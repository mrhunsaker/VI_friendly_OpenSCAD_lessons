% Options for packages loaded elsewhere
\PassOptionsToPackage{unicode}{hyperref}
\PassOptionsToPackage{hyphens}{url}
\documentclass[
  10pt,
  twoside,
        english,
      ]{report}
\usepackage{xcolor}
\usepackage{amsmath,amssymb}
% Detect engine and load fontspec+atkinson only for XeTeX or LuaTeX
\usepackage{iftex}
\ifLuaTeX
  \usepackage{fontspec}
  \usepackage[sfdefault]{atkinson}
\else\ifXeTeX
  \usepackage{fontspec}
  \usepackage[sfdefault]{atkinson}
\fi\fi
% ==================================================================================
% Code listings and table theming (listings + tabularray)
% ==================================================================================
% Listings package for code blocks (configured to use the Alabaster-inspired style below)
\usepackage{listings}
\usepackage{upquote}
\usepackage{tabularray}
\UseTblrLibrary{booktabs}

% ==================================================================================
% Define Alabaster-inspired colors
% ==================================================================================
\definecolor{alabaster-bg}{HTML}{FAFAFA} % Background color - very light gray
\definecolor{alabaster-keyword}{HTML}{C92C2C} % Keywords - red
\definecolor{alabaster-string}{HTML}{008000} % Strings - green
\definecolor{alabaster-comment}{HTML}{0078D7} % Comments - blue
\definecolor{alabaster-number}{HTML}{333333} % Line numbers - dark gray
\definecolor{alabaster-identifier}{HTML}{000000} % Identifiers - black

% Compact arrow for a single tab: compute width of four code-space characters
\newlength{\AlabasterTabWidth}
\settowidth{\AlabasterTabWidth}{{\ttfamily\footnotesize\ \ \ \ }}% width of four spaces in code font
\newcommand*\AlabasterTab{\makebox[\AlabasterTabWidth][c]{\color{BrickRed}\scriptsize\(\rightarrow\)}}% arrow for tab

% ==================================================================================
% Global Listings Settings
% ==================================================================================
\lstset{
  numbers=left,
  numberstyle=\tiny\color{alabaster-number},
  breaklines=true,
  prebreak=\space\mbox{\textcolor{green}{\(\rightarrow\)}},
  postbreak=\mbox{\textcolor{green}{\(\hookrightarrow\)\space}},
  breakatwhitespace=false,
  tabsize=4,
  showtabs=true,
  tab=\AlabasterTab,
  showstringspaces=false,
  showspaces=false,
  firstnumber=1,
  numberfirstline=true,
  stepnumber=2,
  basicstyle=\footnotesize\ttfamily\color{alabaster-identifier},
  columns=fullflexible,
  extendedchars=true,
  inputencoding=utf8,
  aboveskip=0pt, % remove extra vertical space above listings
  belowskip=0pt, % remove extra vertical space below listings
  lineskip=-0.25\baselineskip % collapse blank lines inside listings
}

% ==================================================================================
% Alabaster Style for listings
% ==================================================================================
\lstdefinestyle{Alabaster}{
  frame=lines,
  backgroundcolor=\color{alabaster-bg},
  basicstyle=\footnotesize\ttfamily\color{alabaster-identifier},
  keywordstyle=\bfseries\color{alabaster-keyword},
  identifierstyle=\color{alabaster-identifier},
  stringstyle=\color{alabaster-string},
  commentstyle=\color{alabaster-comment}\itshape,
  showtabs=true,
  belowcaptionskip=0.25em,
  abovecaptionskip=0.25em,
  aboveskip=1em,
  belowskip=0.25em,
  captionpos=b,
  rulesep=0.5em,
  framesep=3pt,
  framerule=0.4pt,
  xleftmargin=3.4pt,
  xrightmargin=3.4pt
}

% Define OpenSCAD language for listings
\lstdefinelanguage{openscad}{
  sensitive=true,
  morekeywords={module,function,if,else,for,intersection,difference,union,translate,rotate,scale,mirror,resize,render,include,use,assign,let,return,echo},
  morecomment=[l]{//},
  morecomment=[s]{/*}{*/},
  morestring=[b]{"},
  morestring=[b]{'},
  keywordstyle=\color{alabaster-keyword}\bfseries,
  commentstyle=\color{alabaster-comment}\itshape,
  stringstyle=\color{alabaster-string},
}

  % PowerShell language definition (use `powershell` in fenced code blocks)
  \lstdefinelanguage{powershell}{
    sensitive=true,
    morekeywords={function,param,begin,process,end,if,else,foreach,while,return,break,continue,try,catch,finally,throw,exit},
      morecomment=[l]{\#},
    morestring=[b]{"},
    morestring=[b]{'},
    keywordstyle=\color{alabaster-keyword}\bfseries,
    commentstyle=\color{alabaster-comment}\itshape,
    stringstyle=\color{alabaster-string},
  }

  % Windows CMD / Batch language definition (use `cmd` in fenced code blocks)
  \lstdefinelanguage{cmd}{
    sensitive=false,
    morekeywords={if,else,for,call,exit,echo,set,cd,copy,del,move,ren,start,dir,type,shift,md,rd,cls},
      morecomment=[l]{REM},
    morecomment=[l]{::},
    morestring=[b]{"},
    alsoletter={@,\%},
    keywordstyle=\color{alabaster-keyword}\bfseries,
    commentstyle=\color{alabaster-comment}\itshape,
    stringstyle=\color{alabaster-string},
  }
% JSON language definition (use `json` in fenced code blocks)
\lstdefinelanguage{json}{
  sensitive=true,
  morestring=[b]{"},
  morecomment=[l]{//},
  morecomment=[s]{/*}{*/},
  literate=
    *{0}{{{\color{alabaster-keyword}0}}}{1}
     {1}{{{\color{alabaster-keyword}1}}}{1}
     {2}{{{\color{alabaster-keyword}2}}}{1}
     {3}{{{\color{alabaster-keyword}3}}}{1}
     {4}{{{\color{alabaster-keyword}4}}}{1}
     {5}{{{\color{alabaster-keyword}5}}}{1}
     {6}{{{\color{alabaster-keyword}6}}}{1}
     {7}{{{\color{alabaster-keyword}7}}}{1}
     {8}{{{\color{alabaster-keyword}8}}}{1}
     {9}{{{\color{alabaster-keyword}9}}}{1}
     {:}{{{\color{alabaster-comment}{:}}}}{1}
     {,}{{{\color{alabaster-comment}{,}}}}{1}
     {\{}{{{\color{alabaster-identifier}{\{}}}}{1}
     {\}}{{{\color{alabaster-identifier}{\}}}}}{1}
     {[}{{{\color{alabaster-identifier}{[}}}}{1}
     {]}{{{\color{alabaster-identifier}{]}}}}{1},
  keywordstyle=\color{alabaster-keyword}\bfseries,
  stringstyle=\color{alabaster-string},
  commentstyle=\color{alabaster-comment}\itshape,
}
% Ensure listings uses the Alabaster style by default
\lstset{style=Alabaster}
\lstalias{openscad}{openscad}
\lstalias{batch}{cmd}
\lstalias{powershell}{powershell}
\lstalias{json}{json}
\usepackage{longtable}
\usepackage{booktabs}
\usepackage{array}

\setcounter{secnumdepth}{-\maxdimen} % remove section numbering
\usepackage{iftex}
\ifPDFTeX
  \usepackage[T1]{fontenc}
  \usepackage[utf8]{inputenc}
  \usepackage{textcomp} % provide euro and other symbols
\else % if luatex or xetex
  \usepackage{unicode-math} % this also loads fontspec
  \defaultfontfeatures{Scale=MatchLowercase}
  \defaultfontfeatures[\rmfamily]{Ligatures=TeX,Scale=1}
\fi
\usepackage{lmodern}
\ifPDFTeX\else
  % xetex/luatex font selection
\fi
% Use upquote if available, for straight quotes in verbatim environments
\IfFileExists{upquote.sty}{\usepackage{upquote}}{}
\IfFileExists{microtype.sty}{% use microtype if available
  \usepackage[]{microtype}
  \UseMicrotypeSet[protrusion]{basicmath} % disable protrusion for tt fonts
}{}
\makeatletter
\@ifundefined{KOMAClassName}{% if non-KOMA class
  \IfFileExists{parskip.sty}{%
    \usepackage{parskip}
  }{% else
    \setlength{\parindent}{0pt}
    \setlength{\parskip}{6pt plus 2pt minus 1pt}}
}{% if KOMA class
  \KOMAoptions{parskip=half}}
\makeatother
\usepackage{longtable,booktabs,array}
\newcounter{none} % for unnumbered tables
\usepackage{calc} % for calculating minipage widths
% Correct order of tables after \paragraph or \subparagraph
\usepackage{etoolbox}
\makeatletter
\patchcmd\longtable{\par}{\if@noskipsec\mbox{}\fi\par}{}{}
\makeatother
% Allow footnotes in longtable head/foot
\IfFileExists{footnotehyper.sty}{\usepackage{footnotehyper}}{\usepackage{footnote}}
\makesavenoteenv{longtable}
\usepackage{graphicx}
\makeatletter
\newsavebox\pandoc@box
\newcommand*\pandocbounded[1]{% scales image to fit in text height/width
  \sbox\pandoc@box{#1}%
  \Gscale@div\@tempa{\textheight}{\dimexpr\ht\pandoc@box+\dp\pandoc@box\relax}%
  \Gscale@div\@tempb{\linewidth}{\wd\pandoc@box}%
  \ifdim\@tempb\p@<\@tempa\p@\let\@tempa\@tempb\fi% select the smaller of both
  \ifdim\@tempa\p@<\p@\scalebox{\@tempa}{\usebox\pandoc@box}%
  \else\usebox{\pandoc@box}%
  \fi%
}
% Set default figure placement to htbp
\def\fps@figure{htbp}
\makeatother
\ifLuaTeX
\usepackage[bidi=basic,shorthands=off]{babel}
\else
\usepackage[bidi=default,shorthands=off]{babel}
\fi
\ifLuaTeX
  \usepackage{selnolig} % disable illegal ligatures
\fi
\setlength{\emergencystretch}{3em} % prevent overfull lines
\providecommand{\tightlist}{%
  \setlength{\itemsep}{0pt}\setlength{\parskip}{0pt}}
% Indexing support
\usepackage{makeidx}
\makeindex

\IfFileExists{fvextra.sty}{% use fvextra if available to break long lines in code blocks
  \usepackage{fvextra}
  \fvset{breaklines}
}{}

\usepackage{url}
\usepackage{makeidx}
\makeindex
\usepackage{bookmark}
\IfFileExists{xurl.sty}{\usepackage{xurl}}{} % add URL line breaks if available
\urlstyle{same}
\hypersetup{
  pdftitle={3D Printing Course for Blind and Visually Impaired Learners},
  pdfauthor={Michael Ryan Hunsaker, M.Ed., Ph.D.},
  pdflang={en},
  hidelinks,
  pdfcreator={LaTeX via pandoc}}

\title{3D Printing Course for Blind and Visually Impaired Learners}
\author{Michael Ryan Hunsaker, M.Ed., Ph.D.}
\date{2026-02-24}

\begin{document}

\renewcommand{\maketitle}{}
\begin{titlepage}
    \centering
    \vspace*{2cm}
    {\Huge\bfseries 3D Printing Course for Blind and Visually Impaired Learners\par}
    \vspace{1.5cm}
    {\Large Michael Ryan Hunsaker, M.Ed., Ph.D.}
    \vspace{1.0cm}
    {\large Updated: 2026-02-24}
    \vfill
\end{titlepage}

% Copyright Page
\newpage
\thispagestyle{empty}
\vspace*{\fill}
{\small \textcopyright\ 2026 Michael Ryan Hunsaker, M.Ed., Ph.D. All rights reserved.}
{\small Creative Commons CC BY-SC-ND license. To view a copy of this license, visit \url{https://creativecommons.org/licenses/by-nc-nd/4.0/}\par}
{\small Book URL: \url{https://mrhunsaker.github.io/VI_3DMake_OpenSCAD_Curriculum/}}
\newpage

{
\setcounter{tocdepth}{2}
\tableofcontents
}
\chapter*{Introduction}\label{docs__pandoc__latex__src__readme.md__readme}
\addcontentsline{toc}{chapter}{Introduction}

This course teaches 3D design and digital fabrication using a fully accessible, command-line-driven toolchain centered on OpenSCAD (text-based CAD), 3DMake (non-visual build automation), and accessible editors (VS Code, Notepad++, command-line editors) with screen reader support.

The curriculum is explicitly designed for blind and visually impaired learners who use screen readers (NVDA, JAWS, VoiceOver). It eliminates GUI navigation and visual feedback in favor of keyboard-driven, text-based workflows that screen readers can fully access. Accessibility is not an add-on. It is the foundation of every tool, workflow, and lesson in this curriculum.

\section*{Curriculum Structure}\label{docs__pandoc__latex__src__readme.md__curriculum-structure}

\subsection*{Screen Reader Setup \& Accessibility Fundamentals}\label{docs__pandoc__latex__src__readme.md__screen-reader-setup--accessibility-fundamentals}

Before choosing a command-line pathway, students learn to optimize their screen reader setup for terminal work and understand accessibility options.

What\textquotesingle s Included:

\begin{itemize}
\tightlist
\item
  \hyperref[docs__pandoc__latex__src__setup__screen_reader_accessibility_guide__screen_reader_accessibility_guide.md__setup_screen_reader_accessibility_guide-screen_reader_accessibility_guide]{Screen Reader Accessibility Guide} - Comprehensive NVDA \& JAWS reference with terminal tips for PowerShell, CMD, and Git Bash
\item
  \hyperref[docs__pandoc__latex__src__setup__screen_reader_choice__screen_reader_choice.md__setup_screen_reader_choice-screen_reader_choice]{Screen Reader Choice: Windows CLI} - Help choosing between NVDA and JAWS
\item
  \hyperref[docs__pandoc__latex__src__setup__braille_displays__braille_displays.md__setup_braille_displays-braille_displays]{Braille Display \& Terminal Mode} - Advanced accessibility setup with refreshable braille displays
\item
  \hyperref[docs__pandoc__latex__src__setup__editor_selection_setup__editor_selection_setup.md__setup_editor_selection_setup-editor_selection_setup]{Editor Selection and Setup} - Choosing and configuring an accessible text editor
\end{itemize}

Why start here? Screen reader optimization is the foundation for all command-line work. Students learn keyboard shortcuts, navigation techniques, and terminal-specific accessibility before diving into any CLI pathway.

\subsection*{Command-Line Foundation (Choose Your Path)}\label{docs__pandoc__latex__src__readme.md__command-line-foundation-choose-your-path}

Students master terminal/command-line fundamentals before learning 3D design. You choose between three equivalent pathways based on your operating system and learning preferences.

\subsubsection*{Compare All Three Pathways}\label{docs__pandoc__latex__src__readme.md__compare-all-three-pathways}

-\textgreater{} \hyperref[docs__pandoc__latex__src__command_line_interface_selection__command_line_interface_selection.md__command_line_interface_selection-command_line_interface_selection]{Command Line Interface Selection Guide}

This guide provides a comprehensive comparison including:

\begin{itemize}
\tightlist
\item
  Feature matrix comparing all three shells
\item
  Command comparison table (navigation, file operations, scripting)
\item
  Learner profiles to help you choose
\item
  Goal-based recommendations
\item
  FAQ addressing common questions
\end{itemize}

\subsubsection*{Part 2A: PowerShell Foundation (Recommended for Windows Users)}\label{docs__pandoc__latex__src__readme.md__part-2a-powershell-foundation-recommended-for-windows-users}

Where to Start: \hyperref[docs__pandoc__latex__src__powershell_foundation__powershell_curriculum_overview__powershell_curriculum_overview.md__powershell_foundation_powershell_curriculum_overview-powershell_curriculum_overview]{PowerShell Curriculum Overview}

Introductory \& Reference Materials:

\begin{itemize}
\tightlist
\item
  \href{https://github.com/mrhunsaker/VI_3DMake_OpenSCAD_Curriculum/PowerShell_Foundation/Powershell_Introduction/Powershell_Introduction.md}{PowerShell Introduction} - Screen-reader-friendly quick-start guide (JAWS/NVDA)
\item
  \href{https://github.com/mrhunsaker/VI_3DMake_OpenSCAD_Curriculum/PowerShell_Foundation/Powershell_Tutorial/Powershell_Tutorial.md}{PowerShell Tutorial} - Hands-on tutorial covering paths, navigation, and common operations
\end{itemize}

Lessons Included:

\begin{itemize}
\tightlist
\item
  PS-Pre: Your First Terminal - Opening PowerShell, first commands, screen reader tricks
\item
  PS-0: Getting Started - Paths, shortcuts, tab completion
\item
  PS-1: Navigation - Moving around the file system confidently
\item
  PS-2: File and Folder Manipulation - Creating, editing, moving files
\item
  PS-3: Input, Output, and Piping - Redirecting output, piping commands
\item
  PS-4: Environment Variables and Aliases - Setting variables, creating shortcuts
\item
  PS-5: Filling in the Gaps - Control flow, profiles, useful tricks
\item
  PS-6: Advanced Terminal Techniques - Scripts, functions, professional workflows
\item
  PowerShell Unit Test \& Practice - Comprehensive self-assessment
\end{itemize}

Time commitment: 30-45 hours (for screen reader users)
Best for: Users who want modern Windows features and advanced automation
Skills: Terminal navigation, piping, advanced scripting, professional automation

\subsubsection*{Part 2B: Windows Command Prompt Foundation (Simplified Alternative)}\label{docs__pandoc__latex__src__readme.md__part-2b-windows-command-prompt-foundation-simplified-alternative}

Where to Start: \hyperref[docs__pandoc__latex__src__cmd_foundation__cmd_curriculum_overview__cmd_curriculum_overview.md__cmd_foundation_cmd_curriculum_overview-cmd_curriculum_overview]{CMD Curriculum Overview}

Lessons Included:

\begin{itemize}
\tightlist
\item
  CMD-Pre: Your First Terminal - Opening CMD, first commands, screen reader tricks
\item
  CMD-0: Getting Started - Paths, shortcuts, command basics
\item
  CMD-1: Navigation - Moving around the file system confidently
\item
  CMD-2: File and Folder Manipulation - Creating, editing, moving files
\item
  CMD-3: Input, Output \& Redirection - Redirecting output, piping commands
\item
  CMD-4: Environment Variables \& Shortcuts - Setting variables, shortcuts
\item
  CMD-5: Filling in the Gaps - Batch files, advanced techniques
\item
  CMD-6: Advanced Terminal Techniques - Scripts, automation, professional workflows
\item
  CMD Unit Test \& Practice - Comprehensive self-assessment
\end{itemize}

Time commitment: 30-45 hours (for screen reader users)
Best for: Absolute beginners or users who prefer simplicity
Skills: Terminal navigation, file operations, batch scripting, basic automation

\subsubsection*{Part 2C: Git Bash Foundation (Cross-Platform Skills)}\label{docs__pandoc__latex__src__readme.md__part-2c-git-bash-foundation-cross-platform-skills}

Where to Start: \hyperref[docs__pandoc__latex__src__gitbash_foundation__gitbash_curriculum_overview__gitbash_curriculum_overview.md__gitbash_foundation_gitbash_curriculum_overview-gitbash_curriculum_overview]{Git Bash Curriculum Overview}

Introductory \& Reference Materials:

\begin{itemize}
\tightlist
\item
  \hyperref[docs__pandoc__latex__src__gitbash_foundation__gitbash_introduction__gitbash_introduction.md__git-bash-introduction]{Git Bash Introduction} - Screen-reader-friendly quick-start guide (JAWS/NVDA)
\item
  \hyperref[docs__pandoc__latex__src__gitbash_foundation__gitbash_tutorial__gitbash_tutorial.md__gitbash_foundation_gitbash_tutorial-gitbash_tutorial]{Git Bash Tutorial} - Hands-on tutorial covering paths, navigation, and common operations
\item
  \hyperref[docs__pandoc__latex__src__gitbash_foundation__screen_reader_accessibility_guide__screen_reader_accessibility_guide.md__screen-reader-accessibility-guide-for-git-bash]{Screen Reader Accessibility Guide for Git Bash} - NVDA and JAWS configuration for Git Bash
\end{itemize}

Lessons Included:

\begin{itemize}
\tightlist
\item
  GitBash-Pre: Your First Terminal - Opening Git Bash, first commands, screen reader tricks
\item
  GitBash-0: Getting Started - Paths, shortcuts, Unix path notation
\item
  GitBash-1: Navigation - Moving around the file system confidently
\item
  GitBash-2: File and Folder Manipulation - Creating, editing, moving files
\item
  GitBash-3: Input, Output \& Piping - Redirecting output, piping commands
\item
  GitBash-4: Environment Variables \& Aliases - Setting variables, creating shortcuts
\item
  GitBash-5: Filling in the Gaps - Shell profiles, history, debugging
\item
  GitBash-6: Advanced Terminal Techniques - Scripts, functions, professional workflows
\item
  GitBash Unit Test \& Practice - Comprehensive self-assessment
\end{itemize}

Time commitment: 30-45 hours (for screen reader users)
Best for: Users who want cross-platform skills (Windows, macOS, Linux)
Skills: Unix/bash commands, shell scripting, cross-platform automation

Important

Choose ONE pathway and complete it fully. All three teach identical fundamental concepts using different tools. Each integrates fully with 3D design workflows.

\subsection*{3dMake Foundation \& Design (11 Lessons + 6 Appendices)}\label{docs__pandoc__latex__src__readme.md__3dmake-foundation--design-11-lessons--6-appendices}

Where to Start: \hyperref[docs__pandoc__latex__src__3dmake_foundation__3dmake_intro__3dmake_intro.md__instructional-framework-for-3dmake-and-openscad-in-secondary-stem-education]{3dMake Introduction}

Students build parametric 3D designs using OpenSCAD and automate the workflow with 3DMake.

\subsubsection*{Main Curriculum: 11 Progressive Lessons}\label{docs__pandoc__latex__src__readme.md__main-curriculum-11-progressive-lessons}

{\def\LTcaptype{none} % do not increment counter
\begin{longtable}[]{@{}
  >{\raggedright\arraybackslash}p{(\linewidth - 6\tabcolsep) * \real{0.1885}}
  >{\raggedright\arraybackslash}p{(\linewidth - 6\tabcolsep) * \real{0.0738}}
  >{\raggedright\arraybackslash}p{(\linewidth - 6\tabcolsep) * \real{0.6475}}
  >{\raggedright\arraybackslash}p{(\linewidth - 6\tabcolsep) * \real{0.0902}}@{}}
\toprule\noalign{}
\begin{minipage}[b]{\linewidth}\raggedright
Part
\end{minipage} & \begin{minipage}[b]{\linewidth}\raggedright
Lessons
\end{minipage} & \begin{minipage}[b]{\linewidth}\raggedright
Focus
\end{minipage} & \begin{minipage}[b]{\linewidth}\raggedright
Duration
\end{minipage} \\
\midrule\noalign{}
\endhead
\bottomrule\noalign{}
\endlastfoot
Foundations & 1-3 &
Environment setup, primitives, parametric design + advanced language features
& 4-5 hours \\
Verification \& Safety & 4-5 &
AI verification, safety protocols, materials & 2-3 hours \\
Applied Projects & 6-8 &
Practical commands, text functions, transforms, assembly patterns &
5-6 hours \\
Advanced Topics & 9-10 &
Automation with file I/O, troubleshooting, mastery & 3-4 hours \\
Leadership & 11 & Stakeholder-centric design & 2-3 hours \\
\end{longtable}
}

Total: 45-55 hours instruction + practice projects

\subsubsection*{Reference Appendices}\label{docs__pandoc__latex__src__readme.md__reference-appendices}

\textbf{Command Line Tools:}

\begin{itemize}
\tightlist
\item
  \href{https://github.com/mrhunsaker/VI_3DMake_OpenSCAD_Curriculum/3dMake_Foundation/Appendix_D_PowerShell_Integration.md}{Appendix A: PowerShell Integration for SCAD Workflows} - Batch processing, automation scripts, and advanced workflow integration
\item
  \href{https://github.com/mrhunsaker/VI_3DMake_OpenSCAD_Curriculum/3dMake_Foundation/Appendix_E_Advanced_OpenSCAD_Concepts.md}{Appendix B: Advanced OpenSCAD Concepts} - Specialized topics including gears, batch processing, performance, print orientation, and recursion
\end{itemize}

\textbf{3dMake:}

\begin{itemize}
\tightlist
\item
  \hyperref[docs__pandoc__latex__src__3dmake_foundation__appendix_a_comprehensive_slicing_guide.md__3dmake_foundation-appendix_a_comprehensive_slicing_guide]{Appendix 1: Comprehensive Slicing Guide} - Complete reference for PrusaSlicer, Bambu Studio, Cura, and OrcaSlicer configuration
\item
  \hyperref[docs__pandoc__latex__src__3dmake_foundation__appendix_b_material_properties.md__3dmake_foundation-appendix_b_material_properties]{Appendix 2: Material Properties \& Selection Guide} - Detailed material reference including shrinkage data, print settings, and properties
\item
  \hyperref[docs__pandoc__latex__src__3dmake_foundation__appendix_c_tolerance_qa.md__3dmake_foundation-appendix_c_tolerance_qa]{Appendix 3: Tolerance Testing \& Quality Assurance Matrix} - Comprehensive QA procedures and tolerance validation methods
\item
  \href{docs/pandoc/latex/src/assets/3dMake_Foundation/README.md}{Appendix 4: 3dMake Code Examples \& Assets} - OpenSCAD code examples and reference designs
\end{itemize}

\section*{The Accessible Toolchain}\label{docs__pandoc__latex__src__readme.md__the-accessible-toolchain}

\subsection*{Screen Reader Compatibility Throughout}\label{docs__pandoc__latex__src__readme.md__screen-reader-compatibility-throughout}

This course uses tools designed for screen reader access:

\begin{itemize}
\tightlist
\item
  Terminal/Command line - Text-based, fully accessible to NVDA, JAWS, VoiceOver
\item
  OpenSCAD - Free, open-source text-based CAD (no visual-only GUI dependency)
\item
  3DMake - Command-line build tool eliminating GUI navigation
\item
  Accessible editors - VS Code, Notepad++, Nano, Vim (all keyboard-driven, screen reader friendly)
\end{itemize}

See \hyperref[docs__pandoc__latex__src__3dmake_foundation__lessons_3dmake_1__nvda-jaws-coding-tips.md__3dmake_foundation_lessons_3dmake_1-nvda-jaws-coding-tips]{Screen Reader Coding Tips (NVDA \& JAWS)} for detailed keyboard shortcuts and configuration.

\subsection*{3DMake: Non-Visual Build Automation}\label{docs__pandoc__latex__src__readme.md__3dmake-non-visual-build-automation}

3DMake makes the entire design-to-print pipeline accessible:

\begin{lstlisting}[style=Alabaster, language=bash]
3dm build        -> Compiles main.scad to main.stl
3dm info         -> Validates geometry and runs diagnostics
3dm slice        -> Prepares model for printing

\end{lstlisting}

\begin{itemize}
\tightlist
\item
  No GUI navigation needed
\item
  Automation eliminates repetitive manual steps
\item
  Configuration files store parameters as human-readable text
\item
  Error reporting is text-based (screen reader accessible)
\item
  Works with command-line slicers
\end{itemize}

\subsection*{Iterative, Non-Visual Design}\label{docs__pandoc__latex__src__readme.md__iterative-non-visual-design}

Students learn to design through code and testing, not visual previews:

\begin{itemize}
\tightlist
\item
  Write parametric OpenSCAD code in accessible editors
\item
  Run \texttt{3dm\ build} to compile to printable file
\item
  Use measurement-based verification (calipers, scales, functional testing)
\item
  Iterate by editing parameters and rebuilding
\item
  No reliance on 3D preview or visual feedback
\end{itemize}

\section*{Project-Based Learning}\label{docs__pandoc__latex__src__readme.md__project-based-learning}

3dMake lessons include hands-on projects:

\begin{itemize}
\tightlist
\item
  Lesson 6: Keycap with embossed text (3dm commands)
\item
  Lesson 7: Phone stand (parametric transforms)
\item
  Lesson 8: Stackable bins (interlocking features, tolerances)
\item
  Lesson 9: Keychain automation (PowerShell batch processing)
\item
  Lesson 10: QA testing + accessibility audit (measurement, troubleshooting)
\item
  Lesson 11: Beaded jewelry holder (stakeholder-driven design)
\end{itemize}

Each project requires:

\begin{itemize}
\tightlist
\item
  Parametric OpenSCAD code (clean, well-commented)
\item
  Functional prototype (tested, iterated)
\item
  Complete documentation (reflections, measurements, decisions)
\end{itemize}

\section*{Learning Support}\label{docs__pandoc__latex__src__readme.md__learning-support}

\subsection*{Reference Materials}\label{docs__pandoc__latex__src__readme.md__reference-materials}

Quick navigation to common topics:

\hyperref[docs__pandoc__latex__src__3dmake_foundation__lessons_3dmake_2__openscad-cheat-sheet.md__3dmake_foundation_lessons_3dmake_2-openscad-cheat-sheet]{OpenSCAD Cheat Sheet} - Syntax quick-reference
\hyperref[docs__pandoc__latex__src__3dmake_foundation__lessons_3dmake_1__3dmake-setup-guide.md__3dmake-setup--workflow]{3dMake Setup Guide} - Installation walkthrough
\hyperref[docs__pandoc__latex__src__3dmake_foundation__lessons_3dmake_1__vscode-setup-guide.md__3dmake_foundation_lessons_3dmake_1-vscode-setup-guide]{VSCode Setup Guide} - Accessibility configuration
\href{https://github.com/mrhunsaker/VI_3DMake_OpenSCAD_Curriculum/3dMake_Foundation/Lessons_3dMake_1/vocabulary-glossary.md}{Vocabulary Glossary} - Course terminology
\href{docs/pandoc/latex/src/3dMake_Foundation/Lessons_3dMake_5/filament-comparison-table.md}{Filament Comparison Table} - Material reference
\hyperref[docs__pandoc__latex__src__3dmake_foundation__lessons_3dmake_11__master-rubric.md__3dmake_foundation_lessons_3dmake_11-master-rubric]{Master Rubric} - Project assessment criteria

\subsection*{Navigation}\label{docs__pandoc__latex__src__readme.md__navigation}

\hyperref[docs__pandoc__latex__src__3dmake_foundation__3dmake_foundation_curriculum_guide.md__3dmake_foundation-3dmake_foundation_curriculum_guide]{Curriculum Guide} - Detailed overview of all lessons and appendices
\hyperref[docs__pandoc__latex__src__3dmake_foundation__3dmake_quick_reference.md__3dmake_foundation-3dmake_quick_reference]{Quick Reference} - At-a-glance command and syntax reference
\href{https://github.com/mrhunsaker/VI_3DMake_OpenSCAD_Curriculum/Appendices/Appendices.md}{Appendices} - 3dMake reference materials

\section*{Supplemental Resources \& Textbooks}\label{docs__pandoc__latex__src__readme.md__supplemental-resources--textbooks}

\subsection*{Textbook options (EPUB Format)}\label{docs__pandoc__latex__src__readme.md__textbook-options-epub-format}

\begin{itemize}
\tightlist
\item
  \href{https://nostarch.com/programmingopenscad}{Programming with OpenSCAD: A Beginner\textquotesingle s Guide to Coding 3D-Printable Objects} - Comprehensive reference covering OpenSCAD syntax, geometry concepts, and design patterns. Ideal for deep dives into specific topics and as a reference guide throughout the course.
\item
  \href{https://www.packtpub.com/en-us/product/simplifying-3d-printing-with-openscad-9781801813174}{Simplifying 3D Printing with OpenSCAD} - Focused on practical workflows, optimization, and real-world printing scenarios.
\end{itemize}

\subsection*{Companion Teaching Resources}\label{docs__pandoc__latex__src__readme.md__companion-teaching-resources}

\href{https://programmingwithopenscad.github.io/learning.html}{Practice Worksheets} - Printable worksheets for visualization practice, decomposition exercises, vocabulary building, and assessment.
\href{https://programmingwithopenscad.github.io/quick-reference.html}{Visual Quick Reference} - Command syntax guides and geometry reference.
\href{https://github.com/ProgrammingWithOpenSCAD/CodeSolutions}{Code Solutions Repository} - Working OpenSCAD examples organized by topic (3D shapes, transformations, loops, modules, if-statements, advanced techniques).

\section*{Getting Started}\label{docs__pandoc__latex__src__readme.md__getting-started}

\subsection*{For Students:}\label{docs__pandoc__latex__src__readme.md__for-students}

\begin{enumerate}
\tightlist
\item
  Start with \hyperref[docs__pandoc__latex__src__setup__screen_reader_accessibility_guide__screen_reader_accessibility_guide.md__setup_screen_reader_accessibility_guide-screen_reader_accessibility_guide]{Setup \& Accessibility Fundamentals}
\item
  Read \hyperref[docs__pandoc__latex__src__command_line_interface_selection__command_line_interface_selection.md__command_line_interface_selection-command_line_interface_selection]{Command Line Interface Selection Guide} to choose your CLI pathway
\item
  Complete your chosen CLI Foundation (PowerShell, CMD, or Git Bash)
\item
  Begin \hyperref[docs__pandoc__latex__src__3dmake_foundation__3dmake_intro__3dmake_intro.md__instructional-framework-for-3dmake-and-openscad-in-secondary-stem-education]{3dMake Introduction}
\item
  Follow \hyperref[docs__pandoc__latex__src__3dmake_foundation__lessons_3dmake_1__lessons_3dmake_1.md__lesson-1-environmental-configuration-and-the-developer-workflow]{Lesson 1: Environmental Configuration}
\item
  Continue through Lesson 11
\end{enumerate}

\section*{For Instructors:}\label{docs__pandoc__latex__src__readme.md__for-instructors}

\begin{enumerate}
\tightlist
\item
  Review \hyperref[docs__pandoc__latex__src__3dmake_foundation__3dmake_foundation_curriculum_guide.md__3dmake_foundation-3dmake_foundation_curriculum_guide]{Curriculum Guide}
\item
  Use \href{docs/pandoc/latex/src/3dMake_Foundation/Templates/Teacher}{11 Teacher Templates} for assessment
\item
  Reference \hyperref[docs__pandoc__latex__src__3dmake_foundation__lessons_3dmake_11__master-rubric.md__3dmake_foundation_lessons_3dmake_11-master-rubric]{Master Rubric} for grading
\item
  Check \hyperref[docs__pandoc__latex__src__syllabus.md__syllabus]{Syllabus} for course policies and learning progression
\end{enumerate}

\section*{Screen Readers}\label{docs__pandoc__latex__src__readme.md__screen-readers}

We know that users of this curriculum will primarily be JAWS and NVDA screenreader users, or else users of Orca if on a Linux-based system. Dolphin SuperNova (commercial) and Windows Narrator (built-in) are also supported; the workflows and recommendations in this document apply to them. See \url{https://yourdolphin.com/ScreenReader-Training} and \url{https://support.microsoft.com/en-us/windows/complete-guide-to-narrator-e4397a0d-ef4f-b386-d8ae-c172f109bdb1} for vendor documentation.

\chapter*{3D Design \& Printing Curriculum - Non-Visual Toolchain Edition}\label{docs__pandoc__latex__src__syllabus.md__syllabus}
\addcontentsline{toc}{chapter}{3D Design \& Printing Curriculum - Non-Visual Toolchain Edition}

Author: Michael Ryan Hunsaker, M.Ed., Ph.D. Last Updated: 2026-02-22
Target Audience: Blind and visually impaired high school students; anyone learning 3D design and printing through screen reader-accessible workflows.

\section*{Overview}\label{docs__pandoc__latex__src__syllabus.md__overview}

This curriculum teaches 3D design and digital fabrication using a fully accessible, command-line-driven toolchain centered on OpenSCAD (text-based CAD), 3DMake (non-visual build automation), and accessible editors (VS Code, Notepad++, command-line editors) with screen reader support. Students progress from foundational command-line skills through guided projects to real-world, stakeholder-driven design challenges.

\subsection*{Who This Course Is For}\label{docs__pandoc__latex__src__syllabus.md__who-this-course-is-for}

This course is explicitly designed for blind and visually impaired learners who use screen readers (NVDA, JAWS, VoiceOver). It eliminates GUI navigation and visual feedback in favor of keyboard-driven, text-based workflows that screen readers can fully access.

Accessibility is not an add-on. It is the foundation of every tool, workflow, and lesson in this curriculum.

\subsection*{Core Philosophy}\label{docs__pandoc__latex__src__syllabus.md__core-philosophy}

\begin{enumerate}
\item
  Text-First Design: All core work happens in text editors and command-line interfaces - no graphical CAD previews, no mouse-dependent menu navigation.
\item
  Parametric Thinking: Students learn to express geometry as code using OpenSCAD, enabling precise, reproducible, and iterable designs without visual feedback.
\item
  Automation and Independence: 3DMake automates the journey from code to printed object, handling compilation, slicing orchestration, and metadata management through simple command-line commands and text configuration files.
\item
  Screen Reader Mastery: Students develop fluency with accessibility technologies (NVDA, JAWS, VoiceOver) and accessible editors, building skills that apply to careers in software, engineering, and digital fabrication.
\item
  Real-World Impact: Projects culminate in designing assistive-technology solutions for real stakeholders, combining technical skill with human-centered design and documentation.
\end{enumerate}

\section*{Curriculum Structure \& Scope/Sequence}\label{docs__pandoc__latex__src__syllabus.md__curriculum-structure--scopesequence}

\subsection*{Setup \& Accessibility Fundamentals (Prerequisite - 2-3 hours)}\label{docs__pandoc__latex__src__syllabus.md__setup--accessibility-fundamentals-prerequisite---2-3-hours}

Start here: \hyperref[docs__pandoc__latex__src__setup__screen_reader_accessibility_guide__screen_reader_accessibility_guide.md__setup_screen_reader_accessibility_guide-screen_reader_accessibility_guide]{Screen Reader Accessibility Guide}

Before choosing a command-line pathway, students optimize their screen reader setup for terminal work.

{\def\LTcaptype{none} % do not increment counter
\begin{longtable}[]{@{}
  >{\raggedright\arraybackslash}p{(\linewidth - 4\tabcolsep) * \real{0.5665}}
  >{\raggedright\arraybackslash}p{(\linewidth - 4\tabcolsep) * \real{0.0640}}
  >{\raggedright\arraybackslash}p{(\linewidth - 4\tabcolsep) * \real{0.3695}}@{}}
\toprule\noalign{}
\begin{minipage}[b]{\linewidth}\raggedright
Component
\end{minipage} & \begin{minipage}[b]{\linewidth}\raggedright
Duration
\end{minipage} & \begin{minipage}[b]{\linewidth}\raggedright
Content
\end{minipage} \\
\midrule\noalign{}
\endhead
\bottomrule\noalign{}
\endlastfoot
\hyperref[docs__pandoc__latex__src__setup__screen_reader_accessibility_guide__screen_reader_accessibility_guide.md__setup_screen_reader_accessibility_guide-screen_reader_accessibility_guide]{Screen Reader Accessibility Guide}
& 1-1.5 hours &
NVDA/JAWS reference for PowerShell, CMD, and Git Bash; keyboard shortcuts \\
\hyperref[docs__pandoc__latex__src__setup__screen_reader_choice__screen_reader_choice.md__setup_screen_reader_choice-screen_reader_choice]{Screen Reader Choice: Windows CLI}
& 30 min &
Comparing NVDA, JAWS, Narrator, and Dolphin; choosing the right tool \\
\hyperref[docs__pandoc__latex__src__setup__braille_displays__braille_displays.md__setup_braille_displays-braille_displays]{Braille Display \& Terminal Mode}
& 30 min &
Optional: configuring refreshable braille displays for terminal work \\
\hyperref[docs__pandoc__latex__src__setup__editor_selection_setup__editor_selection_setup.md__setup_editor_selection_setup-editor_selection_setup]{Editor Selection and Setup}
& 30 min &
Choosing Notepad, Notepad++, or VS Code; configuring indent announcement \\
\end{longtable}
}

\subsection*{Command-Line Foundation (Choose Your Path)}\label{docs__pandoc__latex__src__syllabus.md__command-line-foundation-choose-your-path}

Start here: \hyperref[docs__pandoc__latex__src__command_line_interface_selection__command_line_interface_selection.md__command_line_interface_selection-command_line_interface_selection]{Command Line Interface Selection Guide}

Students master terminal/command-line fundamentals before learning 3D design. Choose one of three equivalent pathways based on your operating system and preferences. All three pathways teach the same concepts and prepare you equally well for 3dMake work.

Important

Choose ONE pathway and complete it fully. All three teach identical fundamental concepts using different tools. Each integrates fully with 3D design workflows.

\subsubsection*{Pathway A: PowerShell Foundation (Recommended for Windows)}\label{docs__pandoc__latex__src__syllabus.md__pathway-a-powershell-foundation-recommended-for-windows}

Total Duration: 30-45 hours
Start here: \hyperref[docs__pandoc__latex__src__powershell_foundation__powershell_curriculum_overview__powershell_curriculum_overview.md__powershell_foundation_powershell_curriculum_overview-powershell_curriculum_overview]{PowerShell Curriculum Overview}

{\def\LTcaptype{none} % do not increment counter
\begin{longtable}[]{@{}
  >{\raggedright\arraybackslash}p{(\linewidth - 4\tabcolsep) * \real{0.6218}}
  >{\raggedright\arraybackslash}p{(\linewidth - 4\tabcolsep) * \real{0.0546}}
  >{\raggedright\arraybackslash}p{(\linewidth - 4\tabcolsep) * \real{0.3235}}@{}}
\toprule\noalign{}
\begin{minipage}[b]{\linewidth}\raggedright
Component
\end{minipage} & \begin{minipage}[b]{\linewidth}\raggedright
Duration
\end{minipage} & \begin{minipage}[b]{\linewidth}\raggedright
Content
\end{minipage} \\
\midrule\noalign{}
\endhead
\bottomrule\noalign{}
\endlastfoot
\href{https://github.com/mrhunsaker/VI_3DMake_OpenSCAD_Curriculum/PowerShell_Foundation/Powershell_Introduction/Powershell_Introduction.md}{PowerShell Introduction}
& 20-30 min &
Screen-reader-friendly quick-start (JAWS/NVDA); essential commands overview \\
\href{https://github.com/mrhunsaker/VI_3DMake_OpenSCAD_Curriculum/PowerShell_Foundation/Powershell_Tutorial/Powershell_Tutorial.md}{PowerShell Tutorial}
& 30-45 min &
Hands-on tutorial: paths, navigation, wildcards, running scripts \\
\hyperref[docs__pandoc__latex__src__powershell_foundation__ps_pre_your_first_terminal__ps_pre_your_first_terminal.md__powershell_foundation_ps_pre_your_first_terminal-ps_pre_your_first_terminal]{PS-Pre: Your First Terminal}
& 2-2.5 hours &
Opening PowerShell, first commands, basic navigation, screen reader tricks \\
\hyperref[docs__pandoc__latex__src__powershell_foundation__ps_0_getting_started_layout_paths__ps_0_getting_started_layout_paths.md__powershell_foundation_ps_0_getting_started_layout_paths-ps_0_getting_started_layout_paths]{PS-0: Getting Started}
& 1.5 hours & Paths, shortcuts, tab completion \\
\hyperref[docs__pandoc__latex__src__powershell_foundation__ps_1_navigation__ps_1_navigation.md__powershell_foundation_ps_1_navigation-ps_1_navigation]{PS-1: Navigation}
& 2-2.5 hours & Moving around the file system confidently \\
\hyperref[docs__pandoc__latex__src__powershell_foundation__ps_2_file_folder_manipulation_modification__ps_2_file_folder_manipulation_modification.md__powershell_foundation_ps_2_file_folder_manipulation_modification-ps_2_file_folder_manipulation_modification]{PS-2: File \& Folder Manipulation}
& 2.5-3 hours &
Creating, editing, moving, copying, deleting files and directories \\
\hyperref[docs__pandoc__latex__src__powershell_foundation__ps_3_input_output_piping__ps_3_input_output_piping.md__powershell_foundation_ps_3_input_output_piping-ps_3_input_output_piping]{PS-3: Input, Output \& Piping}
& 2.5-3 hours &
Redirecting output, piping commands, understanding data flow \\
\hyperref[docs__pandoc__latex__src__powershell_foundation__ps_4_environment_variables_aliases__ps_4_environment_variables_aliases.md__powershell_foundation_ps_4_environment_variables_aliases-ps_4_environment_variables_aliases]{PS-4: Environment Variables \& Aliases}
& 2-2.5 hours &
Setting variables, creating shortcuts, persistent configurations \\
\hyperref[docs__pandoc__latex__src__powershell_foundation__ps_5_filling_in_the_gaps__ps_5_filling_in_the_gaps.md__powershell_foundation_ps_5_filling_in_the_gaps-ps_5_filling_in_the_gaps]{PS-5: Filling in the Gaps}
& 2.5-3 hours &
Control flow, profiles, useful tricks, scripting fundamentals \\
\hyperref[docs__pandoc__latex__src__powershell_foundation__ps_6_advanced_techniques__ps_6_advanced_techniques.md__powershell_foundation_ps_6_advanced_techniques-ps_6_advanced_techniques]{PS-6: Advanced Terminal Techniques}
& 4-4.5 hours &
Scripts, functions, loops, professional workflows, automation patterns \\
\hyperref[docs__pandoc__latex__src__powershell_foundation__ps_unit_test__ps_unit_test.md__powershell_foundation_ps_unit_test-ps_unit_test]{PowerShell Unit Test \& Practice}
& 2-3 hours & Practice exercises, assessment, reinforcement \\
\end{longtable}
}

Outcomes: Terminal fluency, file system mastery, basic scripting, screen reader optimization, automation readiness

\subsubsection*{Pathway B: Windows Command Prompt (CMD) (Simpler alternative)}\label{docs__pandoc__latex__src__syllabus.md__pathway-b-windows-command-prompt-cmd-simpler-alternative}

Total Duration: 30-45 hours
Start here: \hyperref[docs__pandoc__latex__src__cmd_foundation__cmd_curriculum_overview__cmd_curriculum_overview.md__cmd_foundation_cmd_curriculum_overview-cmd_curriculum_overview]{CMD Curriculum Overview}

{\def\LTcaptype{none} % do not increment counter
\begin{longtable}[]{@{}
  >{\raggedright\arraybackslash}p{(\linewidth - 4\tabcolsep) * \real{0.6372}}
  >{\raggedright\arraybackslash}p{(\linewidth - 4\tabcolsep) * \real{0.0575}}
  >{\raggedright\arraybackslash}p{(\linewidth - 4\tabcolsep) * \real{0.3053}}@{}}
\toprule\noalign{}
\begin{minipage}[b]{\linewidth}\raggedright
Component
\end{minipage} & \begin{minipage}[b]{\linewidth}\raggedright
Duration
\end{minipage} & \begin{minipage}[b]{\linewidth}\raggedright
Content
\end{minipage} \\
\midrule\noalign{}
\endhead
\bottomrule\noalign{}
\endlastfoot
\hyperref[docs__pandoc__latex__src__cmd_foundation__cmd_pre_your_first_terminal__cmd_pre_your_first_terminal.md__cmd_foundation_cmd_pre_your_first_terminal-cmd_pre_your_first_terminal]{CMD-Pre: Your First Terminal}
& 2-2.5 hours &
Opening CMD, first commands, basic navigation, screen reader tricks \\
\hyperref[docs__pandoc__latex__src__cmd_foundation__cmd_0_getting_started_layout_paths__cmd_0_getting_started_layout_paths.md__cmd-0-getting-started---layout-paths-and-the-shell]{CMD-0: Getting Started}
& 1.5 hours & Paths, shortcuts, command basics \\
\hyperref[docs__pandoc__latex__src__cmd_foundation__cmd_1_navigation__cmd_1_navigation.md__cmd_foundation_cmd_1_navigation-cmd_1_navigation]{CMD-1: Navigation}
& 2-2.5 hours & Moving around the file system confidently \\
\hyperref[docs__pandoc__latex__src__cmd_foundation__cmd_2_file_folder_manipulation_modification__cmd_2_file_folder_manipulation_modification.md__cmd_foundation_cmd_2_file_folder_manipulation_modification-cmd_2_file_folder_manipulation_modification]{CMD-2: File \& Folder Manipulation}
& 2.5-3 hours &
Creating, editing, moving, copying, deleting files and directories \\
\hyperref[docs__pandoc__latex__src__cmd_foundation__cmd_3_input_output_piping__cmd_3_input_output_piping.md__cmd-3-input-output-and-piping]{CMD-3: Input, Output \& Redirection}
& 2-2.5 hours &
Redirecting output, piping commands, understanding data flow \\
\hyperref[docs__pandoc__latex__src__cmd_foundation__cmd_4_environment_variables_aliases__cmd_4_environment_variables_aliases.md__cmd-4-environment-variables-path-and-aliases]{CMD-4: Environment Variables \& Shortcuts}
& 2-2.5 hours &
Setting variables, creating shortcuts, persistent configurations \\
\hyperref[docs__pandoc__latex__src__cmd_foundation__cmd_5_filling_in_the_gaps__cmd_5_filling_in_the_gaps.md__cmd-5-filling-in-the-gaps---control-flow-startup-scripts-and-useful-tricks]{CMD-5: Filling in the Gaps}
& 2.5-3 hours &
Batch files, advanced techniques, scripting fundamentals \\
\hyperref[docs__pandoc__latex__src__cmd_foundation__cmd_6_advanced_techniques__cmd_6_advanced_techniques.md__cmd-6-advanced-terminal-techniques---batch-scripts-functions--professional-workflows]{CMD-6: Advanced Terminal Techniques}
& 3-3.5 hours & Scripts, automation patterns, professional workflows \\
\hyperref[docs__pandoc__latex__src__cmd_foundation__cmd_unit_test__cmd_unit_test.md__cmd-unit-test---comprehensive-assessment]{CMD Unit Test \& Practice}
& 2-3 hours & Practice exercises, assessment, reinforcement \\
\end{longtable}
}

Outcomes: Terminal fluency, file system mastery, batch scripting, screen reader optimization, automation readiness

\subsubsection*{Pathway C: Git Bash (Best for macOS/Linux or cross-platform development)}\label{docs__pandoc__latex__src__syllabus.md__pathway-c-git-bash-best-for-macoslinux-or-cross-platform-development}

Total Duration: 20-25 hours
Start here: \hyperref[docs__pandoc__latex__src__gitbash_foundation__gitbash_curriculum_overview__gitbash_curriculum_overview.md__gitbash_foundation_gitbash_curriculum_overview-gitbash_curriculum_overview]{Git Bash Curriculum Overview}

{\def\LTcaptype{none} % do not increment counter
\begin{longtable}[]{@{}
  >{\raggedright\arraybackslash}p{(\linewidth - 4\tabcolsep) * \real{0.6454}}
  >{\raggedright\arraybackslash}p{(\linewidth - 4\tabcolsep) * \real{0.0598}}
  >{\raggedright\arraybackslash}p{(\linewidth - 4\tabcolsep) * \real{0.2948}}@{}}
\toprule\noalign{}
\begin{minipage}[b]{\linewidth}\raggedright
Component
\end{minipage} & \begin{minipage}[b]{\linewidth}\raggedright
Duration
\end{minipage} & \begin{minipage}[b]{\linewidth}\raggedright
Content
\end{minipage} \\
\midrule\noalign{}
\endhead
\bottomrule\noalign{}
\endlastfoot
\hyperref[docs__pandoc__latex__src__gitbash_foundation__gitbash_introduction__gitbash_introduction.md__git-bash-introduction]{Git Bash Introduction}
& 20-30 min &
Screen-reader-friendly quick-start (JAWS/NVDA); essential commands \\
\hyperref[docs__pandoc__latex__src__gitbash_foundation__gitbash_tutorial__gitbash_tutorial.md__gitbash_foundation_gitbash_tutorial-gitbash_tutorial]{Git Bash Tutorial}
& 30-45 min &
Hands-on tutorial: paths, navigation, wildcards, running scripts \\
\hyperref[docs__pandoc__latex__src__gitbash_foundation__screen_reader_accessibility_guide__screen_reader_accessibility_guide.md__screen-reader-accessibility-guide-for-git-bash]{Screen Reader Accessibility Guide for Git Bash}
& 30-45 min & NVDA and JAWS configuration specific to Git Bash \\
\hyperref[docs__pandoc__latex__src__gitbash_foundation__gitbash_pre_your_first_terminal__gitbash_pre_your_first_terminal.md__gitbash_foundation_gitbash_pre_your_first_terminal-gitbash_pre_your_first_terminal]{GitBash-Pre: Your First Terminal}
& 2-2.5 hours &
Opening Git Bash, first commands, basic navigation, screen reader tricks \\
\hyperref[docs__pandoc__latex__src__gitbash_foundation__gitbash_0_getting_started_layout_paths__gitbash_0_getting_started_layout_paths.md__gitbash-0-getting-started---layout-paths-and-the-shell]{GitBash-0: Getting Started}
& 1.5 hours &
Unix-style paths, shortcuts, command basics, Windows path conversion \\
\hyperref[docs__pandoc__latex__src__gitbash_foundation__gitbash_1_navigation__gitbash_1_navigation.md__gitbash-1-navigation---moving-around-your-file-system]{GitBash-1: Navigation}
& 2-2.5 hours & Moving around the file system confidently \\
\hyperref[docs__pandoc__latex__src__gitbash_foundation__gitbash_2_file_folder_manipulation_modification__gitbash_2_file_folder_manipulation_modification.md__gitbash-2-file-and-folder-manipulation]{GitBash-2: File and Folder Manipulation}
& 2-2.5 hours &
Creating, editing, moving, copying, deleting files and directories \\
\hyperref[docs__pandoc__latex__src__gitbash_foundation__gitbash_3_input_output_piping__gitbash_3_input_output_piping.md__gitbash-3-input-output-and-piping]{GitBash-3: Input, Output \& Piping}
& 2-2.5 hours &
Redirecting output, piping with grep/sort/wc, understanding data flow \\
\hyperref[docs__pandoc__latex__src__gitbash_foundation__gitbash_4_environment_variables_aliases__gitbash_4_environment_variables_aliases.md__gitbash-4-environment-variables-path-and-aliases]{GitBash-4: Environment Variables \& Aliases}
& 1.5-2 hours & Setting variables, creating aliases, editing .bashrc \\
\hyperref[docs__pandoc__latex__src__gitbash_foundation__gitbash_5_filling_in_the_gaps__gitbash_5_filling_in_the_gaps.md__gitbash-5-filling-in-the-gaps---shell-profiles-history-and-useful-tricks]{GitBash-5: Filling in the Gaps}
& 2-2.5 hours & Shell profiles, command history, debugging \\
\hyperref[docs__pandoc__latex__src__gitbash_foundation__gitbash_6_advanced_techniques__gitbash_6_advanced_techniques.md__gitbash-6-advanced-terminal-techniques---shell-scripts-functions--professional-workflows]{GitBash-6: Advanced Terminal Techniques}
& 2.5-3.5 hours &
Shell scripts, functions, loops, professional workflows \\
\hyperref[docs__pandoc__latex__src__gitbash_foundation__gitbash_unit_test__gitbash_unit_test.md__gitbash-unit-test---comprehensive-assessment]{GitBash Unit Test \& Practice}
& 2-2.5 hours & Practice exercises, assessment, reinforcement \\
\end{longtable}
}

Outcomes: Terminal fluency, file system mastery, bash scripting, version control basics, automation readiness

\subsection*{Common Outcomes (All Pathways)}\label{docs__pandoc__latex__src__syllabus.md__common-outcomes-all-pathways}

\begin{itemize}
\tightlist
\item
  Comfort with terminal/command-line interface
\item
  File system navigation and manipulation
\item
  Basic scripting and automation
\item
  Screen reader optimization for terminal work
\item
  Foundation for 3DMake automation tasks
\end{itemize}

\subsection*{3dMake Foundation (Main Curriculum - 16-20 hours)}\label{docs__pandoc__latex__src__syllabus.md__3dmake-foundation-main-curriculum---16-20-hours}

Start here: \hyperref[docs__pandoc__latex__src__3dmake_foundation__3dmake_intro__3dmake_intro.md__instructional-framework-for-3dmake-and-openscad-in-secondary-stem-education]{3dMake Introduction}

11 progressive lessons building from foundational concepts to leadership-level design thinking, organized in 5 parts. Version 2.1 adds comprehensive advanced programming and design topics throughout.

\subsubsection*{\texorpdfstring{Foundations (Lessons 1-3 \textbar{} \textasciitilde{}4-5 hours)}{Foundations (Lessons 1-3 \textbar{} \textasciitilde4-5 hours)}}\label{docs__pandoc__latex__src__syllabus.md__foundations-lessons-1-3--4-5-hours}

{\def\LTcaptype{none} % do not increment counter
\begin{longtable}[]{@{}
  >{\raggedright\arraybackslash}p{(\linewidth - 6\tabcolsep) * \real{0.4416}}
  >{\raggedright\arraybackslash}p{(\linewidth - 6\tabcolsep) * \real{0.4221}}
  >{\raggedright\arraybackslash}p{(\linewidth - 6\tabcolsep) * \real{0.0779}}
  >{\raggedright\arraybackslash}p{(\linewidth - 6\tabcolsep) * \real{0.0584}}@{}}
\toprule\noalign{}
\begin{minipage}[b]{\linewidth}\raggedright
Lesson
\end{minipage} & \begin{minipage}[b]{\linewidth}\raggedright
Focus
\end{minipage} & \begin{minipage}[b]{\linewidth}\raggedright
Duration
\end{minipage} & \begin{minipage}[b]{\linewidth}\raggedright
Project
\end{minipage} \\
\midrule\noalign{}
\endhead
\bottomrule\noalign{}
\endlastfoot
\hyperref[docs__pandoc__latex__src__3dmake_foundation__lessons_3dmake_1__lessons_3dmake_1.md__lesson-1-environmental-configuration-and-the-developer-workflow]{Lesson 1}
& Environmental Configuration + Code Documentation Standards & 60-90 min
& None \\
\hyperref[docs__pandoc__latex__src__3dmake_foundation__lessons_3dmake_2__lessons_3dmake_2.md__lesson-2-geometric-primitives-and-constructive-solid-geometry]{Lesson 2}
& Primitives \& Boolean Operations + Modifier Characters Debugging &
75-90 min & None \\
\hyperref[docs__pandoc__latex__src__3dmake_foundation__lessons_3dmake_3__lessons_3dmake_3.md__lesson-3-parametric-architecture-and-modular-libraries]{Lesson 3}
& Parametric Architecture + Advanced Programming Concepts & 90-120 min &
None \\
\end{longtable}
}

\subsubsection*{\texorpdfstring{Verification \& Safety (Lessons 4-5 \textbar{} \textasciitilde{}2 hours)}{Verification \& Safety (Lessons 4-5 \textbar{} \textasciitilde2 hours)}}\label{docs__pandoc__latex__src__syllabus.md__verification--safety-lessons-4-5--2-hours}

{\def\LTcaptype{none} % do not increment counter
\begin{longtable}[]{@{}
  >{\raggedright\arraybackslash}p{(\linewidth - 6\tabcolsep) * \real{0.5231}}
  >{\raggedright\arraybackslash}p{(\linewidth - 6\tabcolsep) * \real{0.3231}}
  >{\raggedright\arraybackslash}p{(\linewidth - 6\tabcolsep) * \real{0.0846}}
  >{\raggedright\arraybackslash}p{(\linewidth - 6\tabcolsep) * \real{0.0692}}@{}}
\toprule\noalign{}
\begin{minipage}[b]{\linewidth}\raggedright
Lesson
\end{minipage} & \begin{minipage}[b]{\linewidth}\raggedright
Focus
\end{minipage} & \begin{minipage}[b]{\linewidth}\raggedright
Duration
\end{minipage} & \begin{minipage}[b]{\linewidth}\raggedright
Project
\end{minipage} \\
\midrule\noalign{}
\endhead
\bottomrule\noalign{}
\endlastfoot
\hyperref[docs__pandoc__latex__src__3dmake_foundation__lessons_3dmake_4__lessons_3dmake_4.md__lesson-4-ai-enhanced-verification-and-multimodal-feedback]{Lesson 4}
& AI-Enhanced Verification \& Feedback & 45-60 min & None \\
\hyperref[docs__pandoc__latex__src__3dmake_foundation__lessons_3dmake_5__lessons_3dmake_5.md__lesson-5-safety-protocols-and-the-physical-fabrication-interface]{Lesson 5}
& Safety Protocols \& Material Introduction & 60-90 min & None \\
\end{longtable}
}

\subsubsection*{\texorpdfstring{Applied Projects (Lessons 6-8 \textbar{} \textasciitilde{}5-6 hours)}{Applied Projects (Lessons 6-8 \textbar{} \textasciitilde5-6 hours)}}\label{docs__pandoc__latex__src__syllabus.md__applied-projects-lessons-6-8--5-6-hours}

{\def\LTcaptype{none} % do not increment counter
\begin{longtable}[]{@{}
  >{\raggedright\arraybackslash}p{(\linewidth - 6\tabcolsep) * \real{0.4690}}
  >{\raggedright\arraybackslash}p{(\linewidth - 6\tabcolsep) * \real{0.2966}}
  >{\raggedright\arraybackslash}p{(\linewidth - 6\tabcolsep) * \real{0.0897}}
  >{\raggedright\arraybackslash}p{(\linewidth - 6\tabcolsep) * \real{0.1448}}@{}}
\toprule\noalign{}
\begin{minipage}[b]{\linewidth}\raggedright
Lesson
\end{minipage} & \begin{minipage}[b]{\linewidth}\raggedright
Focus
\end{minipage} & \begin{minipage}[b]{\linewidth}\raggedright
Duration
\end{minipage} & \begin{minipage}[b]{\linewidth}\raggedright
Project
\end{minipage} \\
\midrule\noalign{}
\endhead
\bottomrule\noalign{}
\endlastfoot
\hyperref[docs__pandoc__latex__src__3dmake_foundation__lessons_3dmake_6__lessons_3dmake_6.md__lesson-6-practical-3dm-commands-and-text-embossing]{Lesson 6}
& Practical 3dm Commands + String Functions & 75-105 min &
Customizable Keycap \\
\hyperref[docs__pandoc__latex__src__3dmake_foundation__lessons_3dmake_7__lessons_3dmake_7.md__lesson-7-parametric-transforms-and-the-phone-stand-project]{Lesson 7}
& Parametric Transforms + Math Functions & 90-120 min & Phone Stand \\
\hyperref[docs__pandoc__latex__src__3dmake_foundation__lessons_3dmake_8__lessons_3dmake_8.md__lesson-8-advanced-parametric-design-and-interlocking-features]{Lesson 8}
& Advanced Design + Assembly Best Practices & 105-150 min &
Stackable Bins \\
\end{longtable}
}

\subsubsection*{\texorpdfstring{Advanced Topics (Lessons 9-10 \textbar{} \textasciitilde{}3-4 hours)}{Advanced Topics (Lessons 9-10 \textbar{} \textasciitilde3-4 hours)}}\label{docs__pandoc__latex__src__syllabus.md__advanced-topics-lessons-9-10--3-4-hours}

{\def\LTcaptype{none} % do not increment counter
\begin{longtable}[]{@{}
  >{\raggedright\arraybackslash}p{(\linewidth - 6\tabcolsep) * \real{0.4011}}
  >{\raggedright\arraybackslash}p{(\linewidth - 6\tabcolsep) * \real{0.3333}}
  >{\raggedright\arraybackslash}p{(\linewidth - 6\tabcolsep) * \real{0.0734}}
  >{\raggedright\arraybackslash}p{(\linewidth - 6\tabcolsep) * \real{0.1921}}@{}}
\toprule\noalign{}
\begin{minipage}[b]{\linewidth}\raggedright
Lesson
\end{minipage} & \begin{minipage}[b]{\linewidth}\raggedright
Focus
\end{minipage} & \begin{minipage}[b]{\linewidth}\raggedright
Duration
\end{minipage} & \begin{minipage}[b]{\linewidth}\raggedright
Project
\end{minipage} \\
\midrule\noalign{}
\endhead
\bottomrule\noalign{}
\endlastfoot
\hyperref[docs__pandoc__latex__src__3dmake_foundation__lessons_3dmake_9__lessons_3dmake_9.md__lesson-9-automation-and-3dm-workflows]{Lesson 9}
& Automation + File Import/Export (requires CLI Foundation) & 75-105 min
& Batch Processing Automation \\
\hyperref[docs__pandoc__latex__src__3dmake_foundation__lessons_3dmake_10__lessons_3dmake_10.md__lesson-10-hands-on-practice-exercises-and-troubleshooting]{Lesson 10}
& Troubleshooting \& Mastery with Measurement & 120-150 min &
QA Testing + Accessibility Audit \\
\end{longtable}
}

\subsubsection*{\texorpdfstring{Leadership (Lesson 11 \textbar{} \textasciitilde{}2 hours)}{Leadership (Lesson 11 \textbar{} \textasciitilde2 hours)}}\label{docs__pandoc__latex__src__syllabus.md__leadership-lesson-11--2-hours}

{\def\LTcaptype{none} % do not increment counter
\begin{longtable}[]{@{}
  >{\raggedright\arraybackslash}p{(\linewidth - 6\tabcolsep) * \real{0.4671}}
  >{\raggedright\arraybackslash}p{(\linewidth - 6\tabcolsep) * \real{0.3026}}
  >{\raggedright\arraybackslash}p{(\linewidth - 6\tabcolsep) * \real{0.0789}}
  >{\raggedright\arraybackslash}p{(\linewidth - 6\tabcolsep) * \real{0.1513}}@{}}
\toprule\noalign{}
\begin{minipage}[b]{\linewidth}\raggedright
Lesson
\end{minipage} & \begin{minipage}[b]{\linewidth}\raggedright
Focus
\end{minipage} & \begin{minipage}[b]{\linewidth}\raggedright
Duration
\end{minipage} & \begin{minipage}[b]{\linewidth}\raggedright
Project
\end{minipage} \\
\midrule\noalign{}
\endhead
\bottomrule\noalign{}
\endlastfoot
\hyperref[docs__pandoc__latex__src__3dmake_foundation__lessons_3dmake_11__lessons_3dmake_11.md__lesson-10-hands-on-practice-exercises-and-troubleshooting]{Lesson 11}
& Stakeholder-Centric Design \& Design Thinking & 90-120 min &
Beaded Jewelry Holder \\
\end{longtable}
}

Total: 16-20 hours of instruction + projects

\subsection*{Reference Appendices}\label{docs__pandoc__latex__src__syllabus.md__reference-appendices}

Located in \texttt{3dMake\_Foundation/} alongside the lessons:

{\def\LTcaptype{none} % do not increment counter
\begin{longtable}[]{@{}
  >{\raggedright\arraybackslash}p{(\linewidth - 4\tabcolsep) * \real{0.4190}}
  >{\raggedright\arraybackslash}p{(\linewidth - 4\tabcolsep) * \real{0.2570}}
  >{\raggedright\arraybackslash}p{(\linewidth - 4\tabcolsep) * \real{0.3240}}@{}}
\toprule\noalign{}
\begin{minipage}[b]{\linewidth}\raggedright
Appendix
\end{minipage} & \begin{minipage}[b]{\linewidth}\raggedright
Title
\end{minipage} & \begin{minipage}[b]{\linewidth}\raggedright
Use When
\end{minipage} \\
\midrule\noalign{}
\endhead
\bottomrule\noalign{}
\endlastfoot
\hyperref[docs__pandoc__latex__src__3dmake_foundation__appendix_a_comprehensive_slicing_guide.md__3dmake_foundation-appendix_a_comprehensive_slicing_guide]{Appendix A}
& Comprehensive Slicing Guide &
Slicing questions, switching slicers, quality issues \\
\hyperref[docs__pandoc__latex__src__3dmake_foundation__appendix_b_material_properties.md__3dmake_foundation-appendix_b_material_properties]{Appendix B}
& Material Properties \& Selection Guide &
Choosing material, troubleshooting prints, cost analysis \\
\hyperref[docs__pandoc__latex__src__3dmake_foundation__appendix_c_tolerance_qa.md__3dmake_foundation-appendix_c_tolerance_qa]{Appendix C}
& Tolerance Testing \& Quality Assurance Matrix &
Starting a project, verifying dimensions, quality issues \\
\href{https://github.com/mrhunsaker/VI_3DMake_OpenSCAD_Curriculum/3dMake_Foundation/Appendix_D_PowerShell_Integration.md}{Appendix D}
& PowerShell Integration for SCAD Workflows &
Automating tasks, testing variations, batch printing \\
\href{https://github.com/mrhunsaker/VI_3DMake_OpenSCAD_Curriculum/3dMake_Foundation/Appendix_E_Advanced_OpenSCAD_Concepts.md}{Appendix E}
& Advanced OpenSCAD Concepts &
Building mechanical systems, optimizing complex models \\
\href{docs/pandoc/latex/src/assets/3dMake_Foundation/README.md}{Appendix F}
& 3dMake Code Examples \& Assets &
Reference designs, working code examples \\
\end{longtable}
}

\section*{Learning Progression: Student Roles}\label{docs__pandoc__latex__src__syllabus.md__learning-progression-student-roles}

Students move through roles across the curriculum:

{\def\LTcaptype{none} % do not increment counter
\begin{longtable}[]{@{}
  >{\raggedright\arraybackslash}p{(\linewidth - 6\tabcolsep) * \real{0.1736}}
  >{\raggedright\arraybackslash}p{(\linewidth - 6\tabcolsep) * \real{0.1488}}
  >{\raggedright\arraybackslash}p{(\linewidth - 6\tabcolsep) * \real{0.2810}}
  >{\raggedright\arraybackslash}p{(\linewidth - 6\tabcolsep) * \real{0.3967}}@{}}
\toprule\noalign{}
\begin{minipage}[b]{\linewidth}\raggedright
Phase
\end{minipage} & \begin{minipage}[b]{\linewidth}\raggedright
Role
\end{minipage} & \begin{minipage}[b]{\linewidth}\raggedright
Core Tools
\end{minipage} & \begin{minipage}[b]{\linewidth}\raggedright
Focus
\end{minipage} \\
\midrule\noalign{}
\endhead
\bottomrule\noalign{}
\endlastfoot
CLI Foundation & Observer/Learner & Terminal, command line &
CLI fundamentals and keyboard navigation \\
3dMake Lessons 1-5 & Observer/Learner & OpenSCAD, 3DMake, editor &
Using CLI tools, safety, concepts, measurement \\
3dMake Lessons 6-8 & Operator & Editor, OpenSCAD, 3DMake, slicer &
Hands-on practice with structured projects \\
3dMake Lessons 9-10 & Designer & Full toolchain &
Parametric design, automation, troubleshooting \\
3dMake Lesson 11 & Problem-Solver & Full toolchain + documentation &
Stakeholder design, real-world impact \\
\end{longtable}
}

\section*{The Accessible Toolchain: How It Works}\label{docs__pandoc__latex__src__syllabus.md__the-accessible-toolchain-how-it-works}

\subsection*{OpenSCAD - Text-Based 3D Design}\label{docs__pandoc__latex__src__syllabus.md__openscad---text-based-3d-design}

OpenSCAD is a free, open-source CAD tool that uses a programming language to describe 3D geometry. Students write code that defines shapes, transforms them, and combines them using Boolean operations.

Why OpenSCAD?

\begin{itemize}
\tightlist
\item
  Screen reader friendly: All work happens in a text editor; no visual-only 3D preview.
\item
  Repeatable: Code is version-controlled, documented, and shareable.
\item
  Parametric: Variables allow students to design once and generate variations by changing numbers.
\item
  No visual dependency: Students reason about geometry through code structure and testing.
\end{itemize}

\subsection*{3DMake - The Non-Visual Build Bridge}\label{docs__pandoc__latex__src__syllabus.md__3dmake---the-non-visual-build-bridge}

3DMake is a command-line tool that automates the journey from OpenSCAD code to a printable file:

\begin{lstlisting}[style=Alabaster, language=bash]
3dm build
3dm info
3dm slice

\end{lstlisting}

Why 3DMake?

\begin{itemize}
\tightlist
\item
  No GUI navigation: All interaction is keyboard-driven and text-based.
\item
  Automation: Eliminates repetitive manual steps.
\item
  Metadata tracking: Configuration files store parameters as human-readable text.
\item
  Error reporting: Diagnostic output is text that screen readers can read aloud.
\end{itemize}

\subsection*{Accessible Editors}\label{docs__pandoc__latex__src__syllabus.md__accessible-editors}

Students write OpenSCAD code using screen reader-accessible editors:

\begin{itemize}
\tightlist
\item
  VS Code (Windows, macOS, Linux): Industry-standard with built-in screen reader support
\item
  Notepad++ (Windows): Lightweight, keyboard-driven, excellent screen reader support
\item
  Command-line editors (Nano, Vim, Emacs): Full keyboard control, no mouse needed
\end{itemize}

\section*{Prerequisites by Section}\label{docs__pandoc__latex__src__syllabus.md__prerequisites-by-section}

{\def\LTcaptype{none} % do not increment counter
\begin{longtable}[]{@{}
  >{\raggedright\arraybackslash}p{(\linewidth - 4\tabcolsep) * \real{0.1591}}
  >{\raggedright\arraybackslash}p{(\linewidth - 4\tabcolsep) * \real{0.2955}}
  >{\raggedright\arraybackslash}p{(\linewidth - 4\tabcolsep) * \real{0.5455}}@{}}
\toprule\noalign{}
\begin{minipage}[b]{\linewidth}\raggedright
Section
\end{minipage} & \begin{minipage}[b]{\linewidth}\raggedright
Prerequisites
\end{minipage} & \begin{minipage}[b]{\linewidth}\raggedright
What You\textquotesingle ll Learn
\end{minipage} \\
\midrule\noalign{}
\endhead
\bottomrule\noalign{}
\endlastfoot
Setup & None - start here &
Screen reader optimization, editor selection, accessibility setup \\
CLI Foundation & Setup &
Terminal basics, keyboard navigation, file operations, basic scripting \\
3dMake Lessons 1-5 & Setup (CLI Foundation recommended) &
3D printing concepts, safety, measurement, OpenSCAD basics, debugging \\
3dMake Lessons 6-8 & Lessons 1-5 &
Building projects, parametric design, transforms, tolerances \\
3dMake Lessons 9-10 & Lessons 6-8 + CLI Foundation required &
Automation, troubleshooting, advanced measurement and QA \\
3dMake Lesson 11 & Lessons 9-10 &
Stakeholder design, real-world prototyping, leadership \\
\end{longtable}
}

\section*{Grading Rubric}\label{docs__pandoc__latex__src__syllabus.md__grading-rubric}

All projects are scored on a 0-9 scale across three equally weighted categories (3 points each):

{\def\LTcaptype{none} % do not increment counter
\begin{longtable}[]{@{}
  >{\raggedright\arraybackslash}p{(\linewidth - 4\tabcolsep) * \real{0.1484}}
  >{\raggedright\arraybackslash}p{(\linewidth - 4\tabcolsep) * \real{0.0516}}
  >{\raggedright\arraybackslash}p{(\linewidth - 4\tabcolsep) * \real{0.8000}}@{}}
\toprule\noalign{}
\begin{minipage}[b]{\linewidth}\raggedright
Category
\end{minipage} & \begin{minipage}[b]{\linewidth}\raggedright
Points
\end{minipage} & \begin{minipage}[b]{\linewidth}\raggedright
What We Measure
\end{minipage} \\
\midrule\noalign{}
\endhead
\bottomrule\noalign{}
\endlastfoot
Problem \& Solution & 0-3 &
Does the design solve the stated problem? Are all functional requirements met? \\
Design \& Code Quality & 0-3 &
Is the OpenSCAD code clean, well-commented, and well-structured? Does the print work well? Is there evidence of iteration? \\
Documentation & 0-3 &
Are all sections complete? Are reflections thoughtful and specific? Are measurements recorded? \\
\end{longtable}
}

\subsection*{Category 1: Problem \& Solution (0-3 points)}\label{docs__pandoc__latex__src__syllabus.md__category-1-problem--solution-0-3-points}

{\def\LTcaptype{none} % do not increment counter
\begin{longtable}[]{@{}
  >{\raggedright\arraybackslash}p{(\linewidth - 2\tabcolsep) * \real{0.0400}}
  >{\raggedright\arraybackslash}p{(\linewidth - 2\tabcolsep) * \real{0.9600}}@{}}
\toprule\noalign{}
\begin{minipage}[b]{\linewidth}\raggedright
Score
\end{minipage} & \begin{minipage}[b]{\linewidth}\raggedright
Description
\end{minipage} \\
\midrule\noalign{}
\endhead
\bottomrule\noalign{}
\endlastfoot
3 &
The prototype clearly and effectively solves the stated problem. All functional requirements are met. The solution shows evidence of testing against the requirements. \\
2 &
The prototype mostly meets the problem. Most functional requirements are met. Minor gaps between the design and the requirements. \\
1 &
The prototype partially addresses the problem. Several functional requirements are not met or were not clearly tested. \\
0 &
The prototype does not address the stated problem, or no functional requirements were established. \\
\end{longtable}
}

\subsection*{Category 2: Design \& Code Quality (0-3 points)}\label{docs__pandoc__latex__src__syllabus.md__category-2-design--code-quality-0-3-points}

OpenSCAD code is central to this course. We evaluate the clarity, structure, and documentation of your code as much as the print quality.

{\def\LTcaptype{none} % do not increment counter
\begin{longtable}[]{@{}
  >{\raggedright\arraybackslash}p{(\linewidth - 2\tabcolsep) * \real{0.0341}}
  >{\raggedright\arraybackslash}p{(\linewidth - 2\tabcolsep) * \real{0.9659}}@{}}
\toprule\noalign{}
\begin{minipage}[b]{\linewidth}\raggedright
Score
\end{minipage} & \begin{minipage}[b]{\linewidth}\raggedright
Description
\end{minipage} \\
\midrule\noalign{}
\endhead
\bottomrule\noalign{}
\endlastfoot
3 &
Code is clean, well-organized, and thoroughly commented. Variables/modules are used appropriately. Print quality is excellent. Design shows original thinking and at least one meaningful iteration. \\
2 &
Code works but lacks clear structure or comments. Variables are used but could be better named. Print quality is acceptable. Some iteration evident. \\
1 &
Code is functional but poorly organized. Comments are minimal or missing. Print quality has defects. Little or no iteration. \\
0 &
Code does not work, is not submitted, or shows no original thinking. Print is not functional. \\
\end{longtable}
}

\subsection*{Category 3: Documentation (0-3 points)}\label{docs__pandoc__latex__src__syllabus.md__category-3-documentation-0-3-points}

{\def\LTcaptype{none} % do not increment counter
\begin{longtable}[]{@{}
  >{\raggedright\arraybackslash}p{(\linewidth - 2\tabcolsep) * \real{0.0354}}
  >{\raggedright\arraybackslash}p{(\linewidth - 2\tabcolsep) * \real{0.9646}}@{}}
\toprule\noalign{}
\begin{minipage}[b]{\linewidth}\raggedright
Score
\end{minipage} & \begin{minipage}[b]{\linewidth}\raggedright
Description
\end{minipage} \\
\midrule\noalign{}
\endhead
\bottomrule\noalign{}
\endlastfoot
3 &
All required sections are present, complete, and specific. Reflections are thoughtful and reference specific decisions, problems encountered, and learning. Photos/measurements are included. \\
2 &
Most required sections are present. Some sections are vague or missing detail. Reflections show some thought but are brief or generic. \\
1 &
Documentation is incomplete. Major sections are missing or consist of one-line responses. Reflections are minimal. \\
0 & Documentation is not submitted or is essentially empty. \\
\end{longtable}
}

\subsection*{Score Interpretation}\label{docs__pandoc__latex__src__syllabus.md__score-interpretation}

{\def\LTcaptype{none} % do not increment counter
\begin{longtable}[]{@{}
  >{\raggedright\arraybackslash}p{(\linewidth - 4\tabcolsep) * \real{0.1226}}
  >{\raggedright\arraybackslash}p{(\linewidth - 4\tabcolsep) * \real{0.3491}}
  >{\raggedright\arraybackslash}p{(\linewidth - 4\tabcolsep) * \real{0.5283}}@{}}
\toprule\noalign{}
\begin{minipage}[b]{\linewidth}\raggedright
Total Score
\end{minipage} & \begin{minipage}[b]{\linewidth}\raggedright
Interpretation
\end{minipage} & \begin{minipage}[b]{\linewidth}\raggedright
Next Step
\end{minipage} \\
\midrule\noalign{}
\endhead
\bottomrule\noalign{}
\endlastfoot
8-9 & Excellent work & Move on to next project \\
6-7 & Good work with room for improvement &
Move on; instructor may suggest revisiting one element \\
4-5 & Meets basic expectations &
Resubmission of specific weak areas recommended \\
2-3 & Does not meet expectations & Resubmission required \\
0-1 & Missing major deliverables &
Meet with instructor; create a completion plan \\
\end{longtable}
}

\subsection*{Resubmission Policy}\label{docs__pandoc__latex__src__syllabus.md__resubmission-policy}

Students may resubmit any project as many times as they need to improve their score. Resubmissions must include a one-paragraph explanation of what was changed and why. The resubmission score replaces the original score.

\section*{Quick Links to Essential Tools \& Setup}\label{docs__pandoc__latex__src__syllabus.md__quick-links-to-essential-tools--setup}

\subsection*{Core Design Toolchain}\label{docs__pandoc__latex__src__syllabus.md__core-design-toolchain}

OpenSCAD

\begin{itemize}
\tightlist
\item
  \href{https://openscad.org/downloads.html}{OpenSCAD Download} - Free, cross-platform CAD (all major OS)
\item
  \href{https://openscad.org/documentation.html}{OpenSCAD Documentation} - Official reference
\item
  \hyperref[docs__pandoc__latex__src__3dmake_foundation__lessons_3dmake_2__openscad-cheat-sheet.md__3dmake_foundation_lessons_3dmake_2-openscad-cheat-sheet]{OpenSCAD Cheat Sheet} - Quick syntax reference
\end{itemize}

3DMake

\begin{itemize}
\tightlist
\item
  \href{https://github.com/tdeck/3dmake}{3DMake Documentation \& Installation} - Command-line build tool for OpenSCAD
\item
  \hyperref[docs__pandoc__latex__src__3dmake_foundation__3dmake_quick_reference.md__3dmake_foundation-3dmake_quick_reference]{3dMake Quick Reference} - Command and workflow reference
\end{itemize}

Editors

\begin{itemize}
\tightlist
\item
  \href{https://code.visualstudio.com/}{VS Code Download} - Free, screen-reader-accessible code editor
\item
  \href{https://notepad-plus-plus.org/}{Notepad++ Download} - Free, lightweight Windows editor
\item
  \hyperref[docs__pandoc__latex__src__setup__editor_selection_setup__editor_selection_setup.md__setup_editor_selection_setup-editor_selection_setup]{Editor Selection and Setup Guide} - Accessibility-focused setup guide
\end{itemize}

\subsection*{Screen Reader \& Accessibility}\label{docs__pandoc__latex__src__syllabus.md__screen-reader--accessibility}

Screen Readers

\begin{itemize}
\tightlist
\item
  \href{https://www.nvaccess.org/}{NVDA Download} - Free, open-source screen reader (Windows)
\item
  \href{https://www.freedomscientific.com/products/software/jaws/}{JAWS Screen Reader} - Commercial screen reader (Windows, macOS)
\end{itemize}

Accessibility Configuration

\begin{itemize}
\tightlist
\item
  \hyperref[docs__pandoc__latex__src__setup__screen_reader_accessibility_guide__screen_reader_accessibility_guide.md__setup_screen_reader_accessibility_guide-screen_reader_accessibility_guide]{Screen Reader Accessibility Guide} - Comprehensive terminal accessibility guide
\item
  \hyperref[docs__pandoc__latex__src__3dmake_foundation__lessons_3dmake_1__nvda-jaws-coding-tips.md__3dmake_foundation_lessons_3dmake_1-nvda-jaws-coding-tips]{Screen Reader Coding Tips (NVDA \& JAWS)} - Keyboard shortcuts and configuration
\item
  \hyperref[docs__pandoc__latex__src__3dmake_foundation__lessons_3dmake_1__vscode-setup-guide.md__3dmake_foundation_lessons_3dmake_1-vscode-setup-guide]{VSCode Setup Guide} - Accessibility-focused editor configuration
\item
  \hyperref[docs__pandoc__latex__src__gitbash_foundation__screen_reader_accessibility_guide__screen_reader_accessibility_guide.md__screen-reader-accessibility-guide-for-git-bash]{Git Bash Screen Reader Guide} - NVDA and JAWS configuration for Git Bash
\end{itemize}

\subsection*{Slicing Software}\label{docs__pandoc__latex__src__syllabus.md__slicing-software}

\begin{itemize}
\tightlist
\item
  \href{https://www.prusa3d.com/page/prusaslicer424/}{PrusaSlicer}
\item
  \href{https://bambulab.com/en/download/studio}{Bambu Studio}
\item
  \href{https://ultimaker.com/software/ultimaker-cura}{Cura}
\item
  \href{https://github.com/SoftFever/OrcaSlicer/releases}{OrcaSlicer}
\item
  \hyperref[docs__pandoc__latex__src__3dmake_foundation__appendix_a_comprehensive_slicing_guide.md__3dmake_foundation-appendix_a_comprehensive_slicing_guide]{Appendix A: Comprehensive Slicing Guide} - Detailed setup guides for all major slicers
\end{itemize}

\subsection*{Supplemental Textbooks}\label{docs__pandoc__latex__src__syllabus.md__supplemental-textbooks}

\begin{itemize}
\tightlist
\item
  \href{https://nostarch.com/programmingopenscad}{Programming with OpenSCAD: A Beginner\textquotesingle s Guide}
\item
  \href{https://www.packtpub.com/en-us/product/simplifying-3d-printing-with-openscad-9781801813174}{Simplifying 3D Printing with OpenSCAD}
\item
  \href{https://programmingwithopenscad.github.io/learning.html}{Programming with OpenSCAD Companion Resources}
\item
  \href{https://github.com/ProgrammingWithOpenSCAD/CodeSolutions}{Code Solutions Repository}
\end{itemize}

\section*{Local Resources: Utah Makerspaces \& Community Printing}\label{docs__pandoc__latex__src__syllabus.md__local-resources-utah-makerspaces--community-printing}

\subsection*{Public Library Make Spaces}\label{docs__pandoc__latex__src__syllabus.md__public-library-make-spaces}

Salt Lake City Public Library

\begin{itemize}
\tightlist
\item
  \href{https://services.slcpl.org/creativelab}{SLC Public Creative Lab} - Main Library (Level 1)

  \begin{itemize}
  \tightlist
  \item
    Hardware: Prusa i3 MK3, LulzBot Taz 5, Elegoo Mars 2 (resin)
  \item
    Pricing: Free for prints under 6 hours; \$0.50/hr + material cost otherwise
  \end{itemize}
\end{itemize}

Salt Lake County Library System

\begin{itemize}
\tightlist
\item
  \href{https://www.slcolibrary.org/what-we-have/create}{County Library "Create" Spaces}

  \begin{itemize}
  \tightlist
  \item
    Hardware: Flashforge Adventurer 5M Pro, LulzBot Workhorse, laser cutters
  \item
    Pricing: \$0.06 per gram of filament used
  \end{itemize}
\end{itemize}

\subsection*{Makerspaces \& Community Centers}\label{docs__pandoc__latex__src__syllabus.md__makerspaces--community-centers}

Make Salt Lake

\begin{itemize}
\tightlist
\item
  Location: 663 W 100 S, Salt Lake City, UT 84101
\item
  Website: \url{https://makesaltlake.org/}
\item
  Equipment: Full metal shop, CNC machines, large-scale FDM and resin printing
\end{itemize}

University of Utah Maker Spaces

\begin{itemize}
\tightlist
\item
  \href{https://www.theblackbookofinnovation.com/lassonde-studios}{Lassonde Studios}
\item
  \href{https://lib.utah.edu/protospace.php}{Marriott Library ProtoSpace}
\end{itemize}

\section*{Troubleshooting \& Getting Help}\label{docs__pandoc__latex__src__syllabus.md__troubleshooting--getting-help}

If you\textquotesingle re stuck:

\begin{enumerate}
\tightlist
\item
  Check \hyperref[docs__pandoc__latex__src__3dmake_foundation__lessons_3dmake_10__common_issues_and_solutions.md__3dmake_foundation_lessons_3dmake_10-common_issues_and_solutions]{Common Issues and Solutions}
\item
  Check \hyperref[docs__pandoc__latex__src__3dmake_foundation__lessons_3dmake_10__diagnostic_checklist.md__3dmake_foundation_lessons_3dmake_10-diagnostic_checklist]{Diagnostic Checklist}
\item
  Post in \href{https://discord.gg/F2Nx2VxTB7}{OpenSCAD Discord} or \href{https://www.reddit.com/r/openscad/}{Reddit}
\item
  Visit your local makerspace for hands-on support
\end{enumerate}

For accessibility support:

\begin{itemize}
\tightlist
\item
  Contact your NVDA/JAWS vendor directly for technical assistance
\item
  Refer to the \hyperref[docs__pandoc__latex__src__setup__screen_reader_accessibility_guide__screen_reader_accessibility_guide.md__setup_screen_reader_accessibility_guide-screen_reader_accessibility_guide]{Screen Reader Accessibility Guide}
\item
  Check the \hyperref[docs__pandoc__latex__src__gitbash_foundation__screen_reader_accessibility_guide__screen_reader_accessibility_guide.md__screen-reader-accessibility-guide-for-git-bash]{Git Bash Screen Reader Guide} if using Git Bash
\end{itemize}

\chapter{Screen Reader Accessibility Guide for Command-Line Terminals}\label{docs__pandoc__latex__src__setup__screen_reader_accessibility_guide__screen_reader_accessibility_guide.md__setup_screen_reader_accessibility_guide-screen_reader_accessibility_guide}

\textbf{Target Users:} NVDA, JAWS, Windows Narrator, and Dolphin SuperNova users using PowerShell, CMD, or Git Bash\\
\textbf{Last Updated:} February 2026

This guide is used throughout the Command-Line Foundation curriculum (PowerShell, CMD, and Git Bash pathways) to help screen reader users navigate and work efficiently with the terminal.

\begin{center}\rule{0.5\linewidth}{0.5pt}\end{center}

\section*{Table of Contents}\label{docs__pandoc__latex__src__setup__screen_reader_accessibility_guide__screen_reader_accessibility_guide.md__table-of-contents}

\begin{enumerate}
\tightlist
\item
  \hyperref[docs__pandoc__latex__src__setup__screen_reader_accessibility_guide__screen_reader_accessibility_guide.md__which-terminal]{Which Terminal Should I Use?}
\item
  \hyperref[docs__pandoc__latex__src__setup__screen_reader_accessibility_guide__screen_reader_accessibility_guide.md__getting-started]{Getting Started with Screen Readers}
\item
  \hyperref[docs__pandoc__latex__src__setup__screen_reader_accessibility_guide__screen_reader_accessibility_guide.md__nvda-tips]{NVDA-Specific Tips}
\item
  \hyperref[docs__pandoc__latex__src__setup__screen_reader_accessibility_guide__screen_reader_accessibility_guide.md__jaws-tips]{JAWS-Specific Tips}
\item
  \hyperref[docs__pandoc__latex__src__setup__screen_reader_accessibility_guide__screen_reader_accessibility_guide.md__narrator-tips]{Windows Narrator-Specific Tips}
\item
  \hyperref[docs__pandoc__latex__src__setup__screen_reader_accessibility_guide__screen_reader_accessibility_guide.md__dolphin-tips]{Dolphin SuperNova-Specific Tips}
\item
  \hyperref[docs__pandoc__latex__src__setup__screen_reader_accessibility_guide__screen_reader_accessibility_guide.md__general-terminal]{General Terminal Accessibility}
\item
  \hyperref[docs__pandoc__latex__src__setup__screen_reader_accessibility_guide__screen_reader_accessibility_guide.md__long-output]{Working with Long Output}
\item
  \hyperref[docs__pandoc__latex__src__setup__screen_reader_accessibility_guide__screen_reader_accessibility_guide.md__shortcuts]{Keyboard Shortcuts Reference}
\item
  \hyperref[docs__pandoc__latex__src__setup__screen_reader_accessibility_guide__screen_reader_accessibility_guide.md__shell-differences]{Shell-Specific Differences}
\item
  \hyperref[docs__pandoc__latex__src__setup__screen_reader_accessibility_guide__screen_reader_accessibility_guide.md__troubleshooting]{Troubleshooting}
\end{enumerate}

\begin{center}\rule{0.5\linewidth}{0.5pt}\end{center}

\section*{Which Terminal Should I Use?}\label{docs__pandoc__latex__src__setup__screen_reader_accessibility_guide__screen_reader_accessibility_guide.md__which-terminal}

All three command-line shells (PowerShell, CMD, Git Bash) work well with screen readers. Choose the one that best fits your setup:

{\def\LTcaptype{none} % do not increment counter
\begin{longtable}[]{@{}
  >{\raggedright\arraybackslash}p{(\linewidth - 6\tabcolsep) * \real{0.2621}}
  >{\raggedright\arraybackslash}p{(\linewidth - 6\tabcolsep) * \real{0.2136}}
  >{\raggedright\arraybackslash}p{(\linewidth - 6\tabcolsep) * \real{0.2330}}
  >{\raggedright\arraybackslash}p{(\linewidth - 6\tabcolsep) * \real{0.2913}}@{}}
\toprule\noalign{}
\begin{minipage}[b]{\linewidth}\raggedright
Feature
\end{minipage} & \begin{minipage}[b]{\linewidth}\raggedright
PowerShell
\end{minipage} & \begin{minipage}[b]{\linewidth}\raggedright
CMD
\end{minipage} & \begin{minipage}[b]{\linewidth}\raggedright
Git Bash
\end{minipage} \\
\midrule\noalign{}
\endhead
\bottomrule\noalign{}
\endlastfoot
\textbf{Platform} & Windows only & Windows only &
Windows, macOS, Linux \\
\textbf{Screen Reader Support} & Excellent & Excellent & Excellent \\
\textbf{Learning Curve} & Moderate & Gentle & Moderate-Steep \\
\textbf{Use Case} & Modern automation & Familiar simplicity &
Cross-platform/Git workflows \\
\textbf{Recommended For} & 3D design automation & Getting started easily
& Software developers \\
\end{longtable}
}

\textbf{Key Point:} Choose one pathway and complete it fully. All three teach the same skills and prepare you equally well for 3dMake work. Don\textquotesingle t switch mid-curriculum.

\begin{center}\rule{0.5\linewidth}{0.5pt}\end{center}

\section*{Getting Started with Screen Readers}\label{docs__pandoc__latex__src__setup__screen_reader_accessibility_guide__screen_reader_accessibility_guide.md__getting-started}

\subsection*{Which Screen Reader Should I Use?}\label{docs__pandoc__latex__src__setup__screen_reader_accessibility_guide__screen_reader_accessibility_guide.md__which-screen-reader-should-i-use}

NVDA, JAWS, Windows Narrator, and Dolphin SuperNova all work with the three shells covered in this curriculum. Here is a brief comparison:

{\def\LTcaptype{none} % do not increment counter
\begin{longtable}[]{@{}
  >{\raggedright\arraybackslash}p{(\linewidth - 8\tabcolsep) * \real{0.1880}}
  >{\raggedright\arraybackslash}p{(\linewidth - 8\tabcolsep) * \real{0.2051}}
  >{\raggedright\arraybackslash}p{(\linewidth - 8\tabcolsep) * \real{0.2051}}
  >{\raggedright\arraybackslash}p{(\linewidth - 8\tabcolsep) * \real{0.2308}}
  >{\raggedright\arraybackslash}p{(\linewidth - 8\tabcolsep) * \real{0.1709}}@{}}
\toprule\noalign{}
\begin{minipage}[b]{\linewidth}\raggedright
Feature
\end{minipage} & \begin{minipage}[b]{\linewidth}\raggedright
NVDA
\end{minipage} & \begin{minipage}[b]{\linewidth}\raggedright
JAWS
\end{minipage} & \begin{minipage}[b]{\linewidth}\raggedright
Windows Narrator
\end{minipage} & \begin{minipage}[b]{\linewidth}\raggedright
Dolphin SuperNova
\end{minipage} \\
\midrule\noalign{}
\endhead
\bottomrule\noalign{}
\endlastfoot
\textbf{Cost} & Free (open-source) & Commercial (paid) &
Free (built into Windows) & Commercial (paid) \\
\textbf{Installation} & Simple download & Complex but thorough &
Already installed & Standard installer \\
\textbf{Terminal Support} & Excellent (all shells) &
Excellent (all shells) & Good (all shells) & Good (all shells) \\
\textbf{Learning Curve} & Gentle & Steeper & Minimal & Moderate \\
\textbf{Customization} & Good (add-ons) & Extensive (scripts) & Limited
& Good \\
\end{longtable}
}

\textbf{Recommendation:} If you are new to screen readers, start with NVDA (free, well-documented, good community support) or Windows Narrator (already on your computer, no installation needed). JAWS and Dolphin SuperNova are excellent choices if you already own a license or your organization provides one.

\subsection*{Before You Start}\label{docs__pandoc__latex__src__setup__screen_reader_accessibility_guide__screen_reader_accessibility_guide.md__before-you-start}

\begin{enumerate}
\tightlist
\item
  Make sure your screen reader is running \textbf{before} opening your terminal.
\item
  Open your terminal (PowerShell, CMD, or Git Bash) and let your screen reader read the title and prompt.
\item
  If you do not hear anything, press \textbf{Alt+Tab} to cycle windows and find your terminal.
\item
  Use your screen reader\textquotesingle s review/virtual cursor to understand the layout.
\end{enumerate}

\begin{center}\rule{0.5\linewidth}{0.5pt}\end{center}

\section*{NVDA-Specific Tips}\label{docs__pandoc__latex__src__setup__screen_reader_accessibility_guide__screen_reader_accessibility_guide.md__nvda-tips}

NVDA is free and available from \url{https://www.nvaccess.org/}

These tips work across all shells: PowerShell, CMD, and Git Bash.

\subsection*{Key Commands for Terminals}\label{docs__pandoc__latex__src__setup__screen_reader_accessibility_guide__screen_reader_accessibility_guide.md__key-commands-for-terminals}

{\def\LTcaptype{none} % do not increment counter
\begin{longtable}[]{@{}
  >{\raggedright\arraybackslash}p{(\linewidth - 2\tabcolsep) * \real{0.3684}}
  >{\raggedright\arraybackslash}p{(\linewidth - 2\tabcolsep) * \real{0.6316}}@{}}
\toprule\noalign{}
\begin{minipage}[b]{\linewidth}\raggedright
Command
\end{minipage} & \begin{minipage}[b]{\linewidth}\raggedright
What It Does
\end{minipage} \\
\midrule\noalign{}
\endhead
\bottomrule\noalign{}
\endlastfoot
\textbf{NVDA+Home} & Read the current line (your command or output) \\
\textbf{NVDA+Down Arrow} & Read from cursor position to end of screen \\
\textbf{NVDA+Up Arrow} & Read from top of screen to cursor \\
\textbf{NVDA+Page Down} & Read next page \\
\textbf{NVDA+Page Up} & Read previous page \\
\textbf{NVDA+F7} & Open the Elements List / Review Mode viewer \\
\textbf{NVDA+Shift+Right Arrow} & Read next word \\
\textbf{NVDA+Shift+Down Arrow} & Read entire visible screen \\
\textbf{NVDA+End} & Jump to end of line \\
\textbf{NVDA+Home} & Jump to start of line \\
\textbf{NVDA+Numpad 5} &
Announce the character under the review cursor \\
\textbf{NVDA+Shift+Numpad 5} &
Announce the word under the review cursor \\
\end{longtable}
}

\begin{quote}
\textbf{Note on NVDA key:} On desktop keyboards the NVDA modifier is typically the \textbf{Insert} key. On laptops without a numpad the NVDA key is often \textbf{CapsLock}. You can change this in NVDA Preferences → Keyboard.
\end{quote}

\subsection*{Example: Reading Long Output}\label{docs__pandoc__latex__src__setup__screen_reader_accessibility_guide__screen_reader_accessibility_guide.md__example-reading-long-output}

\textbf{Scenario:} You ran a command and it listed 50 files. You cannot hear them all at once.

\textbf{Solution with NVDA:}

\begin{enumerate}
\tightlist
\item
  After the command finishes, press \textbf{NVDA+Home} to read the current line.
\item
  Press \textbf{NVDA+Down Arrow} repeatedly to read all output line by line.
\item
  Or redirect to a file first: \texttt{command\ \textgreater{}\ output.txt}, then open with \texttt{notepad\ output.txt} for more comfortable reading.
\end{enumerate}

\subsection*{NVDA Settings for Terminals}\label{docs__pandoc__latex__src__setup__screen_reader_accessibility_guide__screen_reader_accessibility_guide.md__nvda-settings-for-terminals}

\begin{enumerate}
\tightlist
\item
  Press \textbf{NVDA+N} to open the NVDA menu.
\item
  Go to \textbf{Preferences → Settings → Speech}.
\item
  Increase or decrease verbosity to taste.
\item
  Under \textbf{Document Formatting}, enable \textbf{"Report indentation"} (important for reading code).
\item
  Under \textbf{Terminal}, enable \textbf{"Speak typed characters"} so you hear each key as you type.
\end{enumerate}

\begin{center}\rule{0.5\linewidth}{0.5pt}\end{center}

\section*{JAWS-Specific Tips}\label{docs__pandoc__latex__src__setup__screen_reader_accessibility_guide__screen_reader_accessibility_guide.md__jaws-tips}

JAWS is a commercial screen reader available from \url{https://www.freedomscientific.com/products/software/jaws/}

These tips work across all shells: PowerShell, CMD, and Git Bash.

\subsection*{Key Commands for Terminals}\label{docs__pandoc__latex__src__setup__screen_reader_accessibility_guide__screen_reader_accessibility_guide.md__key-commands-for-terminals-1}

{\def\LTcaptype{none} % do not increment counter
\begin{longtable}[]{@{}ll@{}}
\toprule\noalign{}
Command & What It Does \\
\midrule\noalign{}
\endhead
\bottomrule\noalign{}
\endlastfoot
\textbf{Insert+Down Arrow} & Read all / say all from cursor downward \\
\textbf{Insert+Up Arrow} & Re-read current line \\
\textbf{Insert+Page Down} & Read next page of terminal output \\
\textbf{Insert+Page Up} & Read previous page of terminal output \\
\textbf{Insert+End} & Jump to end of text on screen \\
\textbf{Insert+Home} & Jump to start of text on screen \\
\textbf{Insert+Ctrl+Down} & Read to end of screen \\
\textbf{Insert+Ctrl+Up} & Read to beginning of screen \\
\textbf{Insert+Shift+Page Down} & Select and read next page \\
\textbf{Insert+F3} & Open JAWS menu \\
\textbf{Insert+Z} & Toggle JAWS virtual cursor on/off \\
\textbf{Insert+F2} & Open JAWS Manager (Settings Center) \\
\end{longtable}
}

\begin{quote}
\textbf{Note on JAWS key:} The JAWS modifier is typically \textbf{Insert} on a standard keyboard. On laptops, JAWS can be configured to use \textbf{CapsLock} as the modifier instead.
\end{quote}

\subsection*{Example: Reading Long Output}\label{docs__pandoc__latex__src__setup__screen_reader_accessibility_guide__screen_reader_accessibility_guide.md__example-reading-long-output-1}

\textbf{Scenario:} You ran a command and saved output to a file.

\textbf{Solution with JAWS:}

\begin{enumerate}
\tightlist
\item
  Open the file in Notepad: \texttt{notepad\ file.txt}
\item
  In Notepad, press \textbf{Insert+Ctrl+Down} to hear all content from the top.
\item
  Use \textbf{Insert+Down Arrow} to read line by line at your own pace.
\item
  Use \textbf{Insert+F} (JAWS Find) to search for specific text within the file.
\end{enumerate}

\subsection*{JAWS Settings for Terminals}\label{docs__pandoc__latex__src__setup__screen_reader_accessibility_guide__screen_reader_accessibility_guide.md__jaws-settings-for-terminals}

\begin{enumerate}
\tightlist
\item
  Press \textbf{Insert+F3} to open the JAWS menu.
\item
  Go to \textbf{Options → Settings Center} (or \textbf{Utilities → Settings Manager} in older versions).
\item
  Search for \textbf{"terminal"} or \textbf{"console"}.
\item
  Enable \textbf{"Announce output"} and verify \textbf{"Speak when program speaks"} is on.
\item
  For indent reporting, search for \textbf{"Indent"} and set to \textbf{"Tones"} or \textbf{"Tones and Speech"}.
\end{enumerate}

\begin{center}\rule{0.5\linewidth}{0.5pt}\end{center}

\section*{Windows Narrator-Specific Tips}\label{docs__pandoc__latex__src__setup__screen_reader_accessibility_guide__screen_reader_accessibility_guide.md__narrator-tips}

Windows Narrator is built into Windows 10 and Windows 11 at no extra cost. Enable it via \textbf{Settings → Accessibility → Narrator} or press \textbf{Windows+Ctrl+Enter}.

These tips work across all shells: PowerShell, CMD, and Git Bash.

\subsection*{Key Commands for Terminals}\label{docs__pandoc__latex__src__setup__screen_reader_accessibility_guide__screen_reader_accessibility_guide.md__key-commands-for-terminals-2}

{\def\LTcaptype{none} % do not increment counter
\begin{longtable}[]{@{}
  >{\raggedright\arraybackslash}p{(\linewidth - 2\tabcolsep) * \real{0.3827}}
  >{\raggedright\arraybackslash}p{(\linewidth - 2\tabcolsep) * \real{0.6173}}@{}}
\toprule\noalign{}
\begin{minipage}[b]{\linewidth}\raggedright
Command
\end{minipage} & \begin{minipage}[b]{\linewidth}\raggedright
What It Does
\end{minipage} \\
\midrule\noalign{}
\endhead
\bottomrule\noalign{}
\endlastfoot
\textbf{Narrator+D} & Read current line \\
\textbf{Narrator+M} & Read next line \\
\textbf{Narrator+I} & Read current item or focused element \\
\textbf{Narrator+R} & Read from cursor / read all from here \\
\textbf{Narrator+Ctrl+R} & Read page / read all \\
\textbf{Narrator+Left/Right Arrow} & Move to previous/next item \\
\textbf{Narrator+Up/Down Arrow} &
Move to previous/next line in scan mode \\
\textbf{Narrator+Enter} & Activate/interact with current item \\
\textbf{Narrator+Space} &
Toggle Narrator scan mode (browse vs. type mode) \\
\textbf{Narrator+F1} & Open Narrator help \\
\textbf{Windows+Ctrl+Enter} & Start or stop Narrator \\
\end{longtable}
}

\begin{quote}
\textbf{Note on Narrator key:} The Narrator modifier is \textbf{CapsLock} or \textbf{Insert} by default. You can change this in \textbf{Settings → Accessibility → Narrator → Keyboard shortcuts}.
\end{quote}

\subsection*{Working with Terminals Using Narrator}\label{docs__pandoc__latex__src__setup__screen_reader_accessibility_guide__screen_reader_accessibility_guide.md__working-with-terminals-using-narrator}

Narrator works well in \textbf{focus mode} (the default when a terminal is active). In focus mode, arrow keys move through command history and output rather than triggering Narrator navigation.

\textbf{Tips:}

\begin{itemize}
\tightlist
\item
  After running a command, press \textbf{Narrator+D} to read the current line.
\item
  Use \textbf{Narrator+R} to read all output from the current position.
\item
  For long outputs, always redirect to a file: \texttt{command\ \textgreater{}\ output.txt}, then open in Notepad where Narrator\textquotesingle s reading experience is more comfortable.
\item
  In Windows Terminal, enable \textbf{"Accessible terminal"} mode in Windows Terminal settings for better Narrator integration.
\end{itemize}

\subsection*{Narrator Settings for Terminals}\label{docs__pandoc__latex__src__setup__screen_reader_accessibility_guide__screen_reader_accessibility_guide.md__narrator-settings-for-terminals}

\begin{enumerate}
\tightlist
\item
  Open \textbf{Settings → Accessibility → Narrator}.
\item
  Under \textbf{"Change what you hear when typing"}, enable \textbf{"Hear characters as you type"}.
\item
  Under \textbf{"Change what Narrator reads"}, enable \textbf{"Read hints for controls and buttons"}.
\item
  Set verbosity level to \textbf{3 or 4} for a good balance of detail.
\item
  Under \textbf{"Choose a Narrator voice"}, select a voice you find comfortable for extended use.
\end{enumerate}

\textbf{Known limitation:} Narrator has fewer customization options than JAWS or NVDA for advanced terminal work. If you find Narrator insufficient for complex workflows, consider switching to NVDA (free) or JAWS.

\begin{center}\rule{0.5\linewidth}{0.5pt}\end{center}

\section*{Dolphin SuperNova-Specific Tips}\label{docs__pandoc__latex__src__setup__screen_reader_accessibility_guide__screen_reader_accessibility_guide.md__dolphin-tips}

Dolphin SuperNova is a commercial screen reader (with optional magnification) available from \url{https://yourdolphin.com/supernova/}

These tips work across all shells: PowerShell, CMD, and Git Bash.

\subsection*{Key Commands for Terminals}\label{docs__pandoc__latex__src__setup__screen_reader_accessibility_guide__screen_reader_accessibility_guide.md__key-commands-for-terminals-3}

{\def\LTcaptype{none} % do not increment counter
\begin{longtable}[]{@{}
  >{\raggedright\arraybackslash}p{(\linewidth - 2\tabcolsep) * \real{0.4878}}
  >{\raggedright\arraybackslash}p{(\linewidth - 2\tabcolsep) * \real{0.5122}}@{}}
\toprule\noalign{}
\begin{minipage}[b]{\linewidth}\raggedright
Command
\end{minipage} & \begin{minipage}[b]{\linewidth}\raggedright
What It Does
\end{minipage} \\
\midrule\noalign{}
\endhead
\bottomrule\noalign{}
\endlastfoot
\textbf{Caps Lock+Down Arrow} (or NumPad 2) & Read current line \\
\textbf{Caps Lock+Numpad 5} & Read current word \\
\textbf{Caps Lock+Numpad 6} & Read next word \\
\textbf{Caps Lock+Numpad 4} & Read previous word \\
\textbf{Caps Lock+Numpad 8} & Read from cursor to top \\
\textbf{Caps Lock+Numpad 2} & Read from cursor to bottom \\
\textbf{Caps Lock+Numpad Plus} &
Say all / read all from current position \\
\textbf{Caps Lock+L} & Re-read current line \\
\textbf{Caps Lock+Right Arrow} & Read next character \\
\textbf{Caps Lock+Left Arrow} & Read previous character \\
\textbf{Caps Lock+Page Down} & Jump to end of document or screen \\
\textbf{Caps Lock+Page Up} & Jump to start of document or screen \\
\end{longtable}
}

\begin{quote}
\textbf{Note on Dolphin key:} Dolphin SuperNova typically uses \textbf{Caps Lock} as its modifier key (referred to as the "Dolphin key"). This can be changed in \textbf{SuperNova Control Panel → Keyboard}.
\end{quote}

\subsection*{Working with Terminals Using Dolphin SuperNova}\label{docs__pandoc__latex__src__setup__screen_reader_accessibility_guide__screen_reader_accessibility_guide.md__working-with-terminals-using-dolphin-supernova}

\begin{enumerate}
\tightlist
\item
  Open SuperNova \textbf{before} opening your terminal application.
\item
  Dolphin SuperNova automatically tracks focus as you move between the editor, terminal, and file explorer.
\item
  In the terminal, use the \textbf{Dolphin key+Numpad Plus} (say all) to have the screen read after running a command.
\item
  For complex output, redirect to a file and open in Notepad: \texttt{command\ \textgreater{}\ output.txt}, then \texttt{notepad\ output.txt}.
\end{enumerate}

\subsection*{Dolphin SuperNova Settings for Terminals}\label{docs__pandoc__latex__src__setup__screen_reader_accessibility_guide__screen_reader_accessibility_guide.md__dolphin-supernova-settings-for-terminals}

\begin{enumerate}
\tightlist
\item
  Open \textbf{SuperNova Control Panel} (press Caps Lock+SpaceBar, or click the SuperNova icon).
\item
  Go to \textbf{Speech → Verbosity} and set a comfortable level (3 or 4 recommended for terminal work).
\item
  Go to \textbf{Speech → Text Processing → Indentation} and enable indentation announcement (important for code).
\item
  Go to \textbf{Braille → Settings} if you are using a braille display alongside speech.
\item
  In \textbf{General → Application Settings}, you can set terminal-specific verbosity so SuperNova behaves differently in your terminal versus other programs.
\end{enumerate}

\textbf{Note:} Dolphin SuperNova\textquotesingle s magnification features can be helpful if you have some remaining vision. Terminal text can be magnified while still receiving speech output.

\begin{center}\rule{0.5\linewidth}{0.5pt}\end{center}

\section*{General Terminal Accessibility}\label{docs__pandoc__latex__src__setup__screen_reader_accessibility_guide__screen_reader_accessibility_guide.md__general-terminal}

\subsection*{Understanding the Terminal Layout}\label{docs__pandoc__latex__src__setup__screen_reader_accessibility_guide__screen_reader_accessibility_guide.md__understanding-the-terminal-layout}

All terminals (PowerShell, CMD, Git Bash) follow the same basic layout:

\begin{enumerate}
\tightlist
\item
  \textbf{Title bar:} Window name (e.g., "Windows PowerShell", "Command Prompt", "Git Bash")
\item
  \textbf{Content area:} Command history and output
\item
  \textbf{Prompt:} The area where you type (e.g., \texttt{PS\textgreater{}}, \texttt{C:\textbackslash{}\textgreater{}}, \texttt{\$})
\end{enumerate}

\textbf{Your screen reader reads from top to bottom, but focus is at the prompt (bottom).}

\subsection*{Navigation Sequence}\label{docs__pandoc__latex__src__setup__screen_reader_accessibility_guide__screen_reader_accessibility_guide.md__navigation-sequence}

\textbf{When you open any terminal:}

\begin{enumerate}
\tightlist
\item
  Your screen reader announces the window title.
\item
  Then it announces the prompt line.
\item
  Anything before the prompt is previous output.
\item
  Anything after the prompt is where new output will appear.
\end{enumerate}

\subsection*{Reading Output Effectively}\label{docs__pandoc__latex__src__setup__screen_reader_accessibility_guide__screen_reader_accessibility_guide.md__reading-output-effectively}

\textbf{Strategy 1: Immediate Output (Small Amount)}

\begin{itemize}
\tightlist
\item
  Run a command.
\item
  Your screen reader announces output immediately.
\item
  This works well for short outputs (a few lines).
\end{itemize}

\textbf{Strategy 2: Large Output (Many Lines)}

\begin{itemize}
\tightlist
\item
  Redirect to a file: \texttt{command\ \textgreater{}\ output.txt}
\item
  Open the file: \texttt{notepad\ output.txt} (works in all shells on Windows)
\item
  Read in Notepad --- easier and more controllable for all screen readers.
\end{itemize}

\textbf{Strategy 3: Filtering Output}

\begin{itemize}
\tightlist
\item
  Use filtering commands to reduce output.
\item
  PowerShell: \texttt{command\ \textbar{}\ Select-String\ "pattern"}
\item
  CMD: \texttt{command\ \textbar{}\ find\ "pattern"}
\item
  Git Bash: \texttt{command\ \textbar{}\ grep\ "pattern"}
\end{itemize}

\begin{center}\rule{0.5\linewidth}{0.5pt}\end{center}

\section*{Working with Long Output}\label{docs__pandoc__latex__src__setup__screen_reader_accessibility_guide__screen_reader_accessibility_guide.md__long-output}

This is one of the most common challenges for screen reader users. Here are proven solutions:

\subsection*{Solution 1: Redirect to a File (Recommended)}\label{docs__pandoc__latex__src__setup__screen_reader_accessibility_guide__screen_reader_accessibility_guide.md__solution-1-redirect-to-a-file-recommended}

\begin{lstlisting}[style=Alabaster]
command > output.txt
notepad output.txt

\end{lstlisting}

\textbf{Advantages:} Easy to navigate at your own pace, works with all screen readers, output does not scroll away, you can save it for later reference.

\subsection*{Solution 2: Use Pagination}\label{docs__pandoc__latex__src__setup__screen_reader_accessibility_guide__screen_reader_accessibility_guide.md__solution-2-use-pagination}

\begin{lstlisting}[style=Alabaster]
command | more

\end{lstlisting}

Press \textbf{Space} to see the next page, \textbf{Q} to quit. Note: some screen readers struggle with \texttt{more}; Solution 1 is generally preferred.

\subsection*{Solution 3: Filter Output}\label{docs__pandoc__latex__src__setup__screen_reader_accessibility_guide__screen_reader_accessibility_guide.md__solution-3-filter-output}

\begin{lstlisting}[style=Alabaster]
# PowerShell
ls | Select-String "\.scad"

# CMD
dir | find ".scad"

# Git Bash
ls -la | grep ".scad"

\end{lstlisting}

\subsection*{Solution 4: Count Before Displaying}\label{docs__pandoc__latex__src__setup__screen_reader_accessibility_guide__screen_reader_accessibility_guide.md__solution-4-count-before-displaying}

\begin{lstlisting}[style=Alabaster, language=powershell]
# PowerShell
(ls).Count

\end{lstlisting}

This tells you how many items there are before deciding whether to redirect to a file.

\begin{center}\rule{0.5\linewidth}{0.5pt}\end{center}

\section*{Keyboard Shortcuts Reference}\label{docs__pandoc__latex__src__setup__screen_reader_accessibility_guide__screen_reader_accessibility_guide.md__shortcuts}

\subsection*{Shell Navigation (Works Regardless of Screen Reader)}\label{docs__pandoc__latex__src__setup__screen_reader_accessibility_guide__screen_reader_accessibility_guide.md__shell-navigation-works-regardless-of-screen-reader}

{\def\LTcaptype{none} % do not increment counter
\begin{longtable}[]{@{}
  >{\raggedright\arraybackslash}p{(\linewidth - 4\tabcolsep) * \real{0.1860}}
  >{\raggedright\arraybackslash}p{(\linewidth - 4\tabcolsep) * \real{0.4186}}
  >{\raggedright\arraybackslash}p{(\linewidth - 4\tabcolsep) * \real{0.3953}}@{}}
\toprule\noalign{}
\begin{minipage}[b]{\linewidth}\raggedright
Key
\end{minipage} & \begin{minipage}[b]{\linewidth}\raggedright
Action
\end{minipage} & \begin{minipage}[b]{\linewidth}\raggedright
Why It Matters
\end{minipage} \\
\midrule\noalign{}
\endhead
\bottomrule\noalign{}
\endlastfoot
\textbf{Up Arrow} & Show previous command &
Repeat commands without retyping \\
\textbf{Down Arrow} & Show next command & Navigate through history \\
\textbf{Tab} & Auto-complete path or command name &
Faster and more accurate \\
\textbf{Shift+Tab} & Cycle backward through completions &
If Tab went too far \\
\textbf{Home} & Jump to start of command line &
Edit beginning of a command \\
\textbf{End} & Jump to end of command line & Edit end of a command \\
\textbf{Ctrl+A} & Select all text on line & Copy entire command \\
\textbf{Ctrl+C} & Stop / interrupt running command &
Abort long-running tasks \\
\textbf{Ctrl+L} & Clear screen & Start fresh visually \\
\textbf{Enter} & Run command & Execute what you typed \\
\end{longtable}
}

\subsection*{Screen Reader Navigation Quick Reference}\label{docs__pandoc__latex__src__setup__screen_reader_accessibility_guide__screen_reader_accessibility_guide.md__screen-reader-navigation-quick-reference}

{\def\LTcaptype{none} % do not increment counter
\begin{longtable}[]{@{}
  >{\raggedright\arraybackslash}p{(\linewidth - 8\tabcolsep) * \real{0.2358}}
  >{\raggedright\arraybackslash}p{(\linewidth - 8\tabcolsep) * \real{0.1981}}
  >{\raggedright\arraybackslash}p{(\linewidth - 8\tabcolsep) * \real{0.1792}}
  >{\raggedright\arraybackslash}p{(\linewidth - 8\tabcolsep) * \real{0.1792}}
  >{\raggedright\arraybackslash}p{(\linewidth - 8\tabcolsep) * \real{0.2075}}@{}}
\toprule\noalign{}
\begin{minipage}[b]{\linewidth}\raggedright
Task
\end{minipage} & \begin{minipage}[b]{\linewidth}\raggedright
NVDA
\end{minipage} & \begin{minipage}[b]{\linewidth}\raggedright
JAWS
\end{minipage} & \begin{minipage}[b]{\linewidth}\raggedright
Windows Narrator
\end{minipage} & \begin{minipage}[b]{\linewidth}\raggedright
Dolphin SuperNova
\end{minipage} \\
\midrule\noalign{}
\endhead
\bottomrule\noalign{}
\endlastfoot
Read current line & NVDA+Home & Insert+Up Arrow & Narrator+D &
CapsLock+L \\
Read all from here & NVDA+Down Arrow & Insert+Ctrl+Down & Narrator+R &
CapsLock+Numpad Plus \\
Read next line & Down Arrow (review) & Insert+Down Arrow & Narrator+M &
CapsLock+Numpad 2 \\
Read previous line & Up Arrow (review) & Insert+Up Arrow &
Narrator+Up Arrow & CapsLock+Numpad 8 \\
Jump to end of output & NVDA+End & Insert+End & Narrator+End &
CapsLock+Page Down \\
Jump to start of output & NVDA+Home & Insert+Home & Narrator+Home &
CapsLock+Page Up \\
Stop reading & Ctrl & Ctrl & Ctrl & Ctrl \\
\end{longtable}
}

\begin{center}\rule{0.5\linewidth}{0.5pt}\end{center}

\section*{Shell-Specific Differences}\label{docs__pandoc__latex__src__setup__screen_reader_accessibility_guide__screen_reader_accessibility_guide.md__shell-differences}

While all three shells work equally well with screen readers, there are differences in commands and syntax:

\subsection*{File Listing and Navigation}\label{docs__pandoc__latex__src__setup__screen_reader_accessibility_guide__screen_reader_accessibility_guide.md__file-listing-and-navigation}

{\def\LTcaptype{none} % do not increment counter
\begin{longtable}[]{@{}
  >{\raggedright\arraybackslash}p{(\linewidth - 6\tabcolsep) * \real{0.2529}}
  >{\raggedright\arraybackslash}p{(\linewidth - 6\tabcolsep) * \real{0.2874}}
  >{\raggedright\arraybackslash}p{(\linewidth - 6\tabcolsep) * \real{0.2299}}
  >{\raggedright\arraybackslash}p{(\linewidth - 6\tabcolsep) * \real{0.2299}}@{}}
\toprule\noalign{}
\begin{minipage}[b]{\linewidth}\raggedright
Task
\end{minipage} & \begin{minipage}[b]{\linewidth}\raggedright
PowerShell
\end{minipage} & \begin{minipage}[b]{\linewidth}\raggedright
CMD
\end{minipage} & \begin{minipage}[b]{\linewidth}\raggedright
Git Bash
\end{minipage} \\
\midrule\noalign{}
\endhead
\bottomrule\noalign{}
\endlastfoot
\textbf{List files} & \texttt{ls} or \texttt{Get-ChildItem} &
\texttt{dir} & \texttt{ls\ -la} \\
\textbf{List names only} & \texttt{ls\ -Name} & \texttt{dir\ /B} &
\texttt{ls\ -1} \\
\textbf{Change directory} & \texttt{cd\ path} & \texttt{cd\ path} &
\texttt{cd\ path} \\
\textbf{Current location} & \texttt{pwd} & \texttt{cd} (no args) &
\texttt{pwd} \\
\textbf{Create folder} & \texttt{mkdir\ foldername} &
\texttt{mkdir\ foldername} & \texttt{mkdir\ foldername} \\
\textbf{Delete file} & \texttt{Remove-Item\ file.txt} &
\texttt{del\ file.txt} & \texttt{rm\ file.txt} \\
\textbf{Home directory} & \texttt{\$HOME} or \texttt{\textasciitilde{}}
& \texttt{\%USERPROFILE\%} & \texttt{\textasciitilde{}} \\
\end{longtable}
}

\subsection*{Output Redirection}\label{docs__pandoc__latex__src__setup__screen_reader_accessibility_guide__screen_reader_accessibility_guide.md__output-redirection}

{\def\LTcaptype{none} % do not increment counter
\begin{longtable}[]{@{}
  >{\raggedright\arraybackslash}p{(\linewidth - 6\tabcolsep) * \real{0.1802}}
  >{\raggedright\arraybackslash}p{(\linewidth - 6\tabcolsep) * \real{0.3153}}
  >{\raggedright\arraybackslash}p{(\linewidth - 6\tabcolsep) * \real{0.2342}}
  >{\raggedright\arraybackslash}p{(\linewidth - 6\tabcolsep) * \real{0.2703}}@{}}
\toprule\noalign{}
\begin{minipage}[b]{\linewidth}\raggedright
Task
\end{minipage} & \begin{minipage}[b]{\linewidth}\raggedright
PowerShell
\end{minipage} & \begin{minipage}[b]{\linewidth}\raggedright
CMD
\end{minipage} & \begin{minipage}[b]{\linewidth}\raggedright
Git Bash
\end{minipage} \\
\midrule\noalign{}
\endhead
\bottomrule\noalign{}
\endlastfoot
\textbf{Save to file} & \texttt{command\ \textgreater{}\ file.txt} &
\texttt{command\ \textgreater{}\ file.txt} &
\texttt{command\ \textgreater{}\ file.txt} \\
\textbf{Append to file} &
\texttt{command\ \textgreater{}\textgreater{}\ file.txt} &
\texttt{command\ \textgreater{}\textgreater{}\ file.txt} &
\texttt{command\ \textgreater{}\textgreater{}\ file.txt} \\
\textbf{Filter output} &
\texttt{command\ \textbar{}\ Select-String\ "text"} &
\texttt{command\ \textbar{}\ find\ "text"} &
\texttt{command\ \textbar{}\ grep\ "text"} \\
\textbf{Open file} & \texttt{notepad\ file.txt} &
\texttt{notepad\ file.txt} & \texttt{notepad\ file.txt} (Windows) \\
\end{longtable}
}

\subsection*{Scripting and Automation}\label{docs__pandoc__latex__src__setup__screen_reader_accessibility_guide__screen_reader_accessibility_guide.md__scripting-and-automation}

{\def\LTcaptype{none} % do not increment counter
\begin{longtable}[]{@{}
  >{\raggedright\arraybackslash}p{(\linewidth - 6\tabcolsep) * \real{0.2740}}
  >{\raggedright\arraybackslash}p{(\linewidth - 6\tabcolsep) * \real{0.2877}}
  >{\raggedright\arraybackslash}p{(\linewidth - 6\tabcolsep) * \real{0.2329}}
  >{\raggedright\arraybackslash}p{(\linewidth - 6\tabcolsep) * \real{0.2055}}@{}}
\toprule\noalign{}
\begin{minipage}[b]{\linewidth}\raggedright
Feature
\end{minipage} & \begin{minipage}[b]{\linewidth}\raggedright
PowerShell
\end{minipage} & \begin{minipage}[b]{\linewidth}\raggedright
CMD
\end{minipage} & \begin{minipage}[b]{\linewidth}\raggedright
Git Bash
\end{minipage} \\
\midrule\noalign{}
\endhead
\bottomrule\noalign{}
\endlastfoot
\textbf{File extension} & \texttt{.ps1} & \texttt{.bat} &
\texttt{.sh} \\
\textbf{Comment} & \texttt{\#} & \texttt{REM} or \texttt{::} &
\texttt{\#} \\
\textbf{Variable} & \texttt{\$var\ =\ "value"} & \texttt{set\ var=value}
& \texttt{var=value} \\
\textbf{Echo text} & \texttt{Write-Host\ "text"} & \texttt{echo\ text} &
\texttt{echo\ text} \\
\textbf{Get help} & \texttt{Get-Help\ command} & \texttt{help\ command}
& \texttt{man\ command} \\
\end{longtable}
}

\textbf{Choose ONE pathway and stick with it.} Each curriculum teaches you the concepts using that specific shell\textquotesingle s syntax. The accessibility experience is virtually identical across all four screen readers in all three shells --- only the command syntax differs.

\begin{center}\rule{0.5\linewidth}{0.5pt}\end{center}

\section*{Troubleshooting}\label{docs__pandoc__latex__src__setup__screen_reader_accessibility_guide__screen_reader_accessibility_guide.md__troubleshooting}

\subsection*{Problem 1: "I Can\textquotesingle t Hear the Output After Running a Command"}\label{docs__pandoc__latex__src__setup__screen_reader_accessibility_guide__screen_reader_accessibility_guide.md__problem-1-i-cant-hear-the-output-after-running-a-command}

\textbf{Causes and Solutions:}

\begin{enumerate}
\tightlist
\item
  \textbf{Cursor is not at the prompt} --- Press \textbf{End} or \textbf{Ctrl+End} to go to the end of text, then use screen reader commands to review.
\item
  \textbf{Output scrolled off-screen} --- Redirect to file: \texttt{command\ \textgreater{}\ output.txt}, then \texttt{notepad\ output.txt}.
\item
  \textbf{Screen reader focus is on window title, not content} --- Press \textbf{Tab} or arrow keys to move into the content area.
\item
  \textbf{Large output overwhelming screen reader} --- Use filtering: \texttt{command\ \textbar{}\ grep\ "pattern"} (Git Bash), \texttt{command\ \textbar{}\ find\ "text"} (CMD), or \texttt{command\ \textbar{}\ Select-String\ "text"} (PowerShell).
\end{enumerate}

\subsection*{Problem 2: "Tab Completion Isn\textquotesingle t Working"}\label{docs__pandoc__latex__src__setup__screen_reader_accessibility_guide__screen_reader_accessibility_guide.md__problem-2-tab-completion-isnt-working}

\begin{enumerate}
\tightlist
\item
  You must type at least one character before pressing Tab.
\item
  The folder or file must exist --- run \texttt{dir} (CMD), \texttt{ls} (PowerShell/Git Bash) to check.
\item
  If multiple matches exist, press Tab again to cycle through them.
\end{enumerate}

\subsection*{Problem 3: "Access Denied or Permission Denied"}\label{docs__pandoc__latex__src__setup__screen_reader_accessibility_guide__screen_reader_accessibility_guide.md__problem-3-access-denied-or-permission-denied}

\begin{enumerate}
\tightlist
\item
  Close your terminal, right-click it, and choose \textbf{Run as administrator}. Confirm the UAC prompt.
\item
  If the file is in use by another program, close that program and try again.
\item
  If the path contains spaces, use quotes: \texttt{cd\ "Program\ Files"}.
\end{enumerate}

\subsection*{Problem 4: "A Command Runs Forever and Won\textquotesingle t Stop"}\label{docs__pandoc__latex__src__setup__screen_reader_accessibility_guide__screen_reader_accessibility_guide.md__problem-4-a-command-runs-forever-and-wont-stop}

Press \textbf{Ctrl+C} to interrupt any running command.

\subsection*{Problem 5: "I Need to Edit My Last Command"}\label{docs__pandoc__latex__src__setup__screen_reader_accessibility_guide__screen_reader_accessibility_guide.md__problem-5-i-need-to-edit-my-last-command}

\begin{enumerate}
\tightlist
\item
  Press \textbf{Up Arrow} to recall the previous command.
\item
  Use \textbf{Left/Right Arrow} keys to move through it.
\item
  Edit as needed and press \textbf{Enter} to run the modified version.
\end{enumerate}

\subsection*{Problem 6: Screen Reader is Not Announcing Indentation}\label{docs__pandoc__latex__src__setup__screen_reader_accessibility_guide__screen_reader_accessibility_guide.md__problem-6-screen-reader-is-not-announcing-indentation}

Proper indent announcement is critical for reading code. Enable it for your screen reader:

\begin{itemize}
\tightlist
\item
  \textbf{NVDA:} NVDA Menu → Preferences → Settings → Document Formatting → enable \textbf{"Report indentation"}, set to \textbf{"Tones and speech"}.
\item
  \textbf{JAWS:} Settings Center → search \textbf{"Indent"} → enable \textbf{"Announce Indentation"}, set mode to \textbf{"Tones"} or \textbf{"Tones and Speech"}.
\item
  \textbf{Windows Narrator:} Settings → Accessibility → Narrator → Advanced Options → enable \textbf{"Report indentation"} (limited options compared to NVDA/JAWS).
\item
  \textbf{Dolphin SuperNova:} SuperNova Control Panel → Speech → Text Processing → Indentation → enable and set preferred announcement style.
\end{itemize}

\begin{center}\rule{0.5\linewidth}{0.5pt}\end{center}

\section*{Pro Tips for Efficiency}\label{docs__pandoc__latex__src__setup__screen_reader_accessibility_guide__screen_reader_accessibility_guide.md__pro-tips-for-efficiency}

\subsection*{Create Aliases for Frequently Used Commands}\label{docs__pandoc__latex__src__setup__screen_reader_accessibility_guide__screen_reader_accessibility_guide.md__create-aliases-for-frequently-used-commands}

\textbf{PowerShell:} \texttt{Set-Alias\ -Name\ ll\ -Value\ "Get-ChildItem\ -Name"}\\
\textbf{CMD:} Create a \texttt{.bat} file in a folder on your PATH.\\
\textbf{Git Bash:} \texttt{alias\ ll=\textquotesingle{}ls\ -la\textquotesingle{}} (add to \texttt{\textasciitilde{}/.bashrc} to make permanent)

\subsection*{Use Command History}\label{docs__pandoc__latex__src__setup__screen_reader_accessibility_guide__screen_reader_accessibility_guide.md__use-command-history}

All shells support \textbf{Up/Down Arrow} to navigate history. In PowerShell, \texttt{history} shows a numbered list and \texttt{Invoke-History\ 5} reruns command number 5. In Git Bash, \texttt{history} and \texttt{!5} work similarly.

\subsection*{Redirect Everything to Files for Accessibility}\label{docs__pandoc__latex__src__setup__screen_reader_accessibility_guide__screen_reader_accessibility_guide.md__redirect-everything-to-files-for-accessibility}

If a command produces output, save it:

\begin{lstlisting}[style=Alabaster]
command > results.txt
notepad results.txt

\end{lstlisting}

This is always more accessible than reading from the terminal directly.

\begin{center}\rule{0.5\linewidth}{0.5pt}\end{center}

\section*{Quick Reference Card}\label{docs__pandoc__latex__src__setup__screen_reader_accessibility_guide__screen_reader_accessibility_guide.md__quick-reference-card}

\begin{lstlisting}[style=Alabaster]
EVERY SESSION STARTS WITH:
1. pwd / cd            (where am I?)
2. dir / ls            (what is here?)
3. cd path             (go there)

LONG OUTPUT?
-> command > file.txt
-> notepad file.txt

STUCK?
-> Ctrl+C

WANT TO REPEAT?
-> Up Arrow

NEED HELP?
-> PowerShell:  `Get-Help command-name`
-> CMD:         `help command-name`
-> Git Bash:    `man command-name`

\end{lstlisting}

\begin{center}\rule{0.5\linewidth}{0.5pt}\end{center}

\section*{Additional Resources}\label{docs__pandoc__latex__src__setup__screen_reader_accessibility_guide__screen_reader_accessibility_guide.md__additional-resources}

\begin{itemize}
\tightlist
\item
  \textbf{NVDA:} \url{https://www.nvaccess.org/}
\item
  \textbf{JAWS:} \url{https://www.freedomscientific.com/products/software/jaws/}
\item
  \textbf{Windows Narrator:} \url{https://support.microsoft.com/narrator}
\item
  \textbf{Dolphin SuperNova:} \url{https://yourdolphin.com/supernova/}
\item
  \textbf{PowerShell Docs:} \url{https://example.com}
\item
  \textbf{CMD Documentation:} \url{https://example.com}
\item
  \textbf{Git Bash / Git for Windows:} \url{https://example.com}
\item
  \textbf{Accessibility Best Practices:} \url{https://example.com}
\end{itemize}

\section{Choosing a Screen Reader for Windows Command-Line Work}\label{docs__pandoc__latex__src__setup__screen_reader_choice__screen_reader_choice.md__setup_screen_reader_choice-screen_reader_choice}

Screen readers provide speech output and, where supported, braille output that make text-based interfaces accessible to people who are blind or have low vision. In command-line environments they convert terminal text, prompts, and keyboard navigation into audible or tactile feedback so users can work independently in shells such as Windows Terminal, Command Prompt, PowerShell, and Git Bash.

Choosing a screen reader is a personal decision. There is no single reader that is universally better or worse for every user. Preferences depend on workflow, training, budget, and comfort with customization. Each product has genuine strengths and trade-offs. If possible, try the available options and pick what feels most natural for you.

\begin{center}\rule{0.5\linewidth}{0.5pt}\end{center}

\subsection*{The Four Screen Readers Covered in This Curriculum}\label{docs__pandoc__latex__src__setup__screen_reader_choice__screen_reader_choice.md__the-four-screen-readers-covered-in-this-curriculum}

\subsubsection*{NVDA (NonVisual Desktop Access)}\label{docs__pandoc__latex__src__setup__screen_reader_choice__screen_reader_choice.md__nvda-nonvisual-desktop-access}

NVDA is a free, open-source screen reader developed by NV Access. It is one of the most widely used screen readers in the world.

\textbf{Advantages:}

\begin{itemize}
\tightlist
\item
  Free to download and use; supported by donations.
\item
  Actively developed with frequent updates.
\item
  Excellent support for Windows terminals (Command Prompt, PowerShell, Windows Terminal, Git Bash).
\item
  Extensible through a rich add-on ecosystem.
\item
  Strong community support, forums, and troubleshooting resources.
\item
  Good braille display support via built-in braille drivers and the BRLTTY/LibLouis library.
\end{itemize}

\textbf{Disadvantages:}

\begin{itemize}
\tightlist
\item
  Advanced paid support is community-driven rather than available from a commercial vendor helpdesk.
\item
  Some edge-case compatibility differences compared to commercial readers.
\end{itemize}

\textbf{Cost:} Free (donations welcomed at \url{https://www.nvaccess.org/}).

\textbf{Trial:} Not applicable --- NVDA is always free.

\textbf{Download:} \url{https://www.nvaccess.org/download/}

\begin{center}\rule{0.5\linewidth}{0.5pt}\end{center}

\subsubsection*{JAWS (Job Access With Speech)}\label{docs__pandoc__latex__src__setup__screen_reader_choice__screen_reader_choice.md__jaws-job-access-with-speech}

JAWS is a commercial screen reader developed by Freedom Scientific (part of Vispero). It is one of the most feature-rich and widely deployed screen readers in enterprise and educational settings.

\textbf{Advantages:}

\begin{itemize}
\tightlist
\item
  Mature product with decades of development and broad application compatibility.
\item
  Extensive scripting system allows deep customization for specific applications.
\item
  Vendor-provided paid support, training, and documentation.
\item
  Advanced braille display support and configuration.
\item
  Widely used in workplaces and schools, so shared knowledge base is large.
\end{itemize}

\textbf{Disadvantages:}

\begin{itemize}
\tightlist
\item
  Commercial product with significant licensing cost.
\item
  JAWS demo mode limits session length to approximately 40 minutes before requiring a computer restart or session reset to continue. This can be disruptive during extended practice sessions.
\item
  Some advanced customization (JAWS scripts) requires a learning curve.
\end{itemize}

\textbf{Cost:} Commercial. Typical U.S. pricing is several hundred dollars per year for a software maintenance agreement (SMA), or a higher one-time perpetual license fee. Check \url{https://www.freedomscientific.com/products/software/jaws/} for current pricing.

\textbf{Trial:} JAWS offers a time-limited demonstration mode (approximately 40 minutes per session). After the demo period expires, you must restart the computer to continue using JAWS without a license. This applies to the full JAWS installer --- there is no separate shorter trial download.

\textbf{Download:} \url{https://www.freedomscientific.com/downloads/jaws/}

\begin{center}\rule{0.5\linewidth}{0.5pt}\end{center}

\subsubsection*{Windows Narrator}\label{docs__pandoc__latex__src__setup__screen_reader_choice__screen_reader_choice.md__windows-narrator}

Windows Narrator is Microsoft\textquotesingle s built-in screen reader, included with Windows 10 and Windows 11 at no extra cost. It has improved significantly with recent Windows releases.

\textbf{Advantages:}

\begin{itemize}
\tightlist
\item
  Already installed on every Windows 10 and Windows 11 computer --- no download or installation needed.
\item
  No cost.
\item
  Simple to start: press \textbf{Windows+Ctrl+Enter} to toggle Narrator on and off.
\item
  Reasonable support for Windows Terminal, Command Prompt, and PowerShell.
\item
  Integrates tightly with Windows accessibility features.
\end{itemize}

\textbf{Disadvantages:}

\begin{itemize}
\tightlist
\item
  Fewer customization options than JAWS or NVDA for complex command-line workflows.
\item
  Braille display support exists but is more limited than dedicated screen readers.
\item
  Less powerful for advanced scripting, code editing, or rapid navigation in large terminal outputs.
\item
  Less community documentation for terminal-specific workflows compared to NVDA and JAWS.
\end{itemize}

\textbf{Cost:} Free (included with Windows).

\textbf{Trial:} Not applicable --- Narrator is always available on Windows.

\textbf{Enable:} Settings → Accessibility → Narrator, or press \textbf{Windows+Ctrl+Enter}.

\begin{center}\rule{0.5\linewidth}{0.5pt}\end{center}

\subsubsection*{Dolphin SuperNova}\label{docs__pandoc__latex__src__setup__screen_reader_choice__screen_reader_choice.md__dolphin-supernova}

Dolphin SuperNova (sometimes referred to as "Dolphin Screen Reader" or simply "SuperNova") is a commercial product from Dolphin Computer Access. It is available in several editions: Magnifier, Magnifier \& Speech, and Screen Reader. The Screen Reader and Magnifier \& Speech editions provide full screen reader functionality.

\textbf{Advantages:}

\begin{itemize}
\tightlist
\item
  Commercial product with vendor support, training options, and documentation.
\item
  Offers combined magnification and speech in a single product --- useful for users with low vision who benefit from both.
\item
  Good braille display support.
\item
  Works with Windows terminals (Command Prompt, PowerShell, Windows Terminal).
\item
  Full-featured 30-day trial license available for evaluation before purchase.
\end{itemize}

\textbf{Disadvantages:}

\begin{itemize}
\tightlist
\item
  Commercial licensing cost.
\item
  Less widely used than NVDA or JAWS, so community resources and third-party documentation are more limited.
\item
  Configuration for some terminal workflows may require additional setup.
\end{itemize}

\textbf{Cost:} Commercial. Pricing varies by edition (Magnifier only, Magnifier \& Speech, or Screen Reader). Check \url{https://yourdolphin.com/supernova/} for current pricing. Expect costs broadly similar to other commercial screen readers.

\textbf{Trial:} Dolphin provides a full-featured 30-day trial license. This is more generous than JAWS\textquotesingle s per-session demo and allows uninterrupted evaluation over a full month.

\textbf{Download:} \url{https://yourdolphin.com/supernova/}

\begin{center}\rule{0.5\linewidth}{0.5pt}\end{center}

\subsection*{Choosing Between Them for Command-Line Work}\label{docs__pandoc__latex__src__setup__screen_reader_choice__screen_reader_choice.md__choosing-between-them-for-command-line-work}

Use this decision guide to narrow your choice:

\textbf{If budget is the primary constraint:}\\
Start with NVDA or Windows Narrator. Both are free. NVDA is generally more capable and better documented for terminal work. Windows Narrator requires no installation and is a good quick-start option.

\textbf{If you already own or have access to JAWS or Dolphin through school or work:}\\
Use what you have. Both work well for this curriculum.

\textbf{If you have low vision (partial sight) in addition to using a screen reader:}\\
Consider Dolphin SuperNova, which combines magnification and speech in one product, or use Windows Magnifier alongside NVDA.

\textbf{If your organization requires vendor-supported software:}\\
JAWS and Dolphin SuperNova both offer commercial support contracts.

\textbf{If you are unsure:}\\
Install NVDA (free, no risk) and try a few lessons. You can always switch later, and the terminal commands you learn work identically regardless of which screen reader you use.

\begin{center}\rule{0.5\linewidth}{0.5pt}\end{center}

\subsection*{Terminal Command Reference by Screen Reader}\label{docs__pandoc__latex__src__setup__screen_reader_choice__screen_reader_choice.md__terminal-command-reference-by-screen-reader}

All four screen readers can perform the same tasks in the terminal. The table below shows the key actions and how to perform them with each reader. Exact key names can vary by keyboard layout and screen reader configuration.

{\def\LTcaptype{none} % do not increment counter
\begin{longtable}[]{@{}
  >{\raggedright\arraybackslash}p{(\linewidth - 8\tabcolsep) * \real{0.1981}}
  >{\raggedright\arraybackslash}p{(\linewidth - 8\tabcolsep) * \real{0.1981}}
  >{\raggedright\arraybackslash}p{(\linewidth - 8\tabcolsep) * \real{0.2029}}
  >{\raggedright\arraybackslash}p{(\linewidth - 8\tabcolsep) * \real{0.1836}}
  >{\raggedright\arraybackslash}p{(\linewidth - 8\tabcolsep) * \real{0.2174}}@{}}
\toprule\noalign{}
\begin{minipage}[b]{\linewidth}\raggedright
Task
\end{minipage} & \begin{minipage}[b]{\linewidth}\raggedright
NVDA
\end{minipage} & \begin{minipage}[b]{\linewidth}\raggedright
JAWS
\end{minipage} & \begin{minipage}[b]{\linewidth}\raggedright
Windows Narrator
\end{minipage} & \begin{minipage}[b]{\linewidth}\raggedright
Dolphin SuperNova
\end{minipage} \\
\midrule\noalign{}
\endhead
\bottomrule\noalign{}
\endlastfoot
\textbf{Read current line or prompt} & NVDA+Home (reads current line) &
Insert+Up Arrow (re-reads current line) &
Narrator+D (reads current line) & CapsLock+L (reads current line) \\
\textbf{Read all output from here} &
NVDA+Down Arrow (read to end of screen) &
Insert+Ctrl+Down (read to end of screen) & Narrator+R (read from cursor)
& CapsLock+Numpad Plus (say all) \\
\textbf{Move to previous line of output} & Up Arrow in review mode &
Insert+Up Arrow (line by line) & Narrator+Up Arrow (scan mode) &
CapsLock+Numpad 8 \\
\textbf{Move to next line of output} & Down Arrow in review mode &
Insert+Down Arrow (line by line) & Narrator+Down Arrow (scan mode) &
CapsLock+Numpad 2 \\
\textbf{Jump to end of screen/buffer} & NVDA+End & Insert+End &
Narrator+End & CapsLock+Page Down \\
\textbf{Jump to start of screen/buffer} & NVDA+Home & Insert+Home &
Narrator+Home & CapsLock+Page Up \\
\textbf{Recall last command (shell history)} &
Up Arrow (shell native; SR reads it) &
Up Arrow (shell native; SR reads it) &
Up Arrow (shell native; SR reads it) &
Up Arrow (shell native; SR reads it) \\
\textbf{Stop reading} & Ctrl & Ctrl & Ctrl & Ctrl \\
\textbf{Open screen reader settings} & NVDA+N → Preferences &
Insert+F3 → Options/Settings & Windows+Ctrl+N (Narrator settings) &
CapsLock+SpaceBar (SuperNova control panel) \\
\end{longtable}
}

\begin{quote}
\textbf{Notes on modifier keys:}

\begin{itemize}
\tightlist
\item
  \textbf{NVDA key} is Insert by default; can be changed to CapsLock in NVDA Preferences → Keyboard.
\item
  \textbf{JAWS key} is Insert by default; can be changed to CapsLock in JAWS Settings Center.
\item
  \textbf{Narrator key} is CapsLock or Insert; configurable in Settings → Accessibility → Narrator → Keyboard shortcuts.
\item
  \textbf{Dolphin key} is CapsLock by default; configurable in SuperNova Control Panel → Keyboard.
\end{itemize}
\end{quote}

\begin{center}\rule{0.5\linewidth}{0.5pt}\end{center}

\subsection*{Important Notes for This Curriculum}\label{docs__pandoc__latex__src__setup__screen_reader_choice__screen_reader_choice.md__important-notes-for-this-curriculum}

\begin{enumerate}
\item
  \textbf{All four screen readers work with all three shells} (CMD, PowerShell, Git Bash) covered in this curriculum. You do not need to change screen readers based on which shell pathway you choose.
\item
  \textbf{The best practice for all four screen readers} when dealing with long terminal output is to redirect it to a file and open it in Notepad:

  \begin{lstlisting}[style=Alabaster]
  command > output.txt
  notepad output.txt

  \end{lstlisting}

  This works in CMD, PowerShell, and Git Bash, and gives you a stable document that all four screen readers can read comfortably.
\item
  \textbf{Braille display users} can use a braille display alongside any of these four screen readers. See the Braille Display Setup guide for detailed instructions.
\item
  \textbf{The screen reader tips in each lesson} of this curriculum are written for NVDA, JAWS, Windows Narrator, and Dolphin SuperNova. If the specific key commands differ from what you hear when practicing, check your screen reader\textquotesingle s documentation for the equivalent command --- the underlying concept is always the same.
\end{enumerate}

\begin{center}\rule{0.5\linewidth}{0.5pt}\end{center}

\subsection*{Additional Resources}\label{docs__pandoc__latex__src__setup__screen_reader_choice__screen_reader_choice.md__additional-resources}

{\def\LTcaptype{none} % do not increment counter
\begin{longtable}[]{@{}
  >{\raggedright\arraybackslash}p{(\linewidth - 4\tabcolsep) * \real{0.1776}}
  >{\raggedright\arraybackslash}p{(\linewidth - 4\tabcolsep) * \real{0.4112}}
  >{\raggedright\arraybackslash}p{(\linewidth - 4\tabcolsep) * \real{0.4112}}@{}}
\toprule\noalign{}
\begin{minipage}[b]{\linewidth}\raggedright
Screen Reader
\end{minipage} & \begin{minipage}[b]{\linewidth}\raggedright
Official Documentation
\end{minipage} & \begin{minipage}[b]{\linewidth}\raggedright
Support / Help
\end{minipage} \\
\midrule\noalign{}
\endhead
\bottomrule\noalign{}
\endlastfoot
NVDA & \url{https://www.nvaccess.org/} &
\url{https://www.nvaccess.org/download/} \\
JAWS & \url{https://www.freedomscientific.com/products/software/jaws/} &
\url{https://www.freedomscientific.com/downloads/jaws/} \\
Windows Narrator & \url{https://support.microsoft.com/narrator} &
\url{https://support.microsoft.com/narrator} \\
Dolphin SuperNova & \url{https://yourdolphin.com/supernova/} &
\url{https://yourdolphin.com/supernova/} \\
\end{longtable}
}

\section{Using a Braille Display with the Curriculum}\label{docs__pandoc__latex__src__setup__braille_displays__braille_displays.md__setup_braille_displays-braille_displays}

Many blind and low-vision programmers prefer tactile output alongside or instead of speech. A refreshable braille display presents terminal text as a row (or multiple rows) of braille cells and allows you to read at your own pace, re-read lines without interrupting speech, and work quietly. This page covers how to connect and configure a braille display with each screen reader used in this curriculum.

\begin{center}\rule{0.5\linewidth}{0.5pt}\end{center}

\subsection*{How a Braille Display Works with a Terminal}\label{docs__pandoc__latex__src__setup__braille_displays__braille_displays.md__how-a-braille-display-works-with-a-terminal}

A refreshable braille display uses small pins that rise and fall to form braille characters. When connected to a computer and paired with a screen reader, the display shows the line of text where the screen reader\textquotesingle s focus or review cursor is located. You can pan left and right across a long line or press routing buttons to move the screen reader cursor to a specific position.

In a terminal environment:

\begin{itemize}
\tightlist
\item
  The display shows the current command line (what you are typing).
\item
  After running a command, you can use your screen reader\textquotesingle s review cursor and the display\textquotesingle s pan keys to read through the output line by line.
\item
  Routing keys (small buttons above or below each braille cell) let you move the cursor directly to that point in the text --- useful for re-running or editing commands from history.
\end{itemize}

\begin{center}\rule{0.5\linewidth}{0.5pt}\end{center}

\subsection*{Connecting a Braille Display}\label{docs__pandoc__latex__src__setup__braille_displays__braille_displays.md__connecting-a-braille-display}

\subsubsection*{USB Connection}\label{docs__pandoc__latex__src__setup__braille_displays__braille_displays.md__usb-connection}

\begin{enumerate}
\tightlist
\item
  Plug the display\textquotesingle s USB cable into your computer.
\item
  Windows will attempt to install drivers automatically. If the display is not recognized, install the vendor\textquotesingle s driver software first (see vendor links at the end of this page).
\item
  Open your screen reader\textquotesingle s braille settings and select the connected display.
\end{enumerate}

\subsubsection*{Bluetooth Connection}\label{docs__pandoc__latex__src__setup__braille_displays__braille_displays.md__bluetooth-connection}

\begin{enumerate}
\tightlist
\item
  Put the braille display into pairing mode (refer to the device\textquotesingle s quick-start guide).
\item
  On Windows, go to \textbf{Settings → Bluetooth \& devices → Add a device} and pair the display.
\item
  Once paired, open your screen reader\textquotesingle s braille settings and select the display.
\end{enumerate}

\subsubsection*{Before Opening the Terminal}\label{docs__pandoc__latex__src__setup__braille_displays__braille_displays.md__before-opening-the-terminal}

Always confirm the display is connected and showing output before opening your terminal. A quick test: open Notepad, type a few words, and verify they appear on the display. This confirms the screen reader is routing output correctly before you add the complexity of a terminal session.

\begin{center}\rule{0.5\linewidth}{0.5pt}\end{center}

\subsection*{Configuring Each Screen Reader for Braille}\label{docs__pandoc__latex__src__setup__braille_displays__braille_displays.md__configuring-each-screen-reader-for-braille}

\subsubsection*{NVDA}\label{docs__pandoc__latex__src__setup__braille_displays__braille_displays.md__nvda}

NVDA includes built-in braille support for a wide range of displays without requiring separate vendor software for many devices.

\begin{enumerate}
\tightlist
\item
  Connect your display via USB or Bluetooth.
\item
  Open the NVDA menu: press \textbf{NVDA+N} (Insert+N or CapsLock+N depending on your modifier key setting).
\item
  Go to \textbf{Preferences → Settings → Braille} (or press \textbf{NVDA+Ctrl+B} as a shortcut on some versions).
\item
  In the \textbf{"Braille display"} dropdown, select your display from the list. NVDA supports automatic detection for many displays --- you can also select \textbf{"Automatic"} and let NVDA detect it.
\item
  Set your preferred \textbf{braille output table} (e.g., Unified English Braille Grade 1 or Grade 2, or your country\textquotesingle s national table).
\item
  Set \textbf{"Braille input table"} if you want to type commands using the display\textquotesingle s braille keyboard (if equipped).
\item
  Enable \textbf{"Show cursor"} so you can see the caret position on the display.
\item
  Click \textbf{OK} and test by moving the cursor in Notepad.
\end{enumerate}

\textbf{Recommended NVDA braille settings for terminal work:}

\begin{itemize}
\tightlist
\item
  Output table: Unified English Braille Grade 1 (uncontracted) is recommended for reading code --- contractions in Grade 2 can make code harder to read.
\item
  Cursor shape: Dots 7 and 8 (underline) is a common choice.
\item
  Enable \textbf{"Word wrap"} to avoid cutting words at the display\textquotesingle s edge.
\end{itemize}

\textbf{NVDA braille keyboard shortcut reference (while braille display is active):}

{\def\LTcaptype{none} % do not increment counter
\begin{longtable}[]{@{}
  >{\raggedright\arraybackslash}p{(\linewidth - 2\tabcolsep) * \real{0.5467}}
  >{\raggedright\arraybackslash}p{(\linewidth - 2\tabcolsep) * \real{0.4533}}@{}}
\toprule\noalign{}
\begin{minipage}[b]{\linewidth}\raggedright
Action
\end{minipage} & \begin{minipage}[b]{\linewidth}\raggedright
NVDA Command
\end{minipage} \\
\midrule\noalign{}
\endhead
\bottomrule\noalign{}
\endlastfoot
Pan braille display right & Display\textquotesingle s right pan key \\
Pan braille display left & Display\textquotesingle s left pan key \\
Toggle braille tethered to focus/review & NVDA+Ctrl+T \\
Move to previous braille line & NVDA+Shift+Up (or display key) \\
Move to next braille line & NVDA+Shift+Down (or display key) \\
Open braille settings & NVDA+Ctrl+B (some versions) \\
\end{longtable}
}

\begin{center}\rule{0.5\linewidth}{0.5pt}\end{center}

\subsubsection*{JAWS}\label{docs__pandoc__latex__src__setup__braille_displays__braille_displays.md__jaws}

JAWS supports a wide range of braille displays through its built-in braille manager and additional vendor drivers.

\begin{enumerate}
\tightlist
\item
  Connect your display via USB or Bluetooth.
\item
  Install the vendor\textquotesingle s display driver if required (check the vendor\textquotesingle s website --- some displays work without a driver in JAWS, others need one).
\item
  Open the JAWS menu: press \textbf{Insert+F3} (or your JAWS key+F3).
\item
  Go to \textbf{Options → Basics → Braille} or use the \textbf{JAWS Settings Center} and search for \textbf{"Braille"}.
\item
  In Braille Settings, select your display model from the list.
\item
  Choose your \textbf{braille translation table} (e.g., English Unified Grade 1 for uncontracted, Grade 2 for contracted).
\item
  Configure \textbf{cursor routing} so that pressing a routing button on the display moves the JAWS cursor to that position.
\item
  Test in Notepad before moving to a terminal.
\end{enumerate}

\textbf{JAWS braille settings for terminal work:}

\begin{itemize}
\tightlist
\item
  Use \textbf{Grade 1 (uncontracted) braille} for reading code and terminal output. Grade 2 contractions can make commands and file paths difficult to parse.
\item
  Enable \textbf{"Show active cursor"} in JAWS braille settings.
\item
  In the JAWS Settings Center, look for \textbf{"Braille for Applications"} --- you can set different braille behavior for specific applications (e.g., a terminal vs. a word processor).
\end{itemize}

\textbf{Accessing braille settings:}

\begin{itemize}
\tightlist
\item
  \textbf{Insert+F3} → Options → Basics → Braille
\item
  Or: \textbf{Insert+F2} (JAWS Manager) → Settings Center → search "braille"
\end{itemize}

\begin{center}\rule{0.5\linewidth}{0.5pt}\end{center}

\subsubsection*{Windows Narrator}\label{docs__pandoc__latex__src__setup__braille_displays__braille_displays.md__windows-narrator}

Windows Narrator supports braille output through the \textbf{Windows braille service} (Windows 10 version 1903 and later, and Windows 11).

\begin{enumerate}
\tightlist
\item
  Connect your display via USB or Bluetooth.
\item
  Install the vendor\textquotesingle s display driver if required.
\item
  Open \textbf{Settings → Accessibility → Narrator → Braille} (Windows 11) or \textbf{Settings → Ease of Access → Narrator → Use Braille} (Windows 10).
\item
  Turn on \textbf{"Install braille"} (first-time setup may require downloading the braille translation component --- approximately 70 MB).
\item
  Select your \textbf{display brand and model} from the list.
\item
  Choose a \textbf{braille table} (Grade 1 recommended for terminal work).
\item
  Test in Notepad before using in a terminal.
\end{enumerate}

\textbf{Important Narrator braille notes:}

\begin{itemize}
\tightlist
\item
  Narrator\textquotesingle s braille support uses the \textbf{BRLTTY} back-end on Windows, which supports most mainstream displays.
\item
  If your display is not listed, check for a Windows Update or visit the display vendor\textquotesingle s website for a Windows Narrator-compatible driver.
\item
  Narrator\textquotesingle s braille customization is less extensive than NVDA\textquotesingle s or JAWS\textquotesingle s. If you need more control, consider using NVDA (free) for braille-heavy work.
\item
  The \textbf{braille display showing Narrator output} does not change the fact that Narrator\textquotesingle s keyboard commands remain the same --- your display shows what Narrator is currently reading.
\end{itemize}

\begin{center}\rule{0.5\linewidth}{0.5pt}\end{center}

\subsubsection*{Dolphin SuperNova}\label{docs__pandoc__latex__src__setup__braille_displays__braille_displays.md__dolphin-supernova}

Dolphin SuperNova includes integrated braille support as part of its screen reader editions (SuperNova Magnifier \& Speech and SuperNova Screen Reader).

\begin{enumerate}
\tightlist
\item
  Connect your display via USB or Bluetooth.
\item
  Install the vendor\textquotesingle s driver if required.
\item
  Open the \textbf{SuperNova Control Panel}: press \textbf{CapsLock+SpaceBar} or click the SuperNova icon in the taskbar.
\item
  Go to \textbf{Braille → Display} and select your display model from the list.
\item
  Go to \textbf{Braille → Translation} and select your braille table (Grade 1 / uncontracted recommended for code).
\item
  Go to \textbf{Braille → Settings} to configure cursor style, word wrap, and other options.
\item
  Click \textbf{Apply} and test in Notepad.
\end{enumerate}

\textbf{Dolphin SuperNova braille tips for terminal work:}

\begin{itemize}
\tightlist
\item
  SuperNova supports a wide range of displays from Freedom Scientific (Focus), HumanWare (Brailliant, BrailleNote as terminal), HIMS, Optelec, and others.
\item
  Use \textbf{Grade 1 (uncontracted)} braille for terminal and code work.
\item
  SuperNova\textquotesingle s \textbf{"Braille cursor tracking"} setting determines whether the braille display follows the text cursor or the screen review cursor --- set this to \textbf{"Cursor"} when working in a terminal.
\item
  Dolphin provides a \textbf{braille viewer} on screen (a visual representation of what is on the braille display) --- useful when setting up or troubleshooting.
\end{itemize}

\begin{center}\rule{0.5\linewidth}{0.5pt}\end{center}

\subsection*{Using the Braille Display in a Terminal}\label{docs__pandoc__latex__src__setup__braille_displays__braille_displays.md__using-the-braille-display-in-a-terminal}

Once your display is configured, here is how to work in a terminal:

\begin{enumerate}
\tightlist
\item
  \textbf{Reading the prompt:} The display shows your current prompt (e.g., \texttt{C:\textbackslash{}Users\textbackslash{}YourName\textgreater{}} or \texttt{PS\ C:\textbackslash{}\textgreater{}}). Pan right if the prompt is longer than your display width.
\item
  \textbf{Reading output:} After running a command, use your screen reader\textquotesingle s review cursor commands (see the Screen Reader Accessibility Guide) to move through the output. The display shows one line at a time and updates as you move.
\item
  \textbf{Redirecting long output:} For long terminal output, redirect to a file and open in Notepad:

  \begin{lstlisting}[style=Alabaster]
  command > output.txt
  notepad output.txt

  \end{lstlisting}

  In Notepad, you can pan through the braille display comfortably without the output scrolling away.
\item
  \textbf{Using routing keys:} Press the routing key above a word or character to move your cursor to that position. This is especially useful for editing commands recalled from shell history.
\end{enumerate}

\begin{center}\rule{0.5\linewidth}{0.5pt}\end{center}

\subsection*{Braille Grade and Code}\label{docs__pandoc__latex__src__setup__braille_displays__braille_displays.md__braille-grade-and-code}

For all screen readers, use \textbf{Grade 1 (uncontracted) braille} when working in terminals and with code. Contracted braille (Grade 2) uses abbreviations designed for reading prose, and those contractions will make command names, file paths, and code syntax difficult or impossible to read correctly.

Example: In Grade 2, the letters "cd" might be represented as a contraction meaning something other than the Change Directory command. In Grade 1, \texttt{cd} is always shown as the letters c and d.

\begin{center}\rule{0.5\linewidth}{0.5pt}\end{center}

\subsection*{Notetakers and BrailleNote / BrailleSense Devices}\label{docs__pandoc__latex__src__setup__braille_displays__braille_displays.md__notetakers-and-braillenote--braillesense-devices}

BrailleNote Touch+ (HumanWare) and BrailleSense (HIMS) devices run Android as their primary operating system. \textbf{They cannot run \texttt{3dMake} or Windows-based terminal workflows natively.} These devices are not supported as standalone authoring environments for this curriculum.

\textbf{Workaround --- using them as braille terminals:} Both device families can function as a braille display when connected to a Windows computer via USB or Bluetooth. In that configuration:

\begin{itemize}
\tightlist
\item
  The device acts as a standard refreshable braille display.
\item
  Your Windows screen reader (NVDA, JAWS, Narrator, or SuperNova) drives the output.
\item
  Commands are typed on the Windows keyboard; the BrailleNote/BrailleSense shows the output as braille.
\end{itemize}

To set this up, refer to your device\textquotesingle s documentation for "Terminal mode" or "PC connection mode" and then follow the screen reader braille configuration steps above for your chosen screen reader.

\begin{center}\rule{0.5\linewidth}{0.5pt}\end{center}

\subsection*{Multiline Braille Displays}\label{docs__pandoc__latex__src__setup__braille_displays__braille_displays.md__multiline-braille-displays}

Standard refreshable braille displays show a single line of braille (typically 32, 40, or 80 cells). Multiline displays show several lines simultaneously, which can significantly improve the experience of reading code or terminal output because you can see context above and below the current line.

\subsubsection*{Examples}\label{docs__pandoc__latex__src__setup__braille_displays__braille_displays.md__examples}

\textbf{Monarch (APH / HumanWare partnership):}\\
A multiline braille device developed by the American Printing House for the Blind (APH) in partnership with HumanWare. It presents multiple lines of braille and supports graphical braille output, making it useful for reviewing blocks of code, diagrams, and structured text. Confirm current driver and screen reader compatibility before purchase.

\textbf{DotPad (Dot Inc.):}\\
A multiline braille and tactile graphics display. Its multiline capability allows programmers to scan several lines of code or terminal output at once. Useful for reviewing structured output without panning repeatedly. Check current Windows compatibility and screen reader support status.

\textbf{Graphiti and Graphiti Plus (Orbit Research):}\\
Multiline tactile displays from Orbit Research designed for reading text and tactile graphics. Their expanded line count helps with reading code structure and terminal output. Check current driver support for NVDA, JAWS, Narrator, and SuperNova.

\subsubsection*{Considerations for Multiline Displays}\label{docs__pandoc__latex__src__setup__braille_displays__braille_displays.md__considerations-for-multiline-displays}

\begin{itemize}
\tightlist
\item
  \textbf{Driver and screen reader support:} Multiline displays are newer technology and support varies. Confirm that your chosen screen reader (NVDA, JAWS, Narrator, or SuperNova) has a driver for the specific device before purchase.
\item
  \textbf{Size and portability:} Multiline displays are larger and heavier than single-line models. Consider your workspace and whether you need portability.
\item
  \textbf{Cost:} Multiline displays are significantly more expensive than single-line displays. Evaluate whether the workflow benefit justifies the cost for your situation.
\item
  \textbf{Learning curve:} Using a multiline display effectively --- navigating across lines, understanding the spatial layout --- requires some practice beyond what is needed for single-line displays.
\end{itemize}

\begin{center}\rule{0.5\linewidth}{0.5pt}\end{center}

\subsection*{Troubleshooting}\label{docs__pandoc__latex__src__setup__braille_displays__braille_displays.md__troubleshooting}

\textbf{Display shows nothing / no braille output:}

\begin{itemize}
\tightlist
\item
  Verify the USB or Bluetooth connection and check that the display is powered on.
\item
  In your screen reader\textquotesingle s braille settings, confirm the correct display model is selected.
\item
  Try disconnecting and reconnecting. Restart the screen reader if needed.
\item
  Install or reinstall the vendor\textquotesingle s display driver.
\end{itemize}

\textbf{Display shows output but navigation is out of sync:}

\begin{itemize}
\tightlist
\item
  Toggle your screen reader\textquotesingle s braille mode off and on (usually in the braille settings).
\item
  Restart the screen reader (not the whole computer) and reopen the terminal.
\item
  Try a different terminal emulator (Windows Terminal, Command Prompt, PowerShell) to see if the issue is terminal-specific.
\end{itemize}

\textbf{Display driver not found:}

\begin{itemize}
\tightlist
\item
  Go to the vendor\textquotesingle s website (links below) and download the latest driver for your operating system version.
\item
  Some displays require firmware updates to work with newer Windows versions.
\end{itemize}

\textbf{Routing keys not moving the cursor:}

\begin{itemize}
\tightlist
\item
  In your screen reader\textquotesingle s braille settings, confirm that cursor routing is enabled.
\item
  In NVDA, this is under Preferences → Settings → Braille → enable "Move system cursor when routing review cursor".
\item
  In JAWS, check the routing settings in the JAWS braille configuration.
\end{itemize}

\begin{center}\rule{0.5\linewidth}{0.5pt}\end{center}

\subsection*{Vendor Resources}\label{docs__pandoc__latex__src__setup__braille_displays__braille_displays.md__vendor-resources}

{\def\LTcaptype{none} % do not increment counter
\begin{longtable}[]{@{}
  >{\raggedright\arraybackslash}p{(\linewidth - 4\tabcolsep) * \real{0.1887}}
  >{\raggedright\arraybackslash}p{(\linewidth - 4\tabcolsep) * \real{0.3962}}
  >{\raggedright\arraybackslash}p{(\linewidth - 4\tabcolsep) * \real{0.4151}}@{}}
\toprule\noalign{}
\begin{minipage}[b]{\linewidth}\raggedright
Vendor
\end{minipage} & \begin{minipage}[b]{\linewidth}\raggedright
Products
\end{minipage} & \begin{minipage}[b]{\linewidth}\raggedright
Support / Drivers
\end{minipage} \\
\midrule\noalign{}
\endhead
\bottomrule\noalign{}
\endlastfoot
HumanWare & Brailliant, BrailleNote Touch+ &
https://www.humanware.com/ \\
Freedom Scientific & Focus Blue series &
https://www.freedomscientific.com/Products/Blindness/FocusBlue/ \\
HIMS & BrailleSense, Braille EDGE, Smart Beetle &
https://www.hims-inc.com/ \\
Optelec & Braille STAR, Easy Link & https://www.optelec.com/ \\
Orbit Research & Graphiti, Orbit 20/40 &
https://www.orbitresearch.com/ \\
APH & Monarch & https://www.aph.org/ \\
Dot Inc. & DotPad & https://dotincorp.com/ \\
Dolphin & GuideConnect, braille accessories &
https://yourdolphin.com/ \\
\end{longtable}
}

\begin{center}\rule{0.5\linewidth}{0.5pt}\end{center}

\subsection*{Connecting a Braille Display: Summary Table}\label{docs__pandoc__latex__src__setup__braille_displays__braille_displays.md__connecting-a-braille-display-summary-table}

{\def\LTcaptype{none} % do not increment counter
\begin{longtable}[]{@{}
  >{\raggedright\arraybackslash}p{(\linewidth - 4\tabcolsep) * \real{0.1583}}
  >{\raggedright\arraybackslash}p{(\linewidth - 4\tabcolsep) * \real{0.5083}}
  >{\raggedright\arraybackslash}p{(\linewidth - 4\tabcolsep) * \real{0.3333}}@{}}
\toprule\noalign{}
\begin{minipage}[b]{\linewidth}\raggedright
Screen Reader
\end{minipage} & \begin{minipage}[b]{\linewidth}\raggedright
Where to Find Braille Settings
\end{minipage} & \begin{minipage}[b]{\linewidth}\raggedright
Recommended Braille Table for Code
\end{minipage} \\
\midrule\noalign{}
\endhead
\bottomrule\noalign{}
\endlastfoot
NVDA & NVDA Menu → Preferences → Settings → Braille &
Unified English Braille Grade 1 \\
JAWS & Insert+F3 → Options → Basics → Braille (or Settings Center) &
English Braille Grade 1 (uncontracted) \\
Windows Narrator & Settings → Accessibility → Narrator → Braille &
English Grade 1 (uncontracted) \\
Dolphin SuperNova &
CapsLock+SpaceBar → Braille → Display and Translation &
Grade 1 (uncontracted) \\
\end{longtable}
}

This page gives a practical overview. Always consult your braille display vendor\textquotesingle s documentation and your screen reader\textquotesingle s braille guide for device-specific setup steps and advanced configuration options.

\section{Editor Selection and Accessibility Setup Guide}\label{docs__pandoc__latex__src__setup__editor_selection_setup__editor_selection_setup.md__setup_editor_selection_setup-editor_selection_setup}

\subsection*{Table of Contents}\label{docs__pandoc__latex__src__setup__editor_selection_setup__editor_selection_setup.md__table-of-contents}

\begin{enumerate}
\tightlist
\item
  \hyperref[docs__pandoc__latex__src__setup__editor_selection_setup__editor_selection_setup.md__editor-comparison]{Editor Comparison}
\item
  \hyperref[docs__pandoc__latex__src__setup__editor_selection_setup__editor_selection_setup.md__editor-setup-for-3dmake]{Editor Setup for 3dMake}
\item
  \hyperref[docs__pandoc__latex__src__setup__editor_selection_setup__editor_selection_setup.md__screen-reader-indent-announcement-configuration]{Screen Reader Indent Announcement Configuration}
\item
  \hyperref[docs__pandoc__latex__src__setup__editor_selection_setup__editor_selection_setup.md__curriculum-specific-editor-recommendations]{Curriculum-Specific Editor Recommendations}
\end{enumerate}

\subsection*{Editor Comparison}\label{docs__pandoc__latex__src__setup__editor_selection_setup__editor_selection_setup.md__editor-comparison}

\subsubsection*{Overview Table}\label{docs__pandoc__latex__src__setup__editor_selection_setup__editor_selection_setup.md__overview-table}

{\def\LTcaptype{none} % do not increment counter
\begin{longtable}[]{@{}
  >{\raggedright\arraybackslash}p{(\linewidth - 6\tabcolsep) * \real{0.2404}}
  >{\raggedright\arraybackslash}p{(\linewidth - 6\tabcolsep) * \real{0.1635}}
  >{\raggedright\arraybackslash}p{(\linewidth - 6\tabcolsep) * \real{0.2500}}
  >{\raggedright\arraybackslash}p{(\linewidth - 6\tabcolsep) * \real{0.3462}}@{}}
\toprule\noalign{}
\begin{minipage}[b]{\linewidth}\raggedright
Feature
\end{minipage} & \begin{minipage}[b]{\linewidth}\raggedright
Notepad
\end{minipage} & \begin{minipage}[b]{\linewidth}\raggedright
Notepad++
\end{minipage} & \begin{minipage}[b]{\linewidth}\raggedright
Visual Studio Code
\end{minipage} \\
\midrule\noalign{}
\endhead
\bottomrule\noalign{}
\endlastfoot
Cost & Free (built-in) & Free & Free \\
Learning Curve & Minimal & Low & Moderate \\
Screen Reader Support & Good (basic) & Good (syntax features) &
Excellent (built-in accessibility) \\
Extension/Plugin System & None & Limited & Extensive \\
Keyboard Navigation & Good & Good & Excellent \\
Customization & None & Moderate & Very high \\
Performance & Excellent & Very good & Good \\
Syntax Highlighting & None & Yes (OpenSCAD available) &
Yes (OpenSCAD available) \\
Terminal Integration & None & None & Built-in \\
Real-time Feedback & None & None & Yes (with extensions) \\
Hot Key Customization & Limited & Good & Excellent \\
File Size Handling & Good & Good & Excellent \\
Built-in Debugging & None & None & Limited \\
\end{longtable}
}

\subsubsection*{Detailed Comparison}\label{docs__pandoc__latex__src__setup__editor_selection_setup__editor_selection_setup.md__detailed-comparison}

\paragraph*{Notepad}\label{docs__pandoc__latex__src__setup__editor_selection_setup__editor_selection_setup.md__notepad}

Advantages:

\begin{itemize}
\tightlist
\item
  Minimal interface with no distractions-excellent for absolute beginners
\item
  Very predictable behavior for screen reader users
\item
  Extremely fast file operations
\item
  No configuration required; works immediately
\item
  Pure text editing with no formatting surprises
\end{itemize}

Disadvantages:

\begin{itemize}
\tightlist
\item
  No syntax highlighting (OpenSCAD code appears as plain text)
\item
  No keyboard shortcuts for common editing tasks
\item
  Limited undo/redo capabilities compared to modern editors
\item
  No integrated terminal (requires separate command prompt window)
\item
  No way to run 3dm commands directly
\end{itemize}

Best For: Users who prefer absolute simplicity and want minimal cognitive load during learning phase

Screen Reader Experience: Windows Narrator reads all content clearly; JAWS and NVDA work well with standard keyboard navigation

\paragraph*{Notepad++}\label{docs__pandoc__latex__src__setup__editor_selection_setup__editor_selection_setup.md__notepad-1}

Advantages:

\begin{itemize}
\tightlist
\item
  Lightweight and fast
\item
  Good syntax highlighting for OpenSCAD code
\item
  Customizable interface with configurable keyboard shortcuts
\item
  Tab management for multiple files
\item
  Better organization than Notepad for managing projects
\item
  Good screen reader support for basic operations
\end{itemize}

Disadvantages:

\begin{itemize}
\tightlist
\item
  No built-in terminal (requires external command prompt)
\item
  Plugin system is limited compared to VSCode
\item
  Screen reader experience with syntax highlighting features can be inconsistent
\item
  Less extensive keyboard customization than VSCode
\item
  No integrated development environment features
\end{itemize}

Best For: Users who want a lightweight editor with syntax highlighting but prefer not to use a full IDE

Screen Reader Experience: JAWS and NVDA handle navigation well; Windows Narrator works but may struggle with complex UI elements

\paragraph*{Visual Studio Code}\label{docs__pandoc__latex__src__setup__editor_selection_setup__editor_selection_setup.md__visual-studio-code}

Advantages:

\begin{itemize}
\tightlist
\item
  Extensive built-in accessibility features (accessibility inspector, keyboard navigation shortcuts)
\item
  Excellent OpenSCAD extension available (scad-preview)
\item
  Integrated terminal allows running 3dm commands without context switching
\item
  Powerful keyboard shortcut customization
\item
  Rich extension ecosystem for workflow enhancement
\item
  Remote development capabilities
\item
  Native support for multiple projects and workspaces
\item
  Excellent search and find/replace functionality
\end{itemize}

Disadvantages:

\begin{itemize}
\tightlist
\item
  Steeper learning curve than Notepad or Notepad++
\item
  Requires initial configuration for accessibility
\item
  More resource-intensive than lighter editors
\item
  Built-in terminal can be distracting for some users (alternative: Alt-Tab to standalone terminal)
\end{itemize}

Best For: Users building comprehensive accessibility workflow and wanting integrated development environment

Screen Reader Experience: Excellent; built specifically with accessibility in mind. NVDA, JAWS, Windows Narrator, and Dolphin all receive high-quality support.

\subsection*{Editor Setup for 3dMake}\label{docs__pandoc__latex__src__setup__editor_selection_setup__editor_selection_setup.md__editor-setup-for-3dmake}

\subsubsection*{Setting Default Editor in 3dMake}\label{docs__pandoc__latex__src__setup__editor_selection_setup__editor_selection_setup.md__setting-default-editor-in-3dmake}

The 3dMake tool uses your system default text editor. However it can be changed with an edit to the global configuration file by typing this into any terminal.

\begin{lstlisting}[style=Alabaster]
3dm edit-global-config

\end{lstlisting}

You get a file like this in your default text edit program, typically notepad on Windows.

\begin{lstlisting}[style=Alabaster]
view = "3sil"
model_name = "main"
auto_Start_prints = true
printer_profile = "bambu_labs_X1_carbon"

\end{lstlisting}

Add one of the following lines to the end of the file:

\texttt{editor\ =\ "code"} for VSCode
\texttt{editor\ =\ "notepad"} for notepad (already the default)
\texttt{editor\ =\ \textquotesingle{}\textquotesingle{}\textquotesingle{}C:\textbackslash{}Program\ Files\ (x86)\textbackslash{}Notepad++\textbackslash{}notepad++.exe\textquotesingle{}\textquotesingle{}\textquotesingle{}} for Notepad++ (the triple apostrophes mean you do not have to escape the spaces)

\subsubsection*{Notepad Setup}\label{docs__pandoc__latex__src__setup__editor_selection_setup__editor_selection_setup.md__notepad-setup}

Configuration Steps:

\begin{enumerate}
\item
  Open 3dMake-generated .scad files directly:

  \begin{itemize}
  \tightlist
  \item
    Navigate to your project folder in File Explorer
  \item
    Right-click the \texttt{.scad} file -\textgreater{} "Open with" -\textgreater{} "Notepad"
  \item
    File opens immediately for editing
  \end{itemize}
\item
  Edit and Save:

  \begin{itemize}
  \tightlist
  \item
    Make changes to your code
  \item
    Press \texttt{Ctrl+S} to save
  \item
    File updates automatically if 3dMake renders in background
  \end{itemize}
\item
  Run 3dMake Commands:

  \begin{itemize}
  \tightlist
  \item
    Save your edits (Ctrl+S)
  \item
    Alt-Tab to Command Prompt or PowerShell window
  \item
    Navigate to project directory: \texttt{cd\ C:\textbackslash{}path\textbackslash{}to\textbackslash{}project}
  \item
    Run 3dMake command: \texttt{3dMake\ render\ project.scad}
  \end{itemize}
\end{enumerate}

Keyboard Shortcuts:

\begin{itemize}
\tightlist
\item
  \texttt{Ctrl+H}: Find and Replace
\item
  \texttt{Ctrl+G}: Go to Line (newer versions)
\item
  \texttt{Ctrl+Z}: Undo
\item
  \texttt{Ctrl+Y}: Redo
\end{itemize}

\subsubsection*{Notepad++ Setup}\label{docs__pandoc__latex__src__setup__editor_selection_setup__editor_selection_setup.md__notepad-setup-1}

Installation and Configuration:

\begin{enumerate}
\item
  Download from: \url{https://notepad-plus-plus.org/downloads/}
\item
  Choose: Standard Installer (for automatic system integration)
\item
  Configure OpenSCAD Language Support:

  \begin{itemize}
  \tightlist
  \item
    Open Notepad++
  \item
    Language -\textgreater{} User Defined Language -\textgreater{} Import... (if OpenSCAD UDL available)
  \item
    Or manually set syntax highlighting:

    \begin{itemize}
    \tightlist
    \item
      Language -\textgreater{} OpenSCAD (if available in language menu)
    \item
      Otherwise Language -\textgreater{} C++ (provides similar highlighting)
    \end{itemize}
  \end{itemize}
\item
  Customize for Accessibility:

  \begin{itemize}
  \tightlist
  \item
    Settings -\textgreater{} Preferences -\textgreater{} General
  \item
    Check: "Minimize to system tray" (optional)
  \item
    Settings -\textgreater{} Preferences -\textgreater{} MISC.
  \item
    Ensure Word Wrap is set to preference
  \item
    Settings -\textgreater{} Preferences -\textgreater{} Backup
  \item
    Enable regular backups of your work
  \end{itemize}
\item
  Set as Default Editor (see registry method above)
\end{enumerate}

Recommended Keyboard Shortcuts (User-Defined):

Create custom shortcuts by:

\begin{itemize}
\tightlist
\item
  Settings -\textgreater{} Shortcut Mapper
\item
  Add shortcuts for frequently used actions:

  \begin{itemize}
  \tightlist
  \item
    Save and Switch to Terminal: Alt+T
  \item
    Copy File Path: Ctrl+Shift+C
  \end{itemize}
\end{itemize}

Tab Management:

\begin{itemize}
\tightlist
\item
  Open multiple files: Each opens in a separate tab
\item
  Switch between tabs: Ctrl+Tab (next) / Ctrl+Shift+Tab (previous)
\item
  Close tab: Ctrl+W
\end{itemize}

Running 3dMake Commands:

\begin{itemize}
\tightlist
\item
  Save file with Ctrl+S
\item
  Alt-Tab to standalone terminal or command prompt
\item
  Run: \texttt{3dMake\ render\ filename.scad}
\end{itemize}

\subsubsection*{Visual Studio Code Setup}\label{docs__pandoc__latex__src__setup__editor_selection_setup__editor_selection_setup.md__visual-studio-code-setup}

Installation:

\begin{enumerate}
\tightlist
\item
  Download from: \url{https://code.visualstudio.com/} or type \texttt{winget\ search\ VSCode} in a terminal and then \texttt{winget\ install\ } whichever option you prefer
\item
  Install: Run installer and follow prompts
\item
  Launch: Open VSCode
\end{enumerate}

Enable OpenSCAD Support:

\begin{enumerate}
\tightlist
\item
  Open Extensions (Ctrl+Shift+X)
\item
  Search: "OpenSCAD"
\item
  Install: "scad-preview" by Antyos
\item
  Install: "OpenSCAD Syntax Highlighter" (optional, for better syntax highlighting)
\end{enumerate}

Initial Accessibility Configuration:

\begin{enumerate}
\item
  Open Settings: Ctrl+,
\item
  Search: "accessibility"
\item
  Enable Key Settings:

  \begin{itemize}
  \tightlist
  \item
    Accessible View: Toggle ON
  \item
    Screen Reader: Select your screen reader (NVDA, JAWS, Narrator, Dolphin)
  \item
    Keyboard Navigation: Ensure enabled
  \item
    Bracket Pair Guides: Can help with code structure understanding
  \end{itemize}
\item
  Configure Editor Font:

  \begin{itemize}
  \tightlist
  \item
    Search: "editor.fontSize"
  \item
    Set to comfortable size (recommend 14-16 for better readability)
  \item
    Search: "editor.fontFamily"
  \item
    Select monospace font (e.g., "Consolas" or "Courier New")
  \end{itemize}
\end{enumerate}

Set as Default Editor (see registry method above)

VSCode Terminal Options:

\paragraph*{Option 1: Using Built-in Terminal (Less Accessible)}\label{docs__pandoc__latex__src__setup__editor_selection_setup__editor_selection_setup.md__option-1-using-built-in-terminal-less-accessible}

\begin{enumerate}
\tightlist
\item
  View -\textgreater{} Terminal (or Ctrl+`)
\item
  Terminal opens at bottom of VSCode window
\item
  Run commands: \texttt{3dMake\ render\ filename.scad}
\item
  Note: Switching focus between editor and terminal requires Tab navigation, which can be cumbersome for screen reader users
\end{enumerate}

Keyboard Navigation:

\begin{itemize}
\tightlist
\item
  Ctrl+` : Toggle terminal visibility
\item
  Ctrl+Shift+` : Create new terminal
\item
  Alt+\^{}/v : Switch between terminals
\end{itemize}

\paragraph*{Option 2: Alt-Tab to Standalone Terminal (RECOMMENDED for Accessibility)}\label{docs__pandoc__latex__src__setup__editor_selection_setup__editor_selection_setup.md__option-2-alt-tab-to-standalone-terminal-recommended-for-accessibility}

\begin{enumerate}
\item
  Keep Command Prompt/PowerShell open:

  \begin{itemize}
  \tightlist
  \item
    Open Command Prompt (Win+R, type "cmd", Enter)
  \item
    Position window or minimize to taskbar
  \item
    Navigate to project directory: \texttt{cd\ C:\textbackslash{}path\textbackslash{}to\textbackslash{}project}
  \end{itemize}
\item
  From VSCode, switch terminals:

  \begin{itemize}
  \tightlist
  \item
    Alt+Tab to Command Prompt
  \item
    Run command: \texttt{3dMake\ render\ filename.scad}
  \item
    Alt+Tab back to VSCode
  \item
    Continue editing
  \end{itemize}
\end{enumerate}

Why This Is More Accessible:

\begin{itemize}
\tightlist
\item
  Screen reader focus switches clearly between two applications
\item
  Terminal output is read without VSCode context interference
\item
  Cleaner context switching for command-line workflows
\item
  Easier to diagnose issues when editor and terminal are separate
\end{itemize}

Keyboard Shortcuts for Common Tasks:

{\def\LTcaptype{none} % do not increment counter
\begin{longtable}[]{@{}ll@{}}
\toprule\noalign{}
Action & Shortcut \\
\midrule\noalign{}
\endhead
\bottomrule\noalign{}
\endlastfoot
Save & Ctrl+S \\
Find & Ctrl+F \\
Find and Replace & Ctrl+H \\
Go to Line & Ctrl+G \\
Open File & Ctrl+O \\
Open Folder & Ctrl+K, Ctrl+O \\
Open Terminal & Ctrl+` \\
Alt-Tab to Another Window & Alt+Tab \\
Command Palette & Ctrl+Shift+P \\
\end{longtable}
}

Project Organization in VSCode:

\begin{enumerate}
\item
  Open Project Folder:

  \begin{itemize}
  \tightlist
  \item
    File -\textgreater{} Open Folder (Ctrl+K, Ctrl+O)
  \item
    Select your project directory
  \item
    All project files appear in Explorer sidebar
  \end{itemize}
\item
  File Navigation:

  \begin{itemize}
  \tightlist
  \item
    Press Ctrl+P for Quick Open
  \item
    Type filename to search and jump to file
  \item
    Press Enter to open
  \end{itemize}
\item
  Quick Switch Between Files:

  \begin{itemize}
  \tightlist
  \item
    Ctrl+Tab : Open recent files list
  \item
    Arrow keys to select
  \item
    Enter to open
  \end{itemize}
\end{enumerate}

\subsection*{Screen Reader Indent Announcement Configuration}\label{docs__pandoc__latex__src__setup__editor_selection_setup__editor_selection_setup.md__screen-reader-indent-announcement-configuration}

Proper indent announcement is critical for OpenSCAD development, as indentation indicates code nesting and structure.

\subsubsection*{NVDA (NonVisual Desktop Access)}\label{docs__pandoc__latex__src__setup__editor_selection_setup__editor_selection_setup.md__nvda-nonvisual-desktop-access}

Enable Indent Announcement:

\begin{enumerate}
\tightlist
\item
  Open NVDA Menu: Alt+N or right-click NVDA icon
\item
  Preferences -\textgreater{} Settings (Ctrl+Comma)
\item
  Document Formatting: Tab to it
\item
  Check: "Report indentation"
\item
  In the "Indentation reporting" dropdown: Select "Tones and speech"
\item
  Tone Description: NVDA will announce indent level as progressively higher tones (or speaking indent amount)
\item
  Apply: Click OK
\end{enumerate}

Additional Tab Stop Configuration:

\begin{enumerate}
\tightlist
\item
  Preferences -\textgreater{} Settings -\textgreater{} Document Formatting
\item
  Check: "Report line indentation"
\item
  This will announce: "Indent level 4" or similar as you navigate code
\end{enumerate}

Testing:

\begin{itemize}
\tightlist
\item
  Open a \texttt{.scad} file with nested code (e.g., difference() \{ cube(); sphere(); \})
\item
  Press Down Arrow to move line by line
\item
  NVDA announces indentation level on each new indented line
\end{itemize}

Keyboard Control:

\begin{itemize}
\tightlist
\item
  NVDA+3 on Numpad: Cycles indent/outline level reporting
\end{itemize}

\subsubsection*{JAWS (Freedom Scientific)}\label{docs__pandoc__latex__src__setup__editor_selection_setup__editor_selection_setup.md__jaws-freedom-scientific}

Enable Indent Announcement:

\begin{enumerate}
\tightlist
\item
  Open JAWS Manager: Press JAWSKey+F2 (or right-click JAWS icon)
\item
  Utilities -\textgreater{} Settings Manager
\item
  Search: "Indent"
\item
  Look for setting: "Announce Indentation" or "Report Indentation"
\item
  Enable: Set to "Tones" or "Tones and Speech"
\item
  Speech Indent Announcement: Speak indent level
\item
  Tone Indent Announcement: Pitch increases with indent level
\end{enumerate}

Advanced Configuration (Custom Scripts):

If built-in settings don\textquotesingle t work:

\begin{enumerate}
\tightlist
\item
  JAWSKey+F2 -\textgreater{} Utilities -\textgreater{} Settings Manager
\item
  Search: "Line Breaks" or "Formatting"
\item
  Ensure: "Report line indentation" is enabled
\item
  Set tone adjustment: Higher pitch for deeper indents
\end{enumerate}

Testing:

\begin{itemize}
\tightlist
\item
  Open \texttt{.scad} file with nested code
\item
  Navigate with arrow keys
\item
  JAWS announces indent changes with tones or speech
\end{itemize}

Keyboard Shortcuts:

\begin{itemize}
\tightlist
\item
  JAWSKey+Alt+I: Toggle indent reporting
\item
  JAWSKey+Alt+Shift+I: Cycle between indent reporting modes (speech/tones/off)
\end{itemize}

\subsubsection*{Windows Narrator}\label{docs__pandoc__latex__src__setup__editor_selection_setup__editor_selection_setup.md__windows-narrator}

Enable Indent Announcement:

\begin{enumerate}
\tightlist
\item
  Open Settings: Win+I
\item
  Ease of Access -\textgreater{} Narrator
\item
  Advanced Options: Scroll down
\item
  Check: "Report indentation"
\item
  Indentation Reporting: Select "Tones" (less intrusive) or "Speech" (explicit)
\item
  Apply settings
\end{enumerate}

Narrator Keyboard Shortcuts:

\begin{itemize}
\tightlist
\item
  Narrator+Page Down: Read from current position to end of window
\item
  Narrator+Alt+Arrow Keys: Navigate text
\item
  Narrator+V, I: Customize indentation reporting (in Narrator settings)
\end{itemize}

Testing:

\begin{itemize}
\tightlist
\item
  Open \texttt{.scad} file
\item
  Use Narrator+Page Down to read through code
\item
  Listen for indent tone changes or announcements
\end{itemize}

Note: Windows Narrator has fewer customization options than JAWS/NVDA; consider NVDA or JAWS for deeper indent control

\subsubsection*{Dolphin EasyConverter (Dolphin Screen Reader)}\label{docs__pandoc__latex__src__setup__editor_selection_setup__editor_selection_setup.md__dolphin-easyconverter-dolphin-screen-reader}

Enable Indent Announcement:

\begin{enumerate}
\tightlist
\item
  Open Dolphin Central: Right-click Dolphin icon or click Dolphin icon in taskbar
\item
  Utilities -\textgreater{} Settings -\textgreater{} Text Processing
\item
  Look for: "Indentation" section
\item
  Enable: "Announce indentation"
\item
  Mode: Select "Tones", "Speech", or "Tones and Speech"
\item
  Tone Pitch: Configure pitch increase for deeper indents
\item
  Apply
\end{enumerate}

ECO (Ease of Cursor Operation) Customization:

\begin{enumerate}
\tightlist
\item
  Dolphin Central -\textgreater{} Utilities -\textgreater{} ECO Settings
\item
  Text Options -\textgreater{} Indentation Reporting
\item
  Set preferred announcement style
\end{enumerate}

Testing:

\begin{itemize}
\tightlist
\item
  Open \texttt{.scad} file with indented code
\item
  Navigate with arrows
\item
  Dolphin announces indent changes
\end{itemize}

Keyboard Shortcuts:

\begin{itemize}
\tightlist
\item
  Ctrl+Dolphin+I: Toggle indent reporting on/off
\item
  Ctrl+Dolphin+Shift+I: Cycle indent reporting mode
\end{itemize}

\subsection*{Curriculum-Specific Editor Recommendations}\label{docs__pandoc__latex__src__setup__editor_selection_setup__editor_selection_setup.md__curriculum-specific-editor-recommendations}

\subsubsection*{For Absolute Beginners (Lesson 1-2)}\label{docs__pandoc__latex__src__setup__editor_selection_setup__editor_selection_setup.md__for-absolute-beginners-lesson-1-2}

Recommended: Notepad or Notepad++

Rationale:

\begin{itemize}
\tightlist
\item
  Minimal interface reduces cognitive load
\item
  Focus stays on learning OpenSCAD syntax, not editor features
\item
  Keyboard navigation is straightforward
\item
  Screen reader experience is predictable
\end{itemize}

Setup:

\begin{enumerate}
\tightlist
\item
  Use Notepad or Notepad++ as default editor
\item
  Configure screen reader indent announcement
\item
  Keep separate Command Prompt window open for 3dMake commands
\item
  Alt-Tab workflow between editor and terminal
\end{enumerate}

Workflow Example:

\begin{lstlisting}[style=Alabaster]
1. Open Command Prompt -> Navigate to project folder
2. Run: 3dMake new myproject
3. Alt+Tab to file explorer, open myproject.scad
4. Notepad++ opens file
5. Edit code
6. Ctrl+S to save
7. Alt+Tab to Command Prompt
8. Run: 3dMake render myproject.scad
9. Check output, return to Notepad++ to refine code

\end{lstlisting}

\subsubsection*{For Intermediate Users (Lesson 3-6)}\label{docs__pandoc__latex__src__setup__editor_selection_setup__editor_selection_setup.md__for-intermediate-users-lesson-3-6}

Recommended: Notepad++ or VSCode

Notepad++:

\begin{itemize}
\tightlist
\item
  Adds project organization without overwhelming complexity
\item
  Tab support for managing multiple files
\item
  Syntax highlighting improves code understanding
\item
  Still lightweight and predictable
\end{itemize}

VSCode:

\begin{itemize}
\tightlist
\item
  Opens doors to more sophisticated workflows
\item
  Extension system enables advanced features (OpenSCAD preview)
\item
  Keyboard customization becomes valuable
\item
  Terminal integration useful but use Alt-Tab method
\end{itemize}

Setup Decision Tree:

\begin{itemize}
\tightlist
\item
  Choose Notepad++ if: You prefer simplicity and want to focus on code logic
\item
  Choose VSCode if: You\textquotesingle re ready to invest time learning editor features for long-term benefit
\end{itemize}

Workflow Example with VSCode:

\begin{lstlisting}[style=Alabaster]
1. Open VSCode (folder view of project)
2. Press Ctrl+P to open file search
3. Type filename and press Enter to open
4. Edit code with autocomplete
5. Ctrl+H for find/replace across project
6. Ctrl+S to save
7. Alt+Tab to Command Prompt
8. Run: 3dMake render filename.scad
9. Alt+Tab back to VSCode to refine code

\end{lstlisting}

\subsubsection*{For Advanced Users (Lesson 7-11)}\label{docs__pandoc__latex__src__setup__editor_selection_setup__editor_selection_setup.md__for-advanced-users-lesson-7-11}

Recommended: Visual Studio Code

Rationale:

\begin{itemize}
\tightlist
\item
  Powerful search/replace across large projects
\item
  Extension system enables specialized workflows
\item
  Keyboard customization reaches full potential
\item
  Workspace management for complex projects
\item
  Debugging capabilities aid troubleshooting
\end{itemize}

Advanced Setup:

\begin{enumerate}
\item
  Create custom keyboard shortcuts:

  \begin{itemize}
  \tightlist
  \item
    Ctrl+Alt+R: Save and render current file
  \item
    Ctrl+Alt+P: Preview (if using scad-preview extension)
  \end{itemize}
\item
  Install Additional Extensions:

  \begin{itemize}
  \tightlist
  \item
    "scad-preview": Real-time 3D preview
  \item
    "Better Comments": Categorize comments with colors/tones
  \item
    "Bracket Pair Colorizer": Visual/tonal bracket matching
  \item
    "GitLens": Track code changes over time
  \end{itemize}
\item
  Create Task Runner for Common Commands:

  \begin{itemize}
  \tightlist
  \item
    Ctrl+Shift+B: Configure build task to run 3dMake
  \item
    Create separate tasks for render, export, etc.
  \end{itemize}
\item
  Use Workspaces:

  \begin{itemize}
  \tightlist
  \item
    File -\textgreater{} Save Workspace As...
  \item
    Save project-specific workspace with all settings
  \item
    Reopen same workspace configuration automatically
  \end{itemize}
\end{enumerate}

Advanced Workflow Example:

\begin{lstlisting}[style=Alabaster]
1. Open VSCode with project workspace
2. Ctrl+Shift+P -> Run Task -> "3dMake Render Current"
3. Renders file and shows output
4. Use scad-preview extension for real-time 3D view
5. Edit code with advanced search/replace
6. Ctrl+Alt+R saves and renders automatically
7. Use version control (Git) for tracking changes

\end{lstlisting}

\subsection*{Quick Reference: Editor Comparison for Curriculum}\label{docs__pandoc__latex__src__setup__editor_selection_setup__editor_selection_setup.md__quick-reference-editor-comparison-for-curriculum}

\subsubsection*{Foundations (Lessons 1-2)}\label{docs__pandoc__latex__src__setup__editor_selection_setup__editor_selection_setup.md__foundations-lessons-1-2}

\begin{itemize}
\tightlist
\item
  Primary: Notepad or Notepad++
\item
  Focus: Learn syntax and basic concepts
\item
  Terminal: Standalone Command Prompt (Alt-Tab)
\end{itemize}

\subsubsection*{Core Skills (Lessons 3-6)}\label{docs__pandoc__latex__src__setup__editor_selection_setup__editor_selection_setup.md__core-skills-lessons-3-6}

\begin{itemize}
\tightlist
\item
  Primary: Notepad++ or VSCode
\item
  New Features: Begin using editor syntax highlighting
\item
  Terminal: Standalone (continue Alt-Tab method)
\item
  Skills: File organization, search/replace basics
\end{itemize}

\subsubsection*{Advanced Projects (Lessons 7-11)}\label{docs__pandoc__latex__src__setup__editor_selection_setup__editor_selection_setup.md__advanced-projects-lessons-7-11}

\begin{itemize}
\tightlist
\item
  Primary: VSCode (strongly recommended)
\item
  Advanced: Use extensions, real-time preview, complex project management
\item
  Terminal: Choose Alt-Tab or built-in based on preference
\item
  Skills: Workspaces, task automation, version control
\end{itemize}

\subsection*{Troubleshooting Common Issues}\label{docs__pandoc__latex__src__setup__editor_selection_setup__editor_selection_setup.md__troubleshooting-common-issues}

\subsubsection*{Problem: Screen Reader Not Announcing Indent}\label{docs__pandoc__latex__src__setup__editor_selection_setup__editor_selection_setup.md__problem-screen-reader-not-announcing-indent}

Solution:

\begin{enumerate}
\tightlist
\item
  Verify indent announcement is enabled in screen reader settings (see above)
\item
  Test with existing \texttt{.scad} file with clear indentation
\item
  Try different announcement modes (tones vs. speech)
\item
  Restart screen reader: Alt+Ctrl+N (NVDA) or app restart (JAWS)
\end{enumerate}

\subsubsection*{Problem: 3dMake Commands Not Running from VSCode Terminal}\label{docs__pandoc__latex__src__setup__editor_selection_setup__editor_selection_setup.md__problem-3dmake-commands-not-running-from-vscode-terminal}

Solution:

\begin{enumerate}
\tightlist
\item
  Ensure 3dMake is in your system PATH
\item
  Use standalone terminal instead (Alt-Tab method) - more reliable
\item
  In VSCode terminal, manually navigate to correct directory first
\item
  Verify command syntax: \texttt{3dMake\ render\ filename.scad}
\end{enumerate}

\subsubsection*{Problem: File Not Saving in Editor}\label{docs__pandoc__latex__src__setup__editor_selection_setup__editor_selection_setup.md__problem-file-not-saving-in-editor}

Solution:

\begin{enumerate}
\tightlist
\item
  Verify you pressed Ctrl+S
\item
  Check file permissions on project folder
\item
  Try "Save As" instead
\item
  Ensure filename includes \texttt{.scad} extension
\end{enumerate}

\subsubsection*{Problem: Syntax Highlighting Not Working}\label{docs__pandoc__latex__src__setup__editor_selection_setup__editor_selection_setup.md__problem-syntax-highlighting-not-working}

Solution:

\begin{enumerate}
\tightlist
\item
  Verify file has \texttt{.scad} extension
\item
  In Notepad++: Language menu -\textgreater{} select OpenSCAD or C++
\item
  In VSCode: Install OpenSCAD extension (search Extensions)
\item
  Restart editor
\end{enumerate}

\subsubsection*{Problem: Alt-Tab Not Switching Between Windows}\label{docs__pandoc__latex__src__setup__editor_selection_setup__editor_selection_setup.md__problem-alt-tab-not-switching-between-windows}

Solution:

\begin{enumerate}
\tightlist
\item
  Ensure Command Prompt is open and minimized (not closed)
\item
  Press Alt+Tab and hold briefly to see window switcher
\item
  Use Alt+Tab multiple times if more than 2 windows open
\item
  Alternatively, click taskbar directly (Alt+Tab usually more accessible)
\end{enumerate}

\subsection*{Next Steps}\label{docs__pandoc__latex__src__setup__editor_selection_setup__editor_selection_setup.md__next-steps}

After completing this setup guide:

\begin{enumerate}
\tightlist
\item
  Choose your editor based on the recommendations for your skill level
\item
  Configure screen reader indent announcement immediately (critical for code structure understanding)
\item
  Set editor as default for \texttt{.scad} files
\item
  Test with a simple file: Create a test project and edit it
\item
  Practice Alt-Tab workflow before moving to Lesson 1
\item
  Document your setup in a personal note for reference
\end{enumerate}

You are now ready to begin Lesson 1: Environmental Configuration and Developer Workflow

\subsection*{Additional Resources}\label{docs__pandoc__latex__src__setup__editor_selection_setup__editor_selection_setup.md__additional-resources}

\begin{itemize}
\tightlist
\item
  NVDA Documentation: \url{https://www.nvaccess.org/documentation/}
\item
  JAWS Documentation: \url{https://www.freedomscientific.com/products/software/jaws/}
\item
  VSCode Accessibility: \url{https://code.visualstudio.com/docs/editor/accessibility}
\item
  Windows Narrator Guide: \url{https://support.microsoft.com/en-us/help/22798/windows-11-narrator-get-started}
\item
  Notepad++ Documentation: \url{https://notepad-plus-plus.org/online-help/}
\end{itemize}

\chapter{Command-Line Fundamentals - Choose Your Path}\label{docs__pandoc__latex__src__command_line_interface_selection__command_line_interface_selection.md__command_line_interface_selection-command_line_interface_selection}

Welcome! Before diving into 3D design with OpenSCAD, you\textquotesingle ll master command-line fundamentals. This page will help you understand what command-line interfaces are and choose the best path for you.

\section*{What is a Command-Line Interface (CLI)?}\label{docs__pandoc__latex__src__command_line_interface_selection__command_line_interface_selection.md__what-is-a-command-line-interface-cli}

A command-line interface is a text-based way to control your computer by typing commands instead of clicking buttons. It\textquotesingle s like sending written instructions to your computer.

\subsection*{Why learn it?}\label{docs__pandoc__latex__src__command_line_interface_selection__command_line_interface_selection.md__why-learn-it}

\begin{itemize}
\tightlist
\item
  Speed: Text commands are often faster than clicking through menus
\item
  Precision: Exact control over what your computer does
\item
  Accessibility: Perfect for screen readers - text is naturally readable
\item
  Automation: Repeat tasks automatically
\item
  3D Printing: Essential for batch processing models and integrating tools
\end{itemize}

\subsection*{Real-world example}\label{docs__pandoc__latex__src__command_line_interface_selection__command_line_interface_selection.md__real-world-example}

Instead of:

\begin{enumerate}
\tightlist
\item
  Opening File Explorer (click)
\item
  Navigating folders (click, click, click)
\item
  Right-clicking a file (click)
\item
  Selecting "Copy" (click)
\item
  Navigating to destination (click, click)
\item
  Right-clicking (click)
\item
  Selecting "Paste" (click)
\end{enumerate}

You type: \texttt{cp\ myfile.txt\ backup/} and press Enter. Done.

\section*{Three Command-Line Options on Windows}\label{docs__pandoc__latex__src__command_line_interface_selection__command_line_interface_selection.md__three-command-line-options-on-windows}

Windows offers three ways to use the command line. All are accessible with screen readers. Here\textquotesingle s how they compare:

\subsection*{Option 1: Windows Command Prompt (CMD)}\label{docs__pandoc__latex__src__command_line_interface_selection__command_line_interface_selection.md__option-1-windows-command-prompt-cmd}

What it is: The original Windows command-line (1981-present)

Best for: Absolute beginners, maximum simplicity

Pros:

\begin{itemize}
\tightlist
\item
  Simple commands and syntax
\item
  Minimal learning curve
\item
  Easy to understand error messages
\item
  Great for basic file operations
\item
  Perfect entry point to command-line world
\end{itemize}

Cons:

\begin{itemize}
\tightlist
\item
  Limited advanced features
\item
  Less powerful than alternatives
\item
  No built-in piping (but available)
\item
  Smaller ecosystem
\end{itemize}

Typical command:

\begin{lstlisting}[style=Alabaster, language=cmd]
copy myfile.txt backup\

\end{lstlisting}

\subsection*{Option 2: PowerShell}\label{docs__pandoc__latex__src__command_line_interface_selection__command_line_interface_selection.md__option-2-powershell}

What it is: Microsoft\textquotesingle s modern, powerful shell (2006-present)

Best for: Intermediate users, advanced automation

Pros:

\begin{itemize}
\tightlist
\item
  Very powerful for scripting
\item
  Modern syntax and features
\item
  Excellent for 3D printing automation
\item
  Professional workflows
\item
  Large community
\end{itemize}

Cons:

\begin{itemize}
\tightlist
\item
  Steeper learning curve than CMD
\item
  More complex syntax
\item
  More "wordy" commands
\item
  Overkill for simple tasks
\end{itemize}

Typical command:

\begin{lstlisting}[style=Alabaster, language=powershell]
Copy-Item -Path myfile.txt -Destination backup/

\end{lstlisting}

\subsection*{Option 3: Git Bash}\label{docs__pandoc__latex__src__command_line_interface_selection__command_line_interface_selection.md__option-3-git-bash}

What it is: A Unix/Linux shell on Windows (runs bash inside Git for Windows)

Best for: Programmers, users familiar with Linux, advanced users

Pros:

\begin{itemize}
\tightlist
\item
  Familiar if you know Linux/Unix
\item
  Powerful piping and text processing
\item
  Consistent with other platforms (macOS, Linux)
\item
  Excellent for advanced workflows
\item
  Industry-standard for developers
\end{itemize}

Cons:

\begin{itemize}
\tightlist
\item
  Requires Git installation
\item
  Steeper learning curve
\item
  Path syntax is different from native Windows
\item
  Less integrated with Windows system tools
\item
  May be "too much" for beginners
\end{itemize}

Typical command:

\begin{lstlisting}[style=Alabaster, language=bash]
cp myfile.txt backup/

\end{lstlisting}

\section*{Command Comparison Table}\label{docs__pandoc__latex__src__command_line_interface_selection__command_line_interface_selection.md__command-comparison-table}

Here\textquotesingle s how common tasks compare across the three options:

{\def\LTcaptype{none} % do not increment counter
\begin{longtable}[]{@{}
  >{\raggedright\arraybackslash}p{(\linewidth - 6\tabcolsep) * \real{0.2110}}
  >{\raggedright\arraybackslash}p{(\linewidth - 6\tabcolsep) * \real{0.2477}}
  >{\raggedright\arraybackslash}p{(\linewidth - 6\tabcolsep) * \real{0.3028}}
  >{\raggedright\arraybackslash}p{(\linewidth - 6\tabcolsep) * \real{0.2385}}@{}}
\toprule\noalign{}
\begin{minipage}[b]{\linewidth}\raggedright
Task
\end{minipage} & \begin{minipage}[b]{\linewidth}\raggedright
Command Prompt
\end{minipage} & \begin{minipage}[b]{\linewidth}\raggedright
PowerShell
\end{minipage} & \begin{minipage}[b]{\linewidth}\raggedright
Git Bash
\end{minipage} \\
\midrule\noalign{}
\endhead
\bottomrule\noalign{}
\endlastfoot
Show current location & \texttt{cd} & \texttt{pwd} & \texttt{pwd} \\
List files & \texttt{dir\ /B} & \texttt{ls\ -n} & \texttt{ls} \\
Go to folder & \texttt{cd\ Documents} & \texttt{cd\ Documents} &
\texttt{cd\ Documents} \\
Go up one level & \texttt{cd\ ..} & \texttt{cd\ ..} & \texttt{cd\ ..} \\
Go home & \texttt{cd\ \%USERPROFILE\%} & \texttt{cd\ \textasciitilde{}}
& \texttt{cd\ \textasciitilde{}} \\
Create folder & \texttt{mkdir\ Projects} & \texttt{mkdir\ Projects} &
\texttt{mkdir\ Projects} \\
Create file & \texttt{echo\ text\ \textgreater{}\ file.txt} &
\texttt{echo\ "text"\ \textgreater{}\ file.txt} &
\texttt{echo\ "text"\ \textgreater{}\ file.txt} \\
Copy file & \texttt{copy\ old.txt\ new.txt} &
\texttt{Copy-Item\ old.txt\ new.txt} & \texttt{cp\ old.txt\ new.txt} \\
Move file & \texttt{move\ old.txt\ folder/} &
\texttt{Move-Item\ old.txt\ folder/} & \texttt{mv\ old.txt\ folder/} \\
Delete file & \texttt{del\ file.txt} & \texttt{Remove-Item\ file.txt} &
\texttt{rm\ file.txt} \\
List with filter & \texttt{dir\ /B\ *.txt} & \texttt{ls\ *.txt} &
\texttt{ls\ *.txt} \\
Save output to file & \texttt{dir\ \textgreater{}\ list.txt} &
\texttt{ls\ \textgreater{}\ list.txt} &
\texttt{ls\ \textgreater{}\ list.txt} \\
Page through output & \texttt{dir\ \textbar{}\ more} &
\texttt{ls\ \textbar{}\ more} & \texttt{ls\ \textbar{}\ less} \\
Search in files & \texttt{findstr\ "text"\ file.txt} &
\texttt{Select-String\ "text"\ file.txt} &
\texttt{grep\ "text"\ file.txt} \\
Show file contents & \texttt{type\ file.txt} &
\texttt{cat\ file.txt} or \texttt{Get-Content} &
\texttt{cat\ file.txt} \\
Create script & \texttt{.bat} files & \texttt{.ps1} files &
\texttt{.sh} files \\
Run script & \texttt{script.bat} & \texttt{.\textbackslash{}script.ps1}
& \texttt{./script.sh} \\
\end{longtable}
}

\section*{Feature Comparison Table}\label{docs__pandoc__latex__src__command_line_interface_selection__command_line_interface_selection.md__feature-comparison-table}

{\def\LTcaptype{none} % do not increment counter
\begin{longtable}[]{@{}
  >{\raggedright\arraybackslash}p{(\linewidth - 6\tabcolsep) * \real{0.3824}}
  >{\raggedright\arraybackslash}p{(\linewidth - 6\tabcolsep) * \real{0.2059}}
  >{\raggedright\arraybackslash}p{(\linewidth - 6\tabcolsep) * \real{0.2059}}
  >{\raggedright\arraybackslash}p{(\linewidth - 6\tabcolsep) * \real{0.2059}}@{}}
\toprule\noalign{}
\begin{minipage}[b]{\linewidth}\raggedright
Feature
\end{minipage} & \begin{minipage}[b]{\linewidth}\raggedright
CMD
\end{minipage} & \begin{minipage}[b]{\linewidth}\raggedright
PowerShell
\end{minipage} & \begin{minipage}[b]{\linewidth}\raggedright
Git Bash
\end{minipage} \\
\midrule\noalign{}
\endhead
\bottomrule\noalign{}
\endlastfoot
Simplicity & Easiest & Moderate & Hardest \\
Beginner-Friendly & Best & Good & Challenging \\
Power/Capability & Basic & Excellent & Excellent \\
Screen Reader Compatible & Perfect & Perfect & Perfect \\
Linux/macOS Skills & Windows-only & Some overlap & Full overlap \\
3D Printing Automation & Adequate & Excellent & Adequate \\
Learning Curve & Gentle & Moderate & Steep \\
Community Support & Moderate & Excellent & Excellent \\
Windows Integration & Perfect & Perfect & Good \\
Installation Difficulty & Built-in & Built-in & Requires Git \\
\end{longtable}
}

\section*{Quick Learner Profile Test}\label{docs__pandoc__latex__src__command_line_interface_selection__command_line_interface_selection.md__quick-learner-profile-test}

Answer these questions to find your best match:

\subsection*{Question 1: Experience Level}\label{docs__pandoc__latex__src__command_line_interface_selection__command_line_interface_selection.md__question-1-experience-level}

\begin{itemize}
\tightlist
\item
  A: I\textquotesingle ve never used a command line Easier paths better (CMD or PowerShell)
\item
  B: I\textquotesingle ve used terminals before Any path works
\item
  C: I use macOS or Linux Git Bash most natural
\end{itemize}

\subsection*{Question 2: What matters most?}\label{docs__pandoc__latex__src__command_line_interface_selection__command_line_interface_selection.md__question-2-what-matters-most}

\begin{itemize}
\tightlist
\item
  A: Simplicity and quick learning Choose CMD
\item
  B: Power and advanced features Choose PowerShell
\item
  C: Consistency across Windows/Mac/Linux Choose Git Bash
\end{itemize}

\subsection*{Question 3: Future goals}\label{docs__pandoc__latex__src__command_line_interface_selection__command_line_interface_selection.md__question-3-future-goals}

\begin{itemize}
\tightlist
\item
  A: Just need to manage files for 3D printing CMD is fine
\item
  B: Advanced automation and scripting PowerShell recommended
\item
  C: Professional development workflows Git Bash best
\end{itemize}

\subsection*{Question 4: Your main concern}\label{docs__pandoc__latex__src__command_line_interface_selection__command_line_interface_selection.md__question-4-your-main-concern}

\begin{itemize}
\tightlist
\item
  A: Don\textquotesingle t want steep learning curve CMD
\item
  B: Want industry-standard skills Git Bash
\item
  C: Want Microsoft\textquotesingle s modern tool PowerShell
\end{itemize}

\section*{Recommendation by Goal}\label{docs__pandoc__latex__src__command_line_interface_selection__command_line_interface_selection.md__recommendation-by-goal}

\subsection*{Goal: "I want to learn the basics and get to 3D printing quickly"}\label{docs__pandoc__latex__src__command_line_interface_selection__command_line_interface_selection.md__goal-i-want-to-learn-the-basics-and-get-to-3d-printing-quickly}

\subsubsection*{Start with CMD (Command Prompt)}\label{docs__pandoc__latex__src__command_line_interface_selection__command_line_interface_selection.md__start-with-cmd-command-prompt}

\begin{itemize}
\tightlist
\item
  Simplest syntax
\item
  Fastest to get productive
\item
  All core concepts transfer to others
\item
  Can switch later if needed
\end{itemize}

\hyperref[docs__pandoc__latex__src__cmd_foundation__cmd_curriculum_overview__cmd_curriculum_overview.md__cmd_foundation_cmd_curriculum_overview-cmd_curriculum_overview]{Start CMD Foundation}

\subsection*{Goal: "I want power and professional automation"}\label{docs__pandoc__latex__src__command_line_interface_selection__command_line_interface_selection.md__goal-i-want-power-and-professional-automation}

\subsubsection*{Start with PowerShell}\label{docs__pandoc__latex__src__command_line_interface_selection__command_line_interface_selection.md__start-with-powershell}

\begin{itemize}
\tightlist
\item
  Microsoft\textquotesingle s modern, recommended tool
\item
  Professional-grade capabilities
\item
  Better for complex 3D printing workflows
\item
  Skills are in-demand
\end{itemize}

\hyperref[docs__pandoc__latex__src__powershell_foundation__powershell_curriculum_overview__powershell_curriculum_overview.md__powershell_foundation_powershell_curriculum_overview-powershell_curriculum_overview]{Start PowerShell Foundation}

\subsection*{Goal: "I want skills that work on Windows, Mac, and Linux"}\label{docs__pandoc__latex__src__command_line_interface_selection__command_line_interface_selection.md__goal-i-want-skills-that-work-on-windows-mac-and-linux}

\subsubsection*{Start with Git Bash}\label{docs__pandoc__latex__src__command_line_interface_selection__command_line_interface_selection.md__start-with-git-bash}

\begin{itemize}
\tightlist
\item
  Unix/bash skills transfer everywhere
\item
  Great preparation for professional development
\item
  Consistent across all platforms
\item
  Growing standard in 3D printing tools
\end{itemize}

\hyperref[docs__pandoc__latex__src__gitbash_foundation__gitbash_curriculum_overview__gitbash_curriculum_overview.md__gitbash_foundation_gitbash_curriculum_overview-gitbash_curriculum_overview]{Start Git Bash Foundation}

\section*{Can I Switch Paths Later?}\label{docs__pandoc__latex__src__command_line_interface_selection__command_line_interface_selection.md__can-i-switch-paths-later}

Yes, absolutely! All three teach the same fundamental concepts:

\begin{itemize}
\tightlist
\item
  File navigation and organization
\item
  Creating and managing files/folders
\item
  Combining commands for powerful workflows
\item
  Scripting and automation basics
\end{itemize}

Once you learn one, switching to another is quick. The concepts are identical; only the syntax changes.

Example: If you learn CMD first, then later want PowerShell\textquotesingle s power, you\textquotesingle ll find it easy. The command \texttt{cd\ Documents} works the same way in all three.

\section*{Important: All Are Equally Accessible}\label{docs__pandoc__latex__src__command_line_interface_selection__command_line_interface_selection.md__important-all-are-equally-accessible}

\subsection*{Screen readers work perfectly with all three}\label{docs__pandoc__latex__src__command_line_interface_selection__command_line_interface_selection.md__screen-readers-work-perfectly-with-all-three}

\begin{itemize}
\tightlist
\item
  Text-based by nature (perfect for NVDA, JAWS)
\item
  No mouse required
\item
  Output is naturally readable
\item
  Keyboard-only workflows
\end{itemize}

Don\textquotesingle t let accessibility concerns influence your choice. All are fully accessible.

\section*{Getting Started: Your Decision}\label{docs__pandoc__latex__src__command_line_interface_selection__command_line_interface_selection.md__getting-started-your-decision}

Take a moment and choose:

\subsection*{1. I want the simplest path}\label{docs__pandoc__latex__src__command_line_interface_selection__command_line_interface_selection.md__1-i-want-the-simplest-path}

\hyperref[docs__pandoc__latex__src__cmd_foundation__part_1.md__cmd_foundation-part_1]{Command Prompt Foundation}

\begin{itemize}
\tightlist
\item
  Time to first success: \textasciitilde{}30 minutes
\item
  Learning curve: Gentlest
\item
  When to upgrade: Once you\textquotesingle re comfortable and want power
\end{itemize}

\subsection*{2. I want modern, powerful Windows tools}\label{docs__pandoc__latex__src__command_line_interface_selection__command_line_interface_selection.md__2-i-want-modern-powerful-windows-tools}

\hyperref[docs__pandoc__latex__src__powershell_foundation__part_1.md__powershell_foundation-part_1]{PowerShell Foundation}

\begin{itemize}
\tightlist
\item
  Time to first success: \textasciitilde{}45 minutes
\item
  Learning curve: Moderate
\item
  Best for: Professional automation, 3D printing workflows
\end{itemize}

\subsection*{3. I want Unix/Linux skills that work everywhere}\label{docs__pandoc__latex__src__command_line_interface_selection__command_line_interface_selection.md__3-i-want-unixlinux-skills-that-work-everywhere}

\hyperref[docs__pandoc__latex__src__gitbash_foundation__part_1.md__gitbash_foundation-part_1]{Git Bash Foundation}

\begin{itemize}
\tightlist
\item
  Time to first success: \textasciitilde{}1 hour
\item
  Learning curve: Steeper but rewarding
\item
  Best for: Professional development, cross-platform work
\end{itemize}

\section*{Not Sure? Here\textquotesingle s What Most People Do}\label{docs__pandoc__latex__src__command_line_interface_selection__command_line_interface_selection.md__not-sure-heres-what-most-people-do}

If you\textquotesingle re reading this and unsure:

\begin{quote}
Start with Command Prompt (CMD). It\textquotesingle s the gentlest introduction. You\textquotesingle ll be productive quickly and can always switch to PowerShell or Git Bash later. The skills transfer completely.
\end{quote}

After completing CMD:

\begin{itemize}
\tightlist
\item
  Want more power? PowerShell is next
\item
  Want Linux skills? Git Bash is next
\item
  Want to stick with CMD? You have all the skills you need
\end{itemize}

\section*{FAQ}\label{docs__pandoc__latex__src__command_line_interface_selection__command_line_interface_selection.md__faq}

Q: Do I need to pick now and stick with it forever?
A: No. Start with one, try another, switch between them. They\textquotesingle re tools. Use what works.

Q: Will my 3D printing skills work in all three?
A: Yes. Once you understand the concepts (file organization, automation, piping), they apply everywhere.

Q: If I pick CMD, can I learn PowerShell later?
A: Absolutely. Many learners do exactly this. CMD gets you productive; PowerShell adds power.

Q: Is Git Bash harder?
A: Slightly, due to path syntax and Unix conventions. But not dramatically. If you take time with it, you\textquotesingle ll learn it.

Q: Which do professional 3D printing developers use?
A: Mix of all three, but Git Bash/Linux is most common in cross-platform teams.

\section*{Ready to Begin?}\label{docs__pandoc__latex__src__command_line_interface_selection__command_line_interface_selection.md__ready-to-begin}

Choose your path above and click to start. Remember:

\begin{itemize}
\tightlist
\item
  Each lesson includes practice exercises
\item
  You can\textquotesingle t break anything
\item
  Mistakes are learning opportunities
\item
  Ask for help if stuck
\end{itemize}

Let\textquotesingle s get you comfortable with the command line!

Other Screen Readers

Dolphin SuperNova (commercial) and Windows Narrator (built-in) are also supported; the workflows and recommendations in this document apply to them. See \url{https://yourdolphin.com/supernova/} and \url{https://support.microsoft.com/narrator} for vendor documentation.

\section{PowerShell}\label{docs__pandoc__latex__src__powershell_foundation__part_1.md__powershell_foundation-part_1}

This section covers terminal fundamentals, screen reader accessibility, and command-line basics needed before diving into 3D design with OpenSCAD.

Time commitment: \textasciitilde{}10 hours

Skills gained: Terminal navigation, file operations, basic scripting, keyboard-only workflow mastery

\subsection{PowerShell for Screen Reader Users - Complete Curriculum Overview}\label{docs__pandoc__latex__src__powershell_foundation__powershell_curriculum_overview__powershell_curriculum_overview.md__powershell_foundation_powershell_curriculum_overview-powershell_curriculum_overview}

Welcome! This curriculum teaches you how to use PowerShell (Windows Terminal) as a screen reader user, starting from zero experience and building to professional-level skills.

Last Updated: February 2026\\
Total Duration: 30-45 hours of instruction + practice (for screen reader users)\\
Target Users: Anyone with a screen reader (NVDA, JAWS, or other)

\emph{Note: Time estimates reflect the additional time needed for screen reader navigation, text-to-speech processing, and careful keyboard-based workflows.}

\subsubsection*{Why Learn PowerShell?}\label{docs__pandoc__latex__src__powershell_foundation__powershell_curriculum_overview__powershell_curriculum_overview.md__why-learn-powershell}

\paragraph*{For Everyone}\label{docs__pandoc__latex__src__powershell_foundation__powershell_curriculum_overview__powershell_curriculum_overview.md__for-everyone}

\begin{itemize}
\tightlist
\item
  Speed: Text commands are often faster than clicking through menus
\item
  Automation: Repeat tasks automatically instead of doing them manually
\item
  Precision: Exact control over what your computer does
\item
  Scripting: Create programs that solve real problems
\end{itemize}

\paragraph*{For 3D Printing (Our Focus)}\label{docs__pandoc__latex__src__powershell_foundation__powershell_curriculum_overview__powershell_curriculum_overview.md__for-3d-printing-our-focus}

\begin{itemize}
\tightlist
\item
  Batch Operations: Process 100s of 3D models at once
\item
  Accessibility: Many 3D design tools are scriptable
\item
  Reproducibility: Same settings, every time
\item
  Integration: Connect OpenSCAD, slicers, and tools together
\end{itemize}

\paragraph*{For Screen Reader Users Specifically}\label{docs__pandoc__latex__src__powershell_foundation__powershell_curriculum_overview__powershell_curriculum_overview.md__for-screen-reader-users-specifically}

\begin{itemize}
\tightlist
\item
  Great Accessibility: PowerShell works perfectly with NVDA, JAWS, and others
\item
  No Mouse Needed: Everything is keyboard-based
\item
  Text-Based: Output is naturally readable by screen readers
\item
  Stability: Unlike GUIs, terminal interactions are consistent
\end{itemize}

\subsubsection*{Curriculum Structure}\label{docs__pandoc__latex__src__powershell_foundation__powershell_curriculum_overview__powershell_curriculum_overview.md__curriculum-structure}

\paragraph*{Phase 1: Absolute Beginner -\textgreater{} Comfortable User}\label{docs__pandoc__latex__src__powershell_foundation__powershell_curriculum_overview__powershell_curriculum_overview.md__phase-1-absolute-beginner---comfortable-user}

{\def\LTcaptype{none} % do not increment counter
\begin{longtable}[]{@{}
  >{\raggedright\arraybackslash}p{(\linewidth - 4\tabcolsep) * \real{0.3365}}
  >{\raggedright\arraybackslash}p{(\linewidth - 4\tabcolsep) * \real{0.1250}}
  >{\raggedright\arraybackslash}p{(\linewidth - 4\tabcolsep) * \real{0.5385}}@{}}
\toprule\noalign{}
\begin{minipage}[b]{\linewidth}\raggedright
Lesson
\end{minipage} & \begin{minipage}[b]{\linewidth}\raggedright
Duration
\end{minipage} & \begin{minipage}[b]{\linewidth}\raggedright
What You\textquotesingle ll Learn
\end{minipage} \\
\midrule\noalign{}
\endhead
\bottomrule\noalign{}
\endlastfoot
Screen Reader Accessibility Guide & 1.5 hours &
Screen reader tips specific to PowerShell (READ FIRST) \\
PS-Pre: Your First Terminal & 2-2.5 hours &
Opening PowerShell, first commands, basic navigation \\
PS-0: Getting Started & 1.5 hours & Paths, shortcuts, tab completion \\
PS-1: Navigation & 2-2.5 hours &
Moving around the file system confidently \\
\end{longtable}
}

Goal: You can navigate to any folder and see what\textquotesingle s in it with your screen reader.

\paragraph*{Phase 2: Intermediate User -\textgreater{} Power User}\label{docs__pandoc__latex__src__powershell_foundation__powershell_curriculum_overview__powershell_curriculum_overview.md__phase-2-intermediate-user---power-user}

{\def\LTcaptype{none} % do not increment counter
\begin{longtable}[]{@{}
  >{\raggedright\arraybackslash}p{(\linewidth - 4\tabcolsep) * \real{0.4149}}
  >{\raggedright\arraybackslash}p{(\linewidth - 4\tabcolsep) * \real{0.1383}}
  >{\raggedright\arraybackslash}p{(\linewidth - 4\tabcolsep) * \real{0.4468}}@{}}
\toprule\noalign{}
\begin{minipage}[b]{\linewidth}\raggedright
Lesson
\end{minipage} & \begin{minipage}[b]{\linewidth}\raggedright
Duration
\end{minipage} & \begin{minipage}[b]{\linewidth}\raggedright
What You\textquotesingle ll Learn
\end{minipage} \\
\midrule\noalign{}
\endhead
\bottomrule\noalign{}
\endlastfoot
PS-2: File \& Folder Manipulation & 2.5-3 hours &
Create, copy, move, delete files/folders \\
PS-3: Input, Output \& Piping & 2.5-3 hours &
Chain commands together, redirect output \\
PS-4: Environment Variables \& Aliases & 2-2.5 hours &
Automate settings, create shortcuts \\
PS-5: Filling in the Gaps & 2-2.5 hours &
Profiles, history, debugging \\
\end{longtable}
}

Goal: You can create folders, manage files, and combine commands to accomplish complex tasks.

\paragraph*{Phase 3: Professional Skills (Beyond Curriculum)}\label{docs__pandoc__latex__src__powershell_foundation__powershell_curriculum_overview__powershell_curriculum_overview.md__phase-3-professional-skills-beyond-curriculum}

These topics extend beyond this curriculum but are natural next steps:

{\def\LTcaptype{none} % do not increment counter
\begin{longtable}[]{@{}ll@{}}
\toprule\noalign{}
Topic & When to Learn \\
\midrule\noalign{}
\endhead
\bottomrule\noalign{}
\endlastfoot
Scripting (.ps1 files) & After PS-5 \\
Functions \& Loops & After PS-5 \\
Error Handling & After PS-5 \\
Remote Administration & Advanced \\
3D Printing Integration & After all above \\
\end{longtable}
}

\subsubsection*{How to Use This Curriculum}\label{docs__pandoc__latex__src__powershell_foundation__powershell_curriculum_overview__powershell_curriculum_overview.md__how-to-use-this-curriculum}

\paragraph*{If You\textquotesingle ve Never Used a Terminal Before}\label{docs__pandoc__latex__src__powershell_foundation__powershell_curriculum_overview__powershell_curriculum_overview.md__if-youve-never-used-a-terminal-before}

Start here and go in order:

\begin{enumerate}
\tightlist
\item
  Read Screen Reader Accessibility Guide completely
\item
  Do PS-Pre: Your First Terminal exercises
\item
  Continue with PS-0, PS-1, etc.
\end{enumerate}

Don\textquotesingle t skip steps - each builds on the previous one.

\paragraph*{If You\textquotesingle ve Used a Terminal Before (But Not with a Screen Reader)}\label{docs__pandoc__latex__src__powershell_foundation__powershell_curriculum_overview__powershell_curriculum_overview.md__if-youve-used-a-terminal-before-but-not-with-a-screen-reader}

Start here:

\begin{enumerate}
\tightlist
\item
  Skim Screen Reader Accessibility Guide (you\textquotesingle ll recognize most tips)
\item
  Quickly review PS-Pre (basics with screen reader focus)
\item
  Move to PS-0 for deeper learning
\end{enumerate}

\paragraph*{If You\textquotesingle re Experienced with Terminal + Screen Reader}\label{docs__pandoc__latex__src__powershell_foundation__powershell_curriculum_overview__powershell_curriculum_overview.md__if-youre-experienced-with-terminal--screen-reader}

You can:

\begin{enumerate}
\tightlist
\item
  Jump to specific lessons you need (PS-2, PS-3, etc.)
\item
  Use the Quick Reference sections
\item
  Skip the practice exercises, do the quizzes to verify knowledge
\end{enumerate}

\subsubsection*{How Each Lesson is Structured}\label{docs__pandoc__latex__src__powershell_foundation__powershell_curriculum_overview__powershell_curriculum_overview.md__how-each-lesson-is-structured}

\paragraph*{Every Lesson Contains:}\label{docs__pandoc__latex__src__powershell_foundation__powershell_curriculum_overview__powershell_curriculum_overview.md__every-lesson-contains}

\begin{enumerate}
\tightlist
\item
  Learning Objectives - What you\textquotesingle ll be able to do
\item
  Key Commands - The important ones to memorize
\item
  Step-by-Step Examples - How to actually do it
\item
  Practice Exercises - Hands-on work
\item
  Quiz Questions - Check your understanding
\item
  Extension Problems - Go deeper if interested
\end{enumerate}

\paragraph*{How to Get Through Each Lesson:}\label{docs__pandoc__latex__src__powershell_foundation__powershell_curriculum_overview__powershell_curriculum_overview.md__how-to-get-through-each-lesson}

\begin{enumerate}
\tightlist
\item
  Read the learning objectives
\item
  Do the step-by-step examples alongside
\item
  Complete the practice exercises (critical!)
\item
  Take the quiz (don\textquotesingle t cheat)
\item
  Try extension problems if you have time
\item
  Move to next lesson when quiz is solid
\end{enumerate}

Estimated time: 2-3 hours per lesson for screen reader users (depends on practice time)

\subsubsection*{Screen Reader Tips Throughout the Curriculum}\label{docs__pandoc__latex__src__powershell_foundation__powershell_curriculum_overview__powershell_curriculum_overview.md__screen-reader-tips-throughout-the-curriculum}

\paragraph*{Every Lesson Includes:}\label{docs__pandoc__latex__src__powershell_foundation__powershell_curriculum_overview__powershell_curriculum_overview.md__every-lesson-includes}

\begin{itemize}
\tightlist
\item
  {[}SR{]} symbols marking screen reader-specific sections
\item
  Tips for NVDA and JAWS users separately
\item
  Solutions for common accessibility issues
\item
  Workarounds for long outputs
\end{itemize}

\paragraph*{Screen Reader Accessibility Guide}\label{docs__pandoc__latex__src__powershell_foundation__powershell_curriculum_overview__powershell_curriculum_overview.md__screen-reader-accessibility-guide}

This is your companion resource used throughout:

\begin{itemize}
\tightlist
\item
  Detailed NVDA keyboard shortcuts
\item
  Detailed JAWS keyboard shortcuts
\item
  Solutions to common problems
\item
  Pro tips for efficiency
\end{itemize}

Keep it open or printed as you work through lessons.

\subsubsection*{Quick Start Guide (First 45-60 Minutes)}\label{docs__pandoc__latex__src__powershell_foundation__powershell_curriculum_overview__powershell_curriculum_overview.md__quick-start-guide-first-45-60-minutes}

\paragraph*{If You Have 45-60 Minutes Right Now:}\label{docs__pandoc__latex__src__powershell_foundation__powershell_curriculum_overview__powershell_curriculum_overview.md__if-you-have-45-60-minutes-right-now}

\begin{enumerate}
\tightlist
\item
  Open PowerShell (Windows key -\textgreater{} type PowerShell -\textgreater{} Enter)
\item
  Run these commands:

  \begin{lstlisting}[style=Alabaster, language=powershell]
  pwd
  ls -n
  cd Documents
  pwd

  \end{lstlisting}
\item
  See how your screen reader reads each output
\item
  Try Tab completion:

  \begin{itemize}
  \tightlist
  \item
    Type \texttt{cd\ D} and press Tab
  \item
    Hear PowerShell auto-complete to Documents (or similar)
  \end{itemize}
\item
  Create a file:

  \begin{lstlisting}[style=Alabaster, language=powershell]
  echo "I am learning PowerShell" > learning.txt
  cat learning.txt

  \end{lstlisting}
\end{enumerate}

That\textquotesingle s it! You\textquotesingle ve done the key concepts. Now read PS-Pre for the details.

\subsubsection*{Common Questions Before Starting}\label{docs__pandoc__latex__src__powershell_foundation__powershell_curriculum_overview__powershell_curriculum_overview.md__common-questions-before-starting}

\paragraph*{Q: Do I have to use PowerShell? What about Command Prompt (cmd.exe)?}\label{docs__pandoc__latex__src__powershell_foundation__powershell_curriculum_overview__powershell_curriculum_overview.md__q-do-i-have-to-use-powershell-what-about-command-prompt-cmdexe}

A: Command Prompt works, but PowerShell is better. PowerShell is:

\begin{itemize}
\tightlist
\item
  More powerful
\item
  Better for modern tools
\item
  Screen-reader-friendly
\item
  The future of Windows automation
\end{itemize}

Use PowerShell for this curriculum.

\paragraph*{Q: What if I use a different screen reader (not NVDA or JAWS)?}\label{docs__pandoc__latex__src__powershell_foundation__powershell_curriculum_overview__powershell_curriculum_overview.md__q-what-if-i-use-a-different-screen-reader-not-nvda-or-jaws}

A: The fundamentals work the same. Check your screen reader\textquotesingle s documentation for the equivalent of these commands:

\begin{itemize}
\tightlist
\item
  Read current line
\item
  Read to end of screen
\item
  Read next/previous page
\end{itemize}

Most screen readers have these features.

\paragraph*{Q: I\textquotesingle m intimidated. Is this really for me?}\label{docs__pandoc__latex__src__powershell_foundation__powershell_curriculum_overview__powershell_curriculum_overview.md__q-im-intimidated-is-this-really-for-me}

A: YES. This curriculum is specifically designed for people with no terminal experience AND with screen readers. You\textquotesingle ll start with absolute basics. There\textquotesingle s nothing to be afraid of - we\textquotesingle ve written this specifically to make it accessible.

\paragraph*{Q: How long will this take?}\label{docs__pandoc__latex__src__powershell_foundation__powershell_curriculum_overview__powershell_curriculum_overview.md__q-how-long-will-this-take}

A: Realistically:

\begin{itemize}
\tightlist
\item
  Minimum (just lessons, no exercises): 15-18 hours
\item
  Normal (lessons + exercises): 30-45 hours
\item
  With extension problems: 45-60+ hours
\end{itemize}

Spread it over weeks or months. Go at your pace.

\paragraph*{Q: What if I forget things?}\label{docs__pandoc__latex__src__powershell_foundation__powershell_curriculum_overview__powershell_curriculum_overview.md__q-what-if-i-forget-things}

A: That\textquotesingle s normal and expected. Solutions:

\begin{enumerate}
\tightlist
\item
  Come back to this page for the overview
\item
  Jump back to that lesson for a quick review
\item
  Use the quiz questions to self-test
\item
  Check the Screen Reader Accessibility Guide for troubleshooting
\end{enumerate}

\paragraph*{Q: Will this help me with 3D printing?}\label{docs__pandoc__latex__src__powershell_foundation__powershell_curriculum_overview__powershell_curriculum_overview.md__q-will-this-help-me-with-3d-printing}

A: Absolutely. Near the end of the 3dMake curriculum, you\textquotesingle ll use PowerShell to:

\begin{itemize}
\tightlist
\item
  Batch-process 3D models
\item
  Automate slicing tasks
\item
  Run scripts that generate designs
\item
  Integrate tools together
\end{itemize}

\subsubsection*{Suggested Study Schedule}\label{docs__pandoc__latex__src__powershell_foundation__powershell_curriculum_overview__powershell_curriculum_overview.md__suggested-study-schedule}

\paragraph*{Beginner Goal (Weeks 1-2)}\label{docs__pandoc__latex__src__powershell_foundation__powershell_curriculum_overview__powershell_curriculum_overview.md__beginner-goal-weeks-1-2}

Week 1:

\begin{itemize}
\tightlist
\item
  Day 1: Read Screen Reader Accessibility Guide
\item
  Day 2: PS-Pre lesson
\item
  Day 3: PS-0 lesson
\item
  Day 4: Practice PS-0 and PS-1 exercises
\item
  Day 5: PS-1 lesson
\end{itemize}

Week 2:

\begin{itemize}
\tightlist
\item
  Review PS-0 and PS-1 quizzes
\item
  Practice navigation exercises daily
\item
  Do extension problems for PS-0 and PS-1
\item
  Feel confident with file system navigation
\end{itemize}

Goal: Know how to navigate to any folder, list its contents, and understand paths.

\paragraph*{Intermediate Goal (Weeks 3-5)}\label{docs__pandoc__latex__src__powershell_foundation__powershell_curriculum_overview__powershell_curriculum_overview.md__intermediate-goal-weeks-3-5}

Week 3:

\begin{itemize}
\tightlist
\item
  PS-2 lesson (file manipulation)
\item
  Complete exercises
\item
  Take quiz
\end{itemize}

Week 4:

\begin{itemize}
\tightlist
\item
  PS-3 lesson (piping and output)
\item
  Complete exercises
\item
  Take quiz
\end{itemize}

Week 5:

\begin{itemize}
\tightlist
\item
  PS-4 and PS-5 lessons
\item
  Complete all quizzes
\item
  Practice combining commands
\end{itemize}

Goal: Create, modify, and move files. Combine commands for complex tasks.

\paragraph*{Advanced Goal (Weeks 6+)}\label{docs__pandoc__latex__src__powershell_foundation__powershell_curriculum_overview__powershell_curriculum_overview.md__advanced-goal-weeks-6}

\begin{itemize}
\tightlist
\item
  Review any lessons you need
\item
  Do all extension problems
\item
  Start learning PowerShell scripting
\item
  Begin 3D printing integration
\end{itemize}

\subsubsection*{Success Criteria}\label{docs__pandoc__latex__src__powershell_foundation__powershell_curriculum_overview__powershell_curriculum_overview.md__success-criteria}

\paragraph*{By the End of PS-1, You Should:}\label{docs__pandoc__latex__src__powershell_foundation__powershell_curriculum_overview__powershell_curriculum_overview.md__by-the-end-of-ps-1-you-should}

\begin{itemize}
\tightlist
\item[$\square$]
  Know where you are at all times (\texttt{pwd})
\item[$\square$]
  See what\textquotesingle s around you (\texttt{ls\ -n})
\item[$\square$]
  Navigate confidently with your screen reader
\item[$\square$]
  Use Tab completion comfortably
\item[$\square$]
  Understand absolute and relative paths
\item[$\square$]
  Pass the PS-0 and PS-1 quizzes
\end{itemize}

\paragraph*{By the End of PS-3, You Should:}\label{docs__pandoc__latex__src__powershell_foundation__powershell_curriculum_overview__powershell_curriculum_overview.md__by-the-end-of-ps-3-you-should}

\begin{itemize}
\tightlist
\item[$\square$]
  Create and delete files and folders
\item[$\square$]
  Copy and move files
\item[$\square$]
  Redirect output to files
\item[$\square$]
  Pipe commands together
\item[$\square$]
  Save long outputs to readable files
\item[$\square$]
  Pass all quizzes PS-0 through PS-3
\end{itemize}

\paragraph*{By the End of PS-5, You Should:}\label{docs__pandoc__latex__src__powershell_foundation__powershell_curriculum_overview__powershell_curriculum_overview.md__by-the-end-of-ps-5-you-should}

\begin{itemize}
\tightlist
\item[$\square$]
  Use command history effectively
\item[$\square$]
  Create aliases and functions
\item[$\square$]
  Understand your PowerShell profile
\item[$\square$]
  Handle screen reader edge cases
\item[$\square$]
  Feel comfortable experimenting
\item[$\square$]
  Pass all quizzes
\end{itemize}

\subsubsection*{Important Rules}\label{docs__pandoc__latex__src__powershell_foundation__powershell_curriculum_overview__powershell_curriculum_overview.md__important-rules}

\paragraph*{Rule 1: Always Know Where You Are}\label{docs__pandoc__latex__src__powershell_foundation__powershell_curriculum_overview__powershell_curriculum_overview.md__rule-1-always-know-where-you-are}

Every session, first thing:

\begin{lstlisting}[style=Alabaster, language=powershell]
pwd

\end{lstlisting}

If you don\textquotesingle t know your path, you\textquotesingle ll get lost. Don\textquotesingle t move until you know where you are.

\paragraph*{Rule 2: Check Before You Delete}\label{docs__pandoc__latex__src__powershell_foundation__powershell_curriculum_overview__powershell_curriculum_overview.md__rule-2-check-before-you-delete}

Before deleting anything:

\begin{lstlisting}[style=Alabaster, language=powershell]
ls -n

\end{lstlisting}

Make sure you\textquotesingle re deleting the right thing. Once it\textquotesingle s gone, it\textquotesingle s gone.

\paragraph*{\texorpdfstring{Rule 3: Use \texttt{-n} with ls}{Rule 3: Use -n with ls}}\label{docs__pandoc__latex__src__powershell_foundation__powershell_curriculum_overview__powershell_curriculum_overview.md__rule-3-use--n-with-ls}

Always:

\begin{lstlisting}[style=Alabaster, language=powershell]
ls -n

\end{lstlisting}

Never:

\begin{lstlisting}[style=Alabaster, language=powershell]
ls

\end{lstlisting}

The \texttt{-n} (names only) is screen reader friendly. The default view is hard to read.

\paragraph*{Rule 4: When Lost, Redirect to a File}\label{docs__pandoc__latex__src__powershell_foundation__powershell_curriculum_overview__powershell_curriculum_overview.md__rule-4-when-lost-redirect-to-a-file}

If output is confusing:

\begin{lstlisting}[style=Alabaster, language=powershell]
command-name > output.txt
notepad.exe output.txt

\end{lstlisting}

This is always clearer for screen readers than terminal output.

\paragraph*{Rule 5: Save Everything You Create}\label{docs__pandoc__latex__src__powershell_foundation__powershell_curriculum_overview__powershell_curriculum_overview.md__rule-5-save-everything-you-create}

Every exercise, save your work:

\begin{lstlisting}[style=Alabaster, language=powershell]
mkdir my-practice-folder
cd my-practice-folder

\end{lstlisting}

Create a "learning" folder and keep everything there.

\subsubsection*{Troubleshooting: "Nothing Works!"}\label{docs__pandoc__latex__src__powershell_foundation__powershell_curriculum_overview__powershell_curriculum_overview.md__troubleshooting-nothing-works}

If you\textquotesingle re stuck:

\begin{enumerate}
\item
  Can\textquotesingle t hear PowerShell at all?

  \begin{itemize}
  \tightlist
  \item
    Make sure screen reader is running BEFORE PowerShell
  \item
    Try Alt+Tab to cycle to PowerShell window
  \item
    Restart both screen reader and PowerShell
  \end{itemize}
\item
  Commands not working?

  \begin{itemize}
  \tightlist
  \item
    Check spelling carefully
  \item
    Make sure you pressed Enter
  \item
    Try \texttt{Get-Help\ command-name}
  \end{itemize}
\item
  Can\textquotesingle t read the output?

  \begin{itemize}
  \tightlist
  \item
    Redirect to file: \texttt{command\ \textgreater{}\ output.txt}
  \item
    Open in Notepad: \texttt{notepad.exe\ output.txt}
  \item
    This always works
  \end{itemize}
\item
  Something ran forever?

  \begin{itemize}
  \tightlist
  \item
    Press Ctrl+C to stop it
  \end{itemize}
\item
  Completely confused?

  \begin{itemize}
  \tightlist
  \item
    Go back to PS-Pre and start over
  \item
    Work through every single exercise slowly
  \item
    Ask for help (ask an instructor or peer)
  \end{itemize}
\end{enumerate}

\subsubsection*{Resources}\label{docs__pandoc__latex__src__powershell_foundation__powershell_curriculum_overview__powershell_curriculum_overview.md__resources}

\paragraph*{Official Documentation}\label{docs__pandoc__latex__src__powershell_foundation__powershell_curriculum_overview__powershell_curriculum_overview.md__official-documentation}

\begin{itemize}
\tightlist
\item
  PowerShell Docs: \url{https://docs.microsoft.com/powershell/}
\item
  Windows Terminal Docs: \url{https://docs.microsoft.com/windows/terminal/}
\end{itemize}

\paragraph*{Screen Reader Guides}\label{docs__pandoc__latex__src__powershell_foundation__powershell_curriculum_overview__powershell_curriculum_overview.md__screen-reader-guides}

\begin{itemize}
\tightlist
\item
  NVDA: \url{https://www.nvaccess.org/documentation/}
\item
  JAWS: \url{https://www.freedomscientific.com/support/}
\end{itemize}

\paragraph*{Learning Resources}\label{docs__pandoc__latex__src__powershell_foundation__powershell_curriculum_overview__powershell_curriculum_overview.md__learning-resources}

\begin{itemize}
\tightlist
\item
  Microsoft Learn: \url{https://learn.microsoft.com/en-us/training/modules/}
\item
  PowerShell ISE: Built-in editor (open with \texttt{ise})
\end{itemize}

\paragraph*{3D Printing Integration}\label{docs__pandoc__latex__src__powershell_foundation__powershell_curriculum_overview__powershell_curriculum_overview.md__3d-printing-integration}

\begin{itemize}
\tightlist
\item
  OpenSCAD Scripting: See 3dMake lessons in this curriculum
\item
  Batch Processing: See PS-3 and PS-5 for real examples
\end{itemize}

\subsubsection*{Getting Help}\label{docs__pandoc__latex__src__powershell_foundation__powershell_curriculum_overview__powershell_curriculum_overview.md__getting-help}

\paragraph*{If You\textquotesingle re Stuck:}\label{docs__pandoc__latex__src__powershell_foundation__powershell_curriculum_overview__powershell_curriculum_overview.md__if-youre-stuck}

\begin{enumerate}
\tightlist
\item
  Read the relevant section of Screen Reader Accessibility Guide
\item
  Try a different approach from the "Solutions" sections
\item
  Go back one lesson and strengthen those concepts
\item
  Ask an instructor or peer with specific questions
\end{enumerate}

\paragraph*{Before You Ask for Help, Prepare:}\label{docs__pandoc__latex__src__powershell_foundation__powershell_curriculum_overview__powershell_curriculum_overview.md__before-you-ask-for-help-prepare}

\begin{enumerate}
\tightlist
\item
  What command are you running?
\item
  What do you expect to happen?
\item
  What actually happened?
\item
  What error message did you hear?
\end{enumerate}

Example: "I ran \texttt{cd\ Desktop} but my screen reader said \textquotesingle Cannot find path\textquotesingle. I expected to go to the Desktop folder."

This helps others help you quickly.

\subsubsection*{Next Steps}\label{docs__pandoc__latex__src__powershell_foundation__powershell_curriculum_overview__powershell_curriculum_overview.md__next-steps}

\begin{enumerate}
\tightlist
\item
  Right now: Read the Screen Reader Accessibility Guide completely
\item
  Next session: Start PS-Pre and do all exercises
\item
  Keep going: One lesson per day/week at your pace
\item
  Practice: Do exercises, not just read
\item
  Check yourself: Take the quizzes honestly
\item
  Celebrate: Each lesson completed is a real skill gained
\end{enumerate}

\subsubsection*{Final Thoughts}\label{docs__pandoc__latex__src__powershell_foundation__powershell_curriculum_overview__powershell_curriculum_overview.md__final-thoughts}

Learning to use PowerShell with a screen reader is absolutely achievable. Many people do it successfully. This curriculum was designed based on real experiences of screen reader users.

You\textquotesingle ve got this. Start with PS-Pre, take it slow, do the exercises, and ask questions when stuck.

Welcome to the PowerShell community!

\subsubsection*{Curriculum Map (For Reference)}\label{docs__pandoc__latex__src__powershell_foundation__powershell_curriculum_overview__powershell_curriculum_overview.md__curriculum-map-for-reference}

\begin{lstlisting}[style=Alabaster]
START HERE v
+---- Screen Reader Accessibility Guide (reference throughout)
+---- PS-Pre: Your First Terminal (absolute beginner entry point)
+---- PS-0: Getting Started (paths & navigation foundations)
+---- PS-1: Navigation (comfortable moving around)
+---- PS-2: File & Folder Manipulation (create/move/delete)
+---- PS-3: Input, Output & Piping (chain commands)
+---- PS-4: Environment Variables & Aliases (automation)
+---- PS-5: Filling in the Gaps (profiles & history)
+---- PS_Unit_Test (comprehensive practice & assessment)
        v
    NEXT: 3D Printing Integration Lessons
        v
    ADVANCED: PowerShell Scripting

\end{lstlisting}

Questions? Feedback? Stuck? Refer back to this page and the Screen Reader Accessibility Guide. You\textquotesingle ve got everything you need.

Now open PS-Pre and let\textquotesingle s get started!

Other Screen Readers

Dolphin SuperNova (commercial) and Windows Narrator (built-in) are also supported; the workflows and recommendations in this document apply to them. See \url{https://yourdolphin.com/supernova/} and \url{https://support.microsoft.com/narrator} for vendor documentation.

\subsection{PS-Pre: Your First Terminal - Screen Reader Navigation Fundamentals}\label{docs__pandoc__latex__src__powershell_foundation__ps_pre_your_first_terminal__ps_pre_your_first_terminal.md__powershell_foundation_ps_pre_your_first_terminal-ps_pre_your_first_terminal}

Duration: 1.5-2 hours (for screen reader users)\\
Prerequisites: None - this is the starting point\\
Accessibility Note: This lesson is designed specifically for screen reader users (NVDA, JAWS)

\subsubsection*{What is a Terminal?}\label{docs__pandoc__latex__src__powershell_foundation__ps_pre_your_first_terminal__ps_pre_your_first_terminal.md__what-is-a-terminal}

A terminal (also called a command line or shell) is a text-based interface where you type commands instead of clicking buttons. Think of it like sending written instructions to your computer instead of pointing and clicking.

Why learn this?

\begin{itemize}
\tightlist
\item
  Faster and more precise work (especially for 3D printing scripts and automation)
\item
  Essential for programming and using tools like OpenSCAD
\item
  Accessibility: Many command line tools work perfectly with screen readers
\item
  Scripting: Automate repetitive tasks
\end{itemize}

\subsubsection*{Opening PowerShell for the First Time}\label{docs__pandoc__latex__src__powershell_foundation__ps_pre_your_first_terminal__ps_pre_your_first_terminal.md__opening-powershell-for-the-first-time}

\paragraph*{On Windows}\label{docs__pandoc__latex__src__powershell_foundation__ps_pre_your_first_terminal__ps_pre_your_first_terminal.md__on-windows}

Method 1: Search (Easiest)

\begin{enumerate}
\tightlist
\item
  Press the Windows key alone
\item
  You should hear "Search"
\item
  Type: \texttt{PowerShell}
\item
  You\textquotesingle ll hear search results appear
\item
  Press Enter to open the first result (Windows PowerShell)
\item
  PowerShell will open in a new window
\end{enumerate}

Method 2: Using the Start Menu

\begin{enumerate}
\tightlist
\item
  Press Windows key + X (opens the Quick Link menu)
\item
  Look for "Windows PowerShell" or "Terminal"
\item
  Press Enter
\end{enumerate}

Method 3: From File Explorer

\begin{enumerate}
\tightlist
\item
  Open File Explorer
\item
  Navigate to any folder
\item
  In the menu bar, select "File" -\textgreater{} "Open Windows PowerShell here"
\item
  PowerShell opens in that folder location
\end{enumerate}

\paragraph*{First Connection: Understanding the Prompt}\label{docs__pandoc__latex__src__powershell_foundation__ps_pre_your_first_terminal__ps_pre_your_first_terminal.md__first-connection-understanding-the-prompt}

When PowerShell opens, your screen reader will announce the window title and then the prompt. The prompt is where you type commands.

What you\textquotesingle ll hear:

\begin{lstlisting}[style=Alabaster]
PS C:\Users\YourName>

\end{lstlisting}

What this means:

\begin{itemize}
\tightlist
\item
  \texttt{PS} = "PowerShell" indicator
\item
  \texttt{C:\textbackslash{}Users\textbackslash{}YourName} = Your current location (the "path")
\item
  \texttt{\textgreater{}} = The prompt is ready for your input
\end{itemize}

Important

Your cursor is blinking right after the \texttt{\textgreater{}}. This is where you type.

\subsubsection*{Your First Commands (Screen Reader Edition)}\label{docs__pandoc__latex__src__powershell_foundation__ps_pre_your_first_terminal__ps_pre_your_first_terminal.md__your-first-commands-screen-reader-edition}

\paragraph*{\texorpdfstring{Command 1: "Where Am I?" - \texttt{pwd}}{Command 1: "Where Am I?" - pwd}}\label{docs__pandoc__latex__src__powershell_foundation__ps_pre_your_first_terminal__ps_pre_your_first_terminal.md__command-1-where-am-i---pwd}

What it does: Tells you your current location

Type this:

\begin{lstlisting}[style=Alabaster, language=powershell]
pwd

\end{lstlisting}

Press Enter

What you\textquotesingle ll hear:
Your screen reader will announce the current path, something like:

\begin{lstlisting}[style=Alabaster]
C:\Users\YourName

\end{lstlisting}

Understanding paths:

\begin{itemize}
\tightlist
\item
  Paths show your location in the file system (like a mailing address)
\item
  Windows paths use backslashes: \texttt{C:\textbackslash{}Users\textbackslash{}YourName\textbackslash{}Documents}
\item
  Think of it like folders inside folders: \texttt{C:\textbackslash{}} (main drive) -\textgreater{} \texttt{Users} -\textgreater{} \texttt{YourName} -\textgreater{} \texttt{Documents}
\end{itemize}

\paragraph*{\texorpdfstring{Command 2: "What\textquotesingle s Here?" - \texttt{ls\ -n}}{Command 2: "What\textquotesingle s Here?" - ls -n}}\label{docs__pandoc__latex__src__powershell_foundation__ps_pre_your_first_terminal__ps_pre_your_first_terminal.md__command-2-whats-here---ls--n}

What it does: Lists all files and folders in your current location. The \texttt{-n} flag makes it screen-reader friendly (names only, one per line)

Type this:

\begin{lstlisting}[style=Alabaster, language=powershell]
ls -n

\end{lstlisting}

Press Enter

What you\textquotesingle ll hear:
Your screen reader will announce each file and folder name, one per line:

\begin{lstlisting}[style=Alabaster]
Desktop
Documents
Downloads
Music
Pictures
...

\end{lstlisting}

Why \texttt{-n}?

\begin{itemize}
\tightlist
\item
  Without \texttt{-n}, PowerShell shows files in columns (hard to read with a screen reader)
\item
  With \texttt{-n}, each file/folder is on its own line (perfect for screen readers)
\end{itemize}

\paragraph*{\texorpdfstring{Command 3: "Go There" - \texttt{cd\ Documents}}{Command 3: "Go There" - cd Documents}}\label{docs__pandoc__latex__src__powershell_foundation__ps_pre_your_first_terminal__ps_pre_your_first_terminal.md__command-3-go-there---cd-documents}

What it does: Changes your location (navigates to a folder)

Type this:

\begin{lstlisting}[style=Alabaster, language=powershell]
cd Documents

\end{lstlisting}

Press Enter

What you\textquotesingle ll hear:
The prompt changes to show your new location. You might hear something like:

\begin{lstlisting}[style=Alabaster]
PS C:\Users\YourName\Documents>

\end{lstlisting}

Practice navigation:

\begin{enumerate}
\tightlist
\item
  Run \texttt{pwd} to confirm you\textquotesingle re in Documents
\item
  Run \texttt{ls\ -n} to see what files are in Documents
\item
  Try going back: \texttt{cd\ ..} (the \texttt{..} means "go up one level")
\item
  Run \texttt{pwd} again to confirm
\item
  Go back to Documents: \texttt{cd\ Documents}
\end{enumerate}

\subsubsection*{Reading Screen Reader Output (Critical Skills)}\label{docs__pandoc__latex__src__powershell_foundation__ps_pre_your_first_terminal__ps_pre_your_first_terminal.md__reading-screen-reader-output-critical-skills}

\paragraph*{Dealing with Long Lists}\label{docs__pandoc__latex__src__powershell_foundation__ps_pre_your_first_terminal__ps_pre_your_first_terminal.md__dealing-with-long-lists}

When you run \texttt{ls\ -n} in a folder with many files, the list might be very long. Your screen reader might announce 50+ items rapidly.

Solution 1: Save to a File

\begin{lstlisting}[style=Alabaster, language=powershell]
ls -n > list.txt
notepad.exe list.txt

\end{lstlisting}

This saves the list to a file and opens it in Notepad where you can read it more slowly.

Solution 2: Search Within the Output

\begin{lstlisting}[style=Alabaster, language=powershell]
ls -n | findstr "search-term"

\end{lstlisting}

Example: If you\textquotesingle re looking for files containing "scad", type:

\begin{lstlisting}[style=Alabaster, language=powershell]
ls -n | findstr "scad"

\end{lstlisting}

\paragraph*{Navigating Tab Completion}\label{docs__pandoc__latex__src__powershell_foundation__ps_pre_your_first_terminal__ps_pre_your_first_terminal.md__navigating-tab-completion}

One of the most powerful screen reader tricks is Tab completion:

How it works:

\begin{enumerate}
\tightlist
\item
  Type the first few letters of a folder or file name
\item
  Press Tab
\item
  PowerShell automatically completes the rest
\end{enumerate}

Example:

\begin{enumerate}
\tightlist
\item
  You\textquotesingle re in \texttt{C:\textbackslash{}Users\textbackslash{}YourName\textgreater{}}
\item
  Type: \texttt{cd\ Doc}
\item
  Press Tab
\item
  PowerShell auto-completes it to: \texttt{cd\ Documents}
\item
  Press Enter to go there
\end{enumerate}

With a screen reader:

\begin{enumerate}
\tightlist
\item
  As you type \texttt{Doc}, your screen reader announces each letter
\item
  When you press Tab, PowerShell types the rest and your screen reader announces the full command
\item
  This is much faster than typing the whole thing
\end{enumerate}

\subsubsection*{Creating and Editing Files}\label{docs__pandoc__latex__src__powershell_foundation__ps_pre_your_first_terminal__ps_pre_your_first_terminal.md__creating-and-editing-files}

\paragraph*{Create a Simple File}\label{docs__pandoc__latex__src__powershell_foundation__ps_pre_your_first_terminal__ps_pre_your_first_terminal.md__create-a-simple-file}

Type this:

\begin{lstlisting}[style=Alabaster, language=powershell]
echo "Hello, PowerShell!" > hello.txt

\end{lstlisting}

What this does:

\begin{itemize}
\tightlist
\item
  \texttt{echo} sends text to the screen (or file)
\item
  \texttt{"Hello,\ PowerShell!"} is the text
\item
  \texttt{\textgreater{}} redirects it to a file called \texttt{hello.txt}
\end{itemize}

\paragraph*{Read the File Back}\label{docs__pandoc__latex__src__powershell_foundation__ps_pre_your_first_terminal__ps_pre_your_first_terminal.md__read-the-file-back}

Type this:

\begin{lstlisting}[style=Alabaster, language=powershell]
cat hello.txt

\end{lstlisting}

What you\textquotesingle ll hear: Your screen reader announces:

\begin{lstlisting}[style=Alabaster]
Hello, PowerShell!

\end{lstlisting}

\paragraph*{Open and Edit the File}\label{docs__pandoc__latex__src__powershell_foundation__ps_pre_your_first_terminal__ps_pre_your_first_terminal.md__open-and-edit-the-file}

Type this:

\begin{lstlisting}[style=Alabaster, language=powershell]
notepad.exe hello.txt

\end{lstlisting}

This opens the file in Notepad where you can edit it with your screen reader.

\subsubsection*{Essential Keyboard Shortcuts}\label{docs__pandoc__latex__src__powershell_foundation__ps_pre_your_first_terminal__ps_pre_your_first_terminal.md__essential-keyboard-shortcuts}

These work in PowerShell and are crucial for screen reader users:

{\def\LTcaptype{none} % do not increment counter
\begin{longtable}[]{@{}
  >{\raggedright\arraybackslash}p{(\linewidth - 2\tabcolsep) * \real{0.2152}}
  >{\raggedright\arraybackslash}p{(\linewidth - 2\tabcolsep) * \real{0.7848}}@{}}
\toprule\noalign{}
\begin{minipage}[b]{\linewidth}\raggedright
Key Combination
\end{minipage} & \begin{minipage}[b]{\linewidth}\raggedright
What It Does
\end{minipage} \\
\midrule\noalign{}
\endhead
\bottomrule\noalign{}
\endlastfoot
Up Arrow &
Shows your previous command (press again to go further back) \\
Down Arrow & Shows your next command (if you went back) \\
Tab & Auto-completes folder/file names \\
Ctrl+C & Stops a running command \\
Ctrl+L & Clears the screen \\
Enter & Runs the command \\
\end{longtable}
}

Screen reader tip: These all work perfectly with your screen reader. Try them!

\subsubsection*{Screen Reader-Specific Tips}\label{docs__pandoc__latex__src__powershell_foundation__ps_pre_your_first_terminal__ps_pre_your_first_terminal.md__screen-reader-specific-tips}

\paragraph*{NVDA Users}\label{docs__pandoc__latex__src__powershell_foundation__ps_pre_your_first_terminal__ps_pre_your_first_terminal.md__nvda-users}

\begin{enumerate}
\item
  Reading Command Output:

  \begin{itemize}
  \tightlist
  \item
    Use NVDA+Home to read the current line
  \item
    Use NVDA+Down Arrow to read to the end of the screen
  \item
    Use NVDA+Page Down to read the next page
  \end{itemize}
\item
  Reviewing Text:

  \begin{itemize}
  \tightlist
  \item
    Use NVDA+Shift+Page Up to review text above
  \end{itemize}
\end{enumerate}

\paragraph*{JAWS Users}\label{docs__pandoc__latex__src__powershell_foundation__ps_pre_your_first_terminal__ps_pre_your_first_terminal.md__jaws-users}

\begin{enumerate}
\item
  Reading Output:

  \begin{itemize}
  \tightlist
  \item
    Use Insert+Down Arrow to read line-by-line
  \item
    Use Insert+Page Down to read by page
  \item
    Use Insert+End to jump to the end of text
  \end{itemize}
\item
  Reading All Text:

  \begin{itemize}
  \tightlist
  \item
    Use Insert+Down Arrow repeatedly
  \item
    Or use Insert+Ctrl+Down to read to the end
  \end{itemize}
\end{enumerate}

\paragraph*{Common Issue: "I Can\textquotesingle t Hear the Output"}\label{docs__pandoc__latex__src__powershell_foundation__ps_pre_your_first_terminal__ps_pre_your_first_terminal.md__common-issue-i-cant-hear-the-output}

Problem: You run a command but don\textquotesingle t hear the output

Solutions:

\begin{enumerate}
\tightlist
\item
  Make sure your cursor is at the prompt (try pressing End or Ctrl+End)
\item
  Use Up Arrow to go back to your previous command and review it
\item
  Try redirecting to a file: \texttt{command\ \textgreater{}\ output.txt} then open the file
\item
  In NVDA: Try pressing NVDA+F7 to open the Review Mode viewer
\end{enumerate}

\subsubsection*{Practice Exercises}\label{docs__pandoc__latex__src__powershell_foundation__ps_pre_your_first_terminal__ps_pre_your_first_terminal.md__practice-exercises}

Complete these in order. Take your time with each one:

\paragraph*{Exercise 1: Basic Navigation}\label{docs__pandoc__latex__src__powershell_foundation__ps_pre_your_first_terminal__ps_pre_your_first_terminal.md__exercise-1-basic-navigation}

\begin{enumerate}
\tightlist
\item
  Open PowerShell
\item
  Run \texttt{pwd} and note your location
\item
  Run \texttt{ls\ -n} and listen to what\textquotesingle s there
\item
  Try \texttt{cd\ Documents} or another folder
\item
  Run \texttt{pwd} to confirm your new location
\item
  Run \texttt{ls\ -n} in this new location
\end{enumerate}

Goal: You should be comfortable knowing where you are and what\textquotesingle s around you

\paragraph*{Exercise 2: Using Tab Completion}\label{docs__pandoc__latex__src__powershell_foundation__ps_pre_your_first_terminal__ps_pre_your_first_terminal.md__exercise-2-using-tab-completion}

\begin{enumerate}
\tightlist
\item
  In your home directory, type \texttt{cd\ D} (just the letter D)
\item
  Press Tab
\item
  PowerShell should auto-complete to a folder starting with D
\item
  Repeat with other folder names
\item
  Try typing a longer name: \texttt{cd\ Down} and Tab to \texttt{Downloads}
\end{enumerate}

Goal: Tab completion should feel natural

\paragraph*{Exercise 3: Creating and Viewing Files}\label{docs__pandoc__latex__src__powershell_foundation__ps_pre_your_first_terminal__ps_pre_your_first_terminal.md__exercise-3-creating-and-viewing-files}

\begin{enumerate}
\tightlist
\item
  Create a file: \texttt{echo\ "Test\ content"\ \textgreater{}\ test.txt}
\item
  View it: \texttt{cat\ test.txt}
\item
  Create another: \texttt{echo\ "Line\ 2"\ \textgreater{}\ another.txt}
\item
  List both: \texttt{ls\ -n}
\end{enumerate}

Goal: You understand create, view, and list operations

\paragraph*{Exercise 4: Going Up Levels}\label{docs__pandoc__latex__src__powershell_foundation__ps_pre_your_first_terminal__ps_pre_your_first_terminal.md__exercise-4-going-up-levels}

\begin{enumerate}
\tightlist
\item
  Navigate into several folders: \texttt{cd\ Documents}, then \texttt{cd\ folder1}, etc.
\item
  From deep inside, use \texttt{cd\ ..} multiple times to go back up
\item
  After each \texttt{cd\ ..}, run \texttt{pwd} to confirm your location
\end{enumerate}

Goal: You understand relative navigation with \texttt{..}

\paragraph*{Exercise 5: Redirecting Output}\label{docs__pandoc__latex__src__powershell_foundation__ps_pre_your_first_terminal__ps_pre_your_first_terminal.md__exercise-5-redirecting-output}

\begin{enumerate}
\tightlist
\item
  Create a list: \texttt{ls\ -n\ \textgreater{}\ directory\_list.txt}
\item
  Open it: \texttt{notepad.exe\ directory\_list.txt}
\item
  Read it with your screen reader
\item
  Close Notepad
\item
  Verify the file exists: \texttt{ls\ -n\ \textbar{}\ findstr\ "directory"}
\end{enumerate}

Goal: You can save long outputs to files for easier reading

\subsubsection*{Checkpoint Questions}\label{docs__pandoc__latex__src__powershell_foundation__ps_pre_your_first_terminal__ps_pre_your_first_terminal.md__checkpoint-questions}

After completing this lesson, you should be able to answer:

\begin{enumerate}
\tightlist
\item
  What does \texttt{pwd} do?
\item
  What does \texttt{ls\ -n} do?
\item
  Why do we use \texttt{-n} with \texttt{ls}?
\item
  What path are you in right now?
\item
  How do you navigate to a new folder?
\item
  How do you go up one level?
\item
  What\textquotesingle s the Tab key for?
\item
  What does \texttt{echo\ "text"\ \textgreater{}\ file.txt} do?
\item
  How do you read a file back?
\item
  How do you stop a command that\textquotesingle s running?
\end{enumerate}

You should be able to answer all 10 with confidence before moving to PS-0.

\subsubsection*{Common Questions}\label{docs__pandoc__latex__src__powershell_foundation__ps_pre_your_first_terminal__ps_pre_your_first_terminal.md__common-questions}

Q: Do I need to use PowerShell? Can I use Command Prompt (cmd.exe)?
A: PowerShell is more powerful and works better with modern tools. We recommend PowerShell, but Command Prompt basics are similar.

Q: Why is my screen reader not reading the output?
A: This is common. Use \texttt{command\ \textgreater{}\ file.txt} to save output to a file, then open it with Notepad for reliable reading.

Q: What if I type something wrong?
A: Just press Enter and you\textquotesingle ll see an error message. Type the correct command on the next line. No harm done!

Q: How do I get help with a command?
A: Type \texttt{Get-Help\ command-name} (we\textquotesingle ll cover this in PS-0)

Q: Can I make PowerShell more accessible?
A: Yes! We\textquotesingle ll cover customization in PS-5.

\subsubsection*{Next Steps}\label{docs__pandoc__latex__src__powershell_foundation__ps_pre_your_first_terminal__ps_pre_your_first_terminal.md__next-steps}

Once you\textquotesingle re comfortable with these basics:

\begin{itemize}
\tightlist
\item
  Move to PS-0: Getting Started for deeper path understanding
\item
  Then continue through PS-1 through PS-5 for full terminal mastery
\end{itemize}

\subsubsection*{Resources}\label{docs__pandoc__latex__src__powershell_foundation__ps_pre_your_first_terminal__ps_pre_your_first_terminal.md__resources}

\begin{itemize}
\tightlist
\item
  Microsoft PowerShell Docs: \url{https://docs.microsoft.com/powershell/}
\item
  NVDA Screen Reader: \url{https://www.nvaccess.org/}
\item
  JAWS Screen Reader: \url{https://www.freedomscientific.com/products/software/jaws/}
\item
  Windows Terminal Accessibility: \url{https://docs.microsoft.com/windows/terminal/}
\end{itemize}

\subsubsection*{Troubleshooting}\label{docs__pandoc__latex__src__powershell_foundation__ps_pre_your_first_terminal__ps_pre_your_first_terminal.md__troubleshooting}

{\def\LTcaptype{none} % do not increment counter
\begin{longtable}[]{@{}
  >{\raggedright\arraybackslash}p{(\linewidth - 2\tabcolsep) * \real{0.2545}}
  >{\raggedright\arraybackslash}p{(\linewidth - 2\tabcolsep) * \real{0.7455}}@{}}
\toprule\noalign{}
\begin{minipage}[b]{\linewidth}\raggedright
Issue
\end{minipage} & \begin{minipage}[b]{\linewidth}\raggedright
Solution
\end{minipage} \\
\midrule\noalign{}
\endhead
\bottomrule\noalign{}
\endlastfoot
PowerShell won\textquotesingle t open &
Try searching Windows, or right-click a folder and select "Open PowerShell here" \\
Can\textquotesingle t hear the output &
Try redirecting to a file: \texttt{command\ \textgreater{}\ output.txt} \\
Tab completion not working &
Make sure you typed at least one character before pressing Tab \\
Command not found &
Make sure you spelled it correctly; try \texttt{Get-Command} to see available commands \\
Stuck in a command & Press Ctrl+C to stop it \\
\end{longtable}
}

Still stuck? The checkpoint questions and exercises are your best teacher. Work through them multiple times until comfortable.

Other Screen Readers

Dolphin SuperNova (commercial) and Windows Narrator (built-in) are also supported; the workflows and recommendations in this document apply to them. See \url{https://yourdolphin.com/supernova/} and \url{https://support.microsoft.com/narrator} for vendor documentation.

\subsection{PS-0: Getting Started - Layout, Paths, and the Shell}\label{docs__pandoc__latex__src__powershell_foundation__ps_0_getting_started_layout_paths__ps_0_getting_started_layout_paths.md__powershell_foundation_ps_0_getting_started_layout_paths-ps_0_getting_started_layout_paths}

Estimated time: 20-30 minutes

\subsubsection*{Learning Objectives}\label{docs__pandoc__latex__src__powershell_foundation__ps_0_getting_started_layout_paths__ps_0_getting_started_layout_paths.md__learning-objectives}

\begin{itemize}
\tightlist
\item
  Launch PowerShell and locate the prompt
\item
  Understand path notation and shortcuts (\texttt{\textasciitilde{}}, \texttt{./}, \texttt{../})
\item
  Use tab completion to navigate quickly
\end{itemize}

\subsubsection*{Materials}\label{docs__pandoc__latex__src__powershell_foundation__ps_0_getting_started_layout_paths__ps_0_getting_started_layout_paths.md__materials}

\begin{itemize}
\tightlist
\item
  Computer with PowerShell
\item
  Editor (Notepad/VS Code)
\end{itemize}

\subsubsection*{Step-by-step Tasks}\label{docs__pandoc__latex__src__powershell_foundation__ps_0_getting_started_layout_paths__ps_0_getting_started_layout_paths.md__step-by-step-tasks}

\begin{enumerate}
\tightlist
\item
  Open PowerShell and note the prompt (it includes the current path).
\item
  Run \texttt{pwd} and say or note the printed path.
\item
  Use \texttt{ls\ -n} to list names in your home directory.
\item
  Practice \texttt{cd\ Documents}, \texttt{cd\ ../} and \texttt{cd\ \textasciitilde{}} until comfortable.
\item
  Try tab-completion: type \texttt{cd\ \textasciitilde{}/D} and press Tab.
\end{enumerate}

\subsubsection*{Checkpoints}\label{docs__pandoc__latex__src__powershell_foundation__ps_0_getting_started_layout_paths__ps_0_getting_started_layout_paths.md__checkpoints}

\begin{itemize}
\tightlist
\item
  Confirm you can state your current path and move to \texttt{Documents}.
\end{itemize}

\subsubsection*{Quiz - Lesson PS.0}\label{docs__pandoc__latex__src__powershell_foundation__ps_0_getting_started_layout_paths__ps_0_getting_started_layout_paths.md__quiz---lesson-ps0}

\begin{enumerate}
\tightlist
\item
  What is a path?
\item
  What does \texttt{\textasciitilde{}} mean?
\item
  How do you autocomplete a path?
\item
  How do you go up one directory?
\item
  What command lists only names (\texttt{ls} flag)?
\item
  True or False: On Windows, PowerShell uses backslashes (\texttt{\textbackslash{}}) in paths, but forward slashes (/) are also accepted.
\item
  Explain the difference between an absolute path and a relative path.
\item
  If you are in \texttt{C:\textbackslash{}Users\textbackslash{}YourName\textbackslash{}Documents} and you type \texttt{cd\ ../}, where do you end up?
\item
  What happens when you press Tab while typing a folder name in PowerShell?
\item
  Describe a practical reason why understanding paths is important for a 3D printing workflow.
\item
  What does \texttt{./} mean in a path, and when would you use it?
\item
  If a folder path contains spaces (e.g., \texttt{Program\ Files}), how do you navigate to it with \texttt{cd}?
\item
  Explain what the prompt \texttt{PS\ C:\textbackslash{}Users\textbackslash{}YourName\textgreater{}} tells you about your current state.
\item
  How would you navigate to your home directory from any location using a single command?
\item
  What is the advantage of using relative paths (like \texttt{../}) versus absolute paths in automation scripts?
\end{enumerate}

\subsubsection*{Extension Problems}\label{docs__pandoc__latex__src__powershell_foundation__ps_0_getting_started_layout_paths__ps_0_getting_started_layout_paths.md__extension-problems}

\begin{enumerate}
\tightlist
\item
  Create a nested folder and practice \texttt{cd} into it by typing partial names and using Tab.
\item
  Use \texttt{ls\ -n\ -af} to list only files in a folder.
\item
  Save \texttt{pwd} output to a file and open it in Notepad.
\item
  Try \texttt{cd} into a folder whose name contains spaces; observe how quotes are handled.
\item
  Create a short note file and open it from PowerShell.
\item
  Build a folder structure that mirrors your project organization; navigate to each level and document the path.
\item
  Create a script that prints your current path and the total number of files in it; run it from different locations.
\item
  Investigate the special paths (e.g., \texttt{\$HOME}, \texttt{\$PSScriptRoot}); write down what each contains and when you\textquotesingle d use them.
\item
  Compare absolute vs. relative paths by navigating to the same folder using each method; explain which is easier for automation.
\item
  Create a PowerShell function that changes to a frequently-used folder and lists its contents in one command; test it from different starting locations.
\item
  Navigate to three different locations and at each one note the prompt, the path from \texttt{pwd}, and verify you understand what each shows.
\item
  Create a complex folder tree (at least 5 levels deep) and navigate it using only relative paths; verify your location at each step.
\item
  Document all shortcuts you know (\texttt{\textasciitilde{}}, \texttt{./}, \texttt{../}, \texttt{\$HOME}) and demonstrate each one works as expected.
\item
  Write a guide for a peer on how to understand the PowerShell prompt and path notation without using GUI file explorer.
\item
  Create a troubleshooting flowchart: if someone says "I don\textquotesingle t know where I am," what commands do you give them to find out?
\end{enumerate}

\subsubsection*{References}\label{docs__pandoc__latex__src__powershell_foundation__ps_0_getting_started_layout_paths__ps_0_getting_started_layout_paths.md__references}

\begin{itemize}
\tightlist
\item
  Microsoft. (2024). \emph{PowerShell scripting overview and documentation}. \url{https://learn.microsoft.com/powershell/scripting/overview}
\item
  Microsoft. (2024). \emph{Filesystem navigation in PowerShell}. \url{https://learn.microsoft.com/powershell/scripting/learn/shell/navigate-the-filesystem}
\item
  Microsoft. (2024). \emph{Accessibility features in PowerShell ISE}. \url{https://learn.microsoft.com/powershell/scripting/windows-powershell/ise/accessibility-in-windows-powershell-ise}
\end{itemize}

\subsubsection*{Helpful Resources}\label{docs__pandoc__latex__src__powershell_foundation__ps_0_getting_started_layout_paths__ps_0_getting_started_layout_paths.md__helpful-resources}

\begin{itemize}
\tightlist
\item
  \href{https://learn.microsoft.com/powershell/scripting/overview}{PowerShell Basics - Microsoft Learn}
\item
  \href{https://learn.microsoft.com/powershell/scripting/learn/shell/navigate-the-filesystem}{Filesystem Navigation Guide}
\item
  \href{https://poshcode.gitbook.io/powershell-faq/src/getting-started/filesystem-navigation}{Understanding Path Notation}
\item
  \href{https://learn.microsoft.com/powershell/module/psreadline/about/about_psreadline_functions}{Tab Completion Reference}
\item
  \href{https://learn.microsoft.com/powershell/scripting/windows-powershell/ise/accessibility-in-windows-powershell-ise}{Accessibility in PowerShell ISE}
\end{itemize}

\subsection{PS-1: Navigation - Moving Around Your File System}\label{docs__pandoc__latex__src__powershell_foundation__ps_1_navigation__ps_1_navigation.md__powershell_foundation_ps_1_navigation-ps_1_navigation}

Duration: 1 class period\\
Prerequisite: PS-0 (Getting Started)

Learning Objectives By the end of this lesson, you will be able to:

\begin{itemize}
\tightlist
\item
  Use \texttt{pwd} to print your current location
\item
  Use \texttt{cd} to move between directories
\item
  Use \texttt{ls} (and its flags) to list files and folders
\item
  Use wildcards \texttt{*} and \texttt{?} to filter listings
\item
  Navigate relative vs. absolute paths
\item
  Search for files by name and extension
\end{itemize}

Materials

\begin{itemize}
\tightlist
\item
  PowerShell
\item
  Text editor (Notepad or VS Code)
\end{itemize}

\subsubsection*{Commands Covered in This Lesson}\label{docs__pandoc__latex__src__powershell_foundation__ps_1_navigation__ps_1_navigation.md__commands-covered-in-this-lesson}

{\def\LTcaptype{none} % do not increment counter
\begin{longtable}[]{@{}
  >{\raggedright\arraybackslash}p{(\linewidth - 2\tabcolsep) * \real{0.2571}}
  >{\raggedright\arraybackslash}p{(\linewidth - 2\tabcolsep) * \real{0.7429}}@{}}
\toprule\noalign{}
\begin{minipage}[b]{\linewidth}\raggedright
Command
\end{minipage} & \begin{minipage}[b]{\linewidth}\raggedright
What It Does
\end{minipage} \\
\midrule\noalign{}
\endhead
\bottomrule\noalign{}
\endlastfoot
\texttt{pwd} & Print Working Directory - shows where you are \\
\texttt{cd\ path} & Change Directory - move to a new location \\
\texttt{ls} & List - shows files and folders in current location \\
\texttt{ls\ -n} & List names only (screen reader friendly) \\
\texttt{ls\ -n\ -af} & List names of files only \\
\texttt{ls\ -n\ -ad} & List names of directories only \\
\texttt{ls\ *.extension} & List files matching a pattern \\
\end{longtable}
}

\subsubsection*{\texorpdfstring{\texttt{pwd} - Where Am I?}{pwd - Where Am I?}}\label{docs__pandoc__latex__src__powershell_foundation__ps_1_navigation__ps_1_navigation.md__pwd---where-am-i}

Type \texttt{pwd} and press \texttt{Enter}. PowerShell prints the full path to your current location.

\begin{lstlisting}[style=Alabaster, language=powershell]
pwd
# Output: C:\Users\YourName

\end{lstlisting}

When to use: Always run this if you\textquotesingle re unsure of your current location.

\subsubsection*{\texorpdfstring{\texttt{cd} - Changing Directories}{cd - Changing Directories}}\label{docs__pandoc__latex__src__powershell_foundation__ps_1_navigation__ps_1_navigation.md__cd---changing-directories}

\texttt{cd} stands for "change directory."

\begin{lstlisting}[style=Alabaster, language=powershell]
# Go to Documents
cd Documents
# Go up one level to parent directory
cd ..
# Go to home directory
cd ~
# Go to a specific path
cd C:\Users\YourName\Documents\3D_Projects

\end{lstlisting}

\subsubsection*{\texorpdfstring{\texttt{ls} - Listing Files and Folders}{ls - Listing Files and Folders}}\label{docs__pandoc__latex__src__powershell_foundation__ps_1_navigation__ps_1_navigation.md__ls---listing-files-and-folders}

Use \texttt{ls\ -n} for screen reader compatibility.

\begin{lstlisting}[style=Alabaster, language=powershell]
# List all files and folders (names only)
ls -n
# List only files (no folders)
ls -n -af
# List only folders (no files)
ls -n -ad

\end{lstlisting}

\subsubsection*{Wildcards - Finding Files by Pattern}\label{docs__pandoc__latex__src__powershell_foundation__ps_1_navigation__ps_1_navigation.md__wildcards---finding-files-by-pattern}

Wildcards help you find files without typing the full name.

\texttt{*} (asterisk) matches any number of characters:

\begin{lstlisting}[style=Alabaster, language=powershell]
# List all .scad files
ls -n *.scad
# List all files starting with "part"
ls -n part*
# List all files ending with "_final"
ls -n *_final*

\end{lstlisting}

\texttt{?} (question mark) matches exactly one character:

\begin{lstlisting}[style=Alabaster, language=powershell]
# Find files like model1.scad, model2.scad (but not model12.scad)
ls -n model?.scad

\end{lstlisting}

\subsubsection*{Step-by-step Practice}\label{docs__pandoc__latex__src__powershell_foundation__ps_1_navigation__ps_1_navigation.md__step-by-step-practice}

\begin{enumerate}
\tightlist
\item
  Run \texttt{pwd} and confirm your location
\item
  Move to \texttt{Documents}: \texttt{cd\ Documents}
\item
  Confirm you moved: \texttt{pwd}
\item
  List files and folders: \texttt{ls\ -n}
\item
  List only files: \texttt{ls\ -n\ -af}
\item
  Go back up: \texttt{cd\ ..}
\item
  Search for files: \texttt{ls\ -n\ *.txt}
\end{enumerate}

\subsubsection*{Checkpoints}\label{docs__pandoc__latex__src__powershell_foundation__ps_1_navigation__ps_1_navigation.md__checkpoints}

After this lesson, you should be able to:

\begin{itemize}
\tightlist
\item[$\square$]
  Navigate to any folder using \texttt{cd}
\item[$\square$]
  Confirm your location with \texttt{pwd}
\item[$\square$]
  List files and folders with \texttt{ls\ -n}
\item[$\square$]
  Use wildcards to find files by pattern
\item[$\square$]
  Move between absolute and relative paths confidently
\end{itemize}

\subsubsection*{Quiz - Lesson PS.1}\label{docs__pandoc__latex__src__powershell_foundation__ps_1_navigation__ps_1_navigation.md__quiz---lesson-ps1}

\begin{enumerate}
\tightlist
\item
  What does \texttt{pwd} show?
\item
  How do you list directories only with \texttt{ls}?
\item
  What wildcard matches any number of characters?
\item
  How do you list files with the \texttt{.scad} extension?
\item
  Give an example of an absolute path and a relative path.
\item
  True or False: The \texttt{*} wildcard matches exactly one character.
\item
  Explain the difference between \texttt{ls\ -n} and \texttt{ls\ -n\ -ad}.
\item
  Write a command that would list all \texttt{.txt} files in your Documents folder using a wildcard.
\item
  How would you search for files containing "part" in their name across multiple files?
\item
  Describe a practical scenario where using wildcards saves time in a 3D printing workflow.
\item
  What happens when you use \texttt{ls\ -n\ part?.scad} versus \texttt{ls\ -n\ part*.scad}?
\item
  How would you navigate to a folder whose name contains both spaces and special characters?
\item
  If you\textquotesingle re in \texttt{/Documents/Projects/3D} and you want to go to \texttt{/Documents/Resources}, what command would you use?
\item
  Write a command sequence that navigates to the Downloads folder, lists only files, then returns to home.
\item
  Explain the purpose of using \texttt{ls\ -n\ -af} specifically in a screen reader context.
\end{enumerate}

\subsubsection*{Extension Problems}\label{docs__pandoc__latex__src__powershell_foundation__ps_1_navigation__ps_1_navigation.md__extension-problems}

\begin{enumerate}
\tightlist
\item
  Write a one-line script that lists \texttt{.scad} files and saves to \texttt{scad\_list.txt}.
\item
  Use \texttt{ls\ -n\ \textasciitilde{}/Documents\ \textbar{}\ more} to page through long listings.
\item
  Combine \texttt{ls} with \texttt{Select-String} to search for a filename pattern.
\item
  Create a shortcut alias in the session for a long path and test it.
\item
  Practice tab-completion in a directory with many similarly named files.
\item
  Build a PowerShell script that recursively lists all \texttt{.scad} and \texttt{.stl} files in a directory tree; save the results to a file.
\item
  Compare the output of \texttt{ls}, \texttt{Get-ChildItem}, and \texttt{gci} to understand PowerShell aliasing; document what each command does.
\item
  Create a filtering command that displays only files modified in the last 7 days; test it on your documents folder.
\item
  Write a non-visual guide to PowerShell navigation; include descriptions of common patterns and how to verify directory contents audibly.
\item
  Develop a navigation workflow for a typical 3D printing project: move between CAD, slicing, and print-log folders efficiently; document the commands.
\item
  Create a complex wildcard search: find all files in a folder and subfolders that match multiple patterns (e.g., \texttt{*\_v1.*\ OR\ *\_final.*}).
\item
  Build a script that navigates through a folder tree, counts files at each level, and reports the structure.
\item
  Document the output differences between \texttt{ls\ -n}, \texttt{ls\ -n\ -af}, \texttt{ls\ -n\ -ad}, and \texttt{Get-ChildItem}; explain when to use each.
\item
  Create a navigation "cheat sheet" as a PowerShell script that prints common paths and how to navigate to them.
\item
  Design a project folder structure on your computer, document each path, then create a script that validates all folders exist.
\end{enumerate}

\subsubsection*{References}\label{docs__pandoc__latex__src__powershell_foundation__ps_1_navigation__ps_1_navigation.md__references}

\begin{itemize}
\tightlist
\item
  Microsoft. (2024). \emph{Get-ChildItem cmdlet reference}. \url{https://learn.microsoft.com/powershell/module/microsoft.powershell.management/get-childitem}
\item
  Microsoft. (2024). \emph{PowerShell wildcards and filtering}. \url{https://learn.microsoft.com/powershell/scripting/learn/shell/using-wildcards}
\item
  Microsoft. (2024). \emph{Navigation best practices in PowerShell}. \url{https://learn.microsoft.com/powershell/scripting/learn/shell/navigate-the-filesystem}
\end{itemize}

\subsubsection*{Helpful Resources}\label{docs__pandoc__latex__src__powershell_foundation__ps_1_navigation__ps_1_navigation.md__helpful-resources}

\begin{itemize}
\tightlist
\item
  \href{https://learn.microsoft.com/powershell/module/microsoft.powershell.management/get-childitem}{Get-ChildItem Cmdlet Reference}
\item
  \href{https://learn.microsoft.com/powershell/scripting/learn/shell/using-wildcards}{PowerShell Wildcards and Filtering}
\item
  \href{https://learn.microsoft.com/powershell/scripting/learn/shell/navigate-the-filesystem}{Navigation Best Practices}
\item
  \href{https://poshcode.gitbook.io/powershell-faq/src/getting-started/filesystem-navigation}{Relative and Absolute Paths}
\item
  \href{https://learn.microsoft.com/powershell/scripting/windows-powershell/ise/accessibility-in-windows-powershell-ise}{Screen Reader Tips for PowerShell}
\end{itemize}

\subsection{PS-2: File and Folder Manipulation}\label{docs__pandoc__latex__src__powershell_foundation__ps_2_file_folder_manipulation_modification__ps_2_file_folder_manipulation_modification.md__powershell_foundation_ps_2_file_folder_manipulation_modification-ps_2_file_folder_manipulation_modification}

Estimated time: 30-45 minutes

Learning Objectives

\begin{itemize}
\tightlist
\item
  Create, copy, move, and delete files and folders from PowerShell
\item
  Use \texttt{ni}, \texttt{mkdir}, \texttt{cp}, \texttt{mv}, \texttt{rm}, and \texttt{rmdir} safely
\item
  Understand when operations are permanent and how to confirm results
\end{itemize}

Materials

\begin{itemize}
\tightlist
\item
  PowerShell
\item
  Small practice folder for exercises
\end{itemize}

Step-by-step Tasks

\begin{enumerate}
\tightlist
\item
  Create a practice directory: \texttt{mkdir\ \textasciitilde{}/Documents/PS\_Practice} and \texttt{cd} into it.
\item
  Create two files: \texttt{ni\ file1.txt} and \texttt{ni\ file2.txt}.
\item
  Copy \texttt{file1.txt} to \texttt{file1\_backup.txt} with \texttt{cp} and confirm with \texttt{ls\ -n}.
\item
  Rename \texttt{file2.txt} to \texttt{notes.txt} using \texttt{mv} and confirm.
\item
  Delete \texttt{file1.txt} with \texttt{rm} and verify the backup remains.
\end{enumerate}

Checkpoints

\begin{itemize}
\tightlist
\item
  After step 3 you should see both the original and the backup file.
\end{itemize}

\subsubsection*{Quiz - Lesson PS.2}\label{docs__pandoc__latex__src__powershell_foundation__ps_2_file_folder_manipulation_modification__ps_2_file_folder_manipulation_modification.md__quiz---lesson-ps2}

\begin{enumerate}
\tightlist
\item
  How do you create an empty file from PowerShell?
\item
  What command copies a file?
\item
  How do you rename a file?
\item
  What does \texttt{rm\ -r} do?
\item
  Why is \texttt{rm} potentially dangerous?
\item
  True or False: \texttt{cp} requires the \texttt{-r} flag to copy both files and folders.
\item
  Explain the difference between \texttt{rm} and \texttt{rmdir}.
\item
  If you delete a file with \texttt{rm}, can you recover it from PowerShell?
\item
  Write a command that would copy an entire folder and all its contents to a new location.
\item
  Describe a practical safety check you would perform before running \texttt{rm\ -r} on a folder.
\item
  What happens if you \texttt{cp} a file to a destination where a file with the same name already exists? How would you handle this safely?
\item
  Compare \texttt{mv\ old\_name.txt\ new\_name.txt} vs \texttt{mv\ old\_name.txt\ \textasciitilde{}/Documents/new\_name.txt}. What is the key difference?
\item
  Design a workflow to safely delete 50 files matching the pattern \texttt{*.bak} from a folder containing 500 files. What commands and verifications would you use?
\item
  Explain how you could back up all \texttt{.scad} files from a project folder into a timestamped backup folder in one command.
\item
  When organizing a 3D printing project, you need to move completed designs to an archive folder and delete failed prototypes. How would you structure this as a safe, auditable process?
\end{enumerate}

\subsubsection*{Extension Problems}\label{docs__pandoc__latex__src__powershell_foundation__ps_2_file_folder_manipulation_modification__ps_2_file_folder_manipulation_modification.md__extension-problems}

\begin{enumerate}
\tightlist
\item
  Create a folder tree and copy it to a new location with \texttt{cp\ -r}.
\item
  Write a one-line command that creates three files named \texttt{a\ b\ c} and lists them.
\item
  Move a file into a new folder and confirm the move.
\item
  Use wildcards to delete files matching a pattern in a safe test folder.
\item
  Export a listing of the practice folder to \texttt{practice\_listing.txt}.
\item
  Create a backup script that copies all \texttt{.scad} files from your project folder to a backup folder with timestamp naming.
\item
  Build a safe deletion workflow: list files matching a pattern, verify count, then delete with confirmation; document the steps.
\item
  Write a PowerShell script that organizes files by extension into subfolders; test it on a sample folder tree.
\item
  Create a file operation audit trail: log all copy, move, and delete operations to a text file for review.
\item
  Develop a project template generator: a script that creates a standard folder structure for a new 3D printing project with essential subfolders.
\item
  Implement a file conflict handler: write a script that handles cases where \texttt{cp} would overwrite an existing file by renaming the existing file with a timestamp before copying.
\item
  Create a batch rename operation: use a script to rename all files in a folder from \texttt{old\_prefix\_*} to \texttt{new\_prefix\_*}; test with actual files and verify the results.
\item
  Build a folder comparison tool: list all files in two folders and identify which files exist in one but not the other; output to a report.
\item
  Write a destructive operation validator: before executing \texttt{rm\ -r}, create a script that lists exactly what will be deleted, shows file counts by type, and requires explicit user confirmation to proceed.
\item
  Design a complete project lifecycle workflow: create folders for active projects, completed designs, and archive; include move operations between folders, backup steps, and verification that all files arrive intact.
\end{enumerate}

\subsubsection*{References}\label{docs__pandoc__latex__src__powershell_foundation__ps_2_file_folder_manipulation_modification__ps_2_file_folder_manipulation_modification.md__references}

\begin{itemize}
\tightlist
\item
  Microsoft. (2024). \emph{New-Item cmdlet reference}. \url{https://learn.microsoft.com/powershell/module/microsoft.powershell.management/new-item}
\item
  Microsoft. (2024). \emph{Copy-Item and Move-Item cmdlets}. \url{https://learn.microsoft.com/powershell/module/microsoft.powershell.management/copy-item}
\item
  Microsoft. (2024). \emph{File system operations guide}. \url{https://learn.microsoft.com/powershell/scripting/learn/shell/manipulating-items}
\end{itemize}

\subsubsection*{Helpful Resources}\label{docs__pandoc__latex__src__powershell_foundation__ps_2_file_folder_manipulation_modification__ps_2_file_folder_manipulation_modification.md__helpful-resources}

\begin{itemize}
\tightlist
\item
  \href{https://learn.microsoft.com/powershell/module/microsoft.powershell.management/new-item}{New-Item Cmdlet Reference}
\item
  \href{https://learn.microsoft.com/powershell/module/microsoft.powershell.management/copy-item}{Copy-Item and Move-Item}
\item
  \href{https://learn.microsoft.com/powershell/module/microsoft.powershell.management/remove-item}{Remove-Item Cmdlet Reference}
\item
  \href{https://learn.microsoft.com/powershell/scripting/learn/shell/manipulating-items}{File System Operations Guide}
\item
  \href{https://poshcode.gitbook.io/powershell-faq/src/getting-started/file-management}{Safe Deletion Practices}
\end{itemize}

\subsection{PS-3: Input, Output, and Piping}\label{docs__pandoc__latex__src__powershell_foundation__ps_3_input_output_piping__ps_3_input_output_piping.md__powershell_foundation_ps_3_input_output_piping-ps_3_input_output_piping}

Duration: 1 class period
Prerequisite: PS-2 (File and Folder Manipulation)

\subsubsection*{Learning Objectives}\label{docs__pandoc__latex__src__powershell_foundation__ps_3_input_output_piping__ps_3_input_output_piping.md__learning-objectives}

By the end of this lesson, you will be able to:

\begin{itemize}
\tightlist
\item
  Use \texttt{echo} to print text to the screen
\item
  Use \texttt{cat} to read file contents
\item
  Use \texttt{\textgreater{}} to redirect output into a file
\item
  Use \texttt{\textbar{}} (pipe) to send one command\textquotesingle s output to another
\item
  Copy output to the clipboard with \texttt{clip}
\item
  Open files with a text editor from the command line
\end{itemize}

\subsubsection*{Commands Covered}\label{docs__pandoc__latex__src__powershell_foundation__ps_3_input_output_piping__ps_3_input_output_piping.md__commands-covered}

{\def\LTcaptype{none} % do not increment counter
\begin{longtable}[]{@{}
  >{\raggedright\arraybackslash}p{(\linewidth - 2\tabcolsep) * \real{0.3288}}
  >{\raggedright\arraybackslash}p{(\linewidth - 2\tabcolsep) * \real{0.6712}}@{}}
\toprule\noalign{}
\begin{minipage}[b]{\linewidth}\raggedright
Command
\end{minipage} & \begin{minipage}[b]{\linewidth}\raggedright
What It Does
\end{minipage} \\
\midrule\noalign{}
\endhead
\bottomrule\noalign{}
\endlastfoot
\texttt{echo\ "text"} & Print text to the screen \\
\texttt{cat\ filename} & Print the contents of a file \\
\texttt{\textgreater{}\ filename} &
Redirect output into a file (overwrites) \\
\texttt{\textgreater{}\textgreater{}\ filename} &
Append output to a file (adds to end) \\
\texttt{\textbar{}} & Pipe - send output from one command to the next \\
\texttt{clip} & Copy piped input to the Windows clipboard \\
\texttt{notepad.exe\ filename} & Open a file in Notepad \\
\end{longtable}
}

\subsubsection*{\texorpdfstring{\texttt{echo} - Printing Text}{echo - Printing Text}}\label{docs__pandoc__latex__src__powershell_foundation__ps_3_input_output_piping__ps_3_input_output_piping.md__echo---printing-text}

\texttt{echo} prints text to the screen. It is useful for testing, for writing text into files, and for understanding how piping works.

\begin{lstlisting}[style=Alabaster, language=powershell]
echo "Hello, World"
echo "This is a test"

\end{lstlisting}

\subsubsection*{\texorpdfstring{\texttt{cat} - Reading Files}{cat - Reading Files}}\label{docs__pandoc__latex__src__powershell_foundation__ps_3_input_output_piping__ps_3_input_output_piping.md__cat---reading-files}

\texttt{cat} prints the contents of a file to the screen.

\begin{lstlisting}[style=Alabaster, language=powershell]
# Read a text file
cat ~/Documents/notes.txt
# Read an OpenSCAD file
cat ~/Documents/OpenSCAD_Projects/project0.scad

\end{lstlisting}

With a long file, use \texttt{cat\ filename\ \textbar{}\ more} to read it page by page (press \texttt{Space} to advance, \texttt{Q} to quit).

\subsubsection*{\texorpdfstring{\texttt{\textgreater{}} - Redirecting Output to a File}{\textgreater{} - Redirecting Output to a File}}\label{docs__pandoc__latex__src__powershell_foundation__ps_3_input_output_piping__ps_3_input_output_piping.md__---redirecting-output-to-a-file}

The \texttt{\textgreater{}} symbol redirects output from the screen into a file instead.

\begin{lstlisting}[style=Alabaster, language=powershell]
# Create a file with a single line
echo "Author: Your Name" > header.txt
# Confirm the file was created and has content
cat header.txt

\end{lstlisting}

Warning

\texttt{\textgreater{}} overwrites the file if it already exists. Use \texttt{\textgreater{}\textgreater{}} to append instead:

\begin{lstlisting}[style=Alabaster, language=powershell]
echo "Date: 2025" >> header.txt
echo "Project: Floor Marker" >> header.txt
cat header.txt

\end{lstlisting}

\subsubsection*{\texorpdfstring{\texttt{\textbar{}} - Piping}{\textbar{} - Piping}}\label{docs__pandoc__latex__src__powershell_foundation__ps_3_input_output_piping__ps_3_input_output_piping.md__---piping}

The pipe symbol \texttt{\textbar{}} sends the output of one command to the input of the next. This lets you chain commands together.

\begin{lstlisting}[style=Alabaster, language=powershell]
# List files and send the list to clip (copies to clipboard)
ls -n | clip
# Now paste with Ctrl + V anywhere

\end{lstlisting}

\begin{lstlisting}[style=Alabaster, language=powershell]
# Search within a file's contents using Select-String (like grep)
cat project0.scad | Select-String "cube"

\end{lstlisting}

\subsubsection*{\texorpdfstring{\texttt{clip} - Copying to Clipboard}{clip - Copying to Clipboard}}\label{docs__pandoc__latex__src__powershell_foundation__ps_3_input_output_piping__ps_3_input_output_piping.md__clip---copying-to-clipboard}

\texttt{clip} takes whatever is piped to it and puts it on the Windows clipboard.

\begin{lstlisting}[style=Alabaster, language=powershell]
# Copy your current directory path to the clipboard
pwd | clip
# Copy a file listing to clipboard
ls -n | clip
# Copy the contents of a file to clipboard
cat notes.txt | clip

\end{lstlisting}

After any of these, press \texttt{Ctrl\ +\ V} in any application to paste.

\subsubsection*{Opening Files in Notepad}\label{docs__pandoc__latex__src__powershell_foundation__ps_3_input_output_piping__ps_3_input_output_piping.md__opening-files-in-notepad}

\begin{lstlisting}[style=Alabaster, language=powershell]
# Open a file in Notepad
notepad.exe ~/Documents/notes.txt
# Open a .scad file
notepad.exe ~/Documents/OpenSCAD_Projects/project0.scad
# Create a new file and open it
notepad new_notes.txt

\end{lstlisting}

\subsubsection*{Step-by-step Tasks}\label{docs__pandoc__latex__src__powershell_foundation__ps_3_input_output_piping__ps_3_input_output_piping.md__step-by-step-tasks}

\begin{enumerate}
\tightlist
\item
  Create \texttt{practice.txt} with three lines using \texttt{echo} and \texttt{\textgreater{}}/\texttt{\textgreater{}\textgreater{}}.
\item
  Read the file with \texttt{cat\ practice.txt}.
\item
  Pipe the file into \texttt{Select-String} to search for a word.
\item
  Copy the file contents to clipboard with \texttt{cat\ practice.txt\ \textbar{}\ clip}.
\item
  Redirect \texttt{ls\ -n} into \texttt{list.txt} and open it in Notepad.
  Checkpoints
\end{enumerate}

\begin{itemize}
\tightlist
\item
  After step 3 you should be able to find a keyword using piping.
\end{itemize}

\subsubsection*{Quiz - Lesson PS.3}\label{docs__pandoc__latex__src__powershell_foundation__ps_3_input_output_piping__ps_3_input_output_piping.md__quiz---lesson-ps3}

\begin{enumerate}
\tightlist
\item
  What is the difference between \texttt{\textgreater{}} and \texttt{\textgreater{}\textgreater{}}?
\item
  What does the pipe \texttt{\textbar{}} do?
\item
  How do you copy output to the clipboard?
\item
  How would you page through long output?
\item
  How do you suppress output to nowhere?
\item
  True or False: The pipe operator \texttt{\textbar{}} connects the output of one command to the input of another.
\item
  Explain why redirecting output to a file is useful for screen reader users.
\item
  Write a command that would search for the word "sphere" in all \texttt{.scad} files in a directory.
\item
  How would you count the number of lines in a file using PowerShell piping?
\item
  Describe a practical scenario in 3D printing where you would pipe or redirect command output.
\item
  What would be the difference in output between \texttt{echo\ "test"\ \textgreater{}\ file.txt} (run twice) vs \texttt{echo\ "test"\ \textgreater{}\textgreater{}\ file.txt} (run twice)? Show the expected file contents.
\item
  Design a three-step piping chain: read a file, filter for specific content, and save the results; explain what each pipe does.
\item
  You have a 500-line \texttt{.scad} file and need to find all instances of \texttt{sphere()} and count them. Write the command.
\item
  Explain how \texttt{clip} is particularly valuable for screen reader users when working with file paths or long output strings.
\item
  Describe how you would use pipes and redirection to create a timestamped backup report of all \texttt{.stl} files in a 3D printing project folder.
\end{enumerate}

\subsubsection*{Extension Problems}\label{docs__pandoc__latex__src__powershell_foundation__ps_3_input_output_piping__ps_3_input_output_piping.md__extension-problems}

\begin{enumerate}
\tightlist
\item
  Use piping to count lines in a file (hint: \texttt{Select-String\ -Pattern\ \textquotesingle{}.\textquotesingle{}\ \textbar{}\ Measure-Object}).
\item
  Save a long \texttt{ls\ -n} output and search it with \texttt{Select-String}.
\item
  Chain multiple pipes to filter and then save results.
\item
  Practice copying different command outputs to clipboard and pasting.
\item
  Create a small script that generates a report (counts of files by extension).
\item
  Build a data processing pipeline: read a CSV file, filter rows, and export results; document each step.
\item
  Write a script that pipes directory listing to Count occurrences of each file extension; create a summary report.
\item
  Create a log analysis command: read a log file, filter for errors, and save matching lines to a separate error log.
\item
  Design a piping workflow for 3D printing file management: find \texttt{.stl} files, extract their sizes, and generate a report.
\item
  Develop a reusable piping function library: create functions for common filtering, sorting, and reporting patterns; test each function with different inputs.
\item
  Build a complex filter pipeline: read a \texttt{.scad} file, extract lines containing specific geometry commands, count each type, and output a summary to both screen and file.
\item
  Create an interactive piping tool: build a script that accepts user input for a search term, pipes through multiple filters, and displays paginated results.
\item
  Develop a performance analysis tool: use piping to combine file listing, metadata extraction, and statistical reporting; export results to a dated report file.
\item
  Implement a comprehensive error-handling pipeline: read output, catch errors, log them separately, and generate a summary of successes vs failures.
\item
  Design and execute a real-world project backup workflow: use piping to verify file integrity, count files by type, generate a backup manifest, and create audit logs-all in one integrated command pipeline.
\end{enumerate}

\subsubsection*{References}\label{docs__pandoc__latex__src__powershell_foundation__ps_3_input_output_piping__ps_3_input_output_piping.md__references}

\begin{itemize}
\tightlist
\item
  Microsoft. (2024). \emph{Out-File cmdlet for redirection}. \url{https://learn.microsoft.com/en-us/powershell/module/microsoft.powershell.utility/out-file}
\item
  Microsoft. (2024). \emph{Select-String piping reference}. \url{https://learn.microsoft.com/en-us/powershell/module/microsoft.powershell.utility/select-string}
\item
  Microsoft. (2024). \emph{PowerShell pipeline concepts}. \url{https://learn.microsoft.com/en-us/powershell/scripting/learn/deep-dives/everything-about-pipeline}
\end{itemize}

\subsubsection*{Helpful Resources}\label{docs__pandoc__latex__src__powershell_foundation__ps_3_input_output_piping__ps_3_input_output_piping.md__helpful-resources}

\begin{itemize}
\tightlist
\item
  \href{https://learn.microsoft.com/powershell/module/microsoft.powershell.utility/out-file}{Using Out-File for Redirection}
\item
  \href{https://learn.microsoft.com/powershell/module/microsoft.powershell.utility/select-string}{Piping and Select-String}
\item
  \href{https://learn.microsoft.com/powershell/module/microsoft.powershell.management/get-content}{Get-Content Cmdlet Reference}
\item
  \href{https://learn.microsoft.com/powershell/module/microsoft.powershell.utility/measure-object}{Measure-Object for Counting}
\item
  \href{https://learn.microsoft.com/powershell/scripting/learn/shell/using-the-pipeline}{PowerShell Pipeline Concept}
\end{itemize}

\subsection{PS-4: Environment Variables, PATH, and Aliases}\label{docs__pandoc__latex__src__powershell_foundation__ps_4_environment_variables_aliases__ps_4_environment_variables_aliases.md__powershell_foundation_ps_4_environment_variables_aliases-ps_4_environment_variables_aliases}

Estimated time: 30-45 minutes

Learning Objectives

\begin{itemize}
\tightlist
\item
  Read environment variables with \texttt{\$env:VARNAME}
\item
  Inspect and verify programs in the \texttt{PATH}
\item
  Create temporary aliases and understand making them persistent via the profile
\end{itemize}

Materials

\begin{itemize}
\tightlist
\item
  PowerShell (with rights to open profile if desired)
\end{itemize}

Step-by-step Tasks

\begin{enumerate}
\tightlist
\item
  Show your username and home path with \texttt{echo\ \$env:USERNAME} and \texttt{echo\ \$env:USERPROFILE}.
\item
  Inspect \texttt{echo\ \$env:PATH} and identify whether \texttt{openscad} or \texttt{code} would be found.
\item
  Run \texttt{Get-Command\ openscad} and note the result.
\item
  Create a temporary alias: \texttt{Set-Alias\ -Name\ preview\ -Value\ openscad} and test \texttt{preview\ myfile.scad}.
\item
  Open your profile (\texttt{notepad.exe\ \$PROFILE}) and add the alias line to make it persistent (advanced).
\end{enumerate}

Checkpoints

\begin{itemize}
\tightlist
\item
  After step 3 you can determine whether a program will be found by \texttt{PATH}.
\end{itemize}

\subsubsection*{Quiz - Lesson PS.4}\label{docs__pandoc__latex__src__powershell_foundation__ps_4_environment_variables_aliases__ps_4_environment_variables_aliases.md__quiz---lesson-ps4}

\begin{enumerate}
\tightlist
\item
  How do you print an environment variable?
\item
  What is the purpose of \texttt{PATH}?
\item
  How do you check whether \texttt{openscad} is available?
\item
  How do you create a temporary alias?
\item
  Where would you make an alias permanent?
\item
  True or False: Environment variables are case-sensitive on all platforms.
\item
  Explain why having a program in your PATH is useful compared to always using its full file path.
\item
  Write a command that would create an alias called \texttt{slicer} for the OpenSCAD executable.
\item
  What file would you edit to make an alias persist across PowerShell sessions?
\item
  Describe a practical benefit of using the \texttt{\$env:TEMP} directory for temporary files in a 3D printing workflow.
\item
  You have a custom script at \texttt{C:\textbackslash{}Scripts\textbackslash{}backup\_models.ps1} that you want to run from anywhere as \texttt{backup-now}. What steps would you take to make this work?
\item
  Explain the difference between setting an environment variable in the current session vs. adding it to your profile for permanence.
\item
  Design a profile strategy for managing multiple 3D printing projects, each with different tool paths and directories; show how to structure environment variables for each.
\item
  If a program is not found by \texttt{Get-Command}, what are the possible reasons, and how would you troubleshoot?
\item
  Describe how you would verify that your PowerShell profile is loading correctly and how to debug issues if aliases or environment variables don\textquotesingle t appear after restarting PowerShell.
\end{enumerate}

\subsubsection*{Extension Problems}\label{docs__pandoc__latex__src__powershell_foundation__ps_4_environment_variables_aliases__ps_4_environment_variables_aliases.md__extension-problems}

\begin{enumerate}
\tightlist
\item
  Add a folder to PATH for a test program (describe steps; do not change system PATH without admin).
\item
  Create a short profile snippet that sets two aliases and test re-opening PowerShell.
\item
  Use \texttt{Get-Command} to list the path for several common programs.
\item
  Explore \texttt{\$env:TEMP} and create a file there.
\item
  Save a copy of your current PATH to a text file and examine it in your editor.
\item
  Create a PowerShell profile script that loads custom aliases and environment variables for your 3D printing workflow; test it in a new session.
\item
  Build a "project profile" that sets environment variables for CAD, slicing, and print directories; switch between profiles for different projects.
\item
  Write a script that audits your current environment variables and creates a summary report of what\textquotesingle s set and why.
\item
  Design a custom alias system for common 3D printing commands; document the aliases and their purposes.
\item
  Create a profile migration guide: document how to export and import your PowerShell profile across machines for consistent workflows.
\item
  Implement a safe PATH modification script: create a utility that allows you to add/remove directories from PATH for the current session only; show how to make it permanent in your profile.
\item
  Build a comprehensive profile framework with modules: create separate .ps1 files for aliases, environment variables, and functions; have your main profile load all of them dynamically.
\item
  Develop an environment validation tool: write a script that checks whether all required programs (OpenSCAD, slicers, etc.) are accessible via PATH; report findings and suggest fixes.
\item
  Create a project-switching alias system: design a function that changes all environment variables and aliases based on the current project; test switching between multiple projects.
\item
  Build a profile troubleshooting guide: create a script that exports your current environment state (variables, aliases, PATH) to a timestamped file, allowing you to compare states before and after changes and identify what broke.
\end{enumerate}

\subsubsection*{References}\label{docs__pandoc__latex__src__powershell_foundation__ps_4_environment_variables_aliases__ps_4_environment_variables_aliases.md__references}

\begin{itemize}
\tightlist
\item
  Microsoft. (2024). \emph{Environment variables in PowerShell}. \url{https://learn.microsoft.com/powershell/scripting/learn/shell/using-environment-variables}
\item
  Microsoft. (2024). \emph{Set-Alias cmdlet reference}. \url{https://learn.microsoft.com/powershell/module/microsoft.powershell.utility/set-alias}
\item
  Microsoft. (2024). \emph{Creating and using PowerShell profiles}. \url{https://learn.microsoft.com/powershell/module/microsoft.powershell.core/about/about_profiles}
\end{itemize}

\subsubsection*{Helpful Resources}\label{docs__pandoc__latex__src__powershell_foundation__ps_4_environment_variables_aliases__ps_4_environment_variables_aliases.md__helpful-resources}

\begin{itemize}
\tightlist
\item
  \href{https://learn.microsoft.com/powershell/scripting/learn/shell/using-environment-variables}{Environment Variables in PowerShell}
\item
  \href{https://poshcode.gitbook.io/powershell-faq/src/getting-started/environment-variables}{Understanding the PATH Variable}
\item
  \href{https://learn.microsoft.com/powershell/module/microsoft.powershell.utility/set-alias}{Set-Alias Cmdlet Reference}
\item
  \href{https://learn.microsoft.com/powershell/module/microsoft.powershell.core/about/about_profiles}{Creating a PowerShell Profile}
\item
  \href{https://learn.microsoft.com/powershell/module/microsoft.powershell.core/get-command}{Get-Command for Locating Programs}
\end{itemize}

\subsection{PS-5: Filling in the Gaps - Control Flow, Profiles, and Useful Tricks}\label{docs__pandoc__latex__src__powershell_foundation__ps_5_filling_in_the_gaps__ps_5_filling_in_the_gaps.md__powershell_foundation_ps_5_filling_in_the_gaps-ps_5_filling_in_the_gaps}

Estimated time: 30-45 minutes

Learning Objectives

\begin{itemize}
\tightlist
\item
  Use history and abort commands (\texttt{history}, \texttt{Ctrl+C})
\item
  Inspect and edit your PowerShell profile for persistent settings
\item
  Run programs by full path using the \texttt{\&} operator
\item
  Handle common screen reader edge cases when using the terminal
\end{itemize}

Materials

\begin{itemize}
\tightlist
\item
  PowerShell and an editor (Notepad/ VS Code)
\end{itemize}

Step-by-step Tasks

\begin{enumerate}
\tightlist
\item
  Run several simple commands (e.g., \texttt{pwd}, \texttt{ls\ -n}, \texttt{echo\ hi}) then run \texttt{history} to view them.
\item
  Use \texttt{Invoke-History\ \textless{}n\textgreater{}} to re-run a previous command (replace \texttt{\textless{}n\textgreater{}} with a history number).
\item
  Practice aborting a long-running command with \texttt{Ctrl\ +\ C} (for example, \texttt{ping\ 8.8.8.8}).
\item
  Open your profile: \texttt{notepad.exe\ \$PROFILE}; if it doesn\textquotesingle t exist, create it: \texttt{ni\ \$PROFILE\ -Force}.
\item
  Add a persistent alias line to your profile (example: \texttt{Set-Alias\ -Name\ preview\ -Value\ openscad}), save, and reopen PowerShell to verify.
\end{enumerate}

Checkpoints

\begin{itemize}
\tightlist
\item
  After step 2 you can re-run a recent command by history number.
\item
  After step 5 your alias should persist across sessions.
\end{itemize}

\subsubsection*{Quiz - Lesson PS.5}\label{docs__pandoc__latex__src__powershell_foundation__ps_5_filling_in_the_gaps__ps_5_filling_in_the_gaps.md__quiz---lesson-ps5}

\begin{enumerate}
\tightlist
\item
  How do you view the command history?
\item
  Which key combination aborts a running command?
\item
  What does \texttt{echo\ \$PROFILE} show?
\item
  How does the \texttt{\&} operator help run executables?
\item
  What is one strategy if terminal output stops being announced by your screen reader?
\item
  True or False: Using \texttt{Ctrl+C} permanently deletes any files created by the command you abort.
\item
  Explain the difference between \texttt{history} and \texttt{Get-History} in PowerShell.
\item
  If you place code in your profile but it doesn\textquotesingle t take effect after opening a new PowerShell window, what should you verify?
\item
  Write a command that would run a program at the path \texttt{C:\textbackslash{}Program\ Files\textbackslash{}OpenSCAD\textbackslash{}openscad.exe} directly.
\item
  Describe a practical workflow scenario where having keyboard shortcuts (aliases) in your profile would save time.
\item
  Explain how to re-run the 5th command from your history, and what would happen if that command had file operations (creates/deletes).
\item
  Design a profile initialization strategy that separates utilities for different projects; explain how you would switch between them.
\item
  Walk through a troubleshooting workflow: your screen reader stops announcing output after running a long command. What steps would you take to diagnose and resolve the issue?
\item
  Create a safety checkpoint system: before any destructive operation (mass delete, overwrite), how would you use profile functions and history to verify the command is correct?
\item
  Develop a comprehensive capstone scenario: integrate everything from PS-0 through PS-5 (navigation, file operations, piping, environment setup, history) to design an automated 3D printing project workflow with error handling and logging.
\end{enumerate}

\subsubsection*{Extension Problems}\label{docs__pandoc__latex__src__powershell_foundation__ps_5_filling_in_the_gaps__ps_5_filling_in_the_gaps.md__extension-problems}

\begin{enumerate}
\tightlist
\item
  Add an alias and an environment variable change to your profile and document the behavior after reopening PowerShell.
\item
  Create a short script that automates creating a project folder and an initial .scad file.
\item
  Experiment with running OpenSCAD by full path using \texttt{\&} and by placing it in PATH; compare results.
\item
  Practice redirecting \texttt{Get-Help} output to a file and reading it in Notepad for screen reader clarity.
\item
  Document three screen reader troubleshooting steps you used and when they helped.
\item
  Build a comprehensive PowerShell profile that includes aliases, environment variables, and helper functions for your 3D printing workflow.
\item
  Create a script that troubleshoots common PowerShell issues (module loading, permission errors, command not found); test at least three scenarios.
\item
  Write a PowerShell function that coordinates multiple tasks: creates a project folder, starts OpenSCAD, and opens slicing software.
\item
  Design a screen-reader accessibility guide for PowerShell: document commands, outputs, and accessible navigation patterns.
\item
  Develop an advanced PowerShell workflow: implement error handling, logging, and confirmation prompts for risky operations.
\item
  Implement a "undo" system using history: create a function that logs destructive commands (rm, mv, cp -Force) and allows you to review/rollback the last operation.
\item
  Build a profile debugger: create a script that compares two PowerShell sessions\textquotesingle{} environment states (variables, aliases, functions) to identify what loaded/failed to load.
\item
  Develop a multi-project profile manager: design a system where you can switch entire environments (paths, aliases, variables) for different 3D printing projects by running a single command.
\item
  Create a comprehensive accessibility analyzer: write a script that tests whether key PowerShell commands produce screen-reader-friendly output; document workarounds for commands that don\textquotesingle t.
\item
  Design a complete capstone project: build an integrated automation suite that manages a 3D printing workflow (project setup, file organization, CAD/slicing tool automation, output logging, error recovery, and audit trails) with full error handling and documentation.
\end{enumerate}

\subsubsection*{References}\label{docs__pandoc__latex__src__powershell_foundation__ps_5_filling_in_the_gaps__ps_5_filling_in_the_gaps.md__references}

\begin{itemize}
\tightlist
\item
  Microsoft. (2024). \emph{PowerShell history and recall functionality}. \url{https://learn.microsoft.com/powershell/scripting/learn/shell/using-history}
\item
  Microsoft. (2024). \emph{Understanding and creating PowerShell profiles}. \url{https://learn.microsoft.com/powershell/module/microsoft.powershell.core/about/about_profiles}
\item
  Microsoft. (2024). \emph{The call operator (\&) for running executables}. \url{https://learn.microsoft.com/powershell/module/microsoft.powershell.core/about/about_operators\#call-operator-}
\end{itemize}

\subsubsection*{Helpful Resources}\label{docs__pandoc__latex__src__powershell_foundation__ps_5_filling_in_the_gaps__ps_5_filling_in_the_gaps.md__helpful-resources}

\begin{itemize}
\tightlist
\item
  \href{https://learn.microsoft.com/powershell/scripting/learn/shell/using-history}{PowerShell History and Recall}
\item
  \href{https://learn.microsoft.com/powershell/module/microsoft.powershell.core/about/about_profiles}{Understanding Profiles}
\item
  \href{https://learn.microsoft.com/powershell/module/microsoft.powershell.core/invoke-history}{Invoke-History Cmdlet Reference}
\item
  \href{https://learn.microsoft.com/powershell/module/microsoft.powershell.core/about/about_operators\#call-operator-}{The Call Operator (\&)}
\item
  \href{https://learn.microsoft.com/powershell/scripting/windows-powershell/ise/accessibility-in-windows-powershell-ise}{Screen Reader Tips and Tricks}
\end{itemize}

\subsection{PS-6: Advanced Terminal Techniques - Scripts, Functions \& Professional Workflows}\label{docs__pandoc__latex__src__powershell_foundation__ps_6_advanced_techniques__ps_6_advanced_techniques.md__powershell_foundation_ps_6_advanced_techniques-ps_6_advanced_techniques}

Duration: 4-4.5 hours (for screen reader users)\\
Prerequisites: Complete PS-0 through PS-5\\
Skill Level: Advanced intermediate

This lesson extends PowerShell skills to professional-level workflows. You\textquotesingle ll learn to automate complex tasks, write reusable code, and integrate tools for 3D printing workflows.

\subsubsection*{Learning Objectives}\label{docs__pandoc__latex__src__powershell_foundation__ps_6_advanced_techniques__ps_6_advanced_techniques.md__learning-objectives}

By the end of this lesson, you will be able to:

\begin{itemize}
\tightlist
\item
  Create and run PowerShell scripts (.ps1 files)
\item
  Write functions that accept parameters
\item
  Use loops to repeat tasks automatically
\item
  Automate batch processing of 3D models
\item
  Debug scripts when something goes wrong
\item
  Create professional workflows combining multiple tools
\end{itemize}

\subsubsection*{PowerShell Scripts Basics}\label{docs__pandoc__latex__src__powershell_foundation__ps_6_advanced_techniques__ps_6_advanced_techniques.md__powershell-scripts-basics}

\paragraph*{What\textquotesingle s a Script?}\label{docs__pandoc__latex__src__powershell_foundation__ps_6_advanced_techniques__ps_6_advanced_techniques.md__whats-a-script}

A script is a file containing PowerShell commands that run in sequence. Instead of typing commands one by one, you put them in a file and run them all at once.

Why use scripts?

\begin{itemize}
\tightlist
\item
  Repeatability: Run the same task 100 times identically
\item
  Documentation: Commands are written down for reference
\item
  Complexity: Combine many commands logically
\item
  Automation: Schedule scripts to run automatically
\end{itemize}

\paragraph*{Creating Your First Script}\label{docs__pandoc__latex__src__powershell_foundation__ps_6_advanced_techniques__ps_6_advanced_techniques.md__creating-your-first-script}

Step 1: Open a text editor

\begin{lstlisting}[style=Alabaster, language=powershell]
notepad.exe my-first-script.ps1

\end{lstlisting}

Step 2: Type this script

\begin{lstlisting}[style=Alabaster, language=powershell]
# This is a comment - screen readers will read it
Write-Output "Script is running!"
pwd
ls -n
Write-Output "Script is done!"

\end{lstlisting}

Step 3: Save the file

\begin{itemize}
\tightlist
\item
  In Notepad: Ctrl+S
\item
  Make sure filename ends in \texttt{.ps1}
\item
  Save in an easy-to-find location (like Documents)
\end{itemize}

Step 4: Run the script

\begin{lstlisting}[style=Alabaster, language=powershell]
.\my-first-script.ps1

\end{lstlisting}

What happens: PowerShell runs each command in sequence and shows output.

\paragraph*{Important: Script Execution Policy}\label{docs__pandoc__latex__src__powershell_foundation__ps_6_advanced_techniques__ps_6_advanced_techniques.md__important-script-execution-policy}

On some Windows systems, you might get an error about "execution policy". This is a security feature.

To fix it temporarily:

\begin{lstlisting}[style=Alabaster, language=powershell]
Set-ExecutionPolicy -ExecutionPolicy Bypass -Scope Process

\end{lstlisting}

Then try your script again:

\begin{lstlisting}[style=Alabaster, language=powershell]
.\my-first-script.ps1

\end{lstlisting}

Note for screen readers: Your screen reader will announce the error if there is one. Use \texttt{Get-Help\ Get-ExecutionPolicy} for more information.

\subsubsection*{Variables and Parameters}\label{docs__pandoc__latex__src__powershell_foundation__ps_6_advanced_techniques__ps_6_advanced_techniques.md__variables-and-parameters}

\paragraph*{Using Variables}\label{docs__pandoc__latex__src__powershell_foundation__ps_6_advanced_techniques__ps_6_advanced_techniques.md__using-variables}

Variables store values you want to use later.

Example script:

\begin{lstlisting}[style=Alabaster, language=powershell]
$mypath = "C:\Users\YourName\Documents"
cd $mypath
Write-Output "I am now in:"
pwd
ls -n

\end{lstlisting}

Breaking it down:

\begin{itemize}
\tightlist
\item
  \texttt{\$mypath} = variable name (always starts with \texttt{\$})
\item
  \texttt{=} = assign the value after this
\item
  \texttt{"C:\textbackslash{}Users..."} = the value (a path)
\item
  \texttt{cd\ \$mypath} = use the variable (replace \texttt{\$mypath} with its value)
\end{itemize}

\paragraph*{Functions with Parameters}\label{docs__pandoc__latex__src__powershell_foundation__ps_6_advanced_techniques__ps_6_advanced_techniques.md__functions-with-parameters}

A function is reusable code that you can run with different inputs.

Example: A function that lists files in a folder

\begin{lstlisting}[style=Alabaster, language=powershell]
function ListFolder {
    param(
        [string]$path
    )
    Write-Output "Contents of: $path"
    cd $path
    ls -n
}
# Use the function:
ListFolder -path "C:\Users\YourName\Documents"
ListFolder -path "C:\Users\YourName\Downloads"

\end{lstlisting}

What\textquotesingle s happening:

\begin{itemize}
\tightlist
\item
  \texttt{function\ ListFolder} = name of the function
\item
  \texttt{param({[}string{]}\$path)} = the function accepts a parameter called \texttt{\$path}
\item
  Inside the function, use \texttt{\$path} like any variable
\item
  Call the function with \texttt{-path\ "value"}
\end{itemize}

Screen reader tip: When you call a function, PowerShell will announce the results just like any command.

\subsubsection*{Loops - Repeating Tasks}\label{docs__pandoc__latex__src__powershell_foundation__ps_6_advanced_techniques__ps_6_advanced_techniques.md__loops---repeating-tasks}

\paragraph*{Loop Over Files}\label{docs__pandoc__latex__src__powershell_foundation__ps_6_advanced_techniques__ps_6_advanced_techniques.md__loop-over-files}

Imagine you have 10 SCAD files and want to print their contents. You could do it 10 times manually, or use a loop.

Example: Print every .scad file in a folder

\begin{lstlisting}[style=Alabaster, language=powershell]
$scadFiles = ls -n *.scad
foreach ($file in $scadFiles) {
    Write-Output "=== File: $file ==="
    cat $file
    Write-Output ""
}

\end{lstlisting}

What\textquotesingle s happening:

\begin{itemize}
\tightlist
\item
  \texttt{\$scadFiles\ =\ ls\ -n\ *.scad} = find all .scad files and store in variable
\item
  \texttt{foreach\ (\$file\ in\ \$scadFiles)} = for each file, do this:

  \begin{itemize}
  \tightlist
  \item
    \texttt{Write-Output\ "===\ File:\ \$file\ ==="} = announce which file
  \item
    \texttt{cat\ \$file} = show contents
  \item
    \texttt{Write-Output\ ""} = blank line between files
  \end{itemize}
\end{itemize}

Result: All files printed one after another, organized and readable.

\paragraph*{Loop with a Counter}\label{docs__pandoc__latex__src__powershell_foundation__ps_6_advanced_techniques__ps_6_advanced_techniques.md__loop-with-a-counter}

Example: Do something 5 times

\begin{lstlisting}[style=Alabaster, language=powershell]
for ($i = 1; $i -le 5; $i++) {
    Write-Output "This is iteration number $i"
    # Do something here
}

\end{lstlisting}

What\textquotesingle s happening:

\begin{itemize}
\tightlist
\item
  \texttt{for\ (\$i\ =\ 1;\ \$i\ -le\ 5;\ \$i++)} = loop from 1 to 5
\item
  \texttt{\$i} = counter variable (starts at 1, increases each loop)
\item
  \texttt{-le} = "less than or equal to" (stop when \$i \textgreater{} 5)
\item
  \texttt{\$i++} = add 1 to \$i each time through
\end{itemize}

\subsubsection*{Real-World Example - Batch Processing SCAD Files}\label{docs__pandoc__latex__src__powershell_foundation__ps_6_advanced_techniques__ps_6_advanced_techniques.md__real-world-example---batch-processing-scad-files}

\paragraph*{Scenario}\label{docs__pandoc__latex__src__powershell_foundation__ps_6_advanced_techniques__ps_6_advanced_techniques.md__scenario}

You have 10 OpenSCAD (.scad) files in a folder. You want to:

\begin{enumerate}
\tightlist
\item
  List them all
\item
  Check how many there are
\item
  For each one, verify it exists
\end{enumerate}

\paragraph*{The Script}\label{docs__pandoc__latex__src__powershell_foundation__ps_6_advanced_techniques__ps_6_advanced_techniques.md__the-script}

\begin{lstlisting}[style=Alabaster, language=powershell]
# Batch Processing Script for SCAD Files
$scadFolder = "C:\Users\YourName\Documents\3D_Projects"
$scadFiles = ls $scadFolder -Filter *.scad -Name
Write-Output "Processing SCAD files in: $scadFolder"
Write-Output "Found $($scadFiles.Count) files"
Write-Output ""
foreach ($file in $scadFiles) {
    $fullPath = Join-Path -Path $scadFolder -ChildPath $file
    if (Test-Path -Path $fullPath) {
        Write-Output " Found: $file"
    } else {
        Write-Output " Missing: $file"
    }
}
Write-Output ""
Write-Output "Batch processing complete!"

\end{lstlisting}

Breaking it down:

\begin{itemize}
\tightlist
\item
  \texttt{\$scadFolder} = where to look
\item
  \texttt{ls\ \$scadFolder\ -Filter\ *.scad\ -Name} = find .scad files, show names only
\item
  \texttt{foreach} = process each file
\item
  \texttt{Join-Path} = combine folder and filename into full path
\item
  \texttt{Test-Path} = check if file exists
\item
  \texttt{if} = do different things based on condition
\end{itemize}

\paragraph*{Running the Script}\label{docs__pandoc__latex__src__powershell_foundation__ps_6_advanced_techniques__ps_6_advanced_techniques.md__running-the-script}

\begin{enumerate}
\tightlist
\item
  Save as \texttt{batch-process.ps1}
\item
  Edit \texttt{\$scadFolder} to match your real folder
\item
  Run it:

  \begin{lstlisting}[style=Alabaster, language=powershell]
  .\batch-process.ps1

  \end{lstlisting}
\end{enumerate}

Screen reader output:

\begin{lstlisting}[style=Alabaster]
Processing SCAD files in: C:\Users\YourName\Documents\3D_Projects
Found 10 files

 Found: model1.scad
 Found: model2.scad
 Found: model3.scad
[... more files ...]
Batch processing complete!

\end{lstlisting}

\subsubsection*{Error Handling}\label{docs__pandoc__latex__src__powershell_foundation__ps_6_advanced_techniques__ps_6_advanced_techniques.md__error-handling}

\paragraph*{Try-Catch Blocks}\label{docs__pandoc__latex__src__powershell_foundation__ps_6_advanced_techniques__ps_6_advanced_techniques.md__try-catch-blocks}

What if something goes wrong? Use try-catch:

Example:

\begin{lstlisting}[style=Alabaster, language=powershell]
try {
    $file = "C:\nonexistent\path\file.txt"
    $content = cat $file
    Write-Output $content
} catch {
    Write-Output "Error: Could not read file"
    Write-Output "Details: $_"
}

\end{lstlisting}

What\textquotesingle s happening:

\begin{itemize}
\tightlist
\item
  \texttt{try} = run these commands
\item
  If an error happens, PowerShell jumps to \texttt{catch}
\item
  \texttt{catch} = handle the error gracefully
\item
  \texttt{\$\_} = the error message
\end{itemize}

Screen reader advantage: Errors are announced clearly instead of crashing silently.

\paragraph*{Validating Input}\label{docs__pandoc__latex__src__powershell_foundation__ps_6_advanced_techniques__ps_6_advanced_techniques.md__validating-input}

Example: Make sure a folder exists before processing

\begin{lstlisting}[style=Alabaster, language=powershell]
function ProcessFolder {
    param([string]$folderPath)
    if (-not (Test-Path -Path $folderPath)) {
        Write-Output "Error: Folder does not exist: $folderPath"
        return
    }
    Write-Output "Processing folder: $folderPath"
    ls -n $folderPath
}
ProcessFolder -folderPath "C:\Users\YourName\Documents"

\end{lstlisting}

What\textquotesingle s happening:

\begin{itemize}
\tightlist
\item
  \texttt{Test-Path} = check if folder exists
\item
  \texttt{-not} = if NOT true
\item
  \texttt{return} = exit the function early if error
\end{itemize}

\subsubsection*{Debugging Scripts}\label{docs__pandoc__latex__src__powershell_foundation__ps_6_advanced_techniques__ps_6_advanced_techniques.md__debugging-scripts}

\paragraph*{Common Errors and Solutions}\label{docs__pandoc__latex__src__powershell_foundation__ps_6_advanced_techniques__ps_6_advanced_techniques.md__common-errors-and-solutions}

\subparagraph*{Error 1: "Command not found"}\label{docs__pandoc__latex__src__powershell_foundation__ps_6_advanced_techniques__ps_6_advanced_techniques.md__error-1-command-not-found}

Cause: Typo in command name

Fix: Check spelling

\begin{lstlisting}[style=Alabaster, language=powershell]
# Wrong:
writ-output "hello"
# Correct:
Write-Output "hello"

\end{lstlisting}

\subparagraph*{Error 2: "Variable is null"}\label{docs__pandoc__latex__src__powershell_foundation__ps_6_advanced_techniques__ps_6_advanced_techniques.md__error-2-variable-is-null}

Cause: Variable was never assigned

Fix: Make sure variable is set before using it

\begin{lstlisting}[style=Alabaster, language=powershell]
$myvar = "hello"  # Set first
Write-Output $myvar  # Then use

\end{lstlisting}

\subparagraph*{Error 3: "Cannot find path"}\label{docs__pandoc__latex__src__powershell_foundation__ps_6_advanced_techniques__ps_6_advanced_techniques.md__error-3-cannot-find-path}

Cause: Wrong folder path

Fix: Verify path exists

\begin{lstlisting}[style=Alabaster, language=powershell]
# Check if path exists:
Test-Path -Path "C:\Users\YourName\Documents"
# If false, the path is wrong

\end{lstlisting}

\subparagraph*{Error 4: "Access denied"}\label{docs__pandoc__latex__src__powershell_foundation__ps_6_advanced_techniques__ps_6_advanced_techniques.md__error-4-access-denied}

Cause: Don\textquotesingle t have permission

Fix: Run PowerShell as administrator

\begin{itemize}
\tightlist
\item
  Right-click PowerShell -\textgreater{} Run as administrator
\end{itemize}

\paragraph*{Debugging Technique: Trace Output}\label{docs__pandoc__latex__src__powershell_foundation__ps_6_advanced_techniques__ps_6_advanced_techniques.md__debugging-technique-trace-output}

Add \texttt{Write-Output} statements to track what\textquotesingle s happening:

\begin{lstlisting}[style=Alabaster, language=powershell]
$path = "C:\Users\YourName\Documents"
Write-Output "Starting script. Path is: $path"
$files = ls -n $path
Write-Output "Found $($files.Count) files"
foreach ($file in $files) {
    Write-Output "Processing: $file"
    # Do something with $file
    Write-Output "Done with: $file"
}
Write-Output "Script complete"

\end{lstlisting}

Your screen reader will announce each step, so you know where errors happen.

\subsubsection*{Creating Professional Workflows}\label{docs__pandoc__latex__src__powershell_foundation__ps_6_advanced_techniques__ps_6_advanced_techniques.md__creating-professional-workflows}

\paragraph*{Example 1: Automated Project Setup}\label{docs__pandoc__latex__src__powershell_foundation__ps_6_advanced_techniques__ps_6_advanced_techniques.md__example-1-automated-project-setup}

Scenario: You start a new 3D printing project regularly. Instead of creating folders manually:

\begin{lstlisting}[style=Alabaster, language=powershell]
function SetupNewProject {
    param([string]$projectName)
    $baseFolder = "C:\Users\YourName\Documents\3D_Projects"
    $projectFolder = Join-Path -Path $baseFolder -ChildPath $projectName
    # Create folder structure
    mkdir $projectFolder -Force
    mkdir "$projectFolder\designs" -Force
    mkdir "$projectFolder\output" -Force
    mkdir "$projectFolder\notes" -Force
    # Create a README
    $readmeContent = @"
# $projectName
Created: $(Get-Date)
## Designs
All .scad files go here.
## Output
STL and other exports go here.
## Notes
Project notes and observations.
"@
    $readmeContent | Out-File -FilePath "$projectFolder\README.txt" -Encoding utf8
    Write-Output "Project setup complete: $projectFolder"
}
# Use it:
SetupNewProject -projectName "MyKeychain"
SetupNewProject -projectName "PhoneStand"

\end{lstlisting}

What it does:

\begin{itemize}
\tightlist
\item
  Creates folder structure for a new project
\item
  Sets up subfolders for designs, output, notes
\item
  Creates a README file automatically
\end{itemize}

\paragraph*{Example 2: Batch File Verification}\label{docs__pandoc__latex__src__powershell_foundation__ps_6_advanced_techniques__ps_6_advanced_techniques.md__example-2-batch-file-verification}

Scenario: Before processing, verify all required files exist:

\begin{lstlisting}[style=Alabaster, language=powershell]
function VerifyProjectFiles {
    param([string]$projectFolder)
    $requiredFiles = @(
        "README.txt",
        "designs",
        "output",
        "notes"
    )
    $allGood = $true
    foreach ($item in $requiredFiles) {
        $path = Join-Path -Path $projectFolder -ChildPath $item
        if (Test-Path -Path $path) {
            Write-Output " Found: $item"
        } else {
            Write-Output " Missing: $item"
            $allGood = $false
        }
    }
    if ($allGood) {
        Write-Output "All checks passed!"
        return $true
    } else {
        Write-Output "Some files are missing!"
        return $false
    }
}
# Use it:
$verified = VerifyProjectFiles -projectFolder "C:\Users\YourName\Documents\3D_Projects\MyKeychain"
if ($verified) {
    Write-Output "Safe to proceed with processing"
}

\end{lstlisting}

\subsubsection*{Screen Reader Tips for Scripts}\label{docs__pandoc__latex__src__powershell_foundation__ps_6_advanced_techniques__ps_6_advanced_techniques.md__screen-reader-tips-for-scripts}

\paragraph*{Making Script Output Readable}\label{docs__pandoc__latex__src__powershell_foundation__ps_6_advanced_techniques__ps_6_advanced_techniques.md__making-script-output-readable}

Problem: Script runs but output scrolls too fast or is hard to follow

Solution 1: Save to file

\begin{lstlisting}[style=Alabaster, language=powershell]
.\my-script.ps1 > output.txt
notepad.exe output.txt

\end{lstlisting}

Solution 2: Use Write-Output with clear sections

\begin{lstlisting}[style=Alabaster, language=powershell]
Write-Output "========== STARTING =========="
Write-Output ""
# ... script ...
Write-Output ""
Write-Output "========== COMPLETE =========="

\end{lstlisting}

Solution 3: Pause between major sections

\begin{lstlisting}[style=Alabaster, language=powershell]
Write-Output "Pausing... Press Enter to continue"
Read-Host

\end{lstlisting}

Your screen reader will announce the pause, give you time to read output.

\paragraph*{Announcing Progress}\label{docs__pandoc__latex__src__powershell_foundation__ps_6_advanced_techniques__ps_6_advanced_techniques.md__announcing-progress}

For long-running scripts:

\begin{lstlisting}[style=Alabaster, language=powershell]
$files = ls -n *.scad
$count = 0
foreach ($file in $files) {
    $count++
    Write-Output "Processing $count of $($files.Count): $file"
    # Do something with $file
}
Write-Output "All $count files processed!"

\end{lstlisting}

\subsubsection*{Practice Exercises}\label{docs__pandoc__latex__src__powershell_foundation__ps_6_advanced_techniques__ps_6_advanced_techniques.md__practice-exercises}

\paragraph*{Exercise 1: Your First Script}\label{docs__pandoc__latex__src__powershell_foundation__ps_6_advanced_techniques__ps_6_advanced_techniques.md__exercise-1-your-first-script}

Goal: Create and run a simple script

Steps:

\begin{enumerate}
\tightlist
\item
  Create file: \texttt{notepad.exe\ hello.ps1}
\item
  Type:

  \begin{lstlisting}[style=Alabaster, language=powershell]
  Write-Output "Hello from my first PowerShell script!"
  pwd
  ls -n

  \end{lstlisting}
\item
  Save and run: \texttt{.\textbackslash{}hello.ps1}
\end{enumerate}

Checkpoint: You should see output for each command.

\paragraph*{Exercise 2: Script with a Variable}\label{docs__pandoc__latex__src__powershell_foundation__ps_6_advanced_techniques__ps_6_advanced_techniques.md__exercise-2-script-with-a-variable}

Goal: Use a variable to make the script flexible

Steps:

\begin{enumerate}
\tightlist
\item
  Create file: \texttt{notepad.exe\ smart-listing.ps1}
\item
  Type:

  \begin{lstlisting}[style=Alabaster, language=powershell]
  $targetFolder = "C:\Users\YourName\Documents"
  Write-Output "Listing contents of: $targetFolder"
  ls -n $targetFolder

  \end{lstlisting}
\item
  Edit \texttt{\$targetFolder} to a real folder on your computer
\item
  Run: \texttt{.\textbackslash{}smart-listing.ps1}
\end{enumerate}

Checkpoint: You should see listing of that specific folder.

\paragraph*{Exercise 3: Function}\label{docs__pandoc__latex__src__powershell_foundation__ps_6_advanced_techniques__ps_6_advanced_techniques.md__exercise-3-function}

Goal: Create a reusable function

Steps:

\begin{enumerate}
\tightlist
\item
  Create file: \texttt{notepad.exe\ navigate.ps1}
\item
  Type:

  \begin{lstlisting}[style=Alabaster, language=powershell]
  function GoTo {
      param([string]$path)
      if (Test-Path -Path $path) {
          cd $path
          Write-Output "Now in: $(pwd)"
          Write-Output "Contents:"
          ls -n
      } else {
          Write-Output "Path does not exist: $path"
      }
  }
  # Test the function:
  GoTo -path "C:\Users\YourName\Documents"
  GoTo -path "C:\Users\YourName\Downloads"

  \end{lstlisting}
\item
  Run: \texttt{.\textbackslash{}navigate.ps1}
\end{enumerate}

Checkpoint: Both commands should work, showing contents of each folder.

\paragraph*{Exercise 4: Loop}\label{docs__pandoc__latex__src__powershell_foundation__ps_6_advanced_techniques__ps_6_advanced_techniques.md__exercise-4-loop}

Goal: Use a loop to repeat an action

Steps:

\begin{enumerate}
\tightlist
\item
  Create file: \texttt{notepad.exe\ repeat.ps1}
\item
  Type:

  \begin{lstlisting}[style=Alabaster, language=powershell]
  Write-Output "Demonstrating a loop:"
  for ($i = 1; $i -le 5; $i++) {
      Write-Output "Iteration $i: Hello!"
  }
  Write-Output "Loop complete!"

  \end{lstlisting}
\item
  Run: \texttt{.\textbackslash{}repeat.ps1}
\end{enumerate}

Checkpoint: Should print "Iteration 1" through "Iteration 5".

\paragraph*{Exercise 5: Real-World Script}\label{docs__pandoc__latex__src__powershell_foundation__ps_6_advanced_techniques__ps_6_advanced_techniques.md__exercise-5-real-world-script}

Goal: Create a useful script for a real task

Steps:

\begin{enumerate}
\tightlist
\item
  Create a folder: \texttt{mkdir\ C:\textbackslash{}Users\textbackslash{}YourName\textbackslash{}Documents\textbackslash{}TestFiles}
\item
  Create some test files:

  \begin{lstlisting}[style=Alabaster, language=powershell]
  echo "test" > C:\Users\YourName\Documents\TestFiles\file1.txt
  echo "test" > C:\Users\YourName\Documents\TestFiles\file2.txt
  echo "test" > C:\Users\YourName\Documents\TestFiles\file3.txt

  \end{lstlisting}
\item
  Create script: \texttt{notepad.exe\ report.ps1}
\item
  Type:

  \begin{lstlisting}[style=Alabaster, language=powershell]
  $folder = "C:\Users\YourName\Documents\TestFiles"
  $files = ls -n $folder
  Write-Output "=== FILE REPORT ==="
  Write-Output "Folder: $folder"
  Write-Output "Total files: $($files.Count)"
  Write-Output ""
  Write-Output "Files:"
  foreach ($file in $files) {
      Write-Output "  - $file"
  }
  Write-Output ""
  Write-Output "=== END REPORT ==="

  \end{lstlisting}
\item
  Run: \texttt{.\textbackslash{}report.ps1}
\end{enumerate}

Checkpoint: Should show report of all files in the test folder.

\subsubsection*{Quiz - Lesson PS-6}\label{docs__pandoc__latex__src__powershell_foundation__ps_6_advanced_techniques__ps_6_advanced_techniques.md__quiz---lesson-ps-6}

\begin{enumerate}
\tightlist
\item
  What is a PowerShell script?
\item
  What file extension do PowerShell scripts use?
\item
  What is a variable and how do you create one?
\item
  What is a function and why would you use one?
\item
  How do you run a script?
\item
  What is a loop and what does \texttt{foreach} do?
\item
  What does \texttt{Test-Path} do?
\item
  How do you handle errors in a script?
\item
  When would you use \texttt{Try-Catch}?
\item
  What technique makes script output readable for screen readers?
\end{enumerate}

\subsubsection*{Extension Problems}\label{docs__pandoc__latex__src__powershell_foundation__ps_6_advanced_techniques__ps_6_advanced_techniques.md__extension-problems}

\begin{enumerate}
\tightlist
\item
  Auto-Backup Script: Create a script that copies all files from one folder to another, announcing progress
\item
  File Counter: Write a function that counts files by extension (.txt, .scad, .stl, etc.)
\item
  Folder Cleaner: Script that deletes files older than 30 days (with user confirmation)
\item
  Project Template: Function that creates a complete project folder structure with all needed files
\item
  Batch Rename: Script that renames all files in a folder according to a pattern
\item
  Log Generator: Create a script that records what it does to a log file for later review
\item
  Scheduled Task: Set up a script to run automatically every day at a specific time
\item
  File Verifier: Check that all SCAD files in a folder have corresponding STL exports
\item
  Report Generator: Create a summary report of all projects in a folder
\item
  Error Tracker: Script that lists all commands that had errors in your recent history
\end{enumerate}

\subsubsection*{Important Notes}\label{docs__pandoc__latex__src__powershell_foundation__ps_6_advanced_techniques__ps_6_advanced_techniques.md__important-notes}

\begin{itemize}
\tightlist
\item
  Always test scripts on small sets of files first before running them on important data
\item
  Save your work regularly - use version control if possible
\item
  Test error handling - make sure errors don\textquotesingle t crash silently
\item
  Document your scripts - use comments so you remember what each part does
\item
  Backup before batch operations - if something goes wrong, you have the original
\end{itemize}

\subsubsection*{References}\label{docs__pandoc__latex__src__powershell_foundation__ps_6_advanced_techniques__ps_6_advanced_techniques.md__references}

\begin{itemize}
\tightlist
\item
  Microsoft PowerShell Scripting Guide: \url{https://docs.microsoft.com/powershell/scripting/}
\item
  Function Documentation: \url{https://docs.microsoft.com/powershell/module/microsoft.powershell.core/about/about_functions_advanced}
\item
  Error Handling: \url{https://docs.microsoft.com/powershell/module/microsoft.powershell.core/about/about_error_handling}
\item
  Loops: \url{https://docs.microsoft.com/powershell/module/microsoft.powershell.core/about/about_foreach}
\end{itemize}

Next Steps: After mastering this lesson, explore PowerShell modules, remoting, and 3D printing integration in the main curriculum.

\subsection{PowerShell Unit Test - Comprehensive Assessment}\label{docs__pandoc__latex__src__powershell_foundation__ps_unit_test__ps_unit_test.md__powershell_foundation_ps_unit_test-ps_unit_test}

Estimated time: 60-90 minutes

\subsubsection*{Key Learning Outcomes Assessed}\label{docs__pandoc__latex__src__powershell_foundation__ps_unit_test__ps_unit_test.md__key-learning-outcomes-assessed}

By completing this unit test, you will demonstrate:

\begin{enumerate}
\tightlist
\item
  Understanding of file system navigation and path concepts
\item
  Proficiency with file and folder manipulation commands
\item
  Ability to redirect and pipe command output
\item
  Knowledge of environment variables and aliases
\item
  Screen-reader accessibility best practices in terminal environments
\item
  Problem-solving and command chaining skills
\end{enumerate}

\subsubsection*{Target Audience:}\label{docs__pandoc__latex__src__powershell_foundation__ps_unit_test__ps_unit_test.md__target-audience}

Users who have completed PS-0 through PS-5 and need to demonstrate mastery of PowerShell fundamentals.

\subsubsection*{Instructions:}\label{docs__pandoc__latex__src__powershell_foundation__ps_unit_test__ps_unit_test.md__instructions}

Complete all sections below. For multiple choice, select the best answer. For short answers, write one to two sentences. For hands-on tasks, capture evidence (screenshots or output files) and submit alongside your answers.

\subsubsection*{Part A: Multiple Choice Questions (20 questions)}\label{docs__pandoc__latex__src__powershell_foundation__ps_unit_test__ps_unit_test.md__part-a-multiple-choice-questions-20-questions}

Select the best answer for each question. Each question is worth 1 point.

\begin{enumerate}
\item
  What is the primary purpose of the \texttt{PATH} environment variable?

  \begin{itemize}
  \tightlist
  \item
    A) Store your home directory location
  \item
    B) Tell the shell where to find executable programs
  \item
    C) Configure the visual appearance of the terminal
  \item
    D) Store the current working directory name
  \end{itemize}
\item
  Which command prints your current working directory?

  \begin{itemize}
  \tightlist
  \item
    A) \texttt{ls\ -n}
  \item
    B) \texttt{cd}
  \item
    C) \texttt{pwd}
  \item
    D) \texttt{whoami}
  \end{itemize}
\item
  What does the \texttt{\textasciitilde{}} symbol represent in PowerShell paths?

  \begin{itemize}
  \tightlist
  \item
    A) The root directory
  \item
    B) The current directory
  \item
    C) The parent directory
  \item
    D) The home directory
  \end{itemize}
\item
  How do you list only file names (not full details) in a way that is screen-reader friendly?

  \begin{itemize}
  \tightlist
  \item
    A) \texttt{ls}
  \item
    B) \texttt{ls\ -n}
  \item
    C) \texttt{ls\ -l}
  \item
    D) \texttt{cat\ -n}
  \end{itemize}
\item
  Which command creates a new empty file?

  \begin{itemize}
  \tightlist
  \item
    A) \texttt{mkdir\ filename}
  \item
    B) \texttt{ni\ filename}
  \item
    C) \texttt{touch\ filename}
  \item
    D) \texttt{echo\ filename}
  \end{itemize}
\item
  What is the difference between \texttt{\textgreater{}} and \texttt{\textgreater{}\textgreater{}}?

  \begin{itemize}
  \tightlist
  \item
    A) \texttt{\textgreater{}} redirects to file, \texttt{\textgreater{}\textgreater{}} displays on screen
  \item
    B) \texttt{\textgreater{}} overwrites a file, \texttt{\textgreater{}\textgreater{}} appends to a file
  \item
    C) They do the same thing
  \item
    D) \texttt{\textgreater{}} is for text, \texttt{\textgreater{}\textgreater{}} is for binary
  \end{itemize}
\item
  What does the pipe operator \texttt{\textbar{}} do?

  \begin{itemize}
  \tightlist
  \item
    A) Creates a folder
  \item
    B) Sends the output of one command to the input of another
  \item
    C) Deletes files matching a pattern
  \item
    D) Lists all processes
  \end{itemize}
\item
  Which command copies a file?

  \begin{itemize}
  \tightlist
  \item
    A) \texttt{mv}
  \item
    B) \texttt{rm}
  \item
    C) \texttt{cp}
  \item
    D) \texttt{cd}
  \end{itemize}
\item
  How do you rename a file from \texttt{oldname.txt} to \texttt{newname.txt}?

  \begin{itemize}
  \tightlist
  \item
    A) \texttt{cp\ oldname.txt\ newname.txt}
  \item
    B) \texttt{mv\ oldname.txt\ newname.txt}
  \item
    C) \texttt{rename\ oldname.txt\ newname.txt}
  \item
    D) \texttt{rn\ oldname.txt\ newname.txt}
  \end{itemize}
\item
  What is the purpose of \texttt{Select-String}?

  \begin{itemize}
  \tightlist
  \item
    A) Select files in a directory
  \item
    B) Search for text patterns within a file
  \item
    C) Select a string to copy to clipboard
  \item
    D) Select which shell to use
  \end{itemize}
\item
  Which key combination allows you to autocomplete a path in PowerShell?

  \begin{itemize}
  \tightlist
  \item
    A) \texttt{Ctrl\ +\ A}
  \item
    B) \texttt{Ctrl\ +\ E}
  \item
    C) \texttt{Tab}
  \item
    D) \texttt{Space}
  \end{itemize}
\item
  How do you copy text to the Windows clipboard from PowerShell?

  \begin{itemize}
  \tightlist
  \item
    A) \texttt{cat\ filename\ \textgreater{}\ clipboard}
  \item
    B) \texttt{cat\ filename\ \textbar{}\ clip}
  \item
    C) \texttt{copy\ filename}
  \item
    D) \texttt{cat\ filename\ \textbar{}\ paste}
  \end{itemize}
\item
  What does \texttt{Get-Command\ openscad} do?

  \begin{itemize}
  \tightlist
  \item
    A) Opens the OpenSCAD application
  \item
    B) Gets help about the OpenSCAD command
  \item
    C) Locates the full path of the openscad executable
  \item
    D) Lists all available commands
  \end{itemize}
\item
  Which wildcard matches any single character?

  \begin{itemize}
  \tightlist
  \item
    A) \texttt{*}
  \item
    B) \texttt{?}
  \item
    C) \texttt{\%}
  \item
    D) \texttt{\#}
  \end{itemize}
\item
  What is the purpose of the \texttt{\&} call operator?

  \begin{itemize}
  \tightlist
  \item
    A) Run a script or executable by full path
  \item
    B) Execute all commands in parallel
  \item
    C) Combine multiple commands
  \item
    D) Create an alias
  \end{itemize}
\item
  How do you create a temporary alias for a command?

  \begin{itemize}
  \tightlist
  \item
    A) \texttt{alias\ preview=\textquotesingle{}openscad\textquotesingle{}}
  \item
    B) \texttt{Set-Alias\ -Name\ preview\ -Value\ openscad}
  \item
    C) \texttt{New-Alias\ preview\ openscad}
  \item
    D) \texttt{Alias\ preview\ =\ openscad}
  \end{itemize}
\item
  Where is your PowerShell profile typically stored?

  \begin{itemize}
  \tightlist
  \item
    A) C:\textbackslash Program Files\textbackslash PowerShell\textbackslash profile.ps1
  \item
    B) The location returned by \texttt{echo\ \$PROFILE}
  \item
    C) \textasciitilde{}/PowerShell/profile.ps1
  \item
    D) \textasciitilde{}/.bashrc
  \end{itemize}
\item
  How do you abort a long-running command in PowerShell?

  \begin{itemize}
  \tightlist
  \item
    A) Press \texttt{Escape}
  \item
    B) Press \texttt{Ctrl\ +\ X}
  \item
    C) Press \texttt{Ctrl\ +\ C}
  \item
    D) Press \texttt{Alt\ +\ F4}
  \end{itemize}
\item
  What command shows the history of previously run commands?

  \begin{itemize}
  \tightlist
  \item
    A) \texttt{history}
  \item
    B) \texttt{Get-History}
  \item
    C) \texttt{Show-History}
  \item
    D) Both A and B
  \end{itemize}
\item
  How do you permanently set an alias so it persists across PowerShell sessions?

  \begin{itemize}
  \tightlist
  \item
    A) Use \texttt{Set-Alias} in the terminal every time
  \item
    B) Add the \texttt{Set-Alias} line to your PowerShell profile
  \item
    C) Use the Windows Control Panel
  \item
    D) Aliases cannot be made permanent
  \end{itemize}
\end{enumerate}

\subsubsection*{Part B: Short Answer Questions (10 questions)}\label{docs__pandoc__latex__src__powershell_foundation__ps_unit_test__ps_unit_test.md__part-b-short-answer-questions-10-questions}

Answer each question in one to two sentences. Each question is worth 2 points.

\begin{enumerate}
\item
  Explain the difference between absolute and relative paths. Give one example of each.
\item
  Why is \texttt{ls\ -n} preferred over \texttt{ls} for screen reader users? Describe what flag you would use to list only files.
\item
  What is the purpose of redirecting output to a file, and give an example of when you would use \texttt{\textgreater{}} instead of \texttt{\textgreater{}\textgreater{}}?
\item
  Describe what would happen if you ran \texttt{rm\ -r\ \textasciitilde{}/Documents/my\_folder} and why this command should be used carefully.
\item
  How would you search for all files with a \texttt{.scad} extension in your current directory? Write the command.
\item
  Explain what happens when you pipe the output of \texttt{ls\ -n} into \texttt{clip}. What would you do next?
\item
  What is an environment variable, and give one example of how you might use it in PowerShell.
\item
  If a program is not in your \texttt{PATH}, what two methods could you use to run it from PowerShell?
\item
  Describe how you would open a file in Notepad and also add a line to it from PowerShell.
\item
  What is one strategy you would use if your screen reader stops announcing terminal output while using PowerShell?
\end{enumerate}

\subsubsection*{Part C: Hands-On Tasks (10 tasks)}\label{docs__pandoc__latex__src__powershell_foundation__ps_unit_test__ps_unit_test.md__part-c-hands-on-tasks-10-tasks}

Complete each task and capture evidence (screenshots, output files, or command transcripts). Each task is worth 3 points.

\paragraph*{Tasks 1-5: File System and Navigation}\label{docs__pandoc__latex__src__powershell_foundation__ps_unit_test__ps_unit_test.md__tasks-1-5-file-system-and-navigation}

\begin{enumerate}
\item
  Create a folder structure \texttt{\textasciitilde{}/Documents/PowerShell\_Assessment/Projects} using a single command. Capture the \texttt{ls\ -n} output showing the creation.
\item
  Create five files named \texttt{project\_1.scad}, \texttt{project\_2.scad}, \texttt{project\_3.txt}, \texttt{notes\_1.txt}, and \texttt{notes\_2.txt} inside the \texttt{Projects} folder. Use wildcards to list only \texttt{.scad} files, then capture the output.
\item
  Copy the entire \texttt{Projects} folder to \texttt{Projects\_Backup} using \texttt{cp\ -r}. Capture the \texttt{ls\ -n} output showing both folders exist.
\item
  Move (rename) \texttt{project\_1.scad} to \texttt{project\_1\_final.scad}. Capture the \texttt{ls\ -n} output showing the renamed file.
\item
  Delete \texttt{notes\_1.txt} and \texttt{notes\_2.txt} using a single \texttt{rm} command with wildcards. Capture the final \texttt{ls\ -n} output.
\end{enumerate}

\paragraph*{Tasks 6-10: Advanced Operations and Scripting}\label{docs__pandoc__latex__src__powershell_foundation__ps_unit_test__ps_unit_test.md__tasks-6-10-advanced-operations-and-scripting}

\begin{enumerate}
\setcounter{enumi}{5}
\item
  Create a file called \texttt{my\_data.txt} with at least four lines using \texttt{echo} and \texttt{\textgreater{}\textgreater{}}. Then read it with \texttt{cat\ my\_data.txt} and capture the output.
\item
  Use \texttt{Select-String} to search for a keyword (e.g., "project") in \texttt{my\_data.txt} and pipe the results to \texttt{clip}. Paste the results into Notepad and capture a screenshot.
\item
  List all files in the \texttt{Projects} folder and redirect the output to \texttt{projects\_list.txt}. Open it in Notepad and capture a screenshot of the file.
\item
  Create a temporary alias called \texttt{myls} that runs \texttt{ls\ -n}, test it, and capture the output. Then explain what would be required to make it permanent.
\item
  Run \texttt{Get-Help\ Get-ChildItem} and redirect the output to \texttt{help\_output.txt}. Open the file in Notepad and capture a screenshot showing at least the first page of help content.
\end{enumerate}

\subsubsection*{Grading Rubric}\label{docs__pandoc__latex__src__powershell_foundation__ps_unit_test__ps_unit_test.md__grading-rubric}

{\def\LTcaptype{none} % do not increment counter
\begin{longtable}[]{@{}llll@{}}
\toprule\noalign{}
Section & Questions & Points Each & Total \\
\midrule\noalign{}
\endhead
\bottomrule\noalign{}
\endlastfoot
Multiple Choice & 20 & 1 & 20 \\
Short Answer & 10 & 2 & 20 \\
Hands-On Tasks & 10 & 3 & 30 \\
Total & 40 & - & 70 \\
\end{longtable}
}

Passing Score: 49 points (70\%)

\subsubsection*{Helpful Resources for Review}\label{docs__pandoc__latex__src__powershell_foundation__ps_unit_test__ps_unit_test.md__helpful-resources-for-review}

\begin{itemize}
\tightlist
\item
  \href{https://learn.microsoft.com/powershell/scripting/overview}{PowerShell Command Reference}
\item
  \href{https://learn.microsoft.com/powershell/scripting/learn/shell/navigate-the-filesystem}{Navigation and File System}
\item
  \href{https://learn.microsoft.com/powershell/scripting/learn/shell/using-the-pipeline}{Using Pipes and Filtering}
\item
  \href{https://learn.microsoft.com/powershell/module/microsoft.powershell.core/about/about_profiles}{Profile and Aliases}
\item
  \href{https://learn.microsoft.com/powershell/scripting/windows-powershell/ise/accessibility-in-windows-powershell-ise}{Screen Reader Accessibility Tips}
\end{itemize}

\subsubsection*{Submission Checklist}\label{docs__pandoc__latex__src__powershell_foundation__ps_unit_test__ps_unit_test.md__submission-checklist}

\begin{itemize}
\tightlist
\item[$\square$]
  All 20 multiple choice questions answered
\item[$\square$]
  All 10 short answer questions answered (1-2 sentences each)
\item[$\square$]
  All 10 hands-on tasks completed with evidence captured
\item[$\square$]
  Files/screenshots organized and labeled clearly
\item[$\square$]
  Submission includes this checklist
\end{itemize}

\subsection{Screen Reader Accessibility Guide for PowerShell}\label{docs__pandoc__latex__src__powershell_foundation__screen_reader_accessibility_guide__screen_reader_accessibility_guide.md__screen-reader-accessibility-guide-for-powershell}

Target Users: NVDA, JAWS, and other screen reader users\\
Last Updated: 2026

This guide supports the PowerShell Foundation curriculum and helps screen reader users navigate and work efficiently with PowerShell on Windows.

\subsubsection*{Table of Contents}\label{docs__pandoc__latex__src__powershell_foundation__screen_reader_accessibility_guide__screen_reader_accessibility_guide.md__table-of-contents}

\begin{enumerate}
\tightlist
\item
  \hyperref[docs__pandoc__latex__src__powershell_foundation__screen_reader_accessibility_guide__screen_reader_accessibility_guide.md__getting-started]{Getting Started with Screen Readers}
\item
  \hyperref[docs__pandoc__latex__src__powershell_foundation__screen_reader_accessibility_guide__screen_reader_accessibility_guide.md__nvda-tips]{NVDA-Specific Tips}
\item
  \hyperref[docs__pandoc__latex__src__powershell_foundation__screen_reader_accessibility_guide__screen_reader_accessibility_guide.md__jaws-tips]{JAWS-Specific Tips}
\item
  \hyperref[docs__pandoc__latex__src__powershell_foundation__screen_reader_accessibility_guide__screen_reader_accessibility_guide.md__general-terminal]{General Terminal Accessibility}
\item
  \hyperref[docs__pandoc__latex__src__powershell_foundation__screen_reader_accessibility_guide__screen_reader_accessibility_guide.md__long-output]{Working with Long Output}
\item
  \hyperref[docs__pandoc__latex__src__powershell_foundation__screen_reader_accessibility_guide__screen_reader_accessibility_guide.md__shortcuts]{Keyboard Shortcuts Reference}
\item
  \hyperref[docs__pandoc__latex__src__powershell_foundation__screen_reader_accessibility_guide__screen_reader_accessibility_guide.md__troubleshooting]{Troubleshooting}
\end{enumerate}

\subsubsection*{Getting Started with Screen Readers}\label{docs__pandoc__latex__src__powershell_foundation__screen_reader_accessibility_guide__screen_reader_accessibility_guide.md__getting-started}

\paragraph*{Which Screen Reader Should I Use?}\label{docs__pandoc__latex__src__powershell_foundation__screen_reader_accessibility_guide__screen_reader_accessibility_guide.md__which-screen-reader-should-i-use}

NVDA is free and often recommended for new users; JAWS is a powerful commercial option. Also consider Dolphin SuperNova and Windows Narrator:

\begin{itemize}
\tightlist
\item
  Dolphin SuperNova: commercial speech, braille and magnification (check vendor docs for keyboard mappings).
\item
  Windows Narrator: built into Windows and useful for quick access without installing third-party software.
\end{itemize}

\paragraph*{Before You Start}\label{docs__pandoc__latex__src__powershell_foundation__screen_reader_accessibility_guide__screen_reader_accessibility_guide.md__before-you-start}

\begin{enumerate}
\tightlist
\item
  Start your screen reader before opening PowerShell.
\item
  Open PowerShell and listen for the window title and prompt.
\item
  If silent, press Alt+Tab to find the window.
\end{enumerate}

\paragraph*{What is PowerShell?}\label{docs__pandoc__latex__src__powershell_foundation__screen_reader_accessibility_guide__screen_reader_accessibility_guide.md__what-is-powershell}

PowerShell is the modern Windows shell and scripting environment. Common commands include \texttt{Get-ChildItem}, \texttt{Get-Content}, and \texttt{Out-File}. PowerShell provides richer objects and piping than CMD.

\subsubsection*{NVDA-Specific Tips}\label{docs__pandoc__latex__src__powershell_foundation__screen_reader_accessibility_guide__screen_reader_accessibility_guide.md__nvda-tips}

NVDA is available from \url{https://www.nvaccess.org/}

\paragraph*{Dolphin SuperNova}\label{docs__pandoc__latex__src__powershell_foundation__screen_reader_accessibility_guide__screen_reader_accessibility_guide.md__dolphin-supernova}

Dolphin SuperNova: \url{https://yourdolphin.com/supernova/} --- commercial option providing speech and magnification; consult vendor guides for features and commands.

\paragraph*{Windows Narrator}\label{docs__pandoc__latex__src__powershell_foundation__screen_reader_accessibility_guide__screen_reader_accessibility_guide.md__windows-narrator}

Windows Narrator: \url{https://support.microsoft.com/narrator} --- built-in Narrator has a different set of commands; it can be enabled via Windows Settings \textgreater{} Accessibility.

\paragraph*{Key Commands for PowerShell}\label{docs__pandoc__latex__src__powershell_foundation__screen_reader_accessibility_guide__screen_reader_accessibility_guide.md__key-commands-for-powershell}

{\def\LTcaptype{none} % do not increment counter
\begin{longtable}[]{@{}
  >{\raggedright\arraybackslash}p{(\linewidth - 2\tabcolsep) * \real{0.2361}}
  >{\raggedright\arraybackslash}p{(\linewidth - 2\tabcolsep) * \real{0.7639}}@{}}
\toprule\noalign{}
\begin{minipage}[b]{\linewidth}\raggedright
Command
\end{minipage} & \begin{minipage}[b]{\linewidth}\raggedright
What It Does
\end{minipage} \\
\midrule\noalign{}
\endhead
\bottomrule\noalign{}
\endlastfoot
NVDA+Home & Read the current line (your command or output) \\
NVDA+Down Arrow & Read from cursor to end of screen \\
NVDA+Up Arrow & Read from top to cursor \\
NVDA+Page Down & Read next page \\
NVDA+Page Up & Read previous page \\
NVDA+F7 & Open the Review Mode viewer (can scroll through text) \\
\end{longtable}
}

\paragraph*{Example: Reading Long Output}\label{docs__pandoc__latex__src__powershell_foundation__screen_reader_accessibility_guide__screen_reader_accessibility_guide.md__example-reading-long-output}

If \texttt{Get-ChildItem} produces many lines, redirect to a file and open it in Notepad:

\begin{lstlisting}[style=Alabaster, language=powershell]
Get-ChildItem -Name > list.txt
notepad list.txt

\end{lstlisting}

\texttt{-Name} prints one item per line (screen reader friendly).

\subsubsection*{JAWS-Specific Tips}\label{docs__pandoc__latex__src__powershell_foundation__screen_reader_accessibility_guide__screen_reader_accessibility_guide.md__jaws-tips}

JAWS is available from \url{https://www.freedomscientific.com/}

\paragraph*{Key Commands for PowerShell}\label{docs__pandoc__latex__src__powershell_foundation__screen_reader_accessibility_guide__screen_reader_accessibility_guide.md__key-commands-for-powershell-1}

{\def\LTcaptype{none} % do not increment counter
\begin{longtable}[]{@{}ll@{}}
\toprule\noalign{}
Command & What It Does \\
\midrule\noalign{}
\endhead
\bottomrule\noalign{}
\endlastfoot
Insert+Down Arrow & Read line by line downward \\
Insert+Up Arrow & Read line by line upward \\
Insert+Page Down & Read next page of text \\
Insert+Page Up & Read previous page of text \\
\end{longtable}
}

\paragraph*{Example: Reading Long Output}\label{docs__pandoc__latex__src__powershell_foundation__screen_reader_accessibility_guide__screen_reader_accessibility_guide.md__example-reading-long-output-1}

\begin{enumerate}
\tightlist
\item
  Redirect: \texttt{Get-ChildItem\ -Name\ \textgreater{}\ list.txt}
\item
  Open Notepad: \texttt{notepad\ list.txt}
\item
  Use Insert+Ctrl+Down to read full contents.
\end{enumerate}

\subsubsection*{General Terminal Accessibility}\label{docs__pandoc__latex__src__powershell_foundation__screen_reader_accessibility_guide__screen_reader_accessibility_guide.md__general-terminal}

\paragraph*{Understanding the PowerShell Layout}\label{docs__pandoc__latex__src__powershell_foundation__screen_reader_accessibility_guide__screen_reader_accessibility_guide.md__understanding-the-powershell-layout}

PowerShell shows a title bar, content area, and a prompt that looks like:

\begin{lstlisting}[style=Alabaster, language=powershell]
PS C:\Users\YourName>

\end{lstlisting}

\paragraph*{Navigation Sequence}\label{docs__pandoc__latex__src__powershell_foundation__screen_reader_accessibility_guide__screen_reader_accessibility_guide.md__navigation-sequence}

\begin{enumerate}
\tightlist
\item
  Screen reader announces the title
\item
  Then it announces the prompt line
\item
  Anything above prompt is prior output
\end{enumerate}

\subsubsection*{Working with Long Output}\label{docs__pandoc__latex__src__powershell_foundation__screen_reader_accessibility_guide__screen_reader_accessibility_guide.md__long-output}

\paragraph*{Solution 1: Redirect to a File}\label{docs__pandoc__latex__src__powershell_foundation__screen_reader_accessibility_guide__screen_reader_accessibility_guide.md__solution-1-redirect-to-a-file}

\begin{lstlisting}[style=Alabaster, language=powershell]
Get-ChildItem -Name > list.txt
notepad list.txt

\end{lstlisting}

\paragraph*{Solution 2: Use Pagination}\label{docs__pandoc__latex__src__powershell_foundation__screen_reader_accessibility_guide__screen_reader_accessibility_guide.md__solution-2-use-pagination}

\begin{lstlisting}[style=Alabaster, language=powershell]
Get-Content largefile.txt | more

\end{lstlisting}

\paragraph*{Solution 3: Filter Output}\label{docs__pandoc__latex__src__powershell_foundation__screen_reader_accessibility_guide__screen_reader_accessibility_guide.md__solution-3-filter-output}

\begin{lstlisting}[style=Alabaster, language=powershell]
Get-ChildItem -Name | Where-Object { $_ -like "*.scad" }

\end{lstlisting}

\paragraph*{Solution 4: Count Before Displaying}\label{docs__pandoc__latex__src__powershell_foundation__screen_reader_accessibility_guide__screen_reader_accessibility_guide.md__solution-4-count-before-displaying}

\begin{lstlisting}[style=Alabaster, language=powershell]
(Get-ChildItem -Name).Count

\end{lstlisting}

\subsubsection*{Keyboard Shortcuts Reference}\label{docs__pandoc__latex__src__powershell_foundation__screen_reader_accessibility_guide__screen_reader_accessibility_guide.md__shortcuts}

{\def\LTcaptype{none} % do not increment counter
\begin{longtable}[]{@{}ll@{}}
\toprule\noalign{}
Key & Action \\
\midrule\noalign{}
\endhead
\bottomrule\noalign{}
\endlastfoot
Up Arrow & Show previous command \\
Down Arrow & Show next command \\
Tab & Auto-complete file/folder names \\
Home & Jump to start of line \\
End & Jump to end of line \\
Ctrl+C & Stop command \\
Enter & Run command \\
\end{longtable}
}

\subsubsection*{Troubleshooting}\label{docs__pandoc__latex__src__powershell_foundation__screen_reader_accessibility_guide__screen_reader_accessibility_guide.md__troubleshooting}

\paragraph*{Problem: "I Can\textquotesingle t Hear the Output"}\label{docs__pandoc__latex__src__powershell_foundation__screen_reader_accessibility_guide__screen_reader_accessibility_guide.md__problem-i-cant-hear-the-output}

\begin{enumerate}
\tightlist
\item
  Redirect to file and open in Notepad.
\item
  Use End to jump to the end of text.
\end{enumerate}

\paragraph*{Problem: "Tab Completion Isn\textquotesingle t Working"}\label{docs__pandoc__latex__src__powershell_foundation__screen_reader_accessibility_guide__screen_reader_accessibility_guide.md__problem-tab-completion-isnt-working}

\begin{enumerate}
\tightlist
\item
  Type some characters before Tab.
\end{enumerate}

\paragraph*{Problem: "Command Not Found"}\label{docs__pandoc__latex__src__powershell_foundation__screen_reader_accessibility_guide__screen_reader_accessibility_guide.md__problem-command-not-found}

\begin{enumerate}
\tightlist
\item
  Use \texttt{Get-Command\ programname} to check availability.
\end{enumerate}

\subsubsection*{Pro Tips}\label{docs__pandoc__latex__src__powershell_foundation__screen_reader_accessibility_guide__screen_reader_accessibility_guide.md__pro-tips}

\begin{enumerate}
\tightlist
\item
  Use \texttt{Get-ChildItem\ -Name} for one-per-line listings.
\item
  Use \texttt{Out-File\ -FilePath\ list.txt} to capture output with encoding options.
\end{enumerate}

\subsubsection*{Recommended Workflow}\label{docs__pandoc__latex__src__powershell_foundation__screen_reader_accessibility_guide__screen_reader_accessibility_guide.md__recommended-workflow}

\begin{enumerate}
\tightlist
\item
  \texttt{Set-Location} (or \texttt{cd}) to the project folder
\item
  \texttt{Get-ChildItem\ -Name} to list files
\item
  Redirect large output to files and open in Notepad
\end{enumerate}

\subsubsection*{Additional Resources}\label{docs__pandoc__latex__src__powershell_foundation__screen_reader_accessibility_guide__screen_reader_accessibility_guide.md__additional-resources}

\begin{itemize}
\tightlist
\item
  NVDA Documentation: \url{https://www.nvaccess.org/documentation/}
\item
  JAWS Documentation: \url{https://www.freedomscientific.com/support/}
\item
  PowerShell Documentation: \url{https://docs.microsoft.com/powershell}
\item
  NVDA Documentation: \url{https://www.nvaccess.org/documentation/}
\item
  JAWS Documentation: \url{https://www.freedomscientific.com/support/}
\item
  Dolphin SuperNova: \url{https://yourdolphin.com/supernova/}
\item
  Windows Narrator: \url{https://support.microsoft.com/narrator}
\end{itemize}

\section{Windows Command Line}\label{docs__pandoc__latex__src__cmd_foundation__part_1.md__cmd_foundation-part_1}

This section covers terminal fundamentals, screen reader accessibility, and command-line basics needed before diving into 3D design with OpenSCAD. This is an alternate pathway to the PowerShell curriculum, using Windows Command Prompt (CMD) instead.

Time commitment: \textasciitilde{}10 hours\\
Skills gained: Terminal navigation, file operations, basic scripting, keyboard-only workflow mastery

Note

CMD is simpler than PowerShell but has fewer advanced features. Both pathways teach the same fundamental concepts.

\subsection{Windows Command Prompt (CMD) for Screen Reader Users - Complete Curriculum Overview}\label{docs__pandoc__latex__src__cmd_foundation__cmd_curriculum_overview__cmd_curriculum_overview.md__cmd_foundation_cmd_curriculum_overview-cmd_curriculum_overview}

\textbf{Welcome!} This curriculum teaches you how to use Windows Command Prompt (CMD) as a screen reader user, starting from zero experience and building to practical skill levels.

\textbf{Last Updated:} February 2026\\
\textbf{Total Duration:} 30-45 hours of instruction and practice (for screen reader users)\\
\textbf{Target Users:} Screen reader users --- NVDA, JAWS, Windows Narrator, and Dolphin SuperNova are all covered throughout this curriculum.

\emph{Note: Time estimates reflect the additional time needed for screen reader navigation, text-to-speech processing, and careful keyboard-based workflows.}\\
\textbf{Alternate to:} PowerShell curriculum (same concepts, different command syntax)

\begin{center}\rule{0.5\linewidth}{0.5pt}\end{center}

\subsubsection*{Why Learn Windows Command Prompt (CMD)?}\label{docs__pandoc__latex__src__cmd_foundation__cmd_curriculum_overview__cmd_curriculum_overview.md__why-learn-windows-command-prompt-cmd}

\paragraph*{For Everyone}\label{docs__pandoc__latex__src__cmd_foundation__cmd_curriculum_overview__cmd_curriculum_overview.md__for-everyone}

\begin{itemize}
\tightlist
\item
  \textbf{Speed:} Text commands are often faster than clicking through menus.
\item
  \textbf{Simplicity:} CMD has fewer commands than PowerShell, but they are straightforward.
\item
  \textbf{Accessibility:} CMD works reliably with all major screen readers.
\item
  \textbf{Precision:} Exact control over what your computer does.
\end{itemize}

\paragraph*{For 3D Printing (Our Focus)}\label{docs__pandoc__latex__src__cmd_foundation__cmd_curriculum_overview__cmd_curriculum_overview.md__for-3d-printing-our-focus}

\begin{itemize}
\tightlist
\item
  \textbf{Batch Operations:} Process multiple 3D models at once.
\item
  \textbf{Accessibility:} Many 3D design tools are scriptable from CMD.
\item
  \textbf{Reproducibility:} Same settings, every time.
\item
  \textbf{Integration:} Connect OpenSCAD and other tools together.
\end{itemize}

\paragraph*{For Screen Reader Users Specifically}\label{docs__pandoc__latex__src__cmd_foundation__cmd_curriculum_overview__cmd_curriculum_overview.md__for-screen-reader-users-specifically}

\begin{itemize}
\tightlist
\item
  \textbf{Great Accessibility:} CMD works well with NVDA, JAWS, Windows Narrator, and Dolphin SuperNova.
\item
  \textbf{No Mouse Needed:} Everything is keyboard-based.
\item
  \textbf{Text-Based:} Output is naturally readable by screen readers.
\item
  \textbf{Stability:} Unlike GUIs, terminal interactions are consistent.
\item
  \textbf{Simpler Syntax:} CMD commands are more straightforward than PowerShell for beginners.
\end{itemize}

\begin{center}\rule{0.5\linewidth}{0.5pt}\end{center}

\subsubsection*{Curriculum Structure}\label{docs__pandoc__latex__src__cmd_foundation__cmd_curriculum_overview__cmd_curriculum_overview.md__curriculum-structure}

\paragraph*{Phase 1: Absolute Beginner → Comfortable User}\label{docs__pandoc__latex__src__cmd_foundation__cmd_curriculum_overview__cmd_curriculum_overview.md__phase-1-absolute-beginner--comfortable-user}

{\def\LTcaptype{none} % do not increment counter
\begin{longtable}[]{@{}
  >{\raggedright\arraybackslash}p{(\linewidth - 4\tabcolsep) * \real{0.3617}}
  >{\raggedright\arraybackslash}p{(\linewidth - 4\tabcolsep) * \real{0.1383}}
  >{\raggedright\arraybackslash}p{(\linewidth - 4\tabcolsep) * \real{0.5000}}@{}}
\toprule\noalign{}
\begin{minipage}[b]{\linewidth}\raggedright
Lesson
\end{minipage} & \begin{minipage}[b]{\linewidth}\raggedright
Duration
\end{minipage} & \begin{minipage}[b]{\linewidth}\raggedright
What You Will Learn
\end{minipage} \\
\midrule\noalign{}
\endhead
\bottomrule\noalign{}
\endlastfoot
\textbf{CMD-Pre: Your First Terminal} & 1.5-2 hours &
Opening CMD, first commands, basic navigation \\
\textbf{CMD-0: Getting Started} & 1.5 hours &
Paths, shortcuts, command basics \\
\textbf{CMD-1: Navigation} & 2-2.5 hours &
Moving around the file system confidently \\
\end{longtable}
}

\textbf{Goal:} You can navigate to any folder and see what is in it with your screen reader.

\paragraph*{Phase 2: Intermediate User → Power User}\label{docs__pandoc__latex__src__cmd_foundation__cmd_curriculum_overview__cmd_curriculum_overview.md__phase-2-intermediate-user--power-user}

{\def\LTcaptype{none} % do not increment counter
\begin{longtable}[]{@{}
  >{\raggedright\arraybackslash}p{(\linewidth - 4\tabcolsep) * \real{0.4554}}
  >{\raggedright\arraybackslash}p{(\linewidth - 4\tabcolsep) * \real{0.1287}}
  >{\raggedright\arraybackslash}p{(\linewidth - 4\tabcolsep) * \real{0.4158}}@{}}
\toprule\noalign{}
\begin{minipage}[b]{\linewidth}\raggedright
Lesson
\end{minipage} & \begin{minipage}[b]{\linewidth}\raggedright
Duration
\end{minipage} & \begin{minipage}[b]{\linewidth}\raggedright
What You Will Learn
\end{minipage} \\
\midrule\noalign{}
\endhead
\bottomrule\noalign{}
\endlastfoot
\textbf{CMD-2: File \& Folder Manipulation} & 2.5-3 hours &
Create, copy, move, delete files/folders \\
\textbf{CMD-3: Input, Output \& Redirection} & 2.5-3 hours &
Redirect output, pipe commands \\
\textbf{CMD-4: Environment Variables \& Shortcuts} & 2-2.5 hours &
Automate settings, create shortcuts \\
\textbf{CMD-5: Filling in the Gaps} & 2-2.5 hours &
Batch files, history, debugging \\
\end{longtable}
}

\textbf{Goal:} You can create folders, manage files, and combine commands to accomplish complex tasks.

\paragraph*{Phase 3: Professional Skills}\label{docs__pandoc__latex__src__cmd_foundation__cmd_curriculum_overview__cmd_curriculum_overview.md__phase-3-professional-skills}

{\def\LTcaptype{none} % do not increment counter
\begin{longtable}[]{@{}
  >{\raggedright\arraybackslash}p{(\linewidth - 4\tabcolsep) * \real{0.3765}}
  >{\raggedright\arraybackslash}p{(\linewidth - 4\tabcolsep) * \real{0.1529}}
  >{\raggedright\arraybackslash}p{(\linewidth - 4\tabcolsep) * \real{0.4706}}@{}}
\toprule\noalign{}
\begin{minipage}[b]{\linewidth}\raggedright
Lesson
\end{minipage} & \begin{minipage}[b]{\linewidth}\raggedright
Duration
\end{minipage} & \begin{minipage}[b]{\linewidth}\raggedright
What You Will Learn
\end{minipage} \\
\midrule\noalign{}
\endhead
\bottomrule\noalign{}
\endlastfoot
\textbf{CMD-6: Advanced Techniques} & 2.5-3 hours &
Scripting, loops, automation workflows \\
\textbf{CMD Unit Test} & 2.5-4 hours & Comprehensive assessment \\
\end{longtable}
}

\begin{center}\rule{0.5\linewidth}{0.5pt}\end{center}

\subsubsection*{How to Use This Curriculum}\label{docs__pandoc__latex__src__cmd_foundation__cmd_curriculum_overview__cmd_curriculum_overview.md__how-to-use-this-curriculum}

\paragraph*{If You Have Never Used a Terminal Before}\label{docs__pandoc__latex__src__cmd_foundation__cmd_curriculum_overview__cmd_curriculum_overview.md__if-you-have-never-used-a-terminal-before}

\textbf{Start here and go in order:}

\begin{enumerate}
\tightlist
\item
  Do \textbf{CMD-Pre: Your First Terminal}.
\item
  Continue with CMD-0, CMD-1, and so on.
\end{enumerate}

Do not skip steps --- each lesson builds on the previous one.

\paragraph*{If You Have Used a Terminal Before (But Not with a Screen Reader)}\label{docs__pandoc__latex__src__cmd_foundation__cmd_curriculum_overview__cmd_curriculum_overview.md__if-you-have-used-a-terminal-before-but-not-with-a-screen-reader}

\begin{enumerate}
\tightlist
\item
  Quickly review \textbf{CMD-Pre} (it has the screen reader focus you need).
\item
  Move to \textbf{CMD-0} for deeper path understanding.
\end{enumerate}

\paragraph*{If You Are Experienced with Terminal and Screen Readers}\label{docs__pandoc__latex__src__cmd_foundation__cmd_curriculum_overview__cmd_curriculum_overview.md__if-you-are-experienced-with-terminal-and-screen-readers}

\begin{enumerate}
\tightlist
\item
  Jump to the specific lesson you need (CMD-2, CMD-3, etc.).
\item
  Use the \textbf{Quick Reference} sections.
\item
  Skip exercises and take the quizzes to verify knowledge.
\end{enumerate}

\begin{center}\rule{0.5\linewidth}{0.5pt}\end{center}

\subsubsection*{How Each Lesson Is Structured}\label{docs__pandoc__latex__src__cmd_foundation__cmd_curriculum_overview__cmd_curriculum_overview.md__how-each-lesson-is-structured}

Every lesson contains:

\begin{enumerate}
\item ~
  \subsubsection*{Learning Objectives --- What you will be able to do after the lesson}\label{docs__pandoc__latex__src__cmd_foundation__cmd_curriculum_overview__cmd_curriculum_overview.md__learning-objectives--what-you-will-be-able-to-do-after-the-lesson}
\item
  \textbf{Key Commands} --- The important ones to memorize.
\item
  \textbf{Step-by-Step Examples} --- How to actually do it.
\item
  \textbf{Practice Exercises} --- Hands-on work (critical: do not skip these).
\item
  \textbf{Quiz Questions} --- Check your understanding.
\item
  \textbf{Extension Problems} --- Go deeper if interested.
\end{enumerate}

\textbf{How to get through each lesson:}

\begin{enumerate}
\tightlist
\item
  Read the learning objectives.
\item
  Do the step-by-step examples alongside (type the commands yourself).
\item
  Complete the practice exercises.
\item
  Take the quiz honestly.
\item
  Try extension problems if you have time.
\item
  Move to the next lesson when you can answer the quiz questions confidently.
\end{enumerate}

\textbf{Estimated time:} 2-3 hours per lesson for screen reader users, depending on practice time.

\begin{center}\rule{0.5\linewidth}{0.5pt}\end{center}

\subsubsection*{Screen Reader Tips Throughout the Curriculum}\label{docs__pandoc__latex__src__cmd_foundation__cmd_curriculum_overview__cmd_curriculum_overview.md__screen-reader-tips-throughout-the-curriculum}

All four screen readers are covered in each lesson:

\begin{itemize}
\tightlist
\item
  \textbf{NVDA} --- Free, excellent terminal support. Available at \url{https://www.nvaccess.org/}
\item
  \textbf{JAWS} --- Commercial, powerful, widely used in workplaces. Available at \url{https://www.freedomscientific.com/products/software/jaws/}
\item
  \textbf{Windows Narrator} --- Free, built into Windows, minimal setup required.
\item
  \textbf{Dolphin SuperNova} --- Commercial, with optional magnification. Available at \url{https://yourdolphin.com/supernova/}
\end{itemize}

If your screen reader is not listed in a specific tip, refer to the \textbf{Screen Reader Accessibility Guide} for equivalent commands across all four readers.

\begin{center}\rule{0.5\linewidth}{0.5pt}\end{center}

\subsubsection*{Quick Start Guide (First 45-60 Minutes)}\label{docs__pandoc__latex__src__cmd_foundation__cmd_curriculum_overview__cmd_curriculum_overview.md__quick-start-guide-first-45-60-minutes}

\textbf{If you have 45-60 minutes right now:}

\begin{enumerate}
\item
  Open Command Prompt (Windows key → type \texttt{cmd} → Enter).
\item
  Run these commands:

  \begin{lstlisting}[style=Alabaster, language=cmd]
  cd
  dir /B
  cd Documents
  cd

  \end{lstlisting}
\item
  Notice how your screen reader reads each output.
\item
  Create a file:

  \begin{lstlisting}[style=Alabaster, language=cmd]
  echo I am learning CMD > learning.txt
  type learning.txt

  \end{lstlisting}
\end{enumerate}

That is the core experience. Now read CMD-Pre for the full explanation.

\begin{center}\rule{0.5\linewidth}{0.5pt}\end{center}

\subsubsection*{Common Questions Before Starting}\label{docs__pandoc__latex__src__cmd_foundation__cmd_curriculum_overview__cmd_curriculum_overview.md__common-questions-before-starting}

\paragraph*{Q: PowerShell or CMD? Which should I learn?}\label{docs__pandoc__latex__src__cmd_foundation__cmd_curriculum_overview__cmd_curriculum_overview.md__q-powershell-or-cmd-which-should-i-learn}

Both teach the same fundamental concepts. CMD has simpler syntax and is better for beginners. PowerShell is more powerful and better for advanced automation. This curriculum is CMD; there is a separate PowerShell curriculum if you want that pathway. \textbf{Choose one and stick with it.}

\paragraph*{Q: What if I use a screen reader not listed here?}\label{docs__pandoc__latex__src__cmd_foundation__cmd_curriculum_overview__cmd_curriculum_overview.md__q-what-if-i-use-a-screen-reader-not-listed-here}

The fundamentals work the same across all screen readers. Check your screen reader\textquotesingle s documentation for the equivalent of these functions: "read current line," "read from here to end," and "move through output line by line." All screen readers have these capabilities.

\paragraph*{Q: I am intimidated. Is this really for me?}\label{docs__pandoc__latex__src__cmd_foundation__cmd_curriculum_overview__cmd_curriculum_overview.md__q-i-am-intimidated-is-this-really-for-me}

Yes. This curriculum is designed specifically for people with no terminal experience AND who use screen readers. You start with absolute basics and build up. There is nothing to be afraid of --- the terminal cannot hurt your computer just by navigating and listing files.

\paragraph*{Q: How long will this take?}\label{docs__pandoc__latex__src__cmd_foundation__cmd_curriculum_overview__cmd_curriculum_overview.md__q-how-long-will-this-take}

Realistically:

\begin{itemize}
\tightlist
\item
  Minimum (lessons only, no exercises): 15-18 hours.
\item
  Normal (lessons and exercises): 30-45 hours.
\item
  With extension problems: 45-60 hours or more.
\end{itemize}

Spread it over weeks or months. Go at your pace.

\paragraph*{Q: What if I forget things?}\label{docs__pandoc__latex__src__cmd_foundation__cmd_curriculum_overview__cmd_curriculum_overview.md__q-what-if-i-forget-things}

That is normal. Solutions:

\begin{enumerate}
\tightlist
\item
  Come back to this overview page.
\item
  Jump back to the relevant lesson for a quick review.
\item
  Use the quiz questions to self-test.
\item
  Check the Screen Reader Accessibility Guide for troubleshooting.
\end{enumerate}

\begin{center}\rule{0.5\linewidth}{0.5pt}\end{center}

\subsubsection*{Important Rules}\label{docs__pandoc__latex__src__cmd_foundation__cmd_curriculum_overview__cmd_curriculum_overview.md__important-rules}

\paragraph*{Rule 1: Always Know Where You Are}\label{docs__pandoc__latex__src__cmd_foundation__cmd_curriculum_overview__cmd_curriculum_overview.md__rule-1-always-know-where-you-are}

Every session, first thing:

\begin{lstlisting}[style=Alabaster, language=cmd]
cd

\end{lstlisting}

If you do not know your path, you will get lost. Do not move until you know where you are.

\paragraph*{Rule 2: Check Before You Delete}\label{docs__pandoc__latex__src__cmd_foundation__cmd_curriculum_overview__cmd_curriculum_overview.md__rule-2-check-before-you-delete}

Before deleting anything:

\begin{lstlisting}[style=Alabaster, language=cmd]
dir /B

\end{lstlisting}

Make sure you are deleting the right thing. Deleted files may not be recoverable from the command line.

\paragraph*{\texorpdfstring{Rule 3: Use \texttt{dir\ /B} for Listings}{Rule 3: Use dir /B for Listings}}\label{docs__pandoc__latex__src__cmd_foundation__cmd_curriculum_overview__cmd_curriculum_overview.md__rule-3-use-dir-b-for-listings}

The \texttt{/B} flag gives a clean, one-per-line listing that is easy for all screen readers to follow:

\begin{lstlisting}[style=Alabaster, language=cmd]
dir /B

\end{lstlisting}

\paragraph*{Rule 4: When Output Is Confusing, Redirect to a File}\label{docs__pandoc__latex__src__cmd_foundation__cmd_curriculum_overview__cmd_curriculum_overview.md__rule-4-when-output-is-confusing-redirect-to-a-file}

\begin{lstlisting}[style=Alabaster, language=cmd]
dir /B > output.txt
notepad.exe output.txt

\end{lstlisting}

This is always clearer for all screen readers than reading terminal output directly.

\paragraph*{Rule 5: Save Everything You Create}\label{docs__pandoc__latex__src__cmd_foundation__cmd_curriculum_overview__cmd_curriculum_overview.md__rule-5-save-everything-you-create}

Create a practice folder and keep everything there:

\begin{lstlisting}[style=Alabaster, language=cmd]
mkdir my-practice-folder
cd my-practice-folder

\end{lstlisting}

\begin{center}\rule{0.5\linewidth}{0.5pt}\end{center}

\subsubsection*{Troubleshooting: "Nothing Works!"}\label{docs__pandoc__latex__src__cmd_foundation__cmd_curriculum_overview__cmd_curriculum_overview.md__troubleshooting-nothing-works}

\begin{enumerate}
\item
  \textbf{Cannot hear CMD at all?}

  \begin{itemize}
  \tightlist
  \item
    Make sure your screen reader is running BEFORE opening CMD.
  \item
    Try \textbf{Alt+Tab} to cycle to the CMD window.
  \item
    Restart both your screen reader and CMD.
  \end{itemize}
\item
  \textbf{Commands not working?}

  \begin{itemize}
  \tightlist
  \item
    Check spelling carefully.
  \item
    Make sure you pressed Enter.
  \item
    Try \texttt{help} for a list of available commands.
  \item
    Try \texttt{command\ /?} (for example, \texttt{dir\ /?}) for help on a specific command.
  \end{itemize}
\item
  \textbf{Cannot read the output?}

  \begin{itemize}
  \tightlist
  \item
    Redirect to file: \texttt{command\ \textgreater{}\ output.txt}
  \item
    Open in Notepad: \texttt{notepad.exe\ output.txt}
  \item
    This works with all four screen readers.
  \end{itemize}
\item
  \textbf{Something ran forever?}

  \begin{itemize}
  \tightlist
  \item
    Press \textbf{Ctrl+C} to stop it.
  \end{itemize}
\item
  \textbf{JAWS stopped working?}

  \begin{itemize}
  \tightlist
  \item
    If using JAWS in demo mode, the session limit is approximately 40 minutes. Restart your computer to reset the demo.
  \end{itemize}
\item
  \textbf{Completely confused?}

  \begin{itemize}
  \tightlist
  \item
    Go back to CMD-Pre and start over.
  \item
    Work through every single exercise slowly.
  \item
    Ask an instructor or peer for help.
  \end{itemize}
\end{enumerate}

\begin{center}\rule{0.5\linewidth}{0.5pt}\end{center}

\subsubsection*{Resources}\label{docs__pandoc__latex__src__cmd_foundation__cmd_curriculum_overview__cmd_curriculum_overview.md__resources}

\paragraph*{Official Documentation}\label{docs__pandoc__latex__src__cmd_foundation__cmd_curriculum_overview__cmd_curriculum_overview.md__official-documentation}

\begin{itemize}
\tightlist
\item
  \textbf{Windows CMD Reference:} \url{https://example.com}
\item
  \textbf{Microsoft Learn:} \url{https://example.com}
\end{itemize}

\paragraph*{Screen Reader Guides}\label{docs__pandoc__latex__src__cmd_foundation__cmd_curriculum_overview__cmd_curriculum_overview.md__screen-reader-guides}

\begin{itemize}
\tightlist
\item
  \textbf{NVDA:} \url{https://www.nvaccess.org/}
\item
  \textbf{JAWS:} \url{https://www.freedomscientific.com/products/software/jaws/}
\item
  \textbf{Windows Narrator:} \url{https://support.microsoft.com/narrator}
\item
  \textbf{Dolphin SuperNova:} \url{https://yourdolphin.com/supernova/}
\end{itemize}

\begin{center}\rule{0.5\linewidth}{0.5pt}\end{center}

\subsubsection*{Curriculum Map}\label{docs__pandoc__latex__src__cmd_foundation__cmd_curriculum_overview__cmd_curriculum_overview.md__curriculum-map}

\begin{lstlisting}[style=Alabaster]
START HERE
    |
CMD-Pre: Your First Terminal  (absolute beginner entry point)
    |
CMD-0: Getting Started        (paths and navigation foundations)
    |
CMD-1: Navigation             (comfortable moving around)
    |
CMD-2: File & Folder          (create, move, delete)
    |
CMD-3: Input & Output         (redirect commands)
    |
CMD-4: Variables & Shortcuts  (automation foundations)
    |
CMD-5: Filling in the Gaps    (batch files and history)
    |
CMD-6: Advanced Techniques    (scripts, loops, workflows)
    |
CMD Unit Test                 (comprehensive assessment)
    |
NEXT: 3D Printing Integration Lessons

\end{lstlisting}

\begin{center}\rule{0.5\linewidth}{0.5pt}\end{center}

\textbf{Questions? Feedback? Stuck?} Refer back to this page and try the Troubleshooting section. All the tools you need are here.

\textbf{Now open CMD-Pre and get started!}

\subsection{CMD-Pre: Your First Terminal - Screen Reader Navigation Fundamentals}\label{docs__pandoc__latex__src__cmd_foundation__cmd_pre_your_first_terminal__cmd_pre_your_first_terminal.md__cmd_foundation_cmd_pre_your_first_terminal-cmd_pre_your_first_terminal}

\textbf{Duration:} 1.5-2 hours (for screen reader users)\\
\textbf{Prerequisites:} None --- this is the starting point\\
\textbf{Accessibility Note:} This lesson is designed specifically for screen reader users. Tips are provided for NVDA, JAWS, Windows Narrator, and Dolphin SuperNova.

\begin{center}\rule{0.5\linewidth}{0.5pt}\end{center}

\subsubsection*{What is a Terminal?}\label{docs__pandoc__latex__src__cmd_foundation__cmd_pre_your_first_terminal__cmd_pre_your_first_terminal.md__what-is-a-terminal}

A terminal (also called command line or Command Prompt) is a text-based interface where you type commands instead of clicking buttons. Think of it like sending written instructions to your computer instead of pointing and clicking.

\textbf{Why learn this?}

\begin{itemize}
\tightlist
\item
  Faster and more precise work, especially for 3D printing scripts and automation.
\item
  Essential for using tools like OpenSCAD.
\item
  Accessibility: command-line tools work reliably with screen readers --- output is plain text.
\item
  Simple automation without complex graphical interfaces.
\end{itemize}

\begin{center}\rule{0.5\linewidth}{0.5pt}\end{center}

\subsubsection*{Opening Command Prompt for the First Time}\label{docs__pandoc__latex__src__cmd_foundation__cmd_pre_your_first_terminal__cmd_pre_your_first_terminal.md__opening-command-prompt-for-the-first-time}

\paragraph*{Method 1: Search (Easiest)}\label{docs__pandoc__latex__src__cmd_foundation__cmd_pre_your_first_terminal__cmd_pre_your_first_terminal.md__method-1-search-easiest}

\begin{enumerate}
\tightlist
\item
  Press the \textbf{Windows key} alone.
\item
  Type: \texttt{cmd}
\item
  You will hear search results appear.
\item
  Press \textbf{Enter} to open the first result (Command Prompt).
\end{enumerate}

\paragraph*{Method 2: Using the Run Dialog}\label{docs__pandoc__latex__src__cmd_foundation__cmd_pre_your_first_terminal__cmd_pre_your_first_terminal.md__method-2-using-the-run-dialog}

\begin{enumerate}
\tightlist
\item
  Press \textbf{Windows key + R} (opens the Run dialog).
\item
  Type: \texttt{cmd}
\item
  Press \textbf{Enter}.
\end{enumerate}

\paragraph*{Method 3: From the Start Menu}\label{docs__pandoc__latex__src__cmd_foundation__cmd_pre_your_first_terminal__cmd_pre_your_first_terminal.md__method-3-from-the-start-menu}

\begin{enumerate}
\tightlist
\item
  Press the \textbf{Windows key}.
\item
  Navigate to \textbf{Windows System} or \textbf{Windows Tools} (depending on your Windows version).
\item
  Open \textbf{Command Prompt}.
\end{enumerate}

\paragraph*{First Connection: Understanding the Prompt}\label{docs__pandoc__latex__src__cmd_foundation__cmd_pre_your_first_terminal__cmd_pre_your_first_terminal.md__first-connection-understanding-the-prompt}

When Command Prompt opens, your screen reader will announce the window title and then the \textbf{prompt}. The prompt is where you type commands.

\textbf{What you will hear:}

\begin{lstlisting}[style=Alabaster]
C:\Users\YourName>

\end{lstlisting}

\textbf{What this means:}

\begin{itemize}
\tightlist
\item
  \texttt{C:\textbackslash{}Users\textbackslash{}YourName} = Your current location (the path).
\item
  \texttt{\textgreater{}} = The prompt is ready for input.
\end{itemize}

Your cursor is right after the \texttt{\textgreater{}}. This is where you type.

\begin{center}\rule{0.5\linewidth}{0.5pt}\end{center}

\subsubsection*{Your First Commands (Screen Reader Edition)}\label{docs__pandoc__latex__src__cmd_foundation__cmd_pre_your_first_terminal__cmd_pre_your_first_terminal.md__your-first-commands-screen-reader-edition}

\paragraph*{\texorpdfstring{Command 1: "Where Am I?" --- \texttt{cd}}{Command 1: "Where Am I?" --- cd}}\label{docs__pandoc__latex__src__cmd_foundation__cmd_pre_your_first_terminal__cmd_pre_your_first_terminal.md__command-1-where-am-i--cd}

Running \texttt{cd} with no arguments shows your current directory.

\textbf{Type this and press Enter:}

\begin{lstlisting}[style=Alabaster, language=cmd]
cd

\end{lstlisting}

\textbf{What you will hear:}
Your screen reader announces the current path, for example:

\begin{lstlisting}[style=Alabaster]
C:\Users\YourName

\end{lstlisting}

\textbf{Understanding paths:}

\begin{itemize}
\tightlist
\item
  A path shows your location in the file system, like a mailing address.
\item
  Windows paths use backslashes: \texttt{C:\textbackslash{}Users\textbackslash{}YourName\textbackslash{}Documents}
\item
  Think of it as nested folders: \texttt{C:\textbackslash{}} (the main drive) → \texttt{Users} → \texttt{YourName} → \texttt{Documents}
\end{itemize}

\paragraph*{\texorpdfstring{Command 2: "What\textquotesingle s Here?" --- \texttt{dir}}{Command 2: "What\textquotesingle s Here?" --- dir}}\label{docs__pandoc__latex__src__cmd_foundation__cmd_pre_your_first_terminal__cmd_pre_your_first_terminal.md__command-2-whats-here--dir}

\textbf{Type this and press Enter:}

\begin{lstlisting}[style=Alabaster, language=cmd]
dir /B

\end{lstlisting}

\textbf{What you will hear:}
Your screen reader announces folder and file names, one per line. The \texttt{/B} flag gives a clean "bare" listing --- names only, no dates or sizes --- which is much easier to follow with a screen reader.

If you want more detail (sizes, dates), run \texttt{dir} without \texttt{/B}. But for navigation, \texttt{dir\ /B} is preferred.

\paragraph*{\texorpdfstring{Command 3: "Go There" --- \texttt{cd\ Documents}}{Command 3: "Go There" --- cd Documents}}\label{docs__pandoc__latex__src__cmd_foundation__cmd_pre_your_first_terminal__cmd_pre_your_first_terminal.md__command-3-go-there--cd-documents}

\textbf{Type this and press Enter:}

\begin{lstlisting}[style=Alabaster, language=cmd]
cd Documents

\end{lstlisting}

\textbf{What you will hear:}
The prompt changes to show your new location:

\begin{lstlisting}[style=Alabaster]
C:\Users\YourName\Documents>

\end{lstlisting}

\textbf{Practice navigation:}

\begin{enumerate}
\tightlist
\item
  Run \texttt{cd} to confirm you are in Documents.
\item
  Run \texttt{dir\ /B} to see what is there.
\item
  Go back up: \texttt{cd\ ..} (the \texttt{..} means "go up one level").
\item
  Run \texttt{cd} again to confirm.
\item
  Return to Documents: \texttt{cd\ Documents}
\end{enumerate}

\begin{center}\rule{0.5\linewidth}{0.5pt}\end{center}

\subsubsection*{Reading Screen Reader Output (Critical Skills)}\label{docs__pandoc__latex__src__cmd_foundation__cmd_pre_your_first_terminal__cmd_pre_your_first_terminal.md__reading-screen-reader-output-critical-skills}

\paragraph*{Dealing with Long Lists}\label{docs__pandoc__latex__src__cmd_foundation__cmd_pre_your_first_terminal__cmd_pre_your_first_terminal.md__dealing-with-long-lists}

When you run \texttt{dir} in a folder with many files, the list may be very long. Here is how to manage it:

\textbf{Solution 1: Save to a File}

\begin{lstlisting}[style=Alabaster, language=cmd]
dir /B > list.txt
notepad.exe list.txt

\end{lstlisting}

This saves the listing to a file and opens it in Notepad, where you can read it calmly with your screen reader.

\textbf{Solution 2: Use Pause}

\begin{lstlisting}[style=Alabaster, language=cmd]
dir /B | more

\end{lstlisting}

Shows output one page at a time. Press \textbf{Space} to go to the next page, \textbf{Q} to quit. Note: some screen readers read \texttt{more} output less reliably than a saved file --- if this is difficult, use Solution 1.

\paragraph*{Tab Completion}\label{docs__pandoc__latex__src__cmd_foundation__cmd_pre_your_first_terminal__cmd_pre_your_first_terminal.md__tab-completion}

Tab completion is one of the most powerful screen reader techniques in the terminal.

\textbf{How it works:}

\begin{enumerate}
\tightlist
\item
  Type the first few letters of a folder or file name.
\item
  Press \textbf{Tab}.
\item
  Command Prompt automatically completes the rest.
\end{enumerate}

\textbf{Example:}

\begin{enumerate}
\tightlist
\item
  You are at \texttt{C:\textbackslash{}Users\textbackslash{}YourName\textgreater{}}
\item
  Type: \texttt{cd\ Doc}
\item
  Press \textbf{Tab}
\item
  Command Prompt completes it to: \texttt{cd\ Documents}
\item
  Press \textbf{Enter} to navigate there.
\end{enumerate}

With a screen reader, when you press Tab the screen reader announces the completed command --- much faster and more accurate than typing the whole thing.

\begin{center}\rule{0.5\linewidth}{0.5pt}\end{center}

\subsubsection*{Creating and Viewing Files}\label{docs__pandoc__latex__src__cmd_foundation__cmd_pre_your_first_terminal__cmd_pre_your_first_terminal.md__creating-and-viewing-files}

\paragraph*{Create a Simple File}\label{docs__pandoc__latex__src__cmd_foundation__cmd_pre_your_first_terminal__cmd_pre_your_first_terminal.md__create-a-simple-file}

\begin{lstlisting}[style=Alabaster, language=cmd]
echo Hello, Command Prompt! > hello.txt

\end{lstlisting}

\begin{itemize}
\tightlist
\item
  \texttt{echo} outputs text.
\item
  \texttt{\textgreater{}} redirects it to a file called \texttt{hello.txt}.
\end{itemize}

\paragraph*{Read the File Back}\label{docs__pandoc__latex__src__cmd_foundation__cmd_pre_your_first_terminal__cmd_pre_your_first_terminal.md__read-the-file-back}

\begin{lstlisting}[style=Alabaster, language=cmd]
type hello.txt

\end{lstlisting}

Your screen reader announces the file contents.

\paragraph*{Open and Edit the File}\label{docs__pandoc__latex__src__cmd_foundation__cmd_pre_your_first_terminal__cmd_pre_your_first_terminal.md__open-and-edit-the-file}

\begin{lstlisting}[style=Alabaster, language=cmd]
notepad.exe hello.txt

\end{lstlisting}

Opens the file in Notepad for comfortable editing with your screen reader.

\begin{center}\rule{0.5\linewidth}{0.5pt}\end{center}

\subsubsection*{Essential Keyboard Shortcuts}\label{docs__pandoc__latex__src__cmd_foundation__cmd_pre_your_first_terminal__cmd_pre_your_first_terminal.md__essential-keyboard-shortcuts}

{\def\LTcaptype{none} % do not increment counter
\begin{longtable}[]{@{}
  >{\raggedright\arraybackslash}p{(\linewidth - 2\tabcolsep) * \real{0.2099}}
  >{\raggedright\arraybackslash}p{(\linewidth - 2\tabcolsep) * \real{0.7901}}@{}}
\toprule\noalign{}
\begin{minipage}[b]{\linewidth}\raggedright
Key Combination
\end{minipage} & \begin{minipage}[b]{\linewidth}\raggedright
What It Does
\end{minipage} \\
\midrule\noalign{}
\endhead
\bottomrule\noalign{}
\endlastfoot
\textbf{Up Arrow} &
Shows your previous command (press again for earlier commands) \\
\textbf{Down Arrow} & Shows a more recent command in history \\
\textbf{Tab} & Auto-completes folder and file names \\
\textbf{Ctrl+C} & Stops a running command \\
\textbf{cls} & Clears the screen (type this and press Enter) \\
\textbf{Enter} & Runs the command you have typed \\
\textbf{Home} & Moves cursor to start of command line \\
\textbf{End} & Moves cursor to end of command line \\
\end{longtable}
}

\begin{center}\rule{0.5\linewidth}{0.5pt}\end{center}

\subsubsection*{Screen Reader-Specific Tips}\label{docs__pandoc__latex__src__cmd_foundation__cmd_pre_your_first_terminal__cmd_pre_your_first_terminal.md__screen-reader-specific-tips}

\paragraph*{NVDA Users}\label{docs__pandoc__latex__src__cmd_foundation__cmd_pre_your_first_terminal__cmd_pre_your_first_terminal.md__nvda-users}

NVDA is available free from \url{https://www.nvaccess.org/}

\begin{enumerate}
\tightlist
\item
  \textbf{After running a command}, press \textbf{NVDA+Home} (Insert+Home or CapsLock+Home) to read the current line.
\item
  \textbf{To read all output}, press \textbf{NVDA+Down Arrow} to read from the cursor to the end of the screen.
\item
  \textbf{To browse previous output}, use the NVDA review cursor: press \textbf{Numpad 8/2} (desktop) or the equivalent laptop keys to move up and down through lines.
\item
  \textbf{If output scrolled past}, redirect to a file: \texttt{command\ \textgreater{}\ output.txt}, then \texttt{notepad.exe\ output.txt}.
\item
  \textbf{To open NVDA settings}: press \textbf{NVDA+N} (opens NVDA menu), then go to Preferences.
\end{enumerate}

\begin{quote}
\textbf{NVDA key note:} The NVDA modifier key is \textbf{Insert} by default on desktops. On laptops it is often \textbf{CapsLock}. Check your NVDA settings if these shortcuts do not work as described.
\end{quote}

\paragraph*{JAWS Users}\label{docs__pandoc__latex__src__cmd_foundation__cmd_pre_your_first_terminal__cmd_pre_your_first_terminal.md__jaws-users}

JAWS is available from \url{https://www.freedomscientific.com/products/software/jaws/}

\begin{enumerate}
\tightlist
\item
  \textbf{After running a command}, press \textbf{Insert+Up Arrow} to re-read the current line.
\item
  \textbf{To read all output}, press \textbf{Insert+Ctrl+Down} to read to the end of the screen.
\item
  \textbf{To move through output line by line}, press \textbf{Insert+Down Arrow} repeatedly.
\item
  \textbf{If output is too long}, redirect to a file: \texttt{command\ \textgreater{}\ output.txt}, then \texttt{notepad.exe\ output.txt}.
\item
  \textbf{JAWS demo mode reminder:} If you are using JAWS without a license, sessions are limited to approximately 40 minutes. Save your work and restart the session if JAWS stops responding.
\end{enumerate}

\begin{quote}
\textbf{JAWS key note:} The JAWS modifier key is \textbf{Insert} by default. On some laptops it can be set to \textbf{CapsLock} in JAWS Settings Center.
\end{quote}

\paragraph*{Windows Narrator Users}\label{docs__pandoc__latex__src__cmd_foundation__cmd_pre_your_first_terminal__cmd_pre_your_first_terminal.md__windows-narrator-users}

Windows Narrator is built into Windows 10 and Windows 11. Enable it with \textbf{Windows+Ctrl+Enter}.

\begin{enumerate}
\tightlist
\item
  \textbf{After running a command}, press \textbf{Narrator+D} (CapsLock+D or Insert+D) to read the current line.
\item
  \textbf{To read all output from here}, press \textbf{Narrator+R} to read from the cursor position.
\item
  \textbf{To move through output}, use \textbf{Narrator+Up/Down Arrow} in scan mode (press \textbf{Narrator+Space} to toggle scan mode on).
\item
  \textbf{For long outputs}, always redirect to a file: \texttt{command\ \textgreater{}\ output.txt}, then \texttt{notepad.exe\ output.txt}. Notepad is more comfortable for Narrator users than reading directly from the terminal.
\item
  \textbf{Narrator settings}: Windows key → Settings → Accessibility → Narrator.
\end{enumerate}

\begin{quote}
\textbf{Narrator key note:} The Narrator modifier key is \textbf{CapsLock} or \textbf{Insert}. You can change it in Settings → Accessibility → Narrator → Keyboard shortcuts.
\end{quote}

\paragraph*{Dolphin SuperNova Users}\label{docs__pandoc__latex__src__cmd_foundation__cmd_pre_your_first_terminal__cmd_pre_your_first_terminal.md__dolphin-supernova-users}

Dolphin SuperNova is available from \url{https://yourdolphin.com/supernova/}

\begin{enumerate}
\tightlist
\item
  \textbf{After running a command}, press \textbf{CapsLock+L} to read the current line.
\item
  \textbf{To read all output}, press \textbf{CapsLock+Numpad Plus} (say all from current position).
\item
  \textbf{To move through output}, press \textbf{CapsLock+Numpad 8} (up) and \textbf{CapsLock+Numpad 2} (down) to move line by line through the review buffer.
\item
  \textbf{For long outputs}, redirect to a file: \texttt{command\ \textgreater{}\ output.txt}, then \texttt{notepad.exe\ output.txt}.
\item
  \textbf{SuperNova settings}: press \textbf{CapsLock+SpaceBar} to open the SuperNova Control Panel.
\end{enumerate}

\begin{quote}
\textbf{Dolphin key note:} The SuperNova modifier key is \textbf{CapsLock} by default. This can be changed in the SuperNova Control Panel → Keyboard.
\end{quote}

\begin{center}\rule{0.5\linewidth}{0.5pt}\end{center}

\subsubsection*{Common Issue: "I Can\textquotesingle t Hear the Output"}\label{docs__pandoc__latex__src__cmd_foundation__cmd_pre_your_first_terminal__cmd_pre_your_first_terminal.md__common-issue-i-cant-hear-the-output}

This happens to everyone starting out. Here are the solutions:

\begin{enumerate}
\tightlist
\item
  \textbf{Make sure your screen reader was running before you opened Command Prompt.} If not, close Command Prompt, confirm your screen reader is running, then reopen it.
\item
  \textbf{Press End} to make sure your cursor is at the prompt.
\item
  \textbf{Redirect to a file}: \texttt{command\ \textgreater{}\ output.txt}, then \texttt{notepad.exe\ output.txt}. This always works with all screen readers.
\item
  \textbf{Use the screen reader\textquotesingle s review commands} listed above to scan back through previous output.
\end{enumerate}

\begin{center}\rule{0.5\linewidth}{0.5pt}\end{center}

\subsubsection*{Practice Exercises}\label{docs__pandoc__latex__src__cmd_foundation__cmd_pre_your_first_terminal__cmd_pre_your_first_terminal.md__practice-exercises}

Complete these in order. Take your time.

\paragraph*{Exercise 1: Basic Navigation}\label{docs__pandoc__latex__src__cmd_foundation__cmd_pre_your_first_terminal__cmd_pre_your_first_terminal.md__exercise-1-basic-navigation}

\begin{enumerate}
\tightlist
\item
  Open Command Prompt.
\item
  Run \texttt{cd} and note your location.
\item
  Run \texttt{dir\ /B} and listen to the listing.
\item
  Type \texttt{cd\ Documents} and press Enter.
\item
  Run \texttt{cd} to confirm your new location.
\item
  Run \texttt{dir\ /B} in this new location.
\end{enumerate}

\textbf{Goal:} You know where you are and what is around you.

\paragraph*{Exercise 2: Using Tab Completion}\label{docs__pandoc__latex__src__cmd_foundation__cmd_pre_your_first_terminal__cmd_pre_your_first_terminal.md__exercise-2-using-tab-completion}

\begin{enumerate}
\tightlist
\item
  In your home directory, type \texttt{cd\ D} (just those two characters).
\item
  Press \textbf{Tab}.
\item
  Command Prompt should auto-complete to a folder starting with D.
\item
  Repeat with other folder names.
\item
  Try \texttt{cd\ Down} and Tab to \texttt{Downloads}.
\end{enumerate}

\textbf{Goal:} Tab completion feels natural.

\paragraph*{Exercise 3: Creating and Viewing Files}\label{docs__pandoc__latex__src__cmd_foundation__cmd_pre_your_first_terminal__cmd_pre_your_first_terminal.md__exercise-3-creating-and-viewing-files}

\begin{enumerate}
\tightlist
\item
  Create a file: \texttt{echo\ Test\ content\ \textgreater{}\ test.txt}
\item
  View it: \texttt{type\ test.txt}
\item
  Create another: \texttt{echo\ Line\ 2\ \textgreater{}\ another.txt}
\item
  List both: \texttt{dir\ /B\ *.txt}
\end{enumerate}

\textbf{Goal:} You understand create, view, and list operations.

\paragraph*{Exercise 4: Going Up Levels}\label{docs__pandoc__latex__src__cmd_foundation__cmd_pre_your_first_terminal__cmd_pre_your_first_terminal.md__exercise-4-going-up-levels}

\begin{enumerate}
\tightlist
\item
  Navigate into several folders: \texttt{cd\ Documents}, then \texttt{cd\ folder1}, etc.
\item
  From inside, run \texttt{cd\ ..} multiple times to come back up.
\item
  After each \texttt{cd\ ..}, run \texttt{cd} to confirm your location.
\end{enumerate}

\textbf{Goal:} You are comfortable with relative navigation using \texttt{..}

\paragraph*{Exercise 5: Redirecting Output}\label{docs__pandoc__latex__src__cmd_foundation__cmd_pre_your_first_terminal__cmd_pre_your_first_terminal.md__exercise-5-redirecting-output}

\begin{enumerate}
\tightlist
\item
  Create a listing: \texttt{dir\ /B\ \textgreater{}\ directorylist.txt}
\item
  Open it: \texttt{notepad.exe\ directorylist.txt}
\item
  Read it with your screen reader.
\item
  Close Notepad.
\end{enumerate}

\textbf{Goal:} You can save long outputs to files for easier reading.

\begin{center}\rule{0.5\linewidth}{0.5pt}\end{center}

\subsubsection*{Checkpoint Questions}\label{docs__pandoc__latex__src__cmd_foundation__cmd_pre_your_first_terminal__cmd_pre_your_first_terminal.md__checkpoint-questions}

After this lesson, you should be able to answer all ten with confidence:

\begin{enumerate}
\tightlist
\item
  What does \texttt{cd} do with no arguments?
\item
  What does \texttt{dir\ /B} do?
\item
  Why do we use \texttt{/B} with \texttt{dir}?
\item
  What is your current path right now?
\item
  How do you navigate to a new folder?
\item
  How do you go up one level?
\item
  What is the Tab key for?
\item
  What does \texttt{echo\ text\ \textgreater{}\ file.txt} do?
\item
  How do you read a file back in the terminal?
\item
  How do you stop a command that is running?
\end{enumerate}

\textbf{Answer all 10 before moving to CMD-0.}

\begin{center}\rule{0.5\linewidth}{0.5pt}\end{center}

\subsubsection*{Common Questions}\label{docs__pandoc__latex__src__cmd_foundation__cmd_pre_your_first_terminal__cmd_pre_your_first_terminal.md__common-questions}

\textbf{Q: Should I use Command Prompt or PowerShell?}\\
A: Command Prompt is simpler and a good starting point. PowerShell is more powerful. Both work well with all four screen readers covered in this curriculum. Start with whichever your curriculum pathway specifies.

\textbf{Q: Why is my screen reader not reading the output?}\\
A: This is common. Use \texttt{command\ \textgreater{}\ file.txt} to save output to a file, then open it with \texttt{notepad.exe\ file.txt} for reliable reading.

\textbf{Q: What if I type something wrong?}\\
A: Press \textbf{Enter} and you will see an error message. Type the correct command on the next line. No harm done.

\textbf{Q: How do I get help with a command?}\\
A: Type \texttt{help\ command-name} (for example, \texttt{help\ cd}). Or type \texttt{command\ /?} for built-in help (for example, \texttt{dir\ /?}).

\begin{center}\rule{0.5\linewidth}{0.5pt}\end{center}

\subsubsection*{Next Steps}\label{docs__pandoc__latex__src__cmd_foundation__cmd_pre_your_first_terminal__cmd_pre_your_first_terminal.md__next-steps}

Once comfortable with these basics, move to \textbf{CMD-0: Getting Started} for deeper path understanding, then continue through CMD-1 to CMD-5 for full terminal mastery.

\begin{center}\rule{0.5\linewidth}{0.5pt}\end{center}

\subsubsection*{Troubleshooting}\label{docs__pandoc__latex__src__cmd_foundation__cmd_pre_your_first_terminal__cmd_pre_your_first_terminal.md__troubleshooting}

{\def\LTcaptype{none} % do not increment counter
\begin{longtable}[]{@{}
  >{\raggedright\arraybackslash}p{(\linewidth - 2\tabcolsep) * \real{0.2991}}
  >{\raggedright\arraybackslash}p{(\linewidth - 2\tabcolsep) * \real{0.7009}}@{}}
\toprule\noalign{}
\begin{minipage}[b]{\linewidth}\raggedright
Issue
\end{minipage} & \begin{minipage}[b]{\linewidth}\raggedright
Solution
\end{minipage} \\
\midrule\noalign{}
\endhead
\bottomrule\noalign{}
\endlastfoot
Command Prompt won\textquotesingle t open &
Press Windows+R, type \texttt{cmd}, press Enter \\
Cannot hear the output &
Redirect to a file: \texttt{command\ \textgreater{}\ output.txt}, then \texttt{notepad.exe\ output.txt} \\
Tab completion not working &
Type at least one character before pressing Tab \\
Command not found &
Check spelling; try \texttt{help} for a list of available commands \\
Stuck in a command & Press \textbf{Ctrl+C} to stop it \\
JAWS stops working mid-session &
If using JAWS demo: restart computer, demo sessions last \textasciitilde{}40 minutes \\
\end{longtable}
}

\begin{center}\rule{0.5\linewidth}{0.5pt}\end{center}

\subsubsection*{Resources}\label{docs__pandoc__latex__src__cmd_foundation__cmd_pre_your_first_terminal__cmd_pre_your_first_terminal.md__resources}

\begin{itemize}
\tightlist
\item
  \textbf{NVDA:} \url{https://www.nvaccess.org/}
\item
  \textbf{JAWS:} \url{https://www.freedomscientific.com/products/software/jaws/}
\item
  \textbf{Windows Narrator guide:} \url{https://support.microsoft.com/narrator}
\item
  \textbf{Dolphin SuperNova:} \url{https://yourdolphin.com/supernova/}
\item
  \textbf{Windows CMD Reference:} \url{https://example.com}
\end{itemize}

\subsection{CMD-0: Getting Started - Layout, Paths, and the Shell}\label{docs__pandoc__latex__src__cmd_foundation__cmd_0_getting_started_layout_paths__cmd_0_getting_started_layout_paths.md__cmd-0-getting-started---layout-paths-and-the-shell}

Estimated time: 20-30 minutes

\subsubsection*{Learning Objectives}\label{docs__pandoc__latex__src__cmd_foundation__cmd_0_getting_started_layout_paths__cmd_0_getting_started_layout_paths.md__learning-objectives}

\begin{itemize}
\tightlist
\item
  Launch Command Prompt and locate the prompt
\item
  Understand path notation and shortcuts (\texttt{C:\textbackslash{}}, \texttt{..}, \texttt{.})
\item
  Use tab completion to navigate quickly
\end{itemize}

\subsubsection*{Materials}\label{docs__pandoc__latex__src__cmd_foundation__cmd_0_getting_started_layout_paths__cmd_0_getting_started_layout_paths.md__materials}

\begin{itemize}
\tightlist
\item
  Computer with Windows
\item
  Editor (Notepad/VS Code)
\end{itemize}

\subsubsection*{Step-by-step Tasks}\label{docs__pandoc__latex__src__cmd_foundation__cmd_0_getting_started_layout_paths__cmd_0_getting_started_layout_paths.md__step-by-step-tasks}

\begin{enumerate}
\tightlist
\item
  Open Command Prompt and note the prompt (it includes the current path).
\item
  Run \texttt{cd} and say or note the printed path.
\item
  Use \texttt{dir\ /B} to list names in your home directory.
\item
  Practice \texttt{cd\ Documents}, \texttt{cd\ ..} and \texttt{cd\ \textbackslash{}} until comfortable.
\item
  Try tab-completion: type \texttt{cd\ D} and press Tab.
\end{enumerate}

\subsubsection*{Checkpoints}\label{docs__pandoc__latex__src__cmd_foundation__cmd_0_getting_started_layout_paths__cmd_0_getting_started_layout_paths.md__checkpoints}

\begin{itemize}
\tightlist
\item
  Confirm you can state your current path and move to \texttt{Documents}.
\end{itemize}

\subsubsection*{Quiz - Lesson CMD.0}\label{docs__pandoc__latex__src__cmd_foundation__cmd_0_getting_started_layout_paths__cmd_0_getting_started_layout_paths.md__quiz---lesson-cmd0}

\begin{enumerate}
\tightlist
\item
  What is a path?
\item
  What does \texttt{..} mean?
\item
  How do you autocomplete a path?
\item
  How do you go up one directory?
\item
  What command lists only names (\texttt{dir} flag)?
\item
  True or False: On Windows, CMD uses backslashes (\texttt{\textbackslash{}}) in paths, but forward slashes (\texttt{/}) are also accepted in some contexts.
\item
  Explain the difference between an absolute path and a relative path.
\item
  If you are in \texttt{C:\textbackslash{}Users\textbackslash{}YourName\textbackslash{}Documents} and you type \texttt{cd\ ..}, where do you end up?
\item
  What happens when you press Tab while typing a folder name in Command Prompt?
\item
  Describe a practical reason why understanding paths is important for a 3D printing workflow.
\item
  What does \texttt{C:\textbackslash{}} mean in a path, and when would you use it?
\item
  If a folder path contains spaces (e.g., \texttt{Program\ Files}), how do you navigate to it with \texttt{cd}?
\item
  Explain what the prompt \texttt{C:\textbackslash{}Users\textbackslash{}YourName\textgreater{}} tells you about your current state.
\item
  How would you navigate to your home directory from any location using a single command?
\item
  What is the advantage of using relative paths (like \texttt{..}) versus absolute paths in automation scripts?
\end{enumerate}

\subsubsection*{Extension Problems}\label{docs__pandoc__latex__src__cmd_foundation__cmd_0_getting_started_layout_paths__cmd_0_getting_started_layout_paths.md__extension-problems}

\begin{enumerate}
\tightlist
\item
  Create a nested folder and practice \texttt{cd} into it by typing partial names and using Tab.
\item
  Use \texttt{dir\ /B\ /A:F} to list only files in a folder.
\item
  Save \texttt{cd} output (the path) to a file and open it in Notepad.
\item
  Try \texttt{cd} into a folder whose name contains spaces; observe how quotes are handled.
\item
  Create a short note file and open it from Command Prompt.
\item
  Build a folder structure that mirrors your project organization; navigate to each level and document the path.
\item
  Use \texttt{echo\ \%cd\%} to print your current path and save it to a file.
\item
  Investigate special paths (e.g., \texttt{\%USERPROFILE\%}, \texttt{\%TEMP\%}); write down what each contains and when you\textquotesingle d use them.
\item
  Compare absolute vs. relative paths by navigating to the same folder using each method; explain which is easier for automation.
\item
  Create a batch file that changes to a frequently-used folder and lists its contents in one command; test it from different starting locations.
\item
  Navigate to three different locations and at each one note the prompt, the path from \texttt{cd}, and verify you understand what each shows.
\item
  Create a complex folder tree (at least 5 levels deep) and navigate it using only relative paths; verify your location at each step.
\item
  Document all path shortcuts you know (\texttt{C:\textbackslash{}}, \texttt{..}, \texttt{.}) and demonstrate each one works as expected.
\item
  Write a guide for a peer on how to understand the Command Prompt and path notation without using GUI file explorer.
\item
  Create a troubleshooting flowchart: if someone says "I don\textquotesingle t know where I am," what commands do you give them to find out?
\end{enumerate}

\subsubsection*{References}\label{docs__pandoc__latex__src__cmd_foundation__cmd_0_getting_started_layout_paths__cmd_0_getting_started_layout_paths.md__references}

\begin{itemize}
\tightlist
\item
  Microsoft. (2024). \emph{Command line reference}. \url{https://learn.microsoft.com/en-us/windows-server/administration/windows-commands/}
\item
  Microsoft. (2024). \emph{Using the cd command}. \url{https://learn.microsoft.com/en-us/windows-server/administration/windows-commands/cd_1}
\end{itemize}

\subsubsection*{Helpful Resources}\label{docs__pandoc__latex__src__cmd_foundation__cmd_0_getting_started_layout_paths__cmd_0_getting_started_layout_paths.md__helpful-resources}

\begin{itemize}
\tightlist
\item
  \href{https://docs.microsoft.com/en-us/windows-server/administration/windows-commands/windows-commands-glossary}{Windows Command Reference}
\item
  \href{https://docs.microsoft.com/en-us/windows-server/administration/windows-commands/cd}{CD Command Guide}
\item
  \href{https://docs.microsoft.com/en-us/windows-server/administration/windows-commands/dir}{DIR Command Reference}
\item
  \href{https://superuser.com/questions/167461/what-are-path-separators-on-windows}{Understanding Windows Paths}
\end{itemize}

\subsection{CMD-1: Navigation - Moving Around Your File System}\label{docs__pandoc__latex__src__cmd_foundation__cmd_1_navigation__cmd_1_navigation.md__cmd_foundation_cmd_1_navigation-cmd_1_navigation}

\textbf{Duration:} 1.5-2 hours (for screen reader users)\\
\textbf{Prerequisites:} CMD-Pre and CMD-0\\
\textbf{Learning Objectives:}

\begin{itemize}
\tightlist
\item
  Navigate confidently to any folder location
\item
  Understand and use relative versus absolute paths
\item
  Use shortcuts to move between common folders
\item
  Organize your file system logically
\end{itemize}

\begin{center}\rule{0.5\linewidth}{0.5pt}\end{center}

\subsubsection*{Core Navigation Commands}\label{docs__pandoc__latex__src__cmd_foundation__cmd_1_navigation__cmd_1_navigation.md__core-navigation-commands}

\paragraph*{\texorpdfstring{Command: \texttt{cd} (Change Directory)}{Command: cd (Change Directory)}}\label{docs__pandoc__latex__src__cmd_foundation__cmd_1_navigation__cmd_1_navigation.md__command-cd-change-directory}

\begin{lstlisting}[style=Alabaster, language=cmd]
cd FolderName          :: Go into a folder
cd ..                  :: Go up one level
cd \                   :: Go to root of current drive (e.g., C:\)
cd %USERPROFILE%       :: Go to home directory

\end{lstlisting}

\paragraph*{\texorpdfstring{Command: \texttt{dir\ /B} (List Contents)}{Command: dir /B (List Contents)}}\label{docs__pandoc__latex__src__cmd_foundation__cmd_1_navigation__cmd_1_navigation.md__command-dir-b-list-contents}

\begin{lstlisting}[style=Alabaster, language=cmd]
dir /B                 :: List all names (one per line, screen-reader friendly)
dir /B /A:D            :: List only directories/folders
dir /B /A:F            :: List only files

\end{lstlisting}

\paragraph*{Command: Show Current Location}\label{docs__pandoc__latex__src__cmd_foundation__cmd_1_navigation__cmd_1_navigation.md__command-show-current-location}

\begin{lstlisting}[style=Alabaster, language=cmd]
cd                     :: Running cd with no arguments shows current path

\end{lstlisting}

\begin{center}\rule{0.5\linewidth}{0.5pt}\end{center}

\subsubsection*{Understanding Paths}\label{docs__pandoc__latex__src__cmd_foundation__cmd_1_navigation__cmd_1_navigation.md__understanding-paths}

\paragraph*{Absolute Paths (Full Address)}\label{docs__pandoc__latex__src__cmd_foundation__cmd_1_navigation__cmd_1_navigation.md__absolute-paths-full-address}

An absolute path starts from the root and shows the complete location:

\begin{lstlisting}[style=Alabaster, language=cmd]
C:\Users\YourName\Documents\3DPrinting

\end{lstlisting}

\begin{itemize}
\tightlist
\item
  \texttt{C:\textbackslash{}} --- The C drive root
\item
  \texttt{Users} --- First folder
\item
  \texttt{YourName} --- Your user folder
\item
  \texttt{Documents} --- Documents folder
\item
  \texttt{3DPrinting} --- Your 3D printing folder
\end{itemize}

\textbf{Navigate there directly:}

\begin{lstlisting}[style=Alabaster, language=cmd]
cd C:\Users\YourName\Documents\3DPrinting

\end{lstlisting}

\paragraph*{Relative Paths (Directions From Here)}\label{docs__pandoc__latex__src__cmd_foundation__cmd_1_navigation__cmd_1_navigation.md__relative-paths-directions-from-here}

\begin{lstlisting}[style=Alabaster, language=cmd]
cd Documents\3DPrinting    :: Go into Documents, then into 3DPrinting
cd ..                      :: Go up one level
cd ..\..                   :: Go up two levels

\end{lstlisting}

\textbf{Shortcuts:}

\begin{itemize}
\tightlist
\item
  \texttt{.} = Current folder
\item
  \texttt{..} = Parent folder (up one level)
\item
  \texttt{\%USERPROFILE\%} = Your home folder
\end{itemize}

\begin{center}\rule{0.5\linewidth}{0.5pt}\end{center}

\subsubsection*{Common Navigation Patterns}\label{docs__pandoc__latex__src__cmd_foundation__cmd_1_navigation__cmd_1_navigation.md__common-navigation-patterns}

\paragraph*{Pattern 1: Navigating Down}\label{docs__pandoc__latex__src__cmd_foundation__cmd_1_navigation__cmd_1_navigation.md__pattern-1-navigating-down}

\begin{lstlisting}[style=Alabaster, language=cmd]
C:\>cd Users
C:\Users>cd YourName
C:\Users\YourName>cd Documents
C:\Users\YourName\Documents>

\end{lstlisting}

\paragraph*{Pattern 2: Navigating Up}\label{docs__pandoc__latex__src__cmd_foundation__cmd_1_navigation__cmd_1_navigation.md__pattern-2-navigating-up}

\begin{lstlisting}[style=Alabaster, language=cmd]
C:\Users\YourName\Documents>cd ..
C:\Users\YourName>cd ..
C:\Users>cd ..
C:\>

\end{lstlisting}

\paragraph*{Pattern 3: Jump to a Known Path}\label{docs__pandoc__latex__src__cmd_foundation__cmd_1_navigation__cmd_1_navigation.md__pattern-3-jump-to-a-known-path}

\begin{lstlisting}[style=Alabaster, language=cmd]
C:\Deep\Folder\Somewhere>cd C:\Users\YourName
C:\Users\YourName>

\end{lstlisting}

\paragraph*{Pattern 4: Going Home}\label{docs__pandoc__latex__src__cmd_foundation__cmd_1_navigation__cmd_1_navigation.md__pattern-4-going-home}

\begin{lstlisting}[style=Alabaster, language=cmd]
cd %USERPROFILE%

\end{lstlisting}

\begin{center}\rule{0.5\linewidth}{0.5pt}\end{center}

\subsubsection*{Special Folders and Shortcuts}\label{docs__pandoc__latex__src__cmd_foundation__cmd_1_navigation__cmd_1_navigation.md__special-folders-and-shortcuts}

\begin{lstlisting}[style=Alabaster, language=cmd]
%USERPROFILE%              :: Your home folder
%USERPROFILE%\Desktop      :: Your Desktop
%USERPROFILE%\Documents    :: Your Documents
%USERPROFILE%\Downloads    :: Your Downloads
%ProgramFiles%             :: Program Files folder
C:\                        :: Root of C drive

\end{lstlisting}

\begin{center}\rule{0.5\linewidth}{0.5pt}\end{center}

\subsubsection*{Tab Completion (Essential Skill)}\label{docs__pandoc__latex__src__cmd_foundation__cmd_1_navigation__cmd_1_navigation.md__tab-completion-essential-skill}

\begin{enumerate}
\tightlist
\item
  Type the first few characters of a folder name.
\item
  Press \textbf{Tab}.
\item
  Command Prompt completes it.
\end{enumerate}

\textbf{Example:}

\begin{lstlisting}[style=Alabaster, language=cmd]
C:\Users\YourName>cd Doc  [press Tab]
C:\Users\YourName>cd Documents

\end{lstlisting}

\begin{itemize}
\tightlist
\item
  Your screen reader announces the completed name immediately.
\item
  Press \textbf{Tab} multiple times to cycle through all matches.
\end{itemize}

\begin{center}\rule{0.5\linewidth}{0.5pt}\end{center}

\subsubsection*{Screen Reader Tips for Navigation}\label{docs__pandoc__latex__src__cmd_foundation__cmd_1_navigation__cmd_1_navigation.md__screen-reader-tips-for-navigation}

\paragraph*{NVDA Users}\label{docs__pandoc__latex__src__cmd_foundation__cmd_1_navigation__cmd_1_navigation.md__nvda-users}

\begin{enumerate}
\tightlist
\item
  After each \texttt{cd} command, run \texttt{cd} alone to confirm location. Press \textbf{NVDA+Home} (Insert+Home) to read the current line.
\item
  Press \textbf{NVDA+Down Arrow} to read through \texttt{dir\ /B} output.
\item
  For long listings: \texttt{dir\ /B\ \textgreater{}\ listing.txt}, then \texttt{notepad.exe\ listing.txt}.
\end{enumerate}

\paragraph*{JAWS Users}\label{docs__pandoc__latex__src__cmd_foundation__cmd_1_navigation__cmd_1_navigation.md__jaws-users}

\begin{enumerate}
\tightlist
\item
  After each \texttt{cd} command, press \textbf{Insert+Up Arrow} to re-read the current line.
\item
  Press \textbf{Insert+Down Arrow} repeatedly to read output line by line.
\item
  For long listings: \texttt{dir\ /B\ \textgreater{}\ listing.txt}, then \texttt{notepad.exe\ listing.txt}.
\end{enumerate}

\paragraph*{Windows Narrator Users}\label{docs__pandoc__latex__src__cmd_foundation__cmd_1_navigation__cmd_1_navigation.md__windows-narrator-users}

\begin{enumerate}
\tightlist
\item
  After each \texttt{cd} command, press \textbf{Narrator+D} (CapsLock+D) to read the current line.
\item
  Press \textbf{Narrator+R} to read output from the current position.
\item
  For long listings, always redirect to a file and open in Notepad --- Notepad is more comfortable for Narrator users than reading directly from the terminal buffer.
\end{enumerate}

\paragraph*{Dolphin SuperNova Users}\label{docs__pandoc__latex__src__cmd_foundation__cmd_1_navigation__cmd_1_navigation.md__dolphin-supernova-users}

\begin{enumerate}
\tightlist
\item
  After each \texttt{cd} command, press \textbf{CapsLock+L} to read the current line.
\item
  Press \textbf{CapsLock+Numpad Plus} (say all) to read all output from the current position.
\item
  For long listings: \texttt{dir\ /B\ \textgreater{}\ listing.txt}, then \texttt{notepad.exe\ listing.txt}.
\end{enumerate}

\paragraph*{Best Practice for All Screen Readers}\label{docs__pandoc__latex__src__cmd_foundation__cmd_1_navigation__cmd_1_navigation.md__best-practice-for-all-screen-readers}

Always confirm your location after moving:

\begin{lstlisting}[style=Alabaster, language=cmd]
cd FolderName
cd                        :: Verify your new location

\end{lstlisting}

\begin{center}\rule{0.5\linewidth}{0.5pt}\end{center}

\subsubsection*{Practice Exercises}\label{docs__pandoc__latex__src__cmd_foundation__cmd_1_navigation__cmd_1_navigation.md__practice-exercises}

\paragraph*{Exercise 1: Basic Down Navigation}\label{docs__pandoc__latex__src__cmd_foundation__cmd_1_navigation__cmd_1_navigation.md__exercise-1-basic-down-navigation}

\begin{enumerate}
\tightlist
\item
  \texttt{cd\ \%USERPROFILE\%}
\item
  Run \texttt{cd} to see location.
\item
  \texttt{dir\ /B}
\item
  \texttt{cd\ Documents}
\item
  Run \texttt{cd} to confirm.
\item
  \texttt{dir\ /B}
\end{enumerate}

\paragraph*{Exercise 2: Using Tab Completion}\label{docs__pandoc__latex__src__cmd_foundation__cmd_1_navigation__cmd_1_navigation.md__exercise-2-using-tab-completion}

\begin{enumerate}
\tightlist
\item
  \texttt{cd\ \%USERPROFILE\%}
\item
  Type \texttt{cd\ Doc} then press \textbf{Tab}.
\item
  Press \textbf{Enter}.
\item
  Run \texttt{cd} to confirm.
\end{enumerate}

\paragraph*{Exercise 3: Navigating Up}\label{docs__pandoc__latex__src__cmd_foundation__cmd_1_navigation__cmd_1_navigation.md__exercise-3-navigating-up}

\begin{enumerate}
\tightlist
\item
  Navigate several levels deep.
\item
  Run \texttt{cd} to see full path.
\item
  Use \texttt{cd\ ..} to go up one level at a time.
\item
  Run \texttt{cd} after each step.
\end{enumerate}

\paragraph*{Exercise 4: Absolute Path Navigation}\label{docs__pandoc__latex__src__cmd_foundation__cmd_1_navigation__cmd_1_navigation.md__exercise-4-absolute-path-navigation}

\begin{enumerate}
\tightlist
\item
  Navigate deep (3+ levels).
\item
  Run \texttt{cd\ \%USERPROFILE\%} to jump home.
\item
  Run \texttt{cd} to confirm.
\end{enumerate}

\paragraph*{Exercise 5: Creating and Navigating a Structure}\label{docs__pandoc__latex__src__cmd_foundation__cmd_1_navigation__cmd_1_navigation.md__exercise-5-creating-and-navigating-a-structure}

\begin{lstlisting}[style=Alabaster, language=cmd]
mkdir 3DPractice
cd 3DPractice
mkdir Models
cd Models
mkdir OpenSCAD
cd OpenSCAD

\end{lstlisting}

Run \texttt{cd} at each level. Then navigate back up using only \texttt{cd\ ..}.

\begin{center}\rule{0.5\linewidth}{0.5pt}\end{center}

\subsubsection*{Checkpoint Questions}\label{docs__pandoc__latex__src__cmd_foundation__cmd_1_navigation__cmd_1_navigation.md__checkpoint-questions}

Answer all 10 before moving to CMD-2:

\begin{enumerate}
\tightlist
\item
  What is the difference between \texttt{cd\ Documents} and \texttt{cd\ \textbackslash{}}?
\item
  What does \texttt{cd\ ..} do?
\item
  How do you go to your home folder from anywhere?
\item
  What is an absolute path? Give an example.
\item
  What is a relative path? Give an example.
\item
  How do you confirm your current location?
\item
  What does Tab completion do?
\item
  How would you navigate 3 levels deep, then back home?
\item
  What is the difference between \texttt{.} and \texttt{..}?
\item
  If you are lost, what command should you run first?
\end{enumerate}

\begin{center}\rule{0.5\linewidth}{0.5pt}\end{center}

\subsubsection*{Extension Problems}\label{docs__pandoc__latex__src__cmd_foundation__cmd_1_navigation__cmd_1_navigation.md__extension-problems}

\begin{enumerate}
\tightlist
\item
  Create a folder with spaces in the name (e.g., \texttt{My\ Projects}) and navigate to it using quotes.
\item
  Use Tab completion to navigate 5+ levels deep without typing full names.
\item
  Create a script that saves \texttt{cd} output at each level to a log file.
\item
  Navigate to the same destination using both absolute and relative paths from different starting locations.
\item
  Build a complex folder tree (5+ levels) and navigate using only relative paths.
\item
  Document the full paths to your 5 most-used folders.
\item
  Create folders named \texttt{01Folder}, \texttt{02Folder}, etc. and practice Tab completion through them.
\item
  Navigate to a folder, save the path with \texttt{cd\ \textgreater{}\ mypath.txt}, navigate away, then return using the saved path.
\item
  Challenge: navigate to a destination using only relative paths without running \texttt{cd} to check at each step.
\item
  Create a batch file that navigates to a project folder and lists its contents in one step.
\end{enumerate}

\begin{center}\rule{0.5\linewidth}{0.5pt}\end{center}

\subsubsection*{Common Issues}\label{docs__pandoc__latex__src__cmd_foundation__cmd_1_navigation__cmd_1_navigation.md__common-issues}

\paragraph*{"The system cannot find the path specified"}\label{docs__pandoc__latex__src__cmd_foundation__cmd_1_navigation__cmd_1_navigation.md__the-system-cannot-find-the-path-specified}

\begin{itemize}
\tightlist
\item
  Check spelling with \texttt{dir\ /B} to see correct folder names.
\item
  Use Tab completion to avoid typos.
\item
  Use the absolute path: \texttt{cd\ C:\textbackslash{}Users\textbackslash{}YourName\textbackslash{}Documents}.
\end{itemize}

\paragraph*{"I\textquotesingle m lost"}\label{docs__pandoc__latex__src__cmd_foundation__cmd_1_navigation__cmd_1_navigation.md__im-lost}

\begin{lstlisting}[style=Alabaster, language=cmd]
cd

\end{lstlisting}

This always shows your current location.

\paragraph*{Tab Completion Not Working}\label{docs__pandoc__latex__src__cmd_foundation__cmd_1_navigation__cmd_1_navigation.md__tab-completion-not-working}

\begin{itemize}
\tightlist
\item
  Type at least one character before pressing Tab.
\item
  Confirm the folder exists with \texttt{dir\ /B}.
\item
  Press Tab again to cycle through multiple matches.
\end{itemize}

\begin{center}\rule{0.5\linewidth}{0.5pt}\end{center}

\subsubsection*{Quick Reference}\label{docs__pandoc__latex__src__cmd_foundation__cmd_1_navigation__cmd_1_navigation.md__quick-reference}

\begin{lstlisting}[style=Alabaster, language=cmd]
cd                         :: Show current location
cd FolderName              :: Go into a folder
cd ..                      :: Go up one level
cd ..\..                   :: Go up two levels
cd \                       :: Go to drive root
cd %USERPROFILE%           :: Go to home folder
dir /B                     :: List folder contents
dir /B /A:D                :: List only folders
dir /B /A:F                :: List only files

\end{lstlisting}

\begin{center}\rule{0.5\linewidth}{0.5pt}\end{center}

\subsubsection*{Next Steps}\label{docs__pandoc__latex__src__cmd_foundation__cmd_1_navigation__cmd_1_navigation.md__next-steps}

Complete all exercises, pass the checkpoint questions, then move to \textbf{CMD-2: File \& Folder Manipulation}.

\subsection{CMD-2: File and Folder Manipulation}\label{docs__pandoc__latex__src__cmd_foundation__cmd_2_file_folder_manipulation_modification__cmd_2_file_folder_manipulation_modification.md__cmd_foundation_cmd_2_file_folder_manipulation_modification-cmd_2_file_folder_manipulation_modification}

\textbf{Duration:} 2-2.5 hours (for screen reader users)\\
\textbf{Prerequisites:} CMD-Pre, CMD-0, CMD-1

\textbf{Learning Objectives:}

\begin{itemize}
\tightlist
\item
  Create, copy, move, and delete files and folders safely
\item
  Use wildcards to operate on multiple files
\item
  Understand dangerous operations and how to stay safe
\item
  Rename files and understand file extensions
\end{itemize}

\begin{center}\rule{0.5\linewidth}{0.5pt}\end{center}

\subsubsection*{Core File Manipulation Commands}\label{docs__pandoc__latex__src__cmd_foundation__cmd_2_file_folder_manipulation_modification__cmd_2_file_folder_manipulation_modification.md__core-file-manipulation-commands}

\paragraph*{\texorpdfstring{Create Folders: \texttt{mkdir}}{Create Folders: mkdir}}\label{docs__pandoc__latex__src__cmd_foundation__cmd_2_file_folder_manipulation_modification__cmd_2_file_folder_manipulation_modification.md__create-folders-mkdir}

\begin{lstlisting}[style=Alabaster, language=cmd]
mkdir FolderName                     :: Create a single folder
mkdir Folder1 Folder2 Folder3        :: Create multiple folders at once
mkdir A\B\C                          :: Create nested folders (A, then B inside A, then C inside B)

\end{lstlisting}

\paragraph*{\texorpdfstring{Create Files: \texttt{echo} with Redirection}{Create Files: echo with Redirection}}\label{docs__pandoc__latex__src__cmd_foundation__cmd_2_file_folder_manipulation_modification__cmd_2_file_folder_manipulation_modification.md__create-files-echo-with-redirection}

\begin{lstlisting}[style=Alabaster, language=cmd]
echo This is some content > filename.txt    :: Create a file with text content
echo. > emptyfile.txt                       :: Create an empty file

\end{lstlisting}

Note: \texttt{echo.} (echo followed immediately by a period, no space) creates an empty file. \texttt{echo} with a space would put a space character in the file.

\paragraph*{\texorpdfstring{Copy Files: \texttt{copy}}{Copy Files: copy}}\label{docs__pandoc__latex__src__cmd_foundation__cmd_2_file_folder_manipulation_modification__cmd_2_file_folder_manipulation_modification.md__copy-files-copy}

\begin{lstlisting}[style=Alabaster, language=cmd]
copy source.txt destination.txt          :: Copy a file to a new name
copy source.txt backup\                  :: Copy a file into a folder
copy *.txt backup\                       :: Copy all .txt files to backup folder

\end{lstlisting}

\paragraph*{\texorpdfstring{Copy Folders: \texttt{xcopy}}{Copy Folders: xcopy}}\label{docs__pandoc__latex__src__cmd_foundation__cmd_2_file_folder_manipulation_modification__cmd_2_file_folder_manipulation_modification.md__copy-folders-xcopy}

\begin{lstlisting}[style=Alabaster, language=cmd]
xcopy sourcefolder destinationfolder /E /I    :: Copy a folder and all its contents

\end{lstlisting}

\begin{itemize}
\tightlist
\item
  \texttt{/E} copies all subdirectories including empty ones.
\item
  \texttt{/I} treats the destination as a folder (not a file) if it doesn\textquotesingle t exist.
\end{itemize}

\paragraph*{\texorpdfstring{Move and Rename: \texttt{move}}{Move and Rename: move}}\label{docs__pandoc__latex__src__cmd_foundation__cmd_2_file_folder_manipulation_modification__cmd_2_file_folder_manipulation_modification.md__move-and-rename-move}

\begin{lstlisting}[style=Alabaster, language=cmd]
move oldname.txt newname.txt             :: Rename a file
move file.txt folder\                    :: Move a file into a folder
move *.txt archive\                      :: Move all .txt files to archive folder

\end{lstlisting}

\texttt{move} renames when both source and destination are in the same folder. It moves when the destination is a different folder.

\paragraph*{\texorpdfstring{Delete Files: \texttt{del}}{Delete Files: del}}\label{docs__pandoc__latex__src__cmd_foundation__cmd_2_file_folder_manipulation_modification__cmd_2_file_folder_manipulation_modification.md__delete-files-del}

\begin{lstlisting}[style=Alabaster, language=cmd]
del filename.txt                         :: Delete one specific file
del *.txt                                :: Delete all .txt files in current folder
del /Q *.tmp                             :: Delete quietly (no confirmation prompt)

\end{lstlisting}

\textbf{Warning:} \texttt{del} does not send files to the Recycle Bin. Deleted files are gone immediately.

\paragraph*{\texorpdfstring{Delete Folders: \texttt{rmdir}}{Delete Folders: rmdir}}\label{docs__pandoc__latex__src__cmd_foundation__cmd_2_file_folder_manipulation_modification__cmd_2_file_folder_manipulation_modification.md__delete-folders-rmdir}

\begin{lstlisting}[style=Alabaster, language=cmd]
rmdir foldername                         :: Delete an empty folder
rmdir /S /Q foldername                   :: Delete a folder and ALL its contents (no undo)

\end{lstlisting}

\textbf{Warning:} \texttt{rmdir\ /S} is permanent. Always verify the folder name before running this command.

\begin{center}\rule{0.5\linewidth}{0.5pt}\end{center}

\subsubsection*{Safe File Operations}\label{docs__pandoc__latex__src__cmd_foundation__cmd_2_file_folder_manipulation_modification__cmd_2_file_folder_manipulation_modification.md__safe-file-operations}

\paragraph*{Rule 1: Always Check Before Deleting}\label{docs__pandoc__latex__src__cmd_foundation__cmd_2_file_folder_manipulation_modification__cmd_2_file_folder_manipulation_modification.md__rule-1-always-check-before-deleting}

\begin{lstlisting}[style=Alabaster, language=cmd]
dir /B                   :: See exactly what is here
dir /B *.txt             :: See exactly which .txt files exist
del *.txt                :: Only then delete

\end{lstlisting}

\paragraph*{Rule 2: Make Backups First}\label{docs__pandoc__latex__src__cmd_foundation__cmd_2_file_folder_manipulation_modification__cmd_2_file_folder_manipulation_modification.md__rule-2-make-backups-first}

\begin{lstlisting}[style=Alabaster, language=cmd]
mkdir backup
copy *.txt backup\       :: Copy all .txt files to a backup folder first
del *.txt                :: Now safe to delete originals

\end{lstlisting}

\paragraph*{Rule 3: Test on One File First}\label{docs__pandoc__latex__src__cmd_foundation__cmd_2_file_folder_manipulation_modification__cmd_2_file_folder_manipulation_modification.md__rule-3-test-on-one-file-first}

\begin{lstlisting}[style=Alabaster, language=cmd]
del test-one-file.txt    :: Test deletion on a single file
dir /B                   :: Confirm it is gone

\end{lstlisting}

Only then proceed with bulk operations.

\begin{center}\rule{0.5\linewidth}{0.5pt}\end{center}

\subsubsection*{Using Wildcards}\label{docs__pandoc__latex__src__cmd_foundation__cmd_2_file_folder_manipulation_modification__cmd_2_file_folder_manipulation_modification.md__using-wildcards}

Wildcards let you operate on multiple files matching a pattern.

\paragraph*{\texorpdfstring{\texttt{*} Wildcard (Match Any Number of Characters)}{* Wildcard (Match Any Number of Characters)}}\label{docs__pandoc__latex__src__cmd_foundation__cmd_2_file_folder_manipulation_modification__cmd_2_file_folder_manipulation_modification.md__-wildcard-match-any-number-of-characters}

\begin{lstlisting}[style=Alabaster, language=cmd]
dir /B *.txt             :: List all files ending in .txt
copy *.scad backup\      :: Copy all .scad files to backup
del *.tmp                :: Delete all .tmp files

\end{lstlisting}

\paragraph*{\texorpdfstring{\texttt{?} Wildcard (Match Exactly One Character)}{? Wildcard (Match Exactly One Character)}}\label{docs__pandoc__latex__src__cmd_foundation__cmd_2_file_folder_manipulation_modification__cmd_2_file_folder_manipulation_modification.md__-wildcard-match-exactly-one-character}

\begin{lstlisting}[style=Alabaster, language=cmd]
dir /B file?.txt         :: Matches file1.txt, file2.txt, filea.txt, etc.
copy model?.scad models\ :: Copy model1.scad, model2.scad, etc.

\end{lstlisting}

\begin{center}\rule{0.5\linewidth}{0.5pt}\end{center}

\subsubsection*{Practical Examples}\label{docs__pandoc__latex__src__cmd_foundation__cmd_2_file_folder_manipulation_modification__cmd_2_file_folder_manipulation_modification.md__practical-examples}

\paragraph*{Example 1: Create a Project Structure}\label{docs__pandoc__latex__src__cmd_foundation__cmd_2_file_folder_manipulation_modification__cmd_2_file_folder_manipulation_modification.md__example-1-create-a-project-structure}

\begin{lstlisting}[style=Alabaster, language=cmd]
mkdir 3DProjects
cd 3DProjects
mkdir Models Prints Documentation Backups
dir /B /A:D

\end{lstlisting}

\paragraph*{Example 2: Backup Before Editing}\label{docs__pandoc__latex__src__cmd_foundation__cmd_2_file_folder_manipulation_modification__cmd_2_file_folder_manipulation_modification.md__example-2-backup-before-editing}

\begin{lstlisting}[style=Alabaster, language=cmd]
copy project.scad project-backup.scad    :: Create a backup copy
:: (edit project.scad)
copy project.scad project-v2.scad        :: Save a new versioned copy

\end{lstlisting}

\paragraph*{Example 3: Organize Files by Type}\label{docs__pandoc__latex__src__cmd_foundation__cmd_2_file_folder_manipulation_modification__cmd_2_file_folder_manipulation_modification.md__example-3-organize-files-by-type}

\begin{lstlisting}[style=Alabaster, language=cmd]
mkdir scad-files
mkdir text-files
move *.scad scad-files\
move *.txt text-files\
dir /B /A:D

\end{lstlisting}

\begin{center}\rule{0.5\linewidth}{0.5pt}\end{center}

\subsubsection*{Practice Exercises}\label{docs__pandoc__latex__src__cmd_foundation__cmd_2_file_folder_manipulation_modification__cmd_2_file_folder_manipulation_modification.md__practice-exercises}

\paragraph*{Exercise 1: Create a Folder Structure}\label{docs__pandoc__latex__src__cmd_foundation__cmd_2_file_folder_manipulation_modification__cmd_2_file_folder_manipulation_modification.md__exercise-1-create-a-folder-structure}

\begin{enumerate}
\tightlist
\item
  \texttt{mkdir\ practice-session}
\item
  \texttt{cd\ practice-session}
\item
  \texttt{mkdir\ files\ documents\ models}
\item
  \texttt{dir\ /B\ /A:D}
\end{enumerate}

\textbf{Goal:} Create organized folder structures confidently.

\paragraph*{Exercise 2: Create and Copy Files}\label{docs__pandoc__latex__src__cmd_foundation__cmd_2_file_folder_manipulation_modification__cmd_2_file_folder_manipulation_modification.md__exercise-2-create-and-copy-files}

\begin{enumerate}
\tightlist
\item
  \texttt{echo\ Hello\ World\ \textgreater{}\ test.txt}
\item
  \texttt{type\ test.txt}
\item
  \texttt{copy\ test.txt\ test-backup.txt}
\item
  \texttt{dir\ /B\ *.txt}
\item
  \texttt{type\ test-backup.txt}
\end{enumerate}

\textbf{Goal:} Create and copy files without errors.

\paragraph*{Exercise 3: Safe Deletion Practice}\label{docs__pandoc__latex__src__cmd_foundation__cmd_2_file_folder_manipulation_modification__cmd_2_file_folder_manipulation_modification.md__exercise-3-safe-deletion-practice}

\begin{enumerate}
\tightlist
\item
  \texttt{echo\ content\ \textgreater{}\ file1.txt}
\item
  \texttt{echo\ content\ \textgreater{}\ file2.txt}
\item
  \texttt{echo\ content\ \textgreater{}\ file3.txt}
\item
  \texttt{dir\ /B\ *.txt} (confirm all three exist)
\item
  \texttt{del\ file1.txt} (delete just one)
\item
  \texttt{dir\ /B\ *.txt} (confirm only two remain)
\end{enumerate}

\textbf{Goal:} Practice the "check, then delete" pattern.

\paragraph*{Exercise 4: Wildcard Operations}\label{docs__pandoc__latex__src__cmd_foundation__cmd_2_file_folder_manipulation_modification__cmd_2_file_folder_manipulation_modification.md__exercise-4-wildcard-operations}

\begin{enumerate}
\tightlist
\item
  Create three files: \texttt{echo\ a\ \textgreater{}\ doc1.txt}, \texttt{echo\ b\ \textgreater{}\ doc2.txt}, \texttt{echo\ c\ \textgreater{}\ doc3.txt}
\item
  \texttt{dir\ /B\ *.txt}
\item
  \texttt{mkdir\ archive}
\item
  \texttt{copy\ *.txt\ archive\textbackslash{}}
\item
  \texttt{cd\ archive}
\item
  \texttt{dir\ /B\ *.txt}
\end{enumerate}

\textbf{Goal:} Comfortable using wildcards for bulk operations.

\begin{center}\rule{0.5\linewidth}{0.5pt}\end{center}

\subsubsection*{Screen Reader Tips}\label{docs__pandoc__latex__src__cmd_foundation__cmd_2_file_folder_manipulation_modification__cmd_2_file_folder_manipulation_modification.md__screen-reader-tips}

\paragraph*{NVDA Users}\label{docs__pandoc__latex__src__cmd_foundation__cmd_2_file_folder_manipulation_modification__cmd_2_file_folder_manipulation_modification.md__nvda-users}

After file operations, run \texttt{dir\ /B} and press \textbf{NVDA+Down Arrow} to verify results. Redirect output to a file for complex verification: \texttt{dir\ /B\ \textgreater{}\ check.txt}, then \texttt{notepad.exe\ check.txt}.

\paragraph*{JAWS Users}\label{docs__pandoc__latex__src__cmd_foundation__cmd_2_file_folder_manipulation_modification__cmd_2_file_folder_manipulation_modification.md__jaws-users}

After file operations, run \texttt{dir\ /B} and press \textbf{Insert+Down Arrow} to read results. For complex output: \texttt{dir\ /B\ \textgreater{}\ check.txt}, then \texttt{notepad.exe\ check.txt}.

\paragraph*{Windows Narrator Users}\label{docs__pandoc__latex__src__cmd_foundation__cmd_2_file_folder_manipulation_modification__cmd_2_file_folder_manipulation_modification.md__windows-narrator-users}

After file operations, run \texttt{dir\ /B} and press \textbf{Narrator+R} to read from the current position. Redirect to a file and open in Notepad for long output.

\paragraph*{Dolphin SuperNova Users}\label{docs__pandoc__latex__src__cmd_foundation__cmd_2_file_folder_manipulation_modification__cmd_2_file_folder_manipulation_modification.md__dolphin-supernova-users}

After file operations, run \texttt{dir\ /B} and press \textbf{CapsLock+Numpad Plus} to read results. For long output: \texttt{dir\ /B\ \textgreater{}\ check.txt}, then \texttt{notepad.exe\ check.txt}.

\paragraph*{General Tip for All Screen Readers}\label{docs__pandoc__latex__src__cmd_foundation__cmd_2_file_folder_manipulation_modification__cmd_2_file_folder_manipulation_modification.md__general-tip-for-all-screen-readers}

After any \texttt{copy}, \texttt{move}, or \texttt{del} operation, always verify with \texttt{dir\ /B} to confirm the expected result. Never assume an operation worked without checking.

\begin{center}\rule{0.5\linewidth}{0.5pt}\end{center}

\subsubsection*{Checkpoint Questions}\label{docs__pandoc__latex__src__cmd_foundation__cmd_2_file_folder_manipulation_modification__cmd_2_file_folder_manipulation_modification.md__checkpoint-questions}

\begin{enumerate}
\tightlist
\item
  How do you create a folder?
\item
  How do you copy a file?
\item
  How do you rename a file?
\item
  How do you delete a file safely (checking first)?
\item
  What does \texttt{*} match in a wildcard pattern?
\item
  What does \texttt{?} match in a wildcard pattern?
\item
  How would you copy all .scad files to a backup folder?
\item
  What should you always do before deleting?
\item
  How do you delete a folder and all its contents?
\item
  Why is it important to create a backup before editing?
\end{enumerate}

\begin{center}\rule{0.5\linewidth}{0.5pt}\end{center}

\subsubsection*{Extension Problems}\label{docs__pandoc__latex__src__cmd_foundation__cmd_2_file_folder_manipulation_modification__cmd_2_file_folder_manipulation_modification.md__extension-problems}

\begin{enumerate}
\tightlist
\item
  Create a nested folder structure (5+ levels) using a single \texttt{mkdir} command with backslashes.
\item
  Create 10 files and organize them into subfolders using \texttt{move} with wildcards.
\item
  Create a date-based backup strategy: copy all files to a folder named with today\textquotesingle s date.
\item
  Use wildcards to select specific file types and copy them to separate organized folders.
\item
  Create a file-renaming system (file001.txt, file002.txt, etc.) and practice moving them between folders.
\end{enumerate}

\begin{center}\rule{0.5\linewidth}{0.5pt}\end{center}

\subsubsection*{Next Steps}\label{docs__pandoc__latex__src__cmd_foundation__cmd_2_file_folder_manipulation_modification__cmd_2_file_folder_manipulation_modification.md__next-steps}

Complete all exercises, pass the checkpoint questions, then move to \textbf{CMD-3: Input, Output \& Redirection}.

\subsection{CMD-3: Input, Output, and Piping}\label{docs__pandoc__latex__src__cmd_foundation__cmd_3_input_output_piping__cmd_3_input_output_piping.md__cmd-3-input-output-and-piping}

\textbf{Duration:} 1 class period
\textbf{Prerequisite:} CMD-2 (File and Folder Manipulation)

\begin{center}\rule{0.5\linewidth}{0.5pt}\end{center}

\subsubsection*{Learning Objectives}\label{docs__pandoc__latex__src__cmd_foundation__cmd_3_input_output_piping__cmd_3_input_output_piping.md__learning-objectives}

By the end of this lesson, you will be able to:

\begin{itemize}
\tightlist
\item
  Use \texttt{echo} to print text to the screen
\item
  Use \texttt{type} to read file contents
\item
  Use \texttt{\textgreater{}} to redirect output into a file
\item
  Use \texttt{\textbar{}} (pipe) to send one command\textquotesingle s output to another
\item
  Copy output to the clipboard with \texttt{clip}
\item
  Open files with a text editor from the command line
\end{itemize}

\begin{center}\rule{0.5\linewidth}{0.5pt}\end{center}

\subsubsection*{Commands Covered}\label{docs__pandoc__latex__src__cmd_foundation__cmd_3_input_output_piping__cmd_3_input_output_piping.md__commands-covered}

{\def\LTcaptype{none} % do not increment counter
\begin{longtable}[]{@{}ll@{}}
\toprule\noalign{}
Command & What It Does \\
\midrule\noalign{}
\endhead
\bottomrule\noalign{}
\endlastfoot
\texttt{echo\ text} & Print text to the screen \\
\texttt{type\ filename} & Print the contents of a file \\
\texttt{\textgreater{}\ filename} &
Redirect output into a file (overwrites) \\
\texttt{\textgreater{}\textgreater{}\ filename} &
Append output to a file (adds to end) \\
\texttt{\textbar{}} & Pipe - send output from one command to the next \\
\texttt{clip} & Copy piped input to the Windows clipboard \\
\texttt{notepad\ filename} & Open a file in Notepad \\
\end{longtable}
}

\begin{center}\rule{0.5\linewidth}{0.5pt}\end{center}

\subsubsection*{\texorpdfstring{\texttt{echo} - Printing Text}{echo - Printing Text}}\label{docs__pandoc__latex__src__cmd_foundation__cmd_3_input_output_piping__cmd_3_input_output_piping.md__echo---printing-text}

\texttt{echo} prints text to the screen. It is useful for testing, for writing text into files, and for understanding how piping works.

\begin{lstlisting}[style=Alabaster, language=cmd]
echo Hello, World
echo This is a test

\end{lstlisting}

\begin{center}\rule{0.5\linewidth}{0.5pt}\end{center}

\subsubsection*{\texorpdfstring{\texttt{type} - Reading Files}{type - Reading Files}}\label{docs__pandoc__latex__src__cmd_foundation__cmd_3_input_output_piping__cmd_3_input_output_piping.md__type---reading-files}

\texttt{type} prints the contents of a file to the screen.

\begin{lstlisting}[style=Alabaster, language=cmd]
:: Read a text file
type %USERPROFILE%\Documents\notes.txt

:: Read an OpenSCAD file
type %USERPROFILE%\Documents\OpenSCAD_Projects\project0.scad

\end{lstlisting}

With a long file, use \texttt{type\ filename\ \textbar{}\ more} to read it page by page (press \texttt{Space} to advance, \texttt{Q} to quit).

\begin{center}\rule{0.5\linewidth}{0.5pt}\end{center}

\subsubsection*{\texorpdfstring{\texttt{\textgreater{}} - Redirecting Output to a File}{\textgreater{} - Redirecting Output to a File}}\label{docs__pandoc__latex__src__cmd_foundation__cmd_3_input_output_piping__cmd_3_input_output_piping.md__---redirecting-output-to-a-file}

The \texttt{\textgreater{}} symbol redirects output from the screen into a file instead.

\begin{lstlisting}[style=Alabaster, language=cmd]
:: Create a file with a single line
echo Author: Your Name > header.txt

:: Confirm the file was created and has content
type header.txt

\end{lstlisting}

\textbf{Warning:} \texttt{\textgreater{}} overwrites the file if it already exists. Use \texttt{\textgreater{}\textgreater{}} to append instead:

\begin{lstlisting}[style=Alabaster, language=cmd]
echo Date: 2025 >> header.txt
echo Project: Floor Marker >> header.txt
type header.txt

\end{lstlisting}

\begin{center}\rule{0.5\linewidth}{0.5pt}\end{center}

\subsubsection*{\texorpdfstring{\texttt{\textbar{}} - Piping}{\textbar{} - Piping}}\label{docs__pandoc__latex__src__cmd_foundation__cmd_3_input_output_piping__cmd_3_input_output_piping.md__---piping}

The pipe symbol \texttt{\textbar{}} sends the output of one command to the input of the next. This lets you chain commands together.

\begin{lstlisting}[style=Alabaster, language=cmd]
:: List files and send the list to clip (copies to clipboard)
dir /B | clip

:: Now paste with Ctrl + V anywhere

\end{lstlisting}

\begin{lstlisting}[style=Alabaster, language=cmd]
:: Search within a file's contents using find
type project0.scad | find "cube"

\end{lstlisting}

\begin{center}\rule{0.5\linewidth}{0.5pt}\end{center}

\subsubsection*{\texorpdfstring{\texttt{clip} - Copying to Clipboard}{clip - Copying to Clipboard}}\label{docs__pandoc__latex__src__cmd_foundation__cmd_3_input_output_piping__cmd_3_input_output_piping.md__clip---copying-to-clipboard}

\texttt{clip} takes whatever is piped to it and puts it on the Windows clipboard.

\begin{lstlisting}[style=Alabaster, language=cmd]
:: Copy your current directory path to the clipboard
cd | clip

:: Copy a file listing to clipboard
dir /B | clip

:: Copy the contents of a file to clipboard
type notes.txt | clip

\end{lstlisting}

After any of these, press \texttt{Ctrl\ +\ V} in any application to paste.

\begin{center}\rule{0.5\linewidth}{0.5pt}\end{center}

\subsubsection*{Opening Files in Notepad}\label{docs__pandoc__latex__src__cmd_foundation__cmd_3_input_output_piping__cmd_3_input_output_piping.md__opening-files-in-notepad}

\begin{lstlisting}[style=Alabaster, language=cmd]
:: Open a file in Notepad
notepad %USERPROFILE%\Documents\notes.txt

:: Open a .scad file
notepad %USERPROFILE%\Documents\OpenSCAD_Projects\project0.scad

:: Create a new file and open it
echo. > new_notes.txt
notepad new_notes.txt

\end{lstlisting}

\begin{center}\rule{0.5\linewidth}{0.5pt}\end{center}

\subsubsection*{Step-by-step Tasks}\label{docs__pandoc__latex__src__cmd_foundation__cmd_3_input_output_piping__cmd_3_input_output_piping.md__step-by-step-tasks}

\begin{enumerate}
\tightlist
\item
  Create \texttt{practice.txt} with three lines using \texttt{echo} and \texttt{\textgreater{}}/\texttt{\textgreater{}\textgreater{}}.
\item
  Read the file with \texttt{type\ practice.txt}.
\item
  Pipe the file into \texttt{find} to search for a word.
\item
  Copy the file contents to clipboard with \texttt{type\ practice.txt\ \textbar{}\ clip}.
\item
  Redirect \texttt{dir\ /B} into \texttt{list.txt} and open it in Notepad.
\end{enumerate}

\subsubsection*{Checkpoints}\label{docs__pandoc__latex__src__cmd_foundation__cmd_3_input_output_piping__cmd_3_input_output_piping.md__checkpoints}

\begin{itemize}
\tightlist
\item
  After step 3 you should be able to find a keyword using piping.
\end{itemize}

\subsubsection*{Quiz - Lesson CMD.3}\label{docs__pandoc__latex__src__cmd_foundation__cmd_3_input_output_piping__cmd_3_input_output_piping.md__quiz---lesson-cmd3}

\begin{enumerate}
\tightlist
\item
  What is the difference between \texttt{\textgreater{}} and \texttt{\textgreater{}\textgreater{}}?
\item
  What does the pipe \texttt{\textbar{}} do?
\item
  How do you copy output to the clipboard?
\item
  How would you page through long output?
\item
  How do you suppress output (send it nowhere)?
\item
  True or False: The pipe operator \texttt{\textbar{}} connects the output of one command to the input of another.
\item
  Explain why redirecting output to a file is useful for screen reader users.
\item
  Write a command that would search for the word "sphere" in all \texttt{.scad} files in a directory.
\item
  How would you count the number of lines in a file using CMD piping?
\item
  Describe a practical scenario in 3D printing where you would pipe or redirect command output.
\item
  What would be the difference in output between \texttt{echo\ test\ \textgreater{}\ file.txt} (run twice) vs \texttt{echo\ test\ \textgreater{}\textgreater{}\ file.txt} (run twice)? Show the expected file contents.
\item
  Design a three-step piping chain: read a file, filter for specific content, and save the results; explain what each pipe does.
\item
  You have a 500-line \texttt{.scad} file and need to find all instances of \texttt{sphere()} and count them. Write the command.
\item
  Explain how \texttt{clip} is particularly valuable for screen reader users when working with file paths or long output strings.
\item
  Describe how you would use pipes and redirection to create a timestamped backup report of all \texttt{.stl} files in a 3D printing project folder.
\end{enumerate}

\subsubsection*{Extension Problems}\label{docs__pandoc__latex__src__cmd_foundation__cmd_3_input_output_piping__cmd_3_input_output_piping.md__extension-problems}

\begin{enumerate}
\tightlist
\item
  Use piping to count lines in a file (hint: \texttt{type\ file.txt\ \textbar{}\ find\ /C\ /V\ ""}).
\item
  Save a long \texttt{dir\ /B} output and search it with \texttt{find}.
\item
  Chain multiple pipes to filter and then save results.
\item
  Practice copying different command outputs to clipboard and pasting.
\item
  Create a small batch script that generates a report (counts of files by extension).
\item
  Build a data processing pipeline: read a text file, filter rows, and export results; document each step.
\item
  Write a batch script that pipes directory listing to count occurrences of each file extension; create a summary report.
\item
  Create a log analysis command: read a log file, filter for errors, and save matching lines to a separate error log.
\item
  Design a piping workflow for 3D printing file management: find \texttt{.stl} files, extract their names, and generate a report.
\item
  Develop a reusable piping pattern library: create batch scripts for common filtering, sorting, and reporting patterns; test each with different inputs.
\item
  Build a complex filter pipeline: read a \texttt{.scad} file, extract lines containing specific geometry commands, count each type, and output a summary to both screen and file.
\item
  Create an interactive filtering tool: build a batch script that accepts a search term, pipes through multiple filters, and displays paginated results.
\item
  Develop a performance analysis tool: use piping to combine file listing, metadata extraction, and statistical reporting; export results to a dated report file.
\item
  Implement a comprehensive error-handling pipeline: read output, catch errors, log them separately, and generate a summary of successes vs failures.
\item
  Design and execute a real-world project backup workflow: use piping to verify file existence, count files by type, generate a backup manifest, and create audit logs --- all in one integrated command pipeline.
\end{enumerate}

\subsubsection*{References}\label{docs__pandoc__latex__src__cmd_foundation__cmd_3_input_output_piping__cmd_3_input_output_piping.md__references}

\begin{itemize}
\tightlist
\item
  Microsoft. (2024). \emph{Using redirection operators in CMD}. \url{https://example.com}
\item
  Microsoft. (2024). \emph{FIND command reference}. \url{https://example.com}
\item
  Microsoft. (2024). \emph{CMD pipeline concepts}. \url{https://example.com}
\end{itemize}

\subsubsection*{Helpful Resources}\label{docs__pandoc__latex__src__cmd_foundation__cmd_3_input_output_piping__cmd_3_input_output_piping.md__helpful-resources}

\begin{itemize}
\tightlist
\item
  \href{https://learn.microsoft.com/en-us/windows-server/administration/windows-commands/redirection}{Using Redirection in CMD}
\item
  \href{https://learn.microsoft.com/en-us/windows-server/administration/windows-commands/find}{Piping and FIND}
\item
  \href{https://learn.microsoft.com/en-us/windows-server/administration/windows-commands/type}{TYPE Command Reference}
\item
  \href{https://learn.microsoft.com/en-us/windows-server/administration/windows-commands/findstr}{FINDSTR for Advanced Searching}
\item
  \href{https://learn.microsoft.com/en-us/windows-server/administration/windows-commands/}{CMD Pipeline Concepts}
\end{itemize}

\subsection{CMD-4: Environment Variables, PATH, and Aliases}\label{docs__pandoc__latex__src__cmd_foundation__cmd_4_environment_variables_aliases__cmd_4_environment_variables_aliases.md__cmd-4-environment-variables-path-and-aliases}

Estimated time: 30-45 minutes

\subsubsection*{Learning Objectives}\label{docs__pandoc__latex__src__cmd_foundation__cmd_4_environment_variables_aliases__cmd_4_environment_variables_aliases.md__learning-objectives}

\begin{itemize}
\tightlist
\item
  Read environment variables with \texttt{\%VARNAME\%}
\item
  Inspect and verify programs in the \texttt{PATH}
\item
  Create temporary aliases with \texttt{doskey} and understand making them persistent via a startup script
\end{itemize}

\subsubsection*{Materials}\label{docs__pandoc__latex__src__cmd_foundation__cmd_4_environment_variables_aliases__cmd_4_environment_variables_aliases.md__materials}

\begin{itemize}
\tightlist
\item
  Command Prompt
\end{itemize}

\subsubsection*{Step-by-step Tasks}\label{docs__pandoc__latex__src__cmd_foundation__cmd_4_environment_variables_aliases__cmd_4_environment_variables_aliases.md__step-by-step-tasks}

\begin{enumerate}
\tightlist
\item
  Show your username and home path with \texttt{echo\ \%USERNAME\%} and \texttt{echo\ \%USERPROFILE\%}.
\item
  Inspect \texttt{echo\ \%PATH\%} and identify whether \texttt{openscad} or \texttt{code} would be found.
\item
  Run \texttt{where\ openscad} and note the result.
\item
  Create a temporary alias: \texttt{doskey\ preview=openscad\ \$*} and test \texttt{preview\ myfile.scad}.
\item
  Open your startup script (\texttt{notepad.exe\ autorun.bat}) and add the doskey line to make it persistent (advanced).
\end{enumerate}

\subsubsection*{Checkpoints}\label{docs__pandoc__latex__src__cmd_foundation__cmd_4_environment_variables_aliases__cmd_4_environment_variables_aliases.md__checkpoints}

\begin{itemize}
\tightlist
\item
  After step 3 you can determine whether a program will be found by \texttt{PATH}.
\end{itemize}

\subsubsection*{Quiz - Lesson CMD.4}\label{docs__pandoc__latex__src__cmd_foundation__cmd_4_environment_variables_aliases__cmd_4_environment_variables_aliases.md__quiz---lesson-cmd4}

\begin{enumerate}
\tightlist
\item
  How do you print an environment variable?
\item
  What is the purpose of \texttt{PATH}?
\item
  How do you check whether \texttt{openscad} is available?
\item
  How do you create a temporary alias in CMD?
\item
  Where would you make an alias permanent?
\item
  True or False: Environment variable names in CMD are case-sensitive.
\item
  Explain why having a program in your PATH is useful compared to always using its full file path.
\item
  Write a command that would create a doskey alias called \texttt{slicer} for the OpenSCAD executable.
\item
  What file or technique would you use to make a doskey alias persist across CMD sessions?
\item
  Describe a practical benefit of using the \texttt{\%TEMP\%} directory for temporary files in a 3D printing workflow.
\item
  You have a custom batch script at \texttt{C:\textbackslash{}Scripts\textbackslash{}backup\_models.bat} that you want to run from anywhere as \texttt{backup-now}. What steps would you take to make this work?
\item
  Explain the difference between setting an environment variable in the current session with \texttt{set} vs. using \texttt{setx} for permanence.
\item
  Design a strategy for managing multiple 3D printing projects, each with different tool paths and directories; show how to structure environment variables for each.
\item
  If a program is not found by \texttt{where}, what are the possible reasons, and how would you troubleshoot?
\item
  Describe how you would verify that your CMD autorun script is loading correctly and how to debug issues if aliases or environment variables don\textquotesingle t appear after opening a new CMD session.
\end{enumerate}

\subsubsection*{Extension Problems}\label{docs__pandoc__latex__src__cmd_foundation__cmd_4_environment_variables_aliases__cmd_4_environment_variables_aliases.md__extension-problems}

\begin{enumerate}
\tightlist
\item
  Add a folder to PATH for a test program (describe steps; do not change system PATH without admin).
\item
  Create a short autorun snippet that sets two aliases and test re-opening CMD.
\item
  Use \texttt{where} to list the path for several common programs.
\item
  Explore \texttt{\%TEMP\%} and create a file there.
\item
  Save a copy of your current PATH to a text file and examine it in your editor.
\item
  Create a CMD autorun script that loads custom aliases and environment variables for your 3D printing workflow; test it in a new session.
\item
  Build a "project profile" batch script that sets environment variables for CAD, slicing, and print directories; switch between profiles for different projects.
\item
  Write a batch script that audits your current environment variables and creates a summary report of what\textquotesingle s set and why.
\item
  Design a custom alias system using doskey for common 3D printing commands; document the aliases and their purposes.
\item
  Create a profile migration guide: document how to export and import your CMD aliases and variables across machines for consistent workflows.
\item
  Implement a safe PATH modification script: create a utility that allows you to add/remove directories from PATH for the current session only; show how to make it permanent with \texttt{setx}.
\item
  Build a comprehensive autorun framework: create separate \texttt{.bat} files for aliases, environment variables, and helper macros; have your main autorun load all of them.
\item
  Develop an environment validation tool: write a batch script that checks whether all required programs (OpenSCAD, slicers, etc.) are accessible via PATH; report findings and suggest fixes.
\item
  Create a project-switching alias system: design a batch script that changes all environment variables and aliases based on the current project; test switching between multiple projects.
\item
  Build a troubleshooting guide: create a batch script that exports your current environment state (variables, aliases, PATH) to a timestamped file, allowing you to compare states before and after changes.
\end{enumerate}

\subsubsection*{References}\label{docs__pandoc__latex__src__cmd_foundation__cmd_4_environment_variables_aliases__cmd_4_environment_variables_aliases.md__references}

\begin{itemize}
\tightlist
\item
  Microsoft. (2024). \emph{Environment variables in CMD}. \url{https://example.com}
\item
  Microsoft. (2024). \emph{DOSKEY command reference}. \url{https://example.com}
\item
  Microsoft. (2024). \emph{WHERE command reference}. \url{https://example.com}
\end{itemize}

\subsubsection*{Helpful Resources}\label{docs__pandoc__latex__src__cmd_foundation__cmd_4_environment_variables_aliases__cmd_4_environment_variables_aliases.md__helpful-resources}

\begin{itemize}
\tightlist
\item
  \href{https://learn.microsoft.com/en-us/windows-server/administration/windows-commands/set}{Environment Variables in CMD}
\item
  \href{https://learn.microsoft.com/en-us/windows-server/administration/windows-commands/path}{Understanding the PATH Variable}
\item
  \href{https://learn.microsoft.com/en-us/windows-server/administration/windows-commands/doskey}{DOSKEY Alias Reference}
\item
  \href{https://learn.microsoft.com/en-us/windows-server/administration/windows-commands/where}{WHERE Command for Locating Programs}
\item
  \href{https://learn.microsoft.com/en-us/windows-server/administration/windows-commands/setx}{SETX for Permanent Variables}
\end{itemize}

\subsection{CMD-5: Filling in the Gaps - Control Flow, Startup Scripts, and Useful Tricks}\label{docs__pandoc__latex__src__cmd_foundation__cmd_5_filling_in_the_gaps__cmd_5_filling_in_the_gaps.md__cmd-5-filling-in-the-gaps---control-flow-startup-scripts-and-useful-tricks}

Estimated time: 30-45 minutes

\subsubsection*{Learning Objectives}\label{docs__pandoc__latex__src__cmd_foundation__cmd_5_filling_in_the_gaps__cmd_5_filling_in_the_gaps.md__learning-objectives}

\begin{itemize}
\tightlist
\item
  Use history and abort commands (\texttt{doskey\ /history}, \texttt{F7}, \texttt{Ctrl+C})
\item
  Inspect and edit your CMD autorun script for persistent settings
\item
  Run programs by full path using the \texttt{start} or call operator
\item
  Handle common screen reader edge cases when using the terminal
\end{itemize}

\subsubsection*{Materials}\label{docs__pandoc__latex__src__cmd_foundation__cmd_5_filling_in_the_gaps__cmd_5_filling_in_the_gaps.md__materials}

\begin{itemize}
\tightlist
\item
  Command Prompt and an editor (Notepad/VS Code)
\end{itemize}

\subsubsection*{Step-by-step Tasks}\label{docs__pandoc__latex__src__cmd_foundation__cmd_5_filling_in_the_gaps__cmd_5_filling_in_the_gaps.md__step-by-step-tasks}

\begin{enumerate}
\tightlist
\item
  Run several simple commands (e.g., \texttt{cd}, \texttt{dir\ /B}, \texttt{echo\ hi}) then press \texttt{F7} or run \texttt{doskey\ /history} to view them.
\item
  Use \texttt{F8} to search back through history, or use the up-arrow to re-run a previous command.
\item
  Practice aborting a long-running command with \texttt{Ctrl\ +\ C} (for example, \texttt{ping\ 8.8.8.8}).
\item
  Open your autorun script: \texttt{notepad.exe\ autorun.bat}; if it doesn\textquotesingle t exist, create it with \texttt{echo.\ \textgreater{}\ autorun.bat}.
\item
  Add a persistent alias line to your autorun script (example: \texttt{doskey\ preview=openscad\ \$*}), save, and configure CMD to use it by registering it in the registry (advanced).
\end{enumerate}

\subsubsection*{Checkpoints}\label{docs__pandoc__latex__src__cmd_foundation__cmd_5_filling_in_the_gaps__cmd_5_filling_in_the_gaps.md__checkpoints}

\begin{itemize}
\tightlist
\item
  After step 2 you can re-run a recent command from history.
\item
  After step 5 your alias should persist across CMD sessions.
\end{itemize}

\subsubsection*{Quiz - Lesson CMD.5}\label{docs__pandoc__latex__src__cmd_foundation__cmd_5_filling_in_the_gaps__cmd_5_filling_in_the_gaps.md__quiz---lesson-cmd5}

\begin{enumerate}
\tightlist
\item
  How do you view the command history in CMD?
\item
  Which key combination aborts a running command?
\item
  What does \texttt{echo\ \%CMDCMDLINE\%} show?
\item
  How does the \texttt{start} command help run executables?
\item
  What is one strategy if terminal output stops being announced by your screen reader?
\item
  True or False: Using \texttt{Ctrl+C} permanently deletes any files created by the command you abort.
\item
  Explain the difference between pressing \texttt{F7} and running \texttt{doskey\ /history} in CMD.
\item
  If you add a doskey macro to your autorun script but it doesn\textquotesingle t take effect after opening a new CMD window, what should you verify?
\item
  Write a command that would run a program at the path \texttt{C:\textbackslash{}Program\ Files\textbackslash{}OpenSCAD\textbackslash{}openscad.exe} directly.
\item
  Describe a practical workflow scenario where having keyboard shortcuts (doskey macros) in your autorun script would save time.
\item
  Explain how to re-run the 5th command from your history using \texttt{F7} and selection, and what would happen if that command had file operations (creates/deletes).
\item
  Design an autorun initialization strategy that separates utilities for different projects; explain how you would switch between them.
\item
  Walk through a troubleshooting workflow: your screen reader stops announcing output after running a long command. What steps would you take to diagnose and resolve the issue?
\item
  Create a safety checkpoint system: before any destructive operation (mass delete, overwrite), how would you use autorun macros and history to verify the command is correct?
\item
  Develop a comprehensive capstone scenario: integrate everything from CMD-0 through CMD-5 (navigation, file operations, piping, environment setup, history) to design an automated 3D printing project workflow with error handling and logging.
\end{enumerate}

\subsubsection*{Extension Problems}\label{docs__pandoc__latex__src__cmd_foundation__cmd_5_filling_in_the_gaps__cmd_5_filling_in_the_gaps.md__extension-problems}

\begin{enumerate}
\tightlist
\item
  Add a doskey macro and an environment variable change to your autorun script and document the behavior after reopening CMD.
\item
  Create a short batch script that automates creating a project folder and an initial \texttt{.scad} file.
\item
  Experiment with running OpenSCAD by full path using \texttt{start} and by placing it in PATH; compare results.
\item
  Practice redirecting \texttt{help} output to a file and reading it in Notepad for screen reader clarity.
\item
  Document three screen reader troubleshooting steps you used and when they helped.
\item
  Build a comprehensive CMD autorun script that includes aliases, environment variables, and helper macros for your 3D printing workflow.
\item
  Create a batch script that troubleshoots common CMD issues (missing commands, permission errors, command not found); test at least three scenarios.
\item
  Write a batch script that coordinates multiple tasks: creates a project folder, starts OpenSCAD, and opens a notes file.
\item
  Design a screen-reader accessibility guide for CMD: document commands, outputs, and accessible navigation patterns.
\item
  Develop an advanced CMD workflow: implement error handling, logging, and confirmation prompts for risky operations.
\item
  Implement an "undo" system using history: create a batch script that logs destructive commands (\texttt{del}, \texttt{move}, \texttt{copy\ /Y}) and allows you to review the last operation.
\item
  Build an autorun debugger: create a script that compares two CMD sessions\textquotesingle{} environment states (variables, aliases, macros) to identify what loaded/failed to load.
\item
  Develop a multi-project autorun manager: design a system where you can switch entire environments (paths, aliases, variables) for different 3D printing projects by running a single script.
\item
  Create a comprehensive accessibility analyzer: write a batch script that tests whether key CMD commands produce screen-reader-friendly output; document workarounds for commands that don\textquotesingle t.
\item
  Design a complete capstone project: build an integrated automation suite that manages a 3D printing workflow (project setup, file organization, CAD/slicing tool automation, output logging, error recovery, and audit trails) with full error handling and documentation.
\end{enumerate}

\subsubsection*{References}\label{docs__pandoc__latex__src__cmd_foundation__cmd_5_filling_in_the_gaps__cmd_5_filling_in_the_gaps.md__references}

\begin{itemize}
\tightlist
\item
  Microsoft. (2024). \emph{CMD history and recall functionality}. \url{https://example.com}
\item
  Microsoft. (2024). \emph{Using CMD autorun scripts}. \url{https://example.com}
\item
  Microsoft. (2024). \emph{The START command for running executables}. \url{https://example.com}
\end{itemize}

\subsubsection*{Helpful Resources}\label{docs__pandoc__latex__src__cmd_foundation__cmd_5_filling_in_the_gaps__cmd_5_filling_in_the_gaps.md__helpful-resources}

\begin{itemize}
\tightlist
\item
  \href{https://learn.microsoft.com/en-us/windows-server/administration/windows-commands/doskey}{DOSKEY History and Recall}
\item
  \href{https://learn.microsoft.com/en-us/windows-server/administration/windows-commands/cmd}{CMD Autorun Scripts}
\item
  \href{https://learn.microsoft.com/en-us/windows-server/administration/windows-commands/start}{START Command Reference}
\item
  \href{https://learn.microsoft.com/en-us/windows-server/administration/windows-commands/}{CMD History Navigation}
\item
  \href{https://learn.microsoft.com/en-us/windows-server/administration/windows-commands/windows-commands-glossary}{Screen Reader Tips for CMD}
\end{itemize}

\subsection{CMD-6: Advanced Terminal Techniques - Batch Scripts, Functions \& Professional Workflows}\label{docs__pandoc__latex__src__cmd_foundation__cmd_6_advanced_techniques__cmd_6_advanced_techniques.md__cmd-6-advanced-terminal-techniques---batch-scripts-functions--professional-workflows}

\textbf{Duration:} 4-4.5 hours (for screen reader users)
\textbf{Prerequisites:} Complete CMD-0 through CMD-5
\textbf{Skill Level:} Advanced intermediate

This lesson extends CMD skills to professional-level workflows. You\textquotesingle ll learn to automate complex tasks, write reusable batch scripts, and integrate tools for 3D printing workflows.

\begin{center}\rule{0.5\linewidth}{0.5pt}\end{center}

\subsubsection*{Learning Objectives}\label{docs__pandoc__latex__src__cmd_foundation__cmd_6_advanced_techniques__cmd_6_advanced_techniques.md__learning-objectives}

By the end of this lesson, you will be able to:

\begin{itemize}
\tightlist
\item
  Create and run batch files (.bat files)
\item
  Write subroutines that accept parameters
\item
  Use loops to repeat tasks automatically
\item
  Automate batch processing of 3D models
\item
  Debug batch scripts when something goes wrong
\item
  Create professional workflows combining multiple tools
\end{itemize}

\begin{center}\rule{0.5\linewidth}{0.5pt}\end{center}

\subsubsection*{Batch Script Basics}\label{docs__pandoc__latex__src__cmd_foundation__cmd_6_advanced_techniques__cmd_6_advanced_techniques.md__batch-script-basics}

\paragraph*{What\textquotesingle s a Batch Script?}\label{docs__pandoc__latex__src__cmd_foundation__cmd_6_advanced_techniques__cmd_6_advanced_techniques.md__whats-a-batch-script}

A batch file (\texttt{.bat}) contains multiple CMD commands that run in sequence. Instead of typing commands one by one, you put them in a file and run them all at once.

\textbf{Why use batch scripts?}

\begin{itemize}
\tightlist
\item
  Repeatability: Run the same task 100 times identically
\item
  Documentation: Commands are written down for reference
\item
  Complexity: Combine many commands logically
\item
  Automation: Schedule scripts to run automatically
\end{itemize}

\paragraph*{Creating Your First Batch Script}\label{docs__pandoc__latex__src__cmd_foundation__cmd_6_advanced_techniques__cmd_6_advanced_techniques.md__creating-your-first-batch-script}

\textbf{Step 1: Open a text editor}

\begin{lstlisting}[style=Alabaster, language=cmd]
notepad.exe my-first-script.bat

\end{lstlisting}

\textbf{Step 2: Type this script}

\begin{lstlisting}[style=Alabaster, language=cmd]
@echo off
:: This is a comment - screen readers will read it
echo Script is running!
cd
dir /B
echo Script is done!

\end{lstlisting}

\textbf{Step 3: Save the file}

\begin{itemize}
\tightlist
\item
  In Notepad: Ctrl+S
\item
  Make sure filename ends in \texttt{.bat}
\item
  Save in an easy-to-find location (like Documents)
\end{itemize}

\textbf{Step 4: Run the script}

\begin{lstlisting}[style=Alabaster, language=cmd]
my-first-script.bat

\end{lstlisting}

\textbf{What happens:}
CMD runs each command in sequence and shows output.

\paragraph*{\texorpdfstring{Important: \texttt{@echo\ off}}{Important: @echo off}}\label{docs__pandoc__latex__src__cmd_foundation__cmd_6_advanced_techniques__cmd_6_advanced_techniques.md__important-echo-off}

On some scripts, each command is echoed before running. \texttt{@echo\ off} at the top suppresses this, showing only output --- not the commands themselves.

\begin{lstlisting}[style=Alabaster, language=cmd]
@echo off
:: Now only output is shown, not each command
echo Hello!

\end{lstlisting}

\begin{center}\rule{0.5\linewidth}{0.5pt}\end{center}

\subsubsection*{Variables and Parameters}\label{docs__pandoc__latex__src__cmd_foundation__cmd_6_advanced_techniques__cmd_6_advanced_techniques.md__variables-and-parameters}

\paragraph*{Using Variables}\label{docs__pandoc__latex__src__cmd_foundation__cmd_6_advanced_techniques__cmd_6_advanced_techniques.md__using-variables}

Variables store values you want to use later.

\textbf{Example script:}

\begin{lstlisting}[style=Alabaster, language=cmd]
@echo off
set mypath=C:\Users\YourName\Documents
cd %mypath%
echo I am now in:
cd
dir /B

\end{lstlisting}

\textbf{Breaking it down:}

\begin{itemize}
\tightlist
\item
  \texttt{set\ mypath=...} assigns the variable
\item
  \texttt{\%mypath\%} uses the variable (wraps it in \texttt{\%})
\item
  Variables in CMD are always referenced with \texttt{\%VARNAME\%}
\end{itemize}

\paragraph*{Subroutines with Parameters}\label{docs__pandoc__latex__src__cmd_foundation__cmd_6_advanced_techniques__cmd_6_advanced_techniques.md__subroutines-with-parameters}

A subroutine is reusable code that you can call with different inputs using \texttt{:label} and \texttt{call}.

\textbf{Example: A subroutine that lists files in a folder}

\begin{lstlisting}[style=Alabaster, language=cmd]
@echo off
call :ListFolder "C:\Users\YourName\Documents"
call :ListFolder "C:\Users\YourName\Downloads"
goto :eof

:ListFolder
echo Contents of: %~1
cd /D %~1
dir /B
goto :eof

\end{lstlisting}

\textbf{What\textquotesingle s happening:}

\begin{itemize}
\tightlist
\item
  \texttt{:ListFolder} marks the start of the subroutine
\item
  \texttt{\%\textasciitilde{}1} is the first argument passed to the subroutine
\item
  \texttt{call\ :ListFolder\ "path"} calls the subroutine with a path
\item
  \texttt{goto\ :eof} exits the subroutine (or script)
\end{itemize}

\textbf{Screen reader tip:} When you call a subroutine, CMD will announce the results just like any command.

\begin{center}\rule{0.5\linewidth}{0.5pt}\end{center}

\subsubsection*{Loops - Repeating Tasks}\label{docs__pandoc__latex__src__cmd_foundation__cmd_6_advanced_techniques__cmd_6_advanced_techniques.md__loops---repeating-tasks}

\paragraph*{Loop Over Files}\label{docs__pandoc__latex__src__cmd_foundation__cmd_6_advanced_techniques__cmd_6_advanced_techniques.md__loop-over-files}

Imagine you have 10 SCAD files and want to print their names. You could do it 10 times manually, or use a loop.

\textbf{Example: Print every .scad file in a folder}

\begin{lstlisting}[style=Alabaster, language=cmd]
@echo off
for %%f in (*.scad) do (
    echo === File: %%f ===
    type %%f
    echo.
)

\end{lstlisting}

\textbf{What\textquotesingle s happening:}

\begin{itemize}
\tightlist
\item
  \texttt{for\ \%\%f\ in\ (*.scad)\ do} iterates over each \texttt{.scad} file
\item
  \texttt{\%\%f} is the loop variable (use \texttt{\%f} in interactive CMD, \texttt{\%\%f} in batch files)
\item
  Inside the parentheses, do something with each file
\end{itemize}

\paragraph*{Loop with a Counter}\label{docs__pandoc__latex__src__cmd_foundation__cmd_6_advanced_techniques__cmd_6_advanced_techniques.md__loop-with-a-counter}

\textbf{Example: Do something 5 times}

\begin{lstlisting}[style=Alabaster, language=cmd]
@echo off
for /L %%i in (1,1,5) do (
    echo This is iteration number %%i
)

\end{lstlisting}

\textbf{What\textquotesingle s happening:}

\begin{itemize}
\tightlist
\item
  \texttt{for\ /L\ \%\%i\ in\ (start,step,end)} counts from start to end
\item
  \texttt{\%\%i} is the counter variable
\end{itemize}

\begin{center}\rule{0.5\linewidth}{0.5pt}\end{center}

\subsubsection*{Real-World Example - Batch Processing SCAD Files}\label{docs__pandoc__latex__src__cmd_foundation__cmd_6_advanced_techniques__cmd_6_advanced_techniques.md__real-world-example---batch-processing-scad-files}

\paragraph*{Scenario}\label{docs__pandoc__latex__src__cmd_foundation__cmd_6_advanced_techniques__cmd_6_advanced_techniques.md__scenario}

You have 10 OpenSCAD (.scad) files in a folder. You want to:

\begin{enumerate}
\tightlist
\item
  List them all
\item
  Check how many there are
\item
  For each one, verify it exists
\end{enumerate}

\paragraph*{The Script}\label{docs__pandoc__latex__src__cmd_foundation__cmd_6_advanced_techniques__cmd_6_advanced_techniques.md__the-script}

\begin{lstlisting}[style=Alabaster, language=cmd]
@echo off
set scadFolder=C:\Users\YourName\Documents\3D_Projects
set count=0

echo Processing SCAD files in: %scadFolder%
echo.

for %%f in (%scadFolder%\*.scad) do (
    if exist "%%f" (
        echo  Found: %%~nxf
        set /A count+=1
    ) else (
        echo  Missing: %%~nxf
    )
)

echo.
echo Total files found: %count%
echo Batch processing complete!

\end{lstlisting}

\textbf{Breaking it down:}

\begin{itemize}
\tightlist
\item
  \texttt{set\ scadFolder=...} = where to look
\item
  \texttt{for\ \%\%f\ in\ (...\textbackslash{}*.scad)} = find all .scad files
\item
  \texttt{if\ exist\ "\%\%f"} = check if file exists
\item
  \texttt{\%\%\textasciitilde{}nxf} = just the filename with extension (not the full path)
\item
  \texttt{set\ /A\ count+=1} = increment the counter
\end{itemize}

\paragraph*{Running the Script}\label{docs__pandoc__latex__src__cmd_foundation__cmd_6_advanced_techniques__cmd_6_advanced_techniques.md__running-the-script}

\begin{enumerate}
\item
  Save as \texttt{batch-process.bat}
\item
  Edit \texttt{scadFolder} to match your real folder
\item
  Run it:

  \begin{lstlisting}[style=Alabaster, language=cmd]
  batch-process.bat

  \end{lstlisting}
\end{enumerate}

\textbf{Screen reader output:}

\begin{lstlisting}[style=Alabaster]
Processing SCAD files in: C:\Users\YourName\Documents\3D_Projects

 Found: model1.scad
 Found: model2.scad
 Found: model3.scad
[... more files ...]

Total files found: 10
Batch processing complete!

\end{lstlisting}

\begin{center}\rule{0.5\linewidth}{0.5pt}\end{center}

\subsubsection*{Error Handling}\label{docs__pandoc__latex__src__cmd_foundation__cmd_6_advanced_techniques__cmd_6_advanced_techniques.md__error-handling}

\paragraph*{Checking Error Levels}\label{docs__pandoc__latex__src__cmd_foundation__cmd_6_advanced_techniques__cmd_6_advanced_techniques.md__checking-error-levels}

What if something goes wrong? Use \texttt{errorlevel} checks:

\textbf{Example:}

\begin{lstlisting}[style=Alabaster, language=cmd]
@echo off
type C:\nonexistent\path\file.txt
if %errorlevel% neq 0 (
    echo Error: Could not read file
    echo Errorlevel: %errorlevel%
)

\end{lstlisting}

\textbf{What\textquotesingle s happening:}

\begin{itemize}
\tightlist
\item
  After any command, \texttt{\%errorlevel\%} holds \texttt{0} for success or non-zero for failure
\item
  \texttt{if\ \%errorlevel\%\ neq\ 0} checks for failure
\item
  Handle the error gracefully
\end{itemize}

\textbf{Screen reader advantage:} Errors are announced clearly instead of crashing silently.

\paragraph*{Validating Input}\label{docs__pandoc__latex__src__cmd_foundation__cmd_6_advanced_techniques__cmd_6_advanced_techniques.md__validating-input}

\textbf{Example: Make sure a folder exists before processing}

\begin{lstlisting}[style=Alabaster, language=cmd]
@echo off
set folderPath=C:\Users\YourName\Documents

if not exist "%folderPath%\" (
    echo Error: Folder does not exist: %folderPath%
    goto :eof
)

echo Processing folder: %folderPath%
dir /B "%folderPath%"

\end{lstlisting}

\textbf{What\textquotesingle s happening:}

\begin{itemize}
\tightlist
\item
  \texttt{if\ not\ exist\ "\%folderPath\%\textbackslash{}"} checks if folder exists (trailing \texttt{\textbackslash{}} ensures it\textquotesingle s a folder)
\item
  \texttt{goto\ :eof} exits the script early if error
\end{itemize}

\begin{center}\rule{0.5\linewidth}{0.5pt}\end{center}

\subsubsection*{Debugging Batch Scripts}\label{docs__pandoc__latex__src__cmd_foundation__cmd_6_advanced_techniques__cmd_6_advanced_techniques.md__debugging-batch-scripts}

\paragraph*{Common Errors and Solutions}\label{docs__pandoc__latex__src__cmd_foundation__cmd_6_advanced_techniques__cmd_6_advanced_techniques.md__common-errors-and-solutions}

\subparagraph*{Error 1: "Command not found"}\label{docs__pandoc__latex__src__cmd_foundation__cmd_6_advanced_techniques__cmd_6_advanced_techniques.md__error-1-command-not-found}

\textbf{Cause:} Typo in command name or program not in PATH

\textbf{Fix:} Check spelling and verify PATH

\begin{lstlisting}[style=Alabaster, language=cmd]
:: Wrong:
dri /B

:: Correct:
dir /B

\end{lstlisting}

\subparagraph*{Error 2: "Variable is empty"}\label{docs__pandoc__latex__src__cmd_foundation__cmd_6_advanced_techniques__cmd_6_advanced_techniques.md__error-2-variable-is-empty}

\textbf{Cause:} Variable was never set or has a typo

\textbf{Fix:} Make sure variable is set before using it

\begin{lstlisting}[style=Alabaster, language=cmd]
set myvar=hello
echo %myvar%

\end{lstlisting}

\subparagraph*{Error 3: "The system cannot find the path specified"}\label{docs__pandoc__latex__src__cmd_foundation__cmd_6_advanced_techniques__cmd_6_advanced_techniques.md__error-3-the-system-cannot-find-the-path-specified}

\textbf{Cause:} Wrong folder path

\textbf{Fix:} Verify path exists

\begin{lstlisting}[style=Alabaster, language=cmd]
:: Check if path exists:
if exist "C:\Users\YourName\Documents\" echo Path exists

\end{lstlisting}

\subparagraph*{Error 4: "Access denied"}\label{docs__pandoc__latex__src__cmd_foundation__cmd_6_advanced_techniques__cmd_6_advanced_techniques.md__error-4-access-denied}

\textbf{Cause:} Don\textquotesingle t have permission

\textbf{Fix:} Run CMD as administrator

\begin{itemize}
\tightlist
\item
  Right-click Command Prompt -\textgreater{} Run as administrator
\end{itemize}

\paragraph*{Debugging Technique: Trace Output}\label{docs__pandoc__latex__src__cmd_foundation__cmd_6_advanced_techniques__cmd_6_advanced_techniques.md__debugging-technique-trace-output}

Add \texttt{echo} statements to track what\textquotesingle s happening:

\begin{lstlisting}[style=Alabaster, language=cmd]
@echo off
set path_var=C:\Users\YourName\Documents
echo Starting script. Path is: %path_var%

for %%f in (%path_var%\*) do (
    echo Processing: %%f
    :: Do something with %%f
    echo Done with: %%f
)

echo Script complete

\end{lstlisting}

Your screen reader will announce each step, so you know where errors happen.

\begin{center}\rule{0.5\linewidth}{0.5pt}\end{center}

\subsubsection*{Creating Professional Workflows}\label{docs__pandoc__latex__src__cmd_foundation__cmd_6_advanced_techniques__cmd_6_advanced_techniques.md__creating-professional-workflows}

\paragraph*{Example 1: Automated Project Setup}\label{docs__pandoc__latex__src__cmd_foundation__cmd_6_advanced_techniques__cmd_6_advanced_techniques.md__example-1-automated-project-setup}

\textbf{Scenario:} You start a new 3D printing project regularly. Instead of creating folders manually:

\begin{lstlisting}[style=Alabaster, language=cmd]
@echo off
set /p projectName=Enter project name: 
set baseFolder=C:\Users\YourName\Documents\3D_Projects
set projectFolder=%baseFolder%\%projectName%

:: Create folder structure
mkdir "%projectFolder%"
mkdir "%projectFolder%\designs"
mkdir "%projectFolder%\output"
mkdir "%projectFolder%\notes"

:: Create a README
echo # %projectName% > "%projectFolder%\README.txt"
echo. >> "%projectFolder%\README.txt"
echo Created: %date% >> "%projectFolder%\README.txt"
echo. >> "%projectFolder%\README.txt"
echo ## Designs >> "%projectFolder%\README.txt"
echo All .scad files go here. >> "%projectFolder%\README.txt"
echo. >> "%projectFolder%\README.txt"
echo ## Output >> "%projectFolder%\README.txt"
echo STL and other exports go here. >> "%projectFolder%\README.txt"

echo Project setup complete: %projectFolder%

\end{lstlisting}

\textbf{What it does:}

\begin{itemize}
\tightlist
\item
  Prompts for a project name
\item
  Creates folder structure for a new project
\item
  Sets up subfolders for designs, output, notes
\item
  Creates a README file automatically
\end{itemize}

\paragraph*{Example 2: Batch File Verification}\label{docs__pandoc__latex__src__cmd_foundation__cmd_6_advanced_techniques__cmd_6_advanced_techniques.md__example-2-batch-file-verification}

\textbf{Scenario:} Before processing, verify all required files exist:

\begin{lstlisting}[style=Alabaster, language=cmd]
@echo off
set projectFolder=C:\Users\YourName\Documents\3D_Projects\MyKeychain
set allGood=1

for %%i in (README.txt designs output notes) do (
    if exist "%projectFolder%\%%i" (
        echo  Found: %%i
    ) else (
        echo  Missing: %%i
        set allGood=0
    )
)

if %allGood%==1 (
    echo All checks passed!
) else (
    echo Some files are missing!
)

\end{lstlisting}

\begin{center}\rule{0.5\linewidth}{0.5pt}\end{center}

\subsubsection*{Screen Reader Tips for Batch Scripts}\label{docs__pandoc__latex__src__cmd_foundation__cmd_6_advanced_techniques__cmd_6_advanced_techniques.md__screen-reader-tips-for-batch-scripts}

\paragraph*{Making Script Output Readable}\label{docs__pandoc__latex__src__cmd_foundation__cmd_6_advanced_techniques__cmd_6_advanced_techniques.md__making-script-output-readable}

\textbf{Problem:} Script runs but output scrolls too fast or is hard to follow

\textbf{Solution 1: Save to file}

\begin{lstlisting}[style=Alabaster, language=cmd]
my-script.bat > output.txt
notepad.exe output.txt

\end{lstlisting}

\textbf{Solution 2: Use echo with clear sections}

\begin{lstlisting}[style=Alabaster, language=cmd]
echo ========== STARTING ==========
echo.
:: ... script ...
echo.
echo ========== COMPLETE ==========

\end{lstlisting}

\textbf{Solution 3: Pause between major sections}

\begin{lstlisting}[style=Alabaster, language=cmd]
echo Pausing... Press any key to continue
pause > nul

\end{lstlisting}

Your screen reader will announce the pause, give you time to read output.

\paragraph*{Announcing Progress}\label{docs__pandoc__latex__src__cmd_foundation__cmd_6_advanced_techniques__cmd_6_advanced_techniques.md__announcing-progress}

\textbf{For long-running scripts:}

\begin{lstlisting}[style=Alabaster, language=cmd]
@echo off
set count=0
for %%f in (*.scad) do (
    set /A count+=1
    echo Processing file %%f
)
echo All %count% files processed!

\end{lstlisting}

\begin{center}\rule{0.5\linewidth}{0.5pt}\end{center}

\subsubsection*{Practice Exercises}\label{docs__pandoc__latex__src__cmd_foundation__cmd_6_advanced_techniques__cmd_6_advanced_techniques.md__practice-exercises}

\paragraph*{Exercise 1: Your First Batch Script}\label{docs__pandoc__latex__src__cmd_foundation__cmd_6_advanced_techniques__cmd_6_advanced_techniques.md__exercise-1-your-first-batch-script}

\textbf{Goal:} Create and run a simple batch script

\textbf{Steps:}

\begin{enumerate}
\item
  Create file: \texttt{notepad.exe\ hello.bat}
\item
  Type:

  \begin{lstlisting}[style=Alabaster, language=cmd]
  @echo off
  echo Hello from my first CMD batch script!
  cd
  dir /B

  \end{lstlisting}
\item
  Save and run: \texttt{hello.bat}
\end{enumerate}

\textbf{Checkpoint:} You should see output for each command.

\paragraph*{Exercise 2: Script with a Variable}\label{docs__pandoc__latex__src__cmd_foundation__cmd_6_advanced_techniques__cmd_6_advanced_techniques.md__exercise-2-script-with-a-variable}

\textbf{Goal:} Use a variable to make the script flexible

\textbf{Steps:}

\begin{enumerate}
\item
  Create file: \texttt{notepad.exe\ smart-listing.bat}
\item
  Type:

  \begin{lstlisting}[style=Alabaster, language=cmd]
  @echo off
  set targetFolder=C:\Users\YourName\Documents
  echo Listing contents of: %targetFolder%
  dir /B "%targetFolder%"

  \end{lstlisting}
\item
  Edit \texttt{targetFolder} to a real folder on your computer
\item
  Run: \texttt{smart-listing.bat}
\end{enumerate}

\textbf{Checkpoint:} You should see listing of that specific folder.

\paragraph*{Exercise 3: Subroutine}\label{docs__pandoc__latex__src__cmd_foundation__cmd_6_advanced_techniques__cmd_6_advanced_techniques.md__exercise-3-subroutine}

\textbf{Goal:} Create a reusable subroutine

\textbf{Steps:}

\begin{enumerate}
\item
  Create file: \texttt{notepad.exe\ navigate.bat}
\item
  Type:

  \begin{lstlisting}[style=Alabaster, language=cmd]
  @echo off
  call :GoTo "C:\Users\YourName\Documents"
  call :GoTo "C:\Users\YourName\Downloads"
  goto :eof

  :GoTo
  if exist "%~1\" (
      cd /D "%~1"
      echo Now in:
      cd
      echo Contents:
      dir /B
  ) else (
      echo Path does not exist: %~1
  )
  goto :eof

  \end{lstlisting}
\item
  Run: \texttt{navigate.bat}
\end{enumerate}

\textbf{Checkpoint:} Both subroutine calls should work, showing contents of each folder.

\paragraph*{Exercise 4: Loop}\label{docs__pandoc__latex__src__cmd_foundation__cmd_6_advanced_techniques__cmd_6_advanced_techniques.md__exercise-4-loop}

\textbf{Goal:} Use a loop to repeat an action

\textbf{Steps:}

\begin{enumerate}
\item
  Create file: \texttt{notepad.exe\ repeat.bat}
\item
  Type:

  \begin{lstlisting}[style=Alabaster, language=cmd]
  @echo off
  echo Demonstrating a loop:
  for /L %%i in (1,1,5) do (
      echo Iteration %%i: Hello!
  )
  echo Loop complete!

  \end{lstlisting}
\item
  Run: \texttt{repeat.bat}
\end{enumerate}

\textbf{Checkpoint:} Should print "Iteration 1" through "Iteration 5".

\paragraph*{Exercise 5: Real-World Script}\label{docs__pandoc__latex__src__cmd_foundation__cmd_6_advanced_techniques__cmd_6_advanced_techniques.md__exercise-5-real-world-script}

\textbf{Goal:} Create a useful script for a real task

\textbf{Steps:}

\begin{enumerate}
\item
  Create a folder: \texttt{mkdir\ C:\textbackslash{}Users\textbackslash{}YourName\textbackslash{}Documents\textbackslash{}TestFiles}
\item
  Create some test files:

  \begin{lstlisting}[style=Alabaster, language=cmd]
  echo test > C:\Users\YourName\Documents\TestFiles\file1.txt
  echo test > C:\Users\YourName\Documents\TestFiles\file2.txt
  echo test > C:\Users\YourName\Documents\TestFiles\file3.txt

  \end{lstlisting}
\item
  Create script: \texttt{notepad.exe\ report.bat}
\item
  Type:

  \begin{lstlisting}[style=Alabaster, language=cmd]
  @echo off
  set folder=C:\Users\YourName\Documents\TestFiles
  set count=0

  echo === FILE REPORT ===
  echo Folder: %folder%
  echo.
  echo Files:
  for %%f in ("%folder%\*") do (
      echo   - %%~nxf
      set /A count+=1
  )
  echo.
  echo Total: %count% files
  echo === END REPORT ===

  \end{lstlisting}
\item
  Run: \texttt{report.bat}
\end{enumerate}

\textbf{Checkpoint:} Should show report of all files in the test folder.

\begin{center}\rule{0.5\linewidth}{0.5pt}\end{center}

\subsubsection*{Quiz - Lesson CMD.6}\label{docs__pandoc__latex__src__cmd_foundation__cmd_6_advanced_techniques__cmd_6_advanced_techniques.md__quiz---lesson-cmd6}

\begin{enumerate}
\tightlist
\item
  What is a CMD batch script?
\item
  What file extension do CMD batch scripts use?
\item
  What is a variable in a batch script and how do you create one?
\item
  What is a subroutine (\texttt{:label}) and why would you use one?
\item
  How do you run a batch script?
\item
  What is a \texttt{for} loop and what does \texttt{for\ \%\%f\ in\ (*.scad)\ do} do?
\item
  What does \texttt{if\ exist} do?
\item
  How do you handle errors in a batch script?
\item
  When would you use \texttt{if\ \%errorlevel\%\ neq\ 0}?
\item
  What technique makes batch script output readable for screen readers?
\end{enumerate}

\begin{center}\rule{0.5\linewidth}{0.5pt}\end{center}

\subsubsection*{Extension Problems}\label{docs__pandoc__latex__src__cmd_foundation__cmd_6_advanced_techniques__cmd_6_advanced_techniques.md__extension-problems}

\begin{enumerate}
\tightlist
\item
  \textbf{Auto-Backup Script:} Create a batch script that copies all files from one folder to another, announcing progress
\item
  \textbf{File Counter:} Write a subroutine that counts files by extension (.txt, .scad, .stl, etc.)
\item
  \textbf{Folder Cleaner:} Batch script that deletes files older than 30 days (with user confirmation)
\item
  \textbf{Project Template:} Subroutine that creates a complete project folder structure with all needed files
\item
  \textbf{Batch Rename:} Script that renames all files in a folder according to a pattern
\item
  \textbf{Log Generator:} Create a script that records what it does to a log file for later review
\item
  \textbf{Scheduled Task:} Set up a batch script to run automatically every day at a specific time
\item
  \textbf{File Verifier:} Check that all SCAD files in a folder have corresponding STL exports
\item
  \textbf{Report Generator:} Create a summary report of all projects in a folder
\item
  \textbf{Error Tracker:} Script that lists all commands that had errors and logs them with timestamps
\end{enumerate}

\begin{center}\rule{0.5\linewidth}{0.5pt}\end{center}

\subsubsection*{Important Notes}\label{docs__pandoc__latex__src__cmd_foundation__cmd_6_advanced_techniques__cmd_6_advanced_techniques.md__important-notes}

\begin{itemize}
\tightlist
\item
  \textbf{Always test scripts on small sets of files first} before running them on important data
\item
  \textbf{Save your work regularly} --- use version naming if possible
\item
  \textbf{Test error handling} --- make sure errors don\textquotesingle t crash silently
\item
  \textbf{Document your scripts} --- use \texttt{::} comments so you remember what each part does
\item
  \textbf{Backup before batch operations} --- if something goes wrong, you have the original
\end{itemize}

\begin{center}\rule{0.5\linewidth}{0.5pt}\end{center}

\subsubsection*{References}\label{docs__pandoc__latex__src__cmd_foundation__cmd_6_advanced_techniques__cmd_6_advanced_techniques.md__references}

\begin{itemize}
\tightlist
\item
  \textbf{Microsoft CMD Batch Scripting Guide:} \url{https://example.com}
\item
  \textbf{FOR Loop Documentation:} \url{https://example.com}
\item
  \textbf{IF Statement Reference:} \url{https://example.com}
\item
  \textbf{SET Variable Reference:} \url{https://example.com}
\end{itemize}

\begin{center}\rule{0.5\linewidth}{0.5pt}\end{center}

\textbf{Next Steps:} After mastering this lesson, explore advanced batch scripting, scheduled tasks, and 3D printing integration in the main curriculum.

\subsection{CMD Unit Test - Comprehensive Assessment}\label{docs__pandoc__latex__src__cmd_foundation__cmd_unit_test__cmd_unit_test.md__cmd-unit-test---comprehensive-assessment}

Estimated time: 60-90 minutes

\subsubsection*{Key Learning Outcomes Assessed}\label{docs__pandoc__latex__src__cmd_foundation__cmd_unit_test__cmd_unit_test.md__key-learning-outcomes-assessed}

By completing this unit test, you will demonstrate:

\begin{enumerate}
\tightlist
\item
  Understanding of file system navigation and path concepts
\item
  Proficiency with file and folder manipulation commands
\item
  Ability to redirect and pipe command output
\item
  Knowledge of environment variables and aliases
\item
  Screen-reader accessibility best practices in terminal environments
\item
  Problem-solving and command chaining skills
\end{enumerate}

\subsubsection*{Target Audience}\label{docs__pandoc__latex__src__cmd_foundation__cmd_unit_test__cmd_unit_test.md__target-audience}

Users who have completed CMD-0 through CMD-6 and need to demonstrate mastery of Command Prompt fundamentals.

\subsubsection*{Instructions}\label{docs__pandoc__latex__src__cmd_foundation__cmd_unit_test__cmd_unit_test.md__instructions}

Complete all sections below. For multiple choice, select the best answer. For short answers, write one to two sentences. For hands-on tasks, capture evidence (screenshots or output files) and submit alongside your answers.

\begin{center}\rule{0.5\linewidth}{0.5pt}\end{center}

\subsubsection*{Part A: Multiple Choice Questions (20 questions)}\label{docs__pandoc__latex__src__cmd_foundation__cmd_unit_test__cmd_unit_test.md__part-a-multiple-choice-questions-20-questions}

Select the best answer for each question. Each question is worth 1 point.

\begin{enumerate}
\item
  What is the primary purpose of the \texttt{PATH} environment variable?

  \begin{itemize}
  \tightlist
  \item
    A) Store your home directory location
  \item
    B) Tell the shell where to find executable programs
  \item
    C) Configure the visual appearance of the terminal
  \item
    D) Store the current working directory name
  \end{itemize}
\item
  Which command shows your current working directory in CMD?

  \begin{itemize}
  \tightlist
  \item
    A) \texttt{dir\ /B}
  \item
    B) \texttt{cd} (with no arguments)
  \item
    C) \texttt{pwd}
  \item
    D) \texttt{whoami}
  \end{itemize}
\item
  What does \texttt{\%USERPROFILE\%} represent in CMD?

  \begin{itemize}
  \tightlist
  \item
    A) The root directory
  \item
    B) The current directory
  \item
    C) The parent directory
  \item
    D) Your home directory
  \end{itemize}
\item
  How do you list only file names (not full details) in a screen-reader-friendly way?

  \begin{itemize}
  \tightlist
  \item
    A) \texttt{dir}
  \item
    B) \texttt{dir\ /B}
  \item
    C) \texttt{dir\ /L}
  \item
    D) \texttt{type\ /B}
  \end{itemize}
\item
  Which command creates a new empty file in CMD?

  \begin{itemize}
  \tightlist
  \item
    A) \texttt{mkdir\ filename}
  \item
    B) \texttt{echo.\ \textgreater{}\ filename}
  \item
    C) \texttt{touch\ filename}
  \item
    D) \texttt{new\ filename}
  \end{itemize}
\item
  What is the difference between \texttt{\textgreater{}} and \texttt{\textgreater{}\textgreater{}}?

  \begin{itemize}
  \tightlist
  \item
    A) \texttt{\textgreater{}} redirects to file, \texttt{\textgreater{}\textgreater{}} displays on screen
  \item
    B) \texttt{\textgreater{}} overwrites a file, \texttt{\textgreater{}\textgreater{}} appends to a file
  \item
    C) They do the same thing
  \item
    D) \texttt{\textgreater{}} is for text, \texttt{\textgreater{}\textgreater{}} is for binary
  \end{itemize}
\item
  What does the pipe operator \texttt{\textbar{}} do?

  \begin{itemize}
  \tightlist
  \item
    A) Creates a folder
  \item
    B) Sends the output of one command to the input of another
  \item
    C) Deletes files matching a pattern
  \item
    D) Lists all processes
  \end{itemize}
\item
  Which command copies a file in CMD?

  \begin{itemize}
  \tightlist
  \item
    A) \texttt{move}
  \item
    B) \texttt{del}
  \item
    C) \texttt{copy}
  \item
    D) \texttt{cd}
  \end{itemize}
\item
  How do you rename a file from \texttt{oldname.txt} to \texttt{newname.txt} in CMD?

  \begin{itemize}
  \tightlist
  \item
    A) \texttt{copy\ oldname.txt\ newname.txt}
  \item
    B) \texttt{move\ oldname.txt\ newname.txt}
  \item
    C) \texttt{rename\ oldname.txt\ newname.txt}
  \item
    D) Both B and C are correct
  \end{itemize}
\item
  What is the purpose of \texttt{find} in CMD piping?

  \begin{itemize}
  \tightlist
  \item
    A) Find files in a directory
  \item
    B) Search for text patterns within output
  \item
    C) Find a string to copy to clipboard
  \item
    D) Find which shell to use
  \end{itemize}
\item
  Which key allows you to autocomplete a path in CMD?

  \begin{itemize}
  \tightlist
  \item
    A) \texttt{Ctrl\ +\ A}
  \item
    B) \texttt{Ctrl\ +\ E}
  \item
    C) \texttt{Tab}
  \item
    D) \texttt{Space}
  \end{itemize}
\item
  How do you copy text to the Windows clipboard from CMD?

  \begin{itemize}
  \tightlist
  \item
    A) \texttt{type\ filename\ \textgreater{}\ clipboard}
  \item
    B) \texttt{type\ filename\ \textbar{}\ clip}
  \item
    C) \texttt{copy\ filename}
  \item
    D) \texttt{type\ filename\ \textbar{}\ paste}
  \end{itemize}
\item
  What does \texttt{where\ openscad} do?

  \begin{itemize}
  \tightlist
  \item
    A) Opens the OpenSCAD application
  \item
    B) Gets help about the openscad command
  \item
    C) Locates the full path of the openscad executable
  \item
    D) Lists all available commands
  \end{itemize}
\item
  Which wildcard matches any single character?

  \begin{itemize}
  \tightlist
  \item
    A) \texttt{*}
  \item
    B) \texttt{?}
  \item
    C) \texttt{\%}
  \item
    D) \texttt{\#}
  \end{itemize}
\item
  What is the purpose of the \texttt{start} command?

  \begin{itemize}
  \tightlist
  \item
    A) Run a script or executable, optionally in a new window
  \item
    B) Execute all commands in parallel
  \item
    C) Combine multiple commands
  \item
    D) Create an alias
  \end{itemize}
\item
  How do you create a temporary alias (macro) in CMD?

  \begin{itemize}
  \tightlist
  \item
    A) \texttt{alias\ preview=\textquotesingle{}openscad\textquotesingle{}}
  \item
    B) \texttt{doskey\ preview=openscad\ \$*}
  \item
    C) \texttt{set\ preview=openscad}
  \item
    D) \texttt{macro\ preview\ openscad}
  \end{itemize}
\item
  How can doskey macros be made to persist across CMD sessions?

  \begin{itemize}
  \tightlist
  \item
    A) They are automatically saved
  \item
    B) By adding them to a startup batch script registered as an autorun
  \item
    C) By using the Windows Control Panel
  \item
    D) Doskey macros cannot be made permanent
  \end{itemize}
\item
  How do you abort a long-running command in CMD?

  \begin{itemize}
  \tightlist
  \item
    A) Press \texttt{Escape}
  \item
    B) Press \texttt{Ctrl\ +\ X}
  \item
    C) Press \texttt{Ctrl\ +\ C}
  \item
    D) Press \texttt{Alt\ +\ F4}
  \end{itemize}
\item
  What command shows the history of previously run commands in CMD?

  \begin{itemize}
  \tightlist
  \item
    A) \texttt{history}
  \item
    B) \texttt{doskey\ /history}
  \item
    C) \texttt{F7} (opens history popup)
  \item
    D) Both B and C are correct
  \end{itemize}
\item
  How do you make an environment variable permanent in CMD (for all future sessions)?

  \begin{itemize}
  \tightlist
  \item
    A) Use \texttt{set} in the terminal every time
  \item
    B) Use \texttt{setx} to write it to the registry
  \item
    C) Use the Windows Control Panel only
  \item
    D) Environment variables cannot be made permanent in CMD
  \end{itemize}
\end{enumerate}

\begin{center}\rule{0.5\linewidth}{0.5pt}\end{center}

\subsubsection*{Part B: Short Answer Questions (10 questions)}\label{docs__pandoc__latex__src__cmd_foundation__cmd_unit_test__cmd_unit_test.md__part-b-short-answer-questions-10-questions}

Answer each question in one to two sentences. Each question is worth 2 points.

\begin{enumerate}
\item
  Explain the difference between absolute and relative paths. Give one example of each.
\item
  Why is \texttt{dir\ /B} preferred over \texttt{dir} for screen reader users? Describe what flag you would add to list only files.
\item
  What is the purpose of redirecting output to a file, and give an example of when you would use \texttt{\textgreater{}} instead of \texttt{\textgreater{}\textgreater{}}?
\item
  Describe what would happen if you ran \texttt{rmdir\ /S\ /Q\ C:\textbackslash{}Documents\textbackslash{}my\_folder} and why this command should be used carefully.
\item
  How would you search for all files with a \texttt{.scad} extension in your current directory? Write the command.
\item
  Explain what happens when you pipe the output of \texttt{dir\ /B} into \texttt{clip}. What would you do next?
\item
  What is an environment variable, and give one example of how you might use it in CMD.
\item
  If a program is not in your \texttt{PATH}, what two methods could you use to run it from CMD?
\item
  Describe how you would open a file in Notepad and also add a line to it from CMD.
\item
  What is one strategy you would use if your screen reader stops announcing terminal output while using CMD?
\end{enumerate}

\begin{center}\rule{0.5\linewidth}{0.5pt}\end{center}

\subsubsection*{Part C: Hands-On Tasks (10 tasks)}\label{docs__pandoc__latex__src__cmd_foundation__cmd_unit_test__cmd_unit_test.md__part-c-hands-on-tasks-10-tasks}

Complete each task and capture evidence (screenshots, output files, or command transcripts). Each task is worth 3 points.

\paragraph*{Tasks 1-5: File System and Navigation}\label{docs__pandoc__latex__src__cmd_foundation__cmd_unit_test__cmd_unit_test.md__tasks-1-5-file-system-and-navigation}

\begin{enumerate}
\item
  Create a folder structure \texttt{\%USERPROFILE\%\textbackslash{}Documents\textbackslash{}CMD\_Assessment\textbackslash{}Projects} using a single command. Capture the \texttt{dir\ /B} output showing the creation.
\item
  Create five files named \texttt{project\_1.scad}, \texttt{project\_2.scad}, \texttt{project\_3.txt}, \texttt{notes\_1.txt}, and \texttt{notes\_2.txt} inside the \texttt{Projects} folder. Use wildcards to list only \texttt{.scad} files, then capture the output.
\item
  Copy the entire \texttt{Projects} folder to \texttt{Projects\_Backup} using \texttt{xcopy\ /E\ /I}. Capture the \texttt{dir\ /B} output showing both folders exist.
\item
  Move (rename) \texttt{project\_1.scad} to \texttt{project\_1\_final.scad}. Capture the \texttt{dir\ /B} output showing the renamed file.
\item
  Delete \texttt{notes\_1.txt} and \texttt{notes\_2.txt} using a single \texttt{del} command with wildcards. Capture the final \texttt{dir\ /B} output.
\end{enumerate}

\paragraph*{Tasks 6-10: Advanced Operations and Scripting}\label{docs__pandoc__latex__src__cmd_foundation__cmd_unit_test__cmd_unit_test.md__tasks-6-10-advanced-operations-and-scripting}

\begin{enumerate}
\item
  Create a file called \texttt{my\_data.txt} with at least four lines using \texttt{echo} and \texttt{\textgreater{}\textgreater{}}. Then read it with \texttt{type\ my\_data.txt} and capture the output.
\item
  Use \texttt{find} to search for a keyword (e.g., "project") in \texttt{my\_data.txt} and pipe the results to \texttt{clip}. Paste the results into Notepad and capture a screenshot.
\item
  List all files in the \texttt{Projects} folder and redirect the output to \texttt{projects\_list.txt}. Open it in Notepad and capture a screenshot of the file.
\item
  Create a temporary \texttt{doskey} alias called \texttt{mydir} that runs \texttt{dir\ /B}, test it, and capture the output. Then explain what would be required to make it persistent.
\item
  Run \texttt{help\ dir} and redirect the output to \texttt{help\_output.txt}. Open the file in Notepad and capture a screenshot showing at least the first page of help content.
\end{enumerate}

\begin{center}\rule{0.5\linewidth}{0.5pt}\end{center}

\subsubsection*{Grading Rubric}\label{docs__pandoc__latex__src__cmd_foundation__cmd_unit_test__cmd_unit_test.md__grading-rubric}

{\def\LTcaptype{none} % do not increment counter
\begin{longtable}[]{@{}llll@{}}
\toprule\noalign{}
Section & Questions & Points Each & Total \\
\midrule\noalign{}
\endhead
\bottomrule\noalign{}
\endlastfoot
Multiple Choice & 20 & 1 & 20 \\
Short Answer & 10 & 2 & 20 \\
Hands-On Tasks & 10 & 3 & 30 \\
\textbf{Total} & \textbf{40} & - & \textbf{70} \\
\end{longtable}
}

\textbf{Passing Score:} 49 points (70\%)

\begin{center}\rule{0.5\linewidth}{0.5pt}\end{center}

\subsubsection*{Helpful Resources for Review}\label{docs__pandoc__latex__src__cmd_foundation__cmd_unit_test__cmd_unit_test.md__helpful-resources-for-review}

\begin{itemize}
\tightlist
\item
  \href{https://learn.microsoft.com/en-us/windows-server/administration/windows-commands/windows-commands-glossary}{CMD Command Reference}
\item
  \href{https://learn.microsoft.com/en-us/windows-server/administration/windows-commands/cd}{Navigation and File System}
\item
  \href{https://learn.microsoft.com/en-us/windows-server/administration/windows-commands/find}{Using Pipes and Filtering}
\item
  \href{https://learn.microsoft.com/en-us/windows-server/administration/windows-commands/doskey}{DOSKEY and Aliases}
\item
  \href{https://learn.microsoft.com/en-us/windows-server/administration/windows-commands/}{Screen Reader Accessibility Tips}
\end{itemize}

\begin{center}\rule{0.5\linewidth}{0.5pt}\end{center}

\subsubsection*{Submission Checklist}\label{docs__pandoc__latex__src__cmd_foundation__cmd_unit_test__cmd_unit_test.md__submission-checklist}

\begin{itemize}
\tightlist
\item[$\square$]
  All 20 multiple choice questions answered
\item[$\square$]
  All 10 short answer questions answered (1-2 sentences each)
\item[$\square$]
  All 10 hands-on tasks completed with evidence captured
\item[$\square$]
  Files/screenshots organized and labeled clearly
\item[$\square$]
  Submission includes this checklist
\end{itemize}

\subsection{Screen Reader Accessibility Guide for Windows Command Prompt (CMD)}\label{docs__pandoc__latex__src__cmd_foundation__screen_reader_accessibility_guide__screen_reader_accessibility_guide.md__screen-reader-accessibility-guide-for-windows-command-prompt-cmd}

Target Users: NVDA, JAWS, and other screen reader users\\
Last Updated: 2026

This guide supports the CMD Foundation curriculum and helps screen reader users navigate and work efficiently with the Windows Command Prompt.

\subsubsection*{Table of Contents}\label{docs__pandoc__latex__src__cmd_foundation__screen_reader_accessibility_guide__screen_reader_accessibility_guide.md__table-of-contents}

\begin{enumerate}
\tightlist
\item
  \hyperref[docs__pandoc__latex__src__cmd_foundation__screen_reader_accessibility_guide__screen_reader_accessibility_guide.md__getting-started]{Getting Started with Screen Readers}
\item
  \hyperref[docs__pandoc__latex__src__cmd_foundation__screen_reader_accessibility_guide__screen_reader_accessibility_guide.md__nvda-tips]{NVDA-Specific Tips}
\item
  \hyperref[docs__pandoc__latex__src__cmd_foundation__screen_reader_accessibility_guide__screen_reader_accessibility_guide.md__jaws-tips]{JAWS-Specific Tips}
\item
  \hyperref[docs__pandoc__latex__src__cmd_foundation__screen_reader_accessibility_guide__screen_reader_accessibility_guide.md__general-terminal]{General Terminal Accessibility}
\item
  \hyperref[docs__pandoc__latex__src__cmd_foundation__screen_reader_accessibility_guide__screen_reader_accessibility_guide.md__long-output]{Working with Long Output}
\item
  \hyperref[docs__pandoc__latex__src__cmd_foundation__screen_reader_accessibility_guide__screen_reader_accessibility_guide.md__shortcuts]{Keyboard Shortcuts Reference}
\item
  \hyperref[docs__pandoc__latex__src__cmd_foundation__screen_reader_accessibility_guide__screen_reader_accessibility_guide.md__troubleshooting]{Troubleshooting}
\end{enumerate}

\subsubsection*{Getting Started with Screen Readers}\label{docs__pandoc__latex__src__cmd_foundation__screen_reader_accessibility_guide__screen_reader_accessibility_guide.md__getting-started}

\paragraph*{Which Screen Reader Should I Use?}\label{docs__pandoc__latex__src__cmd_foundation__screen_reader_accessibility_guide__screen_reader_accessibility_guide.md__which-screen-reader-should-i-use}

Both NVDA and JAWS work with CMD; NVDA is free and often easiest to start with. JAWS provides advanced features for power users. Dolphin SuperNova and Windows Narrator are also options:

\begin{itemize}
\tightlist
\item
  Dolphin SuperNova: commercial speech, braille, and magnification (use vendor docs for key mappings).
\item
  Windows Narrator: built into Windows for quick, no-install access.
\end{itemize}

\paragraph*{Before You Start}\label{docs__pandoc__latex__src__cmd_foundation__screen_reader_accessibility_guide__screen_reader_accessibility_guide.md__before-you-start}

\begin{enumerate}
\tightlist
\item
  Start your screen reader before opening CMD.
\item
  Open Command Prompt and listen for the window title and prompt.
\item
  If silent, press Alt+Tab to find the window.
\end{enumerate}

\paragraph*{What is CMD?}\label{docs__pandoc__latex__src__cmd_foundation__screen_reader_accessibility_guide__screen_reader_accessibility_guide.md__what-is-cmd}

CMD (Command Prompt) is the classic Windows shell. Common commands include \texttt{dir}, \texttt{type}, and \texttt{more}. Paths use backslashes (e.g., \texttt{C:\textbackslash{}Users\textbackslash{}You}).

\subsubsection*{NVDA-Specific Tips}\label{docs__pandoc__latex__src__cmd_foundation__screen_reader_accessibility_guide__screen_reader_accessibility_guide.md__nvda-tips}

NVDA is available from \url{https://www.nvaccess.org/}

\paragraph*{Dolphin SuperNova}\label{docs__pandoc__latex__src__cmd_foundation__screen_reader_accessibility_guide__screen_reader_accessibility_guide.md__dolphin-supernova}

Dolphin SuperNova: \url{https://yourdolphin.com/supernova/} --- commercial screen reader and magnifier; check institutional licensing.

\paragraph*{Windows Narrator}\label{docs__pandoc__latex__src__cmd_foundation__screen_reader_accessibility_guide__screen_reader_accessibility_guide.md__windows-narrator}

Windows Narrator: \url{https://support.microsoft.com/narrator} --- built-in, simpler command set; useful when third-party readers are unavailable.

\paragraph*{Key Commands for CMD}\label{docs__pandoc__latex__src__cmd_foundation__screen_reader_accessibility_guide__screen_reader_accessibility_guide.md__key-commands-for-cmd}

{\def\LTcaptype{none} % do not increment counter
\begin{longtable}[]{@{}
  >{\raggedright\arraybackslash}p{(\linewidth - 2\tabcolsep) * \real{0.2361}}
  >{\raggedright\arraybackslash}p{(\linewidth - 2\tabcolsep) * \real{0.7639}}@{}}
\toprule\noalign{}
\begin{minipage}[b]{\linewidth}\raggedright
Command
\end{minipage} & \begin{minipage}[b]{\linewidth}\raggedright
What It Does
\end{minipage} \\
\midrule\noalign{}
\endhead
\bottomrule\noalign{}
\endlastfoot
NVDA+Home & Read the current line (your command or output) \\
NVDA+Down Arrow & Read from cursor to end of screen \\
NVDA+Up Arrow & Read from top to cursor \\
NVDA+Page Down & Read next page \\
NVDA+Page Up & Read previous page \\
NVDA+F7 & Open the Review Mode viewer (can scroll through text) \\
\end{longtable}
}

\paragraph*{Example: Reading Long Output}\label{docs__pandoc__latex__src__cmd_foundation__screen_reader_accessibility_guide__screen_reader_accessibility_guide.md__example-reading-long-output}

If \texttt{dir} produces many lines, redirect to a file and open it in Notepad:

\begin{lstlisting}[style=Alabaster, language=cmd]
dir /b > list.txt
notepad list.txt

\end{lstlisting}

\texttt{dir\ /b} shows one item per line (screen reader friendly).

\subsubsection*{JAWS-Specific Tips}\label{docs__pandoc__latex__src__cmd_foundation__screen_reader_accessibility_guide__screen_reader_accessibility_guide.md__jaws-tips}

JAWS is available from \url{https://www.freedomscientific.com/}

\paragraph*{Key Commands for CMD}\label{docs__pandoc__latex__src__cmd_foundation__screen_reader_accessibility_guide__screen_reader_accessibility_guide.md__key-commands-for-cmd-1}

{\def\LTcaptype{none} % do not increment counter
\begin{longtable}[]{@{}ll@{}}
\toprule\noalign{}
Command & What It Does \\
\midrule\noalign{}
\endhead
\bottomrule\noalign{}
\endlastfoot
Insert+Down Arrow & Read line by line downward \\
Insert+Up Arrow & Read line by line upward \\
Insert+Page Down & Read next page of text \\
Insert+Page Up & Read previous page of text \\
\end{longtable}
}

\paragraph*{Example: Reading Long Output}\label{docs__pandoc__latex__src__cmd_foundation__screen_reader_accessibility_guide__screen_reader_accessibility_guide.md__example-reading-long-output-1}

\begin{enumerate}
\tightlist
\item
  Redirect: \texttt{dir\ /b\ \textgreater{}\ list.txt}
\item
  Open Notepad: \texttt{notepad\ list.txt}
\item
  Use Insert+Ctrl+Down to read full contents.
\end{enumerate}

\subsubsection*{General Terminal Accessibility}\label{docs__pandoc__latex__src__cmd_foundation__screen_reader_accessibility_guide__screen_reader_accessibility_guide.md__general-terminal}

\paragraph*{Understanding the CMD Layout}\label{docs__pandoc__latex__src__cmd_foundation__screen_reader_accessibility_guide__screen_reader_accessibility_guide.md__understanding-the-cmd-layout}

The Command Prompt window shows a title bar, a content area, and the prompt (e.g., \texttt{C:\textbackslash{}Users\textbackslash{}YourName\textgreater{}}).

\paragraph*{The CMD Prompt}\label{docs__pandoc__latex__src__cmd_foundation__screen_reader_accessibility_guide__screen_reader_accessibility_guide.md__the-cmd-prompt}

Example:

\begin{lstlisting}[style=Alabaster]
C:\Users\YourName>

\end{lstlisting}

\paragraph*{Navigation Sequence}\label{docs__pandoc__latex__src__cmd_foundation__screen_reader_accessibility_guide__screen_reader_accessibility_guide.md__navigation-sequence}

\begin{enumerate}
\tightlist
\item
  Screen reader announces the title
\item
  Then it announces the prompt line
\item
  Anything above prompt is prior output
\end{enumerate}

\subsubsection*{Working with Long Output}\label{docs__pandoc__latex__src__cmd_foundation__screen_reader_accessibility_guide__screen_reader_accessibility_guide.md__long-output}

\paragraph*{Solution 1: Redirect to a File}\label{docs__pandoc__latex__src__cmd_foundation__screen_reader_accessibility_guide__screen_reader_accessibility_guide.md__solution-1-redirect-to-a-file}

\begin{lstlisting}[style=Alabaster, language=cmd]
dir /b > list.txt
notepad list.txt

\end{lstlisting}

\paragraph*{Solution 2: Use Pagination}\label{docs__pandoc__latex__src__cmd_foundation__screen_reader_accessibility_guide__screen_reader_accessibility_guide.md__solution-2-use-pagination}

\begin{lstlisting}[style=Alabaster, language=cmd]
type largefile.txt | more

\end{lstlisting}

Use Space for next page and Q to quit.

\paragraph*{Solution 3: Filter Output}\label{docs__pandoc__latex__src__cmd_foundation__screen_reader_accessibility_guide__screen_reader_accessibility_guide.md__solution-3-filter-output}

\begin{lstlisting}[style=Alabaster, language=cmd]
dir /b | findstr /R "\.scad$"

\end{lstlisting}

\paragraph*{Solution 4: Count Before Displaying}\label{docs__pandoc__latex__src__cmd_foundation__screen_reader_accessibility_guide__screen_reader_accessibility_guide.md__solution-4-count-before-displaying}

\begin{lstlisting}[style=Alabaster, language=cmd]
dir /b | find /v "" /c

\end{lstlisting}

\subsubsection*{Keyboard Shortcuts Reference}\label{docs__pandoc__latex__src__cmd_foundation__screen_reader_accessibility_guide__screen_reader_accessibility_guide.md__shortcuts}

{\def\LTcaptype{none} % do not increment counter
\begin{longtable}[]{@{}ll@{}}
\toprule\noalign{}
Key & Action \\
\midrule\noalign{}
\endhead
\bottomrule\noalign{}
\endlastfoot
Up Arrow & Show previous command \\
Down Arrow & Show next command \\
Tab & Auto-complete file/folder names \\
Home & Jump to start of line \\
End & Jump to end of line \\
Ctrl+C & Stop command \\
Enter & Run command \\
\end{longtable}
}

\subsubsection*{Troubleshooting}\label{docs__pandoc__latex__src__cmd_foundation__screen_reader_accessibility_guide__screen_reader_accessibility_guide.md__troubleshooting}

\paragraph*{Problem: "I Can\textquotesingle t Hear the Output"}\label{docs__pandoc__latex__src__cmd_foundation__screen_reader_accessibility_guide__screen_reader_accessibility_guide.md__problem-i-cant-hear-the-output}

\begin{enumerate}
\tightlist
\item
  Redirect to file and open in Notepad.
\item
  Use End to jump to the end of text.
\end{enumerate}

\paragraph*{Problem: "Tab Completion Isn\textquotesingle t Working"}\label{docs__pandoc__latex__src__cmd_foundation__screen_reader_accessibility_guide__screen_reader_accessibility_guide.md__problem-tab-completion-isnt-working}

\begin{enumerate}
\tightlist
\item
  Type at least one character before Tab.
\end{enumerate}

\paragraph*{Problem: "Command Not Found"}\label{docs__pandoc__latex__src__cmd_foundation__screen_reader_accessibility_guide__screen_reader_accessibility_guide.md__problem-command-not-found}

\begin{enumerate}
\tightlist
\item
  Use \texttt{where\ programname} to find installed programs.
\end{enumerate}

\subsubsection*{Pro Tips}\label{docs__pandoc__latex__src__cmd_foundation__screen_reader_accessibility_guide__screen_reader_accessibility_guide.md__pro-tips}

\begin{enumerate}
\tightlist
\item
  Use \texttt{dir\ /b} for one-per-line listings.
\item
  Create a personal notes file and open it in Notepad for quick reference.
\end{enumerate}

\subsubsection*{Recommended Workflow}\label{docs__pandoc__latex__src__cmd_foundation__screen_reader_accessibility_guide__screen_reader_accessibility_guide.md__recommended-workflow}

\begin{enumerate}
\tightlist
\item
  \texttt{cd} to the project folder
\item
  \texttt{dir\ /b} to list files
\item
  Redirect large output to files and open in Notepad
\end{enumerate}

\subsubsection*{Additional Resources}\label{docs__pandoc__latex__src__cmd_foundation__screen_reader_accessibility_guide__screen_reader_accessibility_guide.md__additional-resources}

\begin{itemize}
\tightlist
\item
  NVDA Documentation: \url{https://www.nvaccess.org/documentation/}
\item
  JAWS Documentation: \url{https://www.freedomscientific.com/support/}
\item
  Windows CMD Reference: \url{https://docs.microsoft.com/windows-server/administration/windows-commands/windows-commands}
\item
  NVDA Documentation: \url{https://www.nvaccess.org/documentation/}
\item
  JAWS Documentation: \url{https://www.freedomscientific.com/support/}
\item
  Dolphin SuperNova: \url{https://yourdolphin.com/supernova/}
\item
  Windows Narrator: \url{https://support.microsoft.com/narrator}
\end{itemize}

\section{Git Bash}\label{docs__pandoc__latex__src__gitbash_foundation__part_1.md__gitbash_foundation-part_1}

This section covers terminal fundamentals, screen reader accessibility, and command-line basics needed before diving into 3D design with OpenSCAD.

Time commitment: \textasciitilde{}10 hours\\
Skills gained: Terminal navigation, file operations, basic scripting, keyboard-only workflow mastery

Note

This curriculum uses Git Bash - a free Unix-style terminal for Windows that comes bundled with Git for Windows \url{https://git-scm.com/}. It can be installed at the provided link or elde via \texttt{winget} in the terminal by typing \texttt{winget\ install\ Git.Git}. All commands in this section use standard bash syntax, which also works on Linux and macOS.

\subsection{Git Bash Curriculum Overview}\label{docs__pandoc__latex__src__gitbash_foundation__gitbash_curriculum_overview__gitbash_curriculum_overview.md__gitbash_foundation_gitbash_curriculum_overview-gitbash_curriculum_overview}

Total Duration: 20-25 hours (for screen reader users)\\
Target Audience: Screen reader users, accessibility-first learners\\
Prerequisites: Basic Windows computer familiarity

\emph{Note: Time estimates reflect the additional time needed for screen reader navigation, text-to-speech processing, and careful keyboard-based workflows.}

\subsubsection*{What is Git Bash?}\label{docs__pandoc__latex__src__gitbash_foundation__gitbash_curriculum_overview__gitbash_curriculum_overview.md__what-is-git-bash}

Git Bash is a Unix/bash shell that runs natively on Windows. It provides the same powerful command-line tools as macOS and Linux while keeping you on Windows.

Why learn Git Bash?

\begin{itemize}
\tightlist
\item
  Cross-platform compatibility: Same commands work on Windows, macOS, and Linux
\item
  Professional standard: Developers worldwide use bash
\item
  Accessibility: Git Bash is 100\% screen reader compatible with NVDA and JAWS
\item
  Powerful: Access to Unix tools that Windows Command Prompt and PowerShell can\textquotesingle t match
\item
  3D printing workflows: Essential for OpenSCAD scripting and automation
\item
  Future-proof: Learning bash makes you adaptable to any computer system
\end{itemize}

\subsubsection*{Curriculum Structure}\label{docs__pandoc__latex__src__gitbash_foundation__gitbash_curriculum_overview__gitbash_curriculum_overview.md__curriculum-structure}

\paragraph*{Part 1 (Intro): Comparing Command Line Interfaces}\label{docs__pandoc__latex__src__gitbash_foundation__gitbash_curriculum_overview__gitbash_curriculum_overview.md__part-1-intro-comparing-command-line-interfaces}

Start here if you\textquotesingle re new to terminals:

\begin{itemize}
\tightlist
\item
  \href{https://github.com/mrhunsaker/VI_3DMake_OpenSCAD_Curriculum/GitBash_Foundation/GitBash_Curriculum_Overview/../Command_Line_Interface_Selection.md}{Command Line Interface Selection Guide} - Decision guide comparing PowerShell, CMD, and Git Bash
\item
  Learn which option fits your learning style
\item
  Understand when to use each tool
\end{itemize}

\subsubsection*{Core Curriculum (11 Lessons)}\label{docs__pandoc__latex__src__gitbash_foundation__gitbash_curriculum_overview__gitbash_curriculum_overview.md__core-curriculum-11-lessons}

The main curriculum teaches you to use Git Bash for real work.

\paragraph*{Foundation Level (Lessons Pre - 1)}\label{docs__pandoc__latex__src__gitbash_foundation__gitbash_curriculum_overview__gitbash_curriculum_overview.md__foundation-level-lessons-pre---1}

\subparagraph*{GitBash-Pre: Your First Terminal (1.5-2.5 hours)}\label{docs__pandoc__latex__src__gitbash_foundation__gitbash_curriculum_overview__gitbash_curriculum_overview.md__gitbash-pre-your-first-terminal-15-25-hours}

What you\textquotesingle ll learn:

\begin{itemize}
\tightlist
\item
  How to open and close Git Bash
\item
  Understanding the prompt and what it tells you
\item
  Your first commands: \texttt{pwd}, \texttt{ls}, \texttt{cd}
\item
  How to navigate folders safely
\item
  Screen reader accessibility features
\end{itemize}

Key Skills:

\begin{itemize}
\tightlist
\item
  Know your current location (\texttt{pwd})
\item
  List what\textquotesingle s around you (\texttt{ls})
\item
  Move to different folders (\texttt{cd})
\item
  Use Tab completion (Game-changer for screen reader users!)
\item
  Redirect output to files for easier reading
\end{itemize}

Hands-On Work:

\begin{itemize}
\tightlist
\item
  Open Git Bash 5 times from different methods
\item
  Navigate to 10 different locations
\item
  List contents in 5 different folders
\item
  Create your first test file
\item
  Read it back with a screen reader
\end{itemize}

\subparagraph*{GitBash-0: Getting Started - Layout, Paths, and the Shell (1-1.5 hours)}\label{docs__pandoc__latex__src__gitbash_foundation__gitbash_curriculum_overview__gitbash_curriculum_overview.md__gitbash-0-getting-started---layout-paths-and-the-shell-1-15-hours}

What you\textquotesingle ll learn:

\begin{itemize}
\tightlist
\item
  Unix-style paths vs Windows paths
\item
  Absolute vs relative paths
\item
  Understanding the prompt
\item
  Path shortcuts: \texttt{\textasciitilde{}}, \texttt{.}, \texttt{..}
\item
  Windows drive letters in bash (\texttt{/c/} for C:)
\end{itemize}

Key Skills:

\begin{itemize}
\tightlist
\item
  Convert Windows paths to Git Bash format
\item
  Navigate using both absolute and relative paths
\item
  Understand path structure: \texttt{/c/Users/Name/Documents}
\item
  Recognize shortcuts in paths
\end{itemize}

Hands-On Work:

\begin{itemize}
\tightlist
\item
  Navigate to 5 Windows locations using bash equivalents
\item
  Create a map of paths you use frequently
\item
  Practice relative paths in a deep folder structure
\item
  Document your project folder paths
\end{itemize}

\subparagraph*{GitBash-1: Navigation - Moving Around Your File System (1.5-2.5 hours)}\label{docs__pandoc__latex__src__gitbash_foundation__gitbash_curriculum_overview__gitbash_curriculum_overview.md__gitbash-1-navigation---moving-around-your-file-system-15-25-hours}

What you\textquotesingle ll learn:

\begin{itemize}
\tightlist
\item
  Mastering \texttt{cd} for folder navigation
\item
  Tab completion strategies
\item
  Going up levels with \texttt{..}
\item
  Creating folder structures
\item
  Using shortcuts efficiently
\end{itemize}

Key Skills:

\begin{itemize}
\tightlist
\item
  Navigate confidently to any location
\item
  Use Tab completion without looking
\item
  Create and organize folder structures
\item
  Jump between frequently-used locations
\end{itemize}

Hands-On Work:

\begin{itemize}
\tightlist
\item
  Create a 5-level folder structure and navigate it
\item
  Navigate 20+ different locations
\item
  Build a "favorites" list of important paths
\item
  Time yourself: navigate blindly using only commands
\end{itemize}

\paragraph*{Intermediate Level (Lessons 2 - 4)}\label{docs__pandoc__latex__src__gitbash_foundation__gitbash_curriculum_overview__gitbash_curriculum_overview.md__intermediate-level-lessons-2---4}

\subparagraph*{GitBash-2: File and Folder Manipulation (2-2.5 hours)}\label{docs__pandoc__latex__src__gitbash_foundation__gitbash_curriculum_overview__gitbash_curriculum_overview.md__gitbash-2-file-and-folder-manipulation-2-25-hours}

What you\textquotesingle ll learn:

\begin{itemize}
\tightlist
\item
  Creating files and folders: \texttt{mkdir}, \texttt{touch}, \texttt{echo\ \textgreater{}\ file}
\item
  Copying files: \texttt{cp}
\item
  Moving and renaming: \texttt{mv}
\item
  Deleting safely: \texttt{rm} (with caution!)
\item
  Listing with details: \texttt{ls\ -l}
\item
  Checking file sizes: \texttt{du}, \texttt{wc}
\end{itemize}

Key Skills:

\begin{itemize}
\tightlist
\item
  Create organized folder structures
\item
  Copy files to new locations
\item
  Rename files safely
\item
  Delete files and folders
\item
  Understand file properties and sizes
\end{itemize}

Hands-On Work:

\begin{itemize}
\tightlist
\item
  Create 20+ files and organize into folders
\item
  Copy entire folder structures
\item
  Rename 10 files at once
\item
  Practice safe deletion with confirmation
\item
  Build a file organization system for 3D printing projects
\end{itemize}

\subparagraph*{GitBash-3: Input, Output \& Piping (2-2.5 hours)}\label{docs__pandoc__latex__src__gitbash_foundation__gitbash_curriculum_overview__gitbash_curriculum_overview.md__gitbash-3-input-output--piping-2-25-hours}

What you\textquotesingle ll learn:

\begin{itemize}
\tightlist
\item
  Redirecting output: \texttt{\textgreater{}}, \texttt{\textgreater{}\textgreater{}}
\item
  Reading files: \texttt{cat}, \texttt{less}
\item
  Searching text: \texttt{grep}
\item
  Sorting data: \texttt{sort}
\item
  Counting lines: \texttt{wc}
\item
  Piping commands together: \texttt{\textbar{}}
\item
  Combining tools for powerful workflows
\end{itemize}

Key Skills:

\begin{itemize}
\tightlist
\item
  Save command output to files
\item
  Search through text efficiently
\item
  Sort and organize data
\item
  Combine multiple commands into workflows
\item
  Debug by redirecting output to files
\end{itemize}

Hands-On Work:

\begin{itemize}
\tightlist
\item
  Create data files and search through them
\item
  Sort lists of file names and information
\item
  Pipe 10+ different command combinations
\item
  Create scripts that redirect output to files
\item
  Build a file-processing workflow
\end{itemize}

\subparagraph*{GitBash-4: Environment Variables \& Aliases (1.5-2 hours)}\label{docs__pandoc__latex__src__gitbash_foundation__gitbash_curriculum_overview__gitbash_curriculum_overview.md__gitbash-4-environment-variables--aliases-15-2-hours}

What you\textquotesingle ll learn:

\begin{itemize}
\tightlist
\item
  Understanding environment variables: \texttt{echo\ \$PATH}, \texttt{echo\ \$HOME}
\item
  Setting variables: \texttt{export\ MYVAR="value"}
\item
  Creating aliases: \texttt{alias\ ll="ls\ -la"}
\item
  Editing \texttt{.bashrc} configuration file
\item
  Making changes permanent
\end{itemize}

Key Skills:

\begin{itemize}
\tightlist
\item
  Understand the \texttt{\$PATH} and why it matters
\item
  Create shortcuts for long commands
\item
  Set up your shell environment
\item
  Make your configuration permanent
\end{itemize}

Hands-On Work:

\begin{itemize}
\tightlist
\item
  View all your environment variables
\item
  Create 5 useful aliases
\item
  Modify your \texttt{.bashrc} file
\item
  Test variables from different locations
\item
  Build a custom shell environment
\end{itemize}

\paragraph*{Advanced Level (Lessons 5 - 6)}\label{docs__pandoc__latex__src__gitbash_foundation__gitbash_curriculum_overview__gitbash_curriculum_overview.md__advanced-level-lessons-5---6}

\subparagraph*{GitBash-5: Filling in the Gaps - Shell Profiles, History, and Debugging (2-2.5 hours)}\label{docs__pandoc__latex__src__gitbash_foundation__gitbash_curriculum_overview__gitbash_curriculum_overview.md__gitbash-5-filling-in-the-gaps---shell-profiles-history-and-debugging-2-25-hours}

What you\textquotesingle ll learn:

\begin{itemize}
\tightlist
\item
  Using command history: \texttt{history}, \texttt{!}, reverse search
\item
  Debugging commands with \texttt{set\ -x}
\item
  Understanding \texttt{.bashrc} vs \texttt{.bashprofile}
\item
  Viewing and modifying history
\item
  Finding and fixing mistakes
\end{itemize}

Key Skills:

\begin{itemize}
\tightlist
\item
  Reuse previous commands efficiently
\item
  Debug scripts and workflows
\item
  Configure shell startup behavior
\item
  Learn from your command history
\end{itemize}

Hands-On Work:

\begin{itemize}
\tightlist
\item
  Review your command history
\item
  Create a custom history system
\item
  Debug 5 failing commands
\item
  Set up a personalized \texttt{.bashrc}
\item
  Create aliases for common debugging tasks
\end{itemize}

\subparagraph*{GitBash-6: Advanced Terminal Techniques - Scripts, Functions \& Professional Workflows (2.5-3.5 hours)}\label{docs__pandoc__latex__src__gitbash_foundation__gitbash_curriculum_overview__gitbash_curriculum_overview.md__gitbash-6-advanced-terminal-techniques---scripts-functions--professional-workflows-25-35-hours}

What you\textquotesingle ll learn:

\begin{itemize}
\tightlist
\item
  Writing bash scripts (\texttt{.sh} files)
\item
  Creating functions for reuse
\item
  Using loops: \texttt{for}, \texttt{while}
\item
  Conditional statements: \texttt{if}, \texttt{else}
\item
  Professional script structure
\item
  Error handling and logging
\end{itemize}

Key Skills:

\begin{itemize}
\tightlist
\item
  Write programs in bash
\item
  Create reusable functions
\item
  Automate repetitive tasks
\item
  Handle errors gracefully
\item
  Professional-grade scripts
\end{itemize}

Hands-On Work:

\begin{itemize}
\tightlist
\item
  Write 5 useful bash scripts
\item
  Create 3 reusable functions
\item
  Build loops that process files
\item
  Write scripts with error checking
\item
  Create a file backup automation system
\end{itemize}

\paragraph*{Comprehensive Assessment (2.5-5 hours)}\label{docs__pandoc__latex__src__gitbash_foundation__gitbash_curriculum_overview__gitbash_curriculum_overview.md__comprehensive-assessment-25-5-hours}

\subparagraph*{GitBash Unit Test \& Practice}\label{docs__pandoc__latex__src__gitbash_foundation__gitbash_curriculum_overview__gitbash_curriculum_overview.md__gitbash-unit-test--practice}

What you\textquotesingle ll do:

\begin{itemize}
\tightlist
\item
  30 practical exercises covering all skills
\item
  Real-world scenarios with 3D printing workflows
\item
  Debugging challenges
\item
  Script writing projects
\item
  Performance optimization tasks
\end{itemize}

Success Criteria:

\begin{itemize}
\tightlist
\item
  Complete all 30 exercises correctly
\item
  Debug 5 failing scripts
\item
  Write 2 advanced automation scripts
\item
  Pass all checkpoint tests
\end{itemize}

\subsubsection*{Lesson Progression Map}\label{docs__pandoc__latex__src__gitbash_foundation__gitbash_curriculum_overview__gitbash_curriculum_overview.md__lesson-progression-map}

\begin{lstlisting}[style=Alabaster]
Pre: First Terminal (accessibility focus)
  v
GitBash-0: Paths & Layout (foundations)
  v
GitBash-1: Navigation (mastery)
  v
GitBash-2: Files & Folders (basic manipulation)
  v
GitBash-3: Piping & Redirection (powerful combinations)
  v
GitBash-4: Variables & Aliases (customization)
  v
GitBash-5: History & Debugging (refinement)
  v
GitBash-6: Scripts & Functions (advanced)
  v
Unit Test & Practice (comprehensive)

\end{lstlisting}

\subsubsection*{Command Reference by Lesson}\label{docs__pandoc__latex__src__gitbash_foundation__gitbash_curriculum_overview__gitbash_curriculum_overview.md__command-reference-by-lesson}

\paragraph*{GitBash-Pre}\label{docs__pandoc__latex__src__gitbash_foundation__gitbash_curriculum_overview__gitbash_curriculum_overview.md__gitbash-pre}

\begin{itemize}
\tightlist
\item
  \texttt{pwd} - Show current location
\item
  \texttt{ls} - List files
\item
  \texttt{cd} - Change directory
\item
  \texttt{echo\ "text"\ \textgreater{}} - Create files
\item
  \texttt{cat} - Read files
\end{itemize}

\paragraph*{GitBash-0}\label{docs__pandoc__latex__src__gitbash_foundation__gitbash_curriculum_overview__gitbash_curriculum_overview.md__gitbash-0}

\begin{itemize}
\tightlist
\item
  \texttt{pwd} - Current location
\item
  \texttt{ls} - List contents
\item
  \texttt{cd\ \textasciitilde{}} - Go home
\item
  \texttt{cd\ ..} - Go up
\item
  \texttt{cd\ /path} - Go to path
\end{itemize}

\paragraph*{GitBash-1}\label{docs__pandoc__latex__src__gitbash_foundation__gitbash_curriculum_overview__gitbash_curriculum_overview.md__gitbash-1}

\begin{itemize}
\tightlist
\item
  \texttt{cd\ FolderName} - Enter folder
\item
  \texttt{cd\ ..} - Go up one level
\item
  \texttt{cd\ \textasciitilde{}} - Go home
\item
  \texttt{ls} - List contents
\item
  Tab completion (Press Tab)
\end{itemize}

\paragraph*{GitBash-2}\label{docs__pandoc__latex__src__gitbash_foundation__gitbash_curriculum_overview__gitbash_curriculum_overview.md__gitbash-2}

\begin{itemize}
\tightlist
\item
  \texttt{mkdir\ FolderName} - Create folder
\item
  \texttt{touch\ FileName} - Create file
\item
  \texttt{cp\ Source\ Dest} - Copy file
\item
  \texttt{mv\ Source\ Dest} - Move/rename
\item
  \texttt{rm\ FileName} - Delete file
\item
  \texttt{rmdir\ FolderName} - Delete empty folder
\item
  \texttt{ls\ -l} - List with details
\end{itemize}

\paragraph*{GitBash-3}\label{docs__pandoc__latex__src__gitbash_foundation__gitbash_curriculum_overview__gitbash_curriculum_overview.md__gitbash-3}

\begin{itemize}
\tightlist
\item
  \texttt{cat\ FileName} - Display file
\item
  \texttt{grep\ "text"\ File} - Search
\item
  \texttt{sort\ FileName} - Sort lines
\item
  \texttt{wc} - Count words/lines
\item
  \texttt{command\ \textgreater{}\ file} - Redirect output
\item
  \texttt{command\ \textbar{}\ other} - Pipe commands
\item
  \texttt{less\ FileName} - Read page by page
\end{itemize}

\paragraph*{GitBash-4}\label{docs__pandoc__latex__src__gitbash_foundation__gitbash_curriculum_overview__gitbash_curriculum_overview.md__gitbash-4}

\begin{itemize}
\tightlist
\item
  \texttt{echo\ \$VAR} - Show variable
\item
  \texttt{export\ VAR="value"} - Set variable
\item
  \texttt{alias\ name="command"} - Create shortcut
\item
  \texttt{.bashrc} - Edit startup file
\item
  \texttt{source\ \textasciitilde{}/.bashrc} - Reload config
\end{itemize}

\paragraph*{GitBash-5}\label{docs__pandoc__latex__src__gitbash_foundation__gitbash_curriculum_overview__gitbash_curriculum_overview.md__gitbash-5}

\begin{itemize}
\tightlist
\item
  \texttt{history} - Show past commands
\item
  \texttt{!CommandNumber} - Rerun command
\item
  \texttt{Ctrl+R} - Search history
\item
  \texttt{set\ -x} - Debug mode
\item
  \texttt{.bashrc} vs \texttt{.bashprofile} - Config files
\end{itemize}

\paragraph*{GitBash-6}\label{docs__pandoc__latex__src__gitbash_foundation__gitbash_curriculum_overview__gitbash_curriculum_overview.md__gitbash-6}

\begin{itemize}
\tightlist
\item
  \texttt{\#!/bin/bash} - Script header
\item
  \texttt{function\ name()\ \{\}} - Define function
\item
  \texttt{for\ var\ in\ list;\ do} - Loops
\item
  \texttt{if\ {[}condition{]};\ then} - Conditionals
\item
  \texttt{source\ script.sh} - Run script
\item
  \texttt{\$1,\ \$2} - Arguments
\end{itemize}

\subsubsection*{Accessibility Features Throughout}\label{docs__pandoc__latex__src__gitbash_foundation__gitbash_curriculum_overview__gitbash_curriculum_overview.md__accessibility-features-throughout}

Every lesson includes:

\begin{itemize}
\tightlist
\item
  Screen reader-tested examples using NVDA and JAWS
\item
  Text-based output - No graphics, all linear text
\item
  Tab completion strategies - Optimized for speech feedback
\item
  File redirection methods - Save output when needed
\item
  Keyboard-only workflows - No mouse required
\item
  Practical tips for both NVDA and JAWS users
\end{itemize}

\subsubsection*{Time Breakdown}\label{docs__pandoc__latex__src__gitbash_foundation__gitbash_curriculum_overview__gitbash_curriculum_overview.md__time-breakdown}

{\def\LTcaptype{none} % do not increment counter
\begin{longtable}[]{@{}ll@{}}
\toprule\noalign{}
Component & Time \\
\midrule\noalign{}
\endhead
\bottomrule\noalign{}
\endlastfoot
GitBash-Pre & 1.5-2.5 hours \\
GitBash-0 & 1-1.5 hours \\
GitBash-1 & 1.5-2.5 hours \\
GitBash-2 & 2-2.5 hours \\
GitBash-3 & 2-2.5 hours \\
GitBash-4 & 1.5-2 hours \\
GitBash-5 & 2-2.5 hours \\
GitBash-6 & 2.5-3.5 hours \\
Unit Test & 2.5-5 hours \\
Total & 20-25 hours \\
\end{longtable}
}

\subsubsection*{Learning Approach}\label{docs__pandoc__latex__src__gitbash_foundation__gitbash_curriculum_overview__gitbash_curriculum_overview.md__learning-approach}

\paragraph*{Core Teaching Method}\label{docs__pandoc__latex__src__gitbash_foundation__gitbash_curriculum_overview__gitbash_curriculum_overview.md__core-teaching-method}

\begin{enumerate}
\tightlist
\item
  Understand WHY - Each lesson starts with purpose
\item
  Learn WHAT - Commands and concepts
\item
  Practice HOW - Hands-on exercises
\item
  Master IT - Checkpoint questions and real-world work
\end{enumerate}

\paragraph*{Recommended Pace}\label{docs__pandoc__latex__src__gitbash_foundation__gitbash_curriculum_overview__gitbash_curriculum_overview.md__recommended-pace}

\begin{itemize}
\tightlist
\item
  Fast track: Complete in 2-3 weeks (2-3 hours daily)
\item
  Standard pace: Complete in 8-12 weeks (2-3 hours, 3x per week)
\item
  Self-paced: Work at your own speed with breaks
\end{itemize}

\paragraph*{Best Practices}\label{docs__pandoc__latex__src__gitbash_foundation__gitbash_curriculum_overview__gitbash_curriculum_overview.md__best-practices}

\begin{itemize}
\tightlist
\item
  Take breaks between lessons
\item
  Do all practice exercises - don\textquotesingle t skip
\item
  Answer checkpoint questions before moving on
\item
  Review previous lessons if stuck
\item
  Use screen reader at full speed once comfortable
\end{itemize}

\subsubsection*{Prerequisites \& Assumptions}\label{docs__pandoc__latex__src__gitbash_foundation__gitbash_curriculum_overview__gitbash_curriculum_overview.md__prerequisites--assumptions}

This curriculum assumes you:

\begin{itemize}
\tightlist
\item
  Have Git Bash installed (free from \url{https://git-scm.com})
\item
  Use a screen reader (NVDA, JAWS, or similar)
\item
  Have Windows 10 or later
\item
  Can use a keyboard efficiently
\item
  Have a text editor (Notepad, VS Code, etc.)
\end{itemize}

\subsubsection*{What You\textquotesingle ll Be Able to Do}\label{docs__pandoc__latex__src__gitbash_foundation__gitbash_curriculum_overview__gitbash_curriculum_overview.md__what-youll-be-able-to-do}

After completing this curriculum:

\paragraph*{Basic Skills (After Lesson 1)}\label{docs__pandoc__latex__src__gitbash_foundation__gitbash_curriculum_overview__gitbash_curriculum_overview.md__basic-skills-after-lesson-1}

\begin{itemize}
\tightlist
\item
  Navigate any folder on your computer
\item
  List files to understand what\textquotesingle s available
\item
  Create and organize folders
\item
  Find files quickly
\end{itemize}

\paragraph*{Intermediate Skills (After Lesson 4)}\label{docs__pandoc__latex__src__gitbash_foundation__gitbash_curriculum_overview__gitbash_curriculum_overview.md__intermediate-skills-after-lesson-4}

\begin{itemize}
\tightlist
\item
  Create, copy, move, and delete files
\item
  Search through text for information
\item
  Combine commands into powerful workflows
\item
  Customize your terminal environment
\end{itemize}

\paragraph*{Advanced Skills (After Lesson 6)}\label{docs__pandoc__latex__src__gitbash_foundation__gitbash_curriculum_overview__gitbash_curriculum_overview.md__advanced-skills-after-lesson-6}

\begin{itemize}
\tightlist
\item
  Write bash programs and scripts
\item
  Automate repetitive tasks
\item
  Build professional tools
\item
  Solve complex problems
\item
  Work efficiently across Windows, macOS, and Linux
\end{itemize}

\subsubsection*{Assessment \& Completion}\label{docs__pandoc__latex__src__gitbash_foundation__gitbash_curriculum_overview__gitbash_curriculum_overview.md__assessment--completion}

\paragraph*{Unit Test (Comprehensive)}\label{docs__pandoc__latex__src__gitbash_foundation__gitbash_curriculum_overview__gitbash_curriculum_overview.md__unit-test-comprehensive}

\begin{itemize}
\tightlist
\item
  30 practical exercises
\item
  Real-world 3D printing scenarios
\item
  Script writing projects
\item
  Performance: 90\%+ correct = mastery
\end{itemize}

\paragraph*{Certification}\label{docs__pandoc__latex__src__gitbash_foundation__gitbash_curriculum_overview__gitbash_curriculum_overview.md__certification}

Upon completion:

\begin{itemize}
\tightlist
\item
  Printable certificate of completion
\item
  Skills assessment document
\item
  Portfolio project (optional)
\end{itemize}

\subsubsection*{Moving Forward}\label{docs__pandoc__latex__src__gitbash_foundation__gitbash_curriculum_overview__gitbash_curriculum_overview.md__moving-forward}

After completing this curriculum, you can:

\paragraph*{Use Git Bash Daily}\label{docs__pandoc__latex__src__gitbash_foundation__gitbash_curriculum_overview__gitbash_curriculum_overview.md__use-git-bash-daily}

\begin{itemize}
\tightlist
\item
  Automate 3D printing workflows
\item
  Manage file systems efficiently
\item
  Create custom tools
\end{itemize}

\paragraph*{Learn More}\label{docs__pandoc__latex__src__gitbash_foundation__gitbash_curriculum_overview__gitbash_curriculum_overview.md__learn-more}

\begin{itemize}
\tightlist
\item
  Advanced bash scripting
\item
  Using Git for version control
\item
  Integrating with OpenSCAD
\item
  Professional shell environments
\end{itemize}

\paragraph*{Transition to Other Systems}\label{docs__pandoc__latex__src__gitbash_foundation__gitbash_curriculum_overview__gitbash_curriculum_overview.md__transition-to-other-systems}

\begin{itemize}
\tightlist
\item
  Skills transfer to macOS Terminal
\item
  Skills transfer to Linux systems
\item
  Knowledge applies to cloud environments (AWS, Azure, GCP)
\end{itemize}

\subsubsection*{Frequently Asked Questions}\label{docs__pandoc__latex__src__gitbash_foundation__gitbash_curriculum_overview__gitbash_curriculum_overview.md__frequently-asked-questions}

Q: Why Git Bash instead of PowerShell or CMD?
A: Git Bash gives you the same commands as macOS and Linux, making it a cross-platform skill that\textquotesingle s more valuable long-term.

Q: How is this different from regular bash?
A: Git Bash is bash running on Windows. 100\% compatible with standard bash. All commands work the same way.

Q: Will I need to use a mouse?
A: No! The entire curriculum is keyboard-based. Perfect for screen reader users.

Q: Can I take breaks between lessons?
A: Yes! Each lesson stands alone. You can take days or weeks between lessons.

Q: What if I get stuck?
A: Each lesson has extension problems for deeper learning and troubleshooting guides for common issues.

Q: Will this help with 3D printing?
A: Absolutely! Many lessons include 3D printing workflows and OpenSCAD automation examples.

\subsubsection*{Support \& Resources}\label{docs__pandoc__latex__src__gitbash_foundation__gitbash_curriculum_overview__gitbash_curriculum_overview.md__support--resources}

\begin{itemize}
\tightlist
\item
  NVDA Screen Reader: \url{https://www.nvaccess.org/}
\item
  JAWS Screen Reader: \url{https://www.freedomscientific.com/products/software/jaws/}
\item
  Bash Manual: \url{https://www.gnu.org/software/bash/manual/}
\item
  Git Bash Documentation: \url{https://git-scm.com/book/}
\item
  Linux Command Reference: \url{https://linuxcommand.org/}
\end{itemize}

\subsubsection*{Curriculum Philosophy}\label{docs__pandoc__latex__src__gitbash_foundation__gitbash_curriculum_overview__gitbash_curriculum_overview.md__curriculum-philosophy}

This curriculum is built on:

\begin{itemize}
\tightlist
\item
  Accessibility First - Every lesson designed for screen readers
\item
  Practical Learning - Real commands, real workflows
\item
  Progressive Complexity - Build skills systematically
\item
  Hands-On Practice - Learn by doing
\item
  Professional Standards - Industry-ready skills
\end{itemize}

\subsubsection*{Ready to Begin?}\label{docs__pandoc__latex__src__gitbash_foundation__gitbash_curriculum_overview__gitbash_curriculum_overview.md__ready-to-begin}

Start with the \href{https://github.com/mrhunsaker/VI_3DMake_OpenSCAD_Curriculum/GitBash_Foundation/GitBash_Curriculum_Overview/../Command_Line_Interface_Selection.md}{Command Line Interface Selection Guide} to see why Git Bash might be the right choice for you.

Then proceed to GitBash-Pre: Your First Terminal to begin your journey!

Goal: You can navigate to any folder and see what\textquotesingle s in it with your screen reader.

\paragraph*{Phase 2: Intermediate User -\textgreater{} Power User}\label{docs__pandoc__latex__src__gitbash_foundation__gitbash_curriculum_overview__gitbash_curriculum_overview.md__phase-2-intermediate-user---power-user}

{\def\LTcaptype{none} % do not increment counter
\begin{longtable}[]{@{}
  >{\raggedright\arraybackslash}p{(\linewidth - 4\tabcolsep) * \real{0.4286}}
  >{\raggedright\arraybackslash}p{(\linewidth - 4\tabcolsep) * \real{0.1099}}
  >{\raggedright\arraybackslash}p{(\linewidth - 4\tabcolsep) * \real{0.4615}}@{}}
\toprule\noalign{}
\begin{minipage}[b]{\linewidth}\raggedright
Lesson
\end{minipage} & \begin{minipage}[b]{\linewidth}\raggedright
Duration
\end{minipage} & \begin{minipage}[b]{\linewidth}\raggedright
What You\textquotesingle ll Learn
\end{minipage} \\
\midrule\noalign{}
\endhead
\bottomrule\noalign{}
\endlastfoot
GB-2: File \& Folder Manipulation & 60 min &
Create, copy, move, delete files/folders \\
GB-3: Input, Output \& Piping & 60 min &
Chain commands together, redirect output \\
GB-4: Environment Variables \& Aliases & 45 min &
Automate settings, create shortcuts \\
GB-5: Filling in the Gaps & 45 min & Profiles, history, debugging \\
\end{longtable}
}

Goal: You can create folders, manage files, and combine commands to accomplish complex tasks.

\paragraph*{Phase 3: Professional Skills (Beyond Curriculum)}\label{docs__pandoc__latex__src__gitbash_foundation__gitbash_curriculum_overview__gitbash_curriculum_overview.md__phase-3-professional-skills-beyond-curriculum}

{\def\LTcaptype{none} % do not increment counter
\begin{longtable}[]{@{}ll@{}}
\toprule\noalign{}
Topic & When to Learn \\
\midrule\noalign{}
\endhead
\bottomrule\noalign{}
\endlastfoot
Shell scripting (.sh files) & After GB-5 \\
Functions \& Loops & After GB-5 \\
Error Handling & After GB-5 \\
3D Printing Integration & After all above \\
\end{longtable}
}

\subsubsection*{How to Use This Curriculum}\label{docs__pandoc__latex__src__gitbash_foundation__gitbash_curriculum_overview__gitbash_curriculum_overview.md__how-to-use-this-curriculum}

\paragraph*{If You\textquotesingle ve Never Used a Terminal Before}\label{docs__pandoc__latex__src__gitbash_foundation__gitbash_curriculum_overview__gitbash_curriculum_overview.md__if-youve-never-used-a-terminal-before}

\begin{enumerate}
\tightlist
\item
  Read Screen Reader Accessibility Guide completely
\item
  Do GB-Pre: Your First Terminal exercises
\item
  Continue with GB-0, GB-1, etc.
\end{enumerate}

\paragraph*{If You\textquotesingle ve Used a Terminal Before (But Not with a Screen Reader)}\label{docs__pandoc__latex__src__gitbash_foundation__gitbash_curriculum_overview__gitbash_curriculum_overview.md__if-youve-used-a-terminal-before-but-not-with-a-screen-reader}

\begin{enumerate}
\tightlist
\item
  Skim Screen Reader Accessibility Guide
\item
  Quickly review GB-Pre
\item
  Move to GB-0 for deeper learning
\end{enumerate}

\paragraph*{If You\textquotesingle re Experienced with Terminal + Screen Reader}\label{docs__pandoc__latex__src__gitbash_foundation__gitbash_curriculum_overview__gitbash_curriculum_overview.md__if-youre-experienced-with-terminal--screen-reader}

\begin{enumerate}
\tightlist
\item
  Jump to specific lessons you need (GB-2, GB-3, etc.)
\item
  Use the Quick Reference sections
\item
  Skip the practice exercises, do the quizzes to verify knowledge
\end{enumerate}

\subsubsection*{How Each Lesson is Structured}\label{docs__pandoc__latex__src__gitbash_foundation__gitbash_curriculum_overview__gitbash_curriculum_overview.md__how-each-lesson-is-structured}

\begin{enumerate}
\tightlist
\item
  Learning Objectives - What you\textquotesingle ll be able to do
\item
  Key Commands - The important ones to memorize
\item
  Step-by-Step Examples - How to actually do it
\item
  Practice Exercises - Hands-on work
\item
  Quiz Questions - Check your understanding
\item
  Extension Problems - Go deeper if interested
\end{enumerate}

\subsubsection*{Screen Reader Tips Throughout the Curriculum}\label{docs__pandoc__latex__src__gitbash_foundation__gitbash_curriculum_overview__gitbash_curriculum_overview.md__screen-reader-tips-throughout-the-curriculum}

Every lesson includes:

\begin{itemize}
\tightlist
\item
  {[}SR{]} symbols marking screen reader-specific sections
\item
  Tips for NVDA and JAWS users separately
\item
  Solutions for common accessibility issues
\item
  Workarounds for long outputs
\end{itemize}

\subsubsection*{Quick Start Guide (First 15 Minutes)}\label{docs__pandoc__latex__src__gitbash_foundation__gitbash_curriculum_overview__gitbash_curriculum_overview.md__quick-start-guide-first-15-minutes}

\begin{enumerate}
\tightlist
\item
  Open Git Bash (Start menu -\textgreater{} type Git Bash -\textgreater{} Enter)
\item
  Run these commands:

  \begin{lstlisting}[style=Alabaster, language=bash]
  pwd
  ls
  cd Documents
  pwd

  \end{lstlisting}
\item
  See how your screen reader reads each output
\item
  Try Tab completion: type \texttt{cd\ D} and press Tab
\item
  Create a file:

  \begin{lstlisting}[style=Alabaster, language=bash]
  echo "I am learning Git Bash" > learning.txt
  cat learning.txt

  \end{lstlisting}
\end{enumerate}

\subsubsection*{Important Rules}\label{docs__pandoc__latex__src__gitbash_foundation__gitbash_curriculum_overview__gitbash_curriculum_overview.md__important-rules}

\paragraph*{Rule 1: Always Know Where You Are}\label{docs__pandoc__latex__src__gitbash_foundation__gitbash_curriculum_overview__gitbash_curriculum_overview.md__rule-1-always-know-where-you-are}

\begin{lstlisting}[style=Alabaster, language=bash]
pwd

\end{lstlisting}

\paragraph*{Rule 2: Check Before You Delete}\label{docs__pandoc__latex__src__gitbash_foundation__gitbash_curriculum_overview__gitbash_curriculum_overview.md__rule-2-check-before-you-delete}

\begin{lstlisting}[style=Alabaster, language=bash]
ls

\end{lstlisting}

\paragraph*{\texorpdfstring{Rule 3: Use \texttt{ls} for listings}{Rule 3: Use ls for listings}}\label{docs__pandoc__latex__src__gitbash_foundation__gitbash_curriculum_overview__gitbash_curriculum_overview.md__rule-3-use-ls-for-listings}

\begin{lstlisting}[style=Alabaster, language=bash]
ls
# or for one-per-line (screen reader friendly):
ls -1

\end{lstlisting}

\paragraph*{Rule 4: When Lost, Redirect to a File}\label{docs__pandoc__latex__src__gitbash_foundation__gitbash_curriculum_overview__gitbash_curriculum_overview.md__rule-4-when-lost-redirect-to-a-file}

\begin{lstlisting}[style=Alabaster, language=bash]
command-name > output.txt
notepad output.txt

\end{lstlisting}

\paragraph*{Rule 5: Save Everything You Create}\label{docs__pandoc__latex__src__gitbash_foundation__gitbash_curriculum_overview__gitbash_curriculum_overview.md__rule-5-save-everything-you-create}

\begin{lstlisting}[style=Alabaster, language=bash]
mkdir my-practice-folder
cd my-practice-folder

\end{lstlisting}

\subsubsection*{Troubleshooting: "Nothing Works!"}\label{docs__pandoc__latex__src__gitbash_foundation__gitbash_curriculum_overview__gitbash_curriculum_overview.md__troubleshooting-nothing-works}

\begin{enumerate}
\tightlist
\item
  Can\textquotesingle t hear Git Bash? Make sure screen reader is running BEFORE Git Bash. Try Alt+Tab.
\item
  Commands not working? Check spelling. Make sure you pressed Enter.
\item
  Can\textquotesingle t read the output? Redirect to file: \texttt{command\ \textgreater{}\ output.txt}, then \texttt{notepad\ output.txt}
\item
  Something ran forever? Press Ctrl+C to stop it.
\item
  Completely confused? Go back to GB-Pre and start over.
\end{enumerate}

\subsubsection*{Curriculum Map}\label{docs__pandoc__latex__src__gitbash_foundation__gitbash_curriculum_overview__gitbash_curriculum_overview.md__curriculum-map}

\begin{lstlisting}[style=Alabaster]
START HERE v
+---- Screen Reader Accessibility Guide (reference throughout)
+---- GB-Pre: Your First Terminal (absolute beginner entry point)
+---- GB-0: Getting Started (paths & navigation foundations)
+---- GB-1: Navigation (comfortable moving around)
+---- GB-2: File & Folder Manipulation (create/move/delete)
+---- GB-3: Input, Output & Piping (chain commands)
+---- GB-4: Environment Variables & Aliases (automation)
+---- GB-5: Filling in the Gaps (profiles & history)
+---- GBUnitTest (comprehensive practice & assessment)
        v
    NEXT: 3D Printing Integration Lessons
        v
    ADVANCED: Bash Scripting

\end{lstlisting}

Questions? Stuck? Refer back to this page and the Screen Reader Accessibility Guide.

Now open GB-Pre and let\textquotesingle s get started!

Other Screen Readers

Dolphin SuperNova (commercial) and Windows Narrator (built-in) are also supported; the workflows and recommendations in this document apply to them. See \url{https://yourdolphin.com/supernova/} and \url{https://support.microsoft.com/narrator} for vendor documentation.

\subsection{Git Bash Introduction}\label{docs__pandoc__latex__src__gitbash_foundation__gitbash_introduction__gitbash_introduction.md__git-bash-introduction}

\subsection{Git Bash Tutorial}\label{docs__pandoc__latex__src__gitbash_foundation__gitbash_tutorial__gitbash_tutorial.md__gitbash_foundation_gitbash_tutorial-gitbash_tutorial}

Estimated time: 30-45 minutes

Learning Objectives

\begin{itemize}
\tightlist
\item
  Launch Git Bash and identify the prompt
\item
  Understand and use basic path notation (\texttt{\textasciitilde{}}, \texttt{./}, \texttt{../})
\item
  Use \texttt{pwd}, \texttt{ls\ -1}, and \texttt{cd} to navigate the filesystem
\item
  Open files in an external editor and run simple commands
\end{itemize}

Materials

\begin{itemize}
\tightlist
\item
  A computer with Git Bash installed (install via \url{https://git-scm.com/})
\item
  Access to a text editor (Notepad, VS Code)
\end{itemize}

\subsubsection*{Pre-Requisite Knowledge}\label{docs__pandoc__latex__src__gitbash_foundation__gitbash_tutorial__gitbash_tutorial.md__pre-requisite-knowledge}

\begin{itemize}
\tightlist
\item
  Typing and basic text-editing skills
\item
  Familiarity with file/folder concepts and basic OS navigation
\item
  Basic screen-reading familiarity (if applicable)
\end{itemize}

\subsubsection*{What is Git Bash?}\label{docs__pandoc__latex__src__gitbash_foundation__gitbash_tutorial__gitbash_tutorial.md__what-is-git-bash}

Git Bash is a free terminal application for Windows that provides a Bash (Unix shell) environment. It is installed as part of Git for Windows. When you open Git Bash, you can use Unix commands like \texttt{ls}, \texttt{cat}, \texttt{grep}, and \texttt{echo} - the same commands used on Linux and macOS - right on your Windows computer.

Git Bash is useful for:

\begin{itemize}
\tightlist
\item
  Running CLI programs (like OpenSCAD, slicers, or 3DMake)
\item
  Navigating the filesystem with keyboard-only commands
\item
  Automating repetitive tasks with scripts
\item
  Accessibility: works cleanly with NVDA, JAWS, and other screen readers
\end{itemize}

\subsubsection*{What We\textquotesingle ll Do and Why}\label{docs__pandoc__latex__src__gitbash_foundation__gitbash_tutorial__gitbash_tutorial.md__what-well-do-and-why}

You\textquotesingle ll use Git Bash to run CLI programs, navigate the filesystem, and manipulate files - tasks that are especially efficient when using a keyboard or a screen reader.

\subsubsection*{Quick Tutorial \& Core Concepts}\label{docs__pandoc__latex__src__gitbash_foundation__gitbash_tutorial__gitbash_tutorial.md__quick-tutorial--core-concepts}

\paragraph*{Paths and Navigation}\label{docs__pandoc__latex__src__gitbash_foundation__gitbash_tutorial__gitbash_tutorial.md__paths-and-navigation}

\begin{itemize}
\tightlist
\item
  \texttt{\textasciitilde{}} - home directory (e.g., \texttt{/c/Users/YourName})
\item
  \texttt{.} - current directory
\item
  \texttt{..} - parent directory
\item
  \texttt{./} - current directory shortcut (used to run scripts: \texttt{./script.sh})
\item
  \texttt{../} - parent directory shortcut
\item
  Use Tab to autocomplete files and folders
\end{itemize}

Git Bash uses forward slashes (\texttt{/}) for paths, just like Linux. Your Windows \texttt{C:\textbackslash{}Users\textbackslash{}YourName} folder is accessible as \texttt{/c/Users/YourName} or simply \texttt{\textasciitilde{}}.

\paragraph*{Useful Commands (Examples)}\label{docs__pandoc__latex__src__gitbash_foundation__gitbash_tutorial__gitbash_tutorial.md__useful-commands-examples}

\begin{lstlisting}[style=Alabaster, language=bash]
pwd                # show current directory
ls -1              # list files, one per line (screen-reader friendly)
cd path/to/dir     # change directory
whoami             # current user

\end{lstlisting}

\paragraph*{Wildcards}\label{docs__pandoc__latex__src__gitbash_foundation__gitbash_tutorial__gitbash_tutorial.md__wildcards}

\begin{itemize}
\tightlist
\item
  \texttt{*} matches zero or more characters
\item
  \texttt{?} matches a single character
\end{itemize}

Use \texttt{ls\ *.scad} to filter by extension, for example.

\subsubsection*{Common Operations}\label{docs__pandoc__latex__src__gitbash_foundation__gitbash_tutorial__gitbash_tutorial.md__common-operations}

\paragraph*{File and Folder Manipulation}\label{docs__pandoc__latex__src__gitbash_foundation__gitbash_tutorial__gitbash_tutorial.md__file-and-folder-manipulation}

\begin{lstlisting}[style=Alabaster, language=bash]
mkdir my-folder          # create folder
mkdir -p a/b/c           # create nested folders at once
cp -r src dest           # copy (use -r for directories)
mv oldname newname       # rename or move
rm file                  # remove file
rm -r folder             # remove folder and contents
touch filename.txt       # create new empty file

\end{lstlisting}

\paragraph*{Input, Output, and Piping}\label{docs__pandoc__latex__src__gitbash_foundation__gitbash_tutorial__gitbash_tutorial.md__input-output-and-piping}

\begin{lstlisting}[style=Alabaster, language=bash]
echo 'hello' | clip      # copy to clipboard
command > output.txt     # redirect output to file
command >> output.txt    # append output to file
command > /dev/null      # discard/suppress output

\end{lstlisting}

Use \texttt{\textbar{}} to pipe output and \texttt{\textgreater{}} to redirect into files.

\paragraph*{Editing and Running Programs}\label{docs__pandoc__latex__src__gitbash_foundation__gitbash_tutorial__gitbash_tutorial.md__editing-and-running-programs}

\begin{lstlisting}[style=Alabaster, language=bash]
notepad file.txt         # open in Notepad (Windows)
code file.txt            # open in VS Code (if in PATH)
./script.sh              # run a script in the current directory
chmod +x script.sh       # make a script executable

\end{lstlisting}

\subsubsection*{Screen-Reader Friendly Tips}\label{docs__pandoc__latex__src__gitbash_foundation__gitbash_tutorial__gitbash_tutorial.md__screen-reader-friendly-tips}

\begin{itemize}
\tightlist
\item
  Prefer \texttt{ls\ -1} for name-only, one-per-line output.
\item
  Filter lists with \texttt{grep}: \texttt{ls\ -1\ \textasciitilde{}/Documents\ \textbar{}\ grep\ "\textbackslash{}.scad\$"} (files only ending in .scad).
\item
  Redirect very long outputs to a file and open it in an editor.
\end{itemize}

\subsubsection*{Error Handling and Control}\label{docs__pandoc__latex__src__gitbash_foundation__gitbash_tutorial__gitbash_tutorial.md__error-handling-and-control}

\begin{itemize}
\tightlist
\item
  Abort a running command: \texttt{Ctrl+C}
\item
  View history: Up/Down arrows or \texttt{history}
\item
  Clear screen: \texttt{clear} (or \texttt{Ctrl+L})
\item
  If an error is long, read the first few lines for the gist and copy short snippets into a file to examine.
\end{itemize}

\subsubsection*{Environment Variables \& PATH}\label{docs__pandoc__latex__src__gitbash_foundation__gitbash_tutorial__gitbash_tutorial.md__environment-variables--path}

Environment variables configure your session. \texttt{PATH} tells the shell where to find executables.

\begin{lstlisting}[style=Alabaster, language=bash]
echo $PATH
echo $HOME
echo $USER

\end{lstlisting}

To add a directory to PATH for the current session:

\begin{lstlisting}[style=Alabaster, language=bash]
export PATH="$PATH:/path/to/my/tools"

\end{lstlisting}

To make it permanent, add that line to \texttt{\textasciitilde{}/.bashrc}.

\subsubsection*{Running CLI Applications and Archives}\label{docs__pandoc__latex__src__gitbash_foundation__gitbash_tutorial__gitbash_tutorial.md__running-cli-applications-and-archives}

To extract ZIP archives in Git Bash:

\begin{lstlisting}[style=Alabaster, language=bash]
# Unzip a file
unzip file.zip -d destinationfolder

# Or use Windows' built-in tool (from Git Bash):
powershell.exe Expand-Archive -Path file.zip -DestinationPath folder

\end{lstlisting}

\subsubsection*{Aliases and Cross-Platform Notes}\label{docs__pandoc__latex__src__gitbash_foundation__gitbash_tutorial__gitbash_tutorial.md__aliases-and-cross-platform-notes}

Git Bash provides Unix commands that work identically on Linux and macOS. This means your Git Bash skills transfer directly if you ever work on a Mac or Linux system.

Useful aliases to add to \texttt{\textasciitilde{}/.bashrc}:

\begin{lstlisting}[style=Alabaster, language=bash]
alias ll='ls -la'
alias la='ls -1a'
alias ..='cd ..'
alias ...='cd ../..'

\end{lstlisting}

\subsubsection*{Step-by-step Tasks (Hands-On)}\label{docs__pandoc__latex__src__gitbash_foundation__gitbash_tutorial__gitbash_tutorial.md__step-by-step-tasks-hands-on}

\begin{enumerate}
\tightlist
\item
  Open Git Bash; listen for the prompt and current path.
\item
  Run \texttt{pwd} to confirm your location.
\item
  Run \texttt{ls\ -1} in your home directory and note the output.
\item
  Practice \texttt{cd\ Documents}, \texttt{cd\ ../}, and \texttt{cd\ \textasciitilde{}} to move between folders.
\item
  Create and open a file: \texttt{touch\ example.txt\ \&\&\ notepad\ example.txt} (or \texttt{code\ example.txt}).
\end{enumerate}

\subsubsection*{Checkpoints}\label{docs__pandoc__latex__src__gitbash_foundation__gitbash_tutorial__gitbash_tutorial.md__checkpoints}

\begin{itemize}
\tightlist
\item
  After step 3 you should be able to state your current directory.
\item
  After step 5 you should be able to create and open a text file from Git Bash.
\end{itemize}

\subsubsection*{Quick Quiz (10 questions)}\label{docs__pandoc__latex__src__gitbash_foundation__gitbash_tutorial__gitbash_tutorial.md__quick-quiz-10-questions}

\begin{enumerate}
\tightlist
\item
  What command prints your current directory?
\item
  What does \texttt{\textasciitilde{}} represent?
\item
  How do you list only names, one per line?
\item
  How do you go up one directory level?
\item
  How would you open \texttt{notes.txt} in Notepad from Git Bash?
\item
  True or False: The pipe operator \texttt{\textbar{}} sends output to a file.
\item
  Explain why running a script requires \texttt{chmod\ +x} first.
\item
  What is the difference between running a script with \texttt{./script.sh} versus \texttt{bash\ script.sh}?
\item
  Describe how you would handle a very long command output when using a screen reader.
\item
  What does the \texttt{PATH} environment variable do, and why is it important when running programs like OpenSCAD?
\end{enumerate}

\subsubsection*{Extension Problems}\label{docs__pandoc__latex__src__gitbash_foundation__gitbash_tutorial__gitbash_tutorial.md__extension-problems}

\begin{enumerate}
\tightlist
\item
  Create a folder \texttt{OpenSCADProjects} in Documents and verify its contents.
\item
  Create three files named \texttt{a.scad}, \texttt{b.scad}, \texttt{c.scad} and list them with a wildcard.
\item
  Save \texttt{ls\ -1\ \textasciitilde{}/Documents} output to \texttt{doclist.txt} and open it.
\item
  Try tab-completion in a deeply nested folder and note behavior.
\item
  Capture \texttt{pwd} output into a file and open it: \texttt{pwd\ \textgreater{}\ cwd.txt\ \&\&\ notepad\ cwd.txt}.
\item
  Build an automated setup script that creates a complete project directory structure, initializes placeholder files, and generates a README.
\item
  Create a Git Bash cheat sheet for your most-used commands; organize by category (navigation, files, scripting, troubleshooting).
\item
  Write a non-visual tutorial for Git Bash basics; use audio descriptions and keyboard-only navigation as the primary learning method.
\item
  Develop a workflow automation script: combines multiple Git Bash concepts (folders, aliases, piping) to solve a real 3D printing task.
\item
  Create a Git Bash proficiency self-assessment: list all concepts covered, provide test commands for each, and reflect on what you learned.
\end{enumerate}

\subsubsection*{References}\label{docs__pandoc__latex__src__gitbash_foundation__gitbash_tutorial__gitbash_tutorial.md__references}

\begin{itemize}
\tightlist
\item
  Git for Windows. (2024). \emph{Git Bash}. \url{https://gitforwindows.org/}
\item
  GNU. (2024). \emph{Bash reference manual}. \url{https://www.gnu.org/software/bash/manual/bash.html}
\item
  The Linux Documentation Project. (2024). \emph{Bash Beginners Guide}. \url{https://tldp.org/LDP/Bash-Beginners-Guide/html/}
\end{itemize}

\subsubsection*{Helpful Resources}\label{docs__pandoc__latex__src__gitbash_foundation__gitbash_tutorial__gitbash_tutorial.md__helpful-resources}

\begin{itemize}
\tightlist
\item
  \href{https://gitforwindows.org/}{Git for Windows}
\item
  \href{https://www.gnu.org/software/bash/manual/bash.html}{Bash Reference Manual (GNU)}
\item
  \href{https://tldp.org/LDP/Bash-Beginners-Guide/html/}{Bash Beginners Guide}
\item
  \href{https://www.gnu.org/software/coreutils/manual/}{GNU Coreutils}
\end{itemize}

Other Screen Readers

Dolphin SuperNova (commercial) and Windows Narrator (built-in) are also supported; the workflows and recommendations in this document apply to them. See \url{https://example.com} and \url{https://example.com} for vendor documentation.

\subsection{Screen Reader Accessibility Guide for Git Bash}\label{docs__pandoc__latex__src__gitbash_foundation__screen_reader_accessibility_guide__screen_reader_accessibility_guide.md__screen-reader-accessibility-guide-for-git-bash}

Target Users: NVDA, JAWS, and other screen reader users\\
Last Updated: 2026

This guide is used throughout the Git Bash Foundation curriculum to help screen reader users navigate and work efficiently with the terminal.

\subsubsection*{Table of Contents}\label{docs__pandoc__latex__src__gitbash_foundation__screen_reader_accessibility_guide__screen_reader_accessibility_guide.md__table-of-contents}

\begin{enumerate}
\tightlist
\item
  \hyperref[docs__pandoc__latex__src__gitbash_foundation__screen_reader_accessibility_guide__screen_reader_accessibility_guide.md__getting-started]{Getting Started with Screen Readers}
\item
  \hyperref[docs__pandoc__latex__src__gitbash_foundation__screen_reader_accessibility_guide__screen_reader_accessibility_guide.md__nvda-tips]{NVDA-Specific Tips}
\item
  \hyperref[docs__pandoc__latex__src__gitbash_foundation__screen_reader_accessibility_guide__screen_reader_accessibility_guide.md__jaws-tips]{JAWS-Specific Tips}
\item
  \hyperref[docs__pandoc__latex__src__gitbash_foundation__screen_reader_accessibility_guide__screen_reader_accessibility_guide.md__general-terminal]{General Terminal Accessibility}
\item
  \hyperref[docs__pandoc__latex__src__gitbash_foundation__screen_reader_accessibility_guide__screen_reader_accessibility_guide.md__long-output]{Working with Long Output}
\item
  \hyperref[docs__pandoc__latex__src__gitbash_foundation__screen_reader_accessibility_guide__screen_reader_accessibility_guide.md__shortcuts]{Keyboard Shortcuts Reference}
\item
  \hyperref[docs__pandoc__latex__src__gitbash_foundation__screen_reader_accessibility_guide__screen_reader_accessibility_guide.md__troubleshooting]{Troubleshooting}
\end{enumerate}

\subsubsection*{Getting Started with Screen Readers}\label{docs__pandoc__latex__src__gitbash_foundation__screen_reader_accessibility_guide__screen_reader_accessibility_guide.md__getting-started}

\paragraph*{Which Screen Reader Should I Use?}\label{docs__pandoc__latex__src__gitbash_foundation__screen_reader_accessibility_guide__screen_reader_accessibility_guide.md__which-screen-reader-should-i-use}

Both NVDA and JAWS work well with Git Bash, but they have different strengths:

{\def\LTcaptype{none} % do not increment counter
\begin{longtable}[]{@{}lll@{}}
\toprule\noalign{}
Feature & NVDA & JAWS \\
\midrule\noalign{}
\endhead
\bottomrule\noalign{}
\endlastfoot
Cost & Free & Commercial (paid) \\
Installation & Simple & Complex but thorough \\
Git Bash Support & Excellent & Excellent \\
Learning Curve & Gentle & Steeper \\
Customization & Good & Extensive \\
\end{longtable}
}

Recommendation: Start with NVDA if you\textquotesingle re new to screen readers. Both will work for this curriculum.

Additional options: Dolphin SuperNova is a commercial screen reader/ magnifier popular in some education settings; Windows Narrator is built into Windows and useful for quick access without installing additional software. Choose based on availability and personal preference.

\paragraph*{Before You Start}\label{docs__pandoc__latex__src__gitbash_foundation__screen_reader_accessibility_guide__screen_reader_accessibility_guide.md__before-you-start}

\begin{enumerate}
\tightlist
\item
  Make sure your screen reader is running before opening Git Bash
\item
  Open Git Bash and let your screen reader read the window title and prompt
\item
  If you don\textquotesingle t hear anything, press Alt+Tab to cycle windows and find Git Bash
\item
  Use your screen reader\textquotesingle s screen review features to understand the layout
\end{enumerate}

\paragraph*{What is Git Bash?}\label{docs__pandoc__latex__src__gitbash_foundation__screen_reader_accessibility_guide__screen_reader_accessibility_guide.md__what-is-git-bash}

Git Bash is a terminal application for Windows that provides a Unix-style command-line experience (Bash shell). It is installed as part of Git for Windows, which is free software available at \url{https://git-scm.com/}. When you open Git Bash, you get the same \texttt{bash}, \texttt{ls}, \texttt{grep}, \texttt{cat}, and other Unix tools used on Linux and macOS - but running on your Windows computer.

\subsubsection*{NVDA-Specific Tips}\label{docs__pandoc__latex__src__gitbash_foundation__screen_reader_accessibility_guide__screen_reader_accessibility_guide.md__nvda-tips}

NVDA is free and available from \url{https://www.nvaccess.org/}

\paragraph*{Key Commands for Git Bash}\label{docs__pandoc__latex__src__gitbash_foundation__screen_reader_accessibility_guide__screen_reader_accessibility_guide.md__key-commands-for-git-bash}

{\def\LTcaptype{none} % do not increment counter
\begin{longtable}[]{@{}
  >{\raggedright\arraybackslash}p{(\linewidth - 2\tabcolsep) * \real{0.3038}}
  >{\raggedright\arraybackslash}p{(\linewidth - 2\tabcolsep) * \real{0.6962}}@{}}
\toprule\noalign{}
\begin{minipage}[b]{\linewidth}\raggedright
Command
\end{minipage} & \begin{minipage}[b]{\linewidth}\raggedright
What It Does
\end{minipage} \\
\midrule\noalign{}
\endhead
\bottomrule\noalign{}
\endlastfoot
NVDA+Home & Read the current line (your command or output) \\
NVDA+Down Arrow & Read from cursor to end of screen \\
NVDA+Up Arrow & Read from top to cursor \\
NVDA+Page Down & Read next page \\
NVDA+Page Up & Read previous page \\
NVDA+F7 & Open the Review Mode viewer (can scroll through text) \\
NVDA+Shift+Right Arrow & Read next word \\
NVDA+Shift+Down Arrow & Read entire screen \\
NVDA+End & Jump to end of line \\
NVDA+Home & Jump to start of line \\
\end{longtable}
}

\paragraph*{Example: Reading Long Output}\label{docs__pandoc__latex__src__gitbash_foundation__screen_reader_accessibility_guide__screen_reader_accessibility_guide.md__example-reading-long-output}

Scenario: You ran \texttt{ls} and it listed 50 files.

Solution with NVDA:

\begin{enumerate}
\tightlist
\item
  After the command finishes, press NVDA+Home to read the current line
\item
  Press NVDA+Down Arrow repeatedly to read all output
\item
  Or press NVDA+F7 to open Review Mode and use arrow keys to scroll
\end{enumerate}

\paragraph*{\texorpdfstring{Tip: Use \texttt{ls\ -1} for Screen Reader Friendly Listings}{Tip: Use ls -1 for Screen Reader Friendly Listings}}\label{docs__pandoc__latex__src__gitbash_foundation__screen_reader_accessibility_guide__screen_reader_accessibility_guide.md__tip-use-ls--1-for-screen-reader-friendly-listings}

By default, \texttt{ls} in Git Bash may display files in columns. Use \texttt{ls\ -1} (the number one, not the letter L) to display one file per line - much easier to follow with a screen reader.

\subsubsection*{JAWS-Specific Tips}\label{docs__pandoc__latex__src__gitbash_foundation__screen_reader_accessibility_guide__screen_reader_accessibility_guide.md__jaws-tips}

JAWS is a commercial screen reader available from \url{https://www.freedomscientific.com/}

\paragraph*{Dolphin SuperNova}\label{docs__pandoc__latex__src__gitbash_foundation__screen_reader_accessibility_guide__screen_reader_accessibility_guide.md__dolphin-supernova}

Dolphin SuperNova (commercial): \url{https://example.com} --- provides speech and braille support and tightly integrated magnification. Refer to vendor docs for keyboard mappings; many users rely on review-mode features similar to NVDA/JAWS.

\paragraph*{Windows Narrator}\label{docs__pandoc__latex__src__gitbash_foundation__screen_reader_accessibility_guide__screen_reader_accessibility_guide.md__windows-narrator}

Windows Narrator (built-in): \url{https://example.com} --- good for quick checks and lightweight usage; Narrator command keys differ by Windows version (use Narrator key + arrow keys to read). Add Narrator to your workflow when a full third-party screen reader isn\textquotesingle t available.

\paragraph*{Key Commands for Git Bash}\label{docs__pandoc__latex__src__gitbash_foundation__screen_reader_accessibility_guide__screen_reader_accessibility_guide.md__key-commands-for-git-bash-1}

{\def\LTcaptype{none} % do not increment counter
\begin{longtable}[]{@{}ll@{}}
\toprule\noalign{}
Command & What It Does \\
\midrule\noalign{}
\endhead
\bottomrule\noalign{}
\endlastfoot
Insert+Down Arrow & Read line by line downward \\
Insert+Up Arrow & Read line by line upward \\
Insert+Page Down & Read next page of text \\
Insert+Page Up & Read previous page of text \\
Insert+End & Jump to end of text on screen \\
Insert+Home & Jump to start of text on screen \\
Insert+Ctrl+Down & Read to end of screen \\
Insert+Ctrl+Up & Read to beginning of screen \\
Insert+F3 & Open JAWS menu \\
\end{longtable}
}

\paragraph*{Example: Reading Long Output}\label{docs__pandoc__latex__src__gitbash_foundation__screen_reader_accessibility_guide__screen_reader_accessibility_guide.md__example-reading-long-output-1}

Scenario: You ran \texttt{ls\ \textgreater{}\ list.txt} and saved output to a file.

Solution with JAWS:

\begin{enumerate}
\tightlist
\item
  Open the file: \texttt{notepad\ list.txt}
\item
  In Notepad, press Insert+Ctrl+Down to hear all content
\item
  Use Insert+Down Arrow to read line by line at your own pace
\end{enumerate}

\subsubsection*{General Terminal Accessibility}\label{docs__pandoc__latex__src__gitbash_foundation__screen_reader_accessibility_guide__screen_reader_accessibility_guide.md__general-terminal}

\paragraph*{Understanding the Git Bash Layout}\label{docs__pandoc__latex__src__gitbash_foundation__screen_reader_accessibility_guide__screen_reader_accessibility_guide.md__understanding-the-git-bash-layout}

The Git Bash window contains:

\begin{enumerate}
\tightlist
\item
  Title bar: Window name (e.g., "MINGW64:/c/Users/YourName")
\item
  Content area: Command history and output
\item
  Prompt: Where you type (e.g., \texttt{YourName@COMPUTER\ MINGW64\ \textasciitilde{}\$})
\end{enumerate}

Your screen reader reads from top to bottom, but focus is at the prompt (bottom).

\paragraph*{The Git Bash Prompt}\label{docs__pandoc__latex__src__gitbash_foundation__screen_reader_accessibility_guide__screen_reader_accessibility_guide.md__the-git-bash-prompt}

The default Git Bash prompt looks like:

\begin{lstlisting}[style=Alabaster, language=bash]
YourName@COMPUTERNAME MINGW64 ~/Documents
$

\end{lstlisting}

\begin{itemize}
\tightlist
\item
  \texttt{YourName} = your Windows username
\item
  \texttt{COMPUTERNAME} = your computer\textquotesingle s name
\item
  \texttt{MINGW64} = the environment type (64-bit)
\item
  \texttt{\textasciitilde{}/Documents} = your current location
\item
  \texttt{\$} = ready for input
\end{itemize}

Note: Paths in Git Bash use forward slashes (\texttt{/}) and your Windows \texttt{C:\textbackslash{}Users\textbackslash{}YourName} folder appears as \texttt{/c/Users/YourName} or simply \texttt{\textasciitilde{}}.

\paragraph*{Navigation Sequence}\label{docs__pandoc__latex__src__gitbash_foundation__screen_reader_accessibility_guide__screen_reader_accessibility_guide.md__navigation-sequence}

When you open Git Bash:

\begin{enumerate}
\tightlist
\item
  Your screen reader announces the window title
\item
  Then it announces the prompt line
\item
  Anything before the prompt is previous output
\item
  Anything after the prompt is where new output will appear
\end{enumerate}

\paragraph*{Reading Output Effectively}\label{docs__pandoc__latex__src__gitbash_foundation__screen_reader_accessibility_guide__screen_reader_accessibility_guide.md__reading-output-effectively}

Strategy 1: Immediate Output (Small Amount)

\begin{itemize}
\tightlist
\item
  Run a command
\item
  Your screen reader announces output immediately
\item
  This works well for short outputs (a few lines)
\end{itemize}

Strategy 2: Large Output (Many Lines)

\begin{itemize}
\tightlist
\item
  Redirect to a file: \texttt{command\ \textgreater{}\ output.txt}
\item
  Open the file: \texttt{notepad\ output.txt}
\item
  Read in Notepad (easier for long text)
\end{itemize}

Strategy 3: Searching Output

\begin{itemize}
\item
  Use \texttt{grep} to filter:

  \begin{lstlisting}[style=Alabaster, language=bash]
  ls | grep "pattern"

  \end{lstlisting}
\item
  Only results matching "pattern" are shown
\end{itemize}

\subsubsection*{Working with Long Output}\label{docs__pandoc__latex__src__gitbash_foundation__screen_reader_accessibility_guide__screen_reader_accessibility_guide.md__long-output}

\paragraph*{Solution 1: Redirect to a File}\label{docs__pandoc__latex__src__gitbash_foundation__screen_reader_accessibility_guide__screen_reader_accessibility_guide.md__solution-1-redirect-to-a-file}

\begin{lstlisting}[style=Alabaster, language=bash]
ls -1 > list.txt
notepad list.txt

\end{lstlisting}

\paragraph*{Solution 2: Use Pagination}\label{docs__pandoc__latex__src__gitbash_foundation__screen_reader_accessibility_guide__screen_reader_accessibility_guide.md__solution-2-use-pagination}

\begin{lstlisting}[style=Alabaster, language=bash]
ls | less

\end{lstlisting}

\begin{itemize}
\tightlist
\item
  Press Space to see next page
\item
  Press Q to quit
\item
  Note: Some screen readers struggle with \texttt{less}, so Solution 1 is preferred
\end{itemize}

\paragraph*{Solution 3: Filter Output}\label{docs__pandoc__latex__src__gitbash_foundation__screen_reader_accessibility_guide__screen_reader_accessibility_guide.md__solution-3-filter-output}

\begin{lstlisting}[style=Alabaster, language=bash]
ls | grep "\.scad"

\end{lstlisting}

\paragraph*{Solution 4: Count Before Displaying}\label{docs__pandoc__latex__src__gitbash_foundation__screen_reader_accessibility_guide__screen_reader_accessibility_guide.md__solution-4-count-before-displaying}

\begin{lstlisting}[style=Alabaster, language=bash]
ls | wc -l

\end{lstlisting}

Tells you how many items there are. If the count is over 20, use the file method.

\subsubsection*{Keyboard Shortcuts Reference}\label{docs__pandoc__latex__src__gitbash_foundation__screen_reader_accessibility_guide__screen_reader_accessibility_guide.md__shortcuts}

\paragraph*{All Users (Works in Git Bash Regardless of Screen Reader)}\label{docs__pandoc__latex__src__gitbash_foundation__screen_reader_accessibility_guide__screen_reader_accessibility_guide.md__all-users-works-in-git-bash-regardless-of-screen-reader}

{\def\LTcaptype{none} % do not increment counter
\begin{longtable}[]{@{}ll@{}}
\toprule\noalign{}
Key & Action \\
\midrule\noalign{}
\endhead
\bottomrule\noalign{}
\endlastfoot
Up Arrow & Show previous command \\
Down Arrow & Show next command \\
Tab & Auto-complete file/folder names \\
Shift+Tab & Cycle backward through completions \\
Home & Jump to start of line \\
End & Jump to end of line \\
Ctrl+A & Jump to start of line (alternate) \\
Ctrl+E & Jump to end of line (alternate) \\
Ctrl+C & Stop command \\
Ctrl+L & Clear screen \\
Enter & Run command \\
Ctrl+R & Search command history interactively \\
\end{longtable}
}

\subsubsection*{Troubleshooting}\label{docs__pandoc__latex__src__gitbash_foundation__screen_reader_accessibility_guide__screen_reader_accessibility_guide.md__troubleshooting}

\paragraph*{Problem 1: "I Can\textquotesingle t Hear the Output After Running a Command"}\label{docs__pandoc__latex__src__gitbash_foundation__screen_reader_accessibility_guide__screen_reader_accessibility_guide.md__problem-1-i-cant-hear-the-output-after-running-a-command}

\begin{enumerate}
\tightlist
\item
  Redirect to file: \texttt{command\ \textgreater{}\ output.txt} then \texttt{notepad\ output.txt}
\item
  Press End or Ctrl+End to go to end of text
\item
  Use Up Arrow to review previous command
\end{enumerate}

\paragraph*{Problem 2: "Tab Completion Isn\textquotesingle t Working"}\label{docs__pandoc__latex__src__gitbash_foundation__screen_reader_accessibility_guide__screen_reader_accessibility_guide.md__problem-2-tab-completion-isnt-working}

\begin{enumerate}
\tightlist
\item
  Need at least one character - type \texttt{cd\ D} then Tab (not just \texttt{cd} then Tab)
\item
  Check if item exists - use \texttt{ls} first to see available items
\item
  Multiple matches - press Tab twice to list all options
\end{enumerate}

\paragraph*{Problem 3: "\textquotesingle Command Not Found\textquotesingle{} Error"}\label{docs__pandoc__latex__src__gitbash_foundation__screen_reader_accessibility_guide__screen_reader_accessibility_guide.md__problem-3-command-not-found-error}

\begin{enumerate}
\tightlist
\item
  Check spelling carefully
\item
  Git Bash uses Unix commands: use \texttt{ls} not \texttt{dir}, use \texttt{cat} not \texttt{type}
\item
  Verify the program is installed and in your PATH: \texttt{which\ openscad}
\end{enumerate}

\paragraph*{Problem 4: "Paths Look Weird"}\label{docs__pandoc__latex__src__gitbash_foundation__screen_reader_accessibility_guide__screen_reader_accessibility_guide.md__problem-4-paths-look-weird}

Git Bash converts Windows paths to Unix style:

\begin{itemize}
\tightlist
\item
  Windows: \texttt{C:\textbackslash{}Users\textbackslash{}YourName\textbackslash{}Documents}
\item
  Git Bash: \texttt{/c/Users/YourName/Documents} or \texttt{\textasciitilde{}/Documents}
\end{itemize}

Both refer to the same location. When opening files in Windows apps (like Notepad), you may need to use the Windows path format.

\paragraph*{Problem 5: "Command Runs Forever and Won\textquotesingle t Stop"}\label{docs__pandoc__latex__src__gitbash_foundation__screen_reader_accessibility_guide__screen_reader_accessibility_guide.md__problem-5-command-runs-forever-and-wont-stop}

Press Ctrl+C

\paragraph*{Problem 6: "I Need to Edit My Last Command"}\label{docs__pandoc__latex__src__gitbash_foundation__screen_reader_accessibility_guide__screen_reader_accessibility_guide.md__problem-6-i-need-to-edit-my-last-command}

\begin{enumerate}
\tightlist
\item
  Press Up Arrow to show previous command
\item
  Use arrow keys to move through it
\item
  Edit the command
\item
  Press Enter to run the modified version
\end{enumerate}

\subsubsection*{Pro Tips for Efficiency}\label{docs__pandoc__latex__src__gitbash_foundation__screen_reader_accessibility_guide__screen_reader_accessibility_guide.md__pro-tips-for-efficiency}

\paragraph*{\texorpdfstring{1. Use \texttt{ls\ -1} for Screen Reader Friendly Listings}{1. Use ls -1 for Screen Reader Friendly Listings}}\label{docs__pandoc__latex__src__gitbash_foundation__screen_reader_accessibility_guide__screen_reader_accessibility_guide.md__1-use-ls--1-for-screen-reader-friendly-listings}

\begin{lstlisting}[style=Alabaster, language=bash]
ls -1

\end{lstlisting}

One file per line - much easier to follow with a screen reader.

\paragraph*{2. Create Aliases for Frequently Used Commands}\label{docs__pandoc__latex__src__gitbash_foundation__screen_reader_accessibility_guide__screen_reader_accessibility_guide.md__2-create-aliases-for-frequently-used-commands}

\begin{lstlisting}[style=Alabaster, language=bash]
alias la='ls -1a'

\end{lstlisting}

Add this to your \texttt{\textasciitilde{}/.bashrc} file to make it permanent.

\paragraph*{3. Use Command History Effectively}\label{docs__pandoc__latex__src__gitbash_foundation__screen_reader_accessibility_guide__screen_reader_accessibility_guide.md__3-use-command-history-effectively}

\begin{lstlisting}[style=Alabaster, language=bash]
history

\end{lstlisting}

Run a previous command by number:

\begin{lstlisting}[style=Alabaster, language=bash]
!5

\end{lstlisting}

(Runs the 5th command in history)

Or search history interactively:

\begin{lstlisting}[style=Alabaster, language=bash]
# Press Ctrl+R then type part of a previous command

\end{lstlisting}

\paragraph*{4. Redirect Everything to Files for Accessibility}\label{docs__pandoc__latex__src__gitbash_foundation__screen_reader_accessibility_guide__screen_reader_accessibility_guide.md__4-redirect-everything-to-files-for-accessibility}

\begin{lstlisting}[style=Alabaster, language=bash]
command-name > results.txt
notepad results.txt

\end{lstlisting}

\paragraph*{5. Create a README for Yourself}\label{docs__pandoc__latex__src__gitbash_foundation__screen_reader_accessibility_guide__screen_reader_accessibility_guide.md__5-create-a-readme-for-yourself}

\begin{lstlisting}[style=Alabaster, language=bash]
echo "ls -1 means list files one per line (screen reader friendly)" > my-notes.txt
echo "cd means change directory" >> my-notes.txt
notepad my-notes.txt

\end{lstlisting}

\subsubsection*{Recommended Workflow}\label{docs__pandoc__latex__src__gitbash_foundation__screen_reader_accessibility_guide__screen_reader_accessibility_guide.md__recommended-workflow}

For every new task:

\begin{enumerate}
\tightlist
\item
  Know where you are: \texttt{pwd}
\item
  See what\textquotesingle s around: \texttt{ls\ -1}
\item
  Plan your next step: Think before typing
\item
  Run the command: Type and press Enter
\item
  Check the output: Use screen reader or redirect to file
\item
  Move forward: Next command or \texttt{cd} to next folder
\end{enumerate}

\subsubsection*{Quick Reference Card}\label{docs__pandoc__latex__src__gitbash_foundation__screen_reader_accessibility_guide__screen_reader_accessibility_guide.md__quick-reference-card}

\begin{lstlisting}[style=Alabaster]
EVERY COMMAND STARTS WITH:
1. pwd (where am I?)
2. ls -1 (what's here?)
3. cd path (go there)

LONG OUTPUT?
-> command > file.txt
-> notepad file.txt

STUCK?
-> Ctrl+C

WANT TO REPEAT?
-> Up Arrow
-> history

NEED HELP?
-> man command-name  (or: command --help)

\end{lstlisting}

\subsubsection*{Additional Resources}\label{docs__pandoc__latex__src__gitbash_foundation__screen_reader_accessibility_guide__screen_reader_accessibility_guide.md__additional-resources}

\begin{itemize}
\tightlist
\item
  NVDA Documentation: \url{https://www.nvaccess.org/documentation/}
\item
  JAWS Documentation: \url{https://www.freedomscientific.com/support/}
\item
  Git for Windows (includes Git Bash): \url{https://git-scm.com/}
\item
  Bash Manual: \url{https://www.gnu.org/software/bash/manual/}
\end{itemize}

\subsection{GitBash-Pre: Your First Terminal - Screen Reader Navigation Fundamentals}\label{docs__pandoc__latex__src__gitbash_foundation__gitbash_pre_your_first_terminal__gitbash_pre_your_first_terminal.md__gitbash_foundation_gitbash_pre_your_first_terminal-gitbash_pre_your_first_terminal}

Duration: 1.5-2 hours (for screen reader users)\\
Prerequisites: None - this is the starting point\\
Accessibility Note: This lesson is designed specifically for screen reader users (NVDA, JAWS)

Important

Git Bash uses Unix/bash commands, different from Windows CMD/PowerShell

\subsubsection*{What is Git Bash?}\label{docs__pandoc__latex__src__gitbash_foundation__gitbash_pre_your_first_terminal__gitbash_pre_your_first_terminal.md__what-is-git-bash}

Git Bash is a Unix/Linux shell running on Windows. It gives you bash commands (same as macOS and Linux) while staying on your Windows computer.

Why learn this?

\begin{itemize}
\tightlist
\item
  Faster and more precise work (especially for 3D printing scripts and automation)
\item
  Essential for programming and using tools like OpenSCAD
\item
  Accessibility: Git Bash works perfectly with screen readers
\item
  Transferable: Same commands work on macOS and Linux
\item
  Industry-standard: Professional developers use this everywhere
\end{itemize}

\subsubsection*{Installing and Opening Git Bash}\label{docs__pandoc__latex__src__gitbash_foundation__gitbash_pre_your_first_terminal__gitbash_pre_your_first_terminal.md__installing-and-opening-git-bash}

\paragraph*{Installation (If Not Already Installed)}\label{docs__pandoc__latex__src__gitbash_foundation__gitbash_pre_your_first_terminal__gitbash_pre_your_first_terminal.md__installation-if-not-already-installed}

\begin{enumerate}
\tightlist
\item
  Download from: \url{https://git-scm.com/download/win}
\item
  Run the installer
\item
  Accept defaults (all options work fine)
\item
  Installation takes \textasciitilde{}5 minutes
\end{enumerate}

\paragraph*{Opening Git Bash}\label{docs__pandoc__latex__src__gitbash_foundation__gitbash_pre_your_first_terminal__gitbash_pre_your_first_terminal.md__opening-git-bash}

Method 1: Search (Easiest)

\begin{enumerate}
\tightlist
\item
  Press the Windows key alone
\item
  You should hear "Search"
\item
  Type: \texttt{Git\ Bash}
\item
  You\textquotesingle ll hear search results appear
\item
  Press Enter to open
\end{enumerate}

Method 2: From File Explorer

\begin{enumerate}
\tightlist
\item
  Open File Explorer
\item
  Navigate to any folder
\item
  Right-click on the folder
\item
  Look for "Git Bash Here"
\item
  Press Enter
\end{enumerate}

\paragraph*{First Connection: Understanding the Prompt}\label{docs__pandoc__latex__src__gitbash_foundation__gitbash_pre_your_first_terminal__gitbash_pre_your_first_terminal.md__first-connection-understanding-the-prompt}

When Git Bash opens, your screen reader will announce the window title and then the prompt.

What you\textquotesingle ll hear (typical example):

\begin{lstlisting}[style=Alabaster]
user@computer MINGW64 ~
$

\end{lstlisting}

What this means:

\begin{itemize}
\tightlist
\item
  \texttt{user} = Your username
\item
  \texttt{@computer} = Your computer name
\item
  \texttt{MINGW64} = Version of Git Bash
\item
  \texttt{\textasciitilde{}} = Your current location (home directory)
\item
  \texttt{\$} = The prompt is ready for your input
\end{itemize}

Important

Your cursor is blinking right after the \texttt{\$}. This is where you type.

\subsubsection*{Your First Commands (Screen Reader Edition)}\label{docs__pandoc__latex__src__gitbash_foundation__gitbash_pre_your_first_terminal__gitbash_pre_your_first_terminal.md__your-first-commands-screen-reader-edition}

\paragraph*{\texorpdfstring{Command 1: "Where Am I?" - \texttt{pwd}}{Command 1: "Where Am I?" - pwd}}\label{docs__pandoc__latex__src__gitbash_foundation__gitbash_pre_your_first_terminal__gitbash_pre_your_first_terminal.md__command-1-where-am-i---pwd}

What it does: Tells you your current location

Type this:

\begin{lstlisting}[style=Alabaster, language=bash]
pwd

\end{lstlisting}

Press Enter

What you\textquotesingle ll hear:
Your screen reader will announce the current path, something like:

\begin{lstlisting}[style=Alabaster]
/c/Users/YourName

\end{lstlisting}

Understanding paths in Git Bash:

\begin{itemize}
\tightlist
\item
  Paths use forward slashes: \texttt{/c/Users/YourName/Documents}
\item
  \texttt{/c/} means the C: drive (Windows drive)
\item
  This is Unix-style, same as macOS and Linux
\item
  Think of it like folders inside folders: \texttt{/c/} (main drive) -\textgreater{} \texttt{Users} -\textgreater{} \texttt{YourName} -\textgreater{} \texttt{Documents}
\end{itemize}

\paragraph*{\texorpdfstring{Command 2: "What\textquotesingle s Here?" - \texttt{ls}}{Command 2: "What\textquotesingle s Here?" - ls}}\label{docs__pandoc__latex__src__gitbash_foundation__gitbash_pre_your_first_terminal__gitbash_pre_your_first_terminal.md__command-2-whats-here---ls}

What it does: Lists all files and folders in your current location

Type this:

\begin{lstlisting}[style=Alabaster, language=bash]
ls

\end{lstlisting}

Press Enter

What you\textquotesingle ll hear:
Your screen reader will announce a list of file and folder names, one per line. Perfect for screen readers!

\paragraph*{\texorpdfstring{Command 3: "Go There" - \texttt{cd\ Documents}}{Command 3: "Go There" - cd Documents}}\label{docs__pandoc__latex__src__gitbash_foundation__gitbash_pre_your_first_terminal__gitbash_pre_your_first_terminal.md__command-3-go-there---cd-documents}

What it does: Changes your location (navigates to a folder)

Type this:

\begin{lstlisting}[style=Alabaster, language=bash]
cd Documents

\end{lstlisting}

Press Enter

What you\textquotesingle ll hear:
The prompt changes to show your new location. You might hear something like:

\begin{lstlisting}[style=Alabaster]
user@computer MINGW64 ~/Documents
$

\end{lstlisting}

Practice navigation:

\begin{enumerate}
\tightlist
\item
  Run \texttt{pwd} to confirm you\textquotesingle re in Documents
\item
  Run \texttt{ls} to see what files are in Documents
\item
  Try going back: \texttt{cd\ ..} (the \texttt{..} means "go up one level")
\item
  Run \texttt{pwd} again to confirm
\item
  Go back to Documents: \texttt{cd\ Documents}
\end{enumerate}

\subsubsection*{Creating and Viewing Files}\label{docs__pandoc__latex__src__gitbash_foundation__gitbash_pre_your_first_terminal__gitbash_pre_your_first_terminal.md__creating-and-viewing-files}

\paragraph*{Create a Simple File}\label{docs__pandoc__latex__src__gitbash_foundation__gitbash_pre_your_first_terminal__gitbash_pre_your_first_terminal.md__create-a-simple-file}

Type this:

\begin{lstlisting}[style=Alabaster, language=bash]
echo "Hello, Git Bash!" > hello.txt

\end{lstlisting}

What this does:

\begin{itemize}
\tightlist
\item
  \texttt{echo} sends text to the screen (or file)
\item
  \texttt{"Hello,\ Git\ Bash!"} is the text
\item
  \texttt{\textgreater{}} redirects it to a file called \texttt{hello.txt}
\end{itemize}

\paragraph*{Read the File Back}\label{docs__pandoc__latex__src__gitbash_foundation__gitbash_pre_your_first_terminal__gitbash_pre_your_first_terminal.md__read-the-file-back}

Type this:

\begin{lstlisting}[style=Alabaster, language=bash]
cat hello.txt

\end{lstlisting}

What you\textquotesingle ll hear: Your screen reader announces:

\begin{lstlisting}[style=Alabaster]
Hello, Git Bash!

\end{lstlisting}

\paragraph*{Open and Edit the File}\label{docs__pandoc__latex__src__gitbash_foundation__gitbash_pre_your_first_terminal__gitbash_pre_your_first_terminal.md__open-and-edit-the-file}

Type this:

\begin{lstlisting}[style=Alabaster, language=bash]
notepad.exe hello.txt

\end{lstlisting}

This opens the file in Notepad where you can edit it with your screen reader.

\subsubsection*{Essential Keyboard Shortcuts}\label{docs__pandoc__latex__src__gitbash_foundation__gitbash_pre_your_first_terminal__gitbash_pre_your_first_terminal.md__essential-keyboard-shortcuts}

These work in Git Bash and are crucial for screen reader users:

{\def\LTcaptype{none} % do not increment counter
\begin{longtable}[]{@{}
  >{\raggedright\arraybackslash}p{(\linewidth - 2\tabcolsep) * \real{0.2152}}
  >{\raggedright\arraybackslash}p{(\linewidth - 2\tabcolsep) * \real{0.7848}}@{}}
\toprule\noalign{}
\begin{minipage}[b]{\linewidth}\raggedright
Key Combination
\end{minipage} & \begin{minipage}[b]{\linewidth}\raggedright
What It Does
\end{minipage} \\
\midrule\noalign{}
\endhead
\bottomrule\noalign{}
\endlastfoot
Up Arrow &
Shows your previous command (press again to go further back) \\
Down Arrow & Shows your next command (if you went back) \\
Tab & Auto-completes folder/file names \\
Ctrl+C & Stops a running command \\
Ctrl+L & Clears the screen \\
Ctrl+A & Go to beginning of line \\
Ctrl+E & Go to end of line \\
Enter & Runs the command \\
\end{longtable}
}

Screen reader tip: These all work perfectly with your screen reader. Try them!

\subsubsection*{Screen Reader-Specific Tips}\label{docs__pandoc__latex__src__gitbash_foundation__gitbash_pre_your_first_terminal__gitbash_pre_your_first_terminal.md__screen-reader-specific-tips}

\paragraph*{NVDA Users}\label{docs__pandoc__latex__src__gitbash_foundation__gitbash_pre_your_first_terminal__gitbash_pre_your_first_terminal.md__nvda-users}

\begin{enumerate}
\item
  Reading Command Output:

  \begin{itemize}
  \tightlist
  \item
    Use NVDA+Home to read the current line
  \item
    Use NVDA+Down Arrow to read to the end of the screen
  \item
    Use NVDA+Page Down to read the next page
  \end{itemize}
\item
  Reviewing Text:

  \begin{itemize}
  \tightlist
  \item
    Use NVDA+Shift+Page Up to review text above
  \end{itemize}
\end{enumerate}

\paragraph*{JAWS Users}\label{docs__pandoc__latex__src__gitbash_foundation__gitbash_pre_your_first_terminal__gitbash_pre_your_first_terminal.md__jaws-users}

\begin{enumerate}
\item
  Reading Output:

  \begin{itemize}
  \tightlist
  \item
    Use Insert+Down Arrow to read line-by-line
  \item
    Use Insert+Page Down to read by page
  \item
    Use Insert+End to jump to the end of text
  \end{itemize}
\item
  Reading All Text:

  \begin{itemize}
  \tightlist
  \item
    Use Insert+Down Arrow repeatedly
  \item
    Or use Insert+Ctrl+Down to read to the end
  \end{itemize}
\end{enumerate}

\paragraph*{Common Issue: "I Can\textquotesingle t Hear the Output"}\label{docs__pandoc__latex__src__gitbash_foundation__gitbash_pre_your_first_terminal__gitbash_pre_your_first_terminal.md__common-issue-i-cant-hear-the-output}

Problem: You run a command but don\textquotesingle t hear the output

Solutions:

\begin{enumerate}
\tightlist
\item
  Make sure your cursor is at the prompt (try pressing End or Ctrl+E)
\item
  Use Up Arrow to go back to your previous command and review it
\item
  Try redirecting to a file: \texttt{command\ \textgreater{}\ output.txt} then open the file
\item
  In NVDA: Try pressing NVDA+F7 to open the Review Mode viewer
\end{enumerate}

\subsubsection*{Practice Exercises}\label{docs__pandoc__latex__src__gitbash_foundation__gitbash_pre_your_first_terminal__gitbash_pre_your_first_terminal.md__practice-exercises}

Complete these in order. Take your time with each one:

\paragraph*{Exercise 1: Basic Navigation}\label{docs__pandoc__latex__src__gitbash_foundation__gitbash_pre_your_first_terminal__gitbash_pre_your_first_terminal.md__exercise-1-basic-navigation}

\begin{enumerate}
\tightlist
\item
  Open Git Bash
\item
  Run \texttt{pwd} and note your location
\item
  Run \texttt{ls} and listen to what\textquotesingle s there
\item
  Try \texttt{cd\ Documents} or another folder
\item
  Run \texttt{pwd} to confirm your new location
\item
  Run \texttt{ls} in this new location
\end{enumerate}

Goal: You should be comfortable knowing where you are and what\textquotesingle s around you

\paragraph*{Exercise 2: Using Tab Completion}\label{docs__pandoc__latex__src__gitbash_foundation__gitbash_pre_your_first_terminal__gitbash_pre_your_first_terminal.md__exercise-2-using-tab-completion}

\begin{enumerate}
\tightlist
\item
  In your home directory, type \texttt{cd\ D} (just the letter D)
\item
  Press Tab
\item
  Git Bash should auto-complete to a folder starting with D
\item
  Repeat with other folder names
\item
  Try typing a longer name: \texttt{cd\ Down} and Tab
\end{enumerate}

Goal: Tab completion should feel natural

\paragraph*{Exercise 3: Creating and Viewing Files}\label{docs__pandoc__latex__src__gitbash_foundation__gitbash_pre_your_first_terminal__gitbash_pre_your_first_terminal.md__exercise-3-creating-and-viewing-files}

\begin{enumerate}
\tightlist
\item
  Create a file: \texttt{echo\ "Test\ content"\ \textgreater{}\ test.txt}
\item
  View it: \texttt{cat\ test.txt}
\item
  Create another: \texttt{echo\ "Line\ 2"\ \textgreater{}\ another.txt}
\item
  List both: \texttt{ls\ *.txt}
\end{enumerate}

Goal: You understand create, view, and list operations

\paragraph*{Exercise 4: Going Up Levels}\label{docs__pandoc__latex__src__gitbash_foundation__gitbash_pre_your_first_terminal__gitbash_pre_your_first_terminal.md__exercise-4-going-up-levels}

\begin{enumerate}
\tightlist
\item
  Navigate into several folders: \texttt{cd\ Documents}, then \texttt{cd\ folder1}, etc.
\item
  From deep inside, use \texttt{cd\ ..} multiple times to go back up
\item
  After each \texttt{cd\ ..}, run \texttt{pwd} to confirm your location
\end{enumerate}

Goal: You understand relative navigation with \texttt{..}

\paragraph*{Exercise 5: Redirecting Output}\label{docs__pandoc__latex__src__gitbash_foundation__gitbash_pre_your_first_terminal__gitbash_pre_your_first_terminal.md__exercise-5-redirecting-output}

\begin{enumerate}
\tightlist
\item
  Create a list: \texttt{ls\ \textgreater{}\ directorylist.txt}
\item
  Open it: \texttt{notepad.exe\ directorylist.txt}
\item
  Read it with your screen reader
\item
  Close Notepad
\item
  Verify the file exists: \texttt{ls\ \textbar{}\ grep\ directory}
\end{enumerate}

Goal: You can save long outputs to files for easier reading

\subsubsection*{Checkpoint Questions}\label{docs__pandoc__latex__src__gitbash_foundation__gitbash_pre_your_first_terminal__gitbash_pre_your_first_terminal.md__checkpoint-questions}

After completing this lesson, you should be able to answer:

\begin{enumerate}
\tightlist
\item
  What does \texttt{pwd} do?
\item
  What does \texttt{ls} do?
\item
  Why do we use \texttt{ls} for listings?
\item
  What path are you in right now?
\item
  How do you navigate to a new folder?
\item
  How do you go up one level?
\item
  What\textquotesingle s the Tab key for?
\item
  What does \texttt{echo\ "text"\ \textgreater{}\ file.txt} do?
\item
  How do you read a file back?
\item
  How do you stop a command that\textquotesingle s running?
\end{enumerate}

You should be able to answer all 10 with confidence before moving to GitBash-0.

\subsubsection*{Common Questions}\label{docs__pandoc__latex__src__gitbash_foundation__gitbash_pre_your_first_terminal__gitbash_pre_your_first_terminal.md__common-questions}

Q: Is Git Bash the same as Command Prompt or PowerShell?
A: No. Git Bash uses Unix/bash commands. CMD and PowerShell use Windows commands. This curriculum teaches bash (Unix) style.

Q: Why is my screen reader not reading the output?
A: This is common. Use \texttt{command\ \textgreater{}\ file.txt} to save output to a file, then open it with Notepad for reliable reading.

Q: What if I type something wrong?
A: Just press Enter and you\textquotesingle ll see an error message. Type the correct command on the next line. No harm done!

Q: How do I get help with a command?
A: Type \texttt{man\ command-name} (we\textquotesingle ll cover this in GitBash-0)

Q: Why do paths look different in Git Bash?
A: Git Bash uses Unix-style paths with forward slashes, not Windows backslashes. Don\textquotesingle t worry - you\textquotesingle ll get used to it quickly.

\subsubsection*{Path Comparison: Windows vs Git Bash}\label{docs__pandoc__latex__src__gitbash_foundation__gitbash_pre_your_first_terminal__gitbash_pre_your_first_terminal.md__path-comparison-windows-vs-git-bash}

{\def\LTcaptype{none} % do not increment counter
\begin{longtable}[]{@{}
  >{\raggedright\arraybackslash}p{(\linewidth - 4\tabcolsep) * \real{0.1389}}
  >{\raggedright\arraybackslash}p{(\linewidth - 4\tabcolsep) * \real{0.3750}}
  >{\raggedright\arraybackslash}p{(\linewidth - 4\tabcolsep) * \real{0.4861}}@{}}
\toprule\noalign{}
\begin{minipage}[b]{\linewidth}\raggedright
Style
\end{minipage} & \begin{minipage}[b]{\linewidth}\raggedright
Example
\end{minipage} & \begin{minipage}[b]{\linewidth}\raggedright
Where Used
\end{minipage} \\
\midrule\noalign{}
\endhead
\bottomrule\noalign{}
\endlastfoot
Windows &
\texttt{C:\textbackslash{}Users\textbackslash{}Name\textbackslash{}Documents}
& CMD, PowerShell, Windows Explorer \\
Git Bash & \texttt{/c/Users/Name/Documents} &
Git Bash, macOS Terminal, Linux \\
\end{longtable}
}

In Git Bash:

\begin{itemize}
\tightlist
\item
  \texttt{C:\textbackslash{}} becomes \texttt{/c/}
\item
  Backslashes \texttt{\textbackslash{}} become forward slashes \texttt{/}
\item
  Otherwise identical
\end{itemize}

\subsubsection*{Next Steps}\label{docs__pandoc__latex__src__gitbash_foundation__gitbash_pre_your_first_terminal__gitbash_pre_your_first_terminal.md__next-steps}

Once you\textquotesingle re comfortable with these basics:

\begin{itemize}
\tightlist
\item
  Move to GitBash-0: Getting Started for deeper path understanding
\item
  Then continue through GitBash-1 through GitBash-5 for full terminal mastery
\end{itemize}

\subsubsection*{Resources}\label{docs__pandoc__latex__src__gitbash_foundation__gitbash_pre_your_first_terminal__gitbash_pre_your_first_terminal.md__resources}

\begin{itemize}
\tightlist
\item
  Git Bash Installation Guide: \url{https://git-scm.com/book/en/v2/Getting-Started-Installing-Git}
\item
  Bash Basics: \url{https://www.gnu.org/software/bash/manual/bash.html\#Basic-Shell-Features}
\item
  NVDA Screen Reader: \url{https://www.nvaccess.org/}
\item
  JAWS Screen Reader: \url{https://www.freedomscientific.com/products/software/jaws/}
\end{itemize}

\subsubsection*{Troubleshooting}\label{docs__pandoc__latex__src__gitbash_foundation__gitbash_pre_your_first_terminal__gitbash_pre_your_first_terminal.md__troubleshooting}

{\def\LTcaptype{none} % do not increment counter
\begin{longtable}[]{@{}
  >{\raggedright\arraybackslash}p{(\linewidth - 2\tabcolsep) * \real{0.2857}}
  >{\raggedright\arraybackslash}p{(\linewidth - 2\tabcolsep) * \real{0.7143}}@{}}
\toprule\noalign{}
\begin{minipage}[b]{\linewidth}\raggedright
Issue
\end{minipage} & \begin{minipage}[b]{\linewidth}\raggedright
Solution
\end{minipage} \\
\midrule\noalign{}
\endhead
\bottomrule\noalign{}
\endlastfoot
Git Bash won\textquotesingle t open &
Make sure it\textquotesingle s installed; search for "Git Bash" in Start menu \\
Can\textquotesingle t hear the output &
Try redirecting to a file: \texttt{command\ \textgreater{}\ output.txt} \\
Tab completion not working &
Make sure you typed at least one character before pressing Tab \\
Command not found &
Make sure you spelled it correctly; try \texttt{man} for available commands \\
Stuck in a command & Press Ctrl+C to stop it \\
\end{longtable}
}

Still stuck? The checkpoint questions and exercises are your best teacher. Work through them multiple times until comfortable.

Other Screen Readers

Dolphin SuperNova (commercial) and Windows Narrator (built-in) are also supported; the workflows and recommendations in this document apply to them. See \url{https://yourdolphin.com/supernova/} and \url{https://support.microsoft.com/narrator} for vendor documentation.

\subsection{GitBash-0: Getting Started - Layout, Paths, and the Shell}\label{docs__pandoc__latex__src__gitbash_foundation__gitbash_0_getting_started_layout_paths__gitbash_0_getting_started_layout_paths.md__gitbash-0-getting-started---layout-paths-and-the-shell}

Estimated time: 20-30 minutes

\subsubsection*{Learning Objectives}\label{docs__pandoc__latex__src__gitbash_foundation__gitbash_0_getting_started_layout_paths__gitbash_0_getting_started_layout_paths.md__learning-objectives}

\begin{itemize}
\tightlist
\item
  Launch Git Bash and locate the prompt
\item
  Understand Unix-style path notation and shortcuts (\texttt{\textasciitilde{}}, \texttt{./}, \texttt{../})
\item
  Use tab completion to navigate quickly
\end{itemize}

\subsubsection*{Materials}\label{docs__pandoc__latex__src__gitbash_foundation__gitbash_0_getting_started_layout_paths__gitbash_0_getting_started_layout_paths.md__materials}

\begin{itemize}
\tightlist
\item
  Computer with Git Bash installed
\item
  Editor (Notepad/VS Code)
\end{itemize}

\subsubsection*{Step-by-step Tasks}\label{docs__pandoc__latex__src__gitbash_foundation__gitbash_0_getting_started_layout_paths__gitbash_0_getting_started_layout_paths.md__step-by-step-tasks}

\begin{enumerate}
\tightlist
\item
  Open Git Bash and note the prompt (it includes the current path).
\item
  Run \texttt{pwd} and say or note the printed path.
\item
  Use \texttt{ls} to list names in your home directory.
\item
  Practice \texttt{cd\ Documents}, \texttt{cd\ ../} and \texttt{cd\ \textasciitilde{}} until comfortable.
\item
  Try tab-completion: type \texttt{cd\ \textasciitilde{}/D} and press Tab.
\end{enumerate}

\subsubsection*{Checkpoints}\label{docs__pandoc__latex__src__gitbash_foundation__gitbash_0_getting_started_layout_paths__gitbash_0_getting_started_layout_paths.md__checkpoints}

\begin{itemize}
\tightlist
\item
  Confirm you can state your current path and move to \texttt{Documents}.
\end{itemize}

\subsubsection*{Quiz - Lesson GitBash.0}\label{docs__pandoc__latex__src__gitbash_foundation__gitbash_0_getting_started_layout_paths__gitbash_0_getting_started_layout_paths.md__quiz---lesson-gitbash0}

\begin{enumerate}
\tightlist
\item
  What is a path?
\item
  What does \texttt{\textasciitilde{}} mean?
\item
  How do you autocomplete a path?
\item
  How do you go up one directory?
\item
  What command lists file names?
\item
  True or False: Git Bash uses backslashes (\texttt{\textbackslash{}}) in paths like Windows CMD.
\item
  Explain the difference between an absolute path and a relative path.
\item
  If you are in \texttt{/c/Users/YourName/Documents} and you type \texttt{cd\ ../}, where do you end up?
\item
  What happens when you press Tab while typing a folder name in Git Bash?
\item
  Describe a practical reason why understanding paths is important for a 3D printing workflow.
\item
  What does \texttt{./} mean in a path, and when would you use it?
\item
  If a folder path contains spaces (e.g., \texttt{My\ Projects}), how do you navigate to it with \texttt{cd}?
\item
  Explain what the prompt \texttt{YourName@COMPUTER\ MINGW64\ \textasciitilde{}/Documents\ \$} tells you about your current state.
\item
  How would you navigate to your home directory from any location using a single command?
\item
  What is the advantage of using relative paths (like \texttt{../}) versus absolute paths in automation scripts?
\end{enumerate}

\subsubsection*{Extension Problems}\label{docs__pandoc__latex__src__gitbash_foundation__gitbash_0_getting_started_layout_paths__gitbash_0_getting_started_layout_paths.md__extension-problems}

\begin{enumerate}
\tightlist
\item
  Create a nested folder and practice \texttt{cd} into it by typing partial names and using Tab.
\item
  Use \texttt{ls\ -la} to list all files including hidden ones in a folder.
\item
  Save \texttt{pwd} output to a file and open it in Notepad.
\item
  Try \texttt{cd} into a folder whose name contains spaces; observe how quotes are handled.
\item
  Create a short note file and open it from Git Bash.
\item
  Build a folder structure that mirrors your project organization; navigate to each level and document the path.
\item
  Create a script that prints your current path and the total number of files in it; run it from different locations.
\item
  Investigate special paths (e.g., \texttt{\$HOME}, \texttt{\$USER}); write down what each contains and when you\textquotesingle d use them.
\item
  Compare absolute vs. relative paths by navigating to the same folder using each method; explain which is easier for automation.
\item
  Create a bash function that changes to a frequently-used folder and lists its contents in one command; test it from different starting locations.
\item
  Navigate to three different locations and at each one note the prompt, the path from \texttt{pwd}, and verify you understand what each shows.
\item
  Create a complex folder tree (at least 5 levels deep) and navigate it using only relative paths; verify your location at each step.
\item
  Document all shortcuts you know (\texttt{\textasciitilde{}}, \texttt{./}, \texttt{../}, \texttt{\$HOME}) and demonstrate each one works as expected.
\item
  Write a guide for a peer on how to understand the Git Bash prompt and path notation without using GUI file explorer.
\item
  Create a troubleshooting flowchart: if someone says "I don\textquotesingle t know where I am," what commands do you give them to find out?
\end{enumerate}

\subsubsection*{References}\label{docs__pandoc__latex__src__gitbash_foundation__gitbash_0_getting_started_layout_paths__gitbash_0_getting_started_layout_paths.md__references}

\begin{itemize}
\tightlist
\item
  GNU. (2024). \emph{Bash Manual}. \url{https://example.com}
\item
  Git SCM. (2024). \emph{Git Bash documentation}. \url{https://example.com}
\item
  Linux Foundation. (2024). \emph{The Linux Command Line}. \url{https://example.com}
\end{itemize}

\subsubsection*{Helpful Resources}\label{docs__pandoc__latex__src__gitbash_foundation__gitbash_0_getting_started_layout_paths__gitbash_0_getting_started_layout_paths.md__helpful-resources}

\begin{itemize}
\tightlist
\item
  \href{https://www.gnu.org/software/bash/manual/htmlnode/Bourne-Shell-Builtins.html}{Bash Manual - Navigation}
\item
  \href{https://git-scm.com/book/en/v2/Git-Basics-Getting-a-Git-Repository}{Git Bash Basics}
\item
  \href{https://linuxcommand.org/lc3lts0020.php}{Linux Path Guide}
\item
  \href{https://www.gnu.org/software/bash/manual/htmlnode/Controlling-the-Prompt.html}{Understanding Bash Prompts}
\item
  \href{https://www.gnu.org/software/bash/manual/htmlnode/Commands-For-Completion.html}{Tab Completion in Bash}
\end{itemize}

\subsection{GitBash-1: Navigation - Moving Around Your File System}\label{docs__pandoc__latex__src__gitbash_foundation__gitbash_1_navigation__gitbash_1_navigation.md__gitbash-1-navigation---moving-around-your-file-system}

\textbf{Duration:} 1 class period
\textbf{Prerequisite:} GitBash-0 (Getting Started)

\subsubsection*{Learning Objectives}\label{docs__pandoc__latex__src__gitbash_foundation__gitbash_1_navigation__gitbash_1_navigation.md__learning-objectives}

By the end of this lesson, you will be able to:

\begin{itemize}
\tightlist
\item
  Use \texttt{pwd} to print your current location
\item
  Use \texttt{cd} to move between directories
\item
  Use \texttt{ls} (and its flags) to list files and folders
\item
  Use wildcards \texttt{*} and \texttt{?} to filter listings
\item
  Navigate relative vs. absolute paths
\item
  Search for files by name and extension
\end{itemize}

\subsubsection*{Materials}\label{docs__pandoc__latex__src__gitbash_foundation__gitbash_1_navigation__gitbash_1_navigation.md__materials}

\begin{itemize}
\tightlist
\item
  Git Bash
\item
  Text editor (Notepad or VS Code)
\end{itemize}

\begin{center}\rule{0.5\linewidth}{0.5pt}\end{center}

\subsubsection*{Commands Covered in This Lesson}\label{docs__pandoc__latex__src__gitbash_foundation__gitbash_1_navigation__gitbash_1_navigation.md__commands-covered-in-this-lesson}

{\def\LTcaptype{none} % do not increment counter
\begin{longtable}[]{@{}
  >{\raggedright\arraybackslash}p{(\linewidth - 2\tabcolsep) * \real{0.2432}}
  >{\raggedright\arraybackslash}p{(\linewidth - 2\tabcolsep) * \real{0.7568}}@{}}
\toprule\noalign{}
\begin{minipage}[b]{\linewidth}\raggedright
Command
\end{minipage} & \begin{minipage}[b]{\linewidth}\raggedright
What It Does
\end{minipage} \\
\midrule\noalign{}
\endhead
\bottomrule\noalign{}
\endlastfoot
\texttt{pwd} & Print Working Directory - shows where you are \\
\texttt{cd\ path} & Change Directory - move to a new location \\
\texttt{ls} & List - shows files and folders in current location \\
\texttt{ls\ -1} &
List names only, one per line (screen reader friendly) \\
\texttt{ls\ -1\ -F} &
List names with type indicators (/ for directories) \\
\texttt{ls\ *.extension} & List files matching a pattern \\
\end{longtable}
}

\begin{center}\rule{0.5\linewidth}{0.5pt}\end{center}

\subsubsection*{\texorpdfstring{\texttt{pwd} - Where Am I?}{pwd - Where Am I?}}\label{docs__pandoc__latex__src__gitbash_foundation__gitbash_1_navigation__gitbash_1_navigation.md__pwd---where-am-i}

Type \texttt{pwd} and press \texttt{Enter}. Git Bash prints the full path to your current location.

\begin{lstlisting}[style=Alabaster, language=bash]
pwd
# Output: /c/Users/YourName

\end{lstlisting}

\textbf{When to use:} Always run this if you\textquotesingle re unsure of your current location.

\begin{center}\rule{0.5\linewidth}{0.5pt}\end{center}

\subsubsection*{\texorpdfstring{\texttt{cd} - Changing Directories}{cd - Changing Directories}}\label{docs__pandoc__latex__src__gitbash_foundation__gitbash_1_navigation__gitbash_1_navigation.md__cd---changing-directories}

\texttt{cd} stands for "change directory."

\begin{lstlisting}[style=Alabaster, language=bash]
# Go to Documents
cd Documents

# Go up one level to parent directory
cd ..

# Go to root of file system
cd /

# Go to home directory
cd ~

# Go to a specific path
cd /c/Users/YourName/Documents/3D_Projects

# Go to previous directory
cd -

\end{lstlisting}

\begin{center}\rule{0.5\linewidth}{0.5pt}\end{center}

\subsubsection*{\texorpdfstring{\texttt{ls} - Listing Files and Folders}{ls - Listing Files and Folders}}\label{docs__pandoc__latex__src__gitbash_foundation__gitbash_1_navigation__gitbash_1_navigation.md__ls---listing-files-and-folders}

Use \texttt{ls\ -1} for screen reader compatibility --- names only, one per line.

\begin{lstlisting}[style=Alabaster, language=bash]
# List all files and folders (names only, one per line)
ls -1

# List only files (no hidden, no directories)
ls -1 -p | grep -v /

# List only directories
ls -1 -d */

\end{lstlisting}

\begin{center}\rule{0.5\linewidth}{0.5pt}\end{center}

\subsubsection*{Wildcards - Finding Files by Pattern}\label{docs__pandoc__latex__src__gitbash_foundation__gitbash_1_navigation__gitbash_1_navigation.md__wildcards---finding-files-by-pattern}

Wildcards help you find files without typing the full name.

\textbf{\texttt{*} (asterisk)} matches any number of characters:

\begin{lstlisting}[style=Alabaster, language=bash]
# List all .scad files
ls *.scad

# List all files starting with "part"
ls part*

# List all files ending with "_final"
ls *_final*

\end{lstlisting}

\textbf{\texttt{?} (question mark)} matches exactly one character:

\begin{lstlisting}[style=Alabaster, language=bash]
# Find files like model1.scad, model2.scad (but not model12.scad)
ls model?.scad

\end{lstlisting}

\begin{center}\rule{0.5\linewidth}{0.5pt}\end{center}

\subsubsection*{Step-by-step Practice}\label{docs__pandoc__latex__src__gitbash_foundation__gitbash_1_navigation__gitbash_1_navigation.md__step-by-step-practice}

\begin{enumerate}
\tightlist
\item
  Run \texttt{pwd} and confirm your location
\item
  Move to \texttt{Documents}: \texttt{cd\ Documents}
\item
  Confirm you moved: \texttt{pwd}
\item
  List files and folders: \texttt{ls\ -1}
\item
  List only files: \texttt{ls\ -1\ -p\ \textbar{}\ grep\ -v\ /}
\item
  Go back up: \texttt{cd\ ..}
\item
  Search for files: \texttt{ls\ *.txt}
\end{enumerate}

\begin{center}\rule{0.5\linewidth}{0.5pt}\end{center}

\subsubsection*{Checkpoints}\label{docs__pandoc__latex__src__gitbash_foundation__gitbash_1_navigation__gitbash_1_navigation.md__checkpoints}

After this lesson, you should be able to:

\begin{itemize}
\tightlist
\item[$\square$]
  Navigate to any folder using \texttt{cd}
\item[$\square$]
  Confirm your location with \texttt{pwd}
\item[$\square$]
  List files and folders with \texttt{ls\ -1}
\item[$\square$]
  Use wildcards to find files by pattern
\item[$\square$]
  Move between absolute and relative paths confidently
\end{itemize}

\subsubsection*{Quiz - Lesson GitBash.1}\label{docs__pandoc__latex__src__gitbash_foundation__gitbash_1_navigation__gitbash_1_navigation.md__quiz---lesson-gitbash1}

\begin{enumerate}
\tightlist
\item
  What does \texttt{pwd} show?
\item
  How do you list directories only with \texttt{ls}?
\item
  What wildcard matches any number of characters?
\item
  How do you list files with the \texttt{.scad} extension?
\item
  Give an example of an absolute path and a relative path.
\item
  True or False: The \texttt{*} wildcard matches exactly one character.
\item
  Explain the difference between \texttt{ls\ -1} and \texttt{ls\ -1\ -d\ */}.
\item
  Write a command that would list all \texttt{.txt} files in your Documents folder using a wildcard.
\item
  How would you search for files containing "part" in their name across multiple files?
\item
  Describe a practical scenario where using wildcards saves time in a 3D printing workflow.
\item
  What happens when you use \texttt{ls\ model?.scad} versus \texttt{ls\ model*.scad}?
\item
  How would you navigate to a folder whose name contains both spaces and special characters?
\item
  If you\textquotesingle re in \texttt{\textasciitilde{}/Documents/Projects/3D} and you want to go to \texttt{\textasciitilde{}/Documents/Resources}, what command would you use?
\item
  Write a command sequence that navigates to the Downloads folder, lists only files, then returns to home.
\item
  Explain the purpose of using \texttt{ls\ -1} specifically in a screen reader context.
\end{enumerate}

\subsubsection*{Extension Problems}\label{docs__pandoc__latex__src__gitbash_foundation__gitbash_1_navigation__gitbash_1_navigation.md__extension-problems}

\begin{enumerate}
\tightlist
\item
  Write a one-line command that lists \texttt{.scad} files and saves to \texttt{scad\_list.txt}.
\item
  Use \texttt{ls\ -1\ \textasciitilde{}/Documents\ \textbar{}\ less} to page through long listings.
\item
  Combine \texttt{ls} with \texttt{grep} to search for a filename pattern.
\item
  Create a shortcut alias in the session for a long path and test it.
\item
  Practice tab-completion in a directory with many similarly named files.
\item
  Build a bash script that recursively lists all \texttt{.scad} and \texttt{.stl} files in a directory tree; save the results to a file.
\item
  Compare the output of \texttt{ls}, \texttt{ls\ -1}, \texttt{ls\ -la}, and \texttt{ls\ -1\ -d\ */} to understand the flags; document what each command does.
\item
  Create a filtering command that displays only files modified in the last 7 days; test it on your documents folder.
\item
  Write a non-visual guide to Git Bash navigation; include descriptions of common patterns and how to verify directory contents audibly.
\item
  Develop a navigation workflow for a typical 3D printing project: move between CAD, slicing, and print-log folders efficiently; document the commands.
\item
  Create a complex wildcard search: find all files in a folder and subfolders that match multiple patterns (e.g., \texttt{*\_v1.*} or \texttt{*\_final.*}).
\item
  Build a script that navigates through a folder tree, counts files at each level, and reports the structure.
\item
  Document the output differences between \texttt{ls}, \texttt{ls\ -1}, \texttt{ls\ -la}, and \texttt{ls\ -1\ -d\ */}; explain when to use each.
\item
  Create a navigation "cheat sheet" as a bash script that prints common paths and how to navigate to them.
\item
  Design a project folder structure on your computer, document each path, then create a script that validates all folders exist.
\end{enumerate}

\subsubsection*{References}\label{docs__pandoc__latex__src__gitbash_foundation__gitbash_1_navigation__gitbash_1_navigation.md__references}

\begin{itemize}
\tightlist
\item
  GNU. (2024). \emph{ls command reference}. \url{https://example.com}
\item
  GNU. (2024). \emph{Bash wildcards and globbing}. \url{https://example.com}
\item
  GNU. (2024). \emph{Navigation best practices in Bash}. \url{https://example.com}
\end{itemize}

\subsubsection*{Helpful Resources}\label{docs__pandoc__latex__src__gitbash_foundation__gitbash_1_navigation__gitbash_1_navigation.md__helpful-resources}

\begin{itemize}
\tightlist
\item
  \href{https://www.gnu.org/software/coreutils/manual/html_node/ls-invocation.html}{ls Command Reference}
\item
  \href{https://www.gnu.org/software/bash/manual/htmlnode/Pattern-Matching.html}{Bash Wildcards and Globbing}
\item
  \href{https://linuxcommand.org/lc3lts0020.php}{Navigation Best Practices}
\item
  \href{https://linuxize.com/post/linux-cd-command/}{Relative and Absolute Paths in Bash}
\item
  \href{https://www.nvaccess.org/documentation/}{Screen Reader Tips for Git Bash}
\end{itemize}

\subsection{GitBash-2: File and Folder Manipulation}\label{docs__pandoc__latex__src__gitbash_foundation__gitbash_2_file_folder_manipulation_modification__gitbash_2_file_folder_manipulation_modification.md__gitbash-2-file-and-folder-manipulation}

Estimated time: 30-45 minutes

\subsubsection*{Learning Objectives}\label{docs__pandoc__latex__src__gitbash_foundation__gitbash_2_file_folder_manipulation_modification__gitbash_2_file_folder_manipulation_modification.md__learning-objectives}

\begin{itemize}
\tightlist
\item
  Create, copy, move, and delete files and folders from Git Bash
\item
  Use \texttt{touch}, \texttt{mkdir}, \texttt{cp}, \texttt{mv}, \texttt{rm}, and \texttt{rmdir} safely
\item
  Understand when operations are permanent and how to confirm results
\end{itemize}

\subsubsection*{Materials}\label{docs__pandoc__latex__src__gitbash_foundation__gitbash_2_file_folder_manipulation_modification__gitbash_2_file_folder_manipulation_modification.md__materials}

\begin{itemize}
\tightlist
\item
  Git Bash
\item
  Small practice folder for exercises
\end{itemize}

\subsubsection*{Step-by-step Tasks}\label{docs__pandoc__latex__src__gitbash_foundation__gitbash_2_file_folder_manipulation_modification__gitbash_2_file_folder_manipulation_modification.md__step-by-step-tasks}

\begin{enumerate}
\tightlist
\item
  Create a practice directory: \texttt{mkdir\ \textasciitilde{}/Documents/GitBash\_Practice} and \texttt{cd} into it.
\item
  Create two files: \texttt{touch\ file1.txt} and \texttt{touch\ file2.txt}.
\item
  Copy \texttt{file1.txt} to \texttt{file1\_backup.txt} with \texttt{cp} and confirm with \texttt{ls\ -1}.
\item
  Rename \texttt{file2.txt} to \texttt{notes.txt} using \texttt{mv} and confirm.
\item
  Delete \texttt{file1.txt} with \texttt{rm} and verify the backup remains.
\end{enumerate}

\subsubsection*{Checkpoints}\label{docs__pandoc__latex__src__gitbash_foundation__gitbash_2_file_folder_manipulation_modification__gitbash_2_file_folder_manipulation_modification.md__checkpoints}

\begin{itemize}
\tightlist
\item
  After step 3 you should see both the original and the backup file.
\end{itemize}

\subsubsection*{Quiz - Lesson GitBash.2}\label{docs__pandoc__latex__src__gitbash_foundation__gitbash_2_file_folder_manipulation_modification__gitbash_2_file_folder_manipulation_modification.md__quiz---lesson-gitbash2}

\begin{enumerate}
\tightlist
\item
  How do you create an empty file from Git Bash?
\item
  What command copies a file?
\item
  How do you rename a file?
\item
  What does \texttt{rm\ -r} do?
\item
  Why is \texttt{rm} potentially dangerous?
\item
  True or False: \texttt{cp} requires the \texttt{-r} flag to copy both files and folders.
\item
  Explain the difference between \texttt{rm} and \texttt{rmdir}.
\item
  If you delete a file with \texttt{rm}, can you recover it from Git Bash?
\item
  Write a command that would copy an entire folder and all its contents to a new location.
\item
  Describe a practical safety check you would perform before running \texttt{rm\ -r} on a folder.
\item
  What happens if you \texttt{cp} a file to a destination where a file with the same name already exists? How would you handle this safely?
\item
  Compare \texttt{mv\ old\_name.txt\ new\_name.txt} vs \texttt{mv\ old\_name.txt\ \textasciitilde{}/Documents/new\_name.txt}. What is the key difference?
\item
  Design a workflow to safely delete 50 files matching the pattern \texttt{*.bak} from a folder containing 500 files. What commands and verifications would you use?
\item
  Explain how you could back up all \texttt{.scad} files from a project folder into a timestamped backup folder in one command.
\item
  When organizing a 3D printing project, you need to move completed designs to an archive folder and delete failed prototypes. How would you structure this as a safe, auditable process?
\end{enumerate}

\subsubsection*{Extension Problems}\label{docs__pandoc__latex__src__gitbash_foundation__gitbash_2_file_folder_manipulation_modification__gitbash_2_file_folder_manipulation_modification.md__extension-problems}

\begin{enumerate}
\tightlist
\item
  Create a folder tree and copy it to a new location with \texttt{cp\ -r}.
\item
  Write a one-line command that creates three files named \texttt{a}, \texttt{b}, \texttt{c} and lists them.
\item
  Move a file into a new folder and confirm the move.
\item
  Use wildcards to delete files matching a pattern in a safe test folder.
\item
  Export a listing of the practice folder to \texttt{practice\_listing.txt}.
\item
  Create a backup shell script that copies all \texttt{.scad} files from your project folder to a backup folder with timestamp naming.
\item
  Build a safe deletion workflow: list files matching a pattern, verify count, then delete with confirmation; document the steps.
\item
  Write a bash script that organizes files by extension into subfolders; test it on a sample folder tree.
\item
  Create a file operation audit trail: log all copy, move, and delete operations to a text file for review.
\item
  Develop a project template generator: a bash script that creates a standard folder structure for a new 3D printing project with essential subfolders.
\item
  Implement a file conflict handler: write a bash script that handles cases where \texttt{cp} would overwrite an existing file by renaming the existing file with a timestamp before copying.
\item
  Create a batch rename operation: use a script to rename all files in a folder from \texttt{old\_prefix\_*} to \texttt{new\_prefix\_*}; test with actual files and verify the results.
\item
  Build a folder comparison tool: list all files in two folders and identify which files exist in one but not the other; output to a report.
\item
  Write a destructive operation validator: before executing \texttt{rm\ -r}, create a script that lists exactly what will be deleted, shows file counts by type, and requires explicit user confirmation to proceed.
\item
  Design a complete project lifecycle workflow: create folders for active projects, completed designs, and archive; include move operations between folders, backup steps, and verification that all files arrive intact.
\end{enumerate}

\subsubsection*{References}\label{docs__pandoc__latex__src__gitbash_foundation__gitbash_2_file_folder_manipulation_modification__gitbash_2_file_folder_manipulation_modification.md__references}

\begin{itemize}
\tightlist
\item
  GNU. (2024). \emph{touch command reference}. \url{https://example.com}
\item
  GNU. (2024). \emph{cp and mv commands}. \url{https://example.com}
\item
  GNU. (2024). \emph{File system operations guide}. \url{https://example.com}
\end{itemize}

\subsubsection*{Helpful Resources}\label{docs__pandoc__latex__src__gitbash_foundation__gitbash_2_file_folder_manipulation_modification__gitbash_2_file_folder_manipulation_modification.md__helpful-resources}

\begin{itemize}
\tightlist
\item
  \href{https://www.gnu.org/software/coreutils/manual/html_node/touch-invocation.html}{touch Command Reference}
\item
  \href{https://www.gnu.org/software/coreutils/manual/html_node/cp-invocation.html}{cp Command Reference}
\item
  \href{https://www.gnu.org/software/coreutils/manual/html_node/mv-invocation.html}{mv Command Reference}
\item
  \href{https://www.gnu.org/software/coreutils/manual/html_node/rm-invocation.html}{rm Command Reference}
\item
  \href{https://linuxcommand.org/lc3lts0050.php}{Safe Deletion Practices}
\end{itemize}

\subsection{GitBash-3: Input, Output, and Piping}\label{docs__pandoc__latex__src__gitbash_foundation__gitbash_3_input_output_piping__gitbash_3_input_output_piping.md__gitbash-3-input-output-and-piping}

\textbf{Duration:} 1 class period
\textbf{Prerequisite:} GitBash-2 (File and Folder Manipulation)

\begin{center}\rule{0.5\linewidth}{0.5pt}\end{center}

\subsubsection*{Learning Objectives}\label{docs__pandoc__latex__src__gitbash_foundation__gitbash_3_input_output_piping__gitbash_3_input_output_piping.md__learning-objectives}

By the end of this lesson, you will be able to:

\begin{itemize}
\tightlist
\item
  Use \texttt{echo} to print text to the screen
\item
  Use \texttt{cat} to read file contents
\item
  Use \texttt{\textgreater{}} to redirect output into a file
\item
  Use \texttt{\textbar{}} (pipe) to send one command\textquotesingle s output to another
\item
  Copy output to the clipboard with \texttt{clip}
\item
  Open files with a text editor from the command line
\end{itemize}

\begin{center}\rule{0.5\linewidth}{0.5pt}\end{center}

\subsubsection*{Commands Covered}\label{docs__pandoc__latex__src__gitbash_foundation__gitbash_3_input_output_piping__gitbash_3_input_output_piping.md__commands-covered}

{\def\LTcaptype{none} % do not increment counter
\begin{longtable}[]{@{}
  >{\raggedright\arraybackslash}p{(\linewidth - 2\tabcolsep) * \real{0.3077}}
  >{\raggedright\arraybackslash}p{(\linewidth - 2\tabcolsep) * \real{0.6923}}@{}}
\toprule\noalign{}
\begin{minipage}[b]{\linewidth}\raggedright
Command
\end{minipage} & \begin{minipage}[b]{\linewidth}\raggedright
What It Does
\end{minipage} \\
\midrule\noalign{}
\endhead
\bottomrule\noalign{}
\endlastfoot
\texttt{echo\ "text"} & Print text to the screen \\
\texttt{cat\ filename} & Print the contents of a file \\
\texttt{\textgreater{}\ filename} &
Redirect output into a file (overwrites) \\
\texttt{\textgreater{}\textgreater{}\ filename} &
Append output to a file (adds to end) \\
\texttt{\textbar{}} & Pipe - send output from one command to the next \\
\texttt{clip} & Copy piped input to the Windows clipboard (Git Bash) \\
\texttt{notepad.exe\ filename} & Open a file in Notepad \\
\end{longtable}
}

\begin{center}\rule{0.5\linewidth}{0.5pt}\end{center}

\subsubsection*{\texorpdfstring{\texttt{echo} - Printing Text}{echo - Printing Text}}\label{docs__pandoc__latex__src__gitbash_foundation__gitbash_3_input_output_piping__gitbash_3_input_output_piping.md__echo---printing-text}

\texttt{echo} prints text to the screen. It is useful for testing, for writing text into files, and for understanding how piping works.

\begin{lstlisting}[style=Alabaster, language=bash]
echo "Hello, World"
echo "This is a test"

\end{lstlisting}

\begin{center}\rule{0.5\linewidth}{0.5pt}\end{center}

\subsubsection*{\texorpdfstring{\texttt{cat} - Reading Files}{cat - Reading Files}}\label{docs__pandoc__latex__src__gitbash_foundation__gitbash_3_input_output_piping__gitbash_3_input_output_piping.md__cat---reading-files}

\texttt{cat} prints the contents of a file to the screen.

\begin{lstlisting}[style=Alabaster, language=bash]
# Read a text file
cat ~/Documents/notes.txt

# Read an OpenSCAD file
cat ~/Documents/OpenSCAD_Projects/project0.scad

\end{lstlisting}

With a long file, use \texttt{cat\ filename\ \textbar{}\ less} to read it page by page (press \texttt{Space} to advance, \texttt{Q} to quit).

\begin{center}\rule{0.5\linewidth}{0.5pt}\end{center}

\subsubsection*{\texorpdfstring{\texttt{\textgreater{}} - Redirecting Output to a File}{\textgreater{} - Redirecting Output to a File}}\label{docs__pandoc__latex__src__gitbash_foundation__gitbash_3_input_output_piping__gitbash_3_input_output_piping.md__---redirecting-output-to-a-file}

The \texttt{\textgreater{}} symbol redirects output from the screen into a file instead.

\begin{lstlisting}[style=Alabaster, language=bash]
# Create a file with a single line
echo "Author: Your Name" > header.txt

# Confirm the file was created and has content
cat header.txt

\end{lstlisting}

\textbf{Warning:} \texttt{\textgreater{}} overwrites the file if it already exists. Use \texttt{\textgreater{}\textgreater{}} to append instead:

\begin{lstlisting}[style=Alabaster, language=bash]
echo "Date: 2025" >> header.txt
echo "Project: Floor Marker" >> header.txt
cat header.txt

\end{lstlisting}

\begin{center}\rule{0.5\linewidth}{0.5pt}\end{center}

\subsubsection*{\texorpdfstring{\texttt{\textbar{}} - Piping}{\textbar{} - Piping}}\label{docs__pandoc__latex__src__gitbash_foundation__gitbash_3_input_output_piping__gitbash_3_input_output_piping.md__---piping}

The pipe symbol \texttt{\textbar{}} sends the output of one command to the input of the next. This lets you chain commands together.

\begin{lstlisting}[style=Alabaster, language=bash]
# List files and send the list to clip (copies to clipboard)
ls -1 | clip

# Now paste with Ctrl + V anywhere

\end{lstlisting}

\begin{lstlisting}[style=Alabaster, language=bash]
# Search within a file's contents using grep
cat project0.scad | grep "cube"

\end{lstlisting}

\begin{center}\rule{0.5\linewidth}{0.5pt}\end{center}

\subsubsection*{\texorpdfstring{\texttt{clip} - Copying to Clipboard}{clip - Copying to Clipboard}}\label{docs__pandoc__latex__src__gitbash_foundation__gitbash_3_input_output_piping__gitbash_3_input_output_piping.md__clip---copying-to-clipboard}

\texttt{clip} takes whatever is piped to it and puts it on the Windows clipboard.

\begin{lstlisting}[style=Alabaster, language=bash]
# Copy your current directory path to the clipboard
pwd | clip

# Copy a file listing to clipboard
ls -1 | clip

# Copy the contents of a file to clipboard
cat notes.txt | clip

\end{lstlisting}

After any of these, press \texttt{Ctrl\ +\ V} in any application to paste.

\begin{center}\rule{0.5\linewidth}{0.5pt}\end{center}

\subsubsection*{Opening Files in Notepad}\label{docs__pandoc__latex__src__gitbash_foundation__gitbash_3_input_output_piping__gitbash_3_input_output_piping.md__opening-files-in-notepad}

\begin{lstlisting}[style=Alabaster, language=bash]
# Open a file in Notepad
notepad.exe ~/Documents/notes.txt

# Open a .scad file
notepad.exe ~/Documents/OpenSCAD_Projects/project0.scad

# Create a new file and open it
touch new_notes.txt
notepad.exe new_notes.txt

\end{lstlisting}

\begin{center}\rule{0.5\linewidth}{0.5pt}\end{center}

\subsubsection*{Step-by-step Tasks}\label{docs__pandoc__latex__src__gitbash_foundation__gitbash_3_input_output_piping__gitbash_3_input_output_piping.md__step-by-step-tasks}

\begin{enumerate}
\tightlist
\item
  Create \texttt{practice.txt} with three lines using \texttt{echo} and \texttt{\textgreater{}}/\texttt{\textgreater{}\textgreater{}}.
\item
  Read the file with \texttt{cat\ practice.txt}.
\item
  Pipe the file into \texttt{grep} to search for a word.
\item
  Copy the file contents to clipboard with \texttt{cat\ practice.txt\ \textbar{}\ clip}.
\item
  Redirect \texttt{ls\ -1} into \texttt{list.txt} and open it in Notepad.
\end{enumerate}

\subsubsection*{Checkpoints}\label{docs__pandoc__latex__src__gitbash_foundation__gitbash_3_input_output_piping__gitbash_3_input_output_piping.md__checkpoints}

\begin{itemize}
\tightlist
\item
  After step 3 you should be able to find a keyword using piping.
\end{itemize}

\subsubsection*{Quiz - Lesson GitBash.3}\label{docs__pandoc__latex__src__gitbash_foundation__gitbash_3_input_output_piping__gitbash_3_input_output_piping.md__quiz---lesson-gitbash3}

\begin{enumerate}
\tightlist
\item
  What is the difference between \texttt{\textgreater{}} and \texttt{\textgreater{}\textgreater{}}?
\item
  What does the pipe \texttt{\textbar{}} do?
\item
  How do you copy output to the clipboard?
\item
  How would you page through long output?
\item
  How do you suppress output (send it to \texttt{/dev/null})?
\item
  True or False: The pipe operator \texttt{\textbar{}} connects the output of one command to the input of another.
\item
  Explain why redirecting output to a file is useful for screen reader users.
\item
  Write a command that would search for the word "sphere" in all \texttt{.scad} files in a directory.
\item
  How would you count the number of lines in a file using bash piping?
\item
  Describe a practical scenario in 3D printing where you would pipe or redirect command output.
\item
  What would be the difference in output between \texttt{echo\ "test"\ \textgreater{}\ file.txt} (run twice) vs \texttt{echo\ "test"\ \textgreater{}\textgreater{}\ file.txt} (run twice)? Show the expected file contents.
\item
  Design a three-step piping chain: read a file, filter for specific content, and save the results; explain what each pipe does.
\item
  You have a 500-line \texttt{.scad} file and need to find all instances of \texttt{sphere()} and count them. Write the command.
\item
  Explain how \texttt{clip} is particularly valuable for screen reader users when working with file paths or long output strings.
\item
  Describe how you would use pipes and redirection to create a timestamped backup report of all \texttt{.stl} files in a 3D printing project folder.
\end{enumerate}

\subsubsection*{Extension Problems}\label{docs__pandoc__latex__src__gitbash_foundation__gitbash_3_input_output_piping__gitbash_3_input_output_piping.md__extension-problems}

\begin{enumerate}
\tightlist
\item
  Use piping to count lines in a file (hint: \texttt{cat\ file.txt\ \textbar{}\ wc\ -l}).
\item
  Save a long \texttt{ls\ -1} output and search it with \texttt{grep}.
\item
  Chain multiple pipes to filter and then save results.
\item
  Practice copying different command outputs to clipboard and pasting.
\item
  Create a small bash script that generates a report (counts of files by extension).
\item
  Build a data processing pipeline: read a text file, filter rows, and export results; document each step.
\item
  Write a script that pipes directory listing to count occurrences of each file extension; create a summary report.
\item
  Create a log analysis command: read a log file, filter for errors, and save matching lines to a separate error log.
\item
  Design a piping workflow for 3D printing file management: find \texttt{.stl} files, extract their names, and generate a report.
\item
  Develop a reusable piping function library: create bash functions for common filtering, sorting, and reporting patterns; test each with different inputs.
\item
  Build a complex filter pipeline: read a \texttt{.scad} file, extract lines containing specific geometry commands, count each type, and output a summary to both screen and file.
\item
  Create an interactive filtering tool: build a bash script that accepts a search term, pipes through multiple filters, and displays paginated results.
\item
  Develop a performance analysis tool: use piping to combine file listing, metadata extraction, and statistical reporting; export results to a dated report file.
\item
  Implement a comprehensive error-handling pipeline: read output, catch errors, log them separately, and generate a summary of successes vs failures.
\item
  Design and execute a real-world project backup workflow: use piping to verify file existence, count files by type, generate a backup manifest, and create audit logs --- all in one integrated command pipeline.
\end{enumerate}

\subsubsection*{References}\label{docs__pandoc__latex__src__gitbash_foundation__gitbash_3_input_output_piping__gitbash_3_input_output_piping.md__references}

\begin{itemize}
\tightlist
\item
  GNU. (2024). \emph{Redirections in Bash}. \url{https://example.com}
\item
  GNU. (2024). \emph{grep command reference}. \url{https://example.com}
\item
  GNU. (2024). \emph{Bash pipeline concepts}. \url{https://example.com}
\end{itemize}

\subsubsection*{Helpful Resources}\label{docs__pandoc__latex__src__gitbash_foundation__gitbash_3_input_output_piping__gitbash_3_input_output_piping.md__helpful-resources}

\begin{itemize}
\tightlist
\item
  \href{https://www.gnu.org/software/bash/manual/htmlnode/Redirections.html}{Bash Redirections}
\item
  \href{https://www.gnu.org/software/grep/manual/grep.html}{Piping and grep}
\item
  \href{https://www.gnu.org/software/coreutils/manual/html_node/cat-invocation.html}{cat Command Reference}
\item
  \href{https://www.gnu.org/software/coreutils/manual/html_node/wc-invocation.html}{wc for Counting}
\item
  \href{https://www.gnu.org/software/bash/manual/htmlnode/Pipelines.html}{Bash Pipeline Concepts}
\end{itemize}

\subsection{GitBash-4: Environment Variables, PATH, and Aliases}\label{docs__pandoc__latex__src__gitbash_foundation__gitbash_4_environment_variables_aliases__gitbash_4_environment_variables_aliases.md__gitbash-4-environment-variables-path-and-aliases}

Estimated time: 30-45 minutes

\subsubsection*{Learning Objectives}\label{docs__pandoc__latex__src__gitbash_foundation__gitbash_4_environment_variables_aliases__gitbash_4_environment_variables_aliases.md__learning-objectives}

\begin{itemize}
\tightlist
\item
  Read environment variables with \texttt{\$VARNAME}
\item
  Inspect and verify programs in the \texttt{PATH}
\item
  Create temporary aliases and understand making them persistent via \texttt{.bashrc}
\end{itemize}

\subsubsection*{Materials}\label{docs__pandoc__latex__src__gitbash_foundation__gitbash_4_environment_variables_aliases__gitbash_4_environment_variables_aliases.md__materials}

\begin{itemize}
\tightlist
\item
  Git Bash (with rights to edit \texttt{.bashrc})
\end{itemize}

\subsubsection*{Step-by-step Tasks}\label{docs__pandoc__latex__src__gitbash_foundation__gitbash_4_environment_variables_aliases__gitbash_4_environment_variables_aliases.md__step-by-step-tasks}

\begin{enumerate}
\tightlist
\item
  Show your username and home path with \texttt{echo\ \$USER} and \texttt{echo\ \$HOME}.
\item
  Inspect \texttt{echo\ \$PATH} and identify whether \texttt{openscad} or \texttt{code} would be found.
\item
  Run \texttt{which\ openscad} and note the result.
\item
  Create a temporary alias: \texttt{alias\ preview=\textquotesingle{}openscad\textquotesingle{}} and test \texttt{preview\ myfile.scad}.
\item
  Open your profile (\texttt{notepad.exe\ \textasciitilde{}/.bashrc}) and add the alias line to make it persistent.
\end{enumerate}

\subsubsection*{Checkpoints}\label{docs__pandoc__latex__src__gitbash_foundation__gitbash_4_environment_variables_aliases__gitbash_4_environment_variables_aliases.md__checkpoints}

\begin{itemize}
\tightlist
\item
  After step 3 you can determine whether a program will be found by \texttt{PATH}.
\end{itemize}

\subsubsection*{Quiz - Lesson GitBash.4}\label{docs__pandoc__latex__src__gitbash_foundation__gitbash_4_environment_variables_aliases__gitbash_4_environment_variables_aliases.md__quiz---lesson-gitbash4}

\begin{enumerate}
\tightlist
\item
  How do you print an environment variable?
\item
  What is the purpose of \texttt{PATH}?
\item
  How do you check whether \texttt{openscad} is available?
\item
  How do you create a temporary alias?
\item
  Where would you make an alias permanent?
\item
  True or False: Environment variables in bash are case-sensitive.
\item
  Explain why having a program in your PATH is useful compared to always using its full file path.
\item
  Write a command that would create an alias called \texttt{slicer} for the OpenSCAD executable.
\item
  What file would you edit to make an alias persist across Git Bash sessions?
\item
  Describe a practical benefit of using the \texttt{\$TMPDIR} or \texttt{/tmp} directory for temporary files in a 3D printing workflow.
\item
  You have a custom script at \texttt{\textasciitilde{}/scripts/backup\_models.sh} that you want to run from anywhere as \texttt{backup-now}. What steps would you take to make this work?
\item
  Explain the difference between setting an environment variable in the current session with \texttt{export} vs. adding it to \texttt{.bashrc} for permanence.
\item
  Design a \texttt{.bashrc} strategy for managing multiple 3D printing projects, each with different tool paths and directories; show how to structure environment variables for each.
\item
  If a program is not found by \texttt{which}, what are the possible reasons, and how would you troubleshoot?
\item
  Describe how you would verify that your \texttt{.bashrc} is loading correctly and how to debug issues if aliases or environment variables don\textquotesingle t appear after restarting Git Bash.
\end{enumerate}

\subsubsection*{Extension Problems}\label{docs__pandoc__latex__src__gitbash_foundation__gitbash_4_environment_variables_aliases__gitbash_4_environment_variables_aliases.md__extension-problems}

\begin{enumerate}
\tightlist
\item
  Add a folder to PATH for a test program (describe steps; do not change system PATH without admin).
\item
  Create a short \texttt{.bashrc} snippet that sets two aliases and test re-opening Git Bash.
\item
  Use \texttt{which} to list the path for several common programs.
\item
  Explore \texttt{\$TMPDIR} or \texttt{/tmp} and create a file there.
\item
  Save a copy of your current PATH to a text file and examine it in your editor.
\item
  Create a \texttt{.bashrc} script that loads custom aliases and environment variables for your 3D printing workflow; test it in a new session.
\item
  Build a "project profile" that sets environment variables for CAD, slicing, and print directories; switch between profiles for different projects.
\item
  Write a script that audits your current environment variables and creates a summary report of what\textquotesingle s set and why.
\item
  Design a custom alias system for common 3D printing commands; document the aliases and their purposes.
\item
  Create a profile migration guide: document how to export and import your \texttt{.bashrc} across machines for consistent workflows.
\item
  Implement a safe PATH modification script: create a utility that allows you to add/remove directories from PATH for the current session only; show how to make it permanent in \texttt{.bashrc}.
\item
  Build a comprehensive \texttt{.bashrc} framework with modular sourcing: create separate \texttt{.sh} files for aliases, environment variables, and functions; have your main \texttt{.bashrc} source all of them.
\item
  Develop an environment validation tool: write a bash script that checks whether all required programs (OpenSCAD, slicers, etc.) are accessible via PATH; report findings and suggest fixes.
\item
  Create a project-switching alias system: design a function that changes all environment variables and aliases based on the current project; test switching between multiple projects.
\item
  Build a \texttt{.bashrc} troubleshooting guide: create a script that exports your current environment state (variables, aliases, PATH) to a timestamped file, allowing you to compare states before and after changes.
\end{enumerate}

\subsubsection*{References}\label{docs__pandoc__latex__src__gitbash_foundation__gitbash_4_environment_variables_aliases__gitbash_4_environment_variables_aliases.md__references}

\begin{itemize}
\tightlist
\item
  GNU. (2024). \emph{Environment variables in Bash}. \url{https://example.com}
\item
  GNU. (2024). \emph{alias command reference}. \url{https://example.com}
\item
  GNU. (2024). \emph{Creating and using Bash profiles}. \url{https://example.com}
\end{itemize}

\subsubsection*{Helpful Resources}\label{docs__pandoc__latex__src__gitbash_foundation__gitbash_4_environment_variables_aliases__gitbash_4_environment_variables_aliases.md__helpful-resources}

\begin{itemize}
\tightlist
\item
  \href{https://www.gnu.org/software/bash/manual/htmlnode/Environment.html}{Environment Variables in Bash}
\item
  \href{https://linuxize.com/post/how-to-add-directory-to-path-in-linux/}{Understanding the PATH Variable}
\item
  \href{https://www.gnu.org/software/bash/manual/htmlnode/Aliases.html}{Bash alias Reference}
\item
  \href{https://www.gnu.org/software/bash/manual/htmlnode/Bash-Startup-Files.html}{Creating a Bash Profile (.bashrc)}
\item
  \href{https://linux.die.net/man/1/which}{which Command for Locating Programs}
\end{itemize}

\subsection{GitBash-5: Filling in the Gaps - Shell Profiles, History, and Useful Tricks}\label{docs__pandoc__latex__src__gitbash_foundation__gitbash_5_filling_in_the_gaps__gitbash_5_filling_in_the_gaps.md__gitbash-5-filling-in-the-gaps---shell-profiles-history-and-useful-tricks}

Estimated time: 30-45 minutes

\subsubsection*{Learning Objectives}\label{docs__pandoc__latex__src__gitbash_foundation__gitbash_5_filling_in_the_gaps__gitbash_5_filling_in_the_gaps.md__learning-objectives}

\begin{itemize}
\tightlist
\item
  Use history and abort commands (\texttt{history}, \texttt{Ctrl+R}, \texttt{Ctrl+C})
\item
  Inspect and edit your \texttt{.bashrc} profile for persistent settings
\item
  Run programs by full path using \texttt{./} or absolute paths
\item
  Handle common screen reader edge cases when using the terminal
\end{itemize}

\subsubsection*{Materials}\label{docs__pandoc__latex__src__gitbash_foundation__gitbash_5_filling_in_the_gaps__gitbash_5_filling_in_the_gaps.md__materials}

\begin{itemize}
\tightlist
\item
  Git Bash and an editor (Notepad/VS Code)
\end{itemize}

\subsubsection*{Step-by-step Tasks}\label{docs__pandoc__latex__src__gitbash_foundation__gitbash_5_filling_in_the_gaps__gitbash_5_filling_in_the_gaps.md__step-by-step-tasks}

\begin{enumerate}
\tightlist
\item
  Run several simple commands (e.g., \texttt{pwd}, \texttt{ls\ -1}, \texttt{echo\ hi}) then run \texttt{history} to view them.
\item
  Use \texttt{!\textless{}n\textgreater{}} to re-run a previous command by its history number (replace \texttt{\textless{}n\textgreater{}} with a history number).
\item
  Practice aborting a long-running command with \texttt{Ctrl\ +\ C} (for example, \texttt{ping\ google.com}).
\item
  Open your profile: \texttt{notepad.exe\ \textasciitilde{}/.bashrc}; if it doesn\textquotesingle t exist, create it: \texttt{touch\ \textasciitilde{}/.bashrc}.
\item
  Add a persistent alias line to your profile (example: \texttt{alias\ preview=\textquotesingle{}openscad\textquotesingle{}}), save, and run \texttt{source\ \textasciitilde{}/.bashrc} or reopen Git Bash to verify.
\end{enumerate}

\subsubsection*{Checkpoints}\label{docs__pandoc__latex__src__gitbash_foundation__gitbash_5_filling_in_the_gaps__gitbash_5_filling_in_the_gaps.md__checkpoints}

\begin{itemize}
\tightlist
\item
  After step 2 you can re-run a recent command by history number.
\item
  After step 5 your alias should persist across sessions.
\end{itemize}

\subsubsection*{Quiz - Lesson GitBash.5}\label{docs__pandoc__latex__src__gitbash_foundation__gitbash_5_filling_in_the_gaps__gitbash_5_filling_in_the_gaps.md__quiz---lesson-gitbash5}

\begin{enumerate}
\tightlist
\item
  How do you view the command history in Git Bash?
\item
  Which key combination aborts a running command?
\item
  What does \texttt{echo\ \$BASH\_VERSION} show?
\item
  How does \texttt{./} help run scripts and executables in the current directory?
\item
  What is one strategy if terminal output stops being announced by your screen reader?
\item
  True or False: Using \texttt{Ctrl+C} permanently deletes any files created by the command you abort.
\item
  Explain the difference between \texttt{history} and \texttt{Ctrl+R} (reverse history search) in Git Bash.
\item
  If you add an alias to \texttt{.bashrc} but it doesn\textquotesingle t take effect after opening a new Git Bash window, what should you verify?
\item
  Write a command that would run a script at the path \texttt{\textasciitilde{}/scripts/openscad\_runner.sh} directly.
\item
  Describe a practical workflow scenario where having keyboard shortcuts (aliases) in your \texttt{.bashrc} would save time.
\item
  Explain how to re-run the 5th command from your history using \texttt{!5}, and what would happen if that command had file operations (creates/deletes).
\item
  Design a \texttt{.bashrc} initialization strategy that separates utilities for different projects; explain how you would switch between them.
\item
  Walk through a troubleshooting workflow: your screen reader stops announcing output after running a long command. What steps would you take to diagnose and resolve the issue?
\item
  Create a safety checkpoint system: before any destructive operation (mass delete, overwrite), how would you use \texttt{.bashrc} functions and history to verify the command is correct?
\item
  Develop a comprehensive capstone scenario: integrate everything from GitBash-0 through GitBash-5 (navigation, file operations, piping, environment setup, history) to design an automated 3D printing project workflow with error handling and logging.
\end{enumerate}

\subsubsection*{Extension Problems}\label{docs__pandoc__latex__src__gitbash_foundation__gitbash_5_filling_in_the_gaps__gitbash_5_filling_in_the_gaps.md__extension-problems}

\begin{enumerate}
\tightlist
\item
  Add an alias and an environment variable change to your \texttt{.bashrc} and document the behavior after reopening Git Bash.
\item
  Create a short bash script that automates creating a project folder and an initial \texttt{.scad} file.
\item
  Experiment with running OpenSCAD by full path using \texttt{./} and by placing it in PATH; compare results.
\item
  Practice redirecting \texttt{man\ ls} output to a file and reading it in Notepad for screen reader clarity.
\item
  Document three screen reader troubleshooting steps you used and when they helped.
\item
  Build a comprehensive \texttt{.bashrc} that includes aliases, environment variables, and helper functions for your 3D printing workflow.
\item
  Create a bash script that troubleshoots common Git Bash issues (module loading, permission errors, command not found); test at least three scenarios.
\item
  Write a bash function that coordinates multiple tasks: creates a project folder, starts OpenSCAD, and opens a notes file.
\item
  Design a screen-reader accessibility guide for Git Bash: document commands, outputs, and accessible navigation patterns.
\item
  Develop an advanced Git Bash workflow: implement error handling, logging, and confirmation prompts for risky operations.
\item
  Implement an "undo" system using history: create a function that logs destructive commands (\texttt{rm}, \texttt{mv}, \texttt{cp}) and allows you to review the last operation.
\item
  Build a \texttt{.bashrc} debugger: create a script that compares two Git Bash sessions\textquotesingle{} environment states (variables, aliases, functions) to identify what loaded/failed to load.
\item
  Develop a multi-project profile manager: design a system where you can switch entire environments (paths, aliases, variables) for different 3D printing projects by running a single command.
\item
  Create a comprehensive accessibility analyzer: write a bash script that tests whether key Git Bash commands produce screen-reader-friendly output; document workarounds for commands that don\textquotesingle t.
\item
  Design a complete capstone project: build an integrated automation suite that manages a 3D printing workflow (project setup, file organization, CAD/slicing tool automation, output logging, error recovery, and audit trails) with full error handling and documentation.
\end{enumerate}

\subsubsection*{References}\label{docs__pandoc__latex__src__gitbash_foundation__gitbash_5_filling_in_the_gaps__gitbash_5_filling_in_the_gaps.md__references}

\begin{itemize}
\tightlist
\item
  GNU. (2024). \emph{Bash history and recall functionality}. \url{https://example.com}
\item
  GNU. (2024). \emph{Understanding and creating Bash profiles}. \url{https://example.com}
\item
  GNU. (2024). \emph{Running scripts and executables in Bash}. \url{https://example.com}
\end{itemize}

\subsubsection*{Helpful Resources}\label{docs__pandoc__latex__src__gitbash_foundation__gitbash_5_filling_in_the_gaps__gitbash_5_filling_in_the_gaps.md__helpful-resources}

\begin{itemize}
\tightlist
\item
  \href{https://www.gnu.org/software/bash/manual/htmlnode/Bash-History-Facilities.html}{Bash History and Recall}
\item
  \href{https://www.gnu.org/software/bash/manual/htmlnode/Bash-Startup-Files.html}{Understanding .bashrc}
\item
  \href{https://www.gnu.org/software/bash/manual/htmlnode/Bash-History-Builtins.html}{history Command Reference}
\item
  \href{https://www.gnu.org/software/bash/manual/htmlnode/Executing-Scripts.html}{Running Scripts with ./}
\item
  \href{https://www.nvaccess.org/documentation/}{Screen Reader Tips and Tricks}
\end{itemize}

\subsection{GitBash-6: Advanced Terminal Techniques - Shell Scripts, Functions \& Professional Workflows}\label{docs__pandoc__latex__src__gitbash_foundation__gitbash_6_advanced_techniques__gitbash_6_advanced_techniques.md__gitbash-6-advanced-terminal-techniques---shell-scripts-functions--professional-workflows}

\textbf{Duration:} 4-4.5 hours (for screen reader users)
\textbf{Prerequisites:} Complete GitBash-0 through GitBash-5
\textbf{Skill Level:} Advanced intermediate

This lesson extends Git Bash skills to professional-level workflows. You\textquotesingle ll learn to automate complex tasks, write reusable shell scripts, and integrate tools for 3D printing workflows.

\begin{center}\rule{0.5\linewidth}{0.5pt}\end{center}

\subsubsection*{Learning Objectives}\label{docs__pandoc__latex__src__gitbash_foundation__gitbash_6_advanced_techniques__gitbash_6_advanced_techniques.md__learning-objectives}

By the end of this lesson, you will be able to:

\begin{itemize}
\tightlist
\item
  Create and run shell scripts (.sh files)
\item
  Write functions that accept parameters
\item
  Use loops to repeat tasks automatically
\item
  Automate batch processing of 3D models
\item
  Debug scripts when something goes wrong
\item
  Create professional workflows combining multiple tools
\end{itemize}

\begin{center}\rule{0.5\linewidth}{0.5pt}\end{center}

\subsubsection*{Shell Script Basics}\label{docs__pandoc__latex__src__gitbash_foundation__gitbash_6_advanced_techniques__gitbash_6_advanced_techniques.md__shell-script-basics}

\paragraph*{What\textquotesingle s a Shell Script?}\label{docs__pandoc__latex__src__gitbash_foundation__gitbash_6_advanced_techniques__gitbash_6_advanced_techniques.md__whats-a-shell-script}

A shell script (\texttt{.sh}) contains multiple bash commands that run in sequence. Instead of typing commands one by one, you put them in a file and run them all at once.

\textbf{Why use shell scripts?}

\begin{itemize}
\tightlist
\item
  Repeatability: Run the same task 100 times identically
\item
  Documentation: Commands are written down for reference
\item
  Complexity: Combine many commands logically
\item
  Automation: Schedule scripts to run automatically
\end{itemize}

\paragraph*{Creating Your First Shell Script}\label{docs__pandoc__latex__src__gitbash_foundation__gitbash_6_advanced_techniques__gitbash_6_advanced_techniques.md__creating-your-first-shell-script}

\textbf{Step 1: Open a text editor}

\begin{lstlisting}[style=Alabaster, language=bash]
notepad.exe my-first-script.sh

\end{lstlisting}

\textbf{Step 2: Type this script}

\begin{lstlisting}[style=Alabaster, language=bash]
#!/bin/bash
# This is a comment - screen readers will read it
echo "Script is running!"
pwd
ls -1
echo "Script is done!"

\end{lstlisting}

\textbf{Step 3: Save the file}

\begin{itemize}
\tightlist
\item
  In Notepad: Ctrl+S
\item
  Make sure filename ends in \texttt{.sh}
\item
  Save in an easy-to-find location (like Documents)
\end{itemize}

\textbf{Step 4: Make it executable and run the script}

\begin{lstlisting}[style=Alabaster, language=bash]
chmod +x my-first-script.sh
./my-first-script.sh

\end{lstlisting}

\textbf{What happens:}
Bash runs each command in sequence and shows output.

\paragraph*{Important: The Shebang Line}\label{docs__pandoc__latex__src__gitbash_foundation__gitbash_6_advanced_techniques__gitbash_6_advanced_techniques.md__important-the-shebang-line}

The \texttt{\#!/bin/bash} at the top of your script (called a "shebang") tells the system which shell to use to run the script.

\begin{lstlisting}[style=Alabaster, language=bash]
#!/bin/bash
# Now bash runs every command below
echo "Hello!"

\end{lstlisting}

\begin{center}\rule{0.5\linewidth}{0.5pt}\end{center}

\subsubsection*{Variables and Parameters}\label{docs__pandoc__latex__src__gitbash_foundation__gitbash_6_advanced_techniques__gitbash_6_advanced_techniques.md__variables-and-parameters}

\paragraph*{Using Variables}\label{docs__pandoc__latex__src__gitbash_foundation__gitbash_6_advanced_techniques__gitbash_6_advanced_techniques.md__using-variables}

Variables store values you want to use later.

\textbf{Example script:}

\begin{lstlisting}[style=Alabaster, language=bash]
#!/bin/bash
mypath="$HOME/Documents"
cd "$mypath"
echo "I am now in:"
pwd
ls -1

\end{lstlisting}

\textbf{Breaking it down:}

\begin{itemize}
\tightlist
\item
  \texttt{mypath="..."} assigns the variable (no spaces around \texttt{=})
\item
  \texttt{"\$mypath"} uses the variable (quote it to handle spaces in paths)
\item
  Variables in bash are always referenced with \texttt{\$}
\end{itemize}

\paragraph*{Functions with Parameters}\label{docs__pandoc__latex__src__gitbash_foundation__gitbash_6_advanced_techniques__gitbash_6_advanced_techniques.md__functions-with-parameters}

A function is reusable code that you can run with different inputs.

\textbf{Example: A function that lists files in a folder}

\begin{lstlisting}[style=Alabaster, language=bash]
#!/bin/bash
list_folder() {
    local path="$1"
    echo "Contents of: $path"
    cd "$path"
    ls -1
}

# Use the function:
list_folder "$HOME/Documents"
list_folder "$HOME/Downloads"

\end{lstlisting}

\textbf{What\textquotesingle s happening:}

\begin{itemize}
\tightlist
\item
  \texttt{list\_folder()} defines the function
\item
  \texttt{local\ path="\$1"} assigns the first argument to a local variable
\item
  \texttt{\$1} is the first argument passed to the function
\item
  Call the function with \texttt{list\_folder\ "path/to/folder"}
\end{itemize}

\textbf{Screen reader tip:} When you call a function, bash will announce the results just like any command.

\begin{center}\rule{0.5\linewidth}{0.5pt}\end{center}

\subsubsection*{Loops - Repeating Tasks}\label{docs__pandoc__latex__src__gitbash_foundation__gitbash_6_advanced_techniques__gitbash_6_advanced_techniques.md__loops---repeating-tasks}

\paragraph*{Loop Over Files}\label{docs__pandoc__latex__src__gitbash_foundation__gitbash_6_advanced_techniques__gitbash_6_advanced_techniques.md__loop-over-files}

Imagine you have 10 SCAD files and want to print their contents. You could do it 10 times manually, or use a loop.

\textbf{Example: Print every .scad file in a folder}

\begin{lstlisting}[style=Alabaster, language=bash]
#!/bin/bash
for file in *.scad; do
    echo "=== File: $file ==="
    cat "$file"
    echo ""
done

\end{lstlisting}

\textbf{What\textquotesingle s happening:}

\begin{itemize}
\tightlist
\item
  \texttt{for\ file\ in\ *.scad;\ do} iterates over each \texttt{.scad} file
\item
  \texttt{\$file} holds the current filename
\item
  \texttt{done} ends the loop
\item
  Inside the loop, do something with each \texttt{\$file}
\end{itemize}

\paragraph*{Loop with a Counter}\label{docs__pandoc__latex__src__gitbash_foundation__gitbash_6_advanced_techniques__gitbash_6_advanced_techniques.md__loop-with-a-counter}

\textbf{Example: Do something 5 times}

\begin{lstlisting}[style=Alabaster, language=bash]
#!/bin/bash
for i in $(seq 1 5); do
    echo "This is iteration number $i"
    # Do something here
done

\end{lstlisting}

\textbf{What\textquotesingle s happening:}

\begin{itemize}
\tightlist
\item
  \texttt{for\ i\ in\ \$(seq\ 1\ 5)} loops with \texttt{i} from 1 to 5
\item
  \texttt{\$i} is the counter variable
\end{itemize}

\begin{center}\rule{0.5\linewidth}{0.5pt}\end{center}

\subsubsection*{Real-World Example - Batch Processing SCAD Files}\label{docs__pandoc__latex__src__gitbash_foundation__gitbash_6_advanced_techniques__gitbash_6_advanced_techniques.md__real-world-example---batch-processing-scad-files}

\paragraph*{Scenario}\label{docs__pandoc__latex__src__gitbash_foundation__gitbash_6_advanced_techniques__gitbash_6_advanced_techniques.md__scenario}

You have 10 OpenSCAD (.scad) files in a folder. You want to:

\begin{enumerate}
\tightlist
\item
  List them all
\item
  Check how many there are
\item
  For each one, verify it exists
\end{enumerate}

\paragraph*{The Script}\label{docs__pandoc__latex__src__gitbash_foundation__gitbash_6_advanced_techniques__gitbash_6_advanced_techniques.md__the-script}

\begin{lstlisting}[style=Alabaster, language=bash]
#!/bin/bash
scad_folder="$HOME/Documents/3D_Projects"
count=0

echo "Processing SCAD files in: $scad_folder"
echo ""

for file in "$scad_folder"/*.scad; do
    if [ -f "$file" ]; then
        echo "  Found: $(basename "$file")"
        count=$((count + 1))
    else
        echo "  Missing: $(basename "$file")"
    fi
done

echo ""
echo "Total files found: $count"
echo "Batch processing complete!"

\end{lstlisting}

\textbf{Breaking it down:}

\begin{itemize}
\tightlist
\item
  \texttt{scad\_folder="..."} = where to look
\item
  \texttt{for\ file\ in\ "\$scad\_folder"/*.scad} = find all .scad files
\item
  \texttt{if\ {[}\ -f\ "\$file"\ {]}} = check if file exists and is a regular file
\item
  \texttt{basename\ "\$file"} = just the filename (not the full path)
\item
  \texttt{count=\$((count\ +\ 1))} = increment the counter
\end{itemize}

\paragraph*{Running the Script}\label{docs__pandoc__latex__src__gitbash_foundation__gitbash_6_advanced_techniques__gitbash_6_advanced_techniques.md__running-the-script}

\begin{enumerate}
\tightlist
\item
  Save as \texttt{batch-process.sh}
\item
  Edit \texttt{scad\_folder} to match your real folder
\item
  Make it executable and run it:

  \begin{lstlisting}[style=Alabaster, language=bash]
  chmod +x batch-process.sh
  ./batch-process.sh

  \end{lstlisting}
\end{enumerate}

\textbf{Screen reader output:}

\begin{lstlisting}[style=Alabaster]
Processing SCAD files in: /c/Users/YourName/Documents/3D_Projects

  Found: model1.scad
  Found: model2.scad
  Found: model3.scad
[... more files ...]

Total files found: 10
Batch processing complete!

\end{lstlisting}

\begin{center}\rule{0.5\linewidth}{0.5pt}\end{center}

\subsubsection*{Error Handling}\label{docs__pandoc__latex__src__gitbash_foundation__gitbash_6_advanced_techniques__gitbash_6_advanced_techniques.md__error-handling}

\paragraph*{Try-Style Checks with Exit Codes}\label{docs__pandoc__latex__src__gitbash_foundation__gitbash_6_advanced_techniques__gitbash_6_advanced_techniques.md__try-style-checks-with-exit-codes}

What if something goes wrong? Use exit code checks:

\textbf{Example:}

\begin{lstlisting}[style=Alabaster, language=bash]
#!/bin/bash
file="$HOME/nonexistent/path/file.txt"

if cat "$file" 2>/dev/null; then
    echo "File read successfully"
else
    echo "Error: Could not read file"
    echo "File path was: $file"
fi

\end{lstlisting}

\textbf{What\textquotesingle s happening:}

\begin{itemize}
\tightlist
\item
  \texttt{2\textgreater{}/dev/null} suppresses error messages from \texttt{cat}
\item
  \texttt{if\ cat\ ...;\ then} checks whether the command succeeded
\item
  \texttt{else} handles the failure gracefully
\end{itemize}

\textbf{Screen reader advantage:} Errors are announced clearly instead of crashing silently.

\paragraph*{Validating Input}\label{docs__pandoc__latex__src__gitbash_foundation__gitbash_6_advanced_techniques__gitbash_6_advanced_techniques.md__validating-input}

\textbf{Example: Make sure a folder exists before processing}

\begin{lstlisting}[style=Alabaster, language=bash]
#!/bin/bash
process_folder() {
    local folder_path="$1"

    if [ ! -d "$folder_path" ]; then
        echo "Error: Folder does not exist: $folder_path"
        return 1
    fi

    echo "Processing folder: $folder_path"
    ls -1 "$folder_path"
}

process_folder "$HOME/Documents"

\end{lstlisting}

\textbf{What\textquotesingle s happening:}

\begin{itemize}
\tightlist
\item
  \texttt{{[}\ !\ -d\ "\$folder\_path"\ {]}} checks if the path is NOT a directory
\item
  \texttt{return\ 1} exits the function early with a non-zero (error) status
\end{itemize}

\begin{center}\rule{0.5\linewidth}{0.5pt}\end{center}

\subsubsection*{Debugging Shell Scripts}\label{docs__pandoc__latex__src__gitbash_foundation__gitbash_6_advanced_techniques__gitbash_6_advanced_techniques.md__debugging-shell-scripts}

\paragraph*{Common Errors and Solutions}\label{docs__pandoc__latex__src__gitbash_foundation__gitbash_6_advanced_techniques__gitbash_6_advanced_techniques.md__common-errors-and-solutions}

\subparagraph*{Error 1: "Command not found"}\label{docs__pandoc__latex__src__gitbash_foundation__gitbash_6_advanced_techniques__gitbash_6_advanced_techniques.md__error-1-command-not-found}

\textbf{Cause:} Typo in command name

\textbf{Fix:} Check spelling

\begin{lstlisting}[style=Alabaster, language=bash]
# Wrong:
ech "hello"

# Correct:
echo "hello"

\end{lstlisting}

\subparagraph*{Error 2: "Variable is empty"}\label{docs__pandoc__latex__src__gitbash_foundation__gitbash_6_advanced_techniques__gitbash_6_advanced_techniques.md__error-2-variable-is-empty}

\textbf{Cause:} Variable was never assigned or has a typo

\textbf{Fix:} Make sure variable is set before using it

\begin{lstlisting}[style=Alabaster, language=bash]
myvar="hello"  # Set first
echo "$myvar"  # Then use

\end{lstlisting}

\subparagraph*{Error 3: "No such file or directory"}\label{docs__pandoc__latex__src__gitbash_foundation__gitbash_6_advanced_techniques__gitbash_6_advanced_techniques.md__error-3-no-such-file-or-directory}

\textbf{Cause:} Wrong folder path

\textbf{Fix:} Verify path exists

\begin{lstlisting}[style=Alabaster, language=bash]
# Check if path exists:
if [ -d "$HOME/Documents" ]; then
    echo "Path exists"
fi

\end{lstlisting}

\subparagraph*{Error 4: "Permission denied"}\label{docs__pandoc__latex__src__gitbash_foundation__gitbash_6_advanced_techniques__gitbash_6_advanced_techniques.md__error-4-permission-denied}

\textbf{Cause:} Script not executable, or no write permission

\textbf{Fix:} Make the script executable, or check file permissions

\begin{lstlisting}[style=Alabaster, language=bash]
chmod +x my-script.sh

\end{lstlisting}

\paragraph*{\texorpdfstring{Debugging Technique: Trace Output with \texttt{set\ -x}}{Debugging Technique: Trace Output with set -x}}\label{docs__pandoc__latex__src__gitbash_foundation__gitbash_6_advanced_techniques__gitbash_6_advanced_techniques.md__debugging-technique-trace-output-with-set--x}

Add \texttt{set\ -x} at the top of your script to print each command before it runs:

\begin{lstlisting}[style=Alabaster, language=bash]
#!/bin/bash
set -x  # Enable trace mode
path_var="$HOME/Documents"
echo "Starting script. Path is: $path_var"

for file in "$path_var"/*; do
    echo "Processing: $file"
    # Do something with $file
    echo "Done with: $file"
done

echo "Script complete"

\end{lstlisting}

Your screen reader will announce each step, so you know where errors happen.

\begin{center}\rule{0.5\linewidth}{0.5pt}\end{center}

\subsubsection*{Creating Professional Workflows}\label{docs__pandoc__latex__src__gitbash_foundation__gitbash_6_advanced_techniques__gitbash_6_advanced_techniques.md__creating-professional-workflows}

\paragraph*{Example 1: Automated Project Setup}\label{docs__pandoc__latex__src__gitbash_foundation__gitbash_6_advanced_techniques__gitbash_6_advanced_techniques.md__example-1-automated-project-setup}

\textbf{Scenario:} You start a new 3D printing project regularly. Instead of creating folders manually:

\begin{lstlisting}[style=Alabaster, language=bash]
#!/bin/bash
read -p "Enter project name: " project_name
base_folder="$HOME/Documents/3D_Projects"
project_folder="$base_folder/$project_name"

# Create folder structure
mkdir -p "$project_folder"
mkdir -p "$project_folder/designs"
mkdir -p "$project_folder/output"
mkdir -p "$project_folder/notes"

# Create a README
cat > "$project_folder/README.txt" << EOF
# $project_name

Created: $(date)

## Designs
All .scad files go here.

## Output
STL and other exports go here.

## Notes
Project notes and observations.
EOF

echo "Project setup complete: $project_folder"

\end{lstlisting}

\textbf{What it does:}

\begin{itemize}
\tightlist
\item
  Prompts for a project name
\item
  Creates folder structure for a new project
\item
  Sets up subfolders for designs, output, notes
\item
  Creates a README file automatically
\end{itemize}

\paragraph*{Example 2: Batch File Verification}\label{docs__pandoc__latex__src__gitbash_foundation__gitbash_6_advanced_techniques__gitbash_6_advanced_techniques.md__example-2-batch-file-verification}

\textbf{Scenario:} Before processing, verify all required files exist:

\begin{lstlisting}[style=Alabaster, language=bash]
#!/bin/bash
verify_project() {
    local project_folder="$1"
    local required_items=("README.txt" "designs" "output" "notes")
    local all_good=true

    for item in "${required_items[@]}"; do
        local path="$project_folder/$item"
        if [ -e "$path" ]; then
            echo "  Found: $item"
        else
            echo "  Missing: $item"
            all_good=false
        fi
    done

    if $all_good; then
        echo "All checks passed!"
        return 0
    else
        echo "Some files are missing!"
        return 1
    fi
}

# Use it:
project="$HOME/Documents/3D_Projects/MyKeychain"
if verify_project "$project"; then
    echo "Safe to proceed with processing"
fi

\end{lstlisting}

\begin{center}\rule{0.5\linewidth}{0.5pt}\end{center}

\subsubsection*{Screen Reader Tips for Shell Scripts}\label{docs__pandoc__latex__src__gitbash_foundation__gitbash_6_advanced_techniques__gitbash_6_advanced_techniques.md__screen-reader-tips-for-shell-scripts}

\paragraph*{Making Script Output Readable}\label{docs__pandoc__latex__src__gitbash_foundation__gitbash_6_advanced_techniques__gitbash_6_advanced_techniques.md__making-script-output-readable}

\textbf{Problem:} Script runs but output scrolls too fast or is hard to follow

\textbf{Solution 1: Save to file}

\begin{lstlisting}[style=Alabaster, language=bash]
./my-script.sh > output.txt
notepad.exe output.txt

\end{lstlisting}

\textbf{Solution 2: Use echo with clear sections}

\begin{lstlisting}[style=Alabaster, language=bash]
echo "========== STARTING =========="
echo ""
# ... script ...
echo ""
echo "========== COMPLETE =========="

\end{lstlisting}

\textbf{Solution 3: Pause between major sections}

\begin{lstlisting}[style=Alabaster, language=bash]
echo "Pausing... Press Enter to continue"
read

\end{lstlisting}

Your screen reader will announce the pause, give you time to read output.

\paragraph*{Announcing Progress}\label{docs__pandoc__latex__src__gitbash_foundation__gitbash_6_advanced_techniques__gitbash_6_advanced_techniques.md__announcing-progress}

\textbf{For long-running scripts:}

\begin{lstlisting}[style=Alabaster, language=bash]
#!/bin/bash
count=0
total=$(ls *.scad | wc -l)

for file in *.scad; do
    count=$((count + 1))
    echo "Processing $count of $total: $file"
    # Do something with $file
done

echo "All $count files processed!"

\end{lstlisting}

\begin{center}\rule{0.5\linewidth}{0.5pt}\end{center}

\subsubsection*{Practice Exercises}\label{docs__pandoc__latex__src__gitbash_foundation__gitbash_6_advanced_techniques__gitbash_6_advanced_techniques.md__practice-exercises}

\paragraph*{Exercise 1: Your First Shell Script}\label{docs__pandoc__latex__src__gitbash_foundation__gitbash_6_advanced_techniques__gitbash_6_advanced_techniques.md__exercise-1-your-first-shell-script}

\textbf{Goal:} Create and run a simple shell script

\textbf{Steps:}

\begin{enumerate}
\tightlist
\item
  Create file: \texttt{notepad.exe\ hello.sh}
\item
  Type:

  \begin{lstlisting}[style=Alabaster, language=bash]
  #!/bin/bash
  echo "Hello from my first Git Bash shell script!"
  pwd
  ls -1

  \end{lstlisting}
\item
  Save, make executable, and run:

  \begin{lstlisting}[style=Alabaster, language=bash]
  chmod +x hello.sh
  ./hello.sh

  \end{lstlisting}
\end{enumerate}

\textbf{Checkpoint:} You should see output for each command.

\paragraph*{Exercise 2: Script with a Variable}\label{docs__pandoc__latex__src__gitbash_foundation__gitbash_6_advanced_techniques__gitbash_6_advanced_techniques.md__exercise-2-script-with-a-variable}

\textbf{Goal:} Use a variable to make the script flexible

\textbf{Steps:}

\begin{enumerate}
\tightlist
\item
  Create file: \texttt{notepad.exe\ smart-listing.sh}
\item
  Type:

  \begin{lstlisting}[style=Alabaster, language=bash]
  #!/bin/bash
  target_folder="$HOME/Documents"
  echo "Listing contents of: $target_folder"
  ls -1 "$target_folder"

  \end{lstlisting}
\item
  Edit \texttt{target\_folder} to a real folder on your computer
\item
  Run:

  \begin{lstlisting}[style=Alabaster, language=bash]
  chmod +x smart-listing.sh
  ./smart-listing.sh

  \end{lstlisting}
\end{enumerate}

\textbf{Checkpoint:} You should see listing of that specific folder.

\paragraph*{Exercise 3: Function}\label{docs__pandoc__latex__src__gitbash_foundation__gitbash_6_advanced_techniques__gitbash_6_advanced_techniques.md__exercise-3-function}

\textbf{Goal:} Create a reusable function

\textbf{Steps:}

\begin{enumerate}
\tightlist
\item
  Create file: \texttt{notepad.exe\ navigate.sh}
\item
  Type:

  \begin{lstlisting}[style=Alabaster, language=bash]
  #!/bin/bash
  go_to() {
      local path="$1"
      if [ -d "$path" ]; then
          cd "$path"
          echo "Now in: $(pwd)"
          echo "Contents:"
          ls -1
      else
          echo "Path does not exist: $path"
      fi
  }

  # Test the function:
  go_to "$HOME/Documents"
  go_to "$HOME/Downloads"

  \end{lstlisting}
\item
  Run:

  \begin{lstlisting}[style=Alabaster, language=bash]
  chmod +x navigate.sh
  ./navigate.sh

  \end{lstlisting}
\end{enumerate}

\textbf{Checkpoint:} Both function calls should work, showing contents of each folder.

\paragraph*{Exercise 4: Loop}\label{docs__pandoc__latex__src__gitbash_foundation__gitbash_6_advanced_techniques__gitbash_6_advanced_techniques.md__exercise-4-loop}

\textbf{Goal:} Use a loop to repeat an action

\textbf{Steps:}

\begin{enumerate}
\tightlist
\item
  Create file: \texttt{notepad.exe\ repeat.sh}
\item
  Type:

  \begin{lstlisting}[style=Alabaster, language=bash]
  #!/bin/bash
  echo "Demonstrating a loop:"

  for i in $(seq 1 5); do
      echo "Iteration $i: Hello!"
  done

  echo "Loop complete!"

  \end{lstlisting}
\item
  Run:

  \begin{lstlisting}[style=Alabaster, language=bash]
  chmod +x repeat.sh
  ./repeat.sh

  \end{lstlisting}
\end{enumerate}

\textbf{Checkpoint:} Should print "Iteration 1" through "Iteration 5".

\paragraph*{Exercise 5: Real-World Script}\label{docs__pandoc__latex__src__gitbash_foundation__gitbash_6_advanced_techniques__gitbash_6_advanced_techniques.md__exercise-5-real-world-script}

\textbf{Goal:} Create a useful script for a real task

\textbf{Steps:}

\begin{enumerate}
\tightlist
\item
  Create a folder: \texttt{mkdir\ \textasciitilde{}/Documents/TestFiles}
\item
  Create some test files:

  \begin{lstlisting}[style=Alabaster, language=bash]
  echo "test" > ~/Documents/TestFiles/file1.txt
  echo "test" > ~/Documents/TestFiles/file2.txt
  echo "test" > ~/Documents/TestFiles/file3.txt

  \end{lstlisting}
\item
  Create script: \texttt{notepad.exe\ report.sh}
\item
  Type:

  \begin{lstlisting}[style=Alabaster, language=bash]
  #!/bin/bash
  folder="$HOME/Documents/TestFiles"
  count=0

  echo "=== FILE REPORT ==="
  echo "Folder: $folder"
  echo ""
  echo "Files:"
  for file in "$folder"/*; do
      echo "  - $(basename "$file")"
      count=$((count + 1))
  done
  echo ""
  echo "Total: $count files"
  echo "=== END REPORT ==="

  \end{lstlisting}
\item
  Run:

  \begin{lstlisting}[style=Alabaster, language=bash]
  chmod +x report.sh
  ./report.sh

  \end{lstlisting}
\end{enumerate}

\textbf{Checkpoint:} Should show report of all files in the test folder.

\begin{center}\rule{0.5\linewidth}{0.5pt}\end{center}

\subsubsection*{Quiz - Lesson GitBash.6}\label{docs__pandoc__latex__src__gitbash_foundation__gitbash_6_advanced_techniques__gitbash_6_advanced_techniques.md__quiz---lesson-gitbash6}

\begin{enumerate}
\tightlist
\item
  What is a shell script?
\item
  What file extension do bash shell scripts use?
\item
  What is a variable in bash and how do you create one?
\item
  What is a function and why would you use one?
\item
  How do you run a shell script?
\item
  What is a \texttt{for} loop and what does \texttt{for\ file\ in\ *.scad;\ do} do?
\item
  What does \texttt{{[}\ -f\ "\$file"\ {]}} check?
\item
  How do you handle errors in a bash script?
\item
  When would you use \texttt{if\ {[}\ !\ -d\ "\$path"\ {]};\ then}?
\item
  What technique makes shell script output readable for screen readers?
\end{enumerate}

\begin{center}\rule{0.5\linewidth}{0.5pt}\end{center}

\subsubsection*{Extension Problems}\label{docs__pandoc__latex__src__gitbash_foundation__gitbash_6_advanced_techniques__gitbash_6_advanced_techniques.md__extension-problems}

\begin{enumerate}
\tightlist
\item
  \textbf{Auto-Backup Script:} Create a bash script that copies all files from one folder to another, announcing progress
\item
  \textbf{File Counter:} Write a function that counts files by extension (.txt, .scad, .stl, etc.)
\item
  \textbf{Folder Cleaner:} Script that deletes files older than 30 days (with user confirmation)
\item
  \textbf{Project Template:} Function that creates a complete project folder structure with all needed files
\item
  \textbf{Batch Rename:} Script that renames all files in a folder according to a pattern
\item
  \textbf{Log Generator:} Create a script that records what it does to a log file for later review
\item
  \textbf{Scheduled Task:} Set up a script to run automatically using cron or Task Scheduler
\item
  \textbf{File Verifier:} Check that all SCAD files in a folder have corresponding STL exports
\item
  \textbf{Report Generator:} Create a summary report of all projects in a folder
\item
  \textbf{Error Tracker:} Script that lists all commands that had errors and logs them with timestamps
\end{enumerate}

\begin{center}\rule{0.5\linewidth}{0.5pt}\end{center}

\subsubsection*{Important Notes}\label{docs__pandoc__latex__src__gitbash_foundation__gitbash_6_advanced_techniques__gitbash_6_advanced_techniques.md__important-notes}

\begin{itemize}
\tightlist
\item
  \textbf{Always test scripts on small sets of files first} before running them on important data
\item
  \textbf{Save your work regularly} --- use version naming if possible
\item
  \textbf{Test error handling} --- make sure errors don\textquotesingle t crash silently
\item
  \textbf{Document your scripts} --- use \texttt{\#} comments so you remember what each part does
\item
  \textbf{Backup before batch operations} --- if something goes wrong, you have the original
\end{itemize}

\begin{center}\rule{0.5\linewidth}{0.5pt}\end{center}

\subsubsection*{References}\label{docs__pandoc__latex__src__gitbash_foundation__gitbash_6_advanced_techniques__gitbash_6_advanced_techniques.md__references}

\begin{itemize}
\tightlist
\item
  \textbf{GNU Bash Scripting Guide:} \url{https://example.com}
\item
  \textbf{Function Documentation:} \url{https://example.com}
\item
  \textbf{Error Handling:} \url{https://example.com}
\item
  \textbf{Loops:} \url{https://example.com}
\end{itemize}

\subsection{GitBash Unit Test - Comprehensive Assessment}\label{docs__pandoc__latex__src__gitbash_foundation__gitbash_unit_test__gitbash_unit_test.md__gitbash-unit-test---comprehensive-assessment}

Estimated time: 60-90 minutes

\subsubsection*{Key Learning Outcomes Assessed}\label{docs__pandoc__latex__src__gitbash_foundation__gitbash_unit_test__gitbash_unit_test.md__key-learning-outcomes-assessed}

By completing this unit test, you will demonstrate:

\begin{enumerate}
\tightlist
\item
  Understanding of file system navigation and path concepts
\item
  Proficiency with file and folder manipulation commands
\item
  Ability to redirect and pipe command output
\item
  Knowledge of environment variables and aliases
\item
  Screen-reader accessibility best practices in terminal environments
\item
  Problem-solving and command chaining skills
\end{enumerate}

\subsubsection*{Target Audience}\label{docs__pandoc__latex__src__gitbash_foundation__gitbash_unit_test__gitbash_unit_test.md__target-audience}

Users who have completed GitBash-0 through GitBash-6 and need to demonstrate mastery of Git Bash fundamentals.

\subsubsection*{Instructions}\label{docs__pandoc__latex__src__gitbash_foundation__gitbash_unit_test__gitbash_unit_test.md__instructions}

Complete all sections below. For multiple choice, select the best answer. For short answers, write one to two sentences. For hands-on tasks, capture evidence (screenshots or output files) and submit alongside your answers.

\begin{center}\rule{0.5\linewidth}{0.5pt}\end{center}

\subsubsection*{Part A: Multiple Choice Questions (20 questions)}\label{docs__pandoc__latex__src__gitbash_foundation__gitbash_unit_test__gitbash_unit_test.md__part-a-multiple-choice-questions-20-questions}

Select the best answer for each question. Each question is worth 1 point.

\begin{enumerate}
\item
  What is the primary purpose of the \texttt{PATH} environment variable?

  \begin{itemize}
  \tightlist
  \item
    A) Store your home directory location
  \item
    B) Tell the shell where to find executable programs
  \item
    C) Configure the visual appearance of the terminal
  \item
    D) Store the current working directory name
  \end{itemize}
\item
  Which command prints your current working directory in Git Bash?

  \begin{itemize}
  \tightlist
  \item
    A) \texttt{ls\ -1}
  \item
    B) \texttt{cd}
  \item
    C) \texttt{pwd}
  \item
    D) \texttt{whoami}
  \end{itemize}
\item
  What does the \texttt{\textasciitilde{}} symbol represent in Git Bash paths?

  \begin{itemize}
  \tightlist
  \item
    A) The root directory
  \item
    B) The current directory
  \item
    C) The parent directory
  \item
    D) The home directory
  \end{itemize}
\item
  How do you list only file names (not full details) in a way that is screen-reader friendly?

  \begin{itemize}
  \tightlist
  \item
    A) \texttt{ls}
  \item
    B) \texttt{ls\ -1}
  \item
    C) \texttt{ls\ -l}
  \item
    D) \texttt{cat\ -1}
  \end{itemize}
\item
  Which command creates a new empty file in Git Bash?

  \begin{itemize}
  \tightlist
  \item
    A) \texttt{mkdir\ filename}
  \item
    B) \texttt{touch\ filename}
  \item
    C) \texttt{new\ filename}
  \item
    D) \texttt{echo\ filename}
  \end{itemize}
\item
  What is the difference between \texttt{\textgreater{}} and \texttt{\textgreater{}\textgreater{}}?

  \begin{itemize}
  \tightlist
  \item
    A) \texttt{\textgreater{}} redirects to file, \texttt{\textgreater{}\textgreater{}} displays on screen
  \item
    B) \texttt{\textgreater{}} overwrites a file, \texttt{\textgreater{}\textgreater{}} appends to a file
  \item
    C) They do the same thing
  \item
    D) \texttt{\textgreater{}} is for text, \texttt{\textgreater{}\textgreater{}} is for binary
  \end{itemize}
\item
  What does the pipe operator \texttt{\textbar{}} do?

  \begin{itemize}
  \tightlist
  \item
    A) Creates a folder
  \item
    B) Sends the output of one command to the input of another
  \item
    C) Deletes files matching a pattern
  \item
    D) Lists all processes
  \end{itemize}
\item
  Which command copies a file in Git Bash?

  \begin{itemize}
  \tightlist
  \item
    A) \texttt{mv}
  \item
    B) \texttt{rm}
  \item
    C) \texttt{cp}
  \item
    D) \texttt{cd}
  \end{itemize}
\item
  How do you rename a file from \texttt{oldname.txt} to \texttt{newname.txt} in Git Bash?

  \begin{itemize}
  \tightlist
  \item
    A) \texttt{cp\ oldname.txt\ newname.txt}
  \item
    B) \texttt{mv\ oldname.txt\ newname.txt}
  \item
    C) \texttt{rename\ oldname.txt\ newname.txt}
  \item
    D) \texttt{rn\ oldname.txt\ newname.txt}
  \end{itemize}
\item
  What is the purpose of \texttt{grep} in Git Bash piping?

  \begin{itemize}
  \tightlist
  \item
    A) Find files in a directory
  \item
    B) Search for text patterns within output or files
  \item
    C) Select a string to copy to clipboard
  \item
    D) Select which shell to use
  \end{itemize}
\item
  Which key allows you to autocomplete a path in Git Bash?

  \begin{itemize}
  \tightlist
  \item
    A) \texttt{Ctrl\ +\ A}
  \item
    B) \texttt{Ctrl\ +\ E}
  \item
    C) \texttt{Tab}
  \item
    D) \texttt{Space}
  \end{itemize}
\item
  How do you copy text to the Windows clipboard from Git Bash?

  \begin{itemize}
  \tightlist
  \item
    A) \texttt{cat\ filename\ \textgreater{}\ clipboard}
  \item
    B) \texttt{cat\ filename\ \textbar{}\ clip}
  \item
    C) \texttt{copy\ filename}
  \item
    D) \texttt{cat\ filename\ \textbar{}\ paste}
  \end{itemize}
\item
  What does \texttt{which\ openscad} do?

  \begin{itemize}
  \tightlist
  \item
    A) Opens the OpenSCAD application
  \item
    B) Gets help about the openscad command
  \item
    C) Locates the full path of the openscad executable
  \item
    D) Lists all available commands
  \end{itemize}
\item
  Which wildcard matches any single character in Git Bash?

  \begin{itemize}
  \tightlist
  \item
    A) \texttt{*}
  \item
    B) \texttt{?}
  \item
    C) \texttt{\%}
  \item
    D) \texttt{\#}
  \end{itemize}
\item
  What is the purpose of \texttt{./} before a script name?

  \begin{itemize}
  \tightlist
  \item
    A) Run a script in the current directory
  \item
    B) Execute all commands in parallel
  \item
    C) Combine multiple commands
  \item
    D) Create an alias
  \end{itemize}
\item
  How do you create a temporary alias in Git Bash?

  \begin{itemize}
  \tightlist
  \item
    A) \texttt{set-alias\ preview=\textquotesingle{}openscad\textquotesingle{}}
  \item
    B) \texttt{alias\ preview=\textquotesingle{}openscad\textquotesingle{}}
  \item
    C) \texttt{new-alias\ preview\ openscad}
  \item
    D) \texttt{preview\ =\ openscad}
  \end{itemize}
\item
  Where is a Git Bash alias typically stored to persist across sessions?

  \begin{itemize}
  \tightlist
  \item
    A) \texttt{C:\textbackslash{}Program\ Files\textbackslash{}Git\textbackslash{}profile.sh}
  \item
    B) In the \texttt{\textasciitilde{}/.bashrc} file
  \item
    C) \texttt{\textasciitilde{}/bash\_profile}
  \item
    D) Aliases cannot be made permanent
  \end{itemize}
\item
  How do you abort a long-running command in Git Bash?

  \begin{itemize}
  \tightlist
  \item
    A) Press \texttt{Escape}
  \item
    B) Press \texttt{Ctrl\ +\ X}
  \item
    C) Press \texttt{Ctrl\ +\ C}
  \item
    D) Press \texttt{Alt\ +\ F4}
  \end{itemize}
\item
  What command shows the history of previously run commands in Git Bash?

  \begin{itemize}
  \tightlist
  \item
    A) \texttt{history}
  \item
    B) \texttt{get-history}
  \item
    C) \texttt{show-history}
  \item
    D) \texttt{bash-history}
  \end{itemize}
\item
  How do you reload your \texttt{.bashrc} without restarting Git Bash?

  \begin{itemize}
  \tightlist
  \item
    A) Use \texttt{reload\ \textasciitilde{}/.bashrc} in the terminal
  \item
    B) Use \texttt{source\ \textasciitilde{}/.bashrc}
  \item
    C) Use the Windows Control Panel
  \item
    D) \texttt{.bashrc} reloads automatically
  \end{itemize}
\end{enumerate}

\begin{center}\rule{0.5\linewidth}{0.5pt}\end{center}

\subsubsection*{Part B: Short Answer Questions (10 questions)}\label{docs__pandoc__latex__src__gitbash_foundation__gitbash_unit_test__gitbash_unit_test.md__part-b-short-answer-questions-10-questions}

Answer each question in one to two sentences. Each question is worth 2 points.

\begin{enumerate}
\item
  Explain the difference between absolute and relative paths. Give one example of each.
\item
  Why is \texttt{ls\ -1} preferred over \texttt{ls} for screen reader users? Describe what flag combination you would use to list only directories.
\item
  What is the purpose of redirecting output to a file, and give an example of when you would use \texttt{\textgreater{}} instead of \texttt{\textgreater{}\textgreater{}}?
\item
  Describe what would happen if you ran \texttt{rm\ -r\ \textasciitilde{}/Documents/my\_folder} and why this command should be used carefully.
\item
  How would you search for all files with a \texttt{.scad} extension in your current directory? Write the command.
\item
  Explain what happens when you pipe the output of \texttt{ls\ -1} into \texttt{clip}. What would you do next?
\item
  What is an environment variable, and give one example of how you might use it in Git Bash.
\item
  If a program is not in your \texttt{PATH}, what two methods could you use to run it from Git Bash?
\item
  Describe how you would open a file in Notepad and also add a line to it from Git Bash.
\item
  What is one strategy you would use if your screen reader stops announcing terminal output while using Git Bash?
\end{enumerate}

\begin{center}\rule{0.5\linewidth}{0.5pt}\end{center}

\subsubsection*{Part C: Hands-On Tasks (10 tasks)}\label{docs__pandoc__latex__src__gitbash_foundation__gitbash_unit_test__gitbash_unit_test.md__part-c-hands-on-tasks-10-tasks}

Complete each task and capture evidence (screenshots, output files, or command transcripts). Each task is worth 3 points.

\paragraph*{Tasks 1-5: File System and Navigation}\label{docs__pandoc__latex__src__gitbash_foundation__gitbash_unit_test__gitbash_unit_test.md__tasks-1-5-file-system-and-navigation}

\begin{enumerate}
\item
  Create a folder structure \texttt{\textasciitilde{}/Documents/GitBash\_Assessment/Projects} using a single command. Capture the \texttt{ls\ -1} output showing the creation.
\item
  Create five files named \texttt{project\_1.scad}, \texttt{project\_2.scad}, \texttt{project\_3.txt}, \texttt{notes\_1.txt}, and \texttt{notes\_2.txt} inside the \texttt{Projects} folder. Use wildcards to list only \texttt{.scad} files, then capture the output.
\item
  Copy the entire \texttt{Projects} folder to \texttt{Projects\_Backup} using \texttt{cp\ -r}. Capture the \texttt{ls\ -1} output showing both folders exist.
\item
  Move (rename) \texttt{project\_1.scad} to \texttt{project\_1\_final.scad}. Capture the \texttt{ls\ -1} output showing the renamed file.
\item
  Delete \texttt{notes\_1.txt} and \texttt{notes\_2.txt} using a single \texttt{rm} command with wildcards. Capture the final \texttt{ls\ -1} output.
\end{enumerate}

\paragraph*{Tasks 6-10: Advanced Operations and Scripting}\label{docs__pandoc__latex__src__gitbash_foundation__gitbash_unit_test__gitbash_unit_test.md__tasks-6-10-advanced-operations-and-scripting}

\begin{enumerate}
\setcounter{enumi}{5}
\item
  Create a file called \texttt{my\_data.txt} with at least four lines using \texttt{echo} and \texttt{\textgreater{}\textgreater{}}. Then read it with \texttt{cat\ my\_data.txt} and capture the output.
\item
  Use \texttt{grep} to search for a keyword (e.g., "project") in \texttt{my\_data.txt} and pipe the results to \texttt{clip}. Paste the results into Notepad and capture a screenshot.
\item
  List all files in the \texttt{Projects} folder and redirect the output to \texttt{projects\_list.txt}. Open it in Notepad and capture a screenshot of the file.
\item
  Create a temporary alias called \texttt{myls} that runs \texttt{ls\ -1}, test it, and capture the output. Then explain what would be required to make it persistent.
\item
  Run \texttt{man\ ls} (or \texttt{ls\ -\/-help}) and redirect the output to \texttt{help\_output.txt}. Open the file in Notepad and capture a screenshot showing at least the first page of help content.
\end{enumerate}

\begin{center}\rule{0.5\linewidth}{0.5pt}\end{center}

\subsubsection*{Grading Rubric}\label{docs__pandoc__latex__src__gitbash_foundation__gitbash_unit_test__gitbash_unit_test.md__grading-rubric}

{\def\LTcaptype{none} % do not increment counter
\begin{longtable}[]{@{}llll@{}}
\toprule\noalign{}
Section & Questions & Points Each & Total \\
\midrule\noalign{}
\endhead
\bottomrule\noalign{}
\endlastfoot
Multiple Choice & 20 & 1 & 20 \\
Short Answer & 10 & 2 & 20 \\
Hands-On Tasks & 10 & 3 & 30 \\
\textbf{Total} & \textbf{40} & - & \textbf{70} \\
\end{longtable}
}

\textbf{Passing Score:} 49 points (70\%)

\begin{center}\rule{0.5\linewidth}{0.5pt}\end{center}

\subsubsection*{Helpful Resources for Review}\label{docs__pandoc__latex__src__gitbash_foundation__gitbash_unit_test__gitbash_unit_test.md__helpful-resources-for-review}

\begin{itemize}
\tightlist
\item
  \href{https://git-scm.com/book/en/v2}{Git Bash Command Reference}
\item
  \href{https://linuxcommand.org/lc3lts0020.php}{Navigation and File System}
\item
  \href{https://www.gnu.org/software/grep/manual/grep.html}{Using Pipes and Filtering}
\item
  \href{https://www.gnu.org/software/bash/manual/htmlnode/Bash-Startup-Files.html}{Bash Profile and Aliases}
\item
  \href{https://www.nvaccess.org/documentation/}{Screen Reader Accessibility Tips}
\end{itemize}

\begin{center}\rule{0.5\linewidth}{0.5pt}\end{center}

\subsubsection*{Submission Checklist}\label{docs__pandoc__latex__src__gitbash_foundation__gitbash_unit_test__gitbash_unit_test.md__submission-checklist}

\begin{itemize}
\tightlist
\item[$\square$]
  All 20 multiple choice questions answered
\item[$\square$]
  All 10 short answer questions answered (1-2 sentences each)
\item[$\square$]
  All 10 hands-on tasks completed with evidence captured
\item[$\square$]
  Files/screenshots organized and labeled clearly
\item[$\square$]
  Submission includes this checklist
\end{itemize}

\chapter{3dMake-Accessible Design with OpenSCAD}\label{docs__pandoc__latex__src__3dmake_foundation__part_2.md__3dmake-accessible-design-with-openscad}

This section covers the complete 3D design and printing workflow using OpenSCAD, 3DMake, and accessible tools. Students progress from basic primitives through advanced parametric design and stakeholder-centric projects.

Time commitment: 30-40 hours instruction + projects

Skills gained: 3D modeling, parametric design, automated workflows, tolerance testing, quality assurance

\section{Instructional Framework for 3DMake and OpenSCAD in Secondary STEM Education}\label{docs__pandoc__latex__src__3dmake_foundation__3dmake_intro__3dmake_intro.md__instructional-framework-for-3dmake-and-openscad-in-secondary-stem-education}

The transition from traditional, direct-manipulation computer-aided design (CAD) to programmatic, declarative modeling represents a fundamental shift in how physical objects are conceptualized and engineered. For the high school junior, this transition is not merely a change in software but a pivot toward computational thinking, where geometry is derived from mathematical logic rather than visual approximation. The 3DMake ecosystem, an open-source command-line toolchain, serves as the critical connective tissue in this lifecycle, automating the translation of source code into physical matter.\footnote{Deck, T. (2025). \emph{3DMake: A command-line tool for 3D printing workflows}. GitHub. Retrieved from \url{https://github.com/tdeck/3dmake}} By integrating OpenSCADs script-based modeling with 3DMakes automation of rendering, slicing, and verification, students engage with a workflow that mirrors professional DevOps and industrial manufacturing pipelines.\footnote{Deck, T. (2025). \emph{3DMake: A command-line tool for 3D printing workflows}. GitHub. Retrieved from \url{https://github.com/tdeck/3dmake}} This report provides an exhaustive pedagogical guide, safety background, and technical analysis of this ecosystem, tailored for an advanced secondary education environment.

\subsection*{The Architecture of Programmatic Design: Introduction to 3DMake}\label{docs__pandoc__latex__src__3dmake_foundation__3dmake_intro__3dmake_intro.md__the-architecture-of-programmatic-design-introduction-to-3dmake}

Programmatic CAD differs from standard industry tools like Fusion 360 or SolidWorks by utilizing a "code-as-model" philosophy. OpenSCAD, the primary engine supported by 3DMake, uses a functional programming language to define three-dimensional volumes.\footnote{Programming with OpenSCAD, accessed February 18, 2026, \url{https://programmingwithopenscad.github.io/}} This approach is particularly robust for parametric design, where the dimensions of an object are defined as variables, allowing for instantaneous reconfiguration without manual rebuilding.\footnote{OpenSCAD Review - Worth learning? - CadHub, accessed February 18, 2026, \url{https://learn.cadhub.xyz/blog/openscad-review/}} 3DMake enhances this by providing a unified command-line interface (CLI) to manage the entire process, from editing source files to triggering remote print jobs through interfaces like OctoPrint or Bambu Connect.\footnote{Deck, T. (2025). \emph{3DMake: A command-line tool for 3D printing workflows}. GitHub. Retrieved from \url{https://github.com/tdeck/3dmake}}\\
The educational value of this toolchain lies in its transparency. Students are not hidden behind a proprietary user interface; they interact directly with the file system, configuration files, and API integrations. This exposure fosters a deeper understanding of how modern software interacts with hardware, a skill set increasingly vital in robotics and aerospace engineering.\footnote{10+ OpenSCAD Online Courses for 2026 \textbar{} Explore Free Courses \& Certifications, accessed February 18, 2026, \url{https://www.classcentral.com/subject/openscad}} However, the shift to a terminal-based environment requires a structured instructional approach to overcome initial barriers to entry.

\subsection*{Background on AI Integration and Model Descriptive Feedback}\label{docs__pandoc__latex__src__3dmake_foundation__3dmake_intro__3dmake_intro.md__background-on-ai-integration-and-model-descriptive-feedback}

The integration of artificial intelligence into the 3DMake workflow represents a significant advancement in democratizing CAD for students with varying levels of spatial reasoning skills. The 3dm info command acts as a multimodal bridge between the deterministic world of OpenSCAD and the probabilistic world of LLMs.\footnote{Deck, T. (2025). \emph{3DMake: A command-line tool for 3D printing workflows}. GitHub. Retrieved from \url{https://github.com/tdeck/3dmake}}\\
When a student executes 3dm info, the system initiates a rendering pipeline. It generates multiple viewpoints of the current model, which are then packaged as image data.\footnote{Deck, T. (2025). \emph{3DMake: A command-line tool for 3D printing workflows}. GitHub. Retrieved from \url{https://github.com/tdeck/3dmake}} If the Gemini API is configured, these images are sent to the model along with a prompt that requires the AI to synthesize a 3D understanding from 2D representations. The AI\textquotesingle s response is then returned to the terminal, providing a descriptive summary that can include the model\textquotesingle s intended function, aesthetic qualities, and potential engineering failures.\footnote{Deck, T. (2025). \emph{3DMake: A command-line tool for 3D printing workflows}. GitHub. Retrieved from \url{https://github.com/tdeck/3dmake}}

\subsubsection*{The Limits of AI Spatial Reasoning}\label{docs__pandoc__latex__src__3dmake_foundation__3dmake_intro__3dmake_intro.md__the-limits-of-ai-spatial-reasoning}

While this feature is powerful, there is a fundamental disconnect in how current LLMs process 3D data. Most models are trained on 2D images and text; they do not possess a true "3D world model".\footnote{Build Great AI: LLM-Powered 3D Model Generation for 3D Printing - ZenML LLMOps Database, accessed February 18, 2026, \url{https://www.zenml.io/llmops-database/llm-powered-3d-model-generation-for-3d-printing}} This leads to several common failures:

\begin{itemize}
\tightlist
\item
  Detached Geometry: The AI might describe a "table" even if the legs are hovering below the tabletop-a common error in OpenSCAD translation.\footnote{Build Great AI: LLM-Powered 3D Model Generation for 3D Printing - ZenML LLMOps Database, accessed February 18, 2026, \url{https://www.zenml.io/llmops-database/llm-powered-3d-model-generation-for-3d-printing}}
\item
  Scale Misinterpretation: Without a reference object in the render, the AI may misjudge the scale of the model, leading to inappropriate feedback on wall thickness.\footnote{Deck, T. (2025). \emph{3DMake: A command-line tool for 3D printing workflows}. GitHub. Retrieved from \url{https://github.com/tdeck/3dmake}}
\item
  Hallucination of Detail: The AI may describe features (like "engraved text") that it expects to see based on the prompt, even if the students code failed to render them.\footnote{Build Great AI: LLM-Powered 3D Model Generation for 3D Printing - ZenML LLMOps Database, accessed February 18, 2026, \url{https://www.zenml.io/llmops-database/llm-powered-3d-model-generation-for-3d-printing}}
\end{itemize}

For the high school student, the takeaway is that AI is a verification assistant, not a source of truth. The deterministic rendering of OpenSCAD remains the final authority on the model\textquotesingle s geometry. The AI is most useful as a "sanity check" to catch obvious mistakes before wasting filament on a failed print.\footnote{Deck, T. (2025). \emph{3DMake: A command-line tool for 3D printing workflows}. GitHub. Retrieved from \url{https://github.com/tdeck/3dmake}}

\subsection*{Occupational Health and Safety in the 3D Printing Environment}\label{docs__pandoc__latex__src__3dmake_foundation__3dmake_intro__3dmake_intro.md__occupational-health-and-safety-in-the-3d-printing-environment}

Safety in the 3DMake workflow is not limited to the digital realm. The physical act of printing involves thermal, chemical, and mechanical risks that must be addressed through rigorous institutional policies. The National Institute for Occupational Safety and Health (NIOSH) and various university environmental health departments provide a clear framework for these risks.\footnote{3D Printer Safety - Environmental Health and Safety - The Ohio State University, accessed February 18, 2026, \url{https://ehs.osu.edu/kb/3d-printer-safety}}

\subsubsection*{Chemical and Particulate Emissions}\label{docs__pandoc__latex__src__3dmake_foundation__3dmake_intro__3dmake_intro.md__chemical-and-particulate-emissions}

The melting of plastic filament is a thermal degradation process. ABS (Acrylonitrile Butadiene Styrene) is particularly hazardous, releasing styrene, a known respiratory irritant and potential carcinogen.\footnote{3D Printer Safety - Environmental Health and Safety - The Ohio State University, accessed February 18, 2026, \url{https://ehs.osu.edu/kb/3d-printer-safety}} Even PLA (Polylactic Acid), often marketed as "safe" and "biodegradable," emits millions of ultrafine particles (UFPs) per minute during extrusion.\footnote{3D Printer Safety - Environmental Health and Safety - The Ohio State University, accessed February 18, 2026, \url{https://ehs.osu.edu/kb/3d-printer-safety}} These particles, smaller than 100 nanometers, can penetrate deep into the lungs and cross into the bloodstream.12

{\def\LTcaptype{none} % do not increment counter
\begin{longtable}[]{@{}
  >{\raggedright\arraybackslash}p{(\linewidth - 4\tabcolsep) * \real{0.2569}}
  >{\raggedright\arraybackslash}p{(\linewidth - 4\tabcolsep) * \real{0.2385}}
  >{\raggedright\arraybackslash}p{(\linewidth - 4\tabcolsep) * \real{0.5046}}@{}}
\toprule\noalign{}
\begin{minipage}[b]{\linewidth}\raggedright
Emission Component
\end{minipage} & \begin{minipage}[b]{\linewidth}\raggedright
Primary Source Filaments
\end{minipage} & \begin{minipage}[b]{\linewidth}\raggedright
Mitigation Strategy
\end{minipage} \\
\midrule\noalign{}
\endhead
\bottomrule\noalign{}
\endlastfoot
Styrene & ABS, ASA &
Enclosed printer with carbon filtration.\footnote{3D Printer Safety - Environmental Health and Safety - The Ohio State University, accessed February 18, 2026, \url{https://ehs.osu.edu/kb/3d-printer-safety}} \\
Formaldehyde & POM, Nylon &
High-efficiency external ventilation.\footnote{3D Printer Safety - Environmental Health and Safety - The Ohio State University, accessed February 18, 2026, \url{https://ehs.osu.edu/kb/3d-printer-safety}} \\
Ultrafine Particles & All filaments &
HEPA filtration and "20-minute" settling period.\footnote{Safe 3D Printing is for Everyone, Everywhere \textbar{} NIOSH Blogs - CDC, accessed February 18, 2026, \url{https://www.cdc.gov/niosh/blogs/2024/safe-3d-printing.html}} \\
Volatile Organic Compounds & All filaments &
Minimum 6 air changes per hour in the room.\footnote{3D Printers \textbar{} Washington State Department of Health, accessed February 18, 2026, \url{https://doh.wa.gov/community-and-environment/schools/3d-printers}} \\
\end{longtable}
}

\subsubsection*{Physical and Mechanical Hazards}\label{docs__pandoc__latex__src__3dmake_foundation__3dmake_intro__3dmake_intro.md__physical-and-mechanical-hazards}

3D printers utilize high-torque stepper motors and heated elements. The extruder nozzle can reach \$260\^{}\textbackslash circ\textbackslash text\{C\}\$, and the heated build plate can reach \$110\^{}\textbackslash circ\textbackslash text\{C\}\$.\footnote{3-D Printer Safety \textbar{} Environmental Health \& Safety \textbar{} RIT, accessed February 18, 2026, \url{https://www.rit.edu/ehs/3-d-printer-safety}} Mechanical hazards include "pinch points" in the gantry system where fingers or loose clothing can be trapped.\footnote{Safe 3D Printing is for Everyone, Everywhere \textbar{} NIOSH Blogs - CDC, accessed February 18, 2026, \url{https://www.cdc.gov/niosh/blogs/2024/safe-3d-printing.html}} Furthermore, post-processing activities-such as removing supports or sanding parts-introduce the risk of cuts from sharp tools and the inhalation of plastic dust.\footnote{3D Printer Safety - Environmental Health and Safety - The Ohio State University, accessed February 18, 2026, \url{https://ehs.osu.edu/kb/3d-printer-safety}}\\
Standard operating procedures (SOPs) for a student lab must include:

\begin{itemize}
\tightlist
\item
  Pre-use Inspection: Checking for frayed wires, loose belts, and a clear build surface.\footnote{3D Printers \textbar{} Washington State Department of Health, accessed February 18, 2026, \url{https://doh.wa.gov/community-and-environment/schools/3d-printers}}
\item
  Environmental Controls: Prohibiting eating or drinking in the print area to avoid ingestion of contaminants.\footnote{3-D Printer Safety \textbar{} Environmental Health \& Safety \textbar{} RIT, accessed February 18, 2026, \url{https://www.rit.edu/ehs/3-d-printer-safety}}
\item
  Emergency Response: Locating the nearest Class D fire extinguisher (for metal prints) or standard ABC extinguisher.\footnote{3D Printer Safety - Environmental Health \& Safety - University of Tennessee, Knoxville, accessed February 18, 2026, \url{https://ehs.utk.edu/index.php/table-of-policies-plans-procedures-guides/3d-printer-safety/}}
\item
  Post-Print Cooling: Ensuring the printer has cooled to below \$30\^{}\textbackslash circ\textbackslash text\{C\}\$ before attempting to remove the model.15
\end{itemize}

\subsection*{Challenges Inherent in the OpenSCAD Language}\label{docs__pandoc__latex__src__3dmake_foundation__3dmake_intro__3dmake_intro.md__challenges-inherent-in-the-openscad-language}

OpenSCAD is often described as "the programmer\textquotesingle s CAD," but its declarative nature and unique rendering kernel present specific hurdles for high school students. Unlike imperative languages (where you tell the computer \emph{how} to do something), OpenSCAD is declarative.\footnote{Understanding the Challenges of OpenSCAD Users for 3D Printing - Thomas Pietrzak, accessed February 18, 2026, \url{https://thomaspietrzak.com/bibliography/gonzalez24.pdf}} This can be counterintuitive for students familiar with Python or JavaScript.

\subsubsection*{The Problem of Immutable State}\label{docs__pandoc__latex__src__3dmake_foundation__3dmake_intro__3dmake_intro.md__the-problem-of-immutable-state}

In OpenSCAD, variables are not truly "variable" in the traditional sense; they are more like constants within a specific scope. If a student writes \texttt{x\ =\ 5;\ x\ =\ 10;}, OpenSCAD will use the last value assigned to \texttt{x} for the entire script (in this case, \texttt{x\ =\ 10}).\footnote{Why is there so little content and community around a tool as powerful and interesting as OpenSCAD? (beyond the awesome folks in this channel) - Reddit, accessed February 18, 2026, \url{https://www.reddit.com/r/openscad/comments/1fxj8xv/why_is_there_so_little_content_and_community/}} This behavior is similar to SQL or XSLT, and requires the student to adopt a functional programming mindset where geometry is defined by its state rather than its sequence of movements.\footnote{Why is there so little content and community around a tool as powerful and interesting as OpenSCAD? (beyond the awesome folks in this channel) - Reddit, accessed February 18, 2026, \url{https://www.reddit.com/r/openscad/comments/1fxj8xv/why_is_there_so_little_content_and_community/}}

\subsubsection*{Performance Bottlenecks and Functional Limits}\label{docs__pandoc__latex__src__3dmake_foundation__3dmake_intro__3dmake_intro.md__performance-bottlenecks-and-functional-limits}

OpenSCAD uses the CGAL (Computational Geometry Algorithms Library) as its geometry kernel. While highly accurate, CGAL is notoriously slow for certain operations.

\begin{itemize}
\tightlist
\item
  Minkowski Sums: This operation, often used to round corners, is computationally explosive. A simple rounded cube can take minutes to render if the resolution (defined by \$fn) is too high.{[}\^{}21{]}
\item
  Hull Operations: The hull() function creates a convex "shrink-wrap" around objects. While faster than minkowski, it cannot be used inside certain loops and can fail if the child objects are non-manifold.\footnote{The great thing about OpenSCAD is that it makes it easy to 3D model things which... \textbar{} Hacker News, accessed February 18, 2026, \url{https://news.ycombinator.com/item?id=46338565}}
\item
  Lack of Native Filleting: Unlike modern CAD tools, OpenSCAD has no native "fillet" command. Students must manually construct these features using boolean subtractions or libraries like BOSL2.\footnote{OpenSCAD Review - Worth learning? - CadHub, accessed February 18, 2026, \url{https://learn.cadhub.xyz/blog/openscad-review/}}
\end{itemize}

\subsubsection*{The "Absolute Coordinate" Barrier}\label{docs__pandoc__latex__src__3dmake_foundation__3dmake_intro__3dmake_intro.md__the-absolute-coordinate-barrier}

In tools like Fusion 360, parts are often "jointed" relative to one another. OpenSCAD has no concept of relative constraints.\footnote{The great thing about OpenSCAD is that it makes it easy to 3D model things which... \textbar{} Hacker News, accessed February 18, 2026, \url{https://news.ycombinator.com/item?id=46338565}} Every object is positioned in absolute \$(\textbackslash text\{X\}, \textbackslash text\{Y\}, \textbackslash text\{Z\})\$ space. If a student moves one part, they must manually calculate and update the translation of every related part.\footnote{The great thing about OpenSCAD is that it makes it easy to 3D model things which... \textbar{} Hacker News, accessed February 18, 2026, \url{https://news.ycombinator.com/item?id=46338565}} This necessitates the heavy use of variables and mathematical offsets, a process that is highly prone to human error.\footnote{The great thing about OpenSCAD is that it makes it easy to 3D model things which... \textbar{} Hacker News, accessed February 18, 2026, \url{https://news.ycombinator.com/item?id=46338565}}

\subsection*{Technical Limitations and Risks of the 3DMake Tool}\label{docs__pandoc__latex__src__3dmake_foundation__3dmake_intro__3dmake_intro.md__technical-limitations-and-risks-of-the-3dmake-tool}

While 3DMake provides a powerful automation layer, it introduces its own set of limitations and risks.

\subsubsection*{Dependency Management and Setup Complexity}\label{docs__pandoc__latex__src__3dmake_foundation__3dmake_intro__3dmake_intro.md__dependency-management-and-setup-complexity}

3DMake is a CLI wrapper, meaning its functionality is entirely dependent on the host machine\textquotesingle s configuration. The ./3dm setup process is a critical point of failure.\footnote{Deck, T. (2025). \emph{3DMake: A command-line tool for 3D printing workflows}. GitHub. Retrieved from \url{https://github.com/tdeck/3dmake}} If the student provides an incorrect path to the OpenSCAD executable or the slicer, the tool will fail or produce cryptic errors. In a school environment with restricted user permissions, setting up the necessary environmental variables can be a significant administrative hurdle.\footnote{Deck, T. (2025). \emph{3DMake: A command-line tool for 3D printing workflows}. GitHub. Retrieved from \url{https://github.com/tdeck/3dmake}}

\subsubsection*{The Risks of Automated Scripting}\label{docs__pandoc__latex__src__3dmake_foundation__3dmake_intro__3dmake_intro.md__the-risks-of-automated-scripting}

The power of 3DMake lies in its ability to string together actions: 3dm build slice print.\footnote{Deck, T. (2025). \emph{3DMake: A command-line tool for 3D printing workflows}. GitHub. Retrieved from \url{https://github.com/tdeck/3dmake}} This "one-command" fabrication is efficient but dangerous if the student bypasses visual verification. If the OpenSCAD code contains a subtle error that results in a "non-manifold" mesh, the slicer may still produce a G-code file that causes the printer to behave erratically-such as extruding into mid-air.24

{\def\LTcaptype{none} % do not increment counter
\begin{longtable}[]{@{}
  >{\raggedright\arraybackslash}p{(\linewidth - 4\tabcolsep) * \real{0.1268}}
  >{\raggedright\arraybackslash}p{(\linewidth - 4\tabcolsep) * \real{0.3803}}
  >{\raggedright\arraybackslash}p{(\linewidth - 4\tabcolsep) * \real{0.4930}}@{}}
\toprule\noalign{}
\begin{minipage}[b]{\linewidth}\raggedright
Limitation
\end{minipage} & \begin{minipage}[b]{\linewidth}\raggedright
Technical Root Cause
\end{minipage} & \begin{minipage}[b]{\linewidth}\raggedright
Educational Impact
\end{minipage} \\
\midrule\noalign{}
\endhead
\bottomrule\noalign{}
\endlastfoot
CLI Barrier & No graphical interface for configuration. &
Steep learning curve for students with zero terminal experience.\footnote{Deck, T. (2025). \emph{3DMake: A command-line tool for 3D printing workflows}. GitHub. Retrieved from \url{https://github.com/tdeck/3dmake}} \\
API Dependency & AI features require external internet and API keys. &
Advanced features fail in offline school networks.\footnote{Deck, T. (2025). \emph{3DMake: A command-line tool for 3D printing workflows}. GitHub. Retrieved from \url{https://github.com/tdeck/3dmake}} \\
Slicer Lock-in & Reliant on external templates for G-code generation. &
Students may not learn the nuance of slicing settings.\footnote{Deck, T. (2025). \emph{3DMake: A command-line tool for 3D printing workflows}. GitHub. Retrieved from \url{https://github.com/tdeck/3dmake}} \\
Feedback Latency & No real-time "live" preview in the editor. &
The "edit-compile-view" cycle is slower than GUI-based CAD.\footnote{Deck, T. (2025). \emph{3DMake: A command-line tool for 3D printing workflows}. GitHub. Retrieved from \url{https://github.com/tdeck/3dmake}} \\
\end{longtable}
}

\subsubsection*{Non-Manifold Geometry and Slicing Errors}\label{docs__pandoc__latex__src__3dmake_foundation__3dmake_intro__3dmake_intro.md__non-manifold-geometry-and-slicing-errors}

A recurring challenge for 3DMake users is the production of "non-manifold" STLs. A manifold object is "water-tight"-it has a clear inside and outside.{[}\^{}28{]} OpenSCAD can easily produce non-manifold geometry through "zero-thickness" walls or improperly closed polyhedrons.{[}\^{}24{]} 3DMakes build command will generate the STL without warning, but the slice command may then fail or produce a corrupt G-code file.{[}\^{}24{]} This requires the student to learn mesh verification skills, often requiring third-party tools like MeshLab or PrusaSlicers repair functions.{[}\^{}28{]}

\subsection*{Local Resources and Community Support: Salt Lake County Ecosystem}\label{docs__pandoc__latex__src__3dmake_foundation__3dmake_intro__3dmake_intro.md__local-resources-and-community-support-salt-lake-county-ecosystem}

For students located in the Salt Lake County area, the transition from digital model to physical object is supported by a robust network of makerspaces and public library resources.

\subsubsection*{Public Library "Create Spaces" and Creative Labs}\label{docs__pandoc__latex__src__3dmake_foundation__3dmake_intro__3dmake_intro.md__public-library-create-spaces-and-creative-labs}

The Salt Lake County and City Library systems offer specialized makerspaces where students can bring their 3DMake-generated files for printing.

{\def\LTcaptype{none} % do not increment counter
\begin{longtable}[]{@{}
  >{\raggedright\arraybackslash}p{(\linewidth - 6\tabcolsep) * \real{0.1381}}
  >{\raggedright\arraybackslash}p{(\linewidth - 6\tabcolsep) * \real{0.1823}}
  >{\raggedright\arraybackslash}p{(\linewidth - 6\tabcolsep) * \real{0.2707}}
  >{\raggedright\arraybackslash}p{(\linewidth - 6\tabcolsep) * \real{0.4088}}@{}}
\toprule\noalign{}
\begin{minipage}[b]{\linewidth}\raggedright
Facility
\end{minipage} & \begin{minipage}[b]{\linewidth}\raggedright
Location
\end{minipage} & \begin{minipage}[b]{\linewidth}\raggedright
Key Hardware
\end{minipage} & \begin{minipage}[b]{\linewidth}\raggedright
Policies/Costs
\end{minipage} \\
\midrule\noalign{}
\endhead
\bottomrule\noalign{}
\endlastfoot
SLC Public Creative Lab & Main Library (Level 1) &
Prusa i3 MK3, LulzBot Taz 5, Elegoo Mars 2 &
Free for prints under 6 hours; Material cost + \$0.50/hr otherwise.{[}\^{}31{]} \\
County Library "Create" & Daybreak, Granite, Kearns, etc. &
Flashforge Adventurer 5M Pro, LulzBot Workhorse &
\$0.06 per gram of filament used.{[}\^{}26{]} \\
Make Salt Lake & 663 W 100 S, SLC &
CNC, Metal Shop, Large-scale FDM and Resin &
Membership-based; offers certification classes for advanced tools.{[}\^{}32{]} \\
\end{longtable}
}

\subsubsection*{Higher Education and Specialized Maker Hubs}\label{docs__pandoc__latex__src__3dmake_foundation__3dmake_intro__3dmake_intro.md__higher-education-and-specialized-maker-hubs}

For advanced students, the University of Utah provides several makerspaces, including the Lassonde Studios and the Eccles Health Sciences Library Technology Hub.{[}\^{}34{]} These centers offer access to "industrial-grade" printing technologies, such as Selective Laser Sintering (SLS), which require even more rigorous safety training regarding inert gas (Argon/Nitrogen) asphyxiation hazards.\footnote{3D Printer Safety - Environmental Health \& Safety - University of Tennessee, Knoxville, accessed February 18, 2026, \url{https://ehs.utk.edu/index.php/table-of-policies-plans-procedures-guides/3d-printer-safety/}}

\subsection*{Conclusion: The Pedagogy of Programmatic Manufacturing}\label{docs__pandoc__latex__src__3dmake_foundation__3dmake_intro__3dmake_intro.md__conclusion-the-pedagogy-of-programmatic-manufacturing}

The integration of OpenSCAD and 3DMake into a high school curriculum is a powerful strategy for developing the next generation of engineers. By shifting the focus from "visual sculpting" to "mathematical definition," students are forced to confront the underlying logic of their designs. The 3DMake toolchain facilitates this by removing the friction of manual rendering and slicing, allowing the student to stay in the "flow state" of coding.\\
However, the success of this instructional model depends on a comprehensive understanding of its limitations. The instructor must balance the efficiency of AI-assisted verification with a healthy skepticism of LLM spatial reasoning. They must enforce rigid safety protocols to mitigate the invisible risks of UFP and VOC emissions. And finally, they must guide the student through the idiosyncratic challenges of the OpenSCAD language-its absolute coordinates and its strict manifold requirements.

\subsection*{References}\label{docs__pandoc__latex__src__3dmake_foundation__3dmake_intro__3dmake_intro.md__references}

Deck, T. (2025). \emph{3DMake: A command-line tool for 3D printing workflows}. GitHub. \url{https://github.com/tdeck/3dmake}\\
Gohde, J., \& Kintel, M. (2021). \emph{Programming with OpenSCAD: A beginner\textquotesingle s guide to coding 3D-printable objects}. No Starch Press.\\
Gonzalez Avila, J. F., Pietrzak, T., \& Casiez, G. (2024). \emph{Understanding the challenges of OpenSCAD users for 3D printing}. Proceedings of the ACM Symposium on User Interface Software and Technology.\\
Google. (2025). \emph{Vertex AI Gemini 3 Pro Preview: Getting started with generative AI}. \url{https://docs.cloud.google.com/vertex-ai/generative-ai/docs/start/get-started-with-gemini-3}\\
National Institute for Occupational Safety and Health (NIOSH). (2024). \emph{Approaches to safe 3D printing: A guide for makerspace users, schools, libraries, and small businesses}. \url{https://www.cdc.gov/niosh/blogs/2024/safe-3d-printing.html}\\
Ohio State University Environmental Health and Safety. (2026). \emph{3D printer safety concerns and ventilation}. \url{https://ehs.osu.edu/kb/3d-printer-safety}\\
Salt Lake City Public Library. (2026). \emph{Creative Lab: Available hardware and 3D printing procedures}. \url{https://services.slcpl.org/creativelab}\\
Salt Lake County Library. (2026). \emph{Create Spaces: Hardware specifications and filament fees}. \url{https://www.slcolibrary.org/what-we-have/create}\\
University of Utah. (2026). \emph{Marriott Library ProtoSpace and Maker hubs}. \url{https://lib.utah.edu/protospace.php}\\
Washington State Department of Health. (2026). \emph{3D printer and filament selection for safe school environments}. \url{https://doh.wa.gov/community-and-environment/schools/3d-printers}

\subsection*{Supplemental Learning Resources}\label{docs__pandoc__latex__src__3dmake_foundation__3dmake_intro__3dmake_intro.md__supplemental-learning-resources}

This introduction is complemented by comprehensive textbooks and code examples:

\begin{itemize}
\tightlist
\item
  \href{docs/pandoc/latex/src/assets/Programming_with_OpenSCAD.epub}{Programming with OpenSCAD: A Beginner\textquotesingle s Guide to Coding 3D-Printable Objects} - Complete reference covering OpenSCAD syntax, geometry concepts, design patterns, and best practices
\item
  \href{https://github.com/mrhunsaker/VI_3DMake_OpenSCAD_Curriculum/3dMake_Foundation/3dMake_Intro/../../assets/Simplifying_3D_Printing_with_OpenSCAD.epub}{Simplifying 3D Printing with OpenSCAD} - Focused guide to practical workflows, optimization techniques, and real-world printing applications
\item
  \href{https://github.com/ProgrammingWithOpenSCAD/CodeSolutions}{CodeSolutions Repository} - Working OpenSCAD examples organized by topic and difficulty level, demonstrating all concepts covered in this curriculum
\item
  \href{https://programmingwithopenscad.github.io/learning.html}{Practice Worksheets and Guides} - Printable materials for visualization practice, decomposition exercises, and conceptual assessment
\end{itemize}

\subsection*{Works cited}\label{docs__pandoc__latex__src__3dmake_foundation__3dmake_intro__3dmake_intro.md__works-cited}

\section{3D Make Foundation: Complete Curriculum Guide}\label{docs__pandoc__latex__src__3dmake_foundation__3dmake_foundation_curriculum_guide.md__3dmake_foundation-3dmake_foundation_curriculum_guide}

A comprehensive, hands-on, screenreader accessible introduction to programmatic 3D design using OpenSCAD and 3DMake

\subsection*{Overview}\label{docs__pandoc__latex__src__3dmake_foundation__3dmake_foundation_curriculum_guide.md__overview}

This 11-lesson curriculum + 4 reference appendices teaches non-visual 3D modeling, parametric design principles, and the complete 3D printing workflow through integrated projects. Each lesson builds on prior knowledge, introducing real-world examples and hands-on activities.

Target Audience: High school and early undergraduate students, makers, and anyone interested in programmatic CAD and 3D printing

Estimated Total Time: 30-40 hours (11 lessons + 4 appendices + projects)

Contents:

\begin{itemize}
\tightlist
\item
  11 progressive lessons (Foundations -\textgreater{} Leadership)
\item
  9 hands-on projects (integrated into lessons)
\item
  4 comprehensive reference appendices (1,200-1,500 lines each)
\item
  Complete learning paths for different skill levels
\end{itemize}

\subsection*{Lesson Structure}\label{docs__pandoc__latex__src__3dmake_foundation__3dmake_foundation_curriculum_guide.md__lesson-structure}

\subsubsection*{Lesson 1: Environmental Configuration and the Developer Workflow}\label{docs__pandoc__latex__src__3dmake_foundation__3dmake_foundation_curriculum_guide.md__lesson-1-environmental-configuration-and-the-developer-workflow}

Duration: 2.5-3.5 hours \textbar{} Level: Beginner

Learn to install and configure 3dMake, create a project scaffold, and run your first build. This lesson establishes the foundation for all subsequent work.

Key Topics:

\begin{itemize}
\tightlist
\item
  Installation and tool verification (\texttt{3dm}, \texttt{openscad})
\item
  Project structure (\texttt{src/}, \texttt{build/}, \texttt{3dmake.toml})
\item
  Parametric design philosophy
\item
  The \texttt{3dm\ build} command
\end{itemize}

Hands-On Activity:

\begin{itemize}
\tightlist
\item
  Create a 3dMake project
\item
  Write a simple parametric cube model
\item
  Build and inspect the generated STL
\end{itemize}

Checkpoint: You can initialize a project, edit a parametric model, and generate an STL file

\subsubsection*{Lesson 2: Geometric Primitives and Constructive Solid Geometry}\label{docs__pandoc__latex__src__3dmake_foundation__3dmake_foundation_curriculum_guide.md__lesson-2-geometric-primitives-and-constructive-solid-geometry}

Duration: 60 minutes \textbar{} Level: Beginner

Discover how to combine basic shapes (cube, sphere, cylinder) using boolean operations (union, difference, intersection). This lesson introduces the mathematical foundation of 3D modeling.

Key Topics:

\begin{itemize}
\tightlist
\item
  Primitive shapes: \texttt{cube()}, \texttt{sphere()}, \texttt{cylinder()}
\item
  CSG operations: \texttt{union()}, \texttt{difference()}, \texttt{intersection()}
\item
  The 0.001 offset rule for avoiding non-manifold geometry
\item
  Low-resolution renders for faster debugging
\end{itemize}

Hands-On Activity:

\begin{itemize}
\tightlist
\item
  Build three simple CSG examples
\item
  Diagnose and fix a failing boolean operation
\item
  Validate geometry in a slicer
\end{itemize}

Checkpoint: You understand CSG operations and can diagnose common geometry issues

\subsubsection*{Lesson 3: Parametric Architecture and Modular Libraries}\label{docs__pandoc__latex__src__3dmake_foundation__3dmake_foundation_curriculum_guide.md__lesson-3-parametric-architecture-and-modular-libraries}

Duration: 2.5 hours \textbar{} Level: Beginner+

Learn to create reusable modules and organize code into libraries. This lesson introduces the "don\textquotesingle t repeat yourself" (DRY) principle for code reuse.

Key Topics:

\begin{itemize}
\tightlist
\item
  Module definitions with parameters
\item
  Derived calculations from parameters
\item
  Library organization (\texttt{lib/} folder)
\item
  Using external libraries (BOSL2)
\item
  Low-resolution testing with \texttt{\$fn}
\end{itemize}

Hands-On Activity:

\begin{itemize}
\tightlist
\item
  Create a parametric bracket module
\item
  Build variants by changing parameters
\item
  Move a module into a reusable library
\end{itemize}

Checkpoint: You can create parametric modules and organize code into libraries

\subsubsection*{Lesson 4: AI-Enhanced Verification and Multimodal Feedback}\label{docs__pandoc__latex__src__3dmake_foundation__3dmake_foundation_curriculum_guide.md__lesson-4-ai-enhanced-verification-and-multimodal-feedback}

Duration: 2-2.5 hours \textbar{} Level: Intermediate

Use \texttt{3dm\ info} to generate AI diagnostics and validate designs. This lesson covers verification workflows and the strengths/limitations of AI in design.

Key Topics:

\begin{itemize}
\tightlist
\item
  The \texttt{3dm\ info} command for model analysis
\item
  AI-generated model descriptions
\item
  Comparing AI suggestions to deterministic validation
\item
  Privacy and governance considerations
\item
  Prompt engineering basics
\end{itemize}

Hands-On Activity:

\begin{itemize}
\tightlist
\item
  Run \texttt{3dm\ info} on a sample model
\item
  Compare AI suggestions to slicer analysis
\item
  Document an AI-assisted design decision
\end{itemize}

Checkpoint: You can use AI tools to supplement your design validation

\subsubsection*{Lesson 5: Safety Protocols and the Physical Fabrication Interface}\label{docs__pandoc__latex__src__3dmake_foundation__3dmake_foundation_curriculum_guide.md__lesson-5-safety-protocols-and-the-physical-fabrication-interface}

Duration: 2.5-3.5 hours \textbar{} Level: Intermediate

Transition from digital design to physical printing. Learn safety procedures, environmental controls, and the complete print workflow.

Key Topics:

\begin{itemize}
\tightlist
\item
  Hierarchy of Controls (elimination, engineering, administrative, PPE)
\item
  Chemical and particulate emissions from printing
\item
  Pre-print environmental and equipment checks
\item
  Post-print inspection and measurement
\item
  Spool metadata and documentation
\end{itemize}

Hands-On Activity:

\begin{itemize}
\tightlist
\item
  Conduct a safety briefing
\item
  Verify environmental controls
\item
  Perform a pre-print checklist
\item
  Monitor and measure a completed print
\end{itemize}

Checkpoint: You understand safety procedures and can safely conduct supervised prints

\subsubsection*{Lesson 6: Practical 3dm Commands and Text Embossing}\label{docs__pandoc__latex__src__3dmake_foundation__3dmake_foundation_curriculum_guide.md__lesson-6-practical-3dm-commands-and-text-embossing}

Duration: 2.5-3.5 hours \textbar{} Level: Intermediate

Master the key 3dm commands by building a practical project-a customizable keycap with embossed text. This lesson ties together design and verification.

Key Topics:

\begin{itemize}
\tightlist
\item
  \texttt{3dm\ describe}: Text-based model analysis
\item
  \texttt{3dm\ preview}: Generating 2D tactile prints
\item
  \texttt{3dm\ orient}: Analyzing optimal print orientation
\item
  \texttt{3dm\ slice}: Generating G-code
\item
  Text embossing with \texttt{linear\_extrude()} and \texttt{text()}
\end{itemize}

Project: Parametric Cube Keycap

\begin{itemize}
\tightlist
\item
  Customize \texttt{key\_size}, \texttt{letter}, and \texttt{emboss\_depth}
\item
  Generate 3+ variants (small, medium, large)
\item
  Test emboss quality in slicer preview
\end{itemize}

Checkpoint: You can generate keycaps with embossed text and understand all major 3dm commands

\subsubsection*{Lesson 7: Parametric Transforms and the Phone Stand Project}\label{docs__pandoc__latex__src__3dmake_foundation__3dmake_foundation_curriculum_guide.md__lesson-7-parametric-transforms-and-the-phone-stand-project}

Duration: 3-3.5 hours \textbar{} Level: Intermediate+

Apply transforms (translate, rotate, scale) to build a multi-part assembly. This lesson covers spatial positioning and practical product design.

Key Topics:

\begin{itemize}
\tightlist
\item
  Transform operations: \texttt{translate()}, \texttt{rotate()}, \texttt{scale()}
\item
  Minkowski sum for edge rounding
\item
  Multi-part assemblies
\item
  Parametric angle and position variations
\item
  Testing and validation
\end{itemize}

Project: Parametric Phone Stand

\begin{itemize}
\tightlist
\item
  Design base, back, and lip components
\item
  Create configurations for phones, tablets, documents
\item
  Test orientation with \texttt{3dm\ orient}
\item
  Validate angles and friction after printing
\end{itemize}

Checkpoint: You can design multi-part assemblies with positioned components

\subsubsection*{Lesson 8: Advanced Parametric Design and Interlocking Features}\label{docs__pandoc__latex__src__3dmake_foundation__3dmake_foundation_curriculum_guide.md__lesson-8-advanced-parametric-design-and-interlocking-features}

Duration: 90-120 minutes \textbar{} Level: Advanced

Design tolerance-critical assemblies where parts snap together. This lesson covers precision manufacturing principles.

Key Topics:

\begin{itemize}
\tightlist
\item
  Tolerance and clearance management
\item
  Stack-up and cumulative tolerances
\item
  Interlocking rims and snap-fit connectors
\item
  Chamfers for quality finishing
\item
  Tolerance sensitivity testing
\end{itemize}

Project: Stackable Storage Bins

\begin{itemize}
\tightlist
\item
  Design bins with interlocking rims
\item
  Test different \texttt{stack\_clear} values
\item
  Create variants: small, medium, large
\item
  Add optional dividers
\item
  Document tolerance data
\end{itemize}

Checkpoint: You understand tolerance management and can design stackable assemblies

\subsubsection*{Lesson 9: Automation and 3dm Workflows}\label{docs__pandoc__latex__src__3dmake_foundation__3dmake_foundation_curriculum_guide.md__lesson-9-automation-and-3dm-workflows}

Duration: 2.5-3.5 hours \textbar{} Level: Advanced

Automate design workflows using shell scripts. This lesson teaches batch processing and continuous integration concepts.

Key Topics:

\begin{itemize}
\tightlist
\item
  Chaining 3dm commands with \texttt{\&\&}
\item
  Shell script basics for batch processing
\item
  Library management (\texttt{3dm\ lib})
\item
  Variant testing matrices
\item
  Production build workflows
\end{itemize}

Activities:

\begin{itemize}
\tightlist
\item
  Create a batch build script
\item
  Automate parameter variant testing
\item
  Generate production build reports
\item
  Learn library management
\end{itemize}

Checkpoint: You can automate design workflows and manage parametric variants at scale

\subsubsection*{Lesson 10: Hands-On Practice Exercises and Troubleshooting}\label{docs__pandoc__latex__src__3dmake_foundation__3dmake_foundation_curriculum_guide.md__lesson-10-hands-on-practice-exercises-and-troubleshooting}

Duration: 5-6 hours \textbar{} Level: Advanced

Complete integrated projects and learn to diagnose and fix common issues. This capstone lesson synthesizes all prior learning.

Exercise Sets:

Set A: Guided Projects

\begin{enumerate}
\tightlist
\item
  Phone Stand (Beginner): Parametric design with weight constraints
\item
  Keycap Set (Intermediate): Family of 5+ customizable caps
\item
  Storage System (Advanced): Stackable bins with tolerance management
\end{enumerate}

Set B: Problem Diagnosis

\begin{itemize}
\tightlist
\item
  Non-manifold geometry detection and fixes
\item
  Print failure prevention
\item
  Dimensional accuracy troubleshooting
\end{itemize}

Set C: Validation \& Documentation

\begin{itemize}
\tightlist
\item
  Design review checklists
\item
  Troubleshooting documentation templates
\item
  Quality assurance workflows
\end{itemize}

Checkpoint: You can complete real projects from concept to verified print

\subsection*{Learning Progressions}\label{docs__pandoc__latex__src__3dmake_foundation__3dmake_foundation_curriculum_guide.md__learning-progressions}

\subsubsection*{By Skill Level}\label{docs__pandoc__latex__src__3dmake_foundation__3dmake_foundation_curriculum_guide.md__by-skill-level}

Beginner Track (Lessons 1-3):

\begin{itemize}
\tightlist
\item
  Learn tools and project structure
\item
  Understand geometric primitives and boolean operations
\item
  Create your first parametric modules
\end{itemize}

Intermediate Track (Lessons 4-6):

\begin{itemize}
\tightlist
\item
  Explore verification and safety
\item
  Master 3dm commands
\item
  Build your first complete project (keycap)
\end{itemize}

Advanced Track (Lessons 7-10):

\begin{itemize}
\tightlist
\item
  Design complex assemblies
\item
  Manage tolerances precisely
\item
  Automate workflows and troubleshoot professionally
\end{itemize}

\subsubsection*{By Project Focus}\label{docs__pandoc__latex__src__3dmake_foundation__3dmake_foundation_curriculum_guide.md__by-project-focus}

Design \& Modeling: Lessons 1-3, 7-8
Verification \& Validation: Lessons 4, 6, 10
Safety \& Printing: Lesson 5, 10 Automation: Lesson 9

\subsection*{Practice Exercises Quick Reference}\label{docs__pandoc__latex__src__3dmake_foundation__3dmake_foundation_curriculum_guide.md__practice-exercises-quick-reference}

\subsubsection*{Structured Exercises by Lesson}\label{docs__pandoc__latex__src__3dmake_foundation__3dmake_foundation_curriculum_guide.md__structured-exercises-by-lesson}

{\def\LTcaptype{none} % do not increment counter
\begin{longtable}[]{@{}lll@{}}
\toprule\noalign{}
Lesson & Exercise Type & Deliverable \\
\midrule\noalign{}
\endhead
\bottomrule\noalign{}
\endlastfoot
1 & Setup \& Configuration & Working project scaffold \\
2 & Geometry Construction & 3 CSG examples + fixes \\
3 & Modular Design & Parametric bracket module \\
4 & AI Verification & \texttt{AI-notes.md} with findings \\
5 & Safety \& Printing & Pre-print checklist + measurements \\
6 & Command Mastery & 3+ keycap variants \\
7 & Multi-Part Assembly & Phone stand for 3+ devices \\
8 & Tolerance Design & Stackable bins + tolerance matrix \\
9 & Automation & Batch build script + variants \\
10 & Integration & 3 complete projects + documentation \\
\end{longtable}
}

\subsection*{Code Examples Repository}\label{docs__pandoc__latex__src__3dmake_foundation__3dmake_foundation_curriculum_guide.md__code-examples-repository}

All code examples from lessons are provided as:

\begin{enumerate}
\tightlist
\item
  Inline OpenSCAD code blocks in each lesson (copy-paste ready)
\item
  Shell script examples for automation tasks
\item
  Documentation templates for reproducible workflows
\end{enumerate}

\subsection*{Accessibility Features}\label{docs__pandoc__latex__src__3dmake_foundation__3dmake_foundation_curriculum_guide.md__accessibility-features}

This curriculum is designed for non-visual learners:

\begin{itemize}
\tightlist
\item
  Text descriptions of all models (\texttt{3dm\ describe})
\item
  Tactile 2D previews (\texttt{3dm\ preview})
\item
  Structured written documentation
\item
  Command-line based (no graphical interface required)
\item
  Parametric organization for clear understanding
\item
  Measurement-based validation without visual inspection
\end{itemize}

\subsection*{Recommended Reading Order}\label{docs__pandoc__latex__src__3dmake_foundation__3dmake_foundation_curriculum_guide.md__recommended-reading-order}

\subsubsection*{Option 1: Linear (Complete Foundation)}\label{docs__pandoc__latex__src__3dmake_foundation__3dmake_foundation_curriculum_guide.md__option-1-linear-complete-foundation}

Lesson 1 -\textgreater{} 2 -\textgreater{} 3 -\textgreater{} 4 -\textgreater{} 5 -\textgreater{} 6 -\textgreater{} 7 -\textgreater{} 8 -\textgreater{} 9 -\textgreater{} 10

Best for: First-time learners wanting comprehensive understanding

\subsubsection*{Option 2: Fast Track (Design Focus)}\label{docs__pandoc__latex__src__3dmake_foundation__3dmake_foundation_curriculum_guide.md__option-2-fast-track-design-focus}

Lesson 1 -\textgreater{} 2 -\textgreater{} 3 -\textgreater{} 6 -\textgreater{} 7 -\textgreater{} 8 -\textgreater{} 9

Best for: Experienced designers new to programmatic CAD

\subsubsection*{Option 3: Project-Based}\label{docs__pandoc__latex__src__3dmake_foundation__3dmake_foundation_curriculum_guide.md__option-3-project-based}

Lesson 1 -\textgreater{} 2 -\textgreater{} 3 -\textgreater{} 6 (Keycap Project)
Then: Lesson 7 (Phone Stand) Then: Lesson 8 (Storage Bins)
Then: Lesson 9 (Automation)

Best for: Learning by doing with integrated projects

\subsection*{Assessment and Completion}\label{docs__pandoc__latex__src__3dmake_foundation__3dmake_foundation_curriculum_guide.md__assessment-and-completion}

\subsubsection*{Lesson Completion Criteria}\label{docs__pandoc__latex__src__3dmake_foundation__3dmake_foundation_curriculum_guide.md__lesson-completion-criteria}

Each lesson includes:

\begin{itemize}
\tightlist
\item
  Learning objectives (concepts you\textquotesingle ll understand)
\item
  Step-by-step tasks (hands-on activities)
\item
  Checkpoints (verification milestones)
\item
  Quiz (10 self-assessment questions)
\item
  Extension problems (10 advanced challenges)
\end{itemize}

\subsubsection*{Project Completion Tracking}\label{docs__pandoc__latex__src__3dmake_foundation__3dmake_foundation_curriculum_guide.md__project-completion-tracking}

{\def\LTcaptype{none} % do not increment counter
\begin{longtable}[]{@{}
  >{\raggedright\arraybackslash}p{(\linewidth - 6\tabcolsep) * \real{0.2029}}
  >{\raggedright\arraybackslash}p{(\linewidth - 6\tabcolsep) * \real{0.1304}}
  >{\raggedright\arraybackslash}p{(\linewidth - 6\tabcolsep) * \real{0.4203}}
  >{\raggedright\arraybackslash}p{(\linewidth - 6\tabcolsep) * \real{0.2464}}@{}}
\toprule\noalign{}
\begin{minipage}[b]{\linewidth}\raggedright
Project
\end{minipage} & \begin{minipage}[b]{\linewidth}\raggedright
Lessons
\end{minipage} & \begin{minipage}[b]{\linewidth}\raggedright
Output Files
\end{minipage} & \begin{minipage}[b]{\linewidth}\raggedright
Est. Print Time
\end{minipage} \\
\midrule\noalign{}
\endhead
\bottomrule\noalign{}
\endlastfoot
Keycap Set & 1-6 & 5+ SCAD files, 3+ STL files & 15-30 min total \\
Phone Stand & 1-7 & 1 SCAD with 3+ variants & 30-60 min total \\
Storage Bins & 1-8 & 1 parametric SCAD, 3 sizes & 60-90 min total \\
\end{longtable}
}

\subsection*{Troubleshooting Guide}\label{docs__pandoc__latex__src__3dmake_foundation__3dmake_foundation_curriculum_guide.md__troubleshooting-guide}

\subsubsection*{Common Issues and References}\label{docs__pandoc__latex__src__3dmake_foundation__3dmake_foundation_curriculum_guide.md__common-issues-and-references}

{\def\LTcaptype{none} % do not increment counter
\begin{longtable}[]{@{}
  >{\raggedright\arraybackslash}p{(\linewidth - 4\tabcolsep) * \real{0.1900}}
  >{\raggedright\arraybackslash}p{(\linewidth - 4\tabcolsep) * \real{0.1600}}
  >{\raggedright\arraybackslash}p{(\linewidth - 4\tabcolsep) * \real{0.6500}}@{}}
\toprule\noalign{}
\begin{minipage}[b]{\linewidth}\raggedright
Issue
\end{minipage} & \begin{minipage}[b]{\linewidth}\raggedright
Related Lesson
\end{minipage} & \begin{minipage}[b]{\linewidth}\raggedright
Solution
\end{minipage} \\
\midrule\noalign{}
\endhead
\bottomrule\noalign{}
\endlastfoot
Model won\textquotesingle t build & Lesson 2 &
Check for non-manifold geometry with \texttt{3dm\ describe} \\
Parts don\textquotesingle t fit & Lesson 8 &
Use tolerance testing matrix to find correct \texttt{stack\_clear} \\
Print fails & Lesson 5 &
Verify pre-print checklist and slicer settings \\
Dimensions off & Lesson 10 &
Conduct tolerance sensitivity study and apply correction factor \\
Slow renders & Lesson 3 & Lower \texttt{\$fn} for faster debugging \\
Script errors & Lesson 9 &
Use \texttt{\&\&} to chain commands properly; check file paths \\
\end{longtable}
}

\subsection*{Next Steps After Completion}\label{docs__pandoc__latex__src__3dmake_foundation__3dmake_foundation_curriculum_guide.md__next-steps-after-completion}

Upon completing this curriculum, you\textquotesingle re ready for:

\begin{enumerate}
\tightlist
\item
  Advanced 3D Design: Explore BOSL2 library features, parametric assemblies
\item
  Multi-Material Printing: Design parts for different materials (TPU, Nylon, etc.)
\item
  Functional Design: Create mechanical assemblies with bearings, gears, mechanisms
\item
  Community Contribution: Share your designs and libraries with OpenSCAD community
\item
  Professional Applications: Apply skills to product design, prototyping, manufacturing
\end{enumerate}

\subsection*{Resources and References}\label{docs__pandoc__latex__src__3dmake_foundation__3dmake_foundation_curriculum_guide.md__resources-and-references}

\subsubsection*{Official Documentation}\label{docs__pandoc__latex__src__3dmake_foundation__3dmake_foundation_curriculum_guide.md__official-documentation}

\begin{itemize}
\tightlist
\item
  \href{https://en.wikibooks.org/wiki/OpenSCAD_User_Manual}{OpenSCAD Manual}
\item
  \href{https://github.com/tdeck/3dmake}{3DMake GitHub}
\item
  \href{https://github.com/revarbat/BOSL2}{BOSL2 Library}
\end{itemize}

\subsubsection*{Printers \& Slicers}\label{docs__pandoc__latex__src__3dmake_foundation__3dmake_foundation_curriculum_guide.md__printers--slicers}

\begin{itemize}
\tightlist
\item
  \href{https://help.prusa3d.com}{PrusaSlicer Documentation}
\item
  \href{https://wiki.bambulab.com}{Bambu Studio Documentation}
\item
  \href{https://github.com/SoftFever/OrcaSlicer/wiki}{OrcaSlicer Documentation}
\end{itemize}

\subsubsection*{3D Printing Guides}\label{docs__pandoc__latex__src__3dmake_foundation__3dmake_foundation_curriculum_guide.md__3d-printing-guides}

\begin{itemize}
\tightlist
\item
  \href{https://www.prusa3d.com/support/}{Prusa3D Troubleshooting}
\item
  \href{https://www.all3dp.com/guides/}{All3DP 3D Printing Guide}
\end{itemize}

\subsubsection*{Community}\label{docs__pandoc__latex__src__3dmake_foundation__3dmake_foundation_curriculum_guide.md__community}

\begin{itemize}
\tightlist
\item
  \href{https://forum.openscad.org/}{OpenSCAD Forums}
\item
  \href{https://community.prusa3d.com/}{Prusa Community}
\item
  \href{https://www.reddit.com/r/3Dprinting/}{Reddit r/3Dprinting}
\end{itemize}

\subsection*{Lesson 11: Stakeholder-Centric Design and the Beaded Jewelry Project}\label{docs__pandoc__latex__src__3dmake_foundation__3dmake_foundation_curriculum_guide.md__lesson-11-stakeholder-centric-design-and-the-beaded-jewelry-project}

Duration: 3.5-4.5 hours \textbar{} Level: Advanced/Leadership

The final lesson teaches you to design for real users, not just yourself. You\textquotesingle ll learn to conduct stakeholder interviews, extract functional requirements, and iterate based on real feedback. This bridges the gap between maker and design professional.

Key Topics:

\begin{itemize}
\tightlist
\item
  Stakeholder identification and interview techniques
\item
  Open-ended questioning strategies
\item
  Extracting functional requirements from interview data
\item
  Defining measurable acceptance criteria
\item
  Design iteration based on user feedback
\item
  Documentation for reproducibility and accessibility
\end{itemize}

Project: Beaded Jewelry Bracelet Holder

\begin{itemize}
\tightlist
\item
  Interview an actual stakeholder (or use provided scenario)
\item
  Extract their specific needs and constraints
\item
  Design a parametric holder based on their requirements
\item
  Test with actual bracelets and iterate based on feedback
\item
  Document the complete design process
\end{itemize}

Checkpoint: You can conduct a real interview, translate needs into measurable requirements, and iterate a design based on user feedback

\subsection*{3D Make Foundation Appendices}\label{docs__pandoc__latex__src__3dmake_foundation__3dmake_foundation_curriculum_guide.md__3d-make-foundation-appendices}

\subsubsection*{Appendix A: Comprehensive Slicing Guide - All Major Slicers}\label{docs__pandoc__latex__src__3dmake_foundation__3dmake_foundation_curriculum_guide.md__appendix-a-comprehensive-slicing-guide---all-major-slicers}

A complete reference covering multiple major slicers (PrusaSlicer, Bambu Studio, Cura, SuperSlicer, OrcaSlicer, IdeaMaker, Fusion 360 built in slicer).

What You\textquotesingle ll Find:

\begin{itemize}
\tightlist
\item
  Setup and workflow for each slicer
\item
  Accessible parameter explanations (non-visual)
\item
  Command-line usage for PowerShell integration
\item
  Troubleshooting guide for common problems
\item
  Comparison table for choosing the right slicer
\end{itemize}

When to Use: Reference whenever slicing, switching slicers, or troubleshooting print quality

\subsubsection*{Appendix B: Material Properties \& Selection Guide}\label{docs__pandoc__latex__src__3dmake_foundation__3dmake_foundation_curriculum_guide.md__appendix-b-material-properties--selection-guide}

Comprehensive reference for 6 common filament materials (PLA, PETG, ABS, TPU, Polycarbonate, Nylon).

What You\textquotesingle ll Find:

\begin{itemize}
\tightlist
\item
  Properties table (strength, flexibility, temperature, cost)
\item
  Ideal projects and use cases for each material
\item
  Printing parameters and settings
\item
  Quality factors and how to verify filament quality
\item
  Storage and maintenance
\item
  Cost analysis and brand recommendations
\item
  Material selection decision tree
\end{itemize}

When to Use: Reference when choosing material for a project or troubleshooting print quality issues

\subsubsection*{Appendix C: Tolerance Testing \& Quality Assurance Matrix}\label{docs__pandoc__latex__src__3dmake_foundation__3dmake_foundation_curriculum_guide.md__appendix-c-tolerance-testing--quality-assurance-matrix}

Measurement-based QA methodology designed to be used non-visually (with calipers, scales, and functional tests).

What You\textquotesingle ll Find:

\begin{itemize}
\tightlist
\item
  Essential measurement tools (digital calipers, scales, go/no-go gauges)
\item
  Step-by-step measurement procedures for all common dimensions
\item
  Functional testing methods (load testing, assembly testing, durability testing)
\item
  Tolerance stack-up calculations for multi-part designs
\item
  Troubleshooting guide for common dimensional problems
\item
  QA checklist template
\item
  Accessibility-focused best practices
\end{itemize}

When to Use: Reference when starting a new project (create test plan), after printing (verify dimensions), or when troubleshooting quality issues

\subsubsection*{Appendix D: PowerShell Integration for SCAD Workflows}\label{docs__pandoc__latex__src__3dmake_foundation__3dmake_foundation_curriculum_guide.md__appendix-d-powershell-integration-for-scad-workflows}

Shows how to automate 3D design workflows using PowerShell scripts (from PowerShell\_Foundation Lessons 1-6).

What You\textquotesingle ll Find:

\begin{itemize}
\tightlist
\item
  Basic workflow automation (SCAD -\textgreater{} STL -\textgreater{} G-code)
\item
  Parametric sweep (test 5, 10, or 100 design variations automatically)
\item
  Batch build for multiple files
\item
  Print logging and quality tracking
\item
  Printer communication (USB transfer, network monitoring)
\item
  Complete full workflow integration
\item
  PowerShell skills mapped to SCAD applications
\item
  Accessibility considerations and best practices
\end{itemize}

When to Use: Reference when automating repetitive tasks, testing parameter variations, or building a batch of designs

\subsubsection*{Appendix E: Advanced OpenSCAD Concepts}\label{docs__pandoc__latex__src__3dmake_foundation__3dmake_foundation_curriculum_guide.md__appendix-e-advanced-openscad-concepts}

Optional topics for experienced designers tackling specialized applications. Five in-depth modules with complete working examples.

What You\textquotesingle ll Find:

\begin{enumerate}
\item
  Gears and Mechanical Components

  \begin{itemize}
  \tightlist
  \item
    Gear terminology and tooth geometry
  \item
    Involute gear algorithm
  \item
    Parametric servo gearbox design
  \item
    Belt and pulley systems
  \end{itemize}
\item
  Batch Processing and Statistical Analysis

  \begin{itemize}
  \tightlist
  \item
    Automated parameter sweep generation
  \item
    Statistical summary scripts
  \item
    Data-driven design selection
  \item
    Comparison and analysis frameworks
  \end{itemize}
\item
  Performance Optimization

  \begin{itemize}
  \tightlist
  \item
    Render time measurement and profiling
  \item
    Resolution parameter strategy
  \item
    Caching complex calculations
  \item
    Preview vs render optimization strategies
  \end{itemize}
\item
  Print Orientation and Support Structure Algorithms

  \begin{itemize}
  \tightlist
  \item
    Strength orientation analysis
  \item
    Support material minimization
  \item
    Bridge span calculations
  \item
    Optimal slicing parameter determination
  \end{itemize}
\item
  Recursive Function Patterns

  \begin{itemize}
  \tightlist
  \item
    Basic recursion with base cases
  \item
    Fractal generation
  \item
    Nested component assembly
  \item
    Performance considerations for deep recursion
  \item
    Practical cable management applications
  \end{itemize}
\end{enumerate}

When to Use:

\begin{itemize}
\tightlist
\item
  When working with mechanical systems requiring gears or synchronized motion
\item
  When testing design variations systematically
\item
  When optimizing complex models for faster iteration
\item
  When designing for specific print orientations or minimizing support
\item
  When building fractal or deeply nested structures
\item
  For advanced library and framework development
\end{itemize}

Accessibility: Each example includes comprehensive comments, starts with simplified versions before advanced techniques, and documents performance implications for both visual and non-visual users

\subsection*{Curriculum Revision History}\label{docs__pandoc__latex__src__3dmake_foundation__3dmake_foundation_curriculum_guide.md__curriculum-revision-history}

{\def\LTcaptype{none} % do not increment counter
\begin{longtable}[]{@{}
  >{\raggedright\arraybackslash}p{(\linewidth - 4\tabcolsep) * \real{0.0818}}
  >{\raggedright\arraybackslash}p{(\linewidth - 4\tabcolsep) * \real{0.0909}}
  >{\raggedright\arraybackslash}p{(\linewidth - 4\tabcolsep) * \real{0.8273}}@{}}
\toprule\noalign{}
\begin{minipage}[b]{\linewidth}\raggedright
Version
\end{minipage} & \begin{minipage}[b]{\linewidth}\raggedright
Date
\end{minipage} & \begin{minipage}[b]{\linewidth}\raggedright
Changes
\end{minipage} \\
\midrule\noalign{}
\endhead
\bottomrule\noalign{}
\endlastfoot
2.1 & Feb 2026 &
Added Appendix E (Advanced OpenSCAD); Enhanced Lessons 3, 6, 7, 8, 9 with advanced topics \\
2.0 & Feb 2026 &
Added Lesson 11 (Stakeholder Design) + 4 Appendices; Consolidated Units 0-3 content \\
1.0 & Feb 2026 &
Initial comprehensive curriculum with 10 lessons + 5 projects \\
\end{longtable}
}

\subsection*{Feedback and Contributions}\label{docs__pandoc__latex__src__3dmake_foundation__3dmake_foundation_curriculum_guide.md__feedback-and-contributions}

Have suggestions to improve this curriculum? Found an issue in a lesson or code example?

Please reach out with:

\begin{itemize}
\tightlist
\item
  Lesson number and specific content reference
\item
  Description of the issue or suggestion
\item
  Proposed solution (if applicable)
\end{itemize}

Happy designing!

\section{3dmake: Non-Visual 3D Printing Tutorial}\label{docs__pandoc__latex__src__3dmake_foundation__3dmake_tutorial__3dmake_tutorial.md__3dmake-non-visual-3d-printing-tutorial}

\subsection*{Estimated time: 45-75 minutes}\label{docs__pandoc__latex__src__3dmake_foundation__3dmake_tutorial__3dmake_tutorial.md__estimated-time-45-75-minutes}

\subsubsection*{Learning Objectives}\label{docs__pandoc__latex__src__3dmake_foundation__3dmake_tutorial__3dmake_tutorial.md__learning-objectives}

\begin{itemize}
\tightlist
\item
  Describe the 3dMake command-line workflow and project structure
\item
  Create a new 3dMake project, edit \texttt{src/main.scad}, and run \texttt{3dm\ build}
\item
  Slice a model and produce a tactile preview or full G-code
\end{itemize}

\subsubsection*{Materials}\label{docs__pandoc__latex__src__3dmake_foundation__3dmake_tutorial__3dmake_tutorial.md__materials}

\begin{itemize}
\tightlist
\item
  Computer with 3dMake installed
\item
  Screen reader (NVDA, JAWS, Orca) configured
\item
  Example project files in \texttt{assets/} or classroom repo
\end{itemize}

\subsubsection*{Step-by-step Tasks}\label{docs__pandoc__latex__src__3dmake_foundation__3dmake_tutorial__3dmake_tutorial.md__step-by-step-tasks}

\begin{enumerate}
\tightlist
\item
  Verify prerequisites: confirm \texttt{3dm}, \texttt{openscad}, and your slicer are discoverable in the terminal (\texttt{which\ 3dm}, \texttt{which\ openscad}).
\item
  Create a new project: \texttt{3dm\ new} -\textgreater{} open \texttt{src/main.scad} with \texttt{3dm\ edit-model} and add a simple cube: \texttt{cube({[}20,20,20{]});}.
\item
  Build the project: \texttt{3dm\ build} and confirm \texttt{build/main.stl} exists.
\item
  Slice a preview: \texttt{3dm\ preview\ slice} then \texttt{3dm\ preview\ print} (or export preview STL) to test tactile output.
\item
  Slice for full print: \texttt{3dm\ build\ slice} -\textgreater{} inspect \texttt{build/main.gcode} with a layer-preview option in your slicer.
\end{enumerate}

\subsubsection*{Checkpoints}\label{docs__pandoc__latex__src__3dmake_foundation__3dmake_tutorial__3dmake_tutorial.md__checkpoints}

\begin{itemize}
\tightlist
\item
  After step 2 you can open and edit \texttt{src/main.scad} from the terminal.
\item
  After step 3 the \texttt{build/} folder contains \texttt{main.stl}.
\end{itemize}

\subsection*{Quick Quiz (5)}\label{docs__pandoc__latex__src__3dmake_foundation__3dmake_tutorial__3dmake_tutorial.md__quick-quiz-5}

\begin{enumerate}
\tightlist
\item
  What command creates a new 3dMake project?
\item
  Where does 3dMake put compiled STLs by default?
\item
  How do you open the model editor from the CLI?
\item
  What command slices a preview for tactile testing?
\item
  Why is it useful to run a preview before a full print?
\end{enumerate}

\subsection*{Extension Problems (10)}\label{docs__pandoc__latex__src__3dmake_foundation__3dmake_tutorial__3dmake_tutorial.md__extension-problems-10}

\begin{enumerate}
\tightlist
\item
  Create two variants of a parameterized cube (different sizes) and export both STLs; compare their file sizes and estimated print times.
\item
  Add a \texttt{3dmake.toml} overlay that changes layer height and document the visible effect on the preview.
\item
  Build and slice a small object, then import the STL into a second slicer and report any differences in estimated time.
\item
  Create a short shell script that automates new -\textgreater{} edit -\textgreater{} build for a classroom scaffold.
\item
  Describe three safety checks you will perform before starting a multi-hour print.
\item
  Develop an accessibility audit of 3DMake\textquotesingle s web interface and CLI tools; test keyboard navigation, screen-reader compatibility, and error message clarity.
\item
  Build a 3DMake workflow automation script: integrates model creation, parameter validation, and batch STL generation.
\item
  Create a 3DMake best practices guide for your classroom: document common patterns, troubleshooting tips, and performance optimization.
\item
  Design a parametric model library in 3DMake with shared modules; test reuse across multiple student projects.
\item
  Compare 3DMake vs. desktop slicers on 5+ models; create a decision matrix for when to use each tool.
\end{enumerate}

Notes: This lesson is intended to be hands-on. If networked AI features are not configured, skip the AI verification steps and focus on deterministic renders and slicer previews.

\subsection*{Helpful Commands}\label{docs__pandoc__latex__src__3dmake_foundation__3dmake_tutorial__3dmake_tutorial.md__helpful-commands}

\subsubsection*{Library commands}\label{docs__pandoc__latex__src__3dmake_foundation__3dmake_tutorial__3dmake_tutorial.md__library-commands}

\begin{lstlisting}[style=Alabaster, language=powershell]
3dm list-libraries
3dm install-libraries

\end{lstlisting}

\subsubsection*{Help and version}\label{docs__pandoc__latex__src__3dmake_foundation__3dmake_tutorial__3dmake_tutorial.md__help-and-version}

\begin{lstlisting}[style=Alabaster, language=powershell]
3dm help
3dm version

\end{lstlisting}

\section*{References (APA)}\label{docs__pandoc__latex__src__3dmake_foundation__3dmake_tutorial__3dmake_tutorial.md__references-apa}

Deck, T. (n.d.). \emph{3dmake: Non-visual 3D design and 3D printing tool}. GitHub. Retrieved February 18, 2026, from \url{http://github.com/tdeck/3dmake}

OpenSCAD. (n.d.). \emph{The programmers solid 3D CAD modeller}. Retrieved February 18, 2026, from \url{https://openscad.org}

\begin{lstlisting}[style=Alabaster]


Other Screen Readers

Dolphin SuperNova (commercial) and Windows Narrator (built-in) are also supported; the workflows and recommendations in this document apply to them. See [https://yourdolphin.com/supernova/](https://yourdolphin.com/supernova/) and [https://support.microsoft.com/narrator](https://support.microsoft.com/narrator) for vendor documentation.

\end{lstlisting}

\section{Lesson 1: Environmental Configuration and the Developer Workflow}\label{docs__pandoc__latex__src__3dmake_foundation__lessons_3dmake_1__lessons_3dmake_1.md__lesson-1-environmental-configuration-and-the-developer-workflow}

Estimated time: 90-120 minutes

\subsection*{Learning Objectives}\label{docs__pandoc__latex__src__3dmake_foundation__lessons_3dmake_1__lessons_3dmake_1.md__learning-objectives}

\begin{itemize}
\tightlist
\item
  Install and verify 3dm\footnote{3DMake GitHub Repository - \url{https://github.com/tdeck/3dmake}}, openscad\footnote{OpenSCAD Manual - \url{https://en.wikibooks.org/wiki/OpenSCAD_User_Manual}}, and a slicer are discoverable in the terminal
\item
  Initialize a 3dMake project and understand the project scaffold (\texttt{src/}, \texttt{build/}, \texttt{3dmake.toml})
\item
  Edit \texttt{src/main.scad} using OpenSCAD\textquotesingle s parametric design capabilities\footnote{OpenSCAD Parametric Design - \url{https://en.wikibooks.org/wiki/OpenSCAD_User_Manual/The_OpenSCAD_Language\#Variables}}, run \texttt{3dm\ build}, and inspect the generated \texttt{build/main.stl}
\end{itemize}

\subsection*{Materials}\label{docs__pandoc__latex__src__3dmake_foundation__lessons_3dmake_1__lessons_3dmake_1.md__materials}

\begin{itemize}
\tightlist
\item
  Terminal with 3dMake installed
\item
  Editor (VS Code or Notepad)
\item
  Example scaffold or classroom repository
\end{itemize}

\subsection*{Step-by-step Tasks}\label{docs__pandoc__latex__src__3dmake_foundation__lessons_3dmake_1__lessons_3dmake_1.md__step-by-step-tasks}

\begin{enumerate}
\item
  Run \texttt{./3dm\ setup} or follow instructor\textquotesingle s installation notes; confirm tools with \texttt{which\ 3dm} and \texttt{which\ openscad}. Verify your 3dMake installation is properly configured\footnote{3DMake GitHub Repository - \url{https://github.com/tdeck/3dmake}}.
\item
  Create a project scaffold with \texttt{3dm\ new} and open \texttt{src/main.scad} using \texttt{3dm\ edit-model}. For a comprehensive introduction to the workflow, consult the OpenSCAD documentation\footnote{OpenSCAD Manual - \url{https://en.wikibooks.org/wiki/OpenSCAD_User_Manual}}.
\item
  Add three top-level parameters (e.g., \texttt{width}, \texttt{height}, \texttt{thickness}) and a minimal model. This demonstrates the parametric design philosophy central to OpenSCAD\footnote{OpenSCAD Parametric Design - \url{https://en.wikibooks.org/wiki/OpenSCAD_User_Manual/The_OpenSCAD_Language\#Variables}}.

  Example \texttt{src/main.scad}:

  \begin{lstlisting}[style=Alabaster, language=openscad]
  // Top-level parameters (change these to customize your model)
  width = 50;      // mm
  height = 30;     // mm
  thickness = 5;   // mm
  // Main model
  cube([width, height, thickness]);

  \end{lstlisting}
\item
  Run \texttt{3dm\ build} and verify \texttt{build/main.stl} exists. Compare this build process to standard OpenSCAD workflows\footnote{OpenSCAD Review - Worth learning? - CadHub, accessed February 18, 2026, \url{https://learn.cadhub.xyz/blog/openscad-review/}}.
\item
  Open the STL in your slicer to check for thin walls or non-manifold geometry\footnote{OpenSCAD User Manual - Non-Manifold Geometry, accessed February 18, 2026, \url{https://en.wikibooks.org/wiki/OpenSCAD_User_Manual/FAQ\#Why_is_my_model_not_manifold.3F}}; if issues appear, iterate on \texttt{main.scad} and rebuild.
\item
  Try modifying the parameters and running \texttt{3dm\ build} again to see how parametric design allows you to quickly create variants.
\end{enumerate}

\subsection*{Code Style and Documentation Standards}\label{docs__pandoc__latex__src__3dmake_foundation__lessons_3dmake_1__lessons_3dmake_1.md__code-style-and-documentation-standards}

Professional OpenSCAD code follows consistent documentation practices for clarity, maintainability, and accessibility. As you write more complex designs, good documentation becomes essential for both you and anyone reading your code.

\subsubsection*{Comment Types and When to Use Them}\label{docs__pandoc__latex__src__3dmake_foundation__lessons_3dmake_1__lessons_3dmake_1.md__comment-types-and-when-to-use-them}

File Header Comments - Describe the entire file\textquotesingle s purpose:

\begin{lstlisting}[style=Alabaster, language=openscad]
// ==============================================================
// Parametric Phone Stand - Design v2.1
// ==============================================================
// Purpose: Multi-angle viewing stand for phones/tablets
// Author: Alex Chen
// Created: February 2026
// Last Modified: February 23, 2026
// 
// Parameters: phone_width, angle, lip_height
// Dependencies: None (standalone file)
// Print Time Estimate: 45-60 min (PLA, 0.20mm layer height)
// Material: ~50g PLA
// ==============================================================

\end{lstlisting}

Section Comments - Organize code into logical blocks:

\begin{lstlisting}[style=Alabaster, language=openscad]
// ============================================
// CUSTOMIZABLE PARAMETERS
// ============================================
phone_width = 75;    // mm
phone_height = 150;  // mm
stand_angle = 60;    // degrees
// ============================================
// DERIVED CALCULATIONS (calculated from parameters)
// ============================================
base_width = phone_width + 20;
lip_height = 15;
// ============================================
// MODULE DEFINITIONS
// ============================================
module base() { ... }
module stand() { ... }

\end{lstlisting}

Inline Comments - Clarify complex logic:

\begin{lstlisting}[style=Alabaster, language=openscad]
// Use minkowski for smooth edges (prevents sharp corners)
module rounded_base() {
  minkowski() {
    cube([width - 2*radius, depth - 2*radius, height], center=true);
    cylinder(r=radius, h=0.01, $fn=32);  // $fn=32 for smooth curve
  }
}

\end{lstlisting}

Parameter Documentation - Explain units and constraints:

\begin{lstlisting}[style=Alabaster, language=openscad]
// Parameter ranges and units
width = 100;         // mm - must be > 50 mm for stability
height = 50;         // mm - can be 20-100 mm
wall_thickness = 2;  // mm - 1.5-3 mm recommended (too thin breaks, too thick wastes material)
angle = 45;          // degrees - 30-80 degrees typical
$fn = 32;            // render quality - use 16-20 for preview, 32+ for export

\end{lstlisting}

\subsubsection*{Documentation Best Practices}\label{docs__pandoc__latex__src__3dmake_foundation__lessons_3dmake_1__lessons_3dmake_1.md__documentation-best-practices}

\begin{enumerate}
\tightlist
\item
  Write for Your Audience
\end{enumerate}

\begin{itemize}
\tightlist
\item
  Document assumptions and constraints
\item
  Explain non-obvious design choices
\item
  Use clear parameter names (\texttt{wall\_thickness} not \texttt{w})
\end{itemize}

\begin{enumerate}
\setcounter{enumi}{1}
\tightlist
\item
  Include Module Documentation
\end{enumerate}

\begin{lstlisting}[style=Alabaster, language=openscad]
// Create a hollow box with walls of specified thickness
// Parameters: outer_w, outer_d, outer_h (mm), wall (mm)
// Example: hollow_box(50, 50, 50, 2) creates 50x50x50 box with 2mm walls
module hollow_box(outer_w, outer_d, outer_h, wall) {
  difference() {
    cube([outer_w, outer_d, outer_h], center=true);
    cube([outer_w - 2*wall, outer_d - 2*wall, outer_h], center=true);
  }
}

\end{lstlisting}

\begin{enumerate}
\setcounter{enumi}{2}
\tightlist
\item
  Document Known Limitations
\end{enumerate}

\begin{lstlisting}[style=Alabaster, language=openscad]
// Phone Stand v2
// Limitations:
// - Not suitable for phones heavier than 200g
// - Recommend PLA or PETG (TPU may warp at recommended angles)
// - Print with 20%+ infill for stability
// - Supports may be needed on underside of angle > 70 degrees

\end{lstlisting}

\begin{enumerate}
\setcounter{enumi}{3}
\tightlist
\item
  Accessibility in Code Documentation
\end{enumerate}

\begin{itemize}
\tightlist
\item
  Use plain language, not jargon (or explain jargon)
\item
  Explain visual concepts in text (e.g., "fillet radius of 3mm rounds sharp corners")
\item
  Include parameter ranges and units always
\item
  Describe geometry relationships (e.g., "lip height 1/3 of stand height")
\end{itemize}

\subsubsection*{Example: Well-Documented File}\label{docs__pandoc__latex__src__3dmake_foundation__lessons_3dmake_1__lessons_3dmake_1.md__example-well-documented-file}

\begin{lstlisting}[style=Alabaster, language=openscad]
// ==============================================================
// Parametric Keycap with Embossed Letter
// ==============================================================
// Purpose: Customizable keyboard keycap for 3D printing
// Applications: Custom keyboards, gaming, accessibility
// Print Time: 3-5 min per cap (varies by size)
// Material: ~2g PLA per cap
//
// PARAMETERS TO CUSTOMIZE:
// - cap_size: 12-28 mm (small to large keycap)
// - cap_height: 6-15 mm
// - letter: any single character to emboss
//
// PRINT RECOMMENDATIONS:
// - Layer height: 0.15 mm (better letter quality)
// - Infill: 15% (sufficient for keyboard use)
// - No supports needed
// - Print with smooth base facing bed
//
// TESTING CHECKLIST:
// - Measure cap_size with calipers (should match design within 0.3mm)
// - Test embossed letter is legible (raised ~0.5mm from surface)
// - Test fit on actual keyboard switch (should snap fit snugly)
// ==============================================================
// ============================================
// CUSTOMIZABLE PARAMETERS
// ============================================
cap_size = 14;        // mm - key size (typical: 12-18 for alphanumeric)
cap_height = 10;      // mm - distance from base to top
wall_thickness = 1.2; // mm - side wall thickness (1-1.5 typical)
letter = "A";         // Character to emboss on top
emboss_depth = 0.8;   // mm - how deep letter is raised (0.5-1.0 recommended)
// ============================================
// DERIVED CALCULATIONS
// ============================================
inner_size = cap_size - 2 * wall_thickness;
// ============================================
// MODULE: Hollow shell
// ============================================
module keycap_shell() {
  difference() {
    // Outer box
    cube([cap_size, cap_size, cap_height], center = false);
    // Hollow interior (removed part)
    translate([wall_thickness, wall_thickness, wall_thickness])
      cube([inner_size, inner_size, cap_height], center = false);
  }
}
// ============================================
// MODULE: Embossed letter on top
// ============================================
module embossed_letter() {
  translate([cap_size / 2, cap_size / 2, cap_height - 0.01])
    linear_extrude(height = emboss_depth)
      text(letter, 
           size = cap_size * 0.5,
           halign = "center",
           valign = "center",
           font = "Impact:style=Regular");
}
// ============================================
// ASSEMBLY: Combine shell + emboss
// ============================================
union() {
  keycap_shell();
  embossed_letter();
}

\end{lstlisting}

This level of documentation makes your code:

\begin{itemize}
\tightlist
\item
  Reusable: Others can use it without reading every line
\item
  Maintainable: You can modify it months later and understand why things are designed a certain way
\item
  Accessible: Non-visual users can understand the design intent and constraints
\item
  Professional: Employers and collaborators trust well-documented code
\end{itemize}

\subsubsection*{Checkpoints}\label{docs__pandoc__latex__src__3dmake_foundation__lessons_3dmake_1__lessons_3dmake_1.md__checkpoints}

\begin{itemize}
\tightlist
\item
  After step 2 you can locate \texttt{3dmake.toml} and the \texttt{build/} directory. Ensure your project scaffold matches the expected structure described in the 3dMake repository\footnote{3DMake GitHub Repository - \url{https://github.com/tdeck/3dmake}}.
\item
  After step 4 the \texttt{build/} folder contains a valid \texttt{main.stl}. Verify the geometry using your slicer\textquotesingle s validation tools\footnote{Slicer Validation Tools - PrusaSlicer Documentation - \url{https://docs.prusa3d.com/en/guide/39012-validation-tools/}}.
\item
  After this section, your code includes file header, parameter documentation, and module descriptions following professional standards.
\end{itemize}

\subsection*{Understanding 3D Printing Technology: The FDM Pipeline}\label{docs__pandoc__latex__src__3dmake_foundation__lessons_3dmake_1__lessons_3dmake_1.md__understanding-3d-printing-technology-the-fdm-pipeline}

Before you send your first print, it\textquotesingle s important to understand how 3D printers work and how the choices you make in the slicer affect the final result. This section builds on the workflow you learned above by explaining \emph{what happens after} \texttt{3dm\ build}.

\subsubsection*{The FDM (Fused Deposition Modeling) Process}\label{docs__pandoc__latex__src__3dmake_foundation__lessons_3dmake_1__lessons_3dmake_1.md__the-fdm-fused-deposition-modeling-process}

FDM printing builds objects layer by layer, where each layer is a thin horizontal slice of your STL file. Here\textquotesingle s the complete pipeline:

\begin{enumerate}
\tightlist
\item
  STL File -\textgreater{} Your 3dMake model exported as an STL geometry file
\item
  Slicer Analysis -\textgreater{} Software like PrusaSlicer reads the STL and divides it into layers
\item
  G-code Generation -\textgreater{} The slicer converts layers into machine instructions (coordinates, temperature, speed)
\item
  Printing -\textgreater{} The printer reads G-code, heats the nozzle to \textasciitilde{}200-230C, and extrudes plastic one layer at a time
\item
  Cooling \& Solidification -\textgreater{} Each layer cools and bonds to the layer below
\end{enumerate}

\subsubsection*{Critical Settings That Affect Your Print}\label{docs__pandoc__latex__src__3dmake_foundation__lessons_3dmake_1__lessons_3dmake_1.md__critical-settings-that-affect-your-print}

When you open your STL in a slicer, you\textquotesingle ll encounter several parameters that directly impact quality, time, and strength:

\paragraph*{Layer Height}\label{docs__pandoc__latex__src__3dmake_foundation__lessons_3dmake_1__lessons_3dmake_1.md__layer-height}

\begin{itemize}
\tightlist
\item
  Definition: The thickness of each printed layer (typically 0.15-0.30 mm)
\item
  Effect on Time: Smaller layers = more detail but longer print time. A layer height of 0.15 mm prints slower than 0.30 mm because more layers must be printed
\item
  Effect on Quality: Smaller layers produce smoother surfaces; larger layers print faster but appear more "stepped"
\item
  Common Choice: 0.20 mm is a good balance for classroom projects
\end{itemize}

\paragraph*{Infill}\label{docs__pandoc__latex__src__3dmake_foundation__lessons_3dmake_1__lessons_3dmake_1.md__infill}

\begin{itemize}
\tightlist
\item
  Definition: The interior solid percentage of your model (0-100\%)
\item
  Purpose: Infill provides internal strength without using solid material throughout (which would be wasteful and heavy)
\item
  Common Values for Classroom: 15-20\% infill is typical; 10\% for very light parts, 50\% for functional parts
\item
  Infill Patterns: Grid, gyroid, or honeycomb patterns determine how the internal structure looks. Grid is simple and fast; gyroid is strong but more complex
\item
  Rule of Thumb: Higher infill = stronger, heavier, and longer print time
\end{itemize}

\paragraph*{Supports}\label{docs__pandoc__latex__src__3dmake_foundation__lessons_3dmake_1__lessons_3dmake_1.md__supports}

\begin{itemize}
\tightlist
\item
  Definition: Temporary structures the printer creates to hold overhanging geometry during printing
\item
  When Needed: Any geometry that "hangs" at a steep angle (typically \textgreater{} 45 from vertical) requires supports
\item
  Post-Processing: Supports must be removed after printing (breaking them off, dissolving them, or picking them away)
\item
  Cost: Supports increase print time and waste material, so good STL design minimizes them
\end{itemize}

\subsubsection*{Why This Matters for Your Design}\label{docs__pandoc__latex__src__3dmake_foundation__lessons_3dmake_1__lessons_3dmake_1.md__why-this-matters-for-your-design}

When you wrote \texttt{src/main.scad} in the tasks above, you created a parametric model. Those parameters become \emph{constraints} that affect how well your part prints:

\begin{itemize}
\tightlist
\item
  A \texttt{thickness} of 0.5 mm might be too thin for FDM (will break easily)
\item
  A \texttt{width} of 300 mm will take many hours to print
\item
  Sharp corners and thin walls can cause printing failures
\end{itemize}

Understanding the FDM pipeline helps you design parts that not only look correct in OpenSCAD but will actually \emph{print successfully}.

\subsubsection*{Next Steps: The Slicer}\label{docs__pandoc__latex__src__3dmake_foundation__lessons_3dmake_1__lessons_3dmake_1.md__next-steps-the-slicer}

After you create an STL with \texttt{3dm\ build}, you open it in a slicer to:

\begin{enumerate}
\tightlist
\item
  Verify geometry (check for thin walls or non-manifold faces)
\item
  Set layer height, infill, and supports
\item
  Preview the layers to see what the printer will do
\item
  Export as G-code to send to the printer
\end{enumerate}

You\textquotesingle ll practice this workflow in Lesson 2, where you\textquotesingle ll iterate on a simple part and resolve common printing issues.

\subsection*{Getting Started: Guided Projects \& Extension Resources}\label{docs__pandoc__latex__src__3dmake_foundation__lessons_3dmake_1__lessons_3dmake_1.md__getting-started-guided-projects--extension-resources}

Once you\textquotesingle ve completed this lesson, you\textquotesingle re ready to work on hands-on projects. The curriculum includes several guided projects and extension resources:

\subsubsection*{Extension Projects (Beginner to Advanced)}\label{docs__pandoc__latex__src__3dmake_foundation__lessons_3dmake_1__lessons_3dmake_1.md__extension-projects-beginner-to-advanced}

These projects are located in the Lesson 1 assets folder and are designed to reinforce your skills in practical contexts:

\begin{itemize}
\item
  Your First Print (\href{docs/pandoc/latex/src/assets/3dMake_Foundation/Lessons_3dMake_1/Your_First_Print/your-first-print.md}{Lesson 1 Assets - Your First Print})

  \begin{itemize}
  \tightlist
  \item
    Goal: Low-friction introduction to the complete printing workflow
  \item
    Skills: Setup, basic slicing, first-time print validation
  \item
    Best for: After completing Lesson 1
  \item
    Asset folder: \href{docs/pandoc/latex/src/assets/3dMake_Foundation/Lessons_3dMake_1/Your_First_Print}{assets/Lessons\_3dMake\_1/Your\_First\_Print/}
  \end{itemize}
\item
  Basic Project Scaffold Template (\href{docs/pandoc/latex/src/assets/3dMake_Foundation/Lessons_3dMake_1/basic_project_scaffold.scad}{Lesson 1 Assets - SCAD Template})

  \begin{itemize}
  \tightlist
  \item
    A starter template for your own 3D printing projects
  \item
    Includes parameter configuration and TODO sections
  \end{itemize}
\end{itemize}

\subsubsection*{Learning Series Sample Projects}\label{docs__pandoc__latex__src__3dmake_foundation__lessons_3dmake_1__lessons_3dmake_1.md__learning-series-sample-projects}

The \texttt{3dmake\_learning\_series/} folder contains worked examples aligned with this curriculum:

\begin{itemize}
\tightlist
\item
  01\_cube\_keycap (Beginner) - Text embossing basics
\item
  02\_parametric\_phone\_stand (Intermediate) - Transforms and Minkowski fillets
\item
  03\_stackable\_bins (Advanced) - Tolerance and assemblies
\end{itemize}

\subsubsection*{Reference Materials}\label{docs__pandoc__latex__src__3dmake_foundation__lessons_3dmake_1__lessons_3dmake_1.md__reference-materials}

Quick-reference guides are available in \texttt{Reference\_Materials/}:

\begin{itemize}
\tightlist
\item
  3dmake-setup-guide.md - Complete setup walkthrough and command reference
\item
  openscad-cheat-sheet.md - Keyboard shortcuts, syntax, and common functions
\item
  filament-comparison-table.md - Material properties for different print scenarios
\item
  master-rubric.md - Assessment criteria for evaluating student work
\end{itemize}

\subsection*{Quiz - Lesson 3dMake.1 (15 questions)}\label{docs__pandoc__latex__src__3dmake_foundation__lessons_3dmake_1__lessons_3dmake_1.md__quiz---lesson-3dmake1-15-questions}

\begin{enumerate}
\tightlist
\item
  What command initializes a 3dMake project?
\item
  What folder holds generated STLs?
\item
  How do you open the main model editor from the CLI?
\item
  Why is it useful to run \texttt{3dm\ build} frequently during development?
\item
  Give one reason to prefer an external editor over editing inline.
\item
  True or False: 3dMake requires a graphical user interface to use effectively.
\item
  Explain what the \texttt{3dmake.toml} file does in your project.
\item
  Describe what the \texttt{src/}, \texttt{build/}, and other project scaffold folders are used for.
\item
  How would you compare the 3dMake build workflow to traditional OpenSCAD workflows?
\item
  What validation steps should you perform after running \texttt{3dm\ build} and before sending a file to print?
\item
  What is the difference between the global configuration file (\texttt{defaults.toml}) and the project configuration file (\texttt{3dmake.toml})? Which takes precedence when both define the same setting?
\item
  Describe what a TOML boolean value looks like and give one example of a 3dMake setting that would use a boolean.
\item
  What command would you run to see all available 3dMake commands, and what command shows the installed version number?
\item
  Explain what FDM stands for and describe the five-step pipeline from STL file to cooled physical part.
\item
  You run \texttt{3dm\ build} and receive the error "No such file or directory: src/main.scad". What are two likely causes and how would you fix each one?
\end{enumerate}

\subsection*{Extension Problems (15)}\label{docs__pandoc__latex__src__3dmake_foundation__lessons_3dmake_1__lessons_3dmake_1.md__extension-problems-15}

\begin{enumerate}
\tightlist
\item
  Add a README entry explaining your top-level parameters and expected units. Reference best practices from the OpenSCAD documentation\footnote{OpenSCAD Manual - \url{https://en.wikibooks.org/wiki/OpenSCAD_User_Manual}}.
\item
  Create a parameter variant by changing \texttt{width} by 20\% and build both variants; compare dimensions with calipers. This demonstrates the power of parametric design discussed in programming resources\footnote{OpenSCAD Parametric Design - \url{https://en.wikibooks.org/wiki/OpenSCAD_User_Manual/The_OpenSCAD_Language\#Variables}}.
\item
  Script a \texttt{3dm} command sequence that automates new -\textgreater{} edit -\textgreater{} build for the scaffold. Review the 3dMake test suite for inspiration\footnote{3dmake/e2e\_test.py at main - GitHub, accessed February 18, 2026, \url{https://github.com/tdeck/3dmake/blob/main/e2e_test.py}}.
\item
  Intentionally create a thin-wall error and document the steps you took to find and fix it. Consult slicing guides\footnote{Slicing Guides and Common Geometry Issues - PrusaSlicer Documentation, accessed February 18, 2026, \url{https://docs.prusa3d.com/en/}} for identifying common geometry issues.
\item
  Prepare a short instructor sign-off checklist describing safety checks before printing.
\item
  Build a variant testing suite: create 5+ parameter combinations, export STLs, and compare file sizes and estimated print times.
\item
  Create a 3dMake project template with best-practice structure, documentation, and reusable modules for future projects.
\item
  Develop a screen-reader accessibility guide for 3dMake CLI commands and parameter syntax.
\item
  Design a parametric part library in 3dMake format; document all parameters, units, and example usage.
\item
  Write a comprehensive troubleshooting guide for common 3dMake build errors, with solutions and prevention tips.
\item
  Add a \texttt{.gitignore} file to a new project, commit the scaffold using Git, then make a parameter change and create a second commit. Write a one-paragraph explanation of why version control is valuable for parametric design.
\item
  Edit the global configuration file using \texttt{3dm\ edit-global-config} to change the default editor. Document the setting name, the value you chose, and how you verified the change took effect.
\item
  Create a project with three \texttt{.scad} files (e.g., \texttt{main.scad}, \texttt{lid.scad}, \texttt{base.scad}) and build each using \texttt{3dm\ build\ -m\ \textless{}name\textgreater{}}. Record the output STL filenames for each.
\item
  Research the FDM layer height trade-off: slice the same model at 0.15 mm, 0.20 mm, and 0.30 mm layer heights. Record estimated print time and filament use for each. Write two sentences explaining the trade-off.
\item
  Write a one-page "new student onboarding guide" for 3dMake that covers installation, project creation, first build, and first print. Use accessible language suitable for someone with no prior terminal experience.
\end{enumerate}

\subsection*{Supplemental Resources}\label{docs__pandoc__latex__src__3dmake_foundation__lessons_3dmake_1__lessons_3dmake_1.md__supplemental-resources}

For deeper exploration of OpenSCAD and parametric design, consult these resources:

\begin{itemize}
\tightlist
\item
  \href{docs/pandoc/latex/src/assets/Programming_with_OpenSCAD.epub}{Programming with OpenSCAD EPUB Textbook} - Comprehensive reference with examples of parametric design, transformations, and modules
\item
  \href{https://github.com/ProgrammingWithOpenSCAD/CodeSolutions}{CodeSolutions Repository} - Working OpenSCAD code organized by topic, including 3D primitives and parametric examples relevant to Lesson 1
\item
  \href{https://programmingwithopenscad.github.io/quick-reference.html}{OpenSCAD Quick Reference} - Visual syntax guide and command reference
\end{itemize}

\subsection{3dMake Setup \& Workflow}\label{docs__pandoc__latex__src__3dmake_foundation__lessons_3dmake_1__3dmake-setup-guide.md__3dmake-setup--workflow}

This guide walks you through installing 3dMake, creating projects, and managing your workflow efficiently.

\subsubsection*{Installing 3dMake}\label{docs__pandoc__latex__src__3dmake_foundation__lessons_3dmake_1__3dmake-setup-guide.md__installing-3dmake}

\paragraph*{Prerequisites}\label{docs__pandoc__latex__src__3dmake_foundation__lessons_3dmake_1__3dmake-setup-guide.md__prerequisites}

Before installing 3dMake, ensure you have:

\begin{itemize}
\tightlist
\item
  Windows or Linux operating system (macOS is not currently supported)
\item
  Terminal or PowerShell access
\item
  Internet connection to download 3dMake
\item
  At least 100 MB of free disk space
\end{itemize}

\paragraph*{Platform Support}\label{docs__pandoc__latex__src__3dmake_foundation__lessons_3dmake_1__3dmake-setup-guide.md__platform-support}

Supported:

\begin{itemize}
\tightlist
\item
  Windows (32-bit and 64-bit)
\item
  Linux (x86-64 architecture)
\end{itemize}

Not Supported:

\begin{itemize}
\tightlist
\item
  macOS - Currently no macOS version is available
\end{itemize}

\paragraph*{Step-by-Step Installation}\label{docs__pandoc__latex__src__3dmake_foundation__lessons_3dmake_1__3dmake-setup-guide.md__step-by-step-installation}

\subparagraph*{Step 1: Download 3dMake for Your Operating System}\label{docs__pandoc__latex__src__3dmake_foundation__lessons_3dmake_1__3dmake-setup-guide.md__step-1-download-3dmake-for-your-operating-system}

Visit the 3dMake releases page and download the appropriate version:

Windows:

\begin{enumerate}
\tightlist
\item
  Go to \href{https://github.com/tdeck/3dmake/releases/latest/download/3dmakewindows.zip}{3dMake Windows Download}
\item
  This downloads \texttt{3dmakewindows.zip} to your computer
\item
  Right-click the file and select "Extract All..." (or use your preferred extraction tool)
\item
  Remember where you extracted it (e.g., \texttt{C:\textbackslash{}Users\textbackslash{}YourName\textbackslash{}3dmake} or \texttt{C:\textbackslash{}Program\ Files\textbackslash{}3dmake})
\end{enumerate}

Linux:

\begin{enumerate}
\tightlist
\item
  Go to \href{https://github.com/tdeck/3dmake/releases/latest/download/3dmakelinux.tar.gz}{3dMake Linux Download}
\item
  This downloads \texttt{3dmakelinux.tar.gz} to your computer
\item
  Extract it using: \texttt{tar\ -xzf\ 3dmakelinux.tar.gz}
\item
  Remember where you extracted it (e.g., \texttt{\textasciitilde{}/3dmake} or \texttt{/opt/3dmake})
\end{enumerate}

\subparagraph*{Step 2: Open Your Terminal}\label{docs__pandoc__latex__src__3dmake_foundation__lessons_3dmake_1__3dmake-setup-guide.md__step-2-open-your-terminal}

Windows:

\begin{itemize}
\tightlist
\item
  Press \texttt{Win\ +\ X} and select "Windows PowerShell" or "Terminal"
\item
  For screen reader users: Use \texttt{Alt\ +\ F2} and type \texttt{powershell} if needed
\end{itemize}

Linux:

\begin{itemize}
\tightlist
\item
  Open your terminal application (Terminal, Konsole, GNOME Terminal, etc.)
\end{itemize}

\subparagraph*{Step 3: Navigate to 3dMake Directory}\label{docs__pandoc__latex__src__3dmake_foundation__lessons_3dmake_1__3dmake-setup-guide.md__step-3-navigate-to-3dmake-directory}

Navigate to where you extracted 3dMake:

Windows:

\begin{lstlisting}[style=Alabaster, language=powershell]
cd C:\Users\YourName\3dmake

\end{lstlisting}

Linux:

\begin{lstlisting}[style=Alabaster, language=bash]
cd ~/3dmake

\end{lstlisting}

For screen reader users: Use \texttt{pwd} (print working directory) to confirm your location.

\subparagraph*{Step 4: Run the Setup Command}\label{docs__pandoc__latex__src__3dmake_foundation__lessons_3dmake_1__3dmake-setup-guide.md__step-4-run-the-setup-command}

From inside the 3dMake directory, run:

Windows:

\begin{lstlisting}[style=Alabaster, language=powershell]
.\3dm setup

\end{lstlisting}

Linux:

\begin{lstlisting}[style=Alabaster, language=bash]
./3dm setup

\end{lstlisting}

Follow the prompts to configure:

\begin{itemize}
\tightlist
\item
  Your default printer profile (e.g., Prusa MK4, Bambu Lab)
\item
  OctoPrint connection (if you use OctoPrint)
\item
  AI integration (optional - for model descriptions)
\item
  Preferred text editor
\end{itemize}

\subparagraph*{Step 5: Complete Installation}\label{docs__pandoc__latex__src__3dmake_foundation__lessons_3dmake_1__3dmake-setup-guide.md__step-5-complete-installation}

After setup finishes, 3dMake will be available from any directory. You can now use the \texttt{3dm} command from your terminal.

Do not delete the original 3dMake directory where you extracted the files, as 3dMake needs to reference it.

\subparagraph*{Step 6: Verify Installation}\label{docs__pandoc__latex__src__3dmake_foundation__lessons_3dmake_1__3dmake-setup-guide.md__step-6-verify-installation}

\begin{lstlisting}[style=Alabaster, language=bash]
3dm --version

\end{lstlisting}

You should see the installed version number. If you see an error, verify:

\begin{itemize}
\tightlist
\item
  You\textquotesingle re in the 3dMake directory
\item
  The extraction completed successfully
\item
  Your terminal has access to the extracted files
\end{itemize}

\subparagraph*{Step 7: Get Help}\label{docs__pandoc__latex__src__3dmake_foundation__lessons_3dmake_1__3dmake-setup-guide.md__step-7-get-help}

\begin{lstlisting}[style=Alabaster, language=bash]
3dm help

\end{lstlisting}

This displays all available 3dMake commands.

\subsubsection*{Creating and Managing Projects}\label{docs__pandoc__latex__src__3dmake_foundation__lessons_3dmake_1__3dmake-setup-guide.md__creating-and-managing-projects}

\paragraph*{What Is a 3dMake Project?}\label{docs__pandoc__latex__src__3dmake_foundation__lessons_3dmake_1__3dmake-setup-guide.md__what-is-a-3dmake-project}

A 3dMake project is a folder structure that organizes:

\begin{itemize}
\tightlist
\item
  src/ folder - contains your OpenSCAD (.scad) files
\item
  build/ folder - stores outputs that 3dMake generates (STL files, sliced GCODE, etc.)
\item
  3dmake.toml file - project configuration and settings
\item
  README.md - project documentation
\end{itemize}

\paragraph*{Creating Your First Project}\label{docs__pandoc__latex__src__3dmake_foundation__lessons_3dmake_1__3dmake-setup-guide.md__creating-your-first-project}

\subparagraph*{Step 1: Choose or Create a Project Directory}\label{docs__pandoc__latex__src__3dmake_foundation__lessons_3dmake_1__3dmake-setup-guide.md__step-1-choose-or-create-a-project-directory}

Decide where you want your project. Examples:

Windows:

\begin{lstlisting}[style=Alabaster, language=powershell]
C:\Users\YourName\Documents\3d-projects\
C:\Users\YourName\Desktop\MyProject\

\end{lstlisting}

Linux:

\begin{lstlisting}[style=Alabaster, language=bash]
~/3d-projects/
~/Documents/3d-projects/

\end{lstlisting}

Create the directory if it doesn\textquotesingle t exist:

Windows:

\begin{lstlisting}[style=Alabaster, language=powershell]
mkdir C:\Users\YourName\Documents\3d-projects\FirstProject

\end{lstlisting}

Linux:

\begin{lstlisting}[style=Alabaster, language=bash]
mkdir -p ~/3d-projects/FirstProject

\end{lstlisting}

\subparagraph*{Step 2: Navigate Into Your Project Directory}\label{docs__pandoc__latex__src__3dmake_foundation__lessons_3dmake_1__3dmake-setup-guide.md__step-2-navigate-into-your-project-directory}

Windows:

\begin{lstlisting}[style=Alabaster, language=powershell]
cd C:\Users\YourName\Documents\3d-projects\FirstProject

\end{lstlisting}

Linux:

\begin{lstlisting}[style=Alabaster, language=bash]
cd ~/3d-projects/FirstProject

\end{lstlisting}

For screen reader users: Use \texttt{pwd} (print working directory) to confirm you\textquotesingle re in the right folder.

\subparagraph*{Step 3: Initialize a 3dMake Project}\label{docs__pandoc__latex__src__3dmake_foundation__lessons_3dmake_1__3dmake-setup-guide.md__step-3-initialize-a-3dmake-project}

\begin{lstlisting}[style=Alabaster, language=bash]
3dm new

\end{lstlisting}

This creates the project structure:

\begin{lstlisting}[style=Alabaster, language=powershell]
FirstProject/
+------ src/              (stores your .scad files)
+------ build/            (stores generated files: STL, GCODE, etc.)
+------ 3dmake.toml       (project configuration)
+------ README.md         (project documentation)

\end{lstlisting}

The \texttt{src/main.scad} file is created as a starting template.

\subparagraph*{Step 4: Verify Project Creation}\label{docs__pandoc__latex__src__3dmake_foundation__lessons_3dmake_1__3dmake-setup-guide.md__step-4-verify-project-creation}

Windows:

\begin{lstlisting}[style=Alabaster, language=powershell]
Get-ChildItem -Force

\end{lstlisting}

Linux:

\begin{lstlisting}[style=Alabaster, language=bash]
ls -la

\end{lstlisting}

You should see the \texttt{src/}, \texttt{build/}, \texttt{3dmake.toml}, and \texttt{README.md} files listed.

\subsubsection*{Working With Your Project}\label{docs__pandoc__latex__src__3dmake_foundation__lessons_3dmake_1__3dmake-setup-guide.md__working-with-your-project}

\paragraph*{Creating OpenSCAD Models}\label{docs__pandoc__latex__src__3dmake_foundation__lessons_3dmake_1__3dmake-setup-guide.md__creating-openscad-models}

\subparagraph*{Step 1: Create or Edit OpenSCAD Files}\label{docs__pandoc__latex__src__3dmake_foundation__lessons_3dmake_1__3dmake-setup-guide.md__step-1-create-or-edit-openscad-files}

The main model file is \texttt{src/main.scad}. You can edit it directly using 3dMake:

\begin{lstlisting}[style=Alabaster, language=bash]
3dm edit-model

\end{lstlisting}

This opens \texttt{src/main.scad} in your configured text editor.

To edit a different model (if you create additional .scad files):

\begin{lstlisting}[style=Alabaster, language=bash]
3dm edit-model -m mymodel

\end{lstlisting}

This opens \texttt{src/mymodel.scad}.

Manual File Creation:

You can also create \texttt{.scad} files directly in the \texttt{src/} folder using your preferred text editor:

\begin{itemize}
\tightlist
\item
  Visual Studio Code
\item
  Notepad++ (Windows)
\item
  Gedit (Linux)
\item
  Nano or Vim (terminal-based)
\end{itemize}

\subparagraph*{Step 2: Building Your Model}\label{docs__pandoc__latex__src__3dmake_foundation__lessons_3dmake_1__3dmake-setup-guide.md__step-2-building-your-model}

Before exporting, build the model from your OpenSCAD code:

\begin{lstlisting}[style=Alabaster, language=bash]
3dm build

\end{lstlisting}

This converts your OpenSCAD code into a 3D mesh (geometry). The output is \texttt{build/main.stl}.

For a different model:

\begin{lstlisting}[style=Alabaster, language=bash]
3dm build -m mymodel

\end{lstlisting}

This creates \texttt{build/mymodel.stl}.

\subparagraph*{Step 3: Viewing Model Information}\label{docs__pandoc__latex__src__3dmake_foundation__lessons_3dmake_1__3dmake-setup-guide.md__step-3-viewing-model-information}

To see statistics about your model:

\begin{lstlisting}[style=Alabaster, language=bash]
3dm info

\end{lstlisting}

Output includes:

\begin{itemize}
\tightlist
\item
  Bounding box - dimensions (X, Y, Z in millimeters)
\item
  Volume - cubic millimeters of material
\item
  Face count - total triangles in the mesh
\item
  Manifold status - whether model is watertight (printable)
\end{itemize}

Example output:

\begin{lstlisting}[style=Alabaster]
Volume: 1234.56 mm
Bounding Box: 50.0 x 40.0 x 30.0 mm
Faces: 2048
Manifold: Yes 

\end{lstlisting}

\subparagraph*{Step 4: Previewing Your Model}\label{docs__pandoc__latex__src__3dmake_foundation__lessons_3dmake_1__3dmake-setup-guide.md__step-4-previewing-your-model}

3dMake can create flat "tactile previews" of your model (fast-printing 2D silhouettes):

\begin{lstlisting}[style=Alabaster, language=bash]
3dm preview slice

\end{lstlisting}

This generates a preview STL and slices it (ready to print in minutes).

To print the preview directly:

\begin{lstlisting}[style=Alabaster, language=bash]
3dm preview print

\end{lstlisting}

Available preview types:

\begin{itemize}
\tightlist
\item
  \texttt{3sil} - Three silhouettes (front, left, top) - default
\item
  \texttt{frontsil} - Front-facing silhouette only
\item
  \texttt{topsil} - Top-down silhouette only
\end{itemize}

Change preview type:

\begin{lstlisting}[style=Alabaster, language=bash]
3dm preview -v topsil print

\end{lstlisting}

\subparagraph*{Step 5: Slicing and Preparing for Print}\label{docs__pandoc__latex__src__3dmake_foundation__lessons_3dmake_1__3dmake-setup-guide.md__step-5-slicing-and-preparing-for-print}

To slice your model (convert STL to GCODE for your printer):

\begin{lstlisting}[style=Alabaster, language=bash]
3dm build slice

\end{lstlisting}

This creates both:

\begin{itemize}
\tightlist
\item
  \texttt{build/main.stl} - the 3D model
\item
  \texttt{build/main.gcode} - sliced for your printer
\end{itemize}

\subparagraph*{Step 6: Printing Directly from 3dMake}\label{docs__pandoc__latex__src__3dmake_foundation__lessons_3dmake_1__3dmake-setup-guide.md__step-6-printing-directly-from-3dmake}

If your printer is connected via OctoPrint or Bambu Labs (LAN mode):

\begin{lstlisting}[style=Alabaster, language=bash]
3dm build slice print

\end{lstlisting}

This builds, slices, and sends directly to your printer in one command.

Or simply:

\begin{lstlisting}[style=Alabaster, language=bash]
3dm build print

\end{lstlisting}

(The \texttt{print} command automatically includes slicing)

\subsubsection*{Managing Multiple Projects}\label{docs__pandoc__latex__src__3dmake_foundation__lessons_3dmake_1__3dmake-setup-guide.md__managing-multiple-projects}

\paragraph*{Switching Between Projects}\label{docs__pandoc__latex__src__3dmake_foundation__lessons_3dmake_1__3dmake-setup-guide.md__switching-between-projects}

\subparagraph*{Method 1: Navigate via Terminal}\label{docs__pandoc__latex__src__3dmake_foundation__lessons_3dmake_1__3dmake-setup-guide.md__method-1-navigate-via-terminal}

\begin{lstlisting}[style=Alabaster, language=bash]
# Leave current project
cd ..

# Enter a different project
cd ../SecondProject
pwd  # Verify you're in the right place

\end{lstlisting}

For screen reader users, always use \texttt{pwd} after navigating to confirm your location.

\subparagraph*{Method 2: Create a Projects Directory Structure}\label{docs__pandoc__latex__src__3dmake_foundation__lessons_3dmake_1__3dmake-setup-guide.md__method-2-create-a-projects-directory-structure}

Create a central folder for all projects:

\begin{lstlisting}[style=Alabaster]
3d-projects/
+------ FirstProject/
|   +------ src/
|   +------ build/
|   +------ 3dmake.toml
+------ SecondProject/
|   +------ src/
|   +------ build/
|   +------ 3dmake.toml
+------ README.md (project index)

\end{lstlisting}

\subparagraph*{Method 3: Use a Project Index}\label{docs__pandoc__latex__src__3dmake_foundation__lessons_3dmake_1__3dmake-setup-guide.md__method-3-use-a-project-index}

Create a \texttt{README.md} in your \texttt{3d-projects/} folder to track all projects:

\begin{lstlisting}[style=Alabaster]
# My 3D Projects

## Project List

1. FirstProject - My initial test models
   - Location: `./FirstProject/src/`
   - Status: In progress
   - Latest model: `main.scad`

2. SecondProject - Parametric keychain designs
   - Location: `./SecondProject/src/`
   - Status: Complete
   - Latest model: `keychain.scad`

3. ThirdProject - Functional brackets for printing
   - Location: `./ThirdProject/src/`
   - Status: In progress
   - Latest model: `bracketassembly.scad`

\end{lstlisting}

\paragraph*{Building from All Projects}\label{docs__pandoc__latex__src__3dmake_foundation__lessons_3dmake_1__3dmake-setup-guide.md__building-from-all-projects}

To build models from all projects:

Windows:

\begin{lstlisting}[style=Alabaster, language=powershell]
Get-ChildItem -Directory | ForEach-Object {
    cd $.FullName
    3dm build
    cd ..
}

\end{lstlisting}

Linux/Bash:

\begin{lstlisting}[style=Alabaster, language=bash]
for project in */; do
    cd "$project"
    3dm build
    cd ..
done

\end{lstlisting}

\subsubsection*{Workflow Best Practices}\label{docs__pandoc__latex__src__3dmake_foundation__lessons_3dmake_1__3dmake-setup-guide.md__workflow-best-practices}

\paragraph*{File Naming Conventions}\label{docs__pandoc__latex__src__3dmake_foundation__lessons_3dmake_1__3dmake-setup-guide.md__file-naming-conventions}

Use clear, descriptive names for your \texttt{.scad} files:

Good:

\begin{itemize}
\tightlist
\item
  \texttt{cube5cm.scad}
\item
  \texttt{parametricboxv2.scad}
\item
  \texttt{bracketformotor.scad}
\item
  \texttt{main.scad} (default project model)
\end{itemize}

Avoid:

\begin{itemize}
\tightlist
\item
  \texttt{test.scad}
\item
  \texttt{model1.scad}
\item
  \texttt{finalfinalFINAL.scad}
\end{itemize}

\paragraph*{Understanding TOML Configuration Files}\label{docs__pandoc__latex__src__3dmake_foundation__lessons_3dmake_1__3dmake-setup-guide.md__understanding-toml-configuration-files}

What is TOML?

TOML (Tom\textquotesingle s Obvious, Minimal Language) is a human-readable configuration file format. It\textquotesingle s designed to be simple and clear, making it easy to read and edit in any text editor.

Basic TOML Formatting Rules:

\begin{enumerate}
\item
  Key-value pairs - Each setting has a name and value separated by an equals sign:

  \begin{lstlisting}[style=Alabaster]
  projectname = "My Project"

  \end{lstlisting}
\item
  Strings use quotes - Text values must be wrapped in double quotes:

  \begin{lstlisting}[style=Alabaster]
  editor = "code"
  printerprofile = "prusaMK4"

  \end{lstlisting}
\item
  Numbers don\textquotesingle t need quotes - Numeric values stand alone:

  \begin{lstlisting}[style=Alabaster]
  scale = 1.05
  copies = 3

  \end{lstlisting}
\item
  Boolean values are true or false - Lowercase, no quotes:

  \begin{lstlisting}[style=Alabaster]
  autostartprints = true
  editinbackground = false

  \end{lstlisting}
\item
  Arrays use square brackets - Lists of values separated by commas:

  \begin{lstlisting}[style=Alabaster]
  overlays = ["supports", "PETG"]
  libraries = ["bosl", "braille-chars"]

  \end{lstlisting}
\item
  One setting per line - Each configuration on its own line
\item
  Comments start with \# - Use for notes (not processed):

  \begin{lstlisting}[style=Alabaster]
  # This is my default printer
  printerprofile = "prusaMK4"

  \end{lstlisting}
\end{enumerate}

How 3dMake Uses TOML:

3dMake has two TOML configuration files:

\begin{itemize}
\tightlist
\item
  Global config (\texttt{defaults.toml}) - Settings for all your projects (run \texttt{3dm\ edit-global-config} to edit)
\item
  Project config (\texttt{3dmake.toml}) - Settings specific to one project (located in your project root)
\end{itemize}

Project settings override global settings. For example, if your global config says \texttt{printerprofile\ =\ "prusaMK4"} but your project\textquotesingle s \texttt{3dmake.toml} says \texttt{printerprofile\ =\ "bambu"}, the project setting wins.

\paragraph*{Project Configuration (3dmake.toml)}\label{docs__pandoc__latex__src__3dmake_foundation__lessons_3dmake_1__3dmake-setup-guide.md__project-configuration-3dmaketoml}

The \texttt{3dmake.toml} file in your project root contains project-specific settings:

\begin{lstlisting}[style=Alabaster]
projectname = "My Project"
modelname = "main"
printerprofile = "prusaMK4"
overlays = []
editor = "code"

\end{lstlisting}

Edit your project configuration:

\begin{lstlisting}[style=Alabaster, language=bash]
3dm edit-global-config

\end{lstlisting}

This opens your configuration file in your text editor.

\paragraph*{Version Control (Optional)}\label{docs__pandoc__latex__src__3dmake_foundation__lessons_3dmake_1__3dmake-setup-guide.md__version-control-optional}

If you\textquotesingle re using Git, add a \texttt{.gitignore} to avoid committing large build files:

Create \texttt{.gitignore} in your project root:

\begin{lstlisting}[style=Alabaster]
build/
*.stl
*.gcode
*.png
.DSStore
pycache/

\end{lstlisting}

Then initialize Git:

\begin{lstlisting}[style=Alabaster, language=bash]
git init
git add .
git commit -m "Initial project setup"

\end{lstlisting}

\paragraph*{Backup Strategy}\label{docs__pandoc__latex__src__3dmake_foundation__lessons_3dmake_1__3dmake-setup-guide.md__backup-strategy}

Regularly backup your \texttt{src/} folder:

Windows:

\begin{lstlisting}[style=Alabaster, language=powershell]
Copy-Item -Path "src" -Destination "backups/src$(Get-Date -Format 'yyyy-MM-dd')" -Recurse

\end{lstlisting}

Linux:

\begin{lstlisting}[style=Alabaster, language=bash]
cp -r src backups/src$(date +%Y-%m-%d)

\end{lstlisting}

\paragraph*{Documentation}\label{docs__pandoc__latex__src__3dmake_foundation__lessons_3dmake_1__3dmake-setup-guide.md__documentation}

Keep a \texttt{src/README.md} describing each model:

\begin{lstlisting}[style=Alabaster]
# Models in This Project

## main.scad
- Purpose: Primary design for this project
- Parameters: width, height, depth
- Last modified: 2026-02-20
- Notes: Standard model for printing

## alternate.scad
- Purpose: Alternative design variant
- Parameters: width, height, depth, wallthickness
- Last modified: 2026-02-15
- Notes: Experimental version with snap-fit features

\end{lstlisting}

\paragraph*{Configuring Your Text Editor}\label{docs__pandoc__latex__src__3dmake_foundation__lessons_3dmake_1__3dmake-setup-guide.md__configuring-your-text-editor}

By default, 3dMake uses:

\begin{itemize}
\tightlist
\item
  Windows: Notepad
\item
  Linux: Nano (or your EDITOR environment variable)
\end{itemize}

To use a different editor, edit your global configuration:

\begin{lstlisting}[style=Alabaster, language=bash]
3dm edit-global-config

\end{lstlisting}

Add or modify the \texttt{editor} line. Examples:

Windows (Visual Studio Code):

\begin{lstlisting}[style=Alabaster]
editor = "code"

\end{lstlisting}

Windows (Notepad++):

\begin{lstlisting}[style=Alabaster]
editor = '''C:\Program Files (x86)\Notepad++\notepad++.exe'''

\end{lstlisting}

Linux (Visual Studio Code):

\begin{lstlisting}[style=Alabaster]
editor = "code"

\end{lstlisting}

Linux (Gedit):

\begin{lstlisting}[style=Alabaster]
editor = "gedit"

\end{lstlisting}

\subsubsection*{Troubleshooting}\label{docs__pandoc__latex__src__3dmake_foundation__lessons_3dmake_1__3dmake-setup-guide.md__troubleshooting}

\paragraph*{Common Issues}\label{docs__pandoc__latex__src__3dmake_foundation__lessons_3dmake_1__3dmake-setup-guide.md__common-issues}

\subparagraph*{"3dm command not found"}\label{docs__pandoc__latex__src__3dmake_foundation__lessons_3dmake_1__3dmake-setup-guide.md__3dm-command-not-found}

Cause: 3dMake isn\textquotesingle t in your PATH or you haven\textquotesingle t completed setup.

Solution:

\begin{enumerate}
\tightlist
\item
  Verify you extracted 3dMake and completed \texttt{3dm\ setup}
\item
  Restart your terminal completely (close and reopen)
\item
  Check that you\textquotesingle re not inside the 3dMake directory when running commands
\item
  On Linux, ensure you\textquotesingle re using \texttt{./3dm} if 3dMake isn\textquotesingle t in your PATH yet
\end{enumerate}

\subparagraph*{"No such file or directory: src/main.scad"}\label{docs__pandoc__latex__src__3dmake_foundation__lessons_3dmake_1__3dmake-setup-guide.md__no-such-file-or-directory-srcmainscad}

Cause: You\textquotesingle re not in a valid 3dMake project directory.

Solution:

Verify you\textquotesingle re in the project directory:

\begin{lstlisting}[style=Alabaster, language=bash]
pwd
ls -la

\end{lstlisting}

You should see \texttt{src/}, \texttt{build/}, and \texttt{3dmake.toml} files. If not, run:

\begin{lstlisting}[style=Alabaster, language=bash]
3dm new

\end{lstlisting}

\subparagraph*{"OpenSCAD error: syntax error at line 5"}\label{docs__pandoc__latex__src__3dmake_foundation__lessons_3dmake_1__3dmake-setup-guide.md__openscad-error-syntax-error-at-line-5}

Cause: Your \texttt{.scad} file has incorrect OpenSCAD syntax.

Solution:

\begin{enumerate}
\tightlist
\item
  Open your model file: \texttt{3dm\ edit-model}
\item
  Check the line number mentioned in the error
\item
  Verify correct OpenSCAD syntax (matching parentheses, semicolons, etc.)
\item
  Save and try building again: \texttt{3dm\ build}
\end{enumerate}

\subparagraph*{"Model won\textquotesingle t render or build"}\label{docs__pandoc__latex__src__3dmake_foundation__lessons_3dmake_1__3dmake-setup-guide.md__model-wont-render-or-build}

Checklist:

\begin{itemize}
\tightlist
\item[$\square$]
  File is saved with \texttt{.scad} extension
\item[$\square$]
  File is in the \texttt{src/} folder
\item[$\square$]
  File has valid OpenSCAD syntax
\item[$\square$]
  Build output has a specific error message (check line number)
\end{itemize}

Debug mode:

\begin{lstlisting}[style=Alabaster, language=bash]
3dm build --debug

\end{lstlisting}

This provides more detailed error messages.

\subparagraph*{Screen reader isn\textquotesingle t reading 3dMake output clearly}\label{docs__pandoc__latex__src__3dmake_foundation__lessons_3dmake_1__3dmake-setup-guide.md__screen-reader-isnt-reading-3dmake-output-clearly}

Solution: Use the \texttt{-\/-debug} flag for more verbose output:

\begin{lstlisting}[style=Alabaster, language=bash]
3dm build --debug

\end{lstlisting}

This logs each step, making it easier for screen readers to follow.

\subparagraph*{"Permission denied" error (Linux)}\label{docs__pandoc__latex__src__3dmake_foundation__lessons_3dmake_1__3dmake-setup-guide.md__permission-denied-error-linux}

Cause: 3dMake executable doesn\textquotesingle t have run permissions.

Solution:

\begin{lstlisting}[style=Alabaster, language=bash]
chmod +x 3dm
./3dm setup

\end{lstlisting}

\subparagraph*{Cannot connect to printer}\label{docs__pandoc__latex__src__3dmake_foundation__lessons_3dmake_1__3dmake-setup-guide.md__cannot-connect-to-printer}

Cause: OctoPrint or Bambu printer settings not configured correctly.

Solution:

\begin{enumerate}
\tightlist
\item
  Verify printer is running and connected to network
\item
  Edit configuration: \texttt{3dm\ edit-global-config}
\item
  Check these settings:

  \begin{itemize}
  \tightlist
  \item
    \texttt{octoprinthost} - correct IP/URL
  \item
    \texttt{octoprintkey} - valid API key
  \item
    \texttt{printmode} - set to "octoprint" or "bambulan"
  \end{itemize}
\item
  Test connection: \texttt{3dm\ build\ print} (without actually printing first)
\end{enumerate}

\subsubsection*{Quick Reference}\label{docs__pandoc__latex__src__3dmake_foundation__lessons_3dmake_1__3dmake-setup-guide.md__quick-reference}

\paragraph*{Essential Commands}\label{docs__pandoc__latex__src__3dmake_foundation__lessons_3dmake_1__3dmake-setup-guide.md__essential-commands}

{\def\LTcaptype{none} % do not increment counter
\begin{longtable}[]{@{}
  >{\raggedright\arraybackslash}p{(\linewidth - 2\tabcolsep) * \real{0.3714}}
  >{\raggedright\arraybackslash}p{(\linewidth - 2\tabcolsep) * \real{0.6286}}@{}}
\toprule\noalign{}
\begin{minipage}[b]{\linewidth}\raggedright
Command
\end{minipage} & \begin{minipage}[b]{\linewidth}\raggedright
Purpose
\end{minipage} \\
\midrule\noalign{}
\endhead
\bottomrule\noalign{}
\endlastfoot
\texttt{3dm\ -\/-version} & Show installed version \\
\texttt{3dm\ help} & Display all available commands \\
\texttt{3dm\ setup} & Initial setup (run in extracted directory) \\
\texttt{3dm\ new} & Initialize a new project \\
\texttt{3dm\ build} & Build OpenSCAD model to STL \\
\texttt{3dm\ build\ slice} & Build and slice to GCODE \\
\texttt{3dm\ build\ print} & Build, slice, and send to printer \\
\texttt{3dm\ edit-model} & Open main.scad in text editor \\
\texttt{3dm\ edit-model\ -m\ name} & Open a specific model file \\
\texttt{3dm\ info} & Show model statistics \\
\texttt{3dm\ preview\ print} & Create and print a tactile preview \\
\texttt{3dm\ list-profiles} & Show available printer profiles \\
\texttt{3dm\ list-overlays} & Show available slicer overlays \\
\texttt{3dm\ edit-global-config} & Edit global settings \\
\texttt{3dm\ orient\ print} & Auto-orient and print model \\
\end{longtable}
}

\paragraph*{Directory Structure Reference}\label{docs__pandoc__latex__src__3dmake_foundation__lessons_3dmake_1__3dmake-setup-guide.md__directory-structure-reference}

\begin{lstlisting}[style=Alabaster]
YourProject/
+------ src/                 <- Place .scad files here
+------ build/               <- Built STL, GCODE auto-save here
+------ 3dmake.toml          <- Project settings
+------ README.md            <- Project documentation

\end{lstlisting}

\paragraph*{Configuration Options (3dmake.toml)}\label{docs__pandoc__latex__src__3dmake_foundation__lessons_3dmake_1__3dmake-setup-guide.md__configuration-options-3dmaketoml}

Common settings you can modify:

\begin{lstlisting}[style=Alabaster]
projectname = "My Project"
modelname = "main"              # Default model to build
printerprofile = "prusaMK4"    # Your printer
overlays = ["supports"]          # Default slicing overlays
editor = "code"                  # Text editor to use

\end{lstlisting}

For full configuration options, see the GitHub repository: \url{https://github.com/tdeck/3dmake}

\subsubsection*{Next Steps}\label{docs__pandoc__latex__src__3dmake_foundation__lessons_3dmake_1__3dmake-setup-guide.md__next-steps}

Once you\textquotesingle re comfortable with basic projects:

\begin{enumerate}
\tightlist
\item
  Learn parametric design - Make reusable models with variables
\item
  Explore modules - Organize code into reusable functions
\item
  Add libraries - Use pre-built OpenSCAD libraries (BOSL, etc.)
\item
  Optimize workflows - Chain commands (\texttt{3dm\ build\ slice\ print})
\item
  Automate models - Use loops and conditionals in OpenSCAD
\item
  Collaborate - Use Git to share projects with others
\end{enumerate}

For advanced topics, see:

\begin{itemize}
\tightlist
\item
  \href{https://en.wikibooks.org/wiki/OpenSCADUserManual}{OpenSCAD Manual}
\item
  \href{https://github.com/tdeck/3dmake}{3dMake GitHub Documentation}
\item
  \href{https://github.com/revarbat/BOSL/wiki}{BOSL Library Documentation}
\end{itemize}

\subsubsection*{Sources}\label{docs__pandoc__latex__src__3dmake_foundation__lessons_3dmake_1__3dmake-setup-guide.md__sources}

3dMake GitHub Repository. (2026). \emph{3dMake - Non-visual 3D design and printing}. Retrieved from \url{https://github.com/tdeck/3dmake}

3dMake Documentation. (2026). \emph{Terminal quick start guide}. Retrieved from \url{https://github.com/tdeck/3dmake/blob/main/docs/terminalquickstart.md}

OpenSCAD Community. (2025). \emph{OpenSCAD user manual}. Retrieved from \url{https://en.wikibooks.org/wiki/OpenSCADUserManual}

Revarbat (Ed.). (2024). \emph{BOSL - Belfry OpenSCAD Library}. Retrieved from \url{https://github.com/revarbat/BOSL/wiki}

\subsection{VSCode Setup Guide}\label{docs__pandoc__latex__src__3dmake_foundation__lessons_3dmake_1__vscode-setup-guide.md__3dmake_foundation_lessons_3dmake_1-vscode-setup-guide}

\subsubsection*{For use with NVDA or JAWS on Windows}\label{docs__pandoc__latex__src__3dmake_foundation__lessons_3dmake_1__vscode-setup-guide.md__for-use-with-nvda-or-jaws-on-windows}

This guide walks you through setting up Visual Studio Code (VSCode) as an accessible code editor for writing OpenSCAD files, with a task runner that automatically previews your \texttt{.scad} code in OpenSCAD whenever you save.

\subsubsection*{Why VSCode Instead of the OpenSCAD Editor?}\label{docs__pandoc__latex__src__3dmake_foundation__lessons_3dmake_1__vscode-setup-guide.md__why-vscode-instead-of-the-openscad-editor}

The built-in OpenSCAD editor has inconsistent behavior with screen readers - focus can jump unexpectedly, and the editor sometimes stops being read after certain actions. VSCode is a mainstream code editor with strong, well-tested accessibility support for both NVDA and JAWS.

You write your code in VSCode. OpenSCAD runs in the background (or in a separate window) to render the preview. You never have to interact with the OpenSCAD editor itself.

\subsubsection*{Install Required Software}\label{docs__pandoc__latex__src__3dmake_foundation__lessons_3dmake_1__vscode-setup-guide.md__install-required-software}

\paragraph*{1.1 Install VSCode}\label{docs__pandoc__latex__src__3dmake_foundation__lessons_3dmake_1__vscode-setup-guide.md__11-install-vscode}

Download from: \url{https://code.visualstudio.com/}

During installation:

\begin{itemize}
\tightlist
\item
  Check "Add to PATH" - this is important for running VSCode from PowerShell
\item
  Check "Register Code as an editor for supported file types"
\end{itemize}

\paragraph*{1.2 Install OpenSCAD}\label{docs__pandoc__latex__src__3dmake_foundation__lessons_3dmake_1__vscode-setup-guide.md__12-install-openscad}

Download from: \url{https://openscad.org/downloads.html}

Use the installer version (not the portable version). After installing, confirm OpenSCAD is in your PATH by opening PowerShell and typing:

\begin{lstlisting}[style=Alabaster, language=powershell]
openscad --version

\end{lstlisting}

You should hear a version number. If you get an error, see the PATH setup section in the PowerShell Foundation guide.

\paragraph*{1.3 Install the OpenSCAD VSCode Extension (Optional but Recommended)}\label{docs__pandoc__latex__src__3dmake_foundation__lessons_3dmake_1__vscode-setup-guide.md__13-install-the-openscad-vscode-extension-optional-but-recommended}

\begin{enumerate}
\tightlist
\item
  Open VSCode
\item
  Press \texttt{Ctrl\ +\ Shift\ +\ X} to open the Extensions panel
\item
  Type \texttt{openscad} in the search box
\item
  Install "OpenSCAD Language Support" - this adds syntax highlighting and keyword completion for \texttt{.scad} files
\end{enumerate}

\subsubsection*{Configure the Task Runner}\label{docs__pandoc__latex__src__3dmake_foundation__lessons_3dmake_1__vscode-setup-guide.md__configure-the-task-runner}

VSCode uses a file called \texttt{tasks.json} to define custom commands you can run from the keyboard. We\textquotesingle ll set up a task that opens your current \texttt{.scad} file in OpenSCAD for preview whenever you press a key.

\paragraph*{2.1 Open or Create Your Workspace Folder}\label{docs__pandoc__latex__src__3dmake_foundation__lessons_3dmake_1__vscode-setup-guide.md__21-open-or-create-your-workspace-folder}

All your \texttt{.scad} files for a project should be in one folder. Open that folder in VSCode:

\begin{lstlisting}[style=Alabaster, language=powershell]
cd ~/Documents/OpenSCAD_Projects
code .

\end{lstlisting}

The \texttt{.} tells VSCode to open the current folder as a workspace.

\paragraph*{2.2 Create the Tasks File}\label{docs__pandoc__latex__src__3dmake_foundation__lessons_3dmake_1__vscode-setup-guide.md__22-create-the-tasks-file}

\begin{enumerate}
\tightlist
\item
  In VSCode, press \texttt{Ctrl\ +\ Shift\ +\ P} to open the Command Palette
\item
  Type \texttt{Tasks:\ Configure\ Task} and press Enter
\item
  Select "Create tasks.json file from template"
\item
  Select "Others"
\end{enumerate}

This creates a \texttt{.vscode/tasks.json} file. Replace its entire contents with the following:

\begin{lstlisting}[style=Alabaster, language=json]
{
  "version": "2.0.0",
  "tasks": [
    {
      "label": "Preview in OpenSCAD",
      "type": "shell",
      "command": "openscad",
      "args": ["${file}"],
      "group": {
        "kind": "build",
        "isDefault": true
      },
      "presentation": {
        "reveal": "silent",
        "panel": "shared"
      },
      "problemMatcher": []
    },
    {
      "label": "Export STL",
      "type": "shell",
      "command": "openscad",
      "args": [
        "-o",
        "${fileDirname}/${fileBasenameNoExtension}.stl",
        "${file}"
      ],
      "group": "build",
      "presentation": {
        "reveal": "always",
        "panel": "shared"
      },
      "problemMatcher": []
    }
  ]
}

\end{lstlisting}

Save the file with \texttt{Ctrl\ +\ S}.

\paragraph*{2.3 Run the Preview Task}\label{docs__pandoc__latex__src__3dmake_foundation__lessons_3dmake_1__vscode-setup-guide.md__23-run-the-preview-task}

With any \texttt{.scad} file open and focused:

\begin{itemize}
\tightlist
\item
  Press \texttt{Ctrl\ +\ Shift\ +\ B} to run the default build task (Preview in OpenSCAD)
\item
  OpenSCAD will open and display your model
\item
  Switch back to VSCode with \texttt{Alt\ +\ Tab} to keep editing
\end{itemize}

To export an STL:

\begin{enumerate}
\tightlist
\item
  Press \texttt{Ctrl\ +\ Shift\ +\ P}
\item
  Type \texttt{Tasks:\ Run\ Task}
\item
  Select "Export STL"
\item
  The \texttt{.stl} file will be saved in the same folder as your \texttt{.scad} file
\end{enumerate}

\subsubsection*{NVDA Settings for VSCode}\label{docs__pandoc__latex__src__3dmake_foundation__lessons_3dmake_1__vscode-setup-guide.md__nvda-settings-for-vscode}

\paragraph*{3.1 Recommended NVDA Settings}\label{docs__pandoc__latex__src__3dmake_foundation__lessons_3dmake_1__vscode-setup-guide.md__31-recommended-nvda-settings}

Open NVDA Menu (\texttt{NVDA\ +\ N}) -\textgreater{} Preferences -\textgreater{} Settings:

Speech category:

\begin{itemize}
\tightlist
\item
  Symbol level: Most (so you hear brackets, semicolons, and other syntax characters)
\end{itemize}

Browse Mode category:

\begin{itemize}
\tightlist
\item
  Uncheck "Use browse mode on page load in web content" - not needed for VSCode
\end{itemize}

\paragraph*{3.2 Useful NVDA + VSCode Keyboard Shortcuts}\label{docs__pandoc__latex__src__3dmake_foundation__lessons_3dmake_1__vscode-setup-guide.md__32-useful-nvda--vscode-keyboard-shortcuts}

{\def\LTcaptype{none} % do not increment counter
\begin{longtable}[]{@{}ll@{}}
\toprule\noalign{}
Action & Keys \\
\midrule\noalign{}
\endhead
\bottomrule\noalign{}
\endlastfoot
Read current line & \texttt{NVDA\ +\ Up\ Arrow} \\
Read from cursor & \texttt{NVDA\ +\ Down\ Arrow} \\
Spell current word & \texttt{NVDA\ +\ Numpad\ 2} (twice quickly) \\
Move to next/previous line & \texttt{Up\ /\ Down\ Arrow} \\
Move by word & \texttt{Ctrl\ +\ Left\ /\ Right\ Arrow} \\
Move to start/end of line & \texttt{Home\ /\ End} \\
Move to start/end of file & \texttt{Ctrl\ +\ Home\ /\ End} \\
Select all & \texttt{Ctrl\ +\ A} \\
Toggle line comment & \texttt{Ctrl\ +\ /} \\
Go to line number & \texttt{Ctrl\ +\ G}, type number, Enter \\
Find in file & \texttt{Ctrl\ +\ F} \\
Open file in workspace & \texttt{Ctrl\ +\ P}, type filename \\
\end{longtable}
}

\paragraph*{3.3 Punctuation Level}\label{docs__pandoc__latex__src__3dmake_foundation__lessons_3dmake_1__vscode-setup-guide.md__33-punctuation-level}

When reading code, you need to hear all punctuation - semicolons, brackets, parentheses, and commas are all part of OpenSCAD syntax.

In NVDA: \texttt{NVDA\ +\ N} -\textgreater{} Preferences -\textgreater{} Settings -\textgreater{} Speech -\textgreater{} Symbol level: Most

You can also toggle punctuation level on the fly: \texttt{NVDA\ +\ P} cycles through None, Some, Most, All.

\subsubsection*{JAWS Settings for VSCode}\label{docs__pandoc__latex__src__3dmake_foundation__lessons_3dmake_1__vscode-setup-guide.md__jaws-settings-for-vscode}

\paragraph*{4.1 Virtual Cursor}\label{docs__pandoc__latex__src__3dmake_foundation__lessons_3dmake_1__vscode-setup-guide.md__41-virtual-cursor}

JAWS may try to activate Virtual/Browse mode in VSCode. If VSCode stops responding to arrow keys for navigation and starts reading the page as HTML, press \texttt{JAWS\ Key\ +\ Z} to toggle Virtual mode off. You want to be in Application mode (not Virtual mode) when using VSCode.

\paragraph*{4.2 Recommended JAWS Settings}\label{docs__pandoc__latex__src__3dmake_foundation__lessons_3dmake_1__vscode-setup-guide.md__42-recommended-jaws-settings}

\begin{itemize}
\tightlist
\item
  Punctuation level: All - \texttt{JAWS\ Key\ +\ Shift\ +\ 2} cycles through levels. Set to All for code editing.
\item
  Reading rate: Adjust with \texttt{Alt\ +\ Ctrl\ +\ Page\ Up\ /\ Page\ Down}
\item
  Spell current line: \texttt{JAWS\ Key\ +\ Up\ Arrow} twice quickly
\end{itemize}

\paragraph*{4.3 Useful JAWS + VSCode Keyboard Shortcuts}\label{docs__pandoc__latex__src__3dmake_foundation__lessons_3dmake_1__vscode-setup-guide.md__43-useful-jaws--vscode-keyboard-shortcuts}

{\def\LTcaptype{none} % do not increment counter
\begin{longtable}[]{@{}ll@{}}
\toprule\noalign{}
Action & Keys \\
\midrule\noalign{}
\endhead
\bottomrule\noalign{}
\endlastfoot
Read current line & \texttt{JAWS\ Key\ +\ Up\ Arrow} \\
Read from cursor & \texttt{JAWS\ Key\ +\ Down\ Arrow} \\
Read current character & \texttt{JAWS\ Key\ +\ Numpad\ 5} \\
Move by word & \texttt{Ctrl\ +\ Left\ /\ Right\ Arrow} \\
Move to start/end of line & \texttt{Home\ /\ End} \\
Toggle comment & \texttt{Ctrl\ +\ /} \\
Go to line & \texttt{Ctrl\ +\ G} \\
Open command palette & \texttt{Ctrl\ +\ Shift\ +\ P} \\
\end{longtable}
}

\subsubsection*{Notepad++ as an Alternative}\label{docs__pandoc__latex__src__3dmake_foundation__lessons_3dmake_1__vscode-setup-guide.md__notepad-as-an-alternative}

If VSCode is too complex to set up, Notepad++ is a simpler screen-reader-friendly option for editing \texttt{.scad} files.

\paragraph*{Setup}\label{docs__pandoc__latex__src__3dmake_foundation__lessons_3dmake_1__vscode-setup-guide.md__setup}

\begin{enumerate}
\tightlist
\item
  Download from: \url{https://notepad-plus-plus.org/}
\item
  Install the OpenSCAD syntax highlighting plugin:

  \begin{itemize}
  \tightlist
  \item
    Go to Plugins -\textgreater{} Plugin Admin
  \item
    Search for "OpenSCAD" - install if available
  \item
    Alternatively, download a UDL (User Defined Language) file from the OpenSCAD community and import it via Language -\textgreater{} User Defined Language -\textgreater{} Import
  \end{itemize}
\end{enumerate}

\paragraph*{Running OpenSCAD from Notepad++}\label{docs__pandoc__latex__src__3dmake_foundation__lessons_3dmake_1__vscode-setup-guide.md__running-openscad-from-notepad}

Use the Run menu (\texttt{F5}) to configure a custom command:

\begin{lstlisting}[style=Alabaster, language=openscad]
openscad "$(FULL_CURRENT_PATH)"

\end{lstlisting}

Name it "Preview in OpenSCAD" and assign it a shortcut key (e.g., \texttt{Ctrl\ +\ F5}).

\paragraph*{NVDA + Notepad++ Tips}\label{docs__pandoc__latex__src__3dmake_foundation__lessons_3dmake_1__vscode-setup-guide.md__nvda--notepad-tips}

\begin{itemize}
\tightlist
\item
  Punctuation level: set to Most or All
\item
  Use \texttt{Ctrl\ +\ G} to go to a specific line
\item
  Use \texttt{Ctrl\ +\ F} to find text
\item
  The status bar at the bottom of Notepad++ announces line and column numbers - useful for finding where errors are
\end{itemize}

\subsubsection*{Workflow Summary}\label{docs__pandoc__latex__src__3dmake_foundation__lessons_3dmake_1__vscode-setup-guide.md__workflow-summary}

Here is the complete workflow from writing code to printing:

\begin{enumerate}
\item
  Open VSCode (or Notepad++)
  code \textasciitilde{}/Documents/OpenSCAD\_Projects
\item
  Open or create a .scad file Ctrl + P -\textgreater{} type filename
\item
  Write your OpenSCAD code
\item
  Preview in OpenSCAD Ctrl + Shift + B  (VSCode)
  Ctrl + F5         (Notepad++)
\item
  If the shape looks right, export STL Run "Export STL" task in VSCode
  OR: In OpenSCAD, press F6 then File \textgreater{} Export \textgreater{} Export as STL
\item
  Open PrusaSlicer, import the STL, slice, export G-code
\item
  Load G-code onto SD card or USB and print
\end{enumerate}

\subsubsection*{Troubleshooting}\label{docs__pandoc__latex__src__3dmake_foundation__lessons_3dmake_1__vscode-setup-guide.md__troubleshooting}

\paragraph*{OpenSCAD doesn\textquotesingle t open when I run the task}\label{docs__pandoc__latex__src__3dmake_foundation__lessons_3dmake_1__vscode-setup-guide.md__openscad-doesnt-open-when-i-run-the-task}

\begin{itemize}
\tightlist
\item
  Confirm OpenSCAD is installed and in your PATH: \texttt{openscad\ -\/-version} in PowerShell
\item
  If not found, add the OpenSCAD install folder to your PATH (see PowerShell Foundation guide)
\end{itemize}

\paragraph*{VSCode isn\textquotesingle t being read by my screen reader}\label{docs__pandoc__latex__src__3dmake_foundation__lessons_3dmake_1__vscode-setup-guide.md__vscode-isnt-being-read-by-my-screen-reader}

\begin{itemize}
\tightlist
\item
  For JAWS: Press \texttt{JAWS\ Key\ +\ Z} to toggle out of Virtual mode
\item
  For NVDA: Make sure you are focused inside the editor panel, not a sidebar
\item
  Try clicking directly in the editor area with the mouse once to confirm focus
\end{itemize}

\paragraph*{I hear "unlabeled" for some VSCode elements}\label{docs__pandoc__latex__src__3dmake_foundation__lessons_3dmake_1__vscode-setup-guide.md__i-hear-unlabeled-for-some-vscode-elements}

\begin{itemize}
\tightlist
\item
  This is a known VSCode accessibility limitation for some UI panels
\item
  Use keyboard shortcuts rather than trying to navigate by element - the editor itself reads well
\end{itemize}

\paragraph*{My .scad file has an error but I can\textquotesingle t find it}\label{docs__pandoc__latex__src__3dmake_foundation__lessons_3dmake_1__vscode-setup-guide.md__my-scad-file-has-an-error-but-i-cant-find-it}

\begin{itemize}
\tightlist
\item
  OpenSCAD will display an error in its console - use \texttt{Alt\ +\ Tab} to switch to OpenSCAD and arrow through the console to read the error
\item
  Errors always include a line number - use \texttt{Ctrl\ +\ G} in VSCode to jump to that line
\end{itemize}

\subsubsection*{References}\label{docs__pandoc__latex__src__3dmake_foundation__lessons_3dmake_1__vscode-setup-guide.md__references}

Microsoft. (2024). \emph{Visual Studio Code accessibility}. \url{https://code.visualstudio.com/docs/editor/accessibility}

NV Access. (2024). \emph{NVDA user guide}. \url{https://www.nvaccess.org/files/nvda/documentation/userGuide.html}

OpenSCAD. (n.d.). \emph{OpenSCAD documentation}. \url{https://openscad.org/documentation.html}

Other Screen Readers

Dolphin SuperNova (commercial) and Windows Narrator (built-in) are also supported; the workflows and recommendations in this document apply to them. See \url{https://yourdolphin.com/supernova/} and \url{https://support.microsoft.com/narrator} for vendor documentation.

\subsection{Navigating This Curriculum - mdBook Guide}\label{docs__pandoc__latex__src__3dmake_foundation__lessons_3dmake_1__mdbook-navigation-guide.md__3dmake_foundation_lessons_3dmake_1-mdbook-navigation-guide}

\emph{This curriculum is published as a web book using mdBook. This page explains how to find what you need, navigate between chapters, and use the book with a screen reader.}

\subsubsection*{What Is mdBook?}\label{docs__pandoc__latex__src__3dmake_foundation__lessons_3dmake_1__mdbook-navigation-guide.md__what-is-mdbook}

mdBook is a tool that turns a collection of Markdown files into a navigable web book - similar to an online textbook. Each lesson or document is its own page (chapter), and they are organized into sections visible in the sidebar table of contents.

You can access the book from any web browser on any device.

\subsubsection*{Basic Navigation}\label{docs__pandoc__latex__src__3dmake_foundation__lessons_3dmake_1__mdbook-navigation-guide.md__basic-navigation}

\paragraph*{Sidebar Table of Contents}\label{docs__pandoc__latex__src__3dmake_foundation__lessons_3dmake_1__mdbook-navigation-guide.md__sidebar-table-of-contents}

The left side of the page contains a table of contents showing all chapters organized by unit and lesson. You can click or tap any chapter title to jump directly to it.

On a small screen (phone or tablet), the sidebar may be hidden. Look for a hamburger menu icon (three horizontal lines) to show it.

\paragraph*{Arrow Navigation}\label{docs__pandoc__latex__src__3dmake_foundation__lessons_3dmake_1__mdbook-navigation-guide.md__arrow-navigation}

At the bottom of every page there are Previous and Next links that take you through the chapters in order. You can also use keyboard arrow keys:

\begin{itemize}
\tightlist
\item
  \texttt{Left\ Arrow} - go to the previous chapter
\item
  \texttt{Right\ Arrow} - go to the next chapter
\end{itemize}

\paragraph*{Search}\label{docs__pandoc__latex__src__3dmake_foundation__lessons_3dmake_1__mdbook-navigation-guide.md__search}

The mdBook search feature indexes all content across all chapters.

\begin{itemize}
\tightlist
\item
  Click the search icon (magnifying glass) in the top bar, or press \texttt{S}
\item
  Type your search term
\item
  Results appear as a dropdown list with the chapter name and a snippet of context
\item
  Click any result to jump to that chapter; the matching term will be highlighted
\item
  Press \texttt{Escape} to close the search panel
\end{itemize}

\paragraph*{Other Keyboard Shortcuts}\label{docs__pandoc__latex__src__3dmake_foundation__lessons_3dmake_1__mdbook-navigation-guide.md__other-keyboard-shortcuts}

{\def\LTcaptype{none} % do not increment counter
\begin{longtable}[]{@{}ll@{}}
\toprule\noalign{}
Key & Action \\
\midrule\noalign{}
\endhead
\bottomrule\noalign{}
\endlastfoot
\texttt{S} & Focus the search box \\
\texttt{Escape} & Close search results \\
\texttt{Left\ Arrow} & Previous chapter \\
\texttt{Right\ Arrow} & Next chapter \\
\texttt{T} & Toggle the table of contents sidebar \\
\end{longtable}
}

\subsubsection*{Screen Reader Navigation}\label{docs__pandoc__latex__src__3dmake_foundation__lessons_3dmake_1__mdbook-navigation-guide.md__screen-reader-navigation}

\paragraph*{With NVDA (Chrome, Firefox, or Edge recommended)}\label{docs__pandoc__latex__src__3dmake_foundation__lessons_3dmake_1__mdbook-navigation-guide.md__with-nvda-chrome-firefox-or-edge-recommended}

Reading the page:

\begin{itemize}
\tightlist
\item
  Use \texttt{Up\ /\ Down\ Arrow} to read line by line
\item
  Use \texttt{H} to jump between headings - this is the fastest way to skim a long lesson
\item
  Use \texttt{Ctrl\ +\ F} (browser Find) or \texttt{S} (mdBook search) to find specific content
\end{itemize}

Table of contents:

\begin{itemize}
\tightlist
\item
  The sidebar is a navigation landmark. Press \texttt{D} to move between landmark regions, or use \texttt{NVDA\ +\ F7} to open the Elements List and select "Landmarks" to navigate to the sidebar directly.
\item
  Within the sidebar, arrow through the list of links and press \texttt{Enter} to follow one.
\end{itemize}

Code blocks:

\begin{itemize}
\tightlist
\item
  Code examples are marked up as \texttt{\textless{}code\textgreater{}} elements. NVDA will read them inline.
\item
  Punctuation level should be set to Most or All to hear semicolons, brackets, and other syntax characters in code examples.
\item
  To copy a code block: navigate to the code, press \texttt{Ctrl\ +\ A} to select all, or use the copy button if present.
\end{itemize}

Tips:

\begin{itemize}
\tightlist
\item
  \texttt{NVDA\ +\ P} to cycle punctuation level - do this before reading code blocks
\item
  \texttt{NVDA\ +\ F7} -\textgreater{} Links list - useful for navigating between major sections quickly
\item
  \texttt{H} key (headings navigation) is your best friend on lesson pages with many sections
\end{itemize}

\paragraph*{With JAWS (Chrome or Edge recommended)}\label{docs__pandoc__latex__src__3dmake_foundation__lessons_3dmake_1__mdbook-navigation-guide.md__with-jaws-chrome-or-edge-recommended}

Reading the page:

\begin{itemize}
\tightlist
\item
  Use \texttt{Up\ /\ Down\ Arrow} in virtual cursor mode to read line by line
\item
  Press \texttt{H} to jump between headings
\item
  Press \texttt{JAWS\ Key\ +\ F6} to get a list of all headings on the page
\end{itemize}

Table of contents:

\begin{itemize}
\tightlist
\item
  Press \texttt{R} to move between landmark regions to reach the sidebar
\item
  Within the sidebar, \texttt{Tab} through the links or use \texttt{Up\ /\ Down\ Arrow}
\end{itemize}

Code blocks:

\begin{itemize}
\tightlist
\item
  Set punctuation to All before reading code: \texttt{JAWS\ Key\ +\ Shift\ +\ 2}
\item
  JAWS reads code blocks as regular text - navigate through them line by line
\end{itemize}

Tips:

\begin{itemize}
\tightlist
\item
  \texttt{JAWS\ Key\ +\ F5} - links list
\item
  \texttt{JAWS\ Key\ +\ F6} - headings list
\item
  \texttt{Ctrl\ +\ F} - browser find, works alongside mdBook search
\end{itemize}

\paragraph*{With VoiceOver (Mac / iOS)}\label{docs__pandoc__latex__src__3dmake_foundation__lessons_3dmake_1__mdbook-navigation-guide.md__with-voiceover-mac--ios}

Mac:

\begin{itemize}
\tightlist
\item
  \texttt{VO\ +\ U} to open the rotor - select Headings to navigate by heading
\item
  \texttt{VO\ +\ Command\ +\ F} to search the page
\item
  \texttt{H} (with Quick Nav on) to move between headings
\end{itemize}

iOS:

\begin{itemize}
\tightlist
\item
  Swipe left/right to navigate elements
\item
  Use the rotor (\texttt{two-finger\ rotate}) to set navigation mode to Headings
\item
  Double-tap to activate links
\end{itemize}

\subsubsection*{Finding What You Need}\label{docs__pandoc__latex__src__3dmake_foundation__lessons_3dmake_1__mdbook-navigation-guide.md__finding-what-you-need}

\paragraph*{If you know which unit or project you need}\label{docs__pandoc__latex__src__3dmake_foundation__lessons_3dmake_1__mdbook-navigation-guide.md__if-you-know-which-unit-or-project-you-need}

Open the table of contents and look for the unit or project name. The structure follows this pattern:

\begin{itemize}
\tightlist
\item
  Unit 0 - Foundation lessons (safety, how printing works, calipers, OpenSCAD basics, slicing)
\item
  Unit 1 - Guided projects (Project 0 and Project 1)
\item
  Unit 2 - Intermediate skills (parametric design, tolerances, advanced slicing, materials)
\item
  Unit 3 - Open-ended projects (Project 2, 3, and 4)
\item
  Reference Materials - Quick-reference sheets you can keep open while working
\item
  PowerShell Foundation - Command-line navigation guide
\end{itemize}

\paragraph*{If you are looking for a specific term or command}\label{docs__pandoc__latex__src__3dmake_foundation__lessons_3dmake_1__mdbook-navigation-guide.md__if-you-are-looking-for-a-specific-term-or-command}

Use the Search function (\texttt{S}). Search for:

\begin{itemize}
\tightlist
\item
  An OpenSCAD command (e.g., \texttt{difference}, \texttt{translate}, \texttt{module})
\item
  A vocabulary word (e.g., \texttt{infill}, \texttt{tolerance}, \texttt{stakeholder})
\item
  A project name (e.g., \texttt{floor\ marker}, \texttt{jewelry}, \texttt{assistive\ technology})
\end{itemize}

\paragraph*{If you are looking for reference material while working}\label{docs__pandoc__latex__src__3dmake_foundation__lessons_3dmake_1__mdbook-navigation-guide.md__if-you-are-looking-for-reference-material-while-working}

Keep a second browser tab open to the Reference Materials section. Useful pages to bookmark:

\begin{itemize}
\tightlist
\item
  OpenSCAD Cheat Sheet
\item
  Slicing Settings Quick Reference
\item
  Filament Comparison Table
\item
  Screen Reader Coding Tips (NVDA/JAWS)
\end{itemize}

\subsubsection*{Printing or Saving Pages}\label{docs__pandoc__latex__src__3dmake_foundation__lessons_3dmake_1__mdbook-navigation-guide.md__printing-or-saving-pages}

To save or print any page for offline use:

\begin{itemize}
\tightlist
\item
  \texttt{Ctrl\ +\ P} opens the print dialog in any browser
\item
  Use "Save as PDF" to save a local copy
\item
  For the whole book: if your instructor has provided a PDF version, use that - it contains all chapters in one file
\end{itemize}

\subsubsection*{Reporting a Problem}\label{docs__pandoc__latex__src__3dmake_foundation__lessons_3dmake_1__mdbook-navigation-guide.md__reporting-a-problem}

If a page is missing content, has a broken link, or is difficult to navigate with your screen reader, let your instructor know:

\begin{itemize}
\tightlist
\item
  Which page (chapter title)
\item
  What you were trying to do
\item
  What happened instead
\end{itemize}

This helps improve the curriculum for future students.

Other Screen Readers

Dolphin SuperNova (commercial) and Windows Narrator (built-in) are also supported; the workflows and recommendations in this document apply to them. See \url{https://yourdolphin.com/supernova/} and \url{https://support.microsoft.com/narrator} for vendor documentation.

\subsection{Screen Reader Coding Tips - NVDA \& JAWS}\label{docs__pandoc__latex__src__3dmake_foundation__lessons_3dmake_1__nvda-jaws-coding-tips.md__3dmake_foundation_lessons_3dmake_1-nvda-jaws-coding-tips}

\subsubsection*{General Principles for Coding with a Screen Reader}\label{docs__pandoc__latex__src__3dmake_foundation__lessons_3dmake_1__nvda-jaws-coding-tips.md__general-principles-for-coding-with-a-screen-reader}

\begin{itemize}
\tightlist
\item
  Turn punctuation up. All OpenSCAD syntax - semicolons, brackets, parentheses, commas - is meaningful. If your screen reader skips punctuation, you will miss syntax errors. Set punctuation to Most or All before coding.
\item
  Navigate by line. Arrow up and down to move through code one line at a time. Use \texttt{Ctrl\ +\ Left/Right} to move word by word within a line.
\item
  Use go-to-line. Both VSCode and Notepad++ let you jump to a specific line number with \texttt{Ctrl\ +\ G}. OpenSCAD errors always give a line number - use this to find them fast.
\item
  Spell when uncertain. If you are not sure what a character is, spell the line character by character using your screen reader\textquotesingle s spell command.
\item
  Comment as you go. Adding a short comment after a block of code lets you navigate by searching for recognizable words with \texttt{Ctrl\ +\ F}.
\end{itemize}

\subsubsection*{NVDA Quick Reference}\label{docs__pandoc__latex__src__3dmake_foundation__lessons_3dmake_1__nvda-jaws-coding-tips.md__nvda-quick-reference}

\paragraph*{Setting Punctuation Level}\label{docs__pandoc__latex__src__3dmake_foundation__lessons_3dmake_1__nvda-jaws-coding-tips.md__setting-punctuation-level}

\texttt{NVDA\ +\ P} - cycles through: None -\textgreater{} Some -\textgreater{} Most -\textgreater{} All

For code: set to Most or All.

You can also set a permanent default:
\texttt{NVDA\ +\ N} -\textgreater{} Preferences -\textgreater{} Settings -\textgreater{} Speech -\textgreater{} Symbol level

\paragraph*{Essential Reading Commands}\label{docs__pandoc__latex__src__3dmake_foundation__lessons_3dmake_1__nvda-jaws-coding-tips.md__essential-reading-commands}

{\def\LTcaptype{none} % do not increment counter
\begin{longtable}[]{@{}ll@{}}
\toprule\noalign{}
Action & Keys \\
\midrule\noalign{}
\endhead
\bottomrule\noalign{}
\endlastfoot
Read current line & \texttt{NVDA\ +\ Up\ Arrow} \\
Spell current line & \texttt{NVDA\ +\ Up\ Arrow} (twice quickly) \\
Read from cursor to end & \texttt{NVDA\ +\ Down\ Arrow} \\
Read current word & \texttt{NVDA\ +\ Numpad\ 5} \\
Spell current word & \texttt{NVDA\ +\ Numpad\ 5} (twice quickly) \\
Read current character & \texttt{NVDA\ +\ Numpad\ 2} \\
Stop reading & \texttt{Ctrl} \\
\end{longtable}
}

\paragraph*{Navigation in VSCode with NVDA}\label{docs__pandoc__latex__src__3dmake_foundation__lessons_3dmake_1__nvda-jaws-coding-tips.md__navigation-in-vscode-with-nvda}

{\def\LTcaptype{none} % do not increment counter
\begin{longtable}[]{@{}ll@{}}
\toprule\noalign{}
Action & Keys \\
\midrule\noalign{}
\endhead
\bottomrule\noalign{}
\endlastfoot
Move by character & \texttt{Left\ /\ Right\ Arrow} \\
Move by word & \texttt{Ctrl\ +\ Left\ /\ Right\ Arrow} \\
Move by line & \texttt{Up\ /\ Down\ Arrow} \\
Start / end of line & \texttt{Home\ /\ End} \\
Start / end of file & \texttt{Ctrl\ +\ Home\ /\ Ctrl\ +\ End} \\
Go to line number & \texttt{Ctrl\ +\ G}, type number, \texttt{Enter} \\
Find text & \texttt{Ctrl\ +\ F} \\
Toggle line comment & \texttt{Ctrl\ +\ /} \\
\end{longtable}
}

\paragraph*{If NVDA Stops Reading the Editor}\label{docs__pandoc__latex__src__3dmake_foundation__lessons_3dmake_1__nvda-jaws-coding-tips.md__if-nvda-stops-reading-the-editor}

\begin{enumerate}
\tightlist
\item
  Click in the editor area once with the mouse (or press \texttt{Escape} and then click)
\item
  Press \texttt{NVDA\ +\ Space} to toggle between Browse and Application mode - for code editors, you want Application mode
\end{enumerate}

\subsubsection*{JAWS Quick Reference}\label{docs__pandoc__latex__src__3dmake_foundation__lessons_3dmake_1__nvda-jaws-coding-tips.md__jaws-quick-reference}

\paragraph*{Setting Punctuation Level}\label{docs__pandoc__latex__src__3dmake_foundation__lessons_3dmake_1__nvda-jaws-coding-tips.md__setting-punctuation-level-1}

\texttt{JAWS\ Key\ +\ Shift\ +\ 2} - cycles through punctuation levels

For code: set to All.

\paragraph*{Essential Reading Commands}\label{docs__pandoc__latex__src__3dmake_foundation__lessons_3dmake_1__nvda-jaws-coding-tips.md__essential-reading-commands-1}

{\def\LTcaptype{none} % do not increment counter
\begin{longtable}[]{@{}
  >{\raggedright\arraybackslash}p{(\linewidth - 2\tabcolsep) * \real{0.3158}}
  >{\raggedright\arraybackslash}p{(\linewidth - 2\tabcolsep) * \real{0.6842}}@{}}
\toprule\noalign{}
\begin{minipage}[b]{\linewidth}\raggedright
Action
\end{minipage} & \begin{minipage}[b]{\linewidth}\raggedright
Keys
\end{minipage} \\
\midrule\noalign{}
\endhead
\bottomrule\noalign{}
\endlastfoot
Read current line & \texttt{JAWS\ Key\ +\ Up\ Arrow} \\
Spell current line & \texttt{JAWS\ Key\ +\ Up\ Arrow} (twice quickly) \\
Read from cursor & \texttt{JAWS\ Key\ +\ A} \\
Read current word & \texttt{JAWS\ Key\ +\ Numpad\ 5} \\
Read next word & \texttt{JAWS\ Key\ +\ Right\ Arrow} \\
Read current character &
\texttt{JAWS\ Key\ +\ Numpad\ 5} (once = word, twice = spell) \\
Stop reading & \texttt{Ctrl} \\
\end{longtable}
}

\paragraph*{Adjusting Speech Rate}\label{docs__pandoc__latex__src__3dmake_foundation__lessons_3dmake_1__nvda-jaws-coding-tips.md__adjusting-speech-rate}

{\def\LTcaptype{none} % do not increment counter
\begin{longtable}[]{@{}
  >{\raggedright\arraybackslash}p{(\linewidth - 2\tabcolsep) * \real{0.2000}}
  >{\raggedright\arraybackslash}p{(\linewidth - 2\tabcolsep) * \real{0.8000}}@{}}
\toprule\noalign{}
\begin{minipage}[b]{\linewidth}\raggedright
Action
\end{minipage} & \begin{minipage}[b]{\linewidth}\raggedright
Keys
\end{minipage} \\
\midrule\noalign{}
\endhead
\bottomrule\noalign{}
\endlastfoot
Increase rate &
\texttt{Alt\ +\ Ctrl\ +\ Page\ Up} (or \texttt{Fn\ +\ Up\ Arrow} on laptops) \\
Decrease rate &
\texttt{Alt\ +\ Ctrl\ +\ Page\ Down} (or \texttt{Fn\ +\ Down\ Arrow} on laptops) \\
\end{longtable}
}

\paragraph*{Navigation in VSCode with JAWS}\label{docs__pandoc__latex__src__3dmake_foundation__lessons_3dmake_1__nvda-jaws-coding-tips.md__navigation-in-vscode-with-jaws}

{\def\LTcaptype{none} % do not increment counter
\begin{longtable}[]{@{}ll@{}}
\toprule\noalign{}
Action & Keys \\
\midrule\noalign{}
\endhead
\bottomrule\noalign{}
\endlastfoot
Move by character & \texttt{Left\ /\ Right\ Arrow} \\
Move by word & \texttt{Ctrl\ +\ Left\ /\ Right\ Arrow} \\
Move by line & \texttt{Up\ /\ Down\ Arrow} \\
Start / end of line & \texttt{Home\ /\ End} \\
Start / end of file & \texttt{Ctrl\ +\ Home\ /\ Ctrl\ +\ End} \\
Go to line number & \texttt{Ctrl\ +\ G} \\
Find text & \texttt{Ctrl\ +\ F} \\
Toggle line comment & \texttt{Ctrl\ +\ /} \\
\end{longtable}
}

\paragraph*{If JAWS Stops Reading the Editor}\label{docs__pandoc__latex__src__3dmake_foundation__lessons_3dmake_1__nvda-jaws-coding-tips.md__if-jaws-stops-reading-the-editor}

\begin{enumerate}
\tightlist
\item
  Press \texttt{JAWS\ Key\ +\ Z} to toggle Virtual/Browse mode off - for VSCode you want Virtual mode off
\item
  Focus the menu bar with \texttt{Alt}, then press \texttt{Escape} to return to the editor
\item
  If still not working, press \texttt{Alt\ +\ F4} to close VSCode and reopen it
\end{enumerate}

\paragraph*{Window Focus}\label{docs__pandoc__latex__src__3dmake_foundation__lessons_3dmake_1__nvda-jaws-coding-tips.md__window-focus}

When you launch OpenSCAD from VSCode (via the task runner), focus stays in VSCode. To switch to OpenSCAD to read the console or error messages:

\begin{itemize}
\tightlist
\item
  \texttt{Alt\ +\ Tab} to cycle through open windows
\item
  JAWS will announce the application name as you cycle
\end{itemize}

\subsubsection*{OpenSCAD-Specific Tips}\label{docs__pandoc__latex__src__3dmake_foundation__lessons_3dmake_1__nvda-jaws-coding-tips.md__openscad-specific-tips}

\paragraph*{Reading Errors}\label{docs__pandoc__latex__src__3dmake_foundation__lessons_3dmake_1__nvda-jaws-coding-tips.md__reading-errors}

When OpenSCAD can\textquotesingle t render your code, it outputs an error. The error message always includes:

\begin{itemize}
\tightlist
\item
  Line number - use \texttt{Ctrl\ +\ G} in VSCode to jump there
\item
  Error type - usually a missing semicolon, a typo in a command name, or mismatched brackets
\end{itemize}

Common errors and what they sound like:

{\def\LTcaptype{none} % do not increment counter
\begin{longtable}[]{@{}
  >{\raggedright\arraybackslash}p{(\linewidth - 2\tabcolsep) * \real{0.3217}}
  >{\raggedright\arraybackslash}p{(\linewidth - 2\tabcolsep) * \real{0.6783}}@{}}
\toprule\noalign{}
\begin{minipage}[b]{\linewidth}\raggedright
Error message
\end{minipage} & \begin{minipage}[b]{\linewidth}\raggedright
What it usually means
\end{minipage} \\
\midrule\noalign{}
\endhead
\bottomrule\noalign{}
\endlastfoot
"Expected \textquotesingle;\textquotesingle{} ..." &
You forgot a semicolon at the end of a statement \\
"Expected \textquotesingle,\textquotesingle{} or \textquotesingle)\textquotesingle{} ..."
& Missing comma between parameters, or unclosed parenthesis \\
"Identifier ... is undefined" &
You typed a variable name wrong, or used a variable before declaring it \\
"WARNING: Normalized tree is empty" &
Your shape has no geometry (e.g., you subtracted more than you started with) \\
\end{longtable}
}

\paragraph*{Bracket Matching}\label{docs__pandoc__latex__src__3dmake_foundation__lessons_3dmake_1__nvda-jaws-coding-tips.md__bracket-matching}

All OpenSCAD shapes and operations use brackets and braces:

\begin{itemize}
\tightlist
\item
  Parentheses \texttt{()} - hold parameters: \texttt{cube({[}10,\ 10,\ 10{]})}
\item
  Square brackets \texttt{{[}{]}} - hold vectors (lists of numbers): \texttt{{[}10,\ 10,\ 10{]}}
\item
  Curly braces \texttt{\{\}} - hold groups of shapes for boolean operations
\end{itemize}

Every opening bracket must have a closing bracket. If you are missing one, OpenSCAD will report an error somewhere near (but not always exactly at) the problem.

VSCode can help: when your cursor is on a bracket, VSCode highlights the matching bracket. With NVDA, you can navigate to the matching bracket with \texttt{Ctrl\ +\ Shift\ +\ \textbackslash{}}.

\paragraph*{Commenting Out Code}\label{docs__pandoc__latex__src__3dmake_foundation__lessons_3dmake_1__nvda-jaws-coding-tips.md__commenting-out-code}

To test part of your code without deleting it, comment it out:

\begin{itemize}
\tightlist
\item
  Single line: position cursor at start of line, type \texttt{//}
\item
  Multiple lines: select the lines, press \texttt{Ctrl\ +\ /} in VSCode (adds \texttt{//} to each selected line)
\item
  Block comment: type \texttt{/*} before and \texttt{*/} after the block
\end{itemize}

To uncomment: select the commented lines, press \texttt{Ctrl\ +\ /} again.

\paragraph*{Navigating Large Files}\label{docs__pandoc__latex__src__3dmake_foundation__lessons_3dmake_1__nvda-jaws-coding-tips.md__navigating-large-files}

Use \texttt{Ctrl\ +\ F} (Find) to locate sections of your code:

\begin{itemize}
\tightlist
\item
  Search for your module names to jump to them: e.g., \texttt{module\ round\_bead}
\item
  Search for comments you wrote: e.g., \texttt{//\ Step\ 2}
\item
  Search for line numbers in error messages
\end{itemize}

\subsubsection*{Caliper and OpenSCAD Workflow Tips}\label{docs__pandoc__latex__src__3dmake_foundation__lessons_3dmake_1__nvda-jaws-coding-tips.md__caliper-and-openscad-workflow-tips}

When measuring an object and entering it into OpenSCAD, say the measurement aloud before typing it to reduce transcription errors. Then read the number back after typing it to confirm.

Recommended sequence:

\begin{enumerate}
\tightlist
\item
  Measure -\textgreater{} say "seventy point three millimeters"
\item
  Type \texttt{70.3} in OpenSCAD
\item
  Read back: "seven zero point three" to confirm
\end{enumerate}

\subsubsection*{mdBook Navigation (Web Version of This Curriculum)}\label{docs__pandoc__latex__src__3dmake_foundation__lessons_3dmake_1__nvda-jaws-coding-tips.md__mdbook-navigation-web-version-of-this-curriculum}

If you are reading this curriculum through the web version (mdBook), here are tips for navigating with a screen reader:

\paragraph*{NVDA + Browser}\label{docs__pandoc__latex__src__3dmake_foundation__lessons_3dmake_1__nvda-jaws-coding-tips.md__nvda--browser}

\begin{itemize}
\tightlist
\item
  \texttt{H} to jump between headings (chapter navigation)
\item
  \texttt{Ctrl\ +\ F} to search within the current page
\item
  The sidebar table of contents is a navigation landmark - use \texttt{NVDA\ +\ F7} to list landmarks, or press \texttt{D} to jump between landmark regions
\item
  Previous/Next chapter links are at the bottom of each page
\end{itemize}

\paragraph*{JAWS + Browser}\label{docs__pandoc__latex__src__3dmake_foundation__lessons_3dmake_1__nvda-jaws-coding-tips.md__jaws--browser}

\begin{itemize}
\tightlist
\item
  \texttt{H} to move between headings
\item
  \texttt{Ctrl\ +\ F} (browser Find) to search
\item
  \texttt{JAWS\ Key\ +\ F6} to list headings
\item
  \texttt{R} to jump to regions/landmarks
\item
  Use the search box in the mdBook header to search across all chapters
\end{itemize}

\paragraph*{Keyboard Navigation (No Screen Reader)}\label{docs__pandoc__latex__src__3dmake_foundation__lessons_3dmake_1__nvda-jaws-coding-tips.md__keyboard-navigation-no-screen-reader}

\begin{itemize}
\tightlist
\item
  \texttt{Left\ Arrow\ /\ Right\ Arrow} - previous/next chapter
\item
  \texttt{S} - focus the search box
\item
  \texttt{Escape} - close search results
\item
  \texttt{T} - toggle the table of contents sidebar
\end{itemize}

\subsubsection*{References}\label{docs__pandoc__latex__src__3dmake_foundation__lessons_3dmake_1__nvda-jaws-coding-tips.md__references}

NV Access. (2024). \emph{NVDA user guide}. \url{https://www.nvaccess.org/files/nvda/documentation/userGuide.html}

Freedom Scientific. (2024). \emph{JAWS for Windows help}. \url{https://support.freedomscientific.com/Content/Documents/Manuals/JAWS/JAWS-Screen-Reader-Help.pdf}

Microsoft. (2024). \emph{Visual Studio Code accessibility}. \url{https://code.visualstudio.com/docs/editor/accessibility}

\paragraph*{More on mdBook navigation}\label{docs__pandoc__latex__src__3dmake_foundation__lessons_3dmake_1__nvda-jaws-coding-tips.md__more-on-mdbook-navigation}

\begin{itemize}
\tightlist
\item
  mdBook (general / keyboard navigation): \url{https://rust-lang.github.io/mdBook/} - includes documentation and basic navigation/usage for the mdBook web UI.
\item
  Curriculum mdBook navigation (with accessibility tips): \hyperref[docs__pandoc__latex__src__3dmake_foundation__lessons_3dmake_1__mdbook-navigation-guide.md__3dmake_foundation_lessons_3dmake_1-mdbook-navigation-guide]{mdBook Navigation Guide} - local guide in this curriculum with notes for using mdBook with and without screen readers.
\end{itemize}

Other Screen Readers

Dolphin SuperNova (commercial) and Windows Narrator (built-in) are also supported; the workflows and recommendations in this document apply to them. See \url{https://yourdolphin.com/supernova/} and \url{https://support.microsoft.com/narrator} for vendor documentation.

\subsection{Your First Print - Guided Extension}\label{docs__pandoc__latex__src__3dmake_foundation__lessons_3dmake_1__your-first-print.md__3dmake_foundation_lessons_3dmake_1-your-first-print}

Estimated time: 2-4 hours (including setup and print monitoring)

\subsubsection*{Learning Objectives}\label{docs__pandoc__latex__src__3dmake_foundation__lessons_3dmake_1__your-first-print.md__learning-objectives}

\begin{itemize}
\tightlist
\item
  Select a simple ready-made model and evaluate its printability for a classroom printer
\item
  Configure slicer settings for a short-duration print and prepare the printer safely
\item
  Document print parameters and reflect on the physical outcome
\end{itemize}

\subsubsection*{Materials}\label{docs__pandoc__latex__src__3dmake_foundation__lessons_3dmake_1__your-first-print.md__materials}

\begin{itemize}
\tightlist
\item
  Computer with slicer and access to the online repository
\item
  Prusa Mini+ or classroom-approved printer, filament spool
\end{itemize}

\subsubsection*{Step-by-step Tasks}\label{docs__pandoc__latex__src__3dmake_foundation__lessons_3dmake_1__your-first-print.md__step-by-step-tasks}

\begin{enumerate}
\tightlist
\item
  Choose a simple model (\textless{}2 hours print) from Thingiverse or Printables and save the STL.
\item
  Inspect the model: note overhangs, thin features, and dimensions; write two short reasons why this model is appropriate for a first print.
\item
  Load the model in your slicer, select the classroom profile, and adjust settings only if necessary (layer height, infill, supports). Record the final print time and filament estimate.
\item
  Perform safety checks, start the print, and monitor the first 10 minutes for adhesion and extrusion problems.
\item
  After cooling, measure three critical dimensions and compare to the models stated dimensions; record deviations.
\end{enumerate}

\subsubsection*{Starter Code}\label{docs__pandoc__latex__src__3dmake_foundation__lessons_3dmake_1__your-first-print.md__starter-code}

You can use this basic project scaffold as your starting point:

\begin{lstlisting}[style=Alabaster, language=openscad]
// Basic Project Scaffold for 3D Printing
// This template provides a starting point for beginner projects
// Complete the TODO sections to create your own design
// ============================================
// PROJECT CONFIGURATION
// ============================================
// TODO: Set your project name
projectname = "My First Project";
// TODO: Define overall dimensions (mm)
objectwidth = 50;
objectdepth = 50;
objectheight = 20;
// Wall thickness for strength (2-3mm recommended)
wallthickness = 2;
// ============================================
// MATERIAL PROPERTIES
// ============================================
// Print resolution (0.2mm layers recommended)
$fn = 30;  // Fragment quality (higher = smoother curves, slower rendering)
// ============================================
// MAIN DESIGN
// ============================================
module baseshape() {
    // TODO: Replace this cube with your design
    cube([objectwidth, objectdepth, objectheight], center = true);
}
module hollowversion() {
    // Creates a hollow version by subtracting an inner cube
    difference() {
        baseshape();
        // Inner void (adjust padding as needed)
        translate([0, 0, 0.5])
            cube([objectwidth - 2*wallthickness, 
                  objectdepth - 2*wallthickness, 
                  objectheight - 2*wallthickness], 
                 center = true);
    }
}
// Render the solid version for your first print
baseshape();

\end{lstlisting}

\subsubsection*{Probing Questions}\label{docs__pandoc__latex__src__3dmake_foundation__lessons_3dmake_1__your-first-print.md__probing-questions}

\begin{itemize}
\tightlist
\item
  Why did you select this model? What risks did you anticipate and how did you mitigate them?
\item
  Which slicer setting most affects print time for this model and why?
\item
  If a thin feature failed, what minimal change would you make to ensure success next time?
\end{itemize}

\subsubsection*{Quiz - Your First Print (10 questions)}\label{docs__pandoc__latex__src__3dmake_foundation__lessons_3dmake_1__your-first-print.md__quiz---your-first-print-10-questions}

\begin{enumerate}
\tightlist
\item
  What is the first-check you do after loading filament? (short answer)
\item
  Name two slicer settings that affect strength. (short answer)
\item
  Why monitor the first layers of a print? (one sentence)
\item
  How do you document filament used for reproducibility? (short answer)
\item
  What is one sign of poor bed adhesion? (one sentence)
\item
  True/False: Selecting a model with support requirements is not recommended for your first print. (Answer: True)
\item
  Short answer: Describe two characteristics of a "printability-friendly" model that would be good for a beginner\textquotesingle s first print.
\item
  Practical scenario: Your first print is showing stringing (thin lines between parts). What are two possible causes and how would you troubleshoot?
\item
  Multiple choice: When measuring a printed dimension, where should you measure multiple times (three different spots) to account for print variation? (A) Never, measure once (B) Only if you suspect error (C) Always measure at least three locations - Answer: C
\item
  Reflection: Explain why carefully selecting your first print model (simple, robust, known to work on your printer) is better than immediately attempting a complex design. What will you learn from success that prepares you for harder projects?
\end{enumerate}

\subsubsection*{Extension Problems (10)}\label{docs__pandoc__latex__src__3dmake_foundation__lessons_3dmake_1__your-first-print.md__extension-problems-10}

\begin{enumerate}
\tightlist
\item
  Re-slice the model with a finer layer height and compare surface finish and print time; document differences.
\item
  Modify the model in OpenSCAD to thicken a failing feature and reprint a small test piece.
\item
  Create a short checklist script (or text checklist) that verifies spool metadata and bed temperature before printing.
\item
  Produce a one-page reflection that includes three lessons learned and one parameter you will change next time.
\item
  Share your measurements and photos in the class folder and give feedback on two peers\textquotesingle{} prints.
\item
  Conduct a post-print analysis: compare actual measurements to STL specifications; identify and document any deviations.
\item
  Print the same model in two different materials or with two different slicer profiles; compare durability, appearance, and accuracy.
\item
  Create a detailed documentation package: CAD file, slicer settings, print log, measurements, and lessons learned.
\item
  Design a quality assurance test: define pass/fail criteria and systematically verify your print meets all requirements.
\item
  Write a "first print troubleshooting guide" based on your experience: common issues you encountered and how you solved them.
\end{enumerate}

\subsubsection*{Deliverables}\label{docs__pandoc__latex__src__3dmake_foundation__lessons_3dmake_1__your-first-print.md__deliverables}

\begin{itemize}
\tightlist
\item
  Short report: model chosen, key slicer settings, measured deviations, and answers to probing questions.
\item
  Photos of the final print and the measured values table.
\end{itemize}

\subsection{Your First Print - Student Documentation Template (Extension Project)}\label{docs__pandoc__latex__src__3dmake_foundation__lessons_3dmake_1__your_first_print_student_template.md__3dmake_foundation_lessons_3dmake_1-your_first_print_student_template}

\begin{itemize}
\tightlist
\item
  Author:
\item
  Date:
\item
  Description: Select a simple 3D model, configure slicer settings, and complete your first independent print job.
\end{itemize}

\subsubsection*{Ideas and Concept}\label{docs__pandoc__latex__src__3dmake_foundation__lessons_3dmake_1__your_first_print_student_template.md__ideas-and-concept}

\paragraph*{Model Selection}\label{docs__pandoc__latex__src__3dmake_foundation__lessons_3dmake_1__your_first_print_student_template.md__model-selection}

\begin{itemize}
\tightlist
\item
  Model name and source (Thingiverse, Printables, etc.):
\item
  Link or file reference:
\item
  Why did you choose this model?
\end{itemize}

\paragraph*{Printability Assessment}\label{docs__pandoc__latex__src__3dmake_foundation__lessons_3dmake_1__your_first_print_student_template.md__printability-assessment}

\begin{itemize}
\tightlist
\item
  Estimated print time (from slicer):
\item
  Estimated filament:
\item
  Expected challenges (overhangs, thin features, supports needed):
\item
  Why is this model appropriate for a first print?
\end{itemize}

\subsubsection*{Measurements and Printer Configuration}\label{docs__pandoc__latex__src__3dmake_foundation__lessons_3dmake_1__your_first_print_student_template.md__measurements-and-printer-configuration}

\paragraph*{Printer Setup Log}\label{docs__pandoc__latex__src__3dmake_foundation__lessons_3dmake_1__your_first_print_student_template.md__printer-setup-log}

{\def\LTcaptype{none} % do not increment counter
\begin{longtable}[]{@{}ll@{}}
\toprule\noalign{}
Parameter & Value \\
\midrule\noalign{}
\endhead
\bottomrule\noalign{}
\endlastfoot
Printer model & \\
Nozzle diameter & \\
Bed temperature & \\
Nozzle temperature & \\
Layer height & \\
Infill percentage & \\
Support settings & \\
\end{longtable}
}

\paragraph*{Safety Checks (Completed)}\label{docs__pandoc__latex__src__3dmake_foundation__lessons_3dmake_1__your_first_print_student_template.md__safety-checks-completed}

\begin{itemize}
\tightlist
\item[$\square$]
  Bed clean and level
\item[$\square$]
  Bed adhesive/prep applied (if needed)
\item[$\square$]
  Filament loaded correctly
\item[$\square$]
  Nozzle at correct height
\item[$\square$]
  Print area clear of obstructions
\item[$\square$]
  Buildplate secured
\end{itemize}

\subsubsection*{Object Notes}\label{docs__pandoc__latex__src__3dmake_foundation__lessons_3dmake_1__your_first_print_student_template.md__object-notes}

\paragraph*{Print Monitoring Notes}\label{docs__pandoc__latex__src__3dmake_foundation__lessons_3dmake_1__your_first_print_student_template.md__print-monitoring-notes}

\begin{itemize}
\tightlist
\item
  What did you observe in the first 10 minutes?
\item
  Any adhesion issues?
\item
  Any extrusion problems?
\item
  Print completed successfully: Yes / No
\end{itemize}

\paragraph*{Printed Part Measurements (After Cooling)}\label{docs__pandoc__latex__src__3dmake_foundation__lessons_3dmake_1__your_first_print_student_template.md__printed-part-measurements-after-cooling}

{\def\LTcaptype{none} % do not increment counter
\begin{longtable}[]{@{}lllll@{}}
\toprule\noalign{}
Feature & Model Spec (mm) & Printed (mm) & Deviation & Notes \\
\midrule\noalign{}
\endhead
\bottomrule\noalign{}
\endlastfoot
& & & & \\
& & & & \\
& & & & \\
\end{longtable}
}

\paragraph*{Use and Assembly Notes}\label{docs__pandoc__latex__src__3dmake_foundation__lessons_3dmake_1__your_first_print_student_template.md__use-and-assembly-notes}

\begin{itemize}
\tightlist
\item
  How does the print feel and look compared to the model?
\item
  Did any features fail or print poorly?
\item
  If there were issues, what do you suspect caused them?
\end{itemize}

\subsubsection*{Reflections}\label{docs__pandoc__latex__src__3dmake_foundation__lessons_3dmake_1__your_first_print_student_template.md__reflections}

\paragraph*{What Went Well}\label{docs__pandoc__latex__src__3dmake_foundation__lessons_3dmake_1__your_first_print_student_template.md__what-went-well}

\begin{itemize}
\tightlist
\item
  Which aspect of the print was most successful?
\item
  Did anything surprise you positively?
\end{itemize}

\paragraph*{What Didn\textquotesingle t Go Well}\label{docs__pandoc__latex__src__3dmake_foundation__lessons_3dmake_1__your_first_print_student_template.md__what-didnt-go-well}

\begin{itemize}
\tightlist
\item
  Were there any defects or failed features?
\item
  What challenges did you encounter?
\end{itemize}

\paragraph*{Learning and Next Steps}\label{docs__pandoc__latex__src__3dmake_foundation__lessons_3dmake_1__your_first_print_student_template.md__learning-and-next-steps}

\begin{itemize}
\tightlist
\item
  What did you learn from this first independent print?
\item
  What would you do differently next time?
\item
  If you printed this again, what settings would you change?
\end{itemize}

\paragraph*{Post-Print Analysis}\label{docs__pandoc__latex__src__3dmake_foundation__lessons_3dmake_1__your_first_print_student_template.md__post-print-analysis}

\begin{itemize}
\tightlist
\item
  Compare this print to others you\textquotesingle ve seen in the classroom
\item
  What variables (temperature, layer height, infill) do you think most affected the outcome?
\item
  How would you test that hypothesis?
\end{itemize}

\subsubsection*{Attachments}\label{docs__pandoc__latex__src__3dmake_foundation__lessons_3dmake_1__your_first_print_student_template.md__attachments}

\begin{itemize}
\tightlist
\item[$\square$]
  Exported \texttt{.stl} model file (or link to source)
\item[$\square$]
  Slicer settings (screenshot or export)
\item[$\square$]
  Photo of final print (multiple angles if possible)
\item[$\square$]
  Safety checklist (signed/dated)
\item[$\square$]
  Print log with timestamps (if available from printer)
\end{itemize}

\subsubsection*{Teacher Feedback}\label{docs__pandoc__latex__src__3dmake_foundation__lessons_3dmake_1__your_first_print_student_template.md__teacher-feedback}

{\def\LTcaptype{none} % do not increment counter
\begin{longtable}[]{@{}lll@{}}
\toprule\noalign{}
Category & Score & Notes \\
\midrule\noalign{}
\endhead
\bottomrule\noalign{}
\endlastfoot
Problem \& Solution (0-3) & & \\
Design \& Code Quality (0-3) & & \\
Documentation (0-3) & & \\
Total (0-9) & & \\
\end{longtable}
}

Feedback:

\subsubsection*{Resubmission (if applicable)}\label{docs__pandoc__latex__src__3dmake_foundation__lessons_3dmake_1__your_first_print_student_template.md__resubmission-if-applicable}

\paragraph*{What Was Changed}\label{docs__pandoc__latex__src__3dmake_foundation__lessons_3dmake_1__your_first_print_student_template.md__what-was-changed}

(One-paragraph explanation of changes made and why)

\paragraph*{Revised Score}\label{docs__pandoc__latex__src__3dmake_foundation__lessons_3dmake_1__your_first_print_student_template.md__revised-score}

{\def\LTcaptype{none} % do not increment counter
\begin{longtable}[]{@{}lll@{}}
\toprule\noalign{}
Category & Revised Score & Notes \\
\midrule\noalign{}
\endhead
\bottomrule\noalign{}
\endlastfoot
Problem \& Solution (0-3) & & \\
Design \& Code Quality (0-3) & & \\
Documentation (0-3) & & \\
Total (0-9) & & \\
\end{longtable}
}

\subsection{Your First Print - Teacher Template (Extension Project)}\label{docs__pandoc__latex__src__3dmake_foundation__lessons_3dmake_1__your_first_print_teacher_template.md__3dmake_foundation_lessons_3dmake_1-your_first_print_teacher_template}

\subsubsection*{Briefing}\label{docs__pandoc__latex__src__3dmake_foundation__lessons_3dmake_1__your_first_print_teacher_template.md__briefing}

A foundational extension project where students select a ready-made 3D model from online repositories, configure slicer settings, and complete their first independent print job. This project builds confidence and establishes safe printing practices.

Key Learning: Printer safety; slicer configuration; print monitoring; dimensional verification.

Real-world Connection: Production environments rely on safe print setup and quality verification. This project establishes habits that transfer to any 3D printing context.

\subsubsection*{Constraints}\label{docs__pandoc__latex__src__3dmake_foundation__lessons_3dmake_1__your_first_print_teacher_template.md__constraints}

\begin{itemize}
\tightlist
\item
  Model must be  2 hours print time
\item
  No or minimal support requirements recommended
\item
  Student must document printer configuration and safety checks
\item
  Student must compare printed dimensions to model specifications
\end{itemize}

\subsubsection*{Functional Requirements}\label{docs__pandoc__latex__src__3dmake_foundation__lessons_3dmake_1__your_first_print_teacher_template.md__functional-requirements}

\begin{itemize}
\tightlist
\item
  Model selected and deemed appropriate for beginner
\item
  Printer configured safely with classroom profile
\item
  Print completes successfully with good adhesion
\item
  Printed part is measured and deviations documented
\end{itemize}

\subsubsection*{Deliverables}\label{docs__pandoc__latex__src__3dmake_foundation__lessons_3dmake_1__your_first_print_teacher_template.md__deliverables}

\begin{itemize}
\tightlist
\item
  Completed documentation template with:

  \begin{itemize}
  \tightlist
  \item
    Model selection rationale and printability assessment
  \item
    Printer setup log (materials, temperature, settings)
  \item
    Print monitoring observations (first 10 minutes, monitoring during print)
  \item
    Dimensional measurements and deviation analysis
  \item
    Reflection on first print experience
  \item
    Post-print analysis and troubleshooting notes
  \end{itemize}
\item
  Photograph of final print
\item
  Slicer settings (screenshot or text export)
\end{itemize}

\subsubsection*{Rubric}\label{docs__pandoc__latex__src__3dmake_foundation__lessons_3dmake_1__your_first_print_teacher_template.md__rubric}

All projects are scored on a 0-9 scale across three equally weighted categories (3 points each):

{\def\LTcaptype{none} % do not increment counter
\begin{longtable}[]{@{}
  >{\raggedright\arraybackslash}p{(\linewidth - 4\tabcolsep) * \real{0.1983}}
  >{\raggedright\arraybackslash}p{(\linewidth - 4\tabcolsep) * \real{0.0690}}
  >{\raggedright\arraybackslash}p{(\linewidth - 4\tabcolsep) * \real{0.7328}}@{}}
\toprule\noalign{}
\begin{minipage}[b]{\linewidth}\raggedright
Category
\end{minipage} & \begin{minipage}[b]{\linewidth}\raggedright
Points
\end{minipage} & \begin{minipage}[b]{\linewidth}\raggedright
What We Measure
\end{minipage} \\
\midrule\noalign{}
\endhead
\bottomrule\noalign{}
\endlastfoot
Problem \& Solution & 0-3 &
Did the student select an appropriate model? Did the print succeed? \\
Design \& Code Quality & 0-3 &
Is documentation of printer setup thorough? Is print quality good? \\
Documentation & 0-3 &
Is all documentation complete? Are measurements recorded? Is reflection thoughtful? \\
\end{longtable}
}

\paragraph*{Category 1: Problem \& Solution (0-3 points)}\label{docs__pandoc__latex__src__3dmake_foundation__lessons_3dmake_1__your_first_print_teacher_template.md__category-1-problem--solution-0-3-points}

{\def\LTcaptype{none} % do not increment counter
\begin{longtable}[]{@{}
  >{\raggedright\arraybackslash}p{(\linewidth - 2\tabcolsep) * \real{0.0493}}
  >{\raggedright\arraybackslash}p{(\linewidth - 2\tabcolsep) * \real{0.9507}}@{}}
\toprule\noalign{}
\begin{minipage}[b]{\linewidth}\raggedright
Score
\end{minipage} & \begin{minipage}[b]{\linewidth}\raggedright
Description
\end{minipage} \\
\midrule\noalign{}
\endhead
\bottomrule\noalign{}
\endlastfoot
3 &
Student selected appropriate model with sound reasoning. Print completed successfully with good adhesion and finish. No major issues. \\
2 &
Model generally appropriate. Print mostly successful with minor adhesion or finish issues. \\
1 &
Model had some difficulties; print partially successful or required intervention. \\
0 & Print failed or model was inappropriate. \\
\end{longtable}
}

\paragraph*{Category 2: Design \& Code Quality (0-3 points)}\label{docs__pandoc__latex__src__3dmake_foundation__lessons_3dmake_1__your_first_print_teacher_template.md__category-2-design--code-quality-0-3-points}

{\def\LTcaptype{none} % do not increment counter
\begin{longtable}[]{@{}
  >{\raggedright\arraybackslash}p{(\linewidth - 2\tabcolsep) * \real{0.0636}}
  >{\raggedright\arraybackslash}p{(\linewidth - 2\tabcolsep) * \real{0.9364}}@{}}
\toprule\noalign{}
\begin{minipage}[b]{\linewidth}\raggedright
Score
\end{minipage} & \begin{minipage}[b]{\linewidth}\raggedright
Description
\end{minipage} \\
\midrule\noalign{}
\endhead
\bottomrule\noalign{}
\endlastfoot
3 &
Printer setup documented thoroughly. Safety checks evident. Print quality excellent. Photos provided. \\
2 &
Setup documented adequately. Print quality acceptable. Some documentation of settings. \\
1 & Minimal documentation of setup. Print quality has defects. \\
0 & Setup not documented or print failed to start. \\
\end{longtable}
}

\paragraph*{Category 3: Documentation (0-3 points)}\label{docs__pandoc__latex__src__3dmake_foundation__lessons_3dmake_1__your_first_print_teacher_template.md__category-3-documentation-0-3-points}

{\def\LTcaptype{none} % do not increment counter
\begin{longtable}[]{@{}
  >{\raggedright\arraybackslash}p{(\linewidth - 2\tabcolsep) * \real{0.0654}}
  >{\raggedright\arraybackslash}p{(\linewidth - 2\tabcolsep) * \real{0.9346}}@{}}
\toprule\noalign{}
\begin{minipage}[b]{\linewidth}\raggedright
Score
\end{minipage} & \begin{minipage}[b]{\linewidth}\raggedright
Description
\end{minipage} \\
\midrule\noalign{}
\endhead
\bottomrule\noalign{}
\endlastfoot
3 &
All sections complete. Measurements recorded precisely. Reflection is specific and shows learning. \\
2 &
Most sections present. Measurements recorded but reflection brief. \\
1 & Incomplete sections. Measurements minimal. Reflection lacking. \\
0 & No documentation. \\
\end{longtable}
}

\paragraph*{Score Interpretation}\label{docs__pandoc__latex__src__3dmake_foundation__lessons_3dmake_1__your_first_print_teacher_template.md__score-interpretation}

{\def\LTcaptype{none} % do not increment counter
\begin{longtable}[]{@{}
  >{\raggedright\arraybackslash}p{(\linewidth - 4\tabcolsep) * \real{0.1444}}
  >{\raggedright\arraybackslash}p{(\linewidth - 4\tabcolsep) * \real{0.3111}}
  >{\raggedright\arraybackslash}p{(\linewidth - 4\tabcolsep) * \real{0.5444}}@{}}
\toprule\noalign{}
\begin{minipage}[b]{\linewidth}\raggedright
Total Score
\end{minipage} & \begin{minipage}[b]{\linewidth}\raggedright
Interpretation
\end{minipage} & \begin{minipage}[b]{\linewidth}\raggedright
Next Step
\end{minipage} \\
\midrule\noalign{}
\endhead
\bottomrule\noalign{}
\endlastfoot
8-9 & Excellent work &
Student ready for independent printing projects \\
6-7 & Good execution & Encourage documentation habit \\
4-5 & Meets basics & Discuss printer setup and safety \\
2-3 & Does not meet expectations &
Resubmission or additional practice \\
0-1 & Missing major components & Meet with instructor \\
\end{longtable}
}

\subsubsection*{Resubmission Policy}\label{docs__pandoc__latex__src__3dmake_foundation__lessons_3dmake_1__your_first_print_teacher_template.md__resubmission-policy}

Students may resubmit to improve their score. Resubmissions must include:

\begin{enumerate}
\tightlist
\item
  A one-paragraph explanation of what was changed and why
\end{enumerate}

The resubmission score replaces the original.

\subsubsection*{Assessment Notes}\label{docs__pandoc__latex__src__3dmake_foundation__lessons_3dmake_1__your_first_print_teacher_template.md__assessment-notes}

\begin{itemize}
\tightlist
\item
  Strong submissions show careful model selection reasoning, thorough printer documentation, and specific observations about print behavior
\item
  Watch for: Poor printer documentation, dismissive attitude toward safety checks, or generic reflections
\item
  Reinforce: Safety first; documentation enables learning and troubleshooting
\item
  Next Step: Extension problem suggestions include print quality analysis or variant material testing
\end{itemize}

\section{Lesson 2: Geometric Primitives and Constructive Solid Geometry}\label{docs__pandoc__latex__src__3dmake_foundation__lessons_3dmake_2__lessons_3dmake_2.md__lesson-2-geometric-primitives-and-constructive-solid-geometry}

Estimated time: 90--120 minutes

\subsection*{Learning Objectives}\label{docs__pandoc__latex__src__3dmake_foundation__lessons_3dmake_2__lessons_3dmake_2.md__learning-objectives}

\begin{itemize}
\tightlist
\item
  Use all six OpenSCAD primitive shapes: \texttt{cube}, \texttt{sphere}, \texttt{cylinder}, \texttt{polyhedron}, \texttt{text}, \texttt{surface}
\item
  Apply the four CSG operations: \texttt{union}, \texttt{difference}, \texttt{intersection}, and \texttt{hull}
\item
  Use modifier characters (\texttt{\#}, \texttt{!}, \texttt{\%}, \texttt{*}) for debugging
\item
  Understand and apply the \texttt{0.001} offset rule for clean Boolean operations\footnote{OpenSCAD User Manual --- Primitive Solids and Boolean Operations - \url{https://en.wikibooks.org/wiki/OpenSCAD_User_Manual/Primitive_Solids}. The 0.001 offset rule is a community convention documented in the OpenSCAD forums to prevent co-planar face artifacts in Boolean operations.}
\end{itemize}

\subsection*{Materials}\label{docs__pandoc__latex__src__3dmake_foundation__lessons_3dmake_2__lessons_3dmake_2.md__materials}

\begin{itemize}
\tightlist
\item
  3dMake project from Lesson 1
\item
  Terminal
\item
  OpenSCAD (for live preview with F5)
\end{itemize}

\subsection*{Step-by-step Tasks}\label{docs__pandoc__latex__src__3dmake_foundation__lessons_3dmake_2__lessons_3dmake_2.md__step-by-step-tasks}

\subsubsection*{1. Build a Compound Object with union and difference}\label{docs__pandoc__latex__src__3dmake_foundation__lessons_3dmake_2__lessons_3dmake_2.md__1-build-a-compound-object-with-union-and-difference}

\begin{lstlisting}[style=Alabaster, language=openscad]
// Simple canister: cylinder body with a sphere on top
difference() {
  union() {
    cylinder(h=40, r=15, $fn=64);
    translate([0, 0, 40]) sphere(r=15, $fn=64);
  }
  // Hollow out the inside (0.001 offset prevents co-planar faces)
  translate([0, 0, 3])
    cylinder(h=38 + 0.001, r=13, $fn=64);
}

\end{lstlisting}

Save as \texttt{src/main.scad} and run \texttt{3dm\ build}. Preview with F5 for fast render; use F6 for final export-quality render.

\subsubsection*{2. Understand the 0.001 Offset Rule}\label{docs__pandoc__latex__src__3dmake_foundation__lessons_3dmake_2__lessons_3dmake_2.md__2-understand-the-0001-offset-rule}

When two surfaces are exactly co-planar, OpenSCAD may produce rendering artifacts or non-manifold faces. Adding a tiny \texttt{0.001\ mm} overlap ensures the Boolean operation cuts completely through:

\begin{lstlisting}[style=Alabaster, language=openscad]
// WRONG - co-planar bottom faces may cause artifacts
difference() {
  cube([20, 20, 10]);
  cube([18, 18, 10]);  // same height - ambiguous
}

// CORRECT - 0.001 ensures clean cut
difference() {
  cube([20, 20, 10]);
  translate([1, 1, -0.001])
    cube([18, 18, 10 + 0.002]);  // 0.001 below and above
}

\end{lstlisting}

This is a widely documented community convention in OpenSCAD for avoiding non-manifold geometry.\footnote{OpenSCAD User Manual --- Primitive Solids and Boolean Operations - \url{https://en.wikibooks.org/wiki/OpenSCAD_User_Manual/Primitive_Solids}. The 0.001 offset rule is a community convention documented in the OpenSCAD forums to prevent co-planar face artifacts in Boolean operations.}

\subsubsection*{3. Apply Modifier Characters for Debugging}\label{docs__pandoc__latex__src__3dmake_foundation__lessons_3dmake_2__lessons_3dmake_2.md__3-apply-modifier-characters-for-debugging}

\begin{lstlisting}[style=Alabaster, language=openscad]
// # = highlight in pink (still rendered)
# cube([10, 10, 10]);

// % = ghost/transparent (shown but not part of model)
% cube([20, 20, 20]);

// ! = render only this object (ignore everything else)
! sphere(r=10);

// * = disable this object entirely
* cube([5, 5, 5]);

\end{lstlisting}

Use \texttt{\#} and \texttt{\%} while debugging to visualize which geometry is being subtracted or added. Remove all modifier characters before final export. See \footnote{OpenSCAD User Manual --- Modifier Characters - \url{https://en.wikibooks.org/wiki/OpenSCAD_User_Manual/Modifier_Characters}} for more on modifier characters.

\subsubsection*{\texorpdfstring{4. Use rotate\_extrude and linear\_extrude \footnote{OpenSCAD User Manual --- Transformations and Extrusions - \url{https://en.wikibooks.org/wiki/OpenSCAD_User_Manual/Using_the_2D_Subsystem}}}{4. Use rotate\_extrude and linear\_extrude }}\label{docs__pandoc__latex__src__3dmake_foundation__lessons_3dmake_2__lessons_3dmake_2.md__4-use-rotate_extrude-and-linear_extrude-}

\begin{lstlisting}[style=Alabaster, language=openscad]
// linear_extrude: 2D profile extruded along Z axis
linear_extrude(height=20, twist=0, scale=1) {
  circle(r=10, $fn=32);
}

// rotate_extrude: 2D profile rotated around Z axis (creates vase/ring shapes)
rotate_extrude(angle=360, $fn=64) {
  translate([15, 0, 0]) circle(r=5, $fn=32);
}

\end{lstlisting}

\subsubsection*{\texorpdfstring{5. Use intersection and hull \footnote{OpenSCAD User Manual --- CSG Modelling - \url{https://en.wikibooks.org/wiki/OpenSCAD_User_Manual/CSG_Modelling}}}{5. Use intersection and hull }}\label{docs__pandoc__latex__src__3dmake_foundation__lessons_3dmake_2__lessons_3dmake_2.md__5-use-intersection-and-hull-}

\begin{lstlisting}[style=Alabaster, language=openscad]
// intersection: keeps only the volume common to both shapes
intersection() {
  cube([20, 20, 20], center=true);
  sphere(r=13, $fn=64);
}

// hull: convex envelope of all child geometry
hull() {
  translate([0, 0, 0]) sphere(r=5);
  translate([30, 0, 0]) sphere(r=5);
  translate([15, 20, 0]) sphere(r=5);
}

\end{lstlisting}

\subsubsection*{Checkpoint}\label{docs__pandoc__latex__src__3dmake_foundation__lessons_3dmake_2__lessons_3dmake_2.md__checkpoint}

\begin{itemize}
\tightlist
\item
  F5 renders quickly in preview mode (not manifold-safe); F6 performs full CGAL render. Always use F6 / \texttt{3dm\ build} before slicing.
\item
  If the slicer reports non-manifold faces, check for missing \texttt{0.001} offsets on co-planar surfaces.
\end{itemize}

\subsection*{Advanced CSG Patterns}\label{docs__pandoc__latex__src__3dmake_foundation__lessons_3dmake_2__lessons_3dmake_2.md__advanced-csg-patterns}

\subsubsection*{Combining Operations for Complex Parts}\label{docs__pandoc__latex__src__3dmake_foundation__lessons_3dmake_2__lessons_3dmake_2.md__combining-operations-for-complex-parts}

Real parts require nested CSG trees. Here is a parametric mounting bracket that combines all four operations:

\begin{lstlisting}[style=Alabaster, language=openscad]
// Parametric Mounting Bracket
width = 40;
height = 30;
depth = 8;
hole_r = 4;
slot_w = 6;
slot_h = 15;
wall = 3;

module bracket() {
  difference() {
    // Main body
    cube([width, depth, height]);

    // Two mounting holes
    translate([10, -0.001, 10])
      rotate([-90, 0, 0]) cylinder(r=hole_r, h=depth + 0.002, $fn=32);
    translate([30, -0.001, 10])
      rotate([-90, 0, 0]) cylinder(r=hole_r, h=depth + 0.002, $fn=32);

    // Lightening slot
    translate([width/2 - slot_w/2, -0.001, height - slot_h - wall])
      cube([slot_w, depth + 0.002, slot_h]);
  }
}

bracket();

\end{lstlisting}

\subsubsection*{Polyhedron for Irregular Shapes}\label{docs__pandoc__latex__src__3dmake_foundation__lessons_3dmake_2__lessons_3dmake_2.md__polyhedron-for-irregular-shapes}

\begin{lstlisting}[style=Alabaster, language=openscad]
// Wedge using polyhedron
polyhedron(
  points = [
    [0, 0, 0], [20, 0, 0], [20, 15, 0], [0, 15, 0],  // bottom
    [0, 0, 10], [20, 0, 10]                             // top edge
  ],
  faces = [
    [0, 1, 2, 3],  // bottom
    [0, 4, 5, 1],  // front
    [1, 5, 2],     // right
    [0, 3, 4],     // left
    [3, 2, 5, 4],  // back/top
  ]
);

\end{lstlisting}

\subsection*{Quiz --- Lesson 3dMake.2 (15 questions)}\label{docs__pandoc__latex__src__3dmake_foundation__lessons_3dmake_2__lessons_3dmake_2.md__quiz--lesson-3dmake2-15-questions}

\begin{enumerate}
\tightlist
\item
  What are the six OpenSCAD primitive shapes?
\item
  What does \texttt{difference()} do in CSG?
\item
  Why do you add \texttt{0.001} mm to the cutting geometry in a \texttt{difference()} operation?
\item
  What does the \texttt{\#} modifier character do, and when would you use it?
\item
  What is the difference between \texttt{F5} and \texttt{F6} in OpenSCAD?
\item
  What does \texttt{hull()} produce, and how is it different from \texttt{union()}?
\item
  What does the \texttt{\%} modifier do to a shape in OpenSCAD?
\item
  Describe what \texttt{rotate\_extrude()} does and give one example of a shape it could produce.
\item
  What does \texttt{intersection()} return when applied to two overlapping cubes?
\item
  True or False: the \texttt{*} modifier renders a shape as a ghost for debugging.
\item
  Describe what \texttt{linear\_extrude()} does and explain the \texttt{twist} parameter.
\item
  What is a non-manifold face, and what common OpenSCAD mistake produces it?
\item
  If you want to subtract a cylinder from a cube and the cylinder is exactly as tall as the cube, what do you need to add to ensure a clean cut?
\item
  Explain when you would use \texttt{polyhedron()} instead of simpler primitives.
\item
  What is the difference between \texttt{union()} combining two overlapping shapes and simply rendering two separate shapes without a CSG operation?
\end{enumerate}

\subsection*{Extension Problems (15)}\label{docs__pandoc__latex__src__3dmake_foundation__lessons_3dmake_2__lessons_3dmake_2.md__extension-problems-15}

\begin{enumerate}
\tightlist
\item
  Build a hollow sphere (a shell with \texttt{1.5\ mm} walls) using \texttt{difference()} and the \texttt{0.001} rule.\footnote{OpenSCAD User Manual --- Primitive Solids and Boolean Operations - \url{https://en.wikibooks.org/wiki/OpenSCAD_User_Manual/Primitive_Solids}. The 0.001 offset rule is a community convention documented in the OpenSCAD forums to prevent co-planar face artifacts in Boolean operations.}
\item
  Create a vase shape using \texttt{rotate\_extrude()} and a custom 2D profile.
\item
  Design a bracket or clip using nested \texttt{difference()} and \texttt{union()} operations. Document each CSG step with inline comments.
\item
  Use \texttt{hull()} to create a smooth transition between two differently-sized shapes.
\item
  Experiment with the \texttt{\%} modifier: build a model where a ghost reference shape helps you position a cut accurately. Screenshot the debugging view and explain it.
\item
  Create a parametric name badge: a flat base with your name text embossed using \texttt{linear\_extrude()} and \texttt{difference()}.
\item
  Build a compound hinge using two cylinders, a \texttt{hull()}, and alignment holes.
\item
  Design a 10-sided (decagonal) prism using \texttt{cylinder()} with \texttt{\$fn=10} and a \texttt{difference()} to cut a through-hole.
\item
  Create a test print that exercises all four CSG operations in a single part (document what each operation does in a header comment).
\item
  Using only \texttt{cube()}, \texttt{sphere()}, \texttt{difference()}, and \texttt{hull()}, build a simple car silhouette (top-down view).
\item
  Create a lattice structure using a \texttt{for} loop combined with \texttt{difference()} to cut a grid of holes in a cube. Document the relationship between hole spacing and wall thickness.
\item
  Build a parametric ring using \texttt{rotate\_extrude()} and make the cross-section shape (circle, square, or triangle) a parameter.
\item
  Research and document the \texttt{surface()} primitive: what input does it accept, what shape does it produce, and when would you use it instead of \texttt{polyhedron()}? Include a working example.
\item
  Design a tolerance test set: 5 pairs of pegs and holes with clearances from 0.0 mm to 0.4 mm in 0.1 mm increments. Print one set and document which clearance allows free movement.
\item
  Write a short guide explaining the four modifier characters (\texttt{\#}, \texttt{!}, \texttt{\%}, \texttt{*}) with one example use case for each and a note on when to remove them before final export.
\end{enumerate}

\subsection*{References and Helpful Resources}\label{docs__pandoc__latex__src__3dmake_foundation__lessons_3dmake_2__lessons_3dmake_2.md__references-and-helpful-resources}

\subsubsection*{Supplemental Resources}\label{docs__pandoc__latex__src__3dmake_foundation__lessons_3dmake_2__lessons_3dmake_2.md__supplemental-resources}

\begin{itemize}
\tightlist
\item
  \href{docs/pandoc/latex/src/assets/Programming_with_OpenSCAD.epub}{Programming with OpenSCAD EPUB Textbook} --- Chapters on CSG and primitives
\item
  \href{https://github.com/ProgrammingWithOpenSCAD/CodeSolutions}{CodeSolutions Repository} --- Worked examples for CSG, hull, and extrusions
\item
  \href{https://programmingwithopenscad.github.io/quick-reference.html}{OpenSCAD Quick Reference} --- All primitive and CSG syntax at a glance
\item
  \href{https://github.com/tdeck/3dmake}{3DMake GitHub Repository} --- Build workflow reference
\end{itemize}

\subsection{OpenSCAD Quick Reference - Cheat Sheet}\label{docs__pandoc__latex__src__3dmake_foundation__lessons_3dmake_2__openscad-cheat-sheet.md__3dmake_foundation_lessons_3dmake_2-openscad-cheat-sheet}

\emph{Keep this handy during all OpenSCAD work. For full documentation: \url{https://openscad.org/documentation.html}}

\subsubsection*{Basic Shapes (Primitives)}\label{docs__pandoc__latex__src__3dmake_foundation__lessons_3dmake_2__openscad-cheat-sheet.md__basic-shapes-primitives}

\begin{lstlisting}[style=Alabaster, language=openscad]
cube([length, width, height]);          // rectangular box
cube([l, w, h], center = true);        // centered at origin

sphere(r = radius);                     // sphere by radius
sphere(d = diameter);                   // sphere by diameter

cylinder(h = height, r = radius);       // cylinder
cylinder(h = height, d = diameter);     // cylinder by diameter
cylinder(h = height, r1 = 5, r2 = 2);  // cone (different top/bottom radii)

$fn = 50;  // sets smoothness for spheres/cylinders (higher = smoother, slower)

\end{lstlisting}

\subsubsection*{Transformations}\label{docs__pandoc__latex__src__3dmake_foundation__lessons_3dmake_2__openscad-cheat-sheet.md__transformations}

\begin{lstlisting}[style=Alabaster, language=openscad]
translate([x, y, z]) shape;             // move
rotate([x_deg, y_deg, z_deg]) shape;    // rotate
scale([x, y, z]) shape;                 // scale (1.5 = 150%)
mirror([1, 0, 0]) shape;               // mirror across YZ plane (use [0,1,0] for XZ, [0,0,1] for XY)

\end{lstlisting}

\subsubsection*{Boolean Operations}\label{docs__pandoc__latex__src__3dmake_foundation__lessons_3dmake_2__openscad-cheat-sheet.md__boolean-operations}

\begin{lstlisting}[style=Alabaster, language=openscad]
union() { shape1; shape2; }            // combine shapes
difference() { base; subtract; }       // remove one shape from another
intersection() { shape1; shape2; }     // keep only overlapping region

\end{lstlisting}

Tip for difference(): Always make the subtracting shape 1 mm taller on both ends than the base shape to avoid zero-thickness artifacts.

\begin{lstlisting}[style=Alabaster, language=openscad]
// Example - box with a hole:
difference() {
    cube([30, 30, 10]);
    translate([15, 15, -1]) cylinder(h = 12, r = 5);  // extends -1 to +11
}

\end{lstlisting}

\subsubsection*{Variables}\label{docs__pandoc__latex__src__3dmake_foundation__lessons_3dmake_2__openscad-cheat-sheet.md__variables}

\begin{lstlisting}[style=Alabaster, language=openscad]
length = 70;      // define a variable
width = 16;
height = 5;

cube([length, width, height]);    // use the variables
cube([length * 2, width, height]); // math works too

\end{lstlisting}

\subsubsection*{Modules (Reusable Functions)}\label{docs__pandoc__latex__src__3dmake_foundation__lessons_3dmake_2__openscad-cheat-sheet.md__modules-reusable-functions}

\begin{lstlisting}[style=Alabaster, language=openscad]
// Define a module:
module my_box(l = 20, w = 15, h = 10) {
    cube([l, w, h]);
}

// Call the module:
my_box();               // uses all defaults
my_box(30, 20, 8);     // positional arguments
my_box(l = 50);         // named argument; others use defaults

\end{lstlisting}

\subsubsection*{Loops}\label{docs__pandoc__latex__src__3dmake_foundation__lessons_3dmake_2__openscad-cheat-sheet.md__loops}

\begin{lstlisting}[style=Alabaster, language=openscad]
// Repeat a shape N times:
for (i = [0 : 4]) {         // i goes 0, 1, 2, 3, 4
    translate([i * 20, 0, 0]) cube([15, 15, 5]);
}

// Step size:
for (i = [0 : 5 : 20]) {   // i goes 0, 5, 10, 15, 20
    translate([i, 0, 0]) sphere(r = 3);
}

\end{lstlisting}

\subsubsection*{2D Shapes (for extrusion)}\label{docs__pandoc__latex__src__3dmake_foundation__lessons_3dmake_2__openscad-cheat-sheet.md__2d-shapes-for-extrusion}

\begin{lstlisting}[style=Alabaster, language=openscad]
circle(r = 10);            // 2D circle
square([w, h]);            // 2D rectangle
polygon([[0,0],[10,0],[5,10]]);  // arbitrary 2D shape

// Extrude a 2D shape into 3D:
linear_extrude(height = 5) circle(r = 10);  // makes a cylinder
rotate_extrude() translate([15, 0]) circle(r = 3);  // makes a torus

\end{lstlisting}

\subsubsection*{Useful Functions}\label{docs__pandoc__latex__src__3dmake_foundation__lessons_3dmake_2__openscad-cheat-sheet.md__useful-functions}

\begin{lstlisting}[style=Alabaster, language=openscad]
len([a, b, c])         // returns 3 (length of a vector)
sqrt(25)               // returns 5
pow(2, 8)              // returns 256 (2^8)
abs(-5)                // returns 5
min(3, 5, 1)           // returns 1
max(3, 5, 1)           // returns 5

\end{lstlisting}

\subsubsection*{Comments}\label{docs__pandoc__latex__src__3dmake_foundation__lessons_3dmake_2__openscad-cheat-sheet.md__comments}

\begin{lstlisting}[style=Alabaster, language=openscad]
// Single-line comment

/*
  Multi-line
  comment
*/

\end{lstlisting}

\subsubsection*{Keyboard Shortcuts}\label{docs__pandoc__latex__src__3dmake_foundation__lessons_3dmake_2__openscad-cheat-sheet.md__keyboard-shortcuts}

{\def\LTcaptype{none} % do not increment counter
\begin{longtable}[]{@{}ll@{}}
\toprule\noalign{}
Key & Action \\
\midrule\noalign{}
\endhead
\bottomrule\noalign{}
\endlastfoot
F5 & Preview (fast) \\
F6 & Full render (for export) \\
Ctrl+S & Save \\
Ctrl+Z & Undo \\
F3 & Reset camera view \\
F5 then scroll & Zoom with mouse \\
\end{longtable}
}

\subsubsection*{Export Workflow}\label{docs__pandoc__latex__src__3dmake_foundation__lessons_3dmake_2__openscad-cheat-sheet.md__export-workflow}

\begin{enumerate}
\tightlist
\item
  Press F6 (full render - wait for it to complete)
\item
  File \textgreater{} Export \textgreater{} Export as STL
\item
  Save with a descriptive filename: \texttt{projectname\_v2.stl}
\end{enumerate}

\subsubsection*{Sources}\label{docs__pandoc__latex__src__3dmake_foundation__lessons_3dmake_2__openscad-cheat-sheet.md__sources}

OpenSCAD. (n.d.). \emph{OpenSCAD cheatsheet}. \url{https://openscad.org/cheatsheet/}\\
OpenSCAD. (n.d.). \emph{OpenSCAD documentation}. \url{https://openscad.org/documentation.html}\\
Gohde, J., \& Kintel, M. (2021). \emph{Programming with OpenSCAD}. No Starch Press. \url{https://nostarch.com/programmingopenscad}

\subsection{Real-Life Problem Solver - Guided Extension}\label{docs__pandoc__latex__src__3dmake_foundation__lessons_3dmake_2__your-second-print.md__3dmake_foundation_lessons_3dmake_2-your-second-print}

Estimated time: 4-8 hours (design, adapt, print, and document)

\subsubsection*{Learning Objectives}\label{docs__pandoc__latex__src__3dmake_foundation__lessons_3dmake_2__your-second-print.md__learning-objectives}

\begin{itemize}
\tightlist
\item
  Identify a user problem and evaluate candidate printed solutions
\item
  Adapt an existing model to meet real constraints and safety requirements
\item
  Test, measure, and iterate on a prototype with documented decisions
\end{itemize}

\subsubsection*{Materials}\label{docs__pandoc__latex__src__3dmake_foundation__lessons_3dmake_2__your-second-print.md__materials}

\begin{itemize}
\tightlist
\item
  Computer with slicer and access to repositories
\item
  Printer, filament, basic hand tools for post-processing
\end{itemize}

\subsubsection*{Step-by-step Tasks}\label{docs__pandoc__latex__src__3dmake_foundation__lessons_3dmake_2__your-second-print.md__step-by-step-tasks}

\begin{enumerate}
\tightlist
\item
  Interview a potential user (or yourself) and write a 1-paragraph problem statement.
\item
  Search repositories for candidate models; list three options and justify which one you will adapt.
\item
  Adapt the model (scale, add mounts, or modify features) and document the changes in a short changelog.
\item
  Slice, print, and run a supervised test of the prototype; log any failures and corrective actions.
\item
  Produce a final report summarizing performance, measured deviations, and next steps.
\end{enumerate}

\subsubsection*{Probing Questions}\label{docs__pandoc__latex__src__3dmake_foundation__lessons_3dmake_2__your-second-print.md__probing-questions}

\begin{itemize}
\tightlist
\item
  What assumptions did you make about the user\textquotesingle s context? How could you validate them?
\item
  Which adaptation had the biggest impact on function and why?
\end{itemize}

\subsubsection*{Quiz - Your Second Print (10 questions)}\label{docs__pandoc__latex__src__3dmake_foundation__lessons_3dmake_2__your-second-print.md__quiz---your-second-print-10-questions}

\begin{enumerate}
\tightlist
\item
  What is a good first question to ask a stakeholder when scoping this project? (short answer)
\item
  Name one safety consideration when adapting a model for daily use. (short answer)
\item
  What is a changelog entry? (one sentence)
\item
  How do you verify a fit for an assembled part? (short answer)
\item
  Why document corrective actions during testing? (one sentence)
\item
  True/False: Once you complete a successful first print, any second project will automatically succeed without iteration. (Answer: False)
\item
  Short answer: Describe one method to gather user feedback on your adapted model before committing to a full production print.
\item
  Practical scenario: Your adapted model prints but feels too fragile in daily use. What are two strategies to improve durability while maintaining the basic form?
\item
  Multiple choice: When you create a changelog, what should you document? (A) Only the successful changes (B) All changes, including failures and iterations (C) Only the slicer settings - Answer: B
\item
  Reflection: Explain how iteration (design -\textgreater{} print -\textgreater{} test -\textgreater{} adapt) leads to better functional outcomes than trying to get the design perfect on the first attempt. Give a specific example from your project.
\end{enumerate}

\subsubsection*{Extension Problems (10)}\label{docs__pandoc__latex__src__3dmake_foundation__lessons_3dmake_2__your-second-print.md__extension-problems-10}

\begin{enumerate}
\tightlist
\item
  Rework your prototype to improve durability and report trade-offs in weight and print time.
\item
  Create a user test script and run it with two participants; summarize results.
\item
  Convert a critical component to parametric OpenSCAD and publish the variant.
\item
  Add a small assembly guide with tactile cues for non-visual users.
\item
  Compare two filament types for the same part and recommend one with justification.
\item
  Execute a formal design iteration cycle: print, test, measure, analyze, revise; repeat at least 3 times and document improvements.
\item
  Build a comparative analysis: print your part in 2 different materials or with 2 different profiles; measure and compare all key properties.
\item
  Create a complete design dossier: CAD files, iteration history, test results, measurements, lessons learned, and final recommendations.
\item
  Develop a user testing protocol: define success metrics, recruit testers, gather systematic feedback, and iterate based on results.
\item
  Write a design case study: document your entire project journey from initial concept through final manufacture, including all mistakes and successes.
\end{enumerate}

\subsection{Your Second Print - Student Documentation Template (Extension Project)}\label{docs__pandoc__latex__src__3dmake_foundation__lessons_3dmake_2__your_second_print_student_template.md__3dmake_foundation_lessons_3dmake_2-your_second_print_student_template}

\begin{itemize}
\tightlist
\item
  Author:
\item
  Date:
\item
  Description: Adapt or modify an existing 3D model to meet a new constraint, improve functionality, or customize for a specific use.
\end{itemize}

\subsubsection*{Ideas and Concept}\label{docs__pandoc__latex__src__3dmake_foundation__lessons_3dmake_2__your_second_print_student_template.md__ideas-and-concept}

\paragraph*{Original Model}\label{docs__pandoc__latex__src__3dmake_foundation__lessons_3dmake_2__your_second_print_student_template.md__original-model}

\begin{itemize}
\tightlist
\item
  Model name and source:
\item
  Link or file reference:
\item
  Why did you choose this model for adaptation?
\end{itemize}

\paragraph*{Adaptation Goal}\label{docs__pandoc__latex__src__3dmake_foundation__lessons_3dmake_2__your_second_print_student_template.md__adaptation-goal}

\begin{itemize}
\tightlist
\item
  What problem or constraint are you addressing?
\item
  How will you know the adaptation is successful?
\item
  Who will use the adapted model?
\end{itemize}

\subsubsection*{Measurements and Design Specifications}\label{docs__pandoc__latex__src__3dmake_foundation__lessons_3dmake_2__your_second_print_student_template.md__measurements-and-design-specifications}

\paragraph*{Original Model Specifications}\label{docs__pandoc__latex__src__3dmake_foundation__lessons_3dmake_2__your_second_print_student_template.md__original-model-specifications}

{\def\LTcaptype{none} % do not increment counter
\begin{longtable}[]{@{}ll@{}}
\toprule\noalign{}
Parameter & Value \\
\midrule\noalign{}
\endhead
\bottomrule\noalign{}
\endlastfoot
Print time & \\
Filament & \\
Key dimension 1 & \\
Key dimension 2 & \\
\end{longtable}
}

\paragraph*{Adapted Model Specifications}\label{docs__pandoc__latex__src__3dmake_foundation__lessons_3dmake_2__your_second_print_student_template.md__adapted-model-specifications}

{\def\LTcaptype{none} % do not increment counter
\begin{longtable}[]{@{}llll@{}}
\toprule\noalign{}
Parameter & Original & Adapted & Reason for Change \\
\midrule\noalign{}
\endhead
\bottomrule\noalign{}
\endlastfoot
& & & \\
& & & \\
& & & \\
\end{longtable}
}

\subsubsection*{Object Notes}\label{docs__pandoc__latex__src__3dmake_foundation__lessons_3dmake_2__your_second_print_student_template.md__object-notes}

\paragraph*{Modifications Made}\label{docs__pandoc__latex__src__3dmake_foundation__lessons_3dmake_2__your_second_print_student_template.md__modifications-made}

\begin{itemize}
\tightlist
\item
  Describe changes to the original design:
\item
  Did you modify the model in OpenSCAD, slicer, or both?
\item
  Include before/after code snippets or screenshots:
\end{itemize}

\paragraph*{Design Iteration Cycle}\label{docs__pandoc__latex__src__3dmake_foundation__lessons_3dmake_2__your_second_print_student_template.md__design-iteration-cycle}

\subparagraph*{Print 1}\label{docs__pandoc__latex__src__3dmake_foundation__lessons_3dmake_2__your_second_print_student_template.md__print-1}

\begin{itemize}
\tightlist
\item
  Date:
\item
  Modifications:
\item
  Results:
\item
  Issues:
\item
  Next action:
\end{itemize}

\subparagraph*{Print 2 (if applicable)}\label{docs__pandoc__latex__src__3dmake_foundation__lessons_3dmake_2__your_second_print_student_template.md__print-2-if-applicable}

\begin{itemize}
\tightlist
\item
  Date:
\item
  Modifications:
\item
  Results:
\item
  Issues:
\item
  Next action:
\end{itemize}

\subparagraph*{Print 3 (if applicable)}\label{docs__pandoc__latex__src__3dmake_foundation__lessons_3dmake_2__your_second_print_student_template.md__print-3-if-applicable}

\begin{itemize}
\tightlist
\item
  Date:
\item
  Modifications:
\item
  Results:
\item
  Final assessment:
\end{itemize}

\paragraph*{User Testing (if applicable)}\label{docs__pandoc__latex__src__3dmake_foundation__lessons_3dmake_2__your_second_print_student_template.md__user-testing-if-applicable}

\begin{itemize}
\tightlist
\item
  Who tested the adapted model?
\item
  What feedback did they provide?
\item
  How did this feedback influence your design?
\end{itemize}

\paragraph*{Assembly and Use Notes}\label{docs__pandoc__latex__src__3dmake_foundation__lessons_3dmake_2__your_second_print_student_template.md__assembly-and-use-notes}

\begin{itemize}
\tightlist
\item
  How is the adapted model assembled or used?
\item
  How does it differ from the original in use?
\item
  What challenges did you encounter?
\end{itemize}

\subsubsection*{Reflections}\label{docs__pandoc__latex__src__3dmake_foundation__lessons_3dmake_2__your_second_print_student_template.md__reflections}

\paragraph*{Design Evolution}\label{docs__pandoc__latex__src__3dmake_foundation__lessons_3dmake_2__your_second_print_student_template.md__design-evolution}

\begin{itemize}
\tightlist
\item
  Describe how your adaptation evolved through the iteration process
\item
  Which iteration felt most successful?
\end{itemize}

\paragraph*{Lessons Learned}\label{docs__pandoc__latex__src__3dmake_foundation__lessons_3dmake_2__your_second_print_student_template.md__lessons-learned}

\begin{itemize}
\tightlist
\item
  What did you learn about adapting existing designs?
\item
  What surprised you during the process?
\end{itemize}

\paragraph*{Comparison: Original vs. Adapted}\label{docs__pandoc__latex__src__3dmake_foundation__lessons_3dmake_2__your_second_print_student_template.md__comparison-original-vs-adapted}

\begin{itemize}
\tightlist
\item
  What are the key functional differences?
\item
  Would you recommend the adapted version? Why?
\end{itemize}

\paragraph*{Future Iterations}\label{docs__pandoc__latex__src__3dmake_foundation__lessons_3dmake_2__your_second_print_student_template.md__future-iterations}

\begin{itemize}
\tightlist
\item
  If you were to continue developing this, what would you change?
\item
  How could you make the design more parametric or reusable?
\end{itemize}

\paragraph*{Accessibility Considerations}\label{docs__pandoc__latex__src__3dmake_foundation__lessons_3dmake_2__your_second_print_student_template.md__accessibility-considerations}

\begin{itemize}
\tightlist
\item
  How did you approach testing this project with screen readers?
\item
  What documentation would help someone without visual access understand the changes?
\end{itemize}

\subsubsection*{Attachments}\label{docs__pandoc__latex__src__3dmake_foundation__lessons_3dmake_2__your_second_print_student_template.md__attachments}

\begin{itemize}
\tightlist
\item[$\square$]
  Original model file (link or reference)
\item[$\square$]
  Adapted model file (\texttt{.scad} or \texttt{.stl})
\item[$\square$]
  Code comparison (before/after with comments)
\item[$\square$]
  Photos of adapted print(s)
\item[$\square$]
  Iteration log with timestamps
\item[$\square$]
  User feedback notes (if applicable)
\item[$\square$]
  Slicer settings for final print
\end{itemize}

\subsubsection*{Teacher Feedback}\label{docs__pandoc__latex__src__3dmake_foundation__lessons_3dmake_2__your_second_print_student_template.md__teacher-feedback}

{\def\LTcaptype{none} % do not increment counter
\begin{longtable}[]{@{}lll@{}}
\toprule\noalign{}
Category & Score & Notes \\
\midrule\noalign{}
\endhead
\bottomrule\noalign{}
\endlastfoot
Problem \& Solution (0-3) & & \\
Design \& Code Quality (0-3) & & \\
Documentation (0-3) & & \\
Total (0-9) & & \\
\end{longtable}
}

Feedback:

\subsubsection*{Resubmission (if applicable)}\label{docs__pandoc__latex__src__3dmake_foundation__lessons_3dmake_2__your_second_print_student_template.md__resubmission-if-applicable}

\paragraph*{What Was Changed}\label{docs__pandoc__latex__src__3dmake_foundation__lessons_3dmake_2__your_second_print_student_template.md__what-was-changed}

(One-paragraph explanation of changes made and why)

\paragraph*{Revised Score}\label{docs__pandoc__latex__src__3dmake_foundation__lessons_3dmake_2__your_second_print_student_template.md__revised-score}

{\def\LTcaptype{none} % do not increment counter
\begin{longtable}[]{@{}lll@{}}
\toprule\noalign{}
Category & Revised Score & Notes \\
\midrule\noalign{}
\endhead
\bottomrule\noalign{}
\endlastfoot
Problem \& Solution (0-3) & & \\
Design \& Code Quality (0-3) & & \\
Documentation (0-3) & & \\
Total (0-9) & & \\
\end{longtable}
}

\subsection{Your Second Print - Teacher Template (Extension Project)}\label{docs__pandoc__latex__src__3dmake_foundation__lessons_3dmake_2__your_second_print_teacher_template.md__3dmake_foundation_lessons_3dmake_2-your_second_print_teacher_template}

\subsubsection*{Briefing}\label{docs__pandoc__latex__src__3dmake_foundation__lessons_3dmake_2__your_second_print_teacher_template.md__briefing}

Building on the first successful print, students now adapt an existing model to meet a specific constraint or improvement goal. This project emphasizes design iteration, parametric thinking, and the ability to modify existing designs for new contexts.

Key Learning: Design iteration; constraint-based modification; documentation of changes.

Real-world Connection: Adaptation and iteration are core engineering practices. Most real products are refinements of earlier versions.

\subsubsection*{Constraints}\label{docs__pandoc__latex__src__3dmake_foundation__lessons_3dmake_2__your_second_print_teacher_template.md__constraints}

\begin{itemize}
\tightlist
\item
  Must be a modification or adaptation of an existing 3D model
\item
  Modifications must be parametric (variables or commented code showing what changed)
\item
  Student must document the modification rationale and testing process
\item
  Iteration should be evidence through multiple print attempts or variant comparisons
\end{itemize}

\subsubsection*{Functional Requirements}\label{docs__pandoc__latex__src__3dmake_foundation__lessons_3dmake_2__your_second_print_teacher_template.md__functional-requirements}

\begin{itemize}
\tightlist
\item
  Modification is clearly documented with before/after code comparison
\item
  Adapted print functions as intended in the modified context
\item
  Student provides evidence of testing the adaptation
\item
  Design shows intentional thought about materials, fit, or functionality
\end{itemize}

\subsubsection*{Deliverables}\label{docs__pandoc__latex__src__3dmake_foundation__lessons_3dmake_2__your_second_print_teacher_template.md__deliverables}

\begin{itemize}
\tightlist
\item
  Completed documentation template with:

  \begin{itemize}
  \tightlist
  \item
    Original model identification and link
  \item
    Modifications made (with code comments or diff)
  \item
    Design iteration cycle (print, test, adapt, repeat)
  \item
    User testing results (if applicable)
  \item
    Reflection on design decisions
  \item
    Comparison of original vs. adapted version
  \end{itemize}
\item
  Modified \texttt{.scad} or \texttt{.stl} files showing changes
\item
  Photos of both original and adapted prints (if possible)
\item
  Test results or user feedback documentation
\end{itemize}

\subsubsection*{Rubric}\label{docs__pandoc__latex__src__3dmake_foundation__lessons_3dmake_2__your_second_print_teacher_template.md__rubric}

All projects are scored on a 0-9 scale across three equally weighted categories (3 points each):

{\def\LTcaptype{none} % do not increment counter
\begin{longtable}[]{@{}
  >{\raggedright\arraybackslash}p{(\linewidth - 4\tabcolsep) * \real{0.1855}}
  >{\raggedright\arraybackslash}p{(\linewidth - 4\tabcolsep) * \real{0.0645}}
  >{\raggedright\arraybackslash}p{(\linewidth - 4\tabcolsep) * \real{0.7500}}@{}}
\toprule\noalign{}
\begin{minipage}[b]{\linewidth}\raggedright
Category
\end{minipage} & \begin{minipage}[b]{\linewidth}\raggedright
Points
\end{minipage} & \begin{minipage}[b]{\linewidth}\raggedright
What We Measure
\end{minipage} \\
\midrule\noalign{}
\endhead
\bottomrule\noalign{}
\endlastfoot
Problem \& Solution & 0-3 &
Is the adapted design functional? Does it solve the stated adaptation goal? \\
Design \& Code Quality & 0-3 &
Are modifications clear and well-documented? Is iteration evident? Does the part work well? \\
Documentation & 0-3 &
Is the iteration cycle documented? Are design decisions explained? Is reflection thorough? \\
\end{longtable}
}

\paragraph*{Category 1: Problem \& Solution (0-3 points)}\label{docs__pandoc__latex__src__3dmake_foundation__lessons_3dmake_2__your_second_print_teacher_template.md__category-1-problem--solution-0-3-points}

{\def\LTcaptype{none} % do not increment counter
\begin{longtable}[]{@{}
  >{\raggedright\arraybackslash}p{(\linewidth - 2\tabcolsep) * \real{0.0579}}
  >{\raggedright\arraybackslash}p{(\linewidth - 2\tabcolsep) * \real{0.9421}}@{}}
\toprule\noalign{}
\begin{minipage}[b]{\linewidth}\raggedright
Score
\end{minipage} & \begin{minipage}[b]{\linewidth}\raggedright
Description
\end{minipage} \\
\midrule\noalign{}
\endhead
\bottomrule\noalign{}
\endlastfoot
3 &
Adaptation successfully addresses design goal. Print is functional and performs as intended in modified context. \\
2 &
Adaptation mostly addresses goal. Print is functional with minor limitations. \\
1 &
Adaptation partially addresses goal. Print has functional limitations. \\
0 & Adaptation does not work or is not attempted. \\
\end{longtable}
}

\paragraph*{Category 2: Design \& Code Quality (0-3 points)}\label{docs__pandoc__latex__src__3dmake_foundation__lessons_3dmake_2__your_second_print_teacher_template.md__category-2-design--code-quality-0-3-points}

{\def\LTcaptype{none} % do not increment counter
\begin{longtable}[]{@{}
  >{\raggedright\arraybackslash}p{(\linewidth - 2\tabcolsep) * \real{0.0483}}
  >{\raggedright\arraybackslash}p{(\linewidth - 2\tabcolsep) * \real{0.9517}}@{}}
\toprule\noalign{}
\begin{minipage}[b]{\linewidth}\raggedright
Score
\end{minipage} & \begin{minipage}[b]{\linewidth}\raggedright
Description
\end{minipage} \\
\midrule\noalign{}
\endhead
\bottomrule\noalign{}
\endlastfoot
3 &
Modifications clearly documented with before/after comparison. Iteration cycle evident (multiple prints/tests). Print quality excellent. \\
2 &
Modifications documented adequately. Some iteration evident. Print quality acceptable. \\
1 &
Minimal modification documentation. Little iteration. Print quality acceptable but lacks refinement. \\
0 & Modifications not documented or design not functional. \\
\end{longtable}
}

\paragraph*{Category 3: Documentation (0-3 points)}\label{docs__pandoc__latex__src__3dmake_foundation__lessons_3dmake_2__your_second_print_teacher_template.md__category-3-documentation-0-3-points}

{\def\LTcaptype{none} % do not increment counter
\begin{longtable}[]{@{}
  >{\raggedright\arraybackslash}p{(\linewidth - 2\tabcolsep) * \real{0.0504}}
  >{\raggedright\arraybackslash}p{(\linewidth - 2\tabcolsep) * \real{0.9496}}@{}}
\toprule\noalign{}
\begin{minipage}[b]{\linewidth}\raggedright
Score
\end{minipage} & \begin{minipage}[b]{\linewidth}\raggedright
Description
\end{minipage} \\
\midrule\noalign{}
\endhead
\bottomrule\noalign{}
\endlastfoot
3 &
All sections complete. Design iteration documented with measurements and testing results. Reflection is specific. Photos included. \\
2 &
Most sections present. Iteration documented but could be more detailed. Reflection adequate. \\
1 &
Incomplete sections. Minimal iteration documentation. Reflection brief. \\
0 & No documentation submitted. \\
\end{longtable}
}

\paragraph*{Score Interpretation}\label{docs__pandoc__latex__src__3dmake_foundation__lessons_3dmake_2__your_second_print_teacher_template.md__score-interpretation}

{\def\LTcaptype{none} % do not increment counter
\begin{longtable}[]{@{}
  >{\raggedright\arraybackslash}p{(\linewidth - 4\tabcolsep) * \real{0.1548}}
  >{\raggedright\arraybackslash}p{(\linewidth - 4\tabcolsep) * \real{0.3929}}
  >{\raggedright\arraybackslash}p{(\linewidth - 4\tabcolsep) * \real{0.4524}}@{}}
\toprule\noalign{}
\begin{minipage}[b]{\linewidth}\raggedright
Total Score
\end{minipage} & \begin{minipage}[b]{\linewidth}\raggedright
Interpretation
\end{minipage} & \begin{minipage}[b]{\linewidth}\raggedright
Next Step
\end{minipage} \\
\midrule\noalign{}
\endhead
\bottomrule\noalign{}
\endlastfoot
8-9 & Excellent adaptation & Student demonstrates design maturity \\
6-7 & Good iteration and adaptation & Encourage further design work \\
4-5 & Meets basics; improve iteration &
Resubmit iteration documentation \\
2-3 & Does not meet expectations & Resubmission or coaching \\
0-1 & Missing components & Meet with instructor \\
\end{longtable}
}

\subsubsection*{Resubmission Policy}\label{docs__pandoc__latex__src__3dmake_foundation__lessons_3dmake_2__your_second_print_teacher_template.md__resubmission-policy}

Students may resubmit to improve their score. Resubmissions must include:

\begin{enumerate}
\tightlist
\item
  A one-paragraph explanation of what was changed and why
\end{enumerate}

The resubmission score replaces the original.

\subsubsection*{Assessment Notes}\label{docs__pandoc__latex__src__3dmake_foundation__lessons_3dmake_2__your_second_print_teacher_template.md__assessment-notes}

\begin{itemize}
\tightlist
\item
  Strong submissions show clear modification intent, multiple iteration cycles with documented changes, and user feedback integration
\item
  Watch for: Minimal modifications, no iteration, or generic reflections
\item
  Reinforce: Why iteration matters; how to document design decisions
\item
  Extension: Portfolio development; design for manufacturability
\end{itemize}

\subsection{Bonus Print - Guided Extension}\label{docs__pandoc__latex__src__3dmake_foundation__lessons_3dmake_2__bonus-print.md__3dmake_foundation_lessons_3dmake_2-bonus-print}

Estimated time: 3-5 hours

\subsubsection*{Learning Objectives}\label{docs__pandoc__latex__src__3dmake_foundation__lessons_3dmake_2__bonus-print.md__learning-objectives}

\begin{itemize}
\tightlist
\item
  Apply scaling and simple modifications to an existing model
\item
  Verify multi-part prints and manage print constraints
\item
  Document design changes and reproduceable print settings
\end{itemize}

\subsubsection*{Materials}\label{docs__pandoc__latex__src__3dmake_foundation__lessons_3dmake_2__bonus-print.md__materials}

\begin{itemize}
\tightlist
\item
  Online repository access, slicer, printer, filament
\end{itemize}

\subsubsection*{Step-by-step Tasks}\label{docs__pandoc__latex__src__3dmake_foundation__lessons_3dmake_2__bonus-print.md__step-by-step-tasks}

\begin{enumerate}
\tightlist
\item
  Choose a model from Thingiverse or Printables and note the original dimensions.
\item
  Decide on a purposeful modification (scale, add mounting holes, combine parts) and explain why.
\item
  Apply changes in OpenSCAD (or by scaling in the slicer) and record the new dimensions.
\item
  Slice and print all parts in one session when possible; log print times and filament used.
\item
  Create a short construction note and photograph the assembled result.
\end{enumerate}

\subsubsection*{Probing Questions}\label{docs__pandoc__latex__src__3dmake_foundation__lessons_3dmake_2__bonus-print.md__probing-questions}

\begin{itemize}
\tightlist
\item
  What motivated the modification and who benefits from it?
\item
  How did scaling affect tolerances or assembly fit?
\end{itemize}

\subsubsection*{Quiz - Bonus Print (10 questions)}\label{docs__pandoc__latex__src__3dmake_foundation__lessons_3dmake_2__bonus-print.md__quiz---bonus-print-10-questions}

\begin{enumerate}
\tightlist
\item
  What is one risk when scaling a model up or down? (short answer)
\item
  Name two checks to perform before printing multi-part models. (short answer)
\item
  How should you record filament usage? (short answer)
\item
  Why document construction steps? (one sentence)
\item
  What is one visual sign a part needs more infill? (one sentence)
\item
  True/False: Scaling a model uniformly (proportionally) in all three dimensions will preserve the original fit tolerances perfectly. (Answer: False - because tolerance stack-up and printer behavior can change)
\item
  Short answer: Explain the difference between scaling a model in the slicer versus modifying the OpenSCAD code to scale a design. Which approach is more reproducible?
\item
  Practical scenario: You want to scale a model from 10 cm to 25 cm (2.5x scale). What should you check regarding print time and support material before committing to the print?
\item
  Multiple choice: When assembling multi-part prints, what should you test first? (A) The final assembly (B) Individual part dimensions, then pairwise assembly (C) Skip testing - Answer: B
\item
  Reflection: Describe how documenting your modifications (scaling factor, OpenSCAD code changes, filament type, print settings) enables other students to reproduce your design and iterate on it further.
\end{enumerate}

\subsubsection*{Extension Problems (10)}\label{docs__pandoc__latex__src__3dmake_foundation__lessons_3dmake_2__bonus-print.md__extension-problems-10}

\begin{enumerate}
\tightlist
\item
  Create a small assembly guide with tactile cues for non-visual users.
\item
  Produce two scaled variants and compare required print time and fit.
\item
  Modify a part to include snap-fit connectors and document fit tolerances.
\item
  Add simple labeling to parts using embossed text in OpenSCAD.
\item
  Publish your variant and short build notes to the class repo and review two peers\textquotesingle{} submissions.
\item
  Build a complete variant library: create 5+ variations of your bonus print; document parameters, reasoning, and use cases.
\item
  Design a variant optimization process: compare variants by cost, print time, quality, and functionality; justify your "best" choice.
\item
  Create a parametric master file that generates all variants automatically; test parameter ranges and edge cases.
\item
  Develop a variant documentation and sharing system: create a portfolio with photos, specs, and instructions for each variant.
\item
  Write a "remix and iterate" guide: explain how future students can modify your designs, what parameters they should change, and how to test improvements.
\end{enumerate}

\subsection{Bonus Print - Student Documentation Template (Extension Project)}\label{docs__pandoc__latex__src__3dmake_foundation__lessons_3dmake_2__bonus_print_student_template.md__3dmake_foundation_lessons_3dmake_2-bonus_print_student_template}

\begin{itemize}
\tightlist
\item
  Author:
\item
  Date:
\item
  Description: Create scaled variants of a design to explore parametric thinking and manufacturing trade-offs.
\end{itemize}

\subsubsection*{Design Concept}\label{docs__pandoc__latex__src__3dmake_foundation__lessons_3dmake_2__bonus_print_student_template.md__design-concept}

\begin{itemize}
\tightlist
\item
  Base design name and description:
\item
  Why did you choose this design?
\item
  How will you generate variants parametrically?
\end{itemize}

\subsubsection*{Variant Specifications}\label{docs__pandoc__latex__src__3dmake_foundation__lessons_3dmake_2__bonus_print_student_template.md__variant-specifications}

{\def\LTcaptype{none} % do not increment counter
\begin{longtable}[]{@{}
  >{\raggedright\arraybackslash}p{(\linewidth - 8\tabcolsep) * \real{0.1324}}
  >{\raggedright\arraybackslash}p{(\linewidth - 8\tabcolsep) * \real{0.1029}}
  >{\raggedright\arraybackslash}p{(\linewidth - 8\tabcolsep) * \real{0.2794}}
  >{\raggedright\arraybackslash}p{(\linewidth - 8\tabcolsep) * \real{0.2500}}
  >{\raggedright\arraybackslash}p{(\linewidth - 8\tabcolsep) * \real{0.2353}}@{}}
\toprule\noalign{}
\begin{minipage}[b]{\linewidth}\raggedright
Variant
\end{minipage} & \begin{minipage}[b]{\linewidth}\raggedright
Scale
\end{minipage} & \begin{minipage}[b]{\linewidth}\raggedright
Print Time (est.)
\end{minipage} & \begin{minipage}[b]{\linewidth}\raggedright
Filament (est.)
\end{minipage} & \begin{minipage}[b]{\linewidth}\raggedright
Key Dimensions
\end{minipage} \\
\midrule\noalign{}
\endhead
\bottomrule\noalign{}
\endlastfoot
v1 & & & & \\
v2 & & & & \\
v3 & & & & \\
\end{longtable}
}

\subsubsection*{Print Results}\label{docs__pandoc__latex__src__3dmake_foundation__lessons_3dmake_2__bonus_print_student_template.md__print-results}

{\def\LTcaptype{none} % do not increment counter
\begin{longtable}[]{@{}lllll@{}}
\toprule\noalign{}
Variant & Actual Print Time & Actual Filament & Quality & Notes \\
\midrule\noalign{}
\endhead
\bottomrule\noalign{}
\endlastfoot
v1 & & & & \\
v2 & & & & \\
v3 & & & & \\
\end{longtable}
}

\subsubsection*{Analysis and Reflections}\label{docs__pandoc__latex__src__3dmake_foundation__lessons_3dmake_2__bonus_print_student_template.md__analysis-and-reflections}

\begin{itemize}
\tightlist
\item
  Which variant is most efficient (time/material ratio)?
\item
  How did scaling affect print quality?
\item
  What trade-offs did you observe (speed vs. quality, material vs. time)?
\item
  If you made a fourth variant, what would it be and why?
\end{itemize}

\subsubsection*{Attachments}\label{docs__pandoc__latex__src__3dmake_foundation__lessons_3dmake_2__bonus_print_student_template.md__attachments}

\begin{itemize}
\tightlist
\item[$\square$]
  \texttt{.scad} file with variant generation code
\item[$\square$]
  Photos of all variants (assembled if applicable)
\item[$\square$]
  Print log for each variant
\item[$\square$]
  Comparison analysis
\end{itemize}

\subsubsection*{Teacher Feedback}\label{docs__pandoc__latex__src__3dmake_foundation__lessons_3dmake_2__bonus_print_student_template.md__teacher-feedback}

{\def\LTcaptype{none} % do not increment counter
\begin{longtable}[]{@{}lll@{}}
\toprule\noalign{}
Category & Score & Notes \\
\midrule\noalign{}
\endhead
\bottomrule\noalign{}
\endlastfoot
Problem \& Solution (0-3) & & \\
Design \& Code Quality (0-3) & & \\
Documentation (0-3) & & \\
Total (0-9) & & \\
\end{longtable}
}

\subsection{Bonus Print - Teacher Template (Extension Project)}\label{docs__pandoc__latex__src__3dmake_foundation__lessons_3dmake_2__bonus_print_teacher_template.md__3dmake_foundation_lessons_3dmake_2-bonus_print_teacher_template}

\subsubsection*{Briefing}\label{docs__pandoc__latex__src__3dmake_foundation__lessons_3dmake_2__bonus_print_teacher_template.md__briefing}

Students design scaled variants of a 3D-printable object to explore parametric design and manufacturing trade-offs. This project emphasizes variant generation, documentation, and the relationship between design parameters and print outcomes.

Key Learning: Parametric scaling; variant generation; trade-off analysis.

Real-world Connection: Manufacturing efficiency depends on understanding how parameter changes affect production time, material use, and functionality.

\subsubsection*{Constraints}\label{docs__pandoc__latex__src__3dmake_foundation__lessons_3dmake_2__bonus_print_teacher_template.md__constraints}

\begin{itemize}
\tightlist
\item
  Must generate at least 3 scaled variants of a single base design
\item
  All variants must be parametric (variables clearly documented)
\item
  Student must document print parameters for each variant
\item
  Variants must be compared on measurable criteria (time, material, fit, etc.)
\end{itemize}

\subsubsection*{Functional Requirements}\label{docs__pandoc__latex__src__3dmake_foundation__lessons_3dmake_2__bonus_print_teacher_template.md__functional-requirements}

\begin{itemize}
\tightlist
\item
  Variants are generated parametrically, not manually resized in slicer
\item
  Each variant prints successfully
\item
  Comparison metrics recorded for all variants
\item
  Code is well-commented and organized
\end{itemize}

\subsubsection*{Deliverables}\label{docs__pandoc__latex__src__3dmake_foundation__lessons_3dmake_2__bonus_print_teacher_template.md__deliverables}

\begin{itemize}
\tightlist
\item
  Parametric \texttt{.scad} with variant generation code
\item
  Completed documentation template with variant specifications
\item
  Print log for all variants
\item
  Comparison matrix (time, material, quality, cost)
\item
  Photos of all variants
\item
  Reflection on design scalability
\end{itemize}

\subsubsection*{Rubric}\label{docs__pandoc__latex__src__3dmake_foundation__lessons_3dmake_2__bonus_print_teacher_template.md__rubric}

(Same three-category 0-9 scale as other projects)

\paragraph*{Category 1: Problem \& Solution (0-3)}\label{docs__pandoc__latex__src__3dmake_foundation__lessons_3dmake_2__bonus_print_teacher_template.md__category-1-problem--solution-0-3}

All variants print successfully and meet comparison criteria.

\paragraph*{Category 2: Design \& Code Quality (0-3)}\label{docs__pandoc__latex__src__3dmake_foundation__lessons_3dmake_2__bonus_print_teacher_template.md__category-2-design--code-quality-0-3}

Code is parametric and well-commented. Variants show thoughtful design thinking.

\paragraph*{Category 3: Documentation (0-3)}\label{docs__pandoc__latex__src__3dmake_foundation__lessons_3dmake_2__bonus_print_teacher_template.md__category-3-documentation-0-3}

Comparison matrix complete. Reflection specific and insightful.

\subsubsection*{Assessment Notes}\label{docs__pandoc__latex__src__3dmake_foundation__lessons_3dmake_2__bonus_print_teacher_template.md__assessment-notes}

\begin{itemize}
\tightlist
\item
  Strong submissions: Show parametric structure, complete comparison matrix, and insightful reflection on scaling trade-offs
\item
  Reinforce: How parametric thinking enables rapid variant generation
\item
  Extension: Cost-benefit analysis; material property comparisons
\end{itemize}

\section{Lesson 3: Parametric Architecture and Modular Libraries}\label{docs__pandoc__latex__src__3dmake_foundation__lessons_3dmake_3__lessons_3dmake_3.md__lesson-3-parametric-architecture-and-modular-libraries}

Estimated time: 90--120 minutes

\subsection*{Learning Objectives}\label{docs__pandoc__latex__src__3dmake_foundation__lessons_3dmake_3__lessons_3dmake_3.md__learning-objectives}

\begin{itemize}
\tightlist
\item
  Understand and apply the DRY (Don\textquotesingle t Repeat Yourself) principle through modules and functions
\item
  Use \texttt{include} vs \texttt{use} for library files
\item
  Apply \texttt{for} loops, conditionals, and list comprehensions
\item
  Write type-safe modules using \texttt{is\_number()}, \texttt{is\_list()}, and \texttt{is\_string()}
\item
  Build a reusable module library
\end{itemize}

\subsection*{Materials}\label{docs__pandoc__latex__src__3dmake_foundation__lessons_3dmake_3__lessons_3dmake_3.md__materials}

\begin{itemize}
\tightlist
\item
  3dMake project from Lessons 1--2
\item
  Terminal and editor
\end{itemize}

\subsection*{Step-by-step Tasks}\label{docs__pandoc__latex__src__3dmake_foundation__lessons_3dmake_3__lessons_3dmake_3.md__step-by-step-tasks}

\subsubsection*{1. Define a Reusable Module}\label{docs__pandoc__latex__src__3dmake_foundation__lessons_3dmake_3__lessons_3dmake_3.md__1-define-a-reusable-module}

Modules create named geometry. They accept parameters and can be called anywhere in the file. See \footnote{OpenSCAD User Manual --- Modules - \url{https://en.wikibooks.org/wiki/OpenSCAD_User_Manual/User-Defined_Functions_and_Modules}}.

\begin{lstlisting}[style=Alabaster, language=openscad]
// Modules create geometry; calling them produces shapes
module rounded_box(w, d, h, r=2) {
  // r is the corner rounding radius, defaulting to 2mm
  minkowski() {
    cube([w - 2*r, d - 2*r, h], center=true);
    cylinder(r=r, h=0.01, $fn=32);
  }
}

// Call the module with different sizes
rounded_box(50, 30, 20);
translate([60, 0, 0]) rounded_box(40, 40, 15, r=5);

\end{lstlisting}

\subsubsection*{2. Write a Function (Computes a Value, No Geometry)}\label{docs__pandoc__latex__src__3dmake_foundation__lessons_3dmake_3__lessons_3dmake_3.md__2-write-a-function-computes-a-value-no-geometry}

\begin{lstlisting}[style=Alabaster, language=openscad]
// Functions return values, never geometry
function clearance_fit(nominal) = nominal + 0.2;
function tight_fit(nominal)    = nominal - 0.1;

hole_diameter = clearance_fit(8);   // = 8.2
peg_diameter  = tight_fit(8);       // = 7.9

cylinder(r=hole_diameter/2, h=10, $fn=64);

\end{lstlisting}

\subsubsection*{3. Use include and use for Libraries}\label{docs__pandoc__latex__src__3dmake_foundation__lessons_3dmake_3__lessons_3dmake_3.md__3-use-include-and-use-for-libraries}

\begin{lstlisting}[style=Alabaster, language=openscad]
// include: executes the file (modules, functions, AND top-level geometry)
include <BOSL2/std.scad>

// use: imports modules and functions only — no top-level geometry executes
use <my_library.scad>

\end{lstlisting}

Key distinction: \texttt{include} will also render any top-level geometry in the library file; \texttt{use} imports only the module and function definitions.
For more on \texttt{include} and \texttt{use}, see \footnote{OpenSCAD User Manual --- Include and Use - \url{https://en.wikibooks.org/wiki/OpenSCAD_User_Manual/Include_Statement}}.

\subsubsection*{4. Apply For Loops and Conditionals}\label{docs__pandoc__latex__src__3dmake_foundation__lessons_3dmake_3__lessons_3dmake_3.md__4-apply-for-loops-and-conditionals}

For loop and conditional syntax is described in \footnote{OpenSCAD User Manual --- For Loops and Conditionals - \url{https://en.wikibooks.org/wiki/OpenSCAD_User_Manual/The_OpenSCAD_Language\#for_loop}}.

\begin{lstlisting}[style=Alabaster, language=openscad]
// Array of peg positions
peg_positions = [[0,0], [20,0], [40,0], [20,20]];

for (pos = peg_positions) {
  translate([pos[0], pos[1], 0]) cylinder(r=3, h=10, $fn=32);
}

// Conditional: only add a lid if show_lid is true
show_lid = true;
if (show_lid) {
  translate([0, 0, 30]) cube([50, 30, 3]);
}

// For loop with range [start:step:end]
for (i = [0 : 5 : 45]) {
  rotate([0, 0, i]) translate([20, 0, 0]) cylinder(r=2, h=5, $fn=16);
}

\end{lstlisting}

\subsubsection*{5. Use Type Testing and List Comprehensions}\label{docs__pandoc__latex__src__3dmake_foundation__lessons_3dmake_3__lessons_3dmake_3.md__5-use-type-testing-and-list-comprehensions}

The BOSL2 library provides many advanced parametric modules and functions \footnote{BOSL2 Library Documentation - \url{https://github.com/BelfrySCAD/BOSL2/wiki}}.

\begin{lstlisting}[style=Alabaster, language=openscad]
// Type testing prevents hard-to-debug errors
module validated_cylinder(r, h) {
  if (!is_number(r)) echo("WARNING: r must be a number");
  if (!is_number(h)) echo("WARNING: h must be a number");
  cylinder(r=r, h=h, $fn=64);
}

// List comprehension: create a list of computed values
angles = [for (i = [0:10:350]) i];
echo(angles);  // [0, 10, 20, ..., 350]

// Use list comprehension to build geometry positions
positions = [for (i=[0:3]) i * 25];
for (x = positions) {
  translate([x, 0, 0]) cube([20, 20, 5]);
}

\end{lstlisting}

\subsubsection*{Checkpoint}\label{docs__pandoc__latex__src__3dmake_foundation__lessons_3dmake_3__lessons_3dmake_3.md__checkpoint}

\begin{itemize}
\tightlist
\item
  After step 1: calling \texttt{rounded\_box(50,\ 30,\ 20)} builds a rounded box. Verify in preview.
\item
  After step 3: \texttt{include} vs \texttt{use} behavior should be tested by placing a top-level \texttt{cube()} in a library file and using both import methods.
\item
  After step 5: \texttt{echo()} output appears in the OpenSCAD console window.
\end{itemize}

\subsection*{Building a Practical Library}\label{docs__pandoc__latex__src__3dmake_foundation__lessons_3dmake_3__lessons_3dmake_3.md__building-a-practical-library}

A well-designed module library is the backbone of efficient parametric design. Here is a starter library structure:

\begin{lstlisting}[style=Alabaster, language=openscad]
// ============================================================
// fasteners.scad — Parametric fastener library
// Usage: use <fasteners.scad>
// ============================================================

// M3 clearance hole (ISO standard: 3.2mm diameter)
module m3_hole(depth=10) {
  cylinder(r=1.6, h=depth + 0.001, $fn=16);
}

// M4 clearance hole (ISO standard: 4.3mm diameter)
module m4_hole(depth=10) {
  cylinder(r=2.15, h=depth + 0.001, $fn=16);
}

// Countersink pocket for M3 flat-head screw
module m3_countersink(depth=3) {
  union() {
    cylinder(r=1.6, h=10, $fn=16);  // shaft
    cylinder(r1=3.2, r2=1.6, h=depth, $fn=32);  // conical seat
  }
}

// Hex nut trap (for embedding M3 hex nut)
module m3_nut_trap(extra_depth=0) {
  // M3 nut: 5.5mm across flats, 2.4mm thick
  cylinder(r=3.2, h=2.4 + extra_depth, $fn=6);
}

\end{lstlisting}

Using the library:

\begin{lstlisting}[style=Alabaster, language=openscad]
use <fasteners.scad>

difference() {
  cube([40, 30, 10]);
  translate([10, 15, -0.001]) m3_hole(depth=10.002);
  translate([30, 15, -0.001]) m3_hole(depth=10.002);
}

\end{lstlisting}

\subsection*{Quiz --- Lesson 3dMake.3 (15 questions)}\label{docs__pandoc__latex__src__3dmake_foundation__lessons_3dmake_3__lessons_3dmake_3.md__quiz--lesson-3dmake3-15-questions}

\begin{enumerate}
\tightlist
\item
  What is the DRY principle and why is it important in parametric design?
\item
  What is the difference between a module and a function in OpenSCAD?
\item
  What does \texttt{include\ \textless{}library.scad\textgreater{}} do differently from \texttt{use\ \textless{}library.scad\textgreater{}}?
\item
  Write a \texttt{for} loop that places a sphere every 30° around a circle of radius 20.
\item
  What does \texttt{is\_number()} return, and give one example of when to use it.
\item
  What does a list comprehension \texttt{{[}for\ (i={[}0:5:20{]})\ i*2{]}} evaluate to?
\item
  What does the \texttt{echo()} function do in OpenSCAD?
\item
  True or False: a function in OpenSCAD can create geometry.
\item
  How do you specify a default parameter value in an OpenSCAD module?
\item
  What does the \texttt{\$preview} special variable allow you to do in OpenSCAD?
\item
  Write an OpenSCAD \texttt{for} loop that creates 8 evenly spaced holes around a circle using \texttt{for\ (i\ =\ {[}0:45:315{]})}.
\item
  What is the difference between \texttt{{[}0:10{]}} and \texttt{{[}0:2:10{]}} in an OpenSCAD range expression?
\item
  Explain why you might use type testing (\texttt{is\_number()}, \texttt{is\_list()}) in a module rather than relying on the caller to pass correct types.
\item
  Describe how recursive functions work in OpenSCAD and give a simple example.
\item
  What would happen if you accidentally used \texttt{include} instead of \texttt{use} for a library that contains a top-level \texttt{cube({[}10,10,10{]});} statement?
\end{enumerate}

\subsection*{Extension Problems (15)}\label{docs__pandoc__latex__src__3dmake_foundation__lessons_3dmake_3__lessons_3dmake_3.md__extension-problems-15}

\begin{enumerate}
\tightlist
\item
  Create a \texttt{bolts.scad} library with at least M3, M4, and M5 clearance holes and nut traps; use it in a test plate.
\item
  Build a parametric pegboard using a \texttt{for} loop. Parameters: \texttt{cols}, \texttt{rows}, \texttt{spacing}, \texttt{peg\_r}, \texttt{peg\_h}.
\item
  Implement a recursive function that generates a Fibonacci sequence up to \texttt{n} terms and use it to space pegs.
\item
  Build a module that validates all its inputs using \texttt{assert()} and logs warnings with \texttt{echo()}; document when \texttt{assert()} halts rendering vs. \texttt{echo()}.
\item
  Create two library files---one for mechanical fasteners and one for artistic decorations---and include both in a combined project.
\item
  Design a parametric gear profile module. Parameters: \texttt{teeth}, \texttt{module\_size}, \texttt{thickness}. (You don\textquotesingle t need to compute true involute gear geometry --- a simplified triangular-tooth approximation is fine.)
\item
  Build a "shelf peg" system: a board with N parametrically positioned holes and matching pegs that press-fit into them.
\item
  Use a list comprehension to generate a spiral path of points and place a small cylinder at each point.
\item
  Extend the \texttt{fasteners.scad} library with threaded insert pockets for M3, M4, and M5 heat-set inserts (look up brass insert dimensions).
\item
  Write a module that draws a parametric text label with a surrounding bordered frame; parameters: \texttt{text\_str}, \texttt{font\_size}, \texttt{padding}, \texttt{border\_thickness}.
\item
  Build a parametric snap-fit lid for a rectangular box: the lid uses \texttt{for} loops to generate evenly-spaced snap clips, and all dimensions are parameters.
\item
  Write a module that takes a list of coordinate pairs as input and draws a connecting chain of cylinders between consecutive points.
\item
  Create a parametric honeycomb panel module using a \texttt{for} loop and \texttt{rotate()}; parameters: \texttt{rows}, \texttt{cols}, \texttt{cell\_r}, \texttt{wall\_t}, \texttt{depth}.
\item
  Research BOSL2\textquotesingle s \texttt{regular\_ngon()} and \texttt{bezier\_path()} modules. Build a shape that uses both and document what each does.
\item
  Design a test suite: build 10 small parametric shapes, each testing a different OpenSCAD feature (extrusion, rotation, for loop, intersection, etc.), and export them all from one main file.
\end{enumerate}

\subsection*{References and Helpful Resources}\label{docs__pandoc__latex__src__3dmake_foundation__lessons_3dmake_3__lessons_3dmake_3.md__references-and-helpful-resources}

\subsubsection*{Supplemental Resources}\label{docs__pandoc__latex__src__3dmake_foundation__lessons_3dmake_3__lessons_3dmake_3.md__supplemental-resources}

\begin{itemize}
\tightlist
\item
  \href{docs/pandoc/latex/src/assets/Programming_with_OpenSCAD.epub}{Programming with OpenSCAD EPUB Textbook} --- Chapters on modules, functions, and libraries
\item
  \href{https://github.com/ProgrammingWithOpenSCAD/CodeSolutions}{CodeSolutions Repository} --- Worked examples for parametric architecture
\item
  \href{https://github.com/BelfrySCAD/BOSL2}{BOSL2 GitHub Repository} --- Comprehensive parametric library
\item
  \href{https://github.com/tdeck/3dmake}{3DMake GitHub Repository} --- Build workflow reference
\end{itemize}

\section{Lesson 4: AI-Enhanced Verification and Multimodal Feedback}\label{docs__pandoc__latex__src__3dmake_foundation__lessons_3dmake_4__lessons_3dmake_4.md__lesson-4-ai-enhanced-verification-and-multimodal-feedback}

Estimated time: 90--120 minutes

\subsection*{Learning Objectives}\label{docs__pandoc__latex__src__3dmake_foundation__lessons_3dmake_4__lessons_3dmake_4.md__learning-objectives}

\begin{itemize}
\tightlist
\item
  Use \texttt{3dm\ describe} to generate AI-assisted descriptions of STL geometry
\item
  Understand deterministic vs. AI-assisted validation
\item
  Apply prompt engineering techniques for design feedback
\item
  Recognize when to trust, question, or reject AI suggestions
\item
  Understand data privacy considerations when using cloud AI tools
\end{itemize}

\subsection*{Materials}\label{docs__pandoc__latex__src__3dmake_foundation__lessons_3dmake_4__lessons_3dmake_4.md__materials}

\begin{itemize}
\tightlist
\item
  3dMake project with a completed STL from Lessons 1--3
\item
  Terminal
\item
  Optional: AI chat interface (Claude, ChatGPT, or similar)
\end{itemize}

\subsection*{Step-by-step Tasks}\label{docs__pandoc__latex__src__3dmake_foundation__lessons_3dmake_4__lessons_3dmake_4.md__step-by-step-tasks}

\subsubsection*{1. Run 3dm describe on Your Current Model}\label{docs__pandoc__latex__src__3dmake_foundation__lessons_3dmake_4__lessons_3dmake_4.md__1-run-3dm-describe-on-your-current-model}

\texttt{3dm\ describe} sends your STL to an AI service and returns a natural-language description of the geometry. This is especially useful for non-visual validation.

\begin{lstlisting}[style=Alabaster, language=bash]
3dm describe

\end{lstlisting}

Expected output (example):

\begin{lstlisting}[style=Alabaster]
Model: build/main.stl
The model appears to be a rectangular box approximately 50mm x 30mm x 5mm.
It has uniform wall thickness and no visible holes or undercuts.
Suggested print orientation: flat on the largest face.

\end{lstlisting}

Review the description. Does it match your design intent? Note any discrepancies.

\subsubsection*{2. Understand Deterministic vs. AI Validation}\label{docs__pandoc__latex__src__3dmake_foundation__lessons_3dmake_4__lessons_3dmake_4.md__2-understand-deterministic-vs-ai-validation}

{\def\LTcaptype{none} % do not increment counter
\begin{longtable}[]{@{}
  >{\raggedright\arraybackslash}p{(\linewidth - 4\tabcolsep) * \real{0.3333}}
  >{\raggedright\arraybackslash}p{(\linewidth - 4\tabcolsep) * \real{0.3333}}
  >{\raggedright\arraybackslash}p{(\linewidth - 4\tabcolsep) * \real{0.3333}}@{}}
\toprule\noalign{}
\begin{minipage}[b]{\linewidth}\raggedright
Validation Type
\end{minipage} & \begin{minipage}[b]{\linewidth}\raggedright
What It Checks
\end{minipage} & \begin{minipage}[b]{\linewidth}\raggedright
Reliability
\end{minipage} \\
\midrule\noalign{}
\endhead
\bottomrule\noalign{}
\endlastfoot
Slicer validation & Manifold geometry, wall thickness, overhang angles &
Deterministic --- same result every run \\
\texttt{3dm\ describe} (AI) &
Human-readable shape description, print orientation suggestions &
Non-deterministic --- may vary slightly between runs \\
Manual calipers & Printed part dimensions & Ground truth \\
\end{longtable}
}

AI-assisted tools like \texttt{3dm\ describe} are useful for generating first-pass descriptions and spotting obvious issues, but they should not replace deterministic checks (slicer validation, geometry analysis).\footnote{OpenSCAD User Manual --- Manifold and Geometry Validation - \url{https://en.wikibooks.org/wiki/OpenSCAD_User_Manual/The_OpenSCAD_Language}}

\subsubsection*{3. Craft Effective Verification Prompts}\label{docs__pandoc__latex__src__3dmake_foundation__lessons_3dmake_4__lessons_3dmake_4.md__3-craft-effective-verification-prompts}

When using an AI assistant (Claude, GPT, etc.) to review OpenSCAD code, prompt quality determines output quality:

\textbf{Weak prompt:}

\begin{lstlisting}[style=Alabaster]
Review my OpenSCAD code.

\end{lstlisting}

\textbf{Strong prompt:}

\begin{lstlisting}[style=Alabaster]
I have an OpenSCAD model that creates a parametric mounting bracket with two M3 holes.
Parameters: width=40, height=30, depth=8, hole_r=3.2.
Please review this code for:
1. Correct use of the 0.001 offset rule in all difference() operations
2. Manifold-safe geometry (no co-planar faces)
3. Whether hole_r=3.2 is an appropriate M3 clearance hole diameter
4. Any parameters that could produce invalid geometry if set to extreme values
Respond with: issues found, severity (critical/warning/note), and suggested fix for each.

\end{lstlisting}

A structured prompt produces structured, actionable feedback.\footnote{Anthropic Prompt Engineering Guide --- \url{https://docs.anthropic.com/en/docs/build-with-claude/prompt-engineering/overview}. Covers structured prompting techniques applicable to code review use cases.}

\subsubsection*{4. Validate AI Suggestions Before Applying Them}\label{docs__pandoc__latex__src__3dmake_foundation__lessons_3dmake_4__lessons_3dmake_4.md__4-validate-ai-suggestions-before-applying-them}

AI tools can produce plausible-sounding but incorrect recommendations. Always apply a verification step:

\begin{lstlisting}[style=Alabaster]
AI suggests: "Change hole_r to 3.0 for an M3 clearance hole"

Verification steps:
1. Check ISO standard: M3 clearance hole = 3.2mm (medium fit), 3.4mm (loose fit)
2. AI recommendation of 3.0mm would be an interference fit, not a clearance hole
3. Reject the suggestion; keep 3.2mm

Conclusion: AI was wrong. Apply human domain knowledge.

\end{lstlisting}

Document your accept/reject decisions in your project notes for future reference.

\subsubsection*{5. Apply Privacy and Data Considerations}\label{docs__pandoc__latex__src__3dmake_foundation__lessons_3dmake_4__lessons_3dmake_4.md__5-apply-privacy-and-data-considerations}

When using cloud AI tools:

\begin{itemize}
\tightlist
\item
  \textbf{Don\textquotesingle t upload proprietary designs} to free-tier AI services without checking their data retention policies
\item
  \textbf{Anonymize or simplify} your model before requesting AI review if IP is a concern
\item
  \texttt{3dm\ describe} uses the same AI backend as configured in your 3dMake settings --- check your instance\textquotesingle s privacy settings
\end{itemize}

For classroom use, \texttt{3dm\ describe} on non-sensitive learning projects is appropriate. For professional designs, consult your organization\textquotesingle s data policy.\footnote{Data Privacy in AI-Assisted Tools --- General guidance: review your AI provider\textquotesingle s terms of service and data retention policy before submitting proprietary or sensitive design files.}

\subsubsection*{Checkpoint}\label{docs__pandoc__latex__src__3dmake_foundation__lessons_3dmake_4__lessons_3dmake_4.md__checkpoint}

\begin{itemize}
\tightlist
\item
  After step 2: You can articulate the difference between deterministic and AI-based validation.
\item
  After step 3: Your AI prompt produces structured, reviewable output.
\item
  After step 4: You have documented at least one accept or reject decision for an AI suggestion.
\end{itemize}

\subsection*{Common AI Validation Patterns}\label{docs__pandoc__latex__src__3dmake_foundation__lessons_3dmake_4__lessons_3dmake_4.md__common-ai-validation-patterns}

\subsubsection*{Pattern 1: Geometry Sanity Check}\label{docs__pandoc__latex__src__3dmake_foundation__lessons_3dmake_4__lessons_3dmake_4.md__pattern-1-geometry-sanity-check}

\begin{lstlisting}[style=Alabaster, language=bash]
# Ask AI to describe what it "sees" in your model
3dm describe

# Then compare to your design checklist:
# - Correct overall dimensions? ✓/✗
# - Expected number of holes/features? ✓/✗
# - No unexpected geometry? ✓/✗
# - Print orientation makes sense? ✓/✗

\end{lstlisting}

\subsubsection*{Pattern 2: Code Review Prompt Template}\label{docs__pandoc__latex__src__3dmake_foundation__lessons_3dmake_4__lessons_3dmake_4.md__pattern-2-code-review-prompt-template}

Use this template when asking an AI to review OpenSCAD code:

\begin{lstlisting}[style=Alabaster]
Context: [describe what the part is and what it will be used for]
Code: [paste your OpenSCAD code]
Parameters: [list key parameter values]
Review for:
  - Boolean operation correctness (0.001 offsets where needed)
  - Parameter range safety (will extreme values break geometry?)
  - Printability (overhangs, thin walls, minimum feature size)
  - Code clarity (variable names, comments)
Output format: numbered list of issues, each with severity and fix

\end{lstlisting}

\subsubsection*{Pattern 3: Iterative Improvement Loop}\label{docs__pandoc__latex__src__3dmake_foundation__lessons_3dmake_4__lessons_3dmake_4.md__pattern-3-iterative-improvement-loop}

\begin{lstlisting}[style=Alabaster]
1. Write initial design
2. Run 3dm build → fix compile errors
3. Run slicer → fix geometry errors (deterministic)
4. Run 3dm describe or AI review → address suggestions (with verification)
5. Print test piece → compare to spec
6. Iterate on parameters if needed

\end{lstlisting}

\subsection*{Quiz --- Lesson 3dMake.4 (15 questions)}\label{docs__pandoc__latex__src__3dmake_foundation__lessons_3dmake_4__lessons_3dmake_4.md__quiz--lesson-3dmake4-15-questions}

\begin{enumerate}
\tightlist
\item
  What does \texttt{3dm\ describe} do?
\item
  What is the difference between deterministic validation and AI-based validation?
\item
  Give two elements of a strong AI code review prompt that a weak prompt lacks.
\item
  Why should you always verify AI suggestions before applying them? Give a specific example.
\item
  What type of geometry errors does slicer validation catch that AI description does not?
\item
  True or False: \texttt{3dm\ describe} will always produce identical output for the same STL file.
\item
  What data privacy concern should you consider when uploading a design to a cloud AI service?
\item
  In the iterative improvement loop described above, at which step should you run slicer validation?
\item
  What does it mean for a model to be "non-deterministic" in the context of AI feedback?
\item
  Describe one scenario where AI validation would be useful and one where you must rely on deterministic tools instead.
\item
  What is "prompt engineering" and why does it matter when using AI for code review?
\item
  If an AI tool suggests a geometry change that contradicts an ISO standard, what should you do?
\item
  What is the NIST AI Risk Management Framework, and in what professional context would it be relevant to AI-assisted design validation?\footnote{NIST AI Risk Management Framework (AI RMF 1.0) --- National Institute of Standards and Technology, 2023. \url{https://www.nist.gov/system/files/documents/2023/01/26/AI\%20RMF\%201.0.pdf}. Provides a structured framework for managing risks in AI-integrated workflows.}
\item
  Describe two ways you could reduce IP exposure when using a cloud AI tool to review a proprietary design.
\item
  In the validation table (deterministic vs. AI vs. calipers), which method provides "ground truth" and why?
\end{enumerate}

\subsection*{Extension Problems (15)}\label{docs__pandoc__latex__src__3dmake_foundation__lessons_3dmake_4__lessons_3dmake_4.md__extension-problems-15}

\begin{enumerate}
\tightlist
\item
  Design a validation checklist for a new 3D printed part. Include at least five items that are deterministic and three that benefit from AI assistance.
\item
  Build a "prompt library" for three common design review scenarios: mounting brackets, enclosures, and text embossing. Test each prompt and document the output quality.
\item
  Run \texttt{3dm\ describe} on the same STL five times. Document any variation in output. What does this tell you about reproducibility?
\item
  Create a structured accept/reject log for one full design cycle (design → AI review → verification → decision). Share the log with a classmate for peer review.
\item
  Research what "hallucination" means in the context of AI language models. Document two examples of how an AI might hallucinate in a design review context.
\item
  Design an experiment to compare AI-suggested print orientation with slicer-suggested orientation for 10 different models. Tabulate the agreement rate.
\item
  Write a one-page data privacy policy for an imaginary classroom 3D printing lab that uses cloud AI tools. Reference your school\textquotesingle s existing digital privacy guidelines.
\item
  Create a set of deliberately flawed OpenSCAD files (missing 0.001 offsets, wrong hole sizes, thin walls) and test whether AI review catches all the flaws.
\item
  Build a "verification script" using \texttt{3dm\ build} and slicer command-line tools that automates the deterministic validation steps in the iterative loop.
\item
  Interview a classmate using the stakeholder interview technique from Lesson 11 to understand their mental model of AI validation. What assumptions do they have that might be incorrect?
\item
  Research the difference between AI classification tasks (is this a manifold mesh?) and AI generation tasks (describe this mesh). Document why classification is more reliable for geometric validation.
\item
  Build a comparison table: for each of five design features (holes, thin walls, overhangs, text, interlocking parts), list whether AI review, slicer validation, or physical test is most reliable.
\item
  Write a "red teaming" exercise: intentionally try to get an AI tool to give you incorrect geometry advice. Document what prompts succeeded and what that means for how you use AI in design work.
\item
  Create a lesson plan (5 minutes) explaining AI validation concepts to a fellow student who has never used an AI tool. Focus on the deterministic vs. non-deterministic distinction.
\item
  Research and document the difference between "AI-assisted" and "AI-automated" workflows. At what point does removing the human from the validation loop become risky?
\end{enumerate}

\subsection*{References and Helpful Resources}\label{docs__pandoc__latex__src__3dmake_foundation__lessons_3dmake_4__lessons_3dmake_4.md__references-and-helpful-resources}

\subsubsection*{Supplemental Resources}\label{docs__pandoc__latex__src__3dmake_foundation__lessons_3dmake_4__lessons_3dmake_4.md__supplemental-resources}

\begin{itemize}
\tightlist
\item
  \href{https://github.com/tdeck/3dmake}{3DMake GitHub Repository} --- Source for \texttt{3dm\ describe} and other commands
\item
  \href{https://docs.anthropic.com/en/docs/build-with-claude/prompt-engineering/overview}{Anthropic Prompt Engineering Overview} --- Prompting best practices
\item
  \href{https://airc.nist.gov/Home}{NIST AI RMF} --- AI risk management in professional contexts
\item
  \href{https://docs.prusa3d.com/en/}{PrusaSlicer Documentation} --- Deterministic slicer validation reference
\end{itemize}

\subsection{Dice Design Workshop - Guided Extension}\label{docs__pandoc__latex__src__3dmake_foundation__lessons_3dmake_4__dice_dice_dice.md__3dmake_foundation_lessons_3dmake_4-dice_dice_dice}

Estimated time: 3-6 hours

\subsubsection*{Learning Objectives}\label{docs__pandoc__latex__src__3dmake_foundation__lessons_3dmake_4__dice_dice_dice.md__learning-objectives}

\begin{itemize}
\tightlist
\item
  Design parametric dice in OpenSCAD with controlled mass distribution
\item
  Test durability and randomness for small, thrown objects
\item
  Document design decisions and measure physical outcomes
\end{itemize}

\subsubsection*{Materials}\label{docs__pandoc__latex__src__3dmake_foundation__lessons_3dmake_4__dice_dice_dice.md__materials}

\begin{itemize}
\tightlist
\item
  Computer with OpenSCAD and slicer
\item
  Printer and filament, small testing surface
\end{itemize}

\subsubsection*{Step-by-step Tasks}\label{docs__pandoc__latex__src__3dmake_foundation__lessons_3dmake_4__dice_dice_dice.md__step-by-step-tasks}

\begin{enumerate}
\tightlist
\item
  Create three distinct die designs in OpenSCAD using parameters (size, face style, internal cavities).
\item
  For each die, explain which parameters you changed and why (one short paragraph each).
\item
  Print one sample of each die and perform a basic durability test (five throws onto a soft surface).
\item
  Record outcomes of each throw and compute a simple frequency table for face results.
\item
  Measure mass and critical dimensions; document any deformation or failure modes.
\end{enumerate}

\subsubsection*{Probing Questions}\label{docs__pandoc__latex__src__3dmake_foundation__lessons_3dmake_4__dice_dice_dice.md__probing-questions}

\begin{itemize}
\tightlist
\item
  How does internal infill or cavity affect mass distribution and roll randomness?
\item
  Which parameter had the largest effect on durability and why?
\end{itemize}

\subsubsection*{Quick Quiz (10 questions)}\label{docs__pandoc__latex__src__3dmake_foundation__lessons_3dmake_4__dice_dice_dice.md__quick-quiz-10-questions}

\begin{enumerate}
\tightlist
\item
  What parameter controls wall thickness in your \texttt{.scad} file? (short answer)
\item
  Name one method to increase a part\textquotesingle s impact resistance. (short answer)
\item
  How would you test randomness in a die? (short answer)
\item
  Why might a hollow die behave differently than a solid one? (one sentence)
\item
  What is a simple sign of print-layer delamination? (one sentence)
\item
  True/False: A perfectly fair die must have uniform weight distribution throughout, including the internal structure. (Answer: True)
\item
  Short answer: Describe two design approaches to a die that could be parametrically scaled - one using \texttt{cube()} with \texttt{difference()} for texture, one using \texttt{linear\_extrude()} for embossed numbers.
\item
  Practical scenario: Your first die printed, but when tested with 100 rolls, it landed on face \#1 significantly more often than other faces. What are two possible causes and how would you diagnose the problem?
\item
  Multiple choice: If you increase infill from 15\% to 50\%, what happens? (A) Dramatically longer print time and higher weight (B) Slightly longer print time, much higher weight (C) Same weight, different durability - Answer: B
\item
  Reflection: Explain why a fair die is an excellent test of design precision and iterative improvement. How does this project teach you about manufacturability, quality assurance, and design validation?
\end{enumerate}

\subsubsection*{Extension Problems (10)}\label{docs__pandoc__latex__src__3dmake_foundation__lessons_3dmake_4__dice_dice_dice.md__extension-problems-10}

\begin{enumerate}
\tightlist
\item
  Create a parametric script that produces dice with adjustable center-of-mass offsets and show results of roll tests.
\item
  Design a bead-style die that assembles from two printed halves and document assembly steps.
\item
  Compare two infill patterns for the same die and report on mass and durability differences.
\item
  Add tactile markers to faces for non-visual testing and document how they affect roll behavior.
\item
  Publish your \texttt{.scad} and test logs to the class repo and provide feedback on two classmates\textquotesingle{} designs.
\item
  Conduct a formal fairness test: roll your die 100+ times, track results, and perform statistical analysis to verify randomness.
\item
  Design multiple die variants (sizes, infills, materials); compare manufacturing feasibility, cost, and performance for each.
\item
  Create a parametric die library: build a system that generates dice of different sizes, materials, and properties on command.
\item
  Develop a comprehensive quality assurance protocol: define tolerances, measurement methods, and acceptance criteria for fair dice.
\item
  Write a die design guide for future students: explain the physics of fair dice, how to test for bias, and how to iterate on designs.
\end{enumerate}

\subsection{Dice, Dice, Dice - Student Documentation Template (Extension Project)}\label{docs__pandoc__latex__src__3dmake_foundation__lessons_3dmake_4__dice_dice_dice_student_template.md__3dmake_foundation_lessons_3dmake_4-dice_dice_dice_student_template}

\begin{itemize}
\tightlist
\item
  Author:
\item
  Date:
\item
  Description: Design parametric 3D-printed dice and test for fairness through statistical analysis.
\end{itemize}

\subsubsection*{Design Concept}\label{docs__pandoc__latex__src__3dmake_foundation__lessons_3dmake_4__dice_dice_dice_student_template.md__design-concept}

\begin{itemize}
\tightlist
\item
  Dice specifications (size, material, infill):
\item
  How will you ensure uniform weight distribution?
\item
  What design details will your dice include (pips, patterns)?
\end{itemize}

\subsubsection*{Design Specifications}\label{docs__pandoc__latex__src__3dmake_foundation__lessons_3dmake_4__dice_dice_dice_student_template.md__design-specifications}

{\def\LTcaptype{none} % do not increment counter
\begin{longtable}[]{@{}ll@{}}
\toprule\noalign{}
Parameter & Value \\
\midrule\noalign{}
\endhead
\bottomrule\noalign{}
\endlastfoot
Die size & \\
Infill \% & \\
Material & \\
Pip depth & \\
Pit/pip count per face & \\
\end{longtable}
}

\subsubsection*{Fairness Testing}\label{docs__pandoc__latex__src__3dmake_foundation__lessons_3dmake_4__dice_dice_dice_student_template.md__fairness-testing}

\paragraph*{Test Setup}\label{docs__pandoc__latex__src__3dmake_foundation__lessons_3dmake_4__dice_dice_dice_student_template.md__test-setup}

\begin{itemize}
\tightlist
\item
  How many rolls will you conduct? (minimum 100)
\item
  Rolling surface and method:
\item
  Date/time of testing:
\end{itemize}

\paragraph*{Roll Results}\label{docs__pandoc__latex__src__3dmake_foundation__lessons_3dmake_4__dice_dice_dice_student_template.md__roll-results}

{\def\LTcaptype{none} % do not increment counter
\begin{longtable}[]{@{}llll@{}}
\toprule\noalign{}
Face & Frequency (out of \_ rolls) & Percentage & Expected \% \\
\midrule\noalign{}
\endhead
\bottomrule\noalign{}
\endlastfoot
1 & & & 16.67\% \\
2 & & & 16.67\% \\
3 & & & 16.67\% \\
4 & & & 16.67\% \\
5 & & & 16.67\% \\
6 & & & 16.67\% \\
\end{longtable}
}

\paragraph*{Statistical Analysis}\label{docs__pandoc__latex__src__3dmake_foundation__lessons_3dmake_4__dice_dice_dice_student_template.md__statistical-analysis}

\begin{itemize}
\tightlist
\item
  Calculate chi-squared value or deviation from expected:
\item
  Is the die fair (within statistical tolerance)?
\item
  What face appears most/least frequently?
\end{itemize}

\subsubsection*{Reflections}\label{docs__pandoc__latex__src__3dmake_foundation__lessons_3dmake_4__dice_dice_dice_student_template.md__reflections}

\begin{itemize}
\tightlist
\item
  Were you surprised by the results?
\item
  What design factors influenced fairness?
\item
  How would you improve the dice design?
\item
  What did you learn about manufacturing precision?
\end{itemize}

\subsubsection*{Attachments}\label{docs__pandoc__latex__src__3dmake_foundation__lessons_3dmake_4__dice_dice_dice_student_template.md__attachments}

\begin{itemize}
\tightlist
\item[$\square$]
  \texttt{.scad} file with parametric dice module
\item[$\square$]
  Photo of test dice
\item[$\square$]
  Roll data (raw results or tally sheet)
\item[$\square$]
  Statistical analysis calculations
\item[$\square$]
  Design iteration notes (if multiple versions tested)
\end{itemize}

\subsubsection*{Teacher Feedback}\label{docs__pandoc__latex__src__3dmake_foundation__lessons_3dmake_4__dice_dice_dice_student_template.md__teacher-feedback}

{\def\LTcaptype{none} % do not increment counter
\begin{longtable}[]{@{}lll@{}}
\toprule\noalign{}
Category & Score & Notes \\
\midrule\noalign{}
\endhead
\bottomrule\noalign{}
\endlastfoot
Problem \& Solution (0-3) & & \\
Design \& Code Quality (0-3) & & \\
Documentation (0-3) & & \\
Total (0-9) & & \\
\end{longtable}
}

\subsection{Dice, Dice, Dice - Teacher Template (Extension Project)}\label{docs__pandoc__latex__src__3dmake_foundation__lessons_3dmake_4__dice_dice_dice_teacher_template.md__3dmake_foundation_lessons_3dmake_4-dice_dice_dice_teacher_template}

\subsubsection*{Briefing}\label{docs__pandoc__latex__src__3dmake_foundation__lessons_3dmake_4__dice_dice_dice_teacher_template.md__briefing}

Students design and test fair 3D-printed dice to explore geometry, symmetry, manufacturing precision, and statistical validation. This project combines physics, materials science, and rigorous testing practices.

Key Learning: Precision manufacturing; fairness validation; quality assurance.

Real-world Connection: Manufacturing quality assurance requires measurement, testing, and statistical analysis. Casino dice are one of the most rigorously tested manufactured objects.

\subsubsection*{Constraints}\label{docs__pandoc__latex__src__3dmake_foundation__lessons_3dmake_4__dice_dice_dice_teacher_template.md__constraints}

\begin{itemize}
\tightlist
\item
  Dice must be generated parametrically in OpenSCAD
\item
  Student must conduct fairness testing (100+ rolls minimum)
\item
  Design must address weight distribution and uniform geometry
\item
  Statistical analysis of test results required
\end{itemize}

\subsubsection*{Functional Requirements}\label{docs__pandoc__latex__src__3dmake_foundation__lessons_3dmake_4__dice_dice_dice_teacher_template.md__functional-requirements}

\begin{itemize}
\tightlist
\item
  Dice geometry is parametric (size, infill, details)
\item
  All faces print with usable pips/markers
\item
  Statistical testing conducted with documented results
\item
  Analysis shows understanding of fairness criteria
\end{itemize}

\subsubsection*{Deliverables}\label{docs__pandoc__latex__src__3dmake_foundation__lessons_3dmake_4__dice_dice_dice_teacher_template.md__deliverables}

\begin{itemize}
\tightlist
\item
  \texttt{.scad} file with parametric dice module
\item
  Completed documentation template with fairness test results
\item
  Statistical analysis (roll frequency data, calculations)
\item
  Photos of test dice and assembly
\item
  Reflection on design precision and manufacturing tolerances
\end{itemize}

\subsubsection*{Rubric}\label{docs__pandoc__latex__src__3dmake_foundation__lessons_3dmake_4__dice_dice_dice_teacher_template.md__rubric}

\paragraph*{Category 1: Problem \& Solution (0-3)}\label{docs__pandoc__latex__src__3dmake_foundation__lessons_3dmake_4__dice_dice_dice_teacher_template.md__category-1-problem--solution-0-3}

Dice print successfully and fairness testing conducted rigorously.

\paragraph*{Category 2: Design \& Code Quality (0-3)}\label{docs__pandoc__latex__src__3dmake_foundation__lessons_3dmake_4__dice_dice_dice_teacher_template.md__category-2-design--code-quality-0-3}

Code is parametric and thoughtful. Dice design shows understanding of symmetry and balance.

\paragraph*{Category 3: Documentation (0-3)}\label{docs__pandoc__latex__src__3dmake_foundation__lessons_3dmake_4__dice_dice_dice_teacher_template.md__category-3-documentation-0-3}

Testing data complete and organized. Statistical analysis accurate and insightful.

\subsubsection*{Assessment Notes}\label{docs__pandoc__latex__src__3dmake_foundation__lessons_3dmake_4__dice_dice_dice_teacher_template.md__assessment-notes}

\begin{itemize}
\tightlist
\item
  Strong submissions: Show rigorous fairness testing (100+ rolls), clear statistical analysis, and reflection on manufacturing precision
\item
  Reinforce: Why uniformity matters; how to test for bias
\item
  Extension: Material comparisons; cost-benefit analysis of infill vs. fairness
\end{itemize}

\section{Lesson 5: Safety Protocols and the Physical Fabrication Interface}\label{docs__pandoc__latex__src__3dmake_foundation__lessons_3dmake_5__lessons_3dmake_5.md__lesson-5-safety-protocols-and-the-physical-fabrication-interface}

Estimated time: 90--120 minutes

\subsection*{Learning Objectives}\label{docs__pandoc__latex__src__3dmake_foundation__lessons_3dmake_5__lessons_3dmake_5.md__learning-objectives}

\begin{itemize}
\tightlist
\item
  Identify and apply the Hierarchy of Controls for 3D printing hazards
\item
  Understand filament-specific hazards and safe material handling
\item
  Follow a safe printer startup, monitoring, and shutdown sequence
\item
  Design parts that are safe to print (minimize supports, overhangs, and VOC-intensive materials)
\end{itemize}

\subsection*{Materials}\label{docs__pandoc__latex__src__3dmake_foundation__lessons_3dmake_5__lessons_3dmake_5.md__materials}

\begin{itemize}
\tightlist
\item
  3D printer (FDM)
\item
  PLA, PETG, or TPU filament
\item
  Safety checklist (see assets folder)
\item
  Ventilation or enclosure (if using PETG, ABS, or ASA)
\end{itemize}

\subsection*{Step-by-step Tasks}\label{docs__pandoc__latex__src__3dmake_foundation__lessons_3dmake_5__lessons_3dmake_5.md__step-by-step-tasks}

\subsubsection*{1. Review the Hierarchy of Controls}\label{docs__pandoc__latex__src__3dmake_foundation__lessons_3dmake_5__lessons_3dmake_5.md__1-review-the-hierarchy-of-controls}

The Hierarchy of Controls is a standard safety framework used in occupational health to rank hazard mitigation strategies from most to least effective.\footnote{OSHA Hierarchy of Controls --- Occupational Safety and Health Administration. \url{https://www.osha.gov/hierarchy-of-controls}. Also see NIOSH\textquotesingle s explanation: \url{https://www.cdc.gov/niosh/topics/hierarchy/default.html}} Applied to 3D printing:

{\def\LTcaptype{none} % do not increment counter
\begin{longtable}[]{@{}
  >{\raggedright\arraybackslash}p{(\linewidth - 2\tabcolsep) * \real{0.5000}}
  >{\raggedright\arraybackslash}p{(\linewidth - 2\tabcolsep) * \real{0.5000}}@{}}
\toprule\noalign{}
\begin{minipage}[b]{\linewidth}\raggedright
Control Level
\end{minipage} & \begin{minipage}[b]{\linewidth}\raggedright
Example in 3D Printing
\end{minipage} \\
\midrule\noalign{}
\endhead
\bottomrule\noalign{}
\endlastfoot
\textbf{Elimination} &
Don\textquotesingle t use materials that produce hazardous fumes (e.g., choose PLA over ABS) \\
\textbf{Substitution} &
Use a low-VOC PETG instead of ABS for a part requiring heat resistance \\
\textbf{Engineering Controls} &
Install an enclosure with active filtration; use a HEPA filter \\
\textbf{Administrative Controls} &
Post material-specific operating procedures; require instructor sign-off \\
\textbf{PPE} &
Wear nitrile gloves when handling uncured resin; use respirator if enclosure is unavailable \\
\end{longtable}
}

The most effective controls reduce or eliminate the hazard at the source. PPE is the least effective because it relies on consistent human behavior.

\subsubsection*{2. Understand Filament-Specific Hazards}\label{docs__pandoc__latex__src__3dmake_foundation__lessons_3dmake_5__lessons_3dmake_5.md__2-understand-filament-specific-hazards}

{\def\LTcaptype{none} % do not increment counter
\begin{longtable}[]{@{}
  >{\raggedright\arraybackslash}p{(\linewidth - 8\tabcolsep) * \real{0.2000}}
  >{\raggedright\arraybackslash}p{(\linewidth - 8\tabcolsep) * \real{0.2000}}
  >{\raggedright\arraybackslash}p{(\linewidth - 8\tabcolsep) * \real{0.2000}}
  >{\raggedright\arraybackslash}p{(\linewidth - 8\tabcolsep) * \real{0.2000}}
  >{\raggedright\arraybackslash}p{(\linewidth - 8\tabcolsep) * \real{0.2000}}@{}}
\toprule\noalign{}
\begin{minipage}[b]{\linewidth}\raggedright
Filament
\end{minipage} & \begin{minipage}[b]{\linewidth}\raggedright
Print Temp (°C)
\end{minipage} & \begin{minipage}[b]{\linewidth}\raggedright
Bed Temp (°C)
\end{minipage} & \begin{minipage}[b]{\linewidth}\raggedright
Key Hazards
\end{minipage} & \begin{minipage}[b]{\linewidth}\raggedright
Ventilation Required?
\end{minipage} \\
\midrule\noalign{}
\endhead
\bottomrule\noalign{}
\endlastfoot
PLA & 200--215 & 0--60 & Minimal; slight sweet smell &
No (recommended) \\
PETG & 230--250 & 70--90 & Low VOC, slight odor & Yes (moderate) \\
TPU & 220--240 & 30--60 & Low VOC & Yes (moderate) \\
ABS & 230--260 & 100--110 & Styrene fumes (potentially harmful) &
Yes (required + enclosure) \\
ASA & 240--260 & 100--110 & Similar to ABS &
Yes (required + enclosure) \\
Resin (SLA) & N/A & N/A & Skin/eye irritant; VOCs & Required + gloves \\
\end{longtable}
}

PLA is the safest classroom filament. ABS and resin require active ventilation and enclosures.\footnote{Filament Safety Properties --- UL Research Institutes: Characterization of Particles and Gases from Common 3D Printing Filaments. \url{https://www.ul.com/news/ul-research-institutes-releases-3d-printing-emissions-study}}

\subsubsection*{3. Safe Printer Startup Sequence}\label{docs__pandoc__latex__src__3dmake_foundation__lessons_3dmake_5__lessons_3dmake_5.md__3-safe-printer-startup-sequence}

Follow this sequence every time you start a print:

\begin{enumerate}
\tightlist
\item
  \textbf{Visual inspection} --- No filament tangles; bed is clear; no debris on nozzle
\item
  \textbf{Bed leveling check} --- Run manual or automatic leveling if the printer has been moved
\item
  \textbf{Filament loaded and feeding} --- Manually extrude 10--20 mm to confirm clean flow
\item
  \textbf{Slicer preview} --- Verify layer preview shows expected geometry before sending G-code \footnote{PrusaSlicer Documentation --- Layer and Print Settings. \url{https://docs.prusa3d.com/en/}}
\item
  \textbf{First layer observation} --- Watch the first 2--3 layers to confirm adhesion
\item
  \textbf{Never leave unattended during the first layer} --- Most failures occur in the first 10 minutes
\end{enumerate}

\subsubsection*{4. Monitoring and Intervention}\label{docs__pandoc__latex__src__3dmake_foundation__lessons_3dmake_5__lessons_3dmake_5.md__4-monitoring-and-intervention}

During a print:

\begin{itemize}
\tightlist
\item
  Check in every 15--30 minutes for long prints
\item
  Watch for: spaghetti (filament not adhering), layer shifts, nozzle clogs (clicking extruder), or smoke
\item
  \textbf{Smoke = stop immediately}: turn off the printer, do not restart until the cause is identified
\item
  Use OctoPrint or built-in camera monitoring if available for remote observation
\end{itemize}

\subsubsection*{5. Safe Shutdown and Part Removal}\label{docs__pandoc__latex__src__3dmake_foundation__lessons_3dmake_5__lessons_3dmake_5.md__5-safe-shutdown-and-part-removal}

\begin{enumerate}
\tightlist
\item
  Wait until the print is complete and the nozzle has cooled to \textless{} 50°C
\item
  Wait until the bed has cooled to \textless{} 30°C before removing the part --- hot beds can warp thin parts
\item
  Use a spatula or palette knife to remove parts; never force them off with bare hands
\item
  Store filament in a sealed bag with desiccant to prevent moisture absorption
\end{enumerate}

\subsubsection*{6. Design for Safety}\label{docs__pandoc__latex__src__3dmake_foundation__lessons_3dmake_5__lessons_3dmake_5.md__6-design-for-safety}

Good design reduces printing risk:

\begin{itemize}
\tightlist
\item
  \textbf{Minimize supports} --- Parts that need \textgreater{} 50\% support coverage may jam the nozzle when support material tangles
\item
  \textbf{Stay within bed size} --- Parts larger than the print bed require splitting and bonding
\item
  \textbf{Avoid ABS when PLA or PETG will work} --- Select the safest material that meets mechanical requirements
\item
  \textbf{Print test coupons first} --- A small test piece (5\% of full size) catches problems before wasting a long print
\end{itemize}

\subsubsection*{Checkpoint}\label{docs__pandoc__latex__src__3dmake_foundation__lessons_3dmake_5__lessons_3dmake_5.md__checkpoint}

\begin{itemize}
\tightlist
\item
  After step 2: You can match each filament type to its required ventilation level.
\item
  After step 3: You have completed at least one printer startup using the six-step sequence.
\item
  After step 6: Your latest model has been evaluated for support minimization.
\end{itemize}

\subsection*{Printability-Focused Design Guidelines}\label{docs__pandoc__latex__src__3dmake_foundation__lessons_3dmake_5__lessons_3dmake_5.md__printability-focused-design-guidelines}

\begin{lstlisting}[style=Alabaster, language=openscad]
// Design with printability in mind from the start
// Rule: any overhang > 45 degrees from vertical needs supports
// Rule: minimum wall thickness for FDM = 0.8mm (2x nozzle width for 0.4mm nozzle)
// Rule: minimum hole diameter = 2mm (smaller holes may not print accurately)
// Rule: add 0.15-0.20mm clearance between mating parts

wall_min = 1.2;        // safe minimum for 0.4mm nozzle: 3x nozzle width
hole_min_r = 1.5;      // 3mm diameter minimum for reliable hole printing
clearance = 0.2;       // fitting clearance between mating faces

module printable_box(w, d, h) {
  wall = max(wall_min, w * 0.05);  // at least 5% of width, minimum 1.2mm
  difference() {
    cube([w, d, h]);
    translate([wall, wall, wall])
      cube([w - 2*wall, d - 2*wall, h]);  // open top
  }
}

printable_box(50, 40, 30);

\end{lstlisting}

\subsection*{Quiz --- Lesson 3dMake.5 (15 questions)}\label{docs__pandoc__latex__src__3dmake_foundation__lessons_3dmake_5__lessons_3dmake_5.md__quiz--lesson-3dmake5-15-questions}

\begin{enumerate}
\tightlist
\item
  What are the five levels of the Hierarchy of Controls, from most to least effective?
\item
  Why is Elimination considered the most effective control?
\item
  What filament requires an enclosure with active ventilation and is NOT recommended for open classrooms?
\item
  At what bed temperature should you wait before removing a print?
\item
  What should you do immediately if you see smoke from your 3D printer?
\item
  Why is PLA considered the safest classroom filament?
\item
  What is the minimum wall thickness for a 0.4 mm nozzle, and why does this minimum exist?
\item
  What is the purpose of desiccant when storing filament?
\item
  True or False: It is safe to leave a 3D printer unattended during the first layer.
\item
  What does "spaghetti" mean in the context of a 3D printing failure?
\item
  What is the OSHA Hierarchy of Controls, and how does it differ from simply requiring PPE?\footnote{OSHA Hierarchy of Controls --- Occupational Safety and Health Administration. \url{https://www.osha.gov/hierarchy-of-controls}. Also see NIOSH\textquotesingle s explanation: \url{https://www.cdc.gov/niosh/topics/hierarchy/default.html}}
\item
  What are the specific hazards associated with resin (SLA/MSLA) printing that differ from FDM?
\item
  What is the bed temperature recommendation for PETG, and why is this higher than PLA?
\item
  Explain why minimizing supports in your design is a safety consideration, not just a time-saving one.
\item
  Describe one engineering control and one administrative control you could implement in a classroom 3D printing lab.
\end{enumerate}

\subsection*{Extension Problems (15)}\label{docs__pandoc__latex__src__3dmake_foundation__lessons_3dmake_5__lessons_3dmake_5.md__extension-problems-15}

\begin{enumerate}
\tightlist
\item
  Create a material selection flowchart: given mechanical requirements (heat resistance, flexibility, strength), the chart guides the user to the safest filament that meets requirements.
\item
  Conduct a ventilation audit of your classroom or makerspace. Document air exchange rates, window proximity, and filter types. Write a one-page recommendation.
\item
  Write a standard operating procedure (SOP) for ABS printing in a classroom. Include pre-print checks, monitoring requirements, and shutdown procedures.
\item
  Design a "first layer test tile" that is 10 cm × 10 cm × 0.4 mm. Print it and assess adhesion quality. Document what you observe.
\item
  Build a parametric filament moisture indicator holder in OpenSCAD: a box that holds a humidity indicator card inside a filament storage bag.
\item
  Research OSHA\textquotesingle s published guidance on 3D printing VOC exposure. Summarize the key recommendations for occupational settings. Cite the specific OSHA or NIOSH publication.\footnote{OSHA Hierarchy of Controls --- Occupational Safety and Health Administration. \url{https://www.osha.gov/hierarchy-of-controls}. Also see NIOSH\textquotesingle s explanation: \url{https://www.cdc.gov/niosh/topics/hierarchy/default.html}}\footnote{NIOSH Science Blog --- Health and Safety Considerations for 3D Printing. \url{https://blogs.cdc.gov/niosh-science-blog/2020/05/14/3d-printing/}}
\item
  Compare the safety data sheets (SDS) for PLA and ABS filament from two manufacturers. Document the differences in recommended ventilation and exposure limits.
\item
  Design a printer enclosure in OpenSCAD. Key requirements: four walls, a door opening, a top panel with a vent hole. Make all dimensions parametric.
\item
  Create a classroom safety poster (on paper or digitally) covering the five most important 3D printing safety rules, using the Hierarchy of Controls as a framework.
\item
  Write a short (one-page) risk assessment for a new classroom printer purchase. Consider: filament type, ventilation, fire suppression, and student training.
\item
  Design and print a filament storage clip: a parametric clip that holds a loose end of filament to a spool. Make the clip diameter a parameter.
\item
  Build a "print monitoring checklist" as a paper form. Columns: time, nozzle temp, bed temp, layer number, observations, action taken.
\item
  Research the difference between particle emissions and volatile organic compound (VOC) emissions in FDM printing. Which is more hazardous at typical classroom distances? Cite your sources.\footnote{NIOSH Science Blog --- Health and Safety Considerations for 3D Printing. \url{https://blogs.cdc.gov/niosh-science-blog/2020/05/14/3d-printing/}}
\item
  Design a parametric spatula guard in OpenSCAD: a thin safety bumper that clips to the edge of a build plate to prevent the spatula from slipping. Make the plate thickness a parameter.
\item
  Write a "near-miss" incident report for a hypothetical 3D printing incident (e.g., a student touched a hot nozzle). Use a standard incident report format: what happened, contributing factors, corrective actions.
\end{enumerate}

\subsection*{References and Helpful Resources}\label{docs__pandoc__latex__src__3dmake_foundation__lessons_3dmake_5__lessons_3dmake_5.md__references-and-helpful-resources}

\subsubsection*{Supplemental Resources}\label{docs__pandoc__latex__src__3dmake_foundation__lessons_3dmake_5__lessons_3dmake_5.md__supplemental-resources}

\begin{itemize}
\tightlist
\item
  \href{https://github.com/tdeck/3dmake}{3DMake GitHub Repository} --- Build workflow reference
\item
  \href{https://en.wikibooks.org/wiki/OpenSCAD_User_Manual}{OpenSCAD User Manual} --- Parametric design for printable parts
\item
  \href{https://github.com/mrhunsaker/VI_3DMake_OpenSCAD_Curriculum/3dMake_Foundation/Lessons_3dMake_5/../../assets/3dMake_Foundation/safety_checklist.md}{Safety Checklist} --- Classroom printer safety checklist asset
\item
  \href{https://github.com/mrhunsaker/VI_3DMake_OpenSCAD_Curriculum/3dMake_Foundation/Lessons_3dMake_5/../../assets/3dMake_Foundation/filament-comparison-table.md}{Filament Comparison Table} --- Filament properties reference
\end{itemize}

\subsection{Safety Checklist for 3D Printing}\label{docs__pandoc__latex__src__3dmake_foundation__lessons_3dmake_5__safety_checklist.md__3dmake_foundation_lessons_3dmake_5-safety_checklist}

Complete this checklist before each printing session to maintain a safe workspace.

\subsubsection*{Pre-Print Setup}\label{docs__pandoc__latex__src__3dmake_foundation__lessons_3dmake_5__safety_checklist.md__pre-print-setup}

\begin{itemize}
\item[$\square$]
  Work Area Clear

  \begin{itemize}
  \tightlist
  \item
    Clear desk/table of unnecessary items
  \item
    Remove tripping hazards around printer
  \item
    Ensure adequate ventilation around the printer
  \end{itemize}
\item[$\square$]
  Printer Inspection

  \begin{itemize}
  \tightlist
  \item
    Check nozzle is clean (no old filament residue)
  \item
    Verify build plate is level
  \item
    Confirm heated bed temperature sensor attached
  \item
    Check all cables are secure and not frayed
  \end{itemize}
\item[$\square$]
  Filament Preparation

  \begin{itemize}
  \tightlist
  \item
    Filament spool rotates freely (no binding)
  \item
    Filament path clear of obstructions
  \item
    Extruder drive gear clean (not clogged with plastic)
  \item
    New filament loaded correctly
  \end{itemize}
\item[$\square$]
  Environmental Conditions

  \begin{itemize}
  \tightlist
  \item
    Room temperature adequate (18-25C ideal)
  \item
    No direct drafts from windows/AC on printer
  \item
    Humidity levels reasonable (not extremely dry)
  \item
    Adequate lighting to see printer clearly
  \end{itemize}
\end{itemize}

\subsubsection*{During Print}\label{docs__pandoc__latex__src__3dmake_foundation__lessons_3dmake_5__safety_checklist.md__during-print}

\begin{itemize}
\item[$\square$]
  First Layer Monitoring

  \begin{itemize}
  \tightlist
  \item
    Watch first 2-3 layers closely
  \item
    Check bed adhesion (not too loose or tight)
  \item
    Confirm nozzle temperature stable
  \item
    Monitor for any unusual sounds
  \end{itemize}
\item[$\square$]
  Regular Checks

  \begin{itemize}
  \tightlist
  \item
    Check print every 15-30 minutes initially
  \item
    Verify filament is feeding smoothly
  \item
    Listen for grinding or skipping noises
  \item
    Watch for layer shifting or warping
  \end{itemize}
\item[$\square$]
  Temperature Stability

  \begin{itemize}
  \tightlist
  \item
    Heated bed maintains consistent temperature
  \item
    Nozzle temperature doesn\textquotesingle t fluctuate
  \item
    No thermal runaway warnings
  \end{itemize}
\end{itemize}

\subsubsection*{Safety Alerts}\label{docs__pandoc__latex__src__3dmake_foundation__lessons_3dmake_5__safety_checklist.md__safety-alerts}

\begin{itemize}
\tightlist
\item[$\square$]
  STOP Print Immediately If

  \begin{itemize}
  \tightlist
  \item
    Nozzle jams or extrudes unevenly
  \item
    Filament completely stops feeding
  \item
    Burning smell detected
  \item
    Visible layer shifting
  \item
    Extruder grinding or skipping sounds
  \item
    Any unusual smells (especially chemical/burning)
  \item
    Power indicators show failure
  \end{itemize}
\end{itemize}

\subsubsection*{Post-Print}\label{docs__pandoc__latex__src__3dmake_foundation__lessons_3dmake_5__safety_checklist.md__post-print}

\begin{itemize}
\item[$\square$]
  Cool Down Period

  \begin{itemize}
  \tightlist
  \item
    Allow heated bed to cool naturally (10-15 min)
  \item
    Keep hands clear until nozzle cools
  \item
    Verify printer is idle before touching
  \end{itemize}
\item[$\square$]
  Print Removal

  \begin{itemize}
  \tightlist
  \item
    Use proper tools (spatula/scraper)
  \item
    Remove print when bed fully cooled
  \item
    Inspect print for defects or sharp edges
  \item
    Sand sharp edges if necessary
  \end{itemize}
\item[$\square$]
  Equipment Cleaning

  \begin{itemize}
  \tightlist
  \item
    Wipe nozzle with brass wire when cooled
  \item
    Remove any plastic debris from build plate
  \item
    Check extruder for filament residue
  \item
    Clean any visible dust from electronics
  \end{itemize}
\item[$\square$]
  Workspace Cleanup

  \begin{itemize}
  \tightlist
  \item
    Return tools to proper storage
  \item
    Dispose of support material safely
  \item
    Tidy workspace and verify clear pathways
  \item
    Store filament properly in dry location
  \end{itemize}
\end{itemize}

\subsubsection*{Hazard Awareness}\label{docs__pandoc__latex__src__3dmake_foundation__lessons_3dmake_5__safety_checklist.md__hazard-awareness}

Hot Surfaces:

\begin{itemize}
\tightlist
\item
  Nozzle reaches 200-250C
\item
  Bed reaches 60-100C
\item
  Allow adequate cooling time before touching
\end{itemize}

Filament Hazards:

\begin{itemize}
\tightlist
\item
  Tangled filament can jam printer
\item
  Always keep filament spool free-spinning
\item
  Check for knots or kinks before printing
\end{itemize}

Electrical Safety:

\begin{itemize}
\tightlist
\item
  Never operate with wet hands
\item
  Keep liquids away from printer
\item
  Unplug before major maintenance
\item
  Verify power supply is grounded
\end{itemize}

Material Hazards:

\begin{itemize}
\tightlist
\item
  Some filaments emit fumes when heated
\item
  Ensure adequate ventilation
\item
  Use HEPA filter if printing indoors
\item
  Dispose of failed prints properly
\end{itemize}

\subsubsection*{Emergency Contacts}\label{docs__pandoc__latex__src__3dmake_foundation__lessons_3dmake_5__safety_checklist.md__emergency-contacts}

\begin{itemize}
\tightlist
\item
  Printer Fire: Unplug immediately, use dry chemical extinguisher (NOT water)
\item
  Burns: Rinse with cool water for 15+ minutes
\item
  Inhalation Issues: Move to fresh air, seek medical attention if symptoms persist
\end{itemize}

Last Reviewed: \_\\
Reviewed By: \_\\
Notes:

\subsection{Printer Maintenance Log}\label{docs__pandoc__latex__src__3dmake_foundation__lessons_3dmake_5__maintenance_log.md__3dmake_foundation_lessons_3dmake_5-maintenance_log}

Track routine maintenance, repairs, and operational issues to keep your printer running smoothly.

\subsubsection*{Printer Information}\label{docs__pandoc__latex__src__3dmake_foundation__lessons_3dmake_5__maintenance_log.md__printer-information}

\begin{itemize}
\tightlist
\item
  Model: \_
\item
  Serial Number: \_
\item
  Purchase Date: \_
\item
  Last Service Date: \_
\end{itemize}

\subsubsection*{Maintenance Schedule}\label{docs__pandoc__latex__src__3dmake_foundation__lessons_3dmake_5__maintenance_log.md__maintenance-schedule}

\paragraph*{Daily (Before Each Use)}\label{docs__pandoc__latex__src__3dmake_foundation__lessons_3dmake_5__maintenance_log.md__daily-before-each-use}

\begin{itemize}
\tightlist
\item[$\square$]
  Visual inspection for damage
\item[$\square$]
  Clean nozzle if needed
\item[$\square$]
  Check build plate level
\item[$\square$]
  Verify filament spool rotates freely
\end{itemize}

\paragraph*{Weekly}\label{docs__pandoc__latex__src__3dmake_foundation__lessons_3dmake_5__maintenance_log.md__weekly}

\begin{itemize}
\tightlist
\item[$\square$]
  Clean extruder drive gear
\item[$\square$]
  Inspect and clean Z-axis rails
\item[$\square$]
  Check all cable connections
\item[$\square$]
  Test emergency stop function
\end{itemize}

\paragraph*{Monthly}\label{docs__pandoc__latex__src__3dmake_foundation__lessons_3dmake_5__maintenance_log.md__monthly}

\begin{itemize}
\tightlist
\item[$\square$]
  Full build plate leveling procedure
\item[$\square$]
  Clean interior of printer chamber
\item[$\square$]
  Inspect heating elements
\item[$\square$]
  Test temperature calibration
\end{itemize}

\paragraph*{Quarterly (Every 3 Months)}\label{docs__pandoc__latex__src__3dmake_foundation__lessons_3dmake_5__maintenance_log.md__quarterly-every-3-months}

\begin{itemize}
\tightlist
\item[$\square$]
  Replace worn nozzle if needed
\item[$\square$]
  Full mechanical inspection
\item[$\square$]
  Software/firmware update check
\item[$\square$]
  Complete system test print
\end{itemize}

\subsubsection*{Maintenance Log}\label{docs__pandoc__latex__src__3dmake_foundation__lessons_3dmake_5__maintenance_log.md__maintenance-log}

{\def\LTcaptype{none} % do not increment counter
\begin{longtable}[]{@{}
  >{\raggedright\arraybackslash}p{(\linewidth - 12\tabcolsep) * \real{0.0674}}
  >{\raggedright\arraybackslash}p{(\linewidth - 12\tabcolsep) * \real{0.2022}}
  >{\raggedright\arraybackslash}p{(\linewidth - 12\tabcolsep) * \real{0.1461}}
  >{\raggedright\arraybackslash}p{(\linewidth - 12\tabcolsep) * \real{0.1348}}
  >{\raggedright\arraybackslash}p{(\linewidth - 12\tabcolsep) * \real{0.1573}}
  >{\raggedright\arraybackslash}p{(\linewidth - 12\tabcolsep) * \real{0.1348}}
  >{\raggedright\arraybackslash}p{(\linewidth - 12\tabcolsep) * \real{0.1573}}@{}}
\toprule\noalign{}
\begin{minipage}[b]{\linewidth}\raggedright
Date
\end{minipage} & \begin{minipage}[b]{\linewidth}\raggedright
Maintenance Type
\end{minipage} & \begin{minipage}[b]{\linewidth}\raggedright
Description
\end{minipage} & \begin{minipage}[b]{\linewidth}\raggedright
Time Spent
\end{minipage} & \begin{minipage}[b]{\linewidth}\raggedright
Issues Found
\end{minipage} & \begin{minipage}[b]{\linewidth}\raggedright
Resolution
\end{minipage} & \begin{minipage}[b]{\linewidth}\raggedright
Performed By
\end{minipage} \\
\midrule\noalign{}
\endhead
\bottomrule\noalign{}
\endlastfoot
& & & & & & \\
& & & & & & \\
& & & & & & \\
& & & & & & \\
& & & & & & \\
& & & & & & \\
& & & & & & \\
& & & & & & \\
\end{longtable}
}

\subsubsection*{Issue Tracking}\label{docs__pandoc__latex__src__3dmake_foundation__lessons_3dmake_5__maintenance_log.md__issue-tracking}

{\def\LTcaptype{none} % do not increment counter
\begin{longtable}[]{@{}llllll@{}}
\toprule\noalign{}
Date & Symptom & Diagnosis & Action Taken & Status & Notes \\
\midrule\noalign{}
\endhead
\bottomrule\noalign{}
\endlastfoot
& & & & & \\
& & & & & \\
& & & & & \\
& & & & & \\
& & & & & \\
\end{longtable}
}

\subsubsection*{Parts Replacement}\label{docs__pandoc__latex__src__3dmake_foundation__lessons_3dmake_5__maintenance_log.md__parts-replacement}

{\def\LTcaptype{none} % do not increment counter
\begin{longtable}[]{@{}llllll@{}}
\toprule\noalign{}
Date & Part & Reason & Supplier & Cost & Notes \\
\midrule\noalign{}
\endhead
\bottomrule\noalign{}
\endlastfoot
& & & & & \\
& & & & & \\
& & & & & \\
& & & & & \\
\end{longtable}
}

\subsubsection*{Filament Compatibility Notes}\label{docs__pandoc__latex__src__3dmake_foundation__lessons_3dmake_5__maintenance_log.md__filament-compatibility-notes}

Test results for different filament types used with this printer:

{\def\LTcaptype{none} % do not increment counter
\begin{longtable}[]{@{}
  >{\raggedright\arraybackslash}p{(\linewidth - 12\tabcolsep) * \real{0.1899}}
  >{\raggedright\arraybackslash}p{(\linewidth - 12\tabcolsep) * \real{0.0886}}
  >{\raggedright\arraybackslash}p{(\linewidth - 12\tabcolsep) * \real{0.1646}}
  >{\raggedright\arraybackslash}p{(\linewidth - 12\tabcolsep) * \real{0.1266}}
  >{\raggedright\arraybackslash}p{(\linewidth - 12\tabcolsep) * \real{0.1646}}
  >{\raggedright\arraybackslash}p{(\linewidth - 12\tabcolsep) * \real{0.1772}}
  >{\raggedright\arraybackslash}p{(\linewidth - 12\tabcolsep) * \real{0.0886}}@{}}
\toprule\noalign{}
\begin{minipage}[b]{\linewidth}\raggedright
Filament Type
\end{minipage} & \begin{minipage}[b]{\linewidth}\raggedright
Brand
\end{minipage} & \begin{minipage}[b]{\linewidth}\raggedright
Nozzle Temp
\end{minipage} & \begin{minipage}[b]{\linewidth}\raggedright
Bed Temp
\end{minipage} & \begin{minipage}[b]{\linewidth}\raggedright
Print Speed
\end{minipage} & \begin{minipage}[b]{\linewidth}\raggedright
Success Rate
\end{minipage} & \begin{minipage}[b]{\linewidth}\raggedright
Notes
\end{minipage} \\
\midrule\noalign{}
\endhead
\bottomrule\noalign{}
\endlastfoot
PLA & & & & & & \\
PETG & & & & & & \\
ABS & & & & & & \\
TPU & & & & & & \\
Other: & & & & & & \\
\end{longtable}
}

\subsubsection*{Troubleshooting Reference}\label{docs__pandoc__latex__src__3dmake_foundation__lessons_3dmake_5__maintenance_log.md__troubleshooting-reference}

\paragraph*{Common Issues and Solutions}\label{docs__pandoc__latex__src__3dmake_foundation__lessons_3dmake_5__maintenance_log.md__common-issues-and-solutions}

Nozzle Clogs:

\begin{itemize}
\tightlist
\item
  Date First Noticed: \_
\item
  Attempts to Clear: \_
\item
  Successful Solution: \_
\end{itemize}

Bed Adhesion Problems:

\begin{itemize}
\tightlist
\item
  Suspected Cause: \_
\item
  Solution Applied: \_
\item
  Result: \_
\end{itemize}

Layer Shifting:

\begin{itemize}
\tightlist
\item
  Frequency: \_
\item
  Likely Causes: \_
\item
  Actions Taken: \_
\end{itemize}

Extrusion Issues:

\begin{itemize}
\tightlist
\item
  Symptoms: \_
\item
  Diagnosed Problem: \_
\item
  Fix Applied: \_
\end{itemize}

\subsubsection*{Printer Performance Metrics}\label{docs__pandoc__latex__src__3dmake_foundation__lessons_3dmake_5__maintenance_log.md__printer-performance-metrics}

Track overall printer health over time:

{\def\LTcaptype{none} % do not increment counter
\begin{longtable}[]{@{}
  >{\raggedright\arraybackslash}p{(\linewidth - 8\tabcolsep) * \real{0.0897}}
  >{\raggedright\arraybackslash}p{(\linewidth - 8\tabcolsep) * \real{0.2564}}
  >{\raggedright\arraybackslash}p{(\linewidth - 8\tabcolsep) * \real{0.2051}}
  >{\raggedright\arraybackslash}p{(\linewidth - 8\tabcolsep) * \real{0.1923}}
  >{\raggedright\arraybackslash}p{(\linewidth - 8\tabcolsep) * \real{0.2564}}@{}}
\toprule\noalign{}
\begin{minipage}[b]{\linewidth}\raggedright
Month
\end{minipage} & \begin{minipage}[b]{\linewidth}\raggedright
Print Success Rate
\end{minipage} & \begin{minipage}[b]{\linewidth}\raggedright
Avg Print Time
\end{minipage} & \begin{minipage}[b]{\linewidth}\raggedright
Common Issues
\end{minipage} & \begin{minipage}[b]{\linewidth}\raggedright
Overall Assessment
\end{minipage} \\
\midrule\noalign{}
\endhead
\bottomrule\noalign{}
\endlastfoot
& & & & \\
& & & & \\
& & & & \\
& & & & \\
\end{longtable}
}

\subsubsection*{Warranty \& Support Information}\label{docs__pandoc__latex__src__3dmake_foundation__lessons_3dmake_5__maintenance_log.md__warranty--support-information}

\begin{itemize}
\tightlist
\item
  Warranty Expires: \_
\item
  Manufacturer Support: \_
\item
  Local Technician: \_
\item
  Emergency Contact: \_
\end{itemize}

Last Log Entry: \_\\
Logged By: \_

\section{Lesson 6: Practical 3dm Commands and Text Embossing}\label{docs__pandoc__latex__src__3dmake_foundation__lessons_3dmake_6__lessons_3dmake_6.md__lesson-6-practical-3dm-commands-and-text-embossing}

Estimated time: 90--120 minutes

\subsection*{Learning Objectives}\label{docs__pandoc__latex__src__3dmake_foundation__lessons_3dmake_6__lessons_3dmake_6.md__learning-objectives}

\begin{itemize}
\tightlist
\item
  Use the full \texttt{3dm} command suite: \texttt{describe}, \texttt{preview}, \texttt{orient}, \texttt{slice}
\item
  Emboss and engrave text using OpenSCAD\textquotesingle s \texttt{text()} function with \texttt{linear\_extrude()}
\item
  Use OpenSCAD string functions: \texttt{str()}, \texttt{len()}, \texttt{search()}, \texttt{substr()}
\item
  Apply \texttt{let()} for scoped variable declarations
\item
  Build a parametric label-making system
\end{itemize}

\subsection*{Materials}\label{docs__pandoc__latex__src__3dmake_foundation__lessons_3dmake_6__lessons_3dmake_6.md__materials}

\begin{itemize}
\tightlist
\item
  3dMake project from previous lessons
\item
  Terminal and editor
\item
  Slicer (PrusaSlicer or equivalent)
\end{itemize}

\subsection*{Step-by-step Tasks}\label{docs__pandoc__latex__src__3dmake_foundation__lessons_3dmake_6__lessons_3dmake_6.md__step-by-step-tasks}

\subsubsection*{1. Master the 3dm Command Suite}\label{docs__pandoc__latex__src__3dmake_foundation__lessons_3dmake_6__lessons_3dmake_6.md__1-master-the-3dm-command-suite}

\begin{lstlisting}[style=Alabaster, language=bash]
# Describe your current model using AI analysis (non-visual validation)
3dm describe

# Generate a preview image (PNG) of the model
3dm preview

# Suggest optimal print orientation
3dm orient

# Slice the model using your configured slicer settings
3dm slice

\end{lstlisting}

Quick Reference:

{\def\LTcaptype{none} % do not increment counter
\begin{longtable}[]{@{}
  >{\raggedright\arraybackslash}p{(\linewidth - 4\tabcolsep) * \real{0.3333}}
  >{\raggedright\arraybackslash}p{(\linewidth - 4\tabcolsep) * \real{0.3333}}
  >{\raggedright\arraybackslash}p{(\linewidth - 4\tabcolsep) * \real{0.3333}}@{}}
\toprule\noalign{}
\begin{minipage}[b]{\linewidth}\raggedright
Command
\end{minipage} & \begin{minipage}[b]{\linewidth}\raggedright
What It Does
\end{minipage} & \begin{minipage}[b]{\linewidth}\raggedright
Output
\end{minipage} \\
\midrule\noalign{}
\endhead
\bottomrule\noalign{}
\endlastfoot
\texttt{3dm\ build} & Compile \texttt{.scad} → \texttt{.stl} &
\texttt{build/main.stl} \\
\texttt{3dm\ describe} & AI geometry description & Console text \\
\texttt{3dm\ preview} & Render model to PNG image &
\texttt{build/preview.png} \\
\texttt{3dm\ orient} & Suggest print orientation &
Console text with recommendation \\
\texttt{3dm\ slice} & Call slicer on current STL & G-code file \\
\end{longtable}
}

Note: \texttt{3dm\ describe} and \texttt{3dm\ orient} both use the AI backend configured in your \texttt{3dmake.toml}.\footnote{3DMake GitHub Repository --- Command Reference. \url{https://github.com/tdeck/3dmake}. See README for full list of available commands.}

\subsubsection*{2. Emboss Text with text() and linear\_extrude()}\label{docs__pandoc__latex__src__3dmake_foundation__lessons_3dmake_6__lessons_3dmake_6.md__2-emboss-text-with-text-and-linear_extrude}

\begin{lstlisting}[style=Alabaster, language=openscad]
// Basic text emboss (text raised above a base). See [^2] for more on text and fonts.
module embossed_label(label_text, font_size=8, depth=1.5) {
  linear_extrude(height=depth) {
    text(label_text,
         size=font_size,
         font="Liberation Sans:style=Bold",
         halign="center",
         valign="center",
         $fn=4   // $fn affects curve resolution on letters
    );
  }
}

// Base plate with embossed text
difference() {
  cube([60, 20, 5], center=true);
  translate([0, 0, 4])
    embossed_label("HELLO", font_size=8, depth=2);  // cuts 2mm into plate (engrave)
}

// OR: emboss (text proud of surface)
union() {
  cube([60, 20, 3], center=true);
  translate([0, 0, 3])
    embossed_label("HELLO", font_size=8, depth=1.5);
}

\end{lstlisting}

\subsubsection*{3. Use String Functions}\label{docs__pandoc__latex__src__3dmake_foundation__lessons_3dmake_6__lessons_3dmake_6.md__3-use-string-functions}

\begin{lstlisting}[style=Alabaster, language=openscad]
// str() - convert and concatenate values into strings
part_id = str("PART-", 2026, "-", 42);
echo(part_id);  // PART-2026-42

// len() - length of a string or list
name = "OpenSCAD";
echo(len(name));  // 8

// search() - find a character's position
echo(search("S", "OpenSCAD"));  // [[4]]

// substr() - extract substring
full = "BATCH-001";
batch = substr(full, 0, 5);   // "BATCH"
number = substr(full, 6, 3);  // "001"
echo(batch, number);

\end{lstlisting}

\subsubsection*{4. Use let() for Scoped Variables}\label{docs__pandoc__latex__src__3dmake_foundation__lessons_3dmake_6__lessons_3dmake_6.md__4-use-let-for-scoped-variables}

\begin{lstlisting}[style=Alabaster, language=openscad]
// let() scopes variables to a block — they don't pollute the global namespace
let (
  base_w = 80,
  base_d = 50,
  text_offset_z = 5
) {
  cube([base_w, base_d, text_offset_z]);
  translate([base_w/2, base_d/2, text_offset_z])
    linear_extrude(2) text("v1.0", size=6, halign="center", valign="center");
}

\end{lstlisting}

\subsubsection*{5. Build a Parametric Label Maker}\label{docs__pandoc__latex__src__3dmake_foundation__lessons_3dmake_6__lessons_3dmake_6.md__5-build-a-parametric-label-maker}

\begin{lstlisting}[style=Alabaster, language=openscad]
// Parametric label: all dimensions and content are parameters
label_text   = "STORAGE BOX";
label_w      = 80;
label_h      = 18;
plate_depth  = 3;
text_depth   = 1.2;
font_size    = 7;
corner_r     = 2;

module label_plate(txt, w, h, t, td, fs, cr) {
  difference() {
    // Rounded base plate
    minkowski() {
      cube([w - 2*cr, h - 2*cr, t], center=true);
      cylinder(r=cr, h=0.01, $fn=16);
    }
    // Engraved text
    translate([0, 0, t/2 - td + 0.001])
      linear_extrude(td + 0.001)
        text(txt, size=fs, font="Liberation Sans:style=Bold",
             halign="center", valign="center", $fn=4);
    // Mounting hole
    translate([w/2 - 6, 0, -0.001])
      cylinder(r=1.5, h=t + 0.002, $fn=16);
  }
}

label_plate(label_text, label_w, label_h, plate_depth, text_depth, font_size, corner_r);

\end{lstlisting}

\subsubsection*{Checkpoint}\label{docs__pandoc__latex__src__3dmake_foundation__lessons_3dmake_6__lessons_3dmake_6.md__checkpoint}

\begin{itemize}
\tightlist
\item
  After step 1: All four commands run without error and produce expected output.
\item
  After step 2: \texttt{3dm\ build} produces an STL with readable text in the slicer preview.
\item
  After step 5: Your label plate shows the text correctly positioned and engraved.
\end{itemize}

\subsection*{Font Handling}\label{docs__pandoc__latex__src__3dmake_foundation__lessons_3dmake_6__lessons_3dmake_6.md__font-handling}

OpenSCAD accesses system fonts. Use \texttt{fontconfig} syntax for precise control:

\begin{lstlisting}[style=Alabaster, language=openscad]
// Font specification format: "FontName:style=StyleName"
text("Hello", font="Liberation Sans:style=Bold");
text("Hello", font="Liberation Mono:style=Regular");
text("Hello", font="DejaVu Serif:style=Italic");

// List available fonts from the OpenSCAD Help menu > Font List
// Or from terminal (Linux):
// fc-list | grep -i "liberation"

\end{lstlisting}

Common classroom-safe fonts (widely available on Linux):

\begin{itemize}
\tightlist
\item
  \texttt{Liberation\ Sans} --- clean, readable sans-serif (Arial-compatible)
\item
  \texttt{Liberation\ Mono} --- monospace (good for part numbers)
\item
  \texttt{DejaVu\ Serif} --- serif option
\end{itemize}

\subsection*{Quiz --- Lesson 3dMake.6 (15 questions)}\label{docs__pandoc__latex__src__3dmake_foundation__lessons_3dmake_6__lessons_3dmake_6.md__quiz--lesson-3dmake6-15-questions}

\begin{enumerate}
\tightlist
\item
  What does \texttt{3dm\ describe} do?
\item
  What does \texttt{3dm\ orient} output, and when would you use it?
\item
  What is the difference between embossed text and engraved text in 3D printing?
\item
  What parameter in \texttt{text()} controls horizontal alignment?
\item
  What does \texttt{str("PART-",\ 2024,\ "-",\ 1)} return?
\item
  What does \texttt{len("OpenSCAD")} return?
\item
  What is the purpose of \texttt{let()} in OpenSCAD?
\item
  What does \texttt{\$fn=4} do to letters rendered with \texttt{text()}?
\item
  True or False: \texttt{3dm\ slice} compiles your \texttt{.scad} file before slicing.
\item
  What font specification format does OpenSCAD use?
\item
  Explain the difference between \texttt{halign="left"} and \texttt{halign="center"} in the \texttt{text()} function and describe when you would use each.
\item
  What does \texttt{substr("BATCH-001",\ 6,\ 3)} return?
\item
  Describe the role of \texttt{linear\_extrude()} when used with \texttt{text()}. What would happen if you omitted it?
\item
  Write the OpenSCAD code to engrave the text "LOT-42" into a 50×20×4 mm base plate, centered, with 1.5 mm engraving depth.
\item
  What is the difference between scoped variables declared with \texttt{let()} and top-level global variables in an OpenSCAD file?
\end{enumerate}

\subsection*{Extension Problems (15)}\label{docs__pandoc__latex__src__3dmake_foundation__lessons_3dmake_6__lessons_3dmake_6.md__extension-problems-15}

\begin{enumerate}
\tightlist
\item
  Build a parametric serial number generator: accept a prefix string and a number, concatenate them with \texttt{str()}, and emboss the result onto a label plate.
\item
  Create a dynamic version label: use \texttt{str()} to combine a product name and a version number, where both are top-level parameters.
\item
  Design a 4-up label sheet: 4 identical labels arranged in a 2×2 grid using \texttt{translate()} and a \texttt{for} loop.
\item
  Build a multi-line label system: stack two \texttt{text()} calls at different Z heights to create a two-line label.
\item
  Create a label with a decorative border using \texttt{difference()} to cut a frame outline around the text.
\item
  Use \texttt{3dm\ orient} on three different models (flat slab, tall cylinder, L-bracket) and document whether the AI orientation suggestion agrees with your own analysis.
\item
  Build a "batch tag" system: a module that generates different serial numbers from a list, placing each on a separate small tag in a row.
\item
  Design a keychain tag: a rounded rectangle with a hole for a ring, parametric text, and two mounting ridges.
\item
  Use \texttt{search()} to find the position of a dash character in a part-number string. Write the code and explain why this might be useful.
\item
  Build a "negative space" text plate: instead of engraving text, create a plate with all letters cut completely through (silhouette/stencil style).
\item
  Design a parametric drawer label holder: a clip that slides over the edge of a drawer and holds a label card. Make all dimensions parameters.
\item
  Create a font comparison tool: render the same text string in Liberation Sans, Liberation Mono, and DejaVu Serif side by side on one plate.
\item
  Build a screen-reader accessibility guide for the five \texttt{3dm} commands taught in this lesson. For each command, document: the command name, what it does, expected output, and how to interpret the output without visual reference.
\item
  Write a parametric module that auto-sizes the font: given a fixed label width and a string, reduce the font size until the text fits within the width.
\item
  Create a part-marking system for a small production batch: design a jig that holds 10 labels in a row, each with an incremented part number, ready to print all at once.
\end{enumerate}

\subsection*{References and Helpful Resources}\label{docs__pandoc__latex__src__3dmake_foundation__lessons_3dmake_6__lessons_3dmake_6.md__references-and-helpful-resources}

\subsubsection*{Supplemental Resources}\label{docs__pandoc__latex__src__3dmake_foundation__lessons_3dmake_6__lessons_3dmake_6.md__supplemental-resources}

\begin{itemize}
\tightlist
\item
  \href{docs/pandoc/latex/src/assets/Programming_with_OpenSCAD.epub}{Programming with OpenSCAD EPUB Textbook} --- String functions and text embossing examples
\item
  \href{https://programmingwithopenscad.github.io/quick-reference.html}{OpenSCAD Quick Reference} --- All string function syntax
\item
  \href{https://github.com/mrhunsaker/VI_3DMake_OpenSCAD_Curriculum/3dMake_Foundation/Lessons_3dMake_6/../../assets/3dMake_Foundation/3dmake-setup-guide.md}{3DMake Terminal Quickstart Guide}
\item
  \href{https://github.com/ProgrammingWithOpenSCAD/CodeSolutions}{CodeSolutions Repository} --- Worked text embossing examples
\end{itemize}

\subsection{Slicing Settings Quick Reference - PrusaSlicer}\label{docs__pandoc__latex__src__3dmake_foundation__lessons_3dmake_6__slicing-settings-reference.md__3dmake_foundation_lessons_3dmake_6-slicing-settings-reference}

\subsubsection*{Recommended Settings by Use Case}\label{docs__pandoc__latex__src__3dmake_foundation__lessons_3dmake_6__slicing-settings-reference.md__recommended-settings-by-use-case}

{\def\LTcaptype{none} % do not increment counter
\begin{longtable}[]{@{}
  >{\raggedright\arraybackslash}p{(\linewidth - 8\tabcolsep) * \real{0.2697}}
  >{\raggedright\arraybackslash}p{(\linewidth - 8\tabcolsep) * \real{0.1573}}
  >{\raggedright\arraybackslash}p{(\linewidth - 8\tabcolsep) * \real{0.0899}}
  >{\raggedright\arraybackslash}p{(\linewidth - 8\tabcolsep) * \real{0.1236}}
  >{\raggedright\arraybackslash}p{(\linewidth - 8\tabcolsep) * \real{0.3596}}@{}}
\toprule\noalign{}
\begin{minipage}[b]{\linewidth}\raggedright
Use Case
\end{minipage} & \begin{minipage}[b]{\linewidth}\raggedright
Layer Height
\end{minipage} & \begin{minipage}[b]{\linewidth}\raggedright
Infill
\end{minipage} & \begin{minipage}[b]{\linewidth}\raggedright
Supports
\end{minipage} & \begin{minipage}[b]{\linewidth}\raggedright
Notes
\end{minipage} \\
\midrule\noalign{}
\endhead
\bottomrule\noalign{}
\endlastfoot
Quick test / prototype & 0.30 mm & 10\% & As needed &
"Draft" - fastest, roughest \\
Standard project & 0.20 mm & 15-20\% & As needed &
Best all-around starting point \\
Functional part & 0.20 mm & 30-40\% & As needed &
Use for parts under stress \\
Fine detail / display & 0.15 mm & 15\% & As needed &
Smoother surface; slower \\
Solid reference part & 0.20 mm & 40-50\% & Rarely &
Rarely needed; long print \\
\end{longtable}
}

\subsubsection*{Filament Temperature Settings}\label{docs__pandoc__latex__src__3dmake_foundation__lessons_3dmake_6__slicing-settings-reference.md__filament-temperature-settings}

{\def\LTcaptype{none} % do not increment counter
\begin{longtable}[]{@{}
  >{\raggedright\arraybackslash}p{(\linewidth - 6\tabcolsep) * \real{0.1299}}
  >{\raggedright\arraybackslash}p{(\linewidth - 6\tabcolsep) * \real{0.1688}}
  >{\raggedright\arraybackslash}p{(\linewidth - 6\tabcolsep) * \real{0.1299}}
  >{\raggedright\arraybackslash}p{(\linewidth - 6\tabcolsep) * \real{0.5714}}@{}}
\toprule\noalign{}
\begin{minipage}[b]{\linewidth}\raggedright
Filament
\end{minipage} & \begin{minipage}[b]{\linewidth}\raggedright
Nozzle Temp
\end{minipage} & \begin{minipage}[b]{\linewidth}\raggedright
Bed Temp
\end{minipage} & \begin{minipage}[b]{\linewidth}\raggedright
Notes
\end{minipage} \\
\midrule\noalign{}
\endhead
\bottomrule\noalign{}
\endlastfoot
PLA & 200-215C & 50-60C & Easiest to print; default choice \\
PETG & 230-250C & 70-85C & Use glue stick on PEI bed \\
TPU & 220-240C & 30-60C & Print at 20-30 mm/s max; direct drive only \\
ABS & 230-250C & 90-110C & Requires enclosure; more fumes \\
\end{longtable}
}

\emph{Always check the temperature range on your filament spool - it may vary by brand.}

\subsubsection*{Support Settings Guide}\label{docs__pandoc__latex__src__3dmake_foundation__lessons_3dmake_6__slicing-settings-reference.md__support-settings-guide}

{\def\LTcaptype{none} % do not increment counter
\begin{longtable}[]{@{}lll@{}}
\toprule\noalign{}
Overhang Angle & Supports Needed? & Recommended Setting \\
\midrule\noalign{}
\endhead
\bottomrule\noalign{}
\endlastfoot
\textless{} 45 & No & None \\
45-60 & Maybe & Preview first; add if sagging \\
\textgreater{} 60 & Yes & Support on build plate only \\
Bridge \textless{} 20 mm & No & Bridges usually fine \\
Bridge \textgreater{} 20 mm & Maybe & Preview; consider reorienting \\
\end{longtable}
}

\subsubsection*{Common Print Problems \& Quick Fixes}\label{docs__pandoc__latex__src__3dmake_foundation__lessons_3dmake_6__slicing-settings-reference.md__common-print-problems--quick-fixes}

{\def\LTcaptype{none} % do not increment counter
\begin{longtable}[]{@{}
  >{\raggedright\arraybackslash}p{(\linewidth - 4\tabcolsep) * \real{0.2655}}
  >{\raggedright\arraybackslash}p{(\linewidth - 4\tabcolsep) * \real{0.3628}}
  >{\raggedright\arraybackslash}p{(\linewidth - 4\tabcolsep) * \real{0.3717}}@{}}
\toprule\noalign{}
\begin{minipage}[b]{\linewidth}\raggedright
Problem
\end{minipage} & \begin{minipage}[b]{\linewidth}\raggedright
Likely Cause
\end{minipage} & \begin{minipage}[b]{\linewidth}\raggedright
Fix
\end{minipage} \\
\midrule\noalign{}
\endhead
\bottomrule\noalign{}
\endlastfoot
Print lifts off bed & Poor adhesion / warping &
Add brim; use glue stick; level bed \\
Stringing between parts & Temperature too high / retraction &
Lower temp 5C; check retraction settings \\
Layer lines very visible & Layer height too thick &
Use 0.15mm or 0.20mm \\
Print takes too long & Layer height too thin / infill too high &
Use 0.30mm draft; reduce infill \\
Holes too small & FDM tolerance - always undersized &
Add 0.2-0.3mm to hole diameter \\
Part broke at layer boundary & Weak axis perpendicular to layers &
Reorient so load is parallel to layers \\
First layer not sticking & Bed not level & Run bed leveling routine \\
Spaghetti / print failure & No supports on overhang &
Add supports or reorient \\
\end{longtable}
}

\subsubsection*{G-code Export Checklist}\label{docs__pandoc__latex__src__3dmake_foundation__lessons_3dmake_6__slicing-settings-reference.md__g-code-export-checklist}

Before exporting, confirm:

\begin{itemize}
\tightlist
\item[$\square$]
  Correct printer profile selected
\item[$\square$]
  Correct filament profile selected
\item[$\square$]
  Layer height appropriate for use case
\item[$\square$]
  Infill percentage set
\item[$\square$]
  Supports enabled if needed
\item[$\square$]
  Layer preview reviewed (no floating parts, supports where needed)
\item[$\square$]
  Print time and filament weight noted for your records
\end{itemize}

\subsubsection*{Sources}\label{docs__pandoc__latex__src__3dmake_foundation__lessons_3dmake_6__slicing-settings-reference.md__sources}

Prusa Research. (2023). \emph{PrusaSlicer knowledge base}. \url{https://help.prusa3d.com/category/prusaslicer_204}\\
Hubs. (2023). \emph{What is FDM 3D printing?} \url{https://www.hubs.com/knowledge-base/what-is-fdm-3d-printing/}\\
ThePrusaSlicer.net. (2025). \emph{How to use PrusaSlicer}. \url{https://theprusaslicer.net/how-to-use-prusaslicer/}

\subsection{Parametric Keychain - Extension Project}\label{docs__pandoc__latex__src__3dmake_foundation__lessons_3dmake_6__parametric-keychain.md__3dmake_foundation_lessons_3dmake_6-parametric-keychain}

Estimated time: 2-4 hours

\subsubsection*{Learning Objectives}\label{docs__pandoc__latex__src__3dmake_foundation__lessons_3dmake_6__parametric-keychain.md__learning-objectives}

By completing this project, you will:

\begin{itemize}
\tightlist
\item
  Create parametric OpenSCAD modules that accept user inputs
\item
  Implement 2D text manipulation and 3D extrusion techniques
\item
  Generate and test multiple design variants systematically
\item
  Document design parameters for reproducibility and user customization
\end{itemize}

\subsubsection*{Objective}\label{docs__pandoc__latex__src__3dmake_foundation__lessons_3dmake_6__parametric-keychain.md__objective}

\begin{itemize}
\tightlist
\item
  Create a parametric keychain design that adapts to custom text, dimensions, and materials.
\end{itemize}

\subsubsection*{Tasks}\label{docs__pandoc__latex__src__3dmake_foundation__lessons_3dmake_6__parametric-keychain.md__tasks}

\begin{enumerate}
\tightlist
\item
  Create \texttt{keychain.scad} with top-level parameters: \texttt{width}, \texttt{height}, \texttt{thickness}, and \texttt{text}.
\item
  Implement embossed or debossed text using \texttt{linear\_extrude()} of a 2D text shape (or simulate with simple geometry if system lacks text support).
\item
  Produce three size variants and export STLs; record print settings.
\item
  Test attachment point for common key rings and report fit.
\end{enumerate}

\subsubsection*{Deliverable}\label{docs__pandoc__latex__src__3dmake_foundation__lessons_3dmake_6__parametric-keychain.md__deliverable}

\begin{itemize}
\tightlist
\item
  Source \texttt{keychain.scad} file with parametric variables documented
\item
  Three STL variants (small, medium, large)
\item
  Print settings log and fit-test report for key ring attachment
\end{itemize}

\subsubsection*{Starter files}\label{docs__pandoc__latex__src__3dmake_foundation__lessons_3dmake_6__parametric-keychain.md__starter-files}

\begin{itemize}
\tightlist
\item
  \href{docs/pandoc/latex/src/assets/Extension_Projects/Parametric_Keychain/starter.scad}{starter.scad} - minimal parametric scaffold to begin.
\end{itemize}

\subsubsection*{Starter Code}\label{docs__pandoc__latex__src__3dmake_foundation__lessons_3dmake_6__parametric-keychain.md__starter-code}

Copy and modify this cube keycap example as your starting point:

\begin{lstlisting}[style=Alabaster, language=openscad]
// Cube Keycap - Beginner Example
// A simple 20mm cube keycap with an embossed letter
// Parameters
keysize = 18;     // mm
keyheight = 12;   // mm
wall = 1.2;        // mm
letter = "R";      // change to your preferred letter
lettersize = 10;  // mm
letterraise = 0.8;// mm
module shell(){
  difference(){
    cube([keysize, keysize, keyheight], center=false);
    translate([wall, wall, wall])
      cube([keysize-2*wall, keysize-2*wall, keyheight], center=false);
  }
}
module emboss(){
  // Emboss letter on top face
  translate([keysize/2, keysize/2, keyheight-0.01])
    linearextrude(height=letterraise)
      text(letter, size=lettersize, halign="center", valign="center");
}
union(){
  shell();
  emboss();
}

\end{lstlisting}

\subsubsection*{Advanced Text Techniques}\label{docs__pandoc__latex__src__3dmake_foundation__lessons_3dmake_6__parametric-keychain.md__advanced-text-techniques}

Beyond simple embossed letters, you can create sophisticated text effects using the patterns from the \emph{Simplifying 3D Printing} textbook:

\paragraph*{Example 1: Circular Text Array}\label{docs__pandoc__latex__src__3dmake_foundation__lessons_3dmake_6__parametric-keychain.md__example-1-circular-text-array}

Arrange text in a circle around a central point:

\begin{lstlisting}[style=Alabaster, language=openscad]
// Circular Text Arrangement
// Text rotates around a center point
module rotate_text(display_text, 
    text_size = 10, 
    distance = 20, 
    rotation_value = 360, 
    tilt = 0)
{
    rotate([0, 0, tilt])
    for(i = [0:len(display_text) - 1])
    {
        rotate([0, 0, -i * rotation_value / len(display_text)])
        translate([0, distance, 0])
        text(display_text[i], 
            font = "Impact:style=Regular", 
            size = text_size,
            halign = "center");
    } 
}
// Use the module to create circular text
linear_extrude(height = 2)
rotate_text("MAKER", text_size = 12, distance = 30, rotation_value = 75, tilt = 30);

\end{lstlisting}

What it does:

\begin{itemize}
\tightlist
\item
  Rotates each letter individually around a center point
\item
  Creates circular or spiral text effects
\item
  Useful for badges, nameplates, and decorative objects
\end{itemize}

\paragraph*{Example 2: Multi-Line Text Composition}\label{docs__pandoc__latex__src__3dmake_foundation__lessons_3dmake_6__parametric-keychain.md__example-2-multi-line-text-composition}

Combine text at different sizes and positions for a professional nameplate:

\begin{lstlisting}[style=Alabaster, language=openscad]
// Professional Nameplate with Multiple Text Layers
module create_nameplate(name, role, company)
{
    union()
    {
        // Base backing plate
        cube([120, 60, 3], center = true);
        // Main name - large, centered
        translate([0, 15, 2])
        linear_extrude(height = 2)
        text(name, size = 24, font = "Impact:style=Regular",
             halign = "center", valign = "center");
        // Role - medium, slightly smaller
        translate([0, 0, 2])
        linear_extrude(height = 2)
        text(role, size = 14, font = "Arial:style=Regular",
             halign = "center", valign = "center");
        // Company name - small, bottom
        translate([0, -15, 2])
        linear_extrude(height = 2)
        text(company, size = 10, font = "Arial:style=Regular",
             halign = "center", valign = "center");
    }
}
// Create a custom nameplate
create_nameplate("Alex Chen", "3D Design Engineer", "MakerCorp");

\end{lstlisting}

Key features:

\begin{itemize}
\tightlist
\item
  Multiple text elements at different scales
\item
  Layered composition for professional appearance
\item
  Each text element can be customized independently
\end{itemize}

\paragraph*{Example 3: Parametric Font Selection}\label{docs__pandoc__latex__src__3dmake_foundation__lessons_3dmake_6__parametric-keychain.md__example-3-parametric-font-selection}

Use different fonts for different effects:

\begin{lstlisting}[style=Alabaster, language=openscad]
// Different fonts create different aesthetics
$fn = 100;
// Impact font (bold, modern)
translate([0, 40, 0])
linear_extrude(height = 2)
text("BOLD", size = 20, font = "Impact:style=Regular", halign = "center");
// Arial font (clean, professional)
translate([0, 10, 0])
linear_extrude(height = 2)
text("Clean", size = 20, font = "Arial:style=Regular", halign = "center");
// Script-like (decorative)
translate([0, -20, 0])
linear_extrude(height = 2)
text("Script", size = 20, font = "Courier:style=Regular", halign = "center");

\end{lstlisting}

Font options (availability depends on your system):

\begin{itemize}
\tightlist
\item
  \texttt{Impact:style=Regular} - Bold, condensed, modern
\item
  \texttt{Arial:style=Regular} - Clean, neutral, professional
\item
  \texttt{Courier:style=Regular} - Monospace, technical
\item
  System fonts vary; check OpenSCAD documentation for your platform
\end{itemize}

\paragraph*{Practical Project: Custom Keychain Nameplate}\label{docs__pandoc__latex__src__3dmake_foundation__lessons_3dmake_6__parametric-keychain.md__practical-project-custom-keychain-nameplate}

Combine everything into a professional keychains with multiple variants:

\begin{lstlisting}[style=Alabaster, language=openscad]
// Customizable Keychain Nameplate
// Parameters - change these to create variants
name = "ALEX";
keysize = 35;
keyheight = 8;
wall = 1.2;
textsize = 16;
module keychain_nameplate(name, width, height, textsize) {
    union() {
        // Shell
        difference() {
            cube([width, width, height], center = false);
            translate([wall, wall, wall])
                cube([width - 2*wall, width - 2*wall, height], center = false);
        }
        // Embossed name
        translate([width/2, width/2, height - 0.01])
        linear_extrude(height = 0.8)
        text(name, size = textsize, font = "Impact:style=Regular",
             halign = "center", valign = "center");
        // Attachment loop
        translate([width/2, -3, height/2])
        rotate([90, 0, 0])
        cylinder(d = 8, h = 6, center = true, $fn = 32);
    }
}
// Create the nameplate
keychain_nameplate(name, keysize, keyheight, textsize);

\end{lstlisting}

\subsubsection*{Assessment Questions (Optional)}\label{docs__pandoc__latex__src__3dmake_foundation__lessons_3dmake_6__parametric-keychain.md__assessment-questions-optional}

\begin{enumerate}
\tightlist
\item
  How did you use OpenSCAD parameters to enable users to customize the keychain without editing code?
\item
  What were the key differences in print time and material usage between your three variants?
\item
  Describe how you tested the key ring attachment and what adjustments you would make for the final design.
\end{enumerate}

\subsection{Parametric Keychain - Student Documentation Template (Extension Project)}\label{docs__pandoc__latex__src__3dmake_foundation__lessons_3dmake_6__parametric_keychain_student_template.md__3dmake_foundation_lessons_3dmake_6-parametric_keychain_student_template}

\begin{itemize}
\tightlist
\item
  Author:
\item
  Date:
\item
  Description: Design a fully parametric keychain that supports personalization and customization.
\end{itemize}

\subsubsection*{Design Concept}\label{docs__pandoc__latex__src__3dmake_foundation__lessons_3dmake_6__parametric_keychain_student_template.md__design-concept}

\begin{itemize}
\tightlist
\item
  Keychain theme or purpose:
\item
  Design elements (shape, attachment, personalization method):
\item
  Parametric strategy (what will be variables?):
\end{itemize}

\subsubsection*{Parametric Variables}\label{docs__pandoc__latex__src__3dmake_foundation__lessons_3dmake_6__parametric_keychain_student_template.md__parametric-variables}

{\def\LTcaptype{none} % do not increment counter
\begin{longtable}[]{@{}lll@{}}
\toprule\noalign{}
Variable & Default Value & Purpose \\
\midrule\noalign{}
\endhead
\bottomrule\noalign{}
\endlastfoot
& & \\
& & \\
& & \\
\end{longtable}
}

\subsubsection*{Variant Configurations}\label{docs__pandoc__latex__src__3dmake_foundation__lessons_3dmake_6__parametric_keychain_student_template.md__variant-configurations}

{\def\LTcaptype{none} % do not increment counter
\begin{longtable}[]{@{}lllll@{}}
\toprule\noalign{}
Variant & Parameter 1 & Parameter 2 & Parameter 3 & Notes \\
\midrule\noalign{}
\endhead
\bottomrule\noalign{}
\endlastfoot
v1 & & & & \\
v2 & & & & \\
v3 & & & & \\
\end{longtable}
}

\subsubsection*{Print and Assembly Results}\label{docs__pandoc__latex__src__3dmake_foundation__lessons_3dmake_6__parametric_keychain_student_template.md__print-and-assembly-results}

\begin{itemize}
\tightlist
\item
  Describe how each variant prints:
\item
  Assembly/attachment method:
\item
  Functionality (does it work as intended?):
\end{itemize}

\subsubsection*{Reflections}\label{docs__pandoc__latex__src__3dmake_foundation__lessons_3dmake_6__parametric_keychain_student_template.md__reflections}

\begin{itemize}
\tightlist
\item
  Which parameterization strategy was most effective?
\item
  How could someone else customize this design?
\item
  What would you add in a future iteration?
\end{itemize}

\subsubsection*{Customization Guide}\label{docs__pandoc__latex__src__3dmake_foundation__lessons_3dmake_6__parametric_keychain_student_template.md__customization-guide}

\begin{itemize}
\tightlist
\item
  Clear instructions for modifying parameters
\item
  Examples of common customizations
\item
  Best practices for variant generation
\end{itemize}

\subsubsection*{Attachments}\label{docs__pandoc__latex__src__3dmake_foundation__lessons_3dmake_6__parametric_keychain_student_template.md__attachments}

\begin{itemize}
\tightlist
\item[$\square$]
  \texttt{.scad} file with full parametric structure
\item[$\square$]
  Photos of 2+ printed variants
\item[$\square$]
  Variant specification table
\item[$\square$]
  Usage/customization guide
\end{itemize}

\subsubsection*{Teacher Feedback}\label{docs__pandoc__latex__src__3dmake_foundation__lessons_3dmake_6__parametric_keychain_student_template.md__teacher-feedback}

{\def\LTcaptype{none} % do not increment counter
\begin{longtable}[]{@{}lll@{}}
\toprule\noalign{}
Category & Score & Notes \\
\midrule\noalign{}
\endhead
\bottomrule\noalign{}
\endlastfoot
Problem \& Solution (0-3) & & \\
Design \& Code Quality (0-3) & & \\
Documentation (0-3) & & \\
Total (0-9) & & \\
\end{longtable}
}

\subsection{Parametric Keychain - Teacher Template (Extension Project)}\label{docs__pandoc__latex__src__3dmake_foundation__lessons_3dmake_6__parametric_keychain_teacher_template.md__3dmake_foundation_lessons_3dmake_6-parametric_keychain_teacher_template}

\subsubsection*{Briefing}\label{docs__pandoc__latex__src__3dmake_foundation__lessons_3dmake_6__parametric_keychain_teacher_template.md__briefing}

Students design a personalized keychain using fully parametric OpenSCAD code. This project emphasizes parameter-driven design, customization, and user-centered iteration.

Key Learning: Parametric modularity; customization; design for manufacturing variation.

Real-world Connection: Mass customization is a growing manufacturing trend. Parametric design enables efficient production of personalized products.

\subsubsection*{Constraints}\label{docs__pandoc__latex__src__3dmake_foundation__lessons_3dmake_6__parametric_keychain_teacher_template.md__constraints}

\begin{itemize}
\tightlist
\item
  Keychain must be fully parametric (text, size, material all variables)
\item
  Design must include at least one test for different parameter values
\item
  Assembly instructions must support varied configurations
\item
  Code must be documented for future customization
\end{itemize}

\subsubsection*{Functional Requirements}\label{docs__pandoc__latex__src__3dmake_foundation__lessons_3dmake_6__parametric_keychain_teacher_template.md__functional-requirements}

\begin{itemize}
\tightlist
\item
  All design elements are parametric variables
\item
  Keychain includes personalization (names, initials, dates)
\item
  Multiple variant configurations tested and documented
\item
  Design is reproducible and shareable
\end{itemize}

\subsubsection*{Deliverables}\label{docs__pandoc__latex__src__3dmake_foundation__lessons_3dmake_6__parametric_keychain_teacher_template.md__deliverables}

\begin{itemize}
\tightlist
\item
  \texttt{.scad} with parametric keychain module
\item
  Variant specification table (3+ configurations)
\item
  Photos of at least 2 printed variants
\item
  Customization guide for future users
\item
  Code documentation and usage examples
\end{itemize}

\subsubsection*{Rubric}\label{docs__pandoc__latex__src__3dmake_foundation__lessons_3dmake_6__parametric_keychain_teacher_template.md__rubric}

\paragraph*{Category 1: Problem \& Solution (0-3)}\label{docs__pandoc__latex__src__3dmake_foundation__lessons_3dmake_6__parametric_keychain_teacher_template.md__category-1-problem--solution-0-3}

Keychains print successfully and are functional/customizable.

\paragraph*{Category 2: Design \& Code Quality (0-3)}\label{docs__pandoc__latex__src__3dmake_foundation__lessons_3dmake_6__parametric_keychain_teacher_template.md__category-2-design--code-quality-0-3}

Code is fully parametric and well-organized. Variants work well.

\paragraph*{Category 3: Documentation (0-3)}\label{docs__pandoc__latex__src__3dmake_foundation__lessons_3dmake_6__parametric_keychain_teacher_template.md__category-3-documentation-0-3}

Variant table complete. Customization guide clear and detailed.

\subsubsection*{Assessment Notes}\label{docs__pandoc__latex__src__3dmake_foundation__lessons_3dmake_6__parametric_keychain_teacher_template.md__assessment-notes}

\begin{itemize}
\tightlist
\item
  Strong submissions: Show comprehensive parametric thinking, multiple tested variants, and clear guidance for customization
\item
  Reinforce: Documentation for design reuse
\item
  Extension: User feedback on customization preferences; commercial potential analysis
\end{itemize}

\section{Lesson 7: Parametric Transforms and the Phone Stand Project}\label{docs__pandoc__latex__src__3dmake_foundation__lessons_3dmake_7__lessons_3dmake_7.md__lesson-7-parametric-transforms-and-the-phone-stand-project}

Estimated time: 90--120 minutes

\subsection*{Learning Objectives}\label{docs__pandoc__latex__src__3dmake_foundation__lessons_3dmake_7__lessons_3dmake_7.md__learning-objectives}

\begin{itemize}
\tightlist
\item
  Apply \texttt{translate()}, \texttt{rotate()}, \texttt{scale()}, and \texttt{mirror()} correctly
\item
  Understand transform order of operations
\item
  Use \texttt{minkowski()} for organic edge rounding
\item
  Apply trigonometric functions (\texttt{sin()}, \texttt{cos()}, \texttt{atan()}, \texttt{atan2()}) for angular positioning
\item
  Build a complete parametric phone stand
\end{itemize}

\subsection*{Materials}\label{docs__pandoc__latex__src__3dmake_foundation__lessons_3dmake_7__lessons_3dmake_7.md__materials}

\begin{itemize}
\tightlist
\item
  3dMake project
\item
  Terminal and editor
\item
  Calipers for measuring your phone
\end{itemize}

\subsection*{Step-by-step Tasks}\label{docs__pandoc__latex__src__3dmake_foundation__lessons_3dmake_7__lessons_3dmake_7.md__step-by-step-tasks}

\subsubsection*{\texorpdfstring{1. Master Transforms and Order of Operations \footnote{OpenSCAD User Manual --- Transformations. \url{https://en.wikibooks.org/wiki/OpenSCAD_User_Manual/Transformations}}}{1. Master Transforms and Order of Operations }}\label{docs__pandoc__latex__src__3dmake_foundation__lessons_3dmake_7__lessons_3dmake_7.md__1-master-transforms-and-order-of-operations-}

Transforms in OpenSCAD apply right-to-left (innermost first). Order matters:

\begin{lstlisting}[style=Alabaster, language=openscad]
// rotate THEN translate (object moves to [20,0,0] first, then rotates in place)
rotate([0, 0, 45]) translate([20, 0, 0]) cube([10, 5, 5]);

// translate THEN rotate (object moves 20mm, then rotates around origin)
translate([20, 0, 0]) rotate([0, 0, 45]) cube([10, 5, 5]);

\end{lstlisting}

These produce different results. A helpful rule: read transforms from the inside out.

\subsubsection*{\texorpdfstring{2. Use Trigonometric and Math Functions \footnote{OpenSCAD User Manual --- Mathematical Functions. \url{https://en.wikibooks.org/wiki/OpenSCAD_User_Manual/Mathematical_Functions}}}{2. Use Trigonometric and Math Functions }}\label{docs__pandoc__latex__src__3dmake_foundation__lessons_3dmake_7__lessons_3dmake_7.md__2-use-trigonometric-and-math-functions-}

\begin{lstlisting}[style=Alabaster, language=openscad]
// Place objects in a circle using sin() and cos()
r = 30;
for (i = [0 : 45 : 315]) {
  x = r * cos(i);
  y = r * sin(i);
  translate([x, y, 0]) cylinder(r=3, h=5, $fn=16);
}

// atan(y/x) gives angle from a ratio (single-argument arctangent)
angle_a = atan(1);      // 45 degrees

// atan2(y, x) gives angle from x and y components directly
// Use atan2() when both x and y are known — it handles all four quadrants correctly
angle_b = atan2(1, 1);  // 45 degrees (equivalent for this case)
angle_c = atan2(1, -1); // 135 degrees (atan(1/-1) would give -45 — wrong quadrant!)

echo(angle_a, angle_b, angle_c);  // 45, 45, 135

// Round/floor/ceil/pow/sqrt
echo(round(3.6));   // 4
echo(floor(3.6));   // 3
echo(ceil(3.6));    // 4
echo(pow(2, 8));    // 256
echo(sqrt(144));    // 12

\end{lstlisting}

\textbf{When to use \texttt{atan2(y,\ x)} vs \texttt{atan(y/x)}:} Use \texttt{atan2()} whenever you have both x and y components and need the correct quadrant. \texttt{atan()} only returns values in the range -90° to +90° and divides by zero when x=0.

\subsubsection*{\texorpdfstring{3. Apply Minkowski Sum for Rounded Edges \footnote{OpenSCAD User Manual --- Minkowski and Hull. \url{https://en.wikibooks.org/wiki/OpenSCAD_User_Manual/Minkowski_and_Hull}}}{3. Apply Minkowski Sum for Rounded Edges }}\label{docs__pandoc__latex__src__3dmake_foundation__lessons_3dmake_7__lessons_3dmake_7.md__3-apply-minkowski-sum-for-rounded-edges-}

\begin{lstlisting}[style=Alabaster, language=openscad]
// Minkowski adds the shape of one object to every point of another
// Result: rounded box with smooth edges
module rounded_cube(w, d, h, r=3) {
  minkowski() {
    cube([w - 2*r, d - 2*r, h - r], center=false);
    sphere(r=r, $fn=16);
  }
}

// Flat-base rounded box (cylinder instead of sphere)
module flat_rounded_cube(w, d, h, r=3) {
  minkowski() {
    cube([w - 2*r, d - 2*r, h], center=false);
    cylinder(r=r, h=0.01, $fn=24);
  }
}

\end{lstlisting}

\subsubsection*{4. Apply scale() and mirror()}\label{docs__pandoc__latex__src__3dmake_foundation__lessons_3dmake_7__lessons_3dmake_7.md__4-apply-scale-and-mirror}

\begin{lstlisting}[style=Alabaster, language=openscad]
// scale() stretches geometry — use when you need non-uniform scaling
scale([1, 1, 2]) sphere(r=10, $fn=32);  // stretch sphere 2x in Z = ellipsoid

// mirror() reflects geometry across a plane
module ear() {
  translate([20, 0, 0]) cylinder(r=5, h=3, $fn=32);
}
// Original + mirrored = symmetric pair
ear();
mirror([1, 0, 0]) ear();  // reflect across YZ plane

\end{lstlisting}

\subsubsection*{5. Build the Complete Phone Stand}\label{docs__pandoc__latex__src__3dmake_foundation__lessons_3dmake_7__lessons_3dmake_7.md__5-build-the-complete-phone-stand}

\begin{lstlisting}[style=Alabaster, language=openscad]
// ============================================================
// Parametric Phone Stand
// ============================================================
// Measure your phone and set these parameters:
phone_w = 75;   // mm — phone width
phone_d = 9;    // mm — phone thickness
angle   = 65;   // degrees — viewing angle from horizontal
lip_h   = 15;   // mm — depth of cradle lip

// Derived dimensions
base_d      = (phone_d + 10) / cos(90 - angle);
base_h      = 5;
cradle_wall = 3;
r_fillet    = 3;

module base_plate() {
  flat_rounded_cube(phone_w + 20, base_d + 10, base_h, r_fillet);
}

module back_support() {
  rotate([angle - 90, 0, 0])
    translate([0, 0, 0])
      flat_rounded_cube(phone_w + 6, cradle_wall, 60, r_fillet);
}

module lip() {
  rotate([angle - 90, 0, 0])
    translate([0, -lip_h, 0])
      flat_rounded_cube(phone_w + 6, lip_h + cradle_wall, cradle_wall, r_fillet);
}

module flat_rounded_cube(w, d, h, r=3) {
  minkowski() {
    cube([max(1, w - 2*r), max(1, d - 2*r), h]);
    cylinder(r=r, h=0.01, $fn=24);
  }
}

translate([0, 0, base_h]) {
  back_support();
  lip();
}
base_plate();

\end{lstlisting}

\subsubsection*{Checkpoint}\label{docs__pandoc__latex__src__3dmake_foundation__lessons_3dmake_7__lessons_3dmake_7.md__checkpoint}

\begin{itemize}
\tightlist
\item
  After step 2: \texttt{echo()} outputs confirm trig function results.
\item
  After step 3: The \texttt{rounded\_cube} module produces a smooth-edged box (check in preview).
\item
  After step 5: The phone stand stands at the correct angle and the lip depth matches your \texttt{lip\_h} parameter.
\end{itemize}

\subsection*{Advanced Transform Patterns}\label{docs__pandoc__latex__src__3dmake_foundation__lessons_3dmake_7__lessons_3dmake_7.md__advanced-transform-patterns}

\subsubsection*{Vector Math Functions}\label{docs__pandoc__latex__src__3dmake_foundation__lessons_3dmake_7__lessons_3dmake_7.md__vector-math-functions}

\begin{lstlisting}[style=Alabaster, language=openscad]
// dot product: measures how aligned two vectors are
a = [1, 0, 0];
b = [0.7, 0.7, 0];
echo(a * b);         // scalar dot product = 0.7

// cross product: finds a vector perpendicular to two input vectors
c = cross([1, 0, 0], [0, 1, 0]);  // = [0, 0, 1]

// norm: vector magnitude (length)
v = [3, 4, 0];
echo(norm(v));       // 5 (Pythagorean theorem)

\end{lstlisting}

\subsubsection*{Adaptive Quality with \$preview}\label{docs__pandoc__latex__src__3dmake_foundation__lessons_3dmake_7__lessons_3dmake_7.md__adaptive-quality-with-preview}

\begin{lstlisting}[style=Alabaster, language=openscad]
// Use lower $fn during preview for fast rendering, higher for export
$fn = $preview ? 16 : 64;

// This speeds up preview without compromising final export quality
sphere(r=20, $fn=$fn);

\end{lstlisting}

\subsection*{Quiz --- Lesson 3dMake.7 (15 questions)}\label{docs__pandoc__latex__src__3dmake_foundation__lessons_3dmake_7__lessons_3dmake_7.md__quiz--lesson-3dmake7-15-questions}

\begin{enumerate}
\tightlist
\item
  In OpenSCAD, do transforms apply left-to-right or right-to-left (innermost or outermost first)?
\item
  What is the result of \texttt{atan(1)}?
\item
  What is the advantage of \texttt{atan2(y,\ x)} over \texttt{atan(y/x)}?
\item
  What does \texttt{minkowski()} do geometrically?
\item
  What does \texttt{mirror({[}1,\ 0,\ 0{]})} do?
\item
  What does \texttt{scale({[}1,\ 1,\ 2{]})} do to a sphere?
\item
  Write the code to place 6 cylinders evenly spaced around a circle of radius 25 mm.
\item
  What does \texttt{norm({[}3,\ 4,\ 0{]})} return?
\item
  True or False: \texttt{translate({[}10,0,0{]})\ rotate({[}0,0,45{]})\ cube(5)} rotates the cube and then moves it.
\item
  Why would you use \texttt{\$preview} to conditionally set \texttt{\$fn}?
\item
  Describe the difference between \texttt{atan(y/x)} and \texttt{atan2(y,\ x)}. In which quadrant does \texttt{atan()} fail to return the correct angle?
\item
  A phone stand design uses \texttt{rotate({[}angle\ -\ 90,\ 0,\ 0{]})} to tilt the back support. If \texttt{angle\ =\ 65}, what is the actual rotation applied?
\item
  What does \texttt{cross({[}1,0,0{]},\ {[}0,1,0{]})} return, and what is the geometric meaning of the cross product?
\item
  Explain why using \texttt{minkowski()} with a sphere produces a different edge profile than using \texttt{minkowski()} with a cylinder.
\item
  A design requires placing components at the vertices of an equilateral triangle centered at the origin. Using \texttt{cos()} and \texttt{sin()}, calculate the coordinates of all three vertices for a circumradius of 30 mm.
\end{enumerate}

\subsection*{Extension Problems (15)}\label{docs__pandoc__latex__src__3dmake_foundation__lessons_3dmake_7__lessons_3dmake_7.md__extension-problems-15}

\begin{enumerate}
\tightlist
\item
  Redesign your phone stand with a cable slot cut through the base. Make the slot width a parameter.
\item
  Add rubber feet pockets to the base of your phone stand: four small rectangular cutouts that could hold adhesive rubber bumpers.
\item
  Create a phone stand variant that holds the phone in landscape (horizontal) orientation. What parameters change?
\item
  Build a radially symmetric decoration: 12 identical fins evenly spaced around a central cylinder using a \texttt{for} loop and \texttt{rotate()}.
\item
  Use \texttt{atan2()} to compute the angle of a slope and apply it with \texttt{rotate()} to align a part precisely with the slope.
\item
  Design a phone stand for a large tablet: update parameters and verify the design still holds at the correct angle.
\item
  Build a "compound arm": three rigid links connected at joints, each rotated by a parameter angle. Display all three links as an assembly.
\item
  Create an ergonomic keyboard wrist rest using \texttt{minkowski()} with an ellipsoid for smooth contouring.
\item
  Demonstrate transform order: build a visual example that shows the difference between \texttt{rotate\ →\ translate} and \texttt{translate\ →\ rotate} using colored shapes.
\item
  Redesign the phone stand as a wall-mount: replace the base with a plate that has two mounting holes, positioned so the phone faces outward.
\item
  Build a mirror-symmetric earring holder: design one arm with pegs, then use \texttt{mirror()} to create the symmetric pair.
\item
  Use \texttt{norm()} to calculate the diagonal of a rectangular prism and use that value to size a through-hole in the design.
\item
  Research OpenSCAD\textquotesingle s \texttt{multmatrix()} function. Build a shear transformation example that could not be achieved with \texttt{translate()} and \texttt{rotate()} alone.
\item
  Add snap-fit clips to the phone stand: two small flexible fingers on either side of the cradle that hold the phone securely. Use the snap-fit principles from Lesson 8.
\item
  Design a dual-phone stand using a \texttt{for} loop and \texttt{translate()} to create two side-by-side stands at different angles, parametrically spaced.
\end{enumerate}

\subsection*{References and Helpful Resources}\label{docs__pandoc__latex__src__3dmake_foundation__lessons_3dmake_7__lessons_3dmake_7.md__references-and-helpful-resources}

\subsubsection*{Supplemental Resources}\label{docs__pandoc__latex__src__3dmake_foundation__lessons_3dmake_7__lessons_3dmake_7.md__supplemental-resources}

\begin{itemize}
\tightlist
\item
  \href{docs/pandoc/latex/src/assets/Programming_with_OpenSCAD.epub}{Programming with OpenSCAD EPUB Textbook} --- Transforms chapter
\item
  \href{https://github.com/ProgrammingWithOpenSCAD/CodeSolutions}{CodeSolutions Repository} --- Phone stand worked example
\item
  \href{https://programmingwithopenscad.github.io/quick-reference.html}{OpenSCAD Quick Reference} --- Math and transform functions
\end{itemize}

\section{Lesson 8: Advanced Parametric Design and Interlocking Features}\label{docs__pandoc__latex__src__3dmake_foundation__lessons_3dmake_8__lessons_3dmake_8.md__lesson-8-advanced-parametric-design-and-interlocking-features}

Estimated time: 90--120 minutes

\subsection*{Learning Objectives}\label{docs__pandoc__latex__src__3dmake_foundation__lessons_3dmake_8__lessons_3dmake_8.md__learning-objectives}

\begin{itemize}
\tightlist
\item
  Apply tolerance and clearance concepts for press-fit and slip-fit assemblies
\item
  Design snap-fit clips using cantilever beam principles
\item
  Create interlocking stackable assemblies
\item
  Use threaded insert pockets for secure metal-to-plastic fastening
\item
  Apply chamfers and true chamfer geometry
\end{itemize}

\subsection*{Materials}\label{docs__pandoc__latex__src__3dmake_foundation__lessons_3dmake_8__lessons_3dmake_8.md__materials}

\begin{itemize}
\tightlist
\item
  3dMake project
\item
  Terminal and calipers
\item
  Reference: standard tolerance table (0.1--0.4 mm clearance guide)
\end{itemize}

\subsection*{Step-by-step Tasks}\label{docs__pandoc__latex__src__3dmake_foundation__lessons_3dmake_8__lessons_3dmake_8.md__step-by-step-tasks}

\subsubsection*{\texorpdfstring{1. Understand Tolerance and Clearance \footnote{Tolerance and Fit in FDM Printing --- All3DP Guide. \url{https://all3dp.com/2/fdm-3d-printing-tolerances/}}}{1. Understand Tolerance and Clearance }}\label{docs__pandoc__latex__src__3dmake_foundation__lessons_3dmake_8__lessons_3dmake_8.md__1-understand-tolerance-and-clearance-}

In FDM printing, dimensions come out slightly different from the digital model due to thermal expansion, layer bonding, and material shrinkage:

\begin{lstlisting}[style=Alabaster, language=openscad]
// Tolerance guide for FDM with 0.4mm nozzle, PLA
// Press-fit (tight):  clearance = 0.0 – 0.1 mm
// Slip-fit (smooth):  clearance = 0.2 – 0.3 mm
// Loose/clearance:    clearance = 0.4 – 0.5 mm

nominal_peg = 10;          // design intent: 10mm peg

press_peg  = nominal_peg - 0.1;   // 9.9mm — tight, requires force to insert
slip_peg   = nominal_peg - 0.2;   // 9.8mm — smooth sliding fit
clear_peg  = nominal_peg - 0.4;   // 9.6mm — loose, rattles slightly

\end{lstlisting}

Always print a tolerance test before committing to a design with many mating parts.

\subsubsection*{\texorpdfstring{2. Design a Stackable Bin System \footnote{OpenSCAD User Manual --- Difference and Boolean Operations. \url{https://en.wikibooks.org/wiki/OpenSCAD_User_Manual/CSG_Modelling}}}{2. Design a Stackable Bin System }}\label{docs__pandoc__latex__src__3dmake_foundation__lessons_3dmake_8__lessons_3dmake_8.md__2-design-a-stackable-bin-system-}

\begin{lstlisting}[style=Alabaster, language=openscad]
// Parametric stackable bin
bin_w     = 80;    // mm
bin_d     = 60;    // mm
bin_h     = 40;    // mm
wall      = 2.5;   // mm
lip_h     = 5;     // mm — height of stacking lip
lip_clear = 0.25;  // mm — slip-fit clearance for stacking

module bin_body() {
  difference() {
    cube([bin_w, bin_d, bin_h]);
    translate([wall, wall, wall])
      cube([bin_w - 2*wall, bin_d - 2*wall, bin_h]);  // open top
  }
}

// Outer lip that fits into the base of the bin above
module stacking_lip() {
  translate([lip_clear, lip_clear, bin_h])
    difference() {
      cube([bin_w - 2*lip_clear, bin_d - 2*lip_clear, lip_h]);
      translate([wall - lip_clear, wall - lip_clear, wall])
        cube([bin_w - 2*wall, bin_d - 2*wall, lip_h]);
    }
}

bin_body();
stacking_lip();

\end{lstlisting}

\subsubsection*{\texorpdfstring{3. Design a Snap-Fit Clip \footnote{Snap-Fit Design Principles --- Bayer MaterialScience Snap-Fit Design Manual (widely cited industry reference). Key principle: cantilever beam deflection formula governs clip strength and flexibility.}}{3. Design a Snap-Fit Clip }}\label{docs__pandoc__latex__src__3dmake_foundation__lessons_3dmake_8__lessons_3dmake_8.md__3-design-a-snap-fit-clip-}

Snap-fits work by deflecting a flexible cantilever beam. The beam deforms during assembly and springs back to lock in place.\footnote{Snap-Fit Design Principles --- Bayer MaterialScience Snap-Fit Design Manual (widely cited industry reference). Key principle: cantilever beam deflection formula governs clip strength and flexibility.}

\begin{lstlisting}[style=Alabaster, language=openscad]
// Snap-fit clip — cantilever beam design
clip_w     = 20;    // mm — clip width
beam_l     = 12;    // mm — beam length (longer = more flexible)
beam_t     = 1.2;   // mm — beam thickness (thinner = easier snap)
barb_h     = 1.5;   // mm — retention barb height

module snap_clip() {
  union() {
    // Base anchor
    cube([clip_w, 5, 5]);
    // Flexible beam
    translate([0, 5, 0])
      cube([clip_w, beam_l, beam_t]);
    // Retention barb (tapered for easier entry)
    translate([0, 5 + beam_l - 0.001, 0])
      cube([clip_w, barb_h, barb_h]);
  }
}

snap_clip();

\end{lstlisting}

Key design rule: for PLA, keep \texttt{beam\_t\ /\ beam\_l\ \textless{}\ 0.15} for adequate flexibility.

\subsubsection*{4. Create Threaded Insert Pockets}\label{docs__pandoc__latex__src__3dmake_foundation__lessons_3dmake_8__lessons_3dmake_8.md__4-create-threaded-insert-pockets}

Heat-set threaded inserts provide strong, reusable metal threads in plastic parts.

\begin{lstlisting}[style=Alabaster, language=openscad]
// Threaded insert pocket for M3 brass heat-set insert
// Typical M3 brass insert: OD ~4.5mm, length ~5.7mm
// Printed hole: OD + 0.1mm for press-in tolerance

module m3_insert_pocket(depth=6) {
  // Straight hole sized for brass insert
  cylinder(r=2.35, h=depth + 0.001, $fn=16);  // (4.5mm OD + 0.1) / 2
}

module m4_insert_pocket(depth=8) {
  // M4 brass insert: typical OD ~5.6mm
  cylinder(r=2.9, h=depth + 0.001, $fn=16);  // (5.6mm + 0.2) / 2
}

// Apply to a part
difference() {
  cube([40, 30, 10]);
  translate([10, 15, -0.001]) m3_insert_pocket(depth=10.002);
  translate([30, 15, -0.001]) m3_insert_pocket(depth=10.002);
}

\end{lstlisting}

\subsubsection*{5. Apply True Chamfers}\label{docs__pandoc__latex__src__3dmake_foundation__lessons_3dmake_8__lessons_3dmake_8.md__5-apply-true-chamfers}

A chamfer is a 45° angled cut at an edge. The correct way to create one in OpenSCAD is with a rotated and translated cutting cylinder or prism --- not just cropping the top of a part.

\begin{lstlisting}[style=Alabaster, language=openscad]
// True chamfer using a difference cut
// Chamfer size c = horizontal distance of chamfer at 45 degrees
module chamfered_box(w, d, h, c=2) {
  difference() {
    cube([w, d, h]);
    // Top edge chamfer (45-degree cut around all four top edges)
    // Cut using a rotate_extrude approach or a hull of edges
    translate([0, 0, h - c])
      difference() {
        cube([w, d, c + 0.001]);
        translate([c, c, 0]) cube([w - 2*c, d - 2*c, c + 0.001]);
      }
  }
}

// Note: the above chamfers the top rim by removing the outer corners.
// For a true 45-degree surface, use:
module edge_chamfer_45(w, d, h, c=2) {
  difference() {
    cube([w, d, h]);
    for (x=[0, w], y=[0, d]) {
      translate([x, y, h - c])
        rotate([0, 0, 45])
          cube([c*2, c*2, c + 0.001], center=true);
    }
  }
}

chamfered_box(50, 40, 20, c=3);

\end{lstlisting}

\subsubsection*{Checkpoint}\label{docs__pandoc__latex__src__3dmake_foundation__lessons_3dmake_8__lessons_3dmake_8.md__checkpoint}

\begin{itemize}
\tightlist
\item
  After step 1: You can explain the difference between press-fit and slip-fit clearance values.
\item
  After step 3: Your snap-fit clip deflects when you check the beam\_t / beam\_l ratio.
\item
  After step 5: Your chamfered box shows angled edges in the slicer preview (not just a flat crop).
\end{itemize}

\subsubsection*{Design Notes}\label{docs__pandoc__latex__src__3dmake_foundation__lessons_3dmake_8__lessons_3dmake_8.md__design-notes}

\begin{itemize}
\tightlist
\item
  Tolerance values (0.1--0.4 mm) are starting points. Always print and test before finalizing a mating design.
\item
  Snap-fit geometry is highly material-dependent: PLA is brittle; PETG and nylon are more resilient.
\item
  Threaded insert dimensions vary by manufacturer --- verify against your actual inserts before printing.
\end{itemize}

\subsection*{Dowel Pins for Alignment}\label{docs__pandoc__latex__src__3dmake_foundation__lessons_3dmake_8__lessons_3dmake_8.md__dowel-pins-for-alignment}

For assemblies that must align precisely without a press-fit, dowel pins provide passive alignment:

\begin{lstlisting}[style=Alabaster, language=openscad]
// Alignment pin and pocket system
pin_r    = 3;
pin_h    = 6;
pin_clear = 0.15;  // slip-fit for alignment pin

module alignment_pin() {
  cylinder(r=pin_r, h=pin_h, $fn=32);
}

module alignment_pocket() {
  cylinder(r=pin_r + pin_clear, h=pin_h + 0.5, $fn=32);
}

// Part A: has pins
difference() {
  cube([50, 40, 5]);
  // holes for other features
}
translate([10, 10, 5]) alignment_pin();
translate([40, 30, 5]) alignment_pin();

// Part B: has pockets (shown offset for clarity)
translate([60, 0, 0])
  difference() {
    cube([50, 40, 5]);
    translate([10, 10, -0.001]) alignment_pocket();
    translate([40, 30, -0.001]) alignment_pocket();
  }

\end{lstlisting}

\subsection*{Quiz --- Lesson 3dMake.8 (15 questions)}\label{docs__pandoc__latex__src__3dmake_foundation__lessons_3dmake_8__lessons_3dmake_8.md__quiz--lesson-3dmake8-15-questions}

\begin{enumerate}
\tightlist
\item
  What clearance range (in mm) produces a slip-fit between two FDM-printed mating parts?
\item
  What is a press-fit, and when would you use one?
\item
  What is a snap-fit clip, and what material property makes it work?
\item
  What is the purpose of a retention barb on a snap-fit beam?
\item
  What is a heat-set threaded insert, and why is it preferred over a tapped plastic hole?
\item
  What is the \texttt{beam\_t\ /\ beam\_l} rule for PLA snap-fit beams?
\item
  What is a chamfer, and how does it differ from a fillet?
\item
  What does a dowel pin achieve in an assembly design?
\item
  True or False: a thicker snap-fit beam is easier to deflect than a thinner one.
\item
  What happens if you design a stacking lip without clearance (clearance = 0)?
\item
  Explain tolerance stack-up: if three parts each have ±0.2 mm dimensional tolerance, what is the worst-case total variation when they are assembled in a chain?
\item
  What is the typical outer diameter of an M3 brass heat-set insert, and what printed hole diameter would you use for a press-in fit?
\item
  Explain the difference between a cropped top edge and a true 45-degree chamfer. Why does the distinction matter for printed aesthetics?
\item
  Describe how you would design and test a tolerance set for a new printer. What measurements would you take and how would you choose the final clearance value?
\item
  For a snap-fit clip made of PLA with \texttt{beam\_l\ =\ 15\ mm}, what is the maximum \texttt{beam\_t} that satisfies the flexibility rule? Show your calculation.
\end{enumerate}

\subsection*{Extension Problems (15)}\label{docs__pandoc__latex__src__3dmake_foundation__lessons_3dmake_8__lessons_3dmake_8.md__extension-problems-15}

\begin{enumerate}
\tightlist
\item
  Design a complete stackable bin set (small, medium, large) where each size nests inside the next. Print and test all three.
\item
  Build a snap-fit box lid that attaches with four clips. Make the number of clips a parameter.
\item
  Design a press-fit peg and socket system; print 5 pairs with clearances from 0.0--0.4 mm and document which clearance gives the best press-fit for your printer.
\item
  Create a wall-mount organizer using snap-fit clips that attach to a standard pegboard (pegboard hole pitch = 25.4 mm, hole diameter = 6.35 mm).
\item
  Build a two-part enclosure with threaded insert pockets for M3 screws; test with real hardware.
\item
  Design a multi-part snap-together dice tray: the tray snaps into a carrying case lid.
\item
  Create an alignment jig using dowel pins and pockets; use it to verify your printer\textquotesingle s dimensional accuracy.
\item
  Build a cable management clip with a snap-fit that opens and closes for easy cable insertion/removal.
\item
  Design a telescoping tube using a slip-fit: an inner cylinder slides freely inside an outer cylinder.
\item
  Create a "bayonet mount" that locks with a 45-degree twist.
\item
  Design a spring-loaded mechanism using only FDM-printed parts; document the material choice and how the spring geometry provides the spring force.
\item
  Build a parametric drawer divider system: interlocking slotted dividers that can be arranged in any grid pattern.
\item
  Design a self-locking dovetail joint that slides together in one direction and cannot be pulled apart.
\item
  Create a complete assembly drawing for a two-part design: an exploded OpenSCAD render showing each component with a \texttt{translate()} offset, plus a tolerance table.
\item
  Conduct a failure mode analysis for a snap-fit clip: list five ways the clip could fail during assembly or use, assign a likelihood and severity to each, and propose a design change to mitigate the top two risks.
\end{enumerate}

\subsection*{References and Helpful Resources}\label{docs__pandoc__latex__src__3dmake_foundation__lessons_3dmake_8__lessons_3dmake_8.md__references-and-helpful-resources}

\subsubsection*{Supplemental Resources}\label{docs__pandoc__latex__src__3dmake_foundation__lessons_3dmake_8__lessons_3dmake_8.md__supplemental-resources}

\begin{itemize}
\tightlist
\item
  \href{docs/pandoc/latex/src/assets/Programming_with_OpenSCAD.epub}{Programming with OpenSCAD EPUB Textbook} --- Assemblies and interlocking features
\item
  \href{https://github.com/mrhunsaker/VI_3DMake_OpenSCAD_Curriculum/3dMake_Foundation/Lessons_3dMake_8/../../assets/3dMake_Foundation/measurement_worksheet.md}{Measurement Worksheet} --- Tolerance testing worksheet
\item
  \href{https://github.com/tdeck/3dmake}{3DMake GitHub Repository} --- Build workflow reference
\end{itemize}

\subsection{Snap-Fit Clip - Extension Project}\label{docs__pandoc__latex__src__3dmake_foundation__lessons_3dmake_8__snap-fit-clip.md__3dmake_foundation_lessons_3dmake_8-snap_fit_clip}

Estimated time: 3-6 hours

\subsubsection*{Learning Objectives}\label{docs__pandoc__latex__src__3dmake_foundation__lessons_3dmake_8__snap-fit-clip.md__learning-objectives}

By completing this project, you will:

\begin{itemize}
\tightlist
\item
  Design flexible components that must bend without failing
\item
  Conduct tolerance and material testing systematically
\item
  Understand the relationship between geometry, material properties, and functional performance
\item
  Document design iterations and trade-offs
\end{itemize}

\subsubsection*{Objective}\label{docs__pandoc__latex__src__3dmake_foundation__lessons_3dmake_8__snap-fit-clip.md__objective}

\begin{itemize}
\tightlist
\item
  Design a snap-fit clip with customizable clip thickness and spacing to determine optimal tolerances for your printer and material.
\end{itemize}

\subsubsection*{Tasks}\label{docs__pandoc__latex__src__3dmake_foundation__lessons_3dmake_8__snap-fit-clip.md__tasks}

\begin{enumerate}
\tightlist
\item
  Design a parametric snap-fit test piece with adjustable clip thickness and spacing.
\item
  Print test pieces for three tolerance values (adjust clip geometry for each variant).
\item
  Test each clip for snap-fit behavior and document success/failure modes.
\item
  Refine dimensions based on testing results and create final clip design.
\end{enumerate}

\subsubsection*{Deliverable}\label{docs__pandoc__latex__src__3dmake_foundation__lessons_3dmake_8__snap-fit-clip.md__deliverable}

\begin{itemize}
\tightlist
\item
  Source \texttt{.scad} file with parametric clip modules
\item
  Test print results and tolerance documentation
\item
  Final clip design with recommended print settings
\end{itemize}

\subsubsection*{Starter files}\label{docs__pandoc__latex__src__3dmake_foundation__lessons_3dmake_8__snap-fit-clip.md__starter-files}

\begin{itemize}
\tightlist
\item
  \href{docs/pandoc/latex/src/assets/Extension_Projects/Snap_Fit_Clip/starter.scad}{starter.scad} - minimal snap-fit test scaffold.
\end{itemize}

\subsubsection*{Assessment Questions (Optional)}\label{docs__pandoc__latex__src__3dmake_foundation__lessons_3dmake_8__snap-fit-clip.md__assessment-questions-optional}

\begin{enumerate}
\tightlist
\item
  What clip thickness and spacing values resulted in a reliable snap-fit for your printer?
\item
  How did you balance flexibility (easy to snap/unsnap) with durability (no premature failure)?
\item
  What would you change in the design if this clip needed to work reliably 1,000+ times?
\end{enumerate}

\subsubsection*{Starter Code}\label{docs__pandoc__latex__src__3dmake_foundation__lessons_3dmake_8__snap-fit-clip.md__starter-code}

Use this stackable bins example as your starting point:

\begin{lstlisting}[style=Alabaster, language=openscad]
// Stackable Storage Bins - Advanced Example
// Parameters
binw = 80;      // width
bind = 120;     // depth
binh = 60;      // height
wall = 2;        // wall thickness
rim = 3;         // interlock rim height
chamfer = 2;
stackclear = 0.6; // clearance between stacks
module outer(){
  cube([binw, bind, binh]);
}
module inner(){
  translate([wall, wall, wall])
    cube([binw-2*wall, bind-2*wall, binh-2*wall]);
}
module body(){
  difference(){ outer(); inner(); }
}
module rimouter(){
  translate([0,0,binh]) cube([binw, bind, rim]);
}
module riminner(){
  translate([wall+stackclear, wall+stackclear, binh])
    cube([binw-2*(wall+stackclear), bind-2*(wall+stackclear), rim]);
}
module chamferedges(){
  // Simple top edge chamfer via difference
  difference(){
    children();
    translate([-1,-1,binh-chamfer]) cube([binw+2, bind+2, chamfer+2]);
  }
}
union(){
  chamferedges(){ body(); }
  rimouter();
  difference(){ rimouter(); riminner(); }
}

\end{lstlisting}

\subsection{Snap-Fit Clip - Student Documentation Template (Extension Project)}\label{docs__pandoc__latex__src__3dmake_foundation__lessons_3dmake_8__snap_fit_clip_student_template.md__3dmake_foundation_lessons_3dmake_8-snap_fit_clip_student_template}

\begin{itemize}
\tightlist
\item
  Author:
\item
  Date:
\item
  Description: Design a snap-fit connector that joins two 3D-printed parts without external fasteners.
\end{itemize}

\subsubsection*{Design Concept}\label{docs__pandoc__latex__src__3dmake_foundation__lessons_3dmake_8__snap_fit_clip_student_template.md__design-concept}

\begin{itemize}
\tightlist
\item
  What two parts will your snap-fit connect?
\item
  Design approach (cantilever, torsion box, other):
\item
  Material and expected stress limits:
\end{itemize}

\subsubsection*{Tolerance Specifications}\label{docs__pandoc__latex__src__3dmake_foundation__lessons_3dmake_8__snap_fit_clip_student_template.md__tolerance-specifications}

{\def\LTcaptype{none} % do not increment counter
\begin{longtable}[]{@{}llll@{}}
\toprule\noalign{}
Parameter & Nominal Value & Tolerance & Rationale \\
\midrule\noalign{}
\endhead
\bottomrule\noalign{}
\endlastfoot
Tab thickness & & & \\
Clip opening & & & \\
Undercut depth & & & \\
Contact area & & & \\
\end{longtable}
}

\subsubsection*{Assembly Testing}\label{docs__pandoc__latex__src__3dmake_foundation__lessons_3dmake_8__snap_fit_clip_student_template.md__assembly-testing}

\paragraph*{Test 1: Initial Assembly}\label{docs__pandoc__latex__src__3dmake_foundation__lessons_3dmake_8__snap_fit_clip_student_template.md__test-1-initial-assembly}

\begin{itemize}
\tightlist
\item
  Date:
\item
  Effort required to snap together:
\item
  Snap strength (resistance to pulling apart):
\item
  Visual inspection (any deformation?):
\end{itemize}

\paragraph*{Test 2-5: Reusability Testing}\label{docs__pandoc__latex__src__3dmake_foundation__lessons_3dmake_8__snap_fit_clip_student_template.md__test-2-5-reusability-testing}

\begin{itemize}
\tightlist
\item
  Cycle 2: (date, observations)
\item
  Cycle 3: (date, observations)
\item
  Cycle 4: (date, observations)
\item
  Cycle 5: (date, observations)
\end{itemize}

\paragraph*{Durability Assessment}\label{docs__pandoc__latex__src__3dmake_foundation__lessons_3dmake_8__snap_fit_clip_student_template.md__durability-assessment}

\begin{itemize}
\tightlist
\item
  After 5 cycles, is the snap-fit still functional?
\item
  Any permanent deformation observed?
\item
  Estimated service life (based on testing):
\end{itemize}

\subsubsection*{Mechanical Analysis}\label{docs__pandoc__latex__src__3dmake_foundation__lessons_3dmake_8__snap_fit_clip_student_template.md__mechanical-analysis}

\begin{itemize}
\tightlist
\item
  Describe the snap mechanism (why does it work?):
\item
  What material properties enable this design?
\item
  What are the stress points?
\end{itemize}

\subsubsection*{Reflections}\label{docs__pandoc__latex__src__3dmake_foundation__lessons_3dmake_8__snap_fit_clip_student_template.md__reflections}

\begin{itemize}
\tightlist
\item
  Did the snap-fit work as expected?
\item
  What tolerance adjustments would you make?
\item
  How would you test durability over thousands of cycles?
\item
  What materials would fail with this design? Why?
\end{itemize}

\subsubsection*{Attachments}\label{docs__pandoc__latex__src__3dmake_foundation__lessons_3dmake_8__snap_fit_clip_student_template.md__attachments}

\begin{itemize}
\tightlist
\item[$\square$]
  \texttt{.scad} file with parametric snap-fit module
\item[$\square$]
  Photos of both parts before assembly
\item[$\square$]
  Photos of assembled connection
\item[$\square$]
  Testing documentation with dates/observations
\item[$\square$]
  Tolerance analysis table
\end{itemize}

\subsubsection*{Teacher Feedback}\label{docs__pandoc__latex__src__3dmake_foundation__lessons_3dmake_8__snap_fit_clip_student_template.md__teacher-feedback}

{\def\LTcaptype{none} % do not increment counter
\begin{longtable}[]{@{}lll@{}}
\toprule\noalign{}
Category & Score & Notes \\
\midrule\noalign{}
\endhead
\bottomrule\noalign{}
\endlastfoot
Problem \& Solution (0-3) & & \\
Design \& Code Quality (0-3) & & \\
Documentation (0-3) & & \\
Total (0-9) & & \\
\end{longtable}
}

\subsection{Snap-Fit Clip - Teacher Template (Extension Project)}\label{docs__pandoc__latex__src__3dmake_foundation__lessons_3dmake_8__snap_fit_clip_teacher_template.md__3dmake_foundation_lessons_3dmake_8-snap_fit_clip_teacher_template}

\subsubsection*{Briefing}\label{docs__pandoc__latex__src__3dmake_foundation__lessons_3dmake_8__snap_fit_clip_teacher_template.md__briefing}

Students design a snap-fit connector that joins two 3D-printed parts without fasteners. This project emphasizes tolerance engineering, material properties, and mechanical design validation.

Key Learning: Tolerance engineering; snap-fit mechanics; non-destructive assembly testing.

Real-world Connection: Snap-fits are ubiquitous in consumer products. Designing them requires understanding material stress, geometry precision, and repeated-use durability.

\subsubsection*{Constraints}\label{docs__pandoc__latex__src__3dmake_foundation__lessons_3dmake_8__snap_fit_clip_teacher_template.md__constraints}

\begin{itemize}
\tightlist
\item
  Snap-fit must connect two printed parts without external fasteners
\item
  Design must be parametric (tolerance variables clearly labeled)
\item
  Assembly must be testable (connection strength, reusability)
\item
  Design must account for material properties (PLA creep, flex limits)
\end{itemize}

\subsubsection*{Functional Requirements}\label{docs__pandoc__latex__src__3dmake_foundation__lessons_3dmake_8__snap_fit_clip_teacher_template.md__functional-requirements}

\begin{itemize}
\tightlist
\item
  Parts snap together securely without over-constraint
\item
  Connection withstands intended use without permanent deformation
\item
  Snap mechanism is reusable (at least 5+ assembly cycles)
\item
  Design demonstrates understanding of tolerance and material behavior
\end{itemize}

\subsubsection*{Deliverables}\label{docs__pandoc__latex__src__3dmake_foundation__lessons_3dmake_8__snap_fit_clip_teacher_template.md__deliverables}

\begin{itemize}
\tightlist
\item
  \texttt{.scad} with parametric snap-fit module
\item
  Completed documentation template
\item
  Assembly testing log (snap strength, cycle durability)
\item
  Tolerance analysis documentation
\item
  Reflection on mechanical design trade-offs
\end{itemize}

\subsubsection*{Rubric}\label{docs__pandoc__latex__src__3dmake_foundation__lessons_3dmake_8__snap_fit_clip_teacher_template.md__rubric}

\paragraph*{Category 1: Problem \& Solution (0-3)}\label{docs__pandoc__latex__src__3dmake_foundation__lessons_3dmake_8__snap_fit_clip_teacher_template.md__category-1-problem--solution-0-3}

Snap-fit works reliably; parts assemble and disassemble as designed.

\paragraph*{Category 2: Design \& Code Quality (0-3)}\label{docs__pandoc__latex__src__3dmake_foundation__lessons_3dmake_8__snap_fit_clip_teacher_template.md__category-2-design--code-quality-0-3}

Code shows tolerance thinking. Design is mechanically sound.

\paragraph*{Category 3: Documentation (0-3)}\label{docs__pandoc__latex__src__3dmake_foundation__lessons_3dmake_8__snap_fit_clip_teacher_template.md__category-3-documentation-0-3}

Assembly testing documented. Tolerance rationale explained.

\subsubsection*{Assessment Notes}\label{docs__pandoc__latex__src__3dmake_foundation__lessons_3dmake_8__snap_fit_clip_teacher_template.md__assessment-notes}

\begin{itemize}
\tightlist
\item
  Strong submissions: Show evidence of tolerance testing, multiple assembly cycles documented, and reflection on material limits
\item
  Reinforce: Snap-fit design principles; tolerancing for manufacturability
\item
  Extension: Material property analysis; cost-benefit comparisons
\end{itemize}

\section{Lesson 9: Automation and 3dm Workflows}\label{docs__pandoc__latex__src__3dmake_foundation__lessons_3dmake_9__lessons_3dmake_9.md__lesson-9-automation-and-3dm-workflows}

Estimated time: 90--120 minutes

\subsection*{Learning Objectives}\label{docs__pandoc__latex__src__3dmake_foundation__lessons_3dmake_9__lessons_3dmake_9.md__learning-objectives}

\begin{itemize}
\tightlist
\item
  Write bash scripts to automate repetitive 3dMake build tasks
\item
  Use \texttt{import()}, \texttt{include}, and \texttt{use} to manage multi-file projects
\item
  Automate parameter sweeps to generate STL variant sets
\item
  Use file system tools to organize and archive builds
\item
  Understand how to detect file changes for watch-and-rebuild loops
\end{itemize}

\subsection*{Materials}\label{docs__pandoc__latex__src__3dmake_foundation__lessons_3dmake_9__lessons_3dmake_9.md__materials}

\begin{itemize}
\tightlist
\item
  3dMake project
\item
  Terminal (bash)
\item
  Text editor
\end{itemize}

\subsection*{Step-by-step Tasks}\label{docs__pandoc__latex__src__3dmake_foundation__lessons_3dmake_9__lessons_3dmake_9.md__step-by-step-tasks}

\subsubsection*{\texorpdfstring{1. Write Your First Build Automation Script \footnote{3DMake GitHub Repository --- Source code and README with full command reference. \url{https://github.com/tdeck/3dmake}}}{1. Write Your First Build Automation Script }}\label{docs__pandoc__latex__src__3dmake_foundation__lessons_3dmake_9__lessons_3dmake_9.md__1-write-your-first-build-automation-script-}

\begin{lstlisting}[style=Alabaster, language=bash]
#!/bin/bash
# build_all.sh — build project and archive the result

set -e  # exit on any error

echo "=== Building project ==="
3dm build

echo "=== Checking output ==="
if [ -f build/main.stl ]; then
  echo "STL created: $(stat -c%s build/main.stl) bytes"
else
  echo "ERROR: build/main.stl not found"
  exit 1
fi

echo "=== Done ==="

\end{lstlisting}

Make it executable:

\begin{lstlisting}[style=Alabaster, language=bash]
chmod +x build_all.sh
./build_all.sh

\end{lstlisting}

\subsubsection*{\texorpdfstring{2. Automate Parameter Variants \footnote{OpenSCAD Command-Line Interface --- OpenSCAD User Manual. \url{https://en.wikibooks.org/wiki/OpenSCAD_User_Manual/Using_OpenSCAD_in_a_command_line_environment}. Documents the \texttt{-D} parameter override flag and other CLI options.}}{2. Automate Parameter Variants }}\label{docs__pandoc__latex__src__3dmake_foundation__lessons_3dmake_9__lessons_3dmake_9.md__2-automate-parameter-variants-}

Generate multiple STL files with different parameter values by using \texttt{sed} or OpenSCAD\textquotesingle s \texttt{-D} command-line override:

\begin{lstlisting}[style=Alabaster, language=bash]
#!/bin/bash
# generate_variants.sh — build variants with different widths

widths=(40 50 60 80)

mkdir -p build/variants

for w in "${widths[@]}"; do
  echo "Building width=${w}..."
  openscad -D "width=${w}" -o "build/variants/main_w${w}.stl" src/main.scad
  echo "  Created: build/variants/main_w${w}.stl"
done

echo "All variants built:"
ls -lh build/variants/

\end{lstlisting}

\subsubsection*{3. Use import() for Multi-File Projects}\label{docs__pandoc__latex__src__3dmake_foundation__lessons_3dmake_9__lessons_3dmake_9.md__3-use-import-for-multi-file-projects}

\begin{lstlisting}[style=Alabaster, language=openscad]
// Import an existing STL into your OpenSCAD assembly
// Useful for combining pre-built parts with new geometry
import("existing_part.stl");

// Or position it within an assembly
translate([50, 0, 0])
  import("../../assets/parts/bracket.stl");

\end{lstlisting}

For OpenSCAD library files, use \texttt{use} (modules and functions only) or \texttt{include} (executes full file):

\begin{lstlisting}[style=Alabaster, language=openscad]
use <my_library.scad>     // import modules/functions only
include <constants.scad>  // also execute top-level statements

\end{lstlisting}

\subsubsection*{4. Archive Builds with Timestamps}\label{docs__pandoc__latex__src__3dmake_foundation__lessons_3dmake_9__lessons_3dmake_9.md__4-archive-builds-with-timestamps}

\begin{lstlisting}[style=Alabaster, language=bash]
#!/bin/bash
# archive_build.sh — timestamped archive of build outputs

TIMESTAMP=$(date +%Y%m%d_%H%M%S)
ARCHIVE_DIR="archives/${TIMESTAMP}"

mkdir -p "$ARCHIVE_DIR"

# Copy build outputs
cp build/main.stl "$ARCHIVE_DIR/"
cp src/main.scad  "$ARCHIVE_DIR/"

# Record metadata
echo "Build archived: ${TIMESTAMP}" > "$ARCHIVE_DIR/README.txt"
echo "Git commit: $(git rev-parse --short HEAD 2>/dev/null || echo 'no git')" >> "$ARCHIVE_DIR/README.txt"

echo "Archived to: ${ARCHIVE_DIR}"

\end{lstlisting}

\subsubsection*{5. Watch-and-Rebuild Loop}\label{docs__pandoc__latex__src__3dmake_foundation__lessons_3dmake_9__lessons_3dmake_9.md__5-watch-and-rebuild-loop}

Automatically rebuild when source files change:

\begin{lstlisting}[style=Alabaster, language=bash]
#!/bin/bash
# watch_and_build.sh — rebuild when .scad files are modified

echo "Watching src/ for changes... (Ctrl+C to stop)"

LAST_MOD=""

while true; do
  # Find the most recently modified .scad file
  CURRENT_MOD=$(find src/ -name "*.scad" -newer build/main.stl 2>/dev/null | head -1)

  if [ -n "$CURRENT_MOD" ] && [ "$CURRENT_MOD" != "$LAST_MOD" ]; then
    echo "[$(date +%H:%M:%S)] Change detected in $CURRENT_MOD — rebuilding..."
    3dm build && echo "Build OK" || echo "Build FAILED"
    LAST_MOD="$CURRENT_MOD"
  fi

  sleep 2
done

\end{lstlisting}

Note: \texttt{find\ -newer} compares modification times. This script polls every 2 seconds; for production use, consider \texttt{inotifywait} (Linux) or \texttt{fswatch} (macOS) for event-driven watching.

\subsubsection*{Checkpoint}\label{docs__pandoc__latex__src__3dmake_foundation__lessons_3dmake_9__lessons_3dmake_9.md__checkpoint}

\begin{itemize}
\tightlist
\item
  After step 1: \texttt{./build\_all.sh} runs successfully and reports the STL file size.
\item
  After step 2: \texttt{build/variants/} contains one STL per width variant.
\item
  After step 5: When you save \texttt{src/main.scad}, the watch script triggers a rebuild within 2 seconds.
\end{itemize}

\subsection*{Bash Quick Reference for 3dMake Automation}\label{docs__pandoc__latex__src__3dmake_foundation__lessons_3dmake_9__lessons_3dmake_9.md__bash-quick-reference-for-3dmake-automation}

{\def\LTcaptype{none} % do not increment counter
\begin{longtable}[]{@{}lll@{}}
\toprule\noalign{}
Construct & Syntax & Example \\
\midrule\noalign{}
\endhead
\bottomrule\noalign{}
\endlastfoot
Variable & \texttt{VAR=value} & \texttt{OUT=build/main.stl} \\
Array & \texttt{ARR=(a\ b\ c)} & \texttt{widths=(40\ 50\ 60)} \\
For loop & \texttt{for\ x\ in\ "\$\{ARR{[}@{]}\}"} &
\texttt{for\ w\ in\ "\$\{widths{[}@{]}\}"} \\
If exists & \texttt{if\ {[}\ -f\ FILE\ {]}} &
\texttt{if\ {[}\ -f\ build/main.stl\ {]}} \\
String concat & \texttt{"\$\{VAR\}\_suffix"} &
\texttt{"main\_w\$\{w\}.stl"} \\
Exit on error & \texttt{set\ -e} & at top of script \\
Command subst. & \texttt{\$(command)} & \texttt{\$(date\ +\%Y\%m\%d)} \\
Redirect output & \texttt{command\ \textgreater{}\textgreater{}\ file} &
\texttt{echo\ "text"\ \textgreater{}\textgreater{}\ log.txt} \\
\end{longtable}
}

\subsection*{Quiz --- Lesson 3dMake.9 (15 questions)}\label{docs__pandoc__latex__src__3dmake_foundation__lessons_3dmake_9__lessons_3dmake_9.md__quiz--lesson-3dmake9-15-questions}

\begin{enumerate}
\tightlist
\item
  What does \texttt{chmod\ +x\ script.sh} do, and why is it needed?
\item
  What does \texttt{set\ -e} do in a bash script?
\item
  What is the difference between \texttt{import()} in OpenSCAD and \texttt{use} or \texttt{include}?
\item
  What does \texttt{\$(date\ +\%Y\%m\%d\_\%H\%M\%S)} produce in a bash script?
\item
  How does \texttt{openscad\ -D\ "width=50"} differ from editing the \texttt{.scad} file directly?
\item
  What does \texttt{find\ src/\ -name\ "*.scad"\ -newer\ build/main.stl} return?
\item
  True or False: \texttt{include\ \textless{}library.scad\textgreater{}} executes any top-level geometry in the library file.
\item
  What is the purpose of archiving builds with timestamps?
\item
  What limitation does \texttt{find\ -newer} have compared to \texttt{inotifywait}?
\item
  Write a bash one-liner that builds 3 variants (height=20, 30, 40) using \texttt{openscad\ -D}.
\item
  What does \texttt{\&\&} do in the line \texttt{3dm\ build\ \&\&\ echo\ "Build\ OK"\ \textbar{}\textbar{}\ echo\ "Build\ FAILED"}?
\item
  Explain the difference between \texttt{{[}\ -f\ FILE\ {]}} and \texttt{{[}\ -d\ DIR\ {]}} in bash conditionals.
\item
  Why would you use \texttt{git\ rev-parse\ -\/-short\ HEAD} in a build archive script?
\item
  Describe a scenario where a watch-and-rebuild loop would save significant time during iterative design.
\item
  What bash construct would you use to loop over a list of 10 filament names and print each one on a separate line?
\end{enumerate}

\subsection*{Extension Problems (15)}\label{docs__pandoc__latex__src__3dmake_foundation__lessons_3dmake_9__lessons_3dmake_9.md__extension-problems-15}

\begin{enumerate}
\tightlist
\item
  Write a \texttt{batch\_label.sh} script that builds label plates for a list of names (read from a text file), one STL per name.
\item
  Create a \texttt{clean.sh} script that deletes all \texttt{.stl} files in \texttt{build/} and all timestamped archive directories older than 7 days.
\item
  Build a variant comparison report: a script that generates 5 STL variants, records each file size, and writes a CSV summary.
\item
  Write a \texttt{deploy.sh} script that copies the current STL to a network-shared slicer folder and logs the transfer with a timestamp.
\item
  Create a \texttt{validate.sh} script that checks whether \texttt{build/main.stl} exists, has a file size \textgreater{} 1000 bytes, and was built in the last 5 minutes.
\item
  Build a multi-project build orchestrator: a script that loops over a list of project directories, runs \texttt{3dm\ build} in each, and reports success/failure.
\item
  Extend the watch-and-rebuild script to send a desktop notification (using \texttt{notify-send} on Linux) when a build succeeds or fails.
\item
  Create a "parameter sweep" runner that generates a 3D matrix of variants: all combinations of 3 widths × 3 heights × 2 wall thicknesses (18 STLs total).
\item
  Write a \texttt{git\_checkpoint.sh} script that runs \texttt{3dm\ build}, then commits \texttt{src/main.scad} and \texttt{build/main.stl} to git with an auto-generated commit message.
\item
  Build a \texttt{diff\_variants.sh} script that compares file sizes of all STLs in \texttt{build/variants/} and flags any that are more than 20\% different from the median.
\item
  Research \texttt{inotifywait} (Linux). Rewrite the watch-and-rebuild loop from step 5 to use \texttt{inotifywait} instead of \texttt{find\ -newer}. Document the advantages of event-driven vs. polling.
\item
  Create a build log: append each build\textquotesingle s timestamp, STL file size, and OpenSCAD parameter values to a \texttt{build\_log.csv} file automatically.
\item
  Write a script that parses your \texttt{src/main.scad} file and extracts all top-level parameter names and values using \texttt{grep} and \texttt{sed}. Output a parameter summary table.
\item
  Build a cross-platform build script (bash + Windows PowerShell) that performs the same validation steps on both platforms. Document the differences.
\item
  Design an automated testing framework for OpenSCAD modules: write a script that builds 10 test cases, each with known expected STL file size ranges, and reports PASS/FAIL for each.
\end{enumerate}

\subsection*{References and Helpful Resources}\label{docs__pandoc__latex__src__3dmake_foundation__lessons_3dmake_9__lessons_3dmake_9.md__references-and-helpful-resources}

\subsubsection*{Supplemental Resources}\label{docs__pandoc__latex__src__3dmake_foundation__lessons_3dmake_9__lessons_3dmake_9.md__supplemental-resources}

\begin{itemize}
\tightlist
\item
  \href{https://www.gnu.org/software/bash/manual/}{Bash Scripting Guide --- GNU Bash Manual}
\item
  \href{https://en.wikibooks.org/wiki/OpenSCAD_User_Manual/Include_Statement}{OpenSCAD User Manual --- Include Statement}
\item
  \href{https://github.com/tdeck/3dmake/blob/main/e2e_test.py}{3DMake End-to-End Test Suite} --- Example automation patterns from the 3dMake project itself
\item
  \href{https://github.com/inotify-tools/inotify-tools}{inotify-tools Linux Documentation}
\end{itemize}

\section{Lesson 10: Hands-On Practice Exercises and Troubleshooting}\label{docs__pandoc__latex__src__3dmake_foundation__lessons_3dmake_10__lessons_3dmake_10.md__lesson-10-hands-on-practice-exercises-and-troubleshooting}

Estimated time: 120--150 minutes

\subsection*{\texorpdfstring{Learning Objectives \footnote{OpenSCAD User Manual --- Hull and Minkowski. \url{https://en.wikibooks.org/wiki/OpenSCAD_User_Manual/Minkowski_and_Hull}}}{Learning Objectives }}\label{docs__pandoc__latex__src__3dmake_foundation__lessons_3dmake_10__lessons_3dmake_10.md__learning-objectives-}

\begin{itemize}
\tightlist
\item
  Apply skills from Lessons 1--9 in three integrated design exercises
\item
  Use calipers to measure and validate printed parts against specifications
\item
  Diagnose and fix non-manifold geometry errors
\item
  Perform tolerance stack-up analysis
\item
  Use \texttt{3dm\ describe} for non-visual validation
\end{itemize}

\subsection*{Materials}\label{docs__pandoc__latex__src__3dmake_foundation__lessons_3dmake_10__lessons_3dmake_10.md__materials}

\begin{itemize}
\tightlist
\item
  3dMake project
\item
  Printer and PLA filament
\item
  Digital calipers
\item
  Printed parts from previous lessons (or new prints from exercises below)
\end{itemize}

\subsection*{Exercise Set A: Phone Stand Refinement}\label{docs__pandoc__latex__src__3dmake_foundation__lessons_3dmake_10__lessons_3dmake_10.md__exercise-set-a-phone-stand-refinement}

\subsubsection*{A1 --- Measure and Iterate}\label{docs__pandoc__latex__src__3dmake_foundation__lessons_3dmake_10__lessons_3dmake_10.md__a1--measure-and-iterate}

Using calipers, measure your printed phone stand against the design specification:

\begin{lstlisting}[style=Alabaster]
Measurement checklist:
[ ] Base width = phone_w + 20 ± 0.3 mm
[ ] Base depth as calculated ± 0.5 mm
[ ] Back support angle (measure with angle gauge or protractor)
[ ] Lip depth = lip_h ± 0.3 mm
[ ] Phone fits and is stable (functional test)

\end{lstlisting}

For each out-of-spec dimension, calculate the correction and update the parameter in \texttt{src/main.scad}. Rebuild and reprint.

\subsubsection*{\texorpdfstring{A2 --- Tolerance Stack-Up Analysis \footnote{Digital Calipers Measurement Technique --- General metrology reference. See also: \href{https://github.com/mrhunsaker/VI_3DMake_OpenSCAD_Curriculum/3dMake_Foundation/Lessons_3dMake_10/../../assets/3dMake_Foundation/measurement_worksheet.md}{Measurement Worksheet Asset}}}{A2 --- Tolerance Stack-Up Analysis }}\label{docs__pandoc__latex__src__3dmake_foundation__lessons_3dmake_10__lessons_3dmake_10.md__a2--tolerance-stack-up-analysis-}

\begin{lstlisting}[style=Alabaster]
Scenario: phone stand cradle with three stacked parts:
- Base plate: designed 5mm, printed 5.12mm (+ 0.12mm)
- Back brace: designed 60mm, printed 59.87mm (- 0.13mm)
- Lip:        designed 15mm, printed 15.09mm (+ 0.09mm)

Total stack height: 5.12 + 59.87 + 15.09 = 80.08mm
Design intent:      5 + 60 + 15            = 80.00mm
Error:              80.08 - 80.00          = +0.08mm  (within 0.5mm spec — PASS)

Worst case (all errors same direction): 0.12 + 0.13 + 0.09 = 0.34mm — still within spec

\end{lstlisting}

Document your own measurements in a similar table.

\subsubsection*{\texorpdfstring{A3 --- Add a Cable Slot \footnote{3DMake GitHub Repository --- Command reference including \texttt{3dm\ describe}. \url{https://github.com/tdeck/3dmake}}}{A3 --- Add a Cable Slot }}\label{docs__pandoc__latex__src__3dmake_foundation__lessons_3dmake_10__lessons_3dmake_10.md__a3--add-a-cable-slot-}

Extend your phone stand design with a cable slot through the base:

\begin{lstlisting}[style=Alabaster, language=openscad]
cable_slot_w  = 12;   // mm
cable_slot_d  = 5;    // mm
cable_slot_z  = -0.001;

// Add to main difference() block:
translate([base_w/2 - cable_slot_w/2, 0, cable_slot_z])
  cube([cable_slot_w, cable_slot_d, base_h + 0.002]);

\end{lstlisting}

\subsection*{Exercise Set B: Keycap with Text}\label{docs__pandoc__latex__src__3dmake_foundation__lessons_3dmake_10__lessons_3dmake_10.md__exercise-set-b-keycap-with-text}

\subsubsection*{B1 --- Build a Mechanical Keyboard Keycap}\label{docs__pandoc__latex__src__3dmake_foundation__lessons_3dmake_10__lessons_3dmake_10.md__b1--build-a-mechanical-keyboard-keycap}

\begin{lstlisting}[style=Alabaster, language=openscad]
// Parametric keycap
key_w      = 18;
key_d      = 18;
key_h      = 7;
stem_r     = 2.75;  // MX stem: 5.5mm diameter
stem_h     = 3.8;
wall       = 1.5;
label_text = "A";

module keycap() {
  difference() {
    // Keycap body with slight top curve
    hull() {
      cube([key_w, key_d, key_h - 2], center=true);
      translate([0, 0, 1]) cube([key_w - 2, key_d - 2, key_h], center=true);
    }
    // Hollow inside
    translate([0, 0, -wall])
      cube([key_w - 2*wall, key_d - 2*wall, key_h], center=true);
    // MX stem hole
    translate([0, 0, -(key_h/2 + 0.001)])
      cylinder(r=stem_r + 0.1, h=stem_h + 0.001, $fn=16);
  }
}

module stem_mount() {
  translate([0, 0, -(key_h/2 + stem_h)])
    difference() {
      cylinder(r=stem_r + wall, h=stem_h, $fn=16);
      cylinder(r=stem_r, h=stem_h + 0.001, $fn=16);
    }
}

keycap();
stem_mount();

// Engrave label
translate([0, 0, key_h/2 - 0.8])
  linear_extrude(1.2)
    text(label_text, size=8, font="Liberation Sans:style=Bold",
         halign="center", valign="center", $fn=4);

\end{lstlisting}

\subsubsection*{B2 --- Validate with 3dm describe}\label{docs__pandoc__latex__src__3dmake_foundation__lessons_3dmake_10__lessons_3dmake_10.md__b2--validate-with-3dm-describe}

\begin{lstlisting}[style=Alabaster, language=bash]
3dm describe

\end{lstlisting}

Expected output should confirm the keycap geometry. Document what the AI description says and compare it to your design intent.

\subsubsection*{B3 --- Print and Test}\label{docs__pandoc__latex__src__3dmake_foundation__lessons_3dmake_10__lessons_3dmake_10.md__b3--print-and-test}

Print the keycap and test it on a Cherry MX switch (or compatible). If the stem is too tight, increase \texttt{stem\_r\ +\ 0.1} to \texttt{stem\_r\ +\ 0.15}. If too loose, decrease to \texttt{stem\_r\ +\ 0.05}.

\subsection*{Exercise Set C: Stackable Bins}\label{docs__pandoc__latex__src__3dmake_foundation__lessons_3dmake_10__lessons_3dmake_10.md__exercise-set-c-stackable-bins}

\subsubsection*{C1 --- Build a Three-Size Bin Set}\label{docs__pandoc__latex__src__3dmake_foundation__lessons_3dmake_10__lessons_3dmake_10.md__c1--build-a-three-size-bin-set}

Using the stackable bin module from Lesson 8, generate three sizes:

\begin{lstlisting}[style=Alabaster, language=openscad]
// Small bin
translate([0, 0, 0])
  bin_assembly(bin_w=60, bin_d=45, bin_h=30);

// Medium bin
translate([80, 0, 0])
  bin_assembly(bin_w=80, bin_d=60, bin_h=40);

// Large bin
translate([180, 0, 0])
  bin_assembly(bin_w=100, bin_d=80, bin_h=50);

\end{lstlisting}

\subsubsection*{C2 --- Diagnose and Fix Non-Manifold Geometry}\label{docs__pandoc__latex__src__3dmake_foundation__lessons_3dmake_10__lessons_3dmake_10.md__c2--diagnose-and-fix-non-manifold-geometry}

Non-manifold geometry occurs when faces share edges inconsistently (T-junctions, missing faces, zero-thickness walls). Common causes:

\begin{lstlisting}[style=Alabaster, language=openscad]
// PROBLEM: two cubes share a face exactly — may produce non-manifold edge
cube([20, 20, 10]);
translate([20, 0, 0]) cube([20, 20, 10]);  // touching at x=20 — ambiguous edge

// FIX 1: use union()
union() {
  cube([20, 20, 10]);
  translate([20, 0, 0]) cube([20, 20, 10]);
}

// FIX 2: overlap slightly
cube([20.001, 20, 10]);
translate([20, 0, 0]) cube([20, 20, 10]);

\end{lstlisting}

Diagnosis tool:

\begin{lstlisting}[style=Alabaster, language=bash]
3dm describe  # AI will often flag non-manifold geometry
# Also open STL in slicer and enable "Check for geometry errors"

\end{lstlisting}

\subsubsection*{C3 --- Advanced Geometry: hull() and minkowski()}\label{docs__pandoc__latex__src__3dmake_foundation__lessons_3dmake_10__lessons_3dmake_10.md__c3--advanced-geometry-hull-and-minkowski}

\begin{lstlisting}[style=Alabaster, language=openscad]
// hull() creates a convex envelope — useful for organic shapes
module smooth_transition() {
  hull() {
    translate([0, 0, 0]) cylinder(r=15, h=5, $fn=64);
    translate([0, 0, 30]) cylinder(r=5, h=2, $fn=64);
  }
}

smooth_transition();

// minkowski() adds the shape of a small object to every surface point
module rounded_hull() {
  minkowski() {
    hull() {
      cylinder(r=10, h=3, $fn=8);      // octagonal prism
      translate([30, 0, 0]) sphere(r=8, $fn=32);
    }
    sphere(r=2, $fn=16);  // rounds all edges by 2mm
  }
}

rounded_hull();

\end{lstlisting}

\subsection*{Quiz --- Lesson 3dMake.10 (15 questions)}\label{docs__pandoc__latex__src__3dmake_foundation__lessons_3dmake_10__lessons_3dmake_10.md__quiz--lesson-3dmake10-15-questions}

\begin{enumerate}
\tightlist
\item
  What tool do you use to measure printed part dimensions against the design specification?
\item
  What is tolerance stack-up, and why does it matter for multi-part assemblies?
\item
  What causes non-manifold geometry in OpenSCAD, and how do you detect it?
\item
  How does \texttt{hull()} differ from \texttt{union()}?
\item
  What does \texttt{3dm\ describe} help you verify about your model?
\item
  What does a Cherry MX stem measure in diameter, and what clearance would you add for a slip-fit keycap?
\item
  True or False: \texttt{find\ -newer} is an event-driven file change detection method.
\item
  If three parts each have ±0.15 mm tolerance, what is the worst-case total error for a three-part stack?
\item
  What does the \texttt{\$fn} parameter control in OpenSCAD?
\item
  Describe two methods for fixing non-manifold geometry caused by two touching (but not overlapping) shapes.
\item
  What is the difference between \texttt{hull()} and \texttt{minkowski()} for creating organic shapes? Give one use case for each.
\item
  What does \texttt{resize({[}50,\ 0,\ 0{]})} do, and why might \texttt{resize()} behave unexpectedly for non-uniform scaling?
\item
  When measuring a printed part with calipers, what is the difference between an inside measurement and an outside measurement, and when does that distinction matter for tolerance analysis?
\item
  Describe the iterative design workflow for dialing in press-fit tolerances: what do you print, what do you measure, and how do you adjust?
\item
  If \texttt{3dm\ describe} reports "the model appears non-manifold," what are three possible causes you would investigate in your OpenSCAD code?
\end{enumerate}

\subsection*{Extension Problems (15)}\label{docs__pandoc__latex__src__3dmake_foundation__lessons_3dmake_10__lessons_3dmake_10.md__extension-problems-15}

\begin{enumerate}
\tightlist
\item
  Create a tolerance sensitivity study: build 5 keycaps with stem clearance from 0.05--0.25 mm in 0.05 mm increments, print them, and record which values fit your switches.
\item
  Design a go/no-go gauge for a 10 mm nominal hole: a part with a "go" pin sized for slip-fit and a "no-go" pin sized for interference fit.
\item
  Write a printer calibration SOP (standard operating procedure): bed leveling, first-layer calibration, and dimension verification. Include a measurement checklist.
\item
  Build a three-tier stackable storage system for art supplies. Each tier has a different inner grid.
\item
  Conduct a tolerance stack-up analysis for your stackable bin system. Calculate worst-case misalignment.
\item
  Build a parametric test coupon that tests four different wall thicknesses (0.8, 1.2, 1.6, 2.0 mm) in a single print.
\item
  Design a caliper stand: a holder that holds your digital calipers at a comfortable angle for one-handed operation.
\item
  Build a non-manifold error catalog: intentionally create 5 different types of non-manifold geometry, document how each was created and how to fix it.
\item
  Use \texttt{hull()} to design a smooth ergonomic tool handle and compare it to a simple cylinder handle.
\item
  Create a printability checklist for new designs: overhangs, wall thickness, minimum feature size, support requirements. Apply it to your keycap and bin designs.
\item
  Research the \texttt{resize()} function in OpenSCAD. Build an example showing how it behaves differently from \texttt{scale()} for non-uniform resizing.
\item
  Design a multi-part assembly tutorial: a three-piece interlocking puzzle that teaches the concepts of tolerance, alignment, and slip-fit.
\item
  Build a "measurement worksheet" template in OpenSCAD: render a flat sheet that lists all key dimensions of a part as text, for printing alongside the part.
\item
  Create a chi-squared goodness-of-fit test for your printer\textquotesingle s dimensional accuracy: measure 20 prints of the same part and determine if the errors are normally distributed.
\item
  Write a comprehensive troubleshooting guide covering the 10 most common 3D printing failures you have encountered (or researched), with causes, prevention, and fixes.
\end{enumerate}

\subsection*{References and Helpful Resources}\label{docs__pandoc__latex__src__3dmake_foundation__lessons_3dmake_10__lessons_3dmake_10.md__references-and-helpful-resources}

\subsubsection*{Supplemental Resources}\label{docs__pandoc__latex__src__3dmake_foundation__lessons_3dmake_10__lessons_3dmake_10.md__supplemental-resources}

\begin{itemize}
\tightlist
\item
  \href{docs/pandoc/latex/src/assets/Programming_with_OpenSCAD.epub}{Programming with OpenSCAD EPUB Textbook} --- Troubleshooting and advanced geometry chapters
\item
  \href{https://github.com/ProgrammingWithOpenSCAD/CodeSolutions}{CodeSolutions Repository} --- Worked practice exercises
\item
  \href{https://programmingwithopenscad.github.io/quick-reference.html}{OpenSCAD Quick Reference} --- Function reference
\item
  \href{https://github.com/mrhunsaker/VI_3DMake_OpenSCAD_Curriculum/3dMake_Foundation/Lessons_3dMake_10/../../assets/3dMake_Foundation/master-rubric.md}{Master Rubric} --- Assessment criteria for practice exercises
\end{itemize}

\subsection{Diagnostic Checklist for 3D Printing}\label{docs__pandoc__latex__src__3dmake_foundation__lessons_3dmake_10__diagnostic_checklist.md__3dmake_foundation_lessons_3dmake_10-diagnostic_checklist}

Use this comprehensive checklist to systematically diagnose and troubleshoot printing issues.

\subsubsection*{Quick Diagnosis Flowchart}\label{docs__pandoc__latex__src__3dmake_foundation__lessons_3dmake_10__diagnostic_checklist.md__quick-diagnosis-flowchart}

\begin{lstlisting}[style=Alabaster]
Print Problem?
|
+---- [Before Print] Issues?
|  +---- Filament won't load -> CHECK: Temperature, drive gear, nozzle
|  +---- Printer won't heat -> CHECK: Power, temperature sensor, firmware
|  +---- Bed not level -> CHECK: Leveling routine, bed surface
|
+---- [First Layer] Issues?
|  +---- Won't stick -> CHECK: Bed temperature, cleanliness, nozzle height
|  +---- Too squished -> CHECK: Nozzle height, bed temperature
|  +---- Gaps/uneven -> CHECK: Bed levelness, hot end alignment
|
+---- [Mid-Print] Issues?
|  +---- Stops extruding -> CHECK: Clog, temperature drop, jam
|  +---- Layers shift -> CHECK: Loose belts, mechanical bind
|  +---- Print wobbles -> CHECK: Build plate, print stability
|
+---- [Print Quality] Issues?
   +---- Weak/brittle -> CHECK: Temperature, flow rate, layer adhesion
   +---- Rough surface -> CHECK: Flow rate, speed, temperature
   +---- Warped -> CHECK: Bed temperature, cooling rate, material

\end{lstlisting}

\subsubsection*{Pre-Print Diagnostics}\label{docs__pandoc__latex__src__3dmake_foundation__lessons_3dmake_10__diagnostic_checklist.md__pre-print-diagnostics}

\paragraph*{Category A: Power \& Connectivity}\label{docs__pandoc__latex__src__3dmake_foundation__lessons_3dmake_10__diagnostic_checklist.md__category-a-power--connectivity}

Checklist:

\begin{itemize}
\tightlist
\item[$\square$]
  Printer powered on
\item[$\square$]
  LED indicators showing normal status
\item[$\square$]
  USB cable connected (if applicable)
\item[$\square$]
  No error codes displaying
\item[$\square$]
  Display/interface responding to input
\end{itemize}

If failed:

\begin{enumerate}
\tightlist
\item
  Check power outlet and cable
\item
  Verify power supply specifications (voltage, current)
\item
  Test with different power outlet
\item
  Try power-cycling (off 30 sec, on)
\item
  Check for blown fuses inside printer
\end{enumerate}

\paragraph*{Category B: Temperature System}\label{docs__pandoc__latex__src__3dmake_foundation__lessons_3dmake_10__diagnostic_checklist.md__category-b-temperature-system}

Heating Element Status:

\begin{itemize}
\tightlist
\item[$\square$]
  Hot end temperature rises when heating commanded
\item[$\square$]
  Bed temperature rises when heating commanded
\item[$\square$]
  Temperature readings stable (not fluctuating 5C+)
\item[$\square$]
  No error messages during heating
\end{itemize}

Measurement Method:

\begin{lstlisting}[style=Alabaster]
1. Set hot end to 200C, observe rise
   - Expected time to reach: 2-4 minutes
   - Steady rise without plateau:  Good
   - Plateau before reaching:  Problem (see below)
2. Set bed to 60C, observe rise
   - Expected time to reach: 5-10 minutes
   - Stable at target:  Good

\end{lstlisting}

If heating slow/incomplete:

\begin{itemize}
\tightlist
\item[$\square$]
  Verify target temperature was set
\item[$\square$]
  Check heating element firmware settings
\item[$\square$]
  Test thermal sensor connectivity
\item[$\square$]
  Inspect heating element for damage
\item[$\square$]
  Measure electrical resistance of heaters
\end{itemize}

Temperature Stability Test:

\begin{lstlisting}[style=Alabaster]
1. Heat to target temperature
2. Wait 10 minutes for stabilization
3. Record temperature every minute
4. Calculate range (max - min)
Results:
- +/-2C range:  Excellent
- +/-5C range:  Acceptable  
- +/-10C range:  Marginal
- >+/-10C range:  Problem

\end{lstlisting}

\paragraph*{Category C: Mechanical Systems}\label{docs__pandoc__latex__src__3dmake_foundation__lessons_3dmake_10__diagnostic_checklist.md__category-c-mechanical-systems}

Movement Diagnostics:

Manual Axis Movement:

\begin{lstlisting}[style=Alabaster]
1. Disable motors (if possible)
2. Manually move each axis
3. Record observations:
   - X-axis:  (smooth/rough/stuck)
   - Y-axis:  (smooth/rough/stuck)
   - Z-axis:  (smooth/rough/stuck)

\end{lstlisting}

Expected results: Smooth, no grinding sounds

If rough/stuck:

\begin{itemize}
\tightlist
\item[$\square$]
  Check for visible obstructions
\item[$\square$]
  Inspect rails for debris
\item[$\square$]
  Verify pulleys turn freely
\item[$\square$]
  Check belt tension (if accessible)
\item[$\square$]
  Lubricate dry joints
\end{itemize}

Powered Movement Test:

\begin{lstlisting}[style=Alabaster]
1. Position nozzle at center
2. Command X+10mm movement
3. Verify nozzle moved ~10mm
4. Repeat for Y+10mm and Z+5mm
5. Check for skipping or missed steps

\end{lstlisting}

\paragraph*{Category D: Leveling \& Alignment}\label{docs__pandoc__latex__src__3dmake_foundation__lessons_3dmake_10__diagnostic_checklist.md__category-d-leveling--alignment}

Build Plate Leveling Test:

Paper Method (Most Common):

\begin{lstlisting}[style=Alabaster]
1. Heat bed to printing temperature
2. Heat nozzle to printing temperature  
3. Position nozzle at first corner
4. Adjust leveling screw until paper drags slightly
5. Move to next corner and repeat
6. Repeat for all 4-9 corners
7. Do center point check last
Target: Consistent slight paper drag all points

\end{lstlisting}

Leveling Validation:

\begin{itemize}
\tightlist
\item[$\square$]
  Level at 4 corners
\item[$\square$]
  Level at bed center
\item[$\square$]
  No high/low points
\item[$\square$]
  Nozzle doesn\textquotesingle t hit bed at any point
\item[$\square$]
  Consistent first-layer appearance across bed
\end{itemize}

\paragraph*{Category E: Filament \& Extruder}\label{docs__pandoc__latex__src__3dmake_foundation__lessons_3dmake_10__diagnostic_checklist.md__category-e-filament--extruder}

Filament Quality Check:

\begin{itemize}
\tightlist
\item[$\square$]
  Filament diameter consistent (visually inspect \textasciitilde{}50cm)
\item[$\square$]
  No visible cracks or damage
\item[$\square$]
  Spool rotates freely without binding
\item[$\square$]
  Filament path clear to extruder
\item[$\square$]
  No tangles in filament path
\end{itemize}

Extruder Test:

\begin{lstlisting}[style=Alabaster]
1. Heat to printing temperature
2. Remove print head (if removable)
3. Push 10-20mm of filament through manually
4. Feel resistance during push
5. Observe material exits cleanly
Expected: Smooth push, consistent extrusion

\end{lstlisting}

\subsubsection*{Filament Loading Test}\label{docs__pandoc__latex__src__3dmake_foundation__lessons_3dmake_10__diagnostic_checklist.md__filament-loading-test}

Test Sequence:

\begin{lstlisting}[style=Alabaster]
1. Heat extruder to material temp
2. Load filament into extruder
3. Watch for material at nozzle tip
Timeline:
- 0-10 sec: Filament engaging
- 10-30 sec: Moving through hot end
- 30-60 sec: Should appear at nozzle tip
- >60 sec: Possible partial clog
If fails:
-> See "Filament Won't Load" troubleshooting

\end{lstlisting}

\subsubsection*{First Layer Diagnostics}\label{docs__pandoc__latex__src__3dmake_foundation__lessons_3dmake_10__diagnostic_checklist.md__first-layer-diagnostics}

\paragraph*{Layer Appearance Test}\label{docs__pandoc__latex__src__3dmake_foundation__lessons_3dmake_10__diagnostic_checklist.md__layer-appearance-test}

After printing first 5-10 layer heights, evaluate:

{\def\LTcaptype{none} % do not increment counter
\begin{longtable}[]{@{}
  >{\raggedright\arraybackslash}p{(\linewidth - 4\tabcolsep) * \real{0.3165}}
  >{\raggedright\arraybackslash}p{(\linewidth - 4\tabcolsep) * \real{0.3544}}
  >{\raggedright\arraybackslash}p{(\linewidth - 4\tabcolsep) * \real{0.3291}}@{}}
\toprule\noalign{}
\begin{minipage}[b]{\linewidth}\raggedright
Appearance
\end{minipage} & \begin{minipage}[b]{\linewidth}\raggedright
Issue
\end{minipage} & \begin{minipage}[b]{\linewidth}\raggedright
Action
\end{minipage} \\
\midrule\noalign{}
\endhead
\bottomrule\noalign{}
\endlastfoot
Wavy/embossed & Bed not level or too close & Relevel bed \\
Gaps between lines & Nozzle too high & Lower Z-offset \\
Completely squished & Nozzle too low & Raise Z-offset \\
Partial adhesion & Bed too cool or dirty & Clean bed, increase temp \\
Consistent squish/lines & Correct & Continue print \\
\end{longtable}
}

\subsubsection*{Mid-Print Issue Diagnostics}\label{docs__pandoc__latex__src__3dmake_foundation__lessons_3dmake_10__diagnostic_checklist.md__mid-print-issue-diagnostics}

\paragraph*{Extrusion Failure Checklist}\label{docs__pandoc__latex__src__3dmake_foundation__lessons_3dmake_10__diagnostic_checklist.md__extrusion-failure-checklist}

When extrusion stops during print:

Immediate Actions:

\begin{itemize}
\tightlist
\item[$\square$]
  Pause print (don\textquotesingle t stop)
\item[$\square$]
  Listen for extruder sounds (grinding = jam)
\item[$\square$]
  Feel nozzle carefully (if cooled slightly)
\item[$\square$]
  Observe filament in extruder (is it feeding?)
\end{itemize}

Diagnostic Decision:

\begin{lstlisting}[style=Alabaster]
Is filament stuck in extruder?
+---- YES -> Nozzle clog likely
|        -> See Nozzle Clog section (common_issues_and_solutions.md)
|        -> Try: Cold pull, retract, clean
|
+---- NO -> Filament loading issue
         -> Is spool binding? -> Fix spool rotation
         -> Is path blocked? -> Clear obstruction
         -> Is drive gear slipping? -> Clean/tension drive gear

\end{lstlisting}

\paragraph*{Mechanical Issue Diagnostics}\label{docs__pandoc__latex__src__3dmake_foundation__lessons_3dmake_10__diagnostic_checklist.md__mechanical-issue-diagnostics}

When movement sounds wrong:

Listen for:

\begin{itemize}
\tightlist
\item
  Grinding/grating: Bearing issue or obstruction
\item
  Clicking/skipping: Lost steps or over-torque
\item
  Squealing: Lubrication needed
\item
  Silence (but no movement): Stalled motor
\end{itemize}

Diagnosis Method:

\begin{lstlisting}[style=Alabaster]
1. Pause print
2. Manually move suspected axis
3. Record resistance type:
   - Smooth:  Normal
   - Rough: Bearing/alignment problem
   - Stuck: Mechanical bind
4. Visually inspect that axis

\end{lstlisting}

\subsubsection*{Layer Quality Diagnostics}\label{docs__pandoc__latex__src__3dmake_foundation__lessons_3dmake_10__diagnostic_checklist.md__layer-quality-diagnostics}

\paragraph*{Visual Inspection During Print}\label{docs__pandoc__latex__src__3dmake_foundation__lessons_3dmake_10__diagnostic_checklist.md__visual-inspection-during-print}

Every 30 minutes of printing, check:

\begin{lstlisting}[style=Alabaster]
[ ] Layer alignment (no X/Y shifting)
[ ] Material flow (consistent lines, not thin or thick)
[ ] Surface appearance (smooth, not rough)
[ ] Support structure (if used, printing properly)
[ ] No material strings between features
[ ] Consistent layer heights visible

\end{lstlisting}

\subsubsection*{Dimensional Accuracy Diagnostics}\label{docs__pandoc__latex__src__3dmake_foundation__lessons_3dmake_10__diagnostic_checklist.md__dimensional-accuracy-diagnostics}

After print completes and cools (24 hours):

\paragraph*{Precision Measurement}\label{docs__pandoc__latex__src__3dmake_foundation__lessons_3dmake_10__diagnostic_checklist.md__precision-measurement}

Materials Needed:

\begin{itemize}
\tightlist
\item
  Digital calipers (+/-0.05mm accuracy)
\item
  Ruler (for larger dimensions)
\item
  Notepad for recording
\end{itemize}

Measurement Protocol:

\begin{lstlisting}[style=Alabaster]
1. Measure each dimension 3 times at different locations
2. Calculate average
3. Compare to design dimension
4. Calculate deviation percentage
Formula:
Deviation % = ((Measured - Design) / Design) x 100%
Example:
- Design: 20.0mm
- Measured: 19.8mm
- Deviation: ((19.8 - 20.0) / 20.0) x 100% = -1%

\end{lstlisting}

\paragraph*{Tolerance Evaluation}\label{docs__pandoc__latex__src__3dmake_foundation__lessons_3dmake_10__diagnostic_checklist.md__tolerance-evaluation}

{\def\LTcaptype{none} % do not increment counter
\begin{longtable}[]{@{}lll@{}}
\toprule\noalign{}
Tolerance & Pass/Fail & Action \\
\midrule\noalign{}
\endhead
\bottomrule\noalign{}
\endlastfoot
+/-0.5mm or better & PASS & No adjustment needed \\
+/-0.5-1mm & MARGINAL & Document and monitor \\
\textgreater+/-1mm & FAIL & Adjust flow/calibration \\
\end{longtable}
}

\subsubsection*{Environmental Diagnostics}\label{docs__pandoc__latex__src__3dmake_foundation__lessons_3dmake_10__diagnostic_checklist.md__environmental-diagnostics}

When quality varies between prints:

Check Conditions:

\begin{itemize}
\tightlist
\item[$\square$]
  Room temperature stable (+/-5C?)
\item[$\square$]
  Humidity reasonable (30-60\%?)
\item[$\square$]
  No drafts from windows/AC near printer
\item[$\square$]
  Consistent vibration level (no external impact)
\item[$\square$]
  Same filament spool/batch used
\item[$\square$]
  Same slicer settings applied
\end{itemize}

Environmental Log:

\begin{lstlisting}[style=Alabaster]
Date:     Time:     Temp: C    Humidity: %
Print Duration:     Result Quality: Poor/Fair/Good/Excellent
Notes: 

\end{lstlisting}

\subsubsection*{Troubleshooting Decision Tree}\label{docs__pandoc__latex__src__3dmake_foundation__lessons_3dmake_10__diagnostic_checklist.md__troubleshooting-decision-tree}

Start here for systematic diagnosis:

\begin{lstlisting}[style=Alabaster]
+---- Printer won't start?
|  +---- Check: Power, connections, firmware
|
+---- Heating won't work?
|  +---- Check: Temperature sensor, firmware, heater element
|
+---- Won't home/move?
|  +---- Check: Endstops, mechanical bind, motors, belts
|
+---- First layer fails?
|  +---- Check: Bed level, nozzle height, cleanliness, temperature
|
+---- Filament won't feed?
|  +---- Check: Temperature, nozzle, drive gear, clog
|
+---- Extrusion stops mid-print?
|  +---- Extruder grinding? -> Clog (cold pull or replace nozzle)
|  +---- Filament slack? -> Drive gear or load issue
|  +---- No sounds? -> Temperature drop or firmware issue
|
+---- Print quality poor?
|  +---- Weak/thin? -> Increase flow/temp/slow down
|  +---- Rough/bloated? -> Decrease flow/temp/speed up
|  +---- Warped? -> Lower bed temp, faster cooling
|  +---- Strings? -> Increase retraction, lower temp
|
+---- Dimensions wrong?
   +---- Check: Flow rate calibration, printer accuracy limits

\end{lstlisting}

\subsubsection*{Diagnostic Report Template}\label{docs__pandoc__latex__src__3dmake_foundation__lessons_3dmake_10__diagnostic_checklist.md__diagnostic-report-template}

Use when seeking help:

\begin{lstlisting}[style=Alabaster]
DIAGNOSTIC REPORT
================
Printer Model: 
Problem Description: 
When it occurs: (always/sometimes/first 5 layers, etc) 
Recent changes: 
DIAGNOSTICS PERFORMED:
[ ] Power/connectivity verified
[ ] Temperatures verified  
[ ] Mechanical movement tested
[ ] Bed leveling checked
[ ] Filament loading tested
[ ] First layer inspected
[ ] Print quality evaluated
Key Findings:
1. 
2. 
3. 
Attempted Solutions:
1. 
2. 
Result: (Solved/Partial/Ongoing) 

\end{lstlisting}

Last Diagnostic Date: Issue Resolved:\\
Diagnostic Performed By:

\subsection{Measurement Calibration Guide}\label{docs__pandoc__latex__src__3dmake_foundation__lessons_3dmake_10__measurement_calibration_guide.md__3dmake_foundation_lessons_3dmake_10-measurement_calibration_guide}

Ensure accurate dimensions in your 3D printed parts through proper calibration and verification procedures.

\subsubsection*{Why Calibration Matters}\label{docs__pandoc__latex__src__3dmake_foundation__lessons_3dmake_10__measurement_calibration_guide.md__why-calibration-matters}

Printed dimensions often deviate from designed dimensions due to:

\begin{itemize}
\tightlist
\item
  Nozzle width variations
\item
  Extrusion flow rate inconsistencies
\item
  Plastic shrinkage during cooling
\item
  Slicer interpretation differences
\item
  Material-specific shrinkage (ABS can shrink 0.3-1\%)
\end{itemize}

Typical tolerances without calibration: +/-0.3-0.5mm
Typical tolerances with calibration: +/-0.1-0.2mm

\subsubsection*{Pre-Print Calibration}\label{docs__pandoc__latex__src__3dmake_foundation__lessons_3dmake_10__measurement_calibration_guide.md__pre-print-calibration}

\paragraph*{1. Nozzle Diameter Verification}\label{docs__pandoc__latex__src__3dmake_foundation__lessons_3dmake_10__measurement_calibration_guide.md__1-nozzle-diameter-verification}

What to verify: Is your nozzle actually 0.4mm?

Test:

\begin{enumerate}
\tightlist
\item
  Create a simple box in OpenSCAD: 10mm x 10mm x 5mm, solid
\item
  Slice with default 0.4mm line width
\item
  Print using 100\% flow rate
\item
  Measure printed box
\end{enumerate}

Adjustment:

\begin{itemize}
\tightlist
\item
  If too small: Nozzle might be partially clogged (clean)
\item
  If significantly different from 0.4mm: Replace nozzle
\end{itemize}

\paragraph*{2. E-Steps Calibration (Extrusion Rate)}\label{docs__pandoc__latex__src__3dmake_foundation__lessons_3dmake_10__measurement_calibration_guide.md__2-e-steps-calibration-extrusion-rate}

What to verify: Does the extruder push the correct amount of filament?

Method:

\begin{enumerate}
\tightlist
\item
  Heat extruder to printing temperature
\item
  Mark filament 100mm from extruder entrance with marker
\item
  Command extrusion of 100mm in firmware (G-code: \texttt{G1\ E100\ F100})
\item
  Measure actual distance filament moved
\end{enumerate}

Formula:

\begin{lstlisting}[style=Alabaster]
New E-steps = Current E-steps x (100mm / Actual distance moved)

\end{lstlisting}

Example:

\begin{itemize}
\tightlist
\item
  Current setting: 93 steps/mm
\item
  Commanded: 100mm
\item
  Actual: 92mm moved
\item
  New: 93 x (100/92) = 101 steps/mm
\end{itemize}

How to apply (varies by printer):

\begin{itemize}
\tightlist
\item
  Marlin: \texttt{M92\ E101} then \texttt{M500} to save
\item
  Klipper: Update configuration and restart
\end{itemize}

\paragraph*{3. First Layer Height Verification}\label{docs__pandoc__latex__src__3dmake_foundation__lessons_3dmake_10__measurement_calibration_guide.md__3-first-layer-height-verification}

What to verify: Is first layer height optimal?

Test: Print a simple single-layer square (just base layer)

Measurements:

\begin{itemize}
\tightlist
\item
  Too high (\textgreater0.3mm): Poor layer adhesion
\item
  Too low (\textless{}0.1mm): Nozzle scratches bed, plastic squeezed
\item
  Optimal: 0.2-0.25mm (roughly paper thickness)
\end{itemize}

Adjustment: Use bed leveling or Z-offset:

\begin{itemize}
\tightlist
\item
  If too high: Reduce Z-offset by 0.05mm
\item
  If too low: Increase Z-offset by 0.05mm
\item
  Test between adjustments
\end{itemize}

\subsubsection*{Dimension Calibration Process}\label{docs__pandoc__latex__src__3dmake_foundation__lessons_3dmake_10__measurement_calibration_guide.md__dimension-calibration-process}

\paragraph*{Standard XY Calibration Test}\label{docs__pandoc__latex__src__3dmake_foundation__lessons_3dmake_10__measurement_calibration_guide.md__standard-xy-calibration-test}

Goal: Create parts with precisely measured dimensions

Test File (OpenSCAD):

\begin{lstlisting}[style=Alabaster]
// Create calibration cube
cube_size = 20;  // 20mm cube
wall_thickness = 2;

difference() {
    cube([cube_size, cube_size, cube_size], center=true);
    cube([cube_size - 2*wall_thickness, 
          cube_size - 2*wall_thickness, 
          cube_size + 1], center=true);  // Top open
}

\end{lstlisting}

Print Instructions:

\begin{itemize}
\tightlist
\item
  Use standard settings (your normal layer height, speed, temp)
\item
  Print with 100\% flow rate
\item
  Allow complete cooling (2+ hours)
\end{itemize}

Measurement Procedure:

\begin{enumerate}
\tightlist
\item
  Measure internal dimensions (hollow part) in 3 locations each axis
\item
  Calculate average internal width: \texttt{avg\_internal}
\item
  Expected internal: \texttt{20\ -\ 2xwall\_thickness\ =\ 16mm}
\end{enumerate}

Calibration Formula:

\begin{lstlisting}[style=Alabaster]
Flow rate adjustment = Expected internal / Actual internal x 100%

Example:
- Expected: 16.00mm
- Actual: 15.75mm
- Adjustment: (16.00 / 15.75) x 100% = 101.6%
- Set flow to: 101.6% in slicer

\end{lstlisting}

\paragraph*{Z-Height Calibration}\label{docs__pandoc__latex__src__3dmake_foundation__lessons_3dmake_10__measurement_calibration_guide.md__z-height-calibration}

Goal: Verify layer heights are accurate

Test: Print calibration tower with varying layer heights

Tower Specifications:

\begin{itemize}
\tightlist
\item
  20mm x 20mm square base
\item
  Height: 40mm
\item
  Layers: Print at 0.2mm nominal
\end{itemize}

Measurements:

\begin{itemize}
\tightlist
\item
  Stack digital calipers on layers and measure height
\item
  Calculate average layer thickness
\item
  Compare with intended 0.2mm
\end{itemize}

Adjustment: If actual layer height differs:

\begin{lstlisting}[style=Alabaster]
New Z-scale = Intended height / Actual height

Example:
- Intended: 0.2mm per layer
- Actual: 0.195mm per layer  
- Adjustment: 0.2 / 0.195 = 1.026 (increase Z by 2.6%)

\end{lstlisting}

\subsubsection*{Tolerance Measurement Matrix}\label{docs__pandoc__latex__src__3dmake_foundation__lessons_3dmake_10__measurement_calibration_guide.md__tolerance-measurement-matrix}

\paragraph*{Critical Measurements to Track}\label{docs__pandoc__latex__src__3dmake_foundation__lessons_3dmake_10__measurement_calibration_guide.md__critical-measurements-to-track}

{\def\LTcaptype{none} % do not increment counter
\begin{longtable}[]{@{}
  >{\raggedright\arraybackslash}p{(\linewidth - 6\tabcolsep) * \real{0.2800}}
  >{\raggedright\arraybackslash}p{(\linewidth - 6\tabcolsep) * \real{0.3600}}
  >{\raggedright\arraybackslash}p{(\linewidth - 6\tabcolsep) * \real{0.1867}}
  >{\raggedright\arraybackslash}p{(\linewidth - 6\tabcolsep) * \real{0.1733}}@{}}
\toprule\noalign{}
\begin{minipage}[b]{\linewidth}\raggedright
Measurement
\end{minipage} & \begin{minipage}[b]{\linewidth}\raggedright
Method
\end{minipage} & \begin{minipage}[b]{\linewidth}\raggedright
Tolerance
\end{minipage} & \begin{minipage}[b]{\linewidth}\raggedright
Frequency
\end{minipage} \\
\midrule\noalign{}
\endhead
\bottomrule\noalign{}
\endlastfoot
Wall thickness & Calipers (multiple spots) & +/-0.1mm & Every print \\
Hole diameter & Calipers or gauge & +/-0.1-0.2mm & Every print \\
Overall dimensions & Ruler/calipers & +/-0.2mm & Monthly \\
Layer height & Stack on calipers & +/-0.02mm & Monthly \\
Vertical dimensions & Measure sides & +/-0.1mm & Every print \\
\end{longtable}
}

\subsubsection*{Advanced Calibration}\label{docs__pandoc__latex__src__3dmake_foundation__lessons_3dmake_10__measurement_calibration_guide.md__advanced-calibration}

\paragraph*{Shrinkage Compensation}\label{docs__pandoc__latex__src__3dmake_foundation__lessons_3dmake_10__measurement_calibration_guide.md__shrinkage-compensation}

Different materials shrink differently after cooling:

{\def\LTcaptype{none} % do not increment counter
\begin{longtable}[]{@{}
  >{\raggedright\arraybackslash}p{(\linewidth - 4\tabcolsep) * \real{0.1220}}
  >{\raggedright\arraybackslash}p{(\linewidth - 4\tabcolsep) * \real{0.2317}}
  >{\raggedright\arraybackslash}p{(\linewidth - 4\tabcolsep) * \real{0.6463}}@{}}
\toprule\noalign{}
\begin{minipage}[b]{\linewidth}\raggedright
Material
\end{minipage} & \begin{minipage}[b]{\linewidth}\raggedright
Typical Shrinkage
\end{minipage} & \begin{minipage}[b]{\linewidth}\raggedright
Compensation
\end{minipage} \\
\midrule\noalign{}
\endhead
\bottomrule\noalign{}
\endlastfoot
PLA & 0.3-0.5\% & Usually acceptable, no action \\
PETG & 0.5-1\% & Scale design up by 0.5-1\% if critical \\
ABS & 0.8-1.5\% & Scale design up by 1\% minimum \\
TPU & 1-2\% & Significant - scale up 1-2\% for critical dimensions \\
\end{longtable}
}

How to apply in design:

\begin{lstlisting}[style=Alabaster]
// In OpenSCAD, scale critical dimensions
final_size = 20;
material_shrinkage = 1.01;  // 1% shrinkage
designed_size = final_size * material_shrinkage;

\end{lstlisting}

\paragraph*{Bed Temperature Compensation}\label{docs__pandoc__latex__src__3dmake_foundation__lessons_3dmake_10__measurement_calibration_guide.md__bed-temperature-compensation}

Different bed temperatures affect final dimensions:

ABS on cold bed (50C) vs warm bed (100C):

\begin{itemize}
\tightlist
\item
  Cold bed: Faster cooling, less shrinkage (but poor adhesion)
\item
  Warm bed: Slower cooling, more shrinkage (better adhesion)
\item
  Difference: Can be 0.2-0.3\% in dimensions
\end{itemize}

Solution: Standardize bed temperature for repeatable results

\paragraph*{Environmental Factors}\label{docs__pandoc__latex__src__3dmake_foundation__lessons_3dmake_10__measurement_calibration_guide.md__environmental-factors}

{\def\LTcaptype{none} % do not increment counter
\begin{longtable}[]{@{}
  >{\raggedright\arraybackslash}p{(\linewidth - 4\tabcolsep) * \real{0.2432}}
  >{\raggedright\arraybackslash}p{(\linewidth - 4\tabcolsep) * \real{0.3919}}
  >{\raggedright\arraybackslash}p{(\linewidth - 4\tabcolsep) * \real{0.3649}}@{}}
\toprule\noalign{}
\begin{minipage}[b]{\linewidth}\raggedright
Factor
\end{minipage} & \begin{minipage}[b]{\linewidth}\raggedright
Effect
\end{minipage} & \begin{minipage}[b]{\linewidth}\raggedright
Mitigation
\end{minipage} \\
\midrule\noalign{}
\endhead
\bottomrule\noalign{}
\endlastfoot
Room temperature & Affects cooling rate & Maintain 20-22C \\
Humidity & Affects material properties & Keep 40-60\% RH \\
Air flow & Inconsistent cooling & Avoid drafts near printer \\
Time of day & Material temperature varies & Print at consistent times \\
\end{longtable}
}

\subsubsection*{Quick Calibration Checklist}\label{docs__pandoc__latex__src__3dmake_foundation__lessons_3dmake_10__measurement_calibration_guide.md__quick-calibration-checklist}

\paragraph*{Before First Print with New Settings}\label{docs__pandoc__latex__src__3dmake_foundation__lessons_3dmake_10__measurement_calibration_guide.md__before-first-print-with-new-settings}

\begin{itemize}
\tightlist
\item[$\square$]
  E-steps calibration complete
\item[$\square$]
  First layer height verified
\item[$\square$]
  Nozzle diameter confirmed
\item[$\square$]
  Test print completed and measured
\end{itemize}

\paragraph*{Monthly Maintenance}\label{docs__pandoc__latex__src__3dmake_foundation__lessons_3dmake_10__measurement_calibration_guide.md__monthly-maintenance}

\begin{itemize}
\tightlist
\item[$\square$]
  Calibration cube printed and measured
\item[$\square$]
  Flow rate adjusted if needed
\item[$\square$]
  Layer height verified
\item[$\square$]
  Temperature consistency checked
\end{itemize}

\paragraph*{When Dimensions Are Critical}\label{docs__pandoc__latex__src__3dmake_foundation__lessons_3dmake_10__measurement_calibration_guide.md__when-dimensions-are-critical}

\begin{itemize}
\tightlist
\item[$\square$]
  Printed test part, let cool 24+ hours
\item[$\square$]
  Measured in multiple locations
\item[$\square$]
  Calculated average deviation
\item[$\square$]
  Flow rate adjusted accordingly
\item[$\square$]
  Re-printed and verified
\end{itemize}

\paragraph*{After Any Changes}\label{docs__pandoc__latex__src__3dmake_foundation__lessons_3dmake_10__measurement_calibration_guide.md__after-any-changes}

\begin{itemize}
\tightlist
\item[$\square$]
  Nozzle replacement -\textgreater{} Re-verify nozzle diameter
\item[$\square$]
  Bed leveling -\textgreater{} Re-verify first layer
\item[$\square$]
  Temperature changes -\textgreater{} Test print required
\item[$\square$]
  Material change -\textgreater{} Full calibration recommended
\end{itemize}

\subsubsection*{Measurement Tools Needed}\label{docs__pandoc__latex__src__3dmake_foundation__lessons_3dmake_10__measurement_calibration_guide.md__measurement-tools-needed}

{\def\LTcaptype{none} % do not increment counter
\begin{longtable}[]{@{}llll@{}}
\toprule\noalign{}
Tool & Cost & Accuracy & Use \\
\midrule\noalign{}
\endhead
\bottomrule\noalign{}
\endlastfoot
Digital Calipers & \$5-15 & +/-0.05mm & Primary measurements \\
Steel Ruler & \$3-10 & +/-1mm & Quick rough checks \\
Vernier Calipers & \$10-30 & +/-0.05mm & Precision work \\
Micrometer & \$20-50 & +/-0.01mm & Critical tolerances \\
Layer Height Gauge & DIY or \$5-10 & +/-0.05mm & Layer verification \\
\end{longtable}
}

Recommendation: Start with digital calipers (most versatile and affordable)

\subsubsection*{Troubleshooting Calibration Issues}\label{docs__pandoc__latex__src__3dmake_foundation__lessons_3dmake_10__measurement_calibration_guide.md__troubleshooting-calibration-issues}

Problem: Measurements still inconsistent after calibration

\begin{itemize}
\tightlist
\item
  Check if bed is level (temperature affects levelness)
\item
  Verify material is dry (moisture affects dimensions)
\item
  Ensure ambient temperature is stable
\item
  Try printing on different bed locations
\end{itemize}

Problem: Can\textquotesingle t achieve target dimensions

\begin{itemize}
\tightlist
\item
  Nozzle may be damaged/worn (try replacement)
\item
  Printer may have fundamental hardware issues
\item
  Review mechanical components (belts, screws)
\item
  Consider printer calibration limits
\end{itemize}

Problem: Dimensions drift over time

\begin{itemize}
\tightlist
\item
  Printer thermal properties changing
\item
  Nozzle wearing out (gradually gets smaller)
\item
  Bed surface degrading
\item
  Normal wear - recalibrate quarterly
\end{itemize}

\subsubsection*{Reference: Standard Test Models}\label{docs__pandoc__latex__src__3dmake_foundation__lessons_3dmake_10__measurement_calibration_guide.md__reference-standard-test-models}

These models are helpful for calibration:

\begin{enumerate}
\tightlist
\item
  Calibration Cube (20mm hollow) - Overall accuracy
\item
  Tolerance Test Box (various hole sizes) - Hole accuracy
\item
  Layer Tower (graduated heights) - Layer consistency
\item
  Thin Wall Test (walls 1-5mm) - Wall thickness accuracy
\end{enumerate}

Last Calibration Date: \_\\
Printer Model: \_\\
Current E-Steps: \_\\
Current Flow Rate: \_\\
Materials Calibrated For: \_

\subsection{Measurement Worksheet}\label{docs__pandoc__latex__src__3dmake_foundation__lessons_3dmake_10__measurement-worksheet.md__3dmake_foundation_lessons_3dmake_10-measurement-worksheet}

Student Name: \_ Date: \_ Project / Assignment: \_

\subsubsection*{How to Use This Worksheet}\label{docs__pandoc__latex__src__3dmake_foundation__lessons_3dmake_10__measurement-worksheet.md__how-to-use-this-worksheet}

\begin{enumerate}
\tightlist
\item
  Measure each feature three times and record all three values
\item
  Calculate the average: add the three values and divide by 3
\item
  Round to one decimal place (e.g., 23.4 mm)
\item
  Use the average in your OpenSCAD code - not a single measurement
\end{enumerate}

Units: All measurements in millimeters (mm) unless otherwise noted.

\subsubsection*{Object 1}\label{docs__pandoc__latex__src__3dmake_foundation__lessons_3dmake_10__measurement-worksheet.md__object-1}

Object description: \_

{\def\LTcaptype{none} % do not increment counter
\begin{longtable}[]{@{}
  >{\raggedright\arraybackslash}p{(\linewidth - 10\tabcolsep) * \real{0.1579}}
  >{\raggedright\arraybackslash}p{(\linewidth - 10\tabcolsep) * \real{0.3026}}
  >{\raggedright\arraybackslash}p{(\linewidth - 10\tabcolsep) * \real{0.1184}}
  >{\raggedright\arraybackslash}p{(\linewidth - 10\tabcolsep) * \real{0.1184}}
  >{\raggedright\arraybackslash}p{(\linewidth - 10\tabcolsep) * \real{0.1184}}
  >{\raggedright\arraybackslash}p{(\linewidth - 10\tabcolsep) * \real{0.1842}}@{}}
\toprule\noalign{}
\begin{minipage}[b]{\linewidth}\raggedright
Feature
\end{minipage} & \begin{minipage}[b]{\linewidth}\raggedright
What You\textquotesingle re Measuring
\end{minipage} & \begin{minipage}[b]{\linewidth}\raggedright
M1 (mm)
\end{minipage} & \begin{minipage}[b]{\linewidth}\raggedright
M2 (mm)
\end{minipage} & \begin{minipage}[b]{\linewidth}\raggedright
M3 (mm)
\end{minipage} & \begin{minipage}[b]{\linewidth}\raggedright
Average (mm)
\end{minipage} \\
\midrule\noalign{}
\endhead
\bottomrule\noalign{}
\endlastfoot
Length (X) & & & & & \\
Width (Y) & & & & & \\
Height (Z) & & & & & \\
Feature 4 & & & & & \\
Feature 5 & & & & & \\
\end{longtable}
}

Notes / sketches:

\begin{lstlisting}[style=Alabaster]
(describe the object here - label which direction is X, Y, Z)


\end{lstlisting}

\subsubsection*{Object 2}\label{docs__pandoc__latex__src__3dmake_foundation__lessons_3dmake_10__measurement-worksheet.md__object-2}

Object description: \_

{\def\LTcaptype{none} % do not increment counter
\begin{longtable}[]{@{}
  >{\raggedright\arraybackslash}p{(\linewidth - 10\tabcolsep) * \real{0.1579}}
  >{\raggedright\arraybackslash}p{(\linewidth - 10\tabcolsep) * \real{0.3026}}
  >{\raggedright\arraybackslash}p{(\linewidth - 10\tabcolsep) * \real{0.1184}}
  >{\raggedright\arraybackslash}p{(\linewidth - 10\tabcolsep) * \real{0.1184}}
  >{\raggedright\arraybackslash}p{(\linewidth - 10\tabcolsep) * \real{0.1184}}
  >{\raggedright\arraybackslash}p{(\linewidth - 10\tabcolsep) * \real{0.1842}}@{}}
\toprule\noalign{}
\begin{minipage}[b]{\linewidth}\raggedright
Feature
\end{minipage} & \begin{minipage}[b]{\linewidth}\raggedright
What You\textquotesingle re Measuring
\end{minipage} & \begin{minipage}[b]{\linewidth}\raggedright
M1 (mm)
\end{minipage} & \begin{minipage}[b]{\linewidth}\raggedright
M2 (mm)
\end{minipage} & \begin{minipage}[b]{\linewidth}\raggedright
M3 (mm)
\end{minipage} & \begin{minipage}[b]{\linewidth}\raggedright
Average (mm)
\end{minipage} \\
\midrule\noalign{}
\endhead
\bottomrule\noalign{}
\endlastfoot
Length (X) & & & & & \\
Width (Y) & & & & & \\
Height (Z) & & & & & \\
Feature 4 & & & & & \\
Feature 5 & & & & & \\
\end{longtable}
}

Notes / sketches:

\begin{lstlisting}[style=Alabaster]
(describe the object here)


\end{lstlisting}

\subsubsection*{Object 3}\label{docs__pandoc__latex__src__3dmake_foundation__lessons_3dmake_10__measurement-worksheet.md__object-3}

Object description: \_

{\def\LTcaptype{none} % do not increment counter
\begin{longtable}[]{@{}
  >{\raggedright\arraybackslash}p{(\linewidth - 10\tabcolsep) * \real{0.1579}}
  >{\raggedright\arraybackslash}p{(\linewidth - 10\tabcolsep) * \real{0.3026}}
  >{\raggedright\arraybackslash}p{(\linewidth - 10\tabcolsep) * \real{0.1184}}
  >{\raggedright\arraybackslash}p{(\linewidth - 10\tabcolsep) * \real{0.1184}}
  >{\raggedright\arraybackslash}p{(\linewidth - 10\tabcolsep) * \real{0.1184}}
  >{\raggedright\arraybackslash}p{(\linewidth - 10\tabcolsep) * \real{0.1842}}@{}}
\toprule\noalign{}
\begin{minipage}[b]{\linewidth}\raggedright
Feature
\end{minipage} & \begin{minipage}[b]{\linewidth}\raggedright
What You\textquotesingle re Measuring
\end{minipage} & \begin{minipage}[b]{\linewidth}\raggedright
M1 (mm)
\end{minipage} & \begin{minipage}[b]{\linewidth}\raggedright
M2 (mm)
\end{minipage} & \begin{minipage}[b]{\linewidth}\raggedright
M3 (mm)
\end{minipage} & \begin{minipage}[b]{\linewidth}\raggedright
Average (mm)
\end{minipage} \\
\midrule\noalign{}
\endhead
\bottomrule\noalign{}
\endlastfoot
Length (X) & & & & & \\
Width (Y) & & & & & \\
Height (Z) & & & & & \\
Feature 4 & & & & & \\
Feature 5 & & & & & \\
\end{longtable}
}

Notes / sketches:

\begin{lstlisting}[style=Alabaster]
(describe the object here)


\end{lstlisting}

\subsubsection*{Object 4}\label{docs__pandoc__latex__src__3dmake_foundation__lessons_3dmake_10__measurement-worksheet.md__object-4}

Object description: \_

{\def\LTcaptype{none} % do not increment counter
\begin{longtable}[]{@{}
  >{\raggedright\arraybackslash}p{(\linewidth - 10\tabcolsep) * \real{0.1579}}
  >{\raggedright\arraybackslash}p{(\linewidth - 10\tabcolsep) * \real{0.3026}}
  >{\raggedright\arraybackslash}p{(\linewidth - 10\tabcolsep) * \real{0.1184}}
  >{\raggedright\arraybackslash}p{(\linewidth - 10\tabcolsep) * \real{0.1184}}
  >{\raggedright\arraybackslash}p{(\linewidth - 10\tabcolsep) * \real{0.1184}}
  >{\raggedright\arraybackslash}p{(\linewidth - 10\tabcolsep) * \real{0.1842}}@{}}
\toprule\noalign{}
\begin{minipage}[b]{\linewidth}\raggedright
Feature
\end{minipage} & \begin{minipage}[b]{\linewidth}\raggedright
What You\textquotesingle re Measuring
\end{minipage} & \begin{minipage}[b]{\linewidth}\raggedright
M1 (mm)
\end{minipage} & \begin{minipage}[b]{\linewidth}\raggedright
M2 (mm)
\end{minipage} & \begin{minipage}[b]{\linewidth}\raggedright
M3 (mm)
\end{minipage} & \begin{minipage}[b]{\linewidth}\raggedright
Average (mm)
\end{minipage} \\
\midrule\noalign{}
\endhead
\bottomrule\noalign{}
\endlastfoot
Length (X) & & & & & \\
Width (Y) & & & & & \\
Height (Z) & & & & & \\
Feature 4 & & & & & \\
Feature 5 & & & & & \\
\end{longtable}
}

Notes / sketches:

\begin{lstlisting}[style=Alabaster]
(describe the object here)


\end{lstlisting}

\subsubsection*{Object 5}\label{docs__pandoc__latex__src__3dmake_foundation__lessons_3dmake_10__measurement-worksheet.md__object-5}

Object description: \_

{\def\LTcaptype{none} % do not increment counter
\begin{longtable}[]{@{}
  >{\raggedright\arraybackslash}p{(\linewidth - 10\tabcolsep) * \real{0.1579}}
  >{\raggedright\arraybackslash}p{(\linewidth - 10\tabcolsep) * \real{0.3026}}
  >{\raggedright\arraybackslash}p{(\linewidth - 10\tabcolsep) * \real{0.1184}}
  >{\raggedright\arraybackslash}p{(\linewidth - 10\tabcolsep) * \real{0.1184}}
  >{\raggedright\arraybackslash}p{(\linewidth - 10\tabcolsep) * \real{0.1184}}
  >{\raggedright\arraybackslash}p{(\linewidth - 10\tabcolsep) * \real{0.1842}}@{}}
\toprule\noalign{}
\begin{minipage}[b]{\linewidth}\raggedright
Feature
\end{minipage} & \begin{minipage}[b]{\linewidth}\raggedright
What You\textquotesingle re Measuring
\end{minipage} & \begin{minipage}[b]{\linewidth}\raggedright
M1 (mm)
\end{minipage} & \begin{minipage}[b]{\linewidth}\raggedright
M2 (mm)
\end{minipage} & \begin{minipage}[b]{\linewidth}\raggedright
M3 (mm)
\end{minipage} & \begin{minipage}[b]{\linewidth}\raggedright
Average (mm)
\end{minipage} \\
\midrule\noalign{}
\endhead
\bottomrule\noalign{}
\endlastfoot
Length (X) & & & & & \\
Width (Y) & & & & & \\
Height (Z) & & & & & \\
Feature 4 & & & & & \\
Feature 5 & & & & & \\
\end{longtable}
}

Notes / sketches:

\begin{lstlisting}[style=Alabaster]
(describe the object here)


\end{lstlisting}

\subsubsection*{Accuracy Check}\label{docs__pandoc__latex__src__3dmake_foundation__lessons_3dmake_10__measurement-worksheet.md__accuracy-check}

After completing all measurements, compare your averages with a partner who measured the same objects.

{\def\LTcaptype{none} % do not increment counter
\begin{longtable}[]{@{}
  >{\raggedright\arraybackslash}p{(\linewidth - 8\tabcolsep) * \real{0.1000}}
  >{\raggedright\arraybackslash}p{(\linewidth - 8\tabcolsep) * \real{0.2125}}
  >{\raggedright\arraybackslash}p{(\linewidth - 8\tabcolsep) * \real{0.3000}}
  >{\raggedright\arraybackslash}p{(\linewidth - 8\tabcolsep) * \real{0.2125}}
  >{\raggedright\arraybackslash}p{(\linewidth - 8\tabcolsep) * \real{0.1750}}@{}}
\toprule\noalign{}
\begin{minipage}[b]{\linewidth}\raggedright
Object
\end{minipage} & \begin{minipage}[b]{\linewidth}\raggedright
My Average (mm)
\end{minipage} & \begin{minipage}[b]{\linewidth}\raggedright
Partner\textquotesingle s Average (mm)
\end{minipage} & \begin{minipage}[b]{\linewidth}\raggedright
Difference (mm)
\end{minipage} & \begin{minipage}[b]{\linewidth}\raggedright
Within 1 mm?
\end{minipage} \\
\midrule\noalign{}
\endhead
\bottomrule\noalign{}
\endlastfoot
1 & & & & \\
2 & & & & \\
3 & & & & \\
4 & & & & \\
5 & & & & \\
\end{longtable}
}

If any difference is greater than 1 mm, remeasure together and find the source of the discrepancy.

\subsubsection*{Percent Error (Optional / Extension)}\label{docs__pandoc__latex__src__3dmake_foundation__lessons_3dmake_10__measurement-worksheet.md__percent-error-optional--extension}

If your instructor provides the "true" dimension of an object (measured with a reference instrument), you can calculate your percent error:

Formula: \texttt{\%\ error\ =\ (\textbar{}your\ average\ \ true\ value\textbar{}\ /\ true\ value)\ x\ 100}

{\def\LTcaptype{none} % do not increment counter
\begin{longtable}[]{@{}lllll@{}}
\toprule\noalign{}
Object & Your Average & True Value & Difference & \% Error \\
\midrule\noalign{}
\endhead
\bottomrule\noalign{}
\endlastfoot
& & & & \\
& & & & \\
\end{longtable}
}

A percent error under 2\% is excellent for caliper work at this level.

\subsubsection*{Reflection}\label{docs__pandoc__latex__src__3dmake_foundation__lessons_3dmake_10__measurement-worksheet.md__reflection}

\emph{Answer in complete sentences.}

\begin{enumerate}
\item
  Which measurement was most difficult to take and why?
\item
  Did your three measurements for any feature vary significantly? What might cause variation between repeated measurements?
\item
  If you were designing an object in OpenSCAD that needed to fit over one of these objects, which measurement would you use - your smallest, your largest, or your average? Why?
\end{enumerate}

\subsection{Common 3D Printing Issues and Solutions}\label{docs__pandoc__latex__src__3dmake_foundation__lessons_3dmake_10__common_issues_and_solutions.md__3dmake_foundation_lessons_3dmake_10-common_issues_and_solutions}

Comprehensive troubleshooting guide for diagnosing and fixing common 3D printing problems.

\subsubsection*{Pre-Print Issues}\label{docs__pandoc__latex__src__3dmake_foundation__lessons_3dmake_10__common_issues_and_solutions.md__pre-print-issues}

\paragraph*{Problem: Filament Won\textquotesingle t Load}\label{docs__pandoc__latex__src__3dmake_foundation__lessons_3dmake_10__common_issues_and_solutions.md__problem-filament-wont-load}

Symptoms:

\begin{itemize}
\tightlist
\item
  Extruder clicks/grinds but no filament moves
\item
  Nozzle temperature displays but stays clean
\item
  No material extrudes when manual extrude command sent
\end{itemize}

Diagnosis Checklist:

\begin{itemize}
\tightlist
\item[$\square$]
  Nozzle temperature high enough? (Check material specs)
\item[$\square$]
  Filament path clear of obstructions?
\item[$\square$]
  Drive gear has grip on filament (not polished smooth)?
\item[$\square$]
  Extruder tension set correctly?
\end{itemize}

Solutions:

\begin{enumerate}
\tightlist
\item
  Increase nozzle temperature (too cold = harder extrusion)
\item
  Clean extruder drive gear - brush with wire brush to restore grip
\item
  Check filament diameter - should be 1.75mm (+/-0.03mm)
\item
  Verify extruder tension - should grip firmly but not crush filament
\item
  Inspect nozzle - may be partially clogged (see nozzle clog section)
\end{enumerate}

\paragraph*{Problem: Warped/Damaged Filament}\label{docs__pandoc__latex__src__3dmake_foundation__lessons_3dmake_10__common_issues_and_solutions.md__problem-warpeddamaged-filament}

Symptoms:

\begin{itemize}
\tightlist
\item
  Filament appears bent or kinked on spool
\item
  Filament diameter inconsistent
\item
  Melted spots on filament surface
\end{itemize}

Diagnosis Checklist:

\begin{itemize}
\tightlist
\item[$\square$]
  Was filament stored properly (dry, cool)?
\item[$\square$]
  Spool has been sitting unused?
\item[$\square$]
  Transport/handling damage visible?
\end{itemize}

Solutions:

\begin{enumerate}
\tightlist
\item
  Discard damaged section - cut off \textasciitilde{}30cm if kinked
\item
  Increase humidity - try drying filament in low oven (50C for 2 hours PLA)
\item
  Check storage - PLA especially absorbs moisture
\item
  Replace spool - if damage is extensive
\end{enumerate}

\subsubsection*{Print Quality Issues}\label{docs__pandoc__latex__src__3dmake_foundation__lessons_3dmake_10__common_issues_and_solutions.md__print-quality-issues}

\paragraph*{Problem: Poor Bed Adhesion (First Layer Lifts)}\label{docs__pandoc__latex__src__3dmake_foundation__lessons_3dmake_10__common_issues_and_solutions.md__problem-poor-bed-adhesion-first-layer-lifts}

Symptoms:

\begin{itemize}
\tightlist
\item
  Print lifts off bed during first layer
\item
  Material rolls up at edges
\item
  Print comes free during printing
\end{itemize}

Diagnosis Checklist:

\begin{itemize}
\tightlist
\item[$\square$]
  Build plate level? (Should drag paper at all points)
\item[$\square$]
  Nozzle at correct height for first layer?
\item[$\square$]
  Bed clean and free of dust/oil?
\item[$\square$]
  Bed temperature adequate?
\end{itemize}

Solutions:

\begin{enumerate}
\tightlist
\item
  Re-level build plate - cold plate adjustment for accuracy
\item
  Clean build plate - wipe with isopropyl alcohol
\item
  Lower nozzle (if too high) - reduce Z by 0.05-0.1mm
\item
  Increase bed temperature - add 5-10C
\item
  Use adhesion aids:

  \begin{itemize}
  \tightlist
  \item
    PLA: Painter\textquotesingle s tape, glue stick, or bare clean plastic
  \item
    PETG: Textured bed or thin glue layer
  \item
    ABS: Heated bed + blue tape + ABS slurry (acetone + scraps)
  \item
    TPU: Clean bed, possibly elevated temperature
  \end{itemize}
\end{enumerate}

\paragraph*{Problem: Nozzle Clog}\label{docs__pandoc__latex__src__3dmake_foundation__lessons_3dmake_10__common_issues_and_solutions.md__problem-nozzle-clog}

Symptoms:

\begin{itemize}
\tightlist
\item
  No extrusion after initial layers
\item
  Consistent gap between nozzle and print
\item
  Pressure building in nozzle (smoking or popping)
\item
  Clicking sounds from extruder
\end{itemize}

Diagnosis Checklist:

\begin{itemize}
\tightlist
\item[$\square$]
  Filament jam visible in extruder?
\item[$\square$]
  Nozzle temperature high enough?
\item[$\square$]
  Print moved away from nozzle?
\item[$\square$]
  Recent failed print or debris?
\end{itemize}

Solutions (Increasing Intensity):

\begin{enumerate}
\item
  Immediate - Cold Pull:

  \begin{itemize}
  \tightlist
  \item
    Heat nozzle to 200C
  \item
    Grab filament and give firm pull while cooling
  \item
    Repeat 5-10 times
  \item
    May extract filament with debris
  \end{itemize}
\item
  Hot Swap:

  \begin{itemize}
  \tightlist
  \item
    Heat nozzle to printing temperature
  \item
    Remove extruder completely
  \item
    Use small drill bit (0.3-0.4mm) to carefully poke from top
  \item
    Don\textquotesingle t force - may damage nozzle
  \end{itemize}
\item
  Soak \& Poke:

  \begin{itemize}
  \tightlist
  \item
    Remove nozzle with wrench while hot
  \item
    Soak in heated acetone or strong cleaner (20-30 min)
  \item
    Use ultrasonic cleaner if available
  \item
    Use thin needle to clear passage
  \end{itemize}
\item
  Replacement:

  \begin{itemize}
  \tightlist
  \item
    If clog persists, nozzle may be damaged internally
  \item
    Replace with new nozzle (usually \$3-10)
  \item
    Always keep spare nozzles
  \end{itemize}
\end{enumerate}

\paragraph*{Problem: Under-Extrusion (Thin Walls, Weak Print)}\label{docs__pandoc__latex__src__3dmake_foundation__lessons_3dmake_10__common_issues_and_solutions.md__problem-under-extrusion-thin-walls-weak-print}

Symptoms:

\begin{itemize}
\tightlist
\item
  Print walls thinner than expected
\item
  Weak layer bonding (layers separate easily)
\item
  Visible horizontal lines in walls
\item
  Light-colored sections in print
\end{itemize}

Diagnosis Checklist:

\begin{itemize}
\tightlist
\item[$\square$]
  Nozzle partially clogged?
\item[$\square$]
  Extrusion multiplier/width set correctly?
\item[$\square$]
  Hot end temperature too low?
\item[$\square$]
  Filament diameter inconsistent?
\item[$\square$]
  Drive gear slipping?
\end{itemize}

Solutions:

\begin{enumerate}
\tightlist
\item
  Check extruder steps/mm - calibrate e-steps if possible
\item
  Increase flow rate - try 105-110\% in slicer
\item
  Raise temperature - add 5-10C
\item
  Slow down print speed - reduce 10-20\%
\item
  Check filament quality - test with different brand/batch
\item
  Clean drive gear - remove plastic buildup
\end{enumerate}

\paragraph*{Problem: Over-Extrusion (Blobs, Rough Texture)}\label{docs__pandoc__latex__src__3dmake_foundation__lessons_3dmake_10__common_issues_and_solutions.md__problem-over-extrusion-blobs-rough-texture}

Symptoms:

\begin{itemize}
\tightlist
\item
  Excess material squishing out between layers
\item
  Rough, bumpy texture on surface
\item
  Blobs or zits on walls
\item
  Layers slightly translucent
\end{itemize}

Diagnosis Checklist:

\begin{itemize}
\tightlist
\item[$\square$]
  Temperature too high?
\item[$\square$]
  Extrusion width too large?
\item[$\square$]
  Line width set wider than nozzle?
\item[$\square$]
  Flow rate too high?
\end{itemize}

Solutions:

\begin{enumerate}
\tightlist
\item
  Reduce flow rate - try 95\% in slicer
\item
  Lower temperature - reduce 5-10C
\item
  Check line width - should be \textasciitilde{}1.2x nozzle diameter (0.4mm nozzle = 0.48mm width)
\item
  Speed up print - higher speeds reduce oozing
\item
  Check nozzle size - confirm you\textquotesingle re using correct setting
\end{enumerate}

\paragraph*{Problem: Layer Shifting}\label{docs__pandoc__latex__src__3dmake_foundation__lessons_3dmake_10__common_issues_and_solutions.md__problem-layer-shifting}

Symptoms:

\begin{itemize}
\tightlist
\item
  Layers offset horizontally mid-print
\item
  X or Y axis suddenly jumps
\item
  Top portion of print misaligned
\item
  Usually happens at specific layer height
\end{itemize}

Diagnosis Checklist:

\begin{itemize}
\tightlist
\item[$\square$]
  Mechanical: Check X/Y pulley teeth, belts, screws loose?
\item[$\square$]
  Collision: Does print hit extruder/frame?
\item[$\square$]
  Speed: Trying to move too fast, losing steps?
\item[$\square$]
  Support: Is print unstable/wobbly?
\item[$\square$]
  Firmware: Recent changes to acceleration settings?
\end{itemize}

Solutions:

\begin{enumerate}
\item
  Check mechanics:

  \begin{itemize}
  \tightlist
  \item
    Manually move X/Y freely (no resistance)
  \item
    Tighten all visible screws
  \item
    Check belts for fraying (replace if damaged)
  \end{itemize}
\item
  Reduce speed:

  \begin{itemize}
  \tightlist
  \item
    Lower travel speed by 20-30\%
  \item
    Reduce acceleration in firmware
  \item
    Slow down print speed by 20\%
  \end{itemize}
\item
  Check print orientation:

  \begin{itemize}
  \tightlist
  \item
    Rotate model to reduce overhang
  \item
    Add supports to prevent wobbling
  \item
    Ensure solid perimeters around shift point
  \end{itemize}
\item
  Firmware:

  \begin{itemize}
  \tightlist
  \item
    Check acceleration settings (try 1000 mm/s)
  \item
    Verify step/mm calibration
  \item
    Update firmware if outdated
  \end{itemize}
\end{enumerate}

\subsubsection*{Advanced Troubleshooting}\label{docs__pandoc__latex__src__3dmake_foundation__lessons_3dmake_10__common_issues_and_solutions.md__advanced-troubleshooting}

\paragraph*{Problem: Stringing (Fine Filament Between Prints)}\label{docs__pandoc__latex__src__3dmake_foundation__lessons_3dmake_10__common_issues_and_solutions.md__problem-stringing-fine-filament-between-prints}

Cause: Nozzle oozes while traveling between print sections

Solutions:

\begin{enumerate}
\tightlist
\item
  Reduce temperature by 5-10C
\item
  Increase retraction distance (0.5-1.5mm more)
\item
  Increase retraction speed
\item
  Enable wipe on retract (slicer setting)
\item
  Check nozzle diameter (worn nozzles ooze more)
\end{enumerate}

\paragraph*{Problem: Warping (Corners Curl Up)}\label{docs__pandoc__latex__src__3dmake_foundation__lessons_3dmake_10__common_issues_and_solutions.md__problem-warping-corners-curl-up}

Cause: Material cools unevenly, contracts differently

Solutions (by material):

\begin{itemize}
\tightlist
\item
  ABS: Heated enclosure, slow cooling, higher bed temp
\item
  PETG: Reduce cooling fan, increase bed temp
\item
  PLA: Usually warps if bed too hot - reduce temperature
\end{itemize}

\paragraph*{Problem: Inconsistent Print Quality}\label{docs__pandoc__latex__src__3dmake_foundation__lessons_3dmake_10__common_issues_and_solutions.md__problem-inconsistent-print-quality}

Cause: Variable conditions print-to-print

Solutions:

\begin{enumerate}
\tightlist
\item
  Environmental: Maintain consistent room temperature
\item
  Material: Use same brand/batch filament
\item
  Calibration: Re-level bed before each print
\item
  Firmware: Disable bed leveling if unreliable
\item
  Maintenance: Clean nozzle after each print
\end{enumerate}

\subsubsection*{Diagnosis Decision Tree}\label{docs__pandoc__latex__src__3dmake_foundation__lessons_3dmake_10__common_issues_and_solutions.md__diagnosis-decision-tree}

\begin{lstlisting}[style=Alabaster]
Print fails?
+---- No extrusion -> Check nozzle temperature -> Clog? -> Cold pull/poke/replace
+---- Lifts off bed -> Clean bed -> Level bed -> Adjust height -> Increase temp
+---- Breaks mid-print -> Layer shift? -> Check mechanics, speed
|                    -> Under-extrusion? -> Increase flow/temp/speed
|                    -> Over-extrusion? -> Decrease flow/temp
+---- Bad surface quality -> Stringing? -> Reduce temp/increase retraction
|                      -> Blobs? -> Check flow rate, temperature
|                      -> Warping? -> Adjust bed temp, enclosed space
+---- Print weak -> Under-extrusion -> Increase flow rate, temperature
              -> Poor layer bonding -> Increase bed temp, first layer height

\end{lstlisting}

\subsubsection*{Maintenance to Prevent Issues}\label{docs__pandoc__latex__src__3dmake_foundation__lessons_3dmake_10__common_issues_and_solutions.md__maintenance-to-prevent-issues}

{\def\LTcaptype{none} % do not increment counter
\begin{longtable}[]{@{}
  >{\raggedright\arraybackslash}p{(\linewidth - 4\tabcolsep) * \real{0.3521}}
  >{\raggedright\arraybackslash}p{(\linewidth - 4\tabcolsep) * \real{0.3803}}
  >{\raggedright\arraybackslash}p{(\linewidth - 4\tabcolsep) * \real{0.2676}}@{}}
\toprule\noalign{}
\begin{minipage}[b]{\linewidth}\raggedright
Issue
\end{minipage} & \begin{minipage}[b]{\linewidth}\raggedright
Prevention
\end{minipage} & \begin{minipage}[b]{\linewidth}\raggedright
Frequency
\end{minipage} \\
\midrule\noalign{}
\endhead
\bottomrule\noalign{}
\endlastfoot
Clogs & Clean nozzle & Before each print \\
Bed adhesion & Level bed, clean plate & Before each print \\
Layer shift & Check mechanics, belts & Monthly \\
Inconsistent quality & Calibrate e-steps & Quarterly \\
Worn nozzle & Monitor extrusion quality & Every 6 months \\
Temperature fluctuation & Stable environment & Ongoing \\
\end{longtable}
}

Last Reviewed:\\
Printer Model:\\
Notes:

\subsection{Accessibility Audit Project}\label{docs__pandoc__latex__src__3dmake_foundation__lessons_3dmake_10__accessibility_audit.md__accessibility-audit-project}

\subsection{Accessibility Audit - Student Documentation Template (Extension Project)}\label{docs__pandoc__latex__src__3dmake_foundation__lessons_3dmake_10__accessibility_audit_student_template.md__3dmake_foundation_lessons_3dmake_10-accessibility_audit_student_template}

\begin{itemize}
\tightlist
\item
  Author:
\item
  Date:
\item
  Description: Audit 3D printing tools and workflows for accessibility barriers and recommend improvements.
\end{itemize}

\subsubsection*{Audit Scope}\label{docs__pandoc__latex__src__3dmake_foundation__lessons_3dmake_10__accessibility_audit_student_template.md__audit-scope}

Tools/workflows to audit:

\begin{enumerate}
\tightlist
\item
  \_
\item
  \_
\item
  \_
\end{enumerate}

Testing methodology (screen reader, keyboard-only, other):

\subsubsection*{Detailed Findings}\label{docs__pandoc__latex__src__3dmake_foundation__lessons_3dmake_10__accessibility_audit_student_template.md__detailed-findings}

\paragraph*{Tool 1: \_}\label{docs__pandoc__latex__src__3dmake_foundation__lessons_3dmake_10__accessibility_audit_student_template.md__tool-1-_}

\subparagraph*{Automated Accessibility Checks}\label{docs__pandoc__latex__src__3dmake_foundation__lessons_3dmake_10__accessibility_audit_student_template.md__automated-accessibility-checks}

\begin{itemize}
\tightlist
\item
  (Screenshots/results of testing)
\end{itemize}

\subparagraph*{Screen Reader Testing}\label{docs__pandoc__latex__src__3dmake_foundation__lessons_3dmake_10__accessibility_audit_student_template.md__screen-reader-testing}

\begin{itemize}
\tightlist
\item
  Navigation: (clear/unclear/broken)
\item
  Output readability: (readable/partially readable/unreadable)
\item
  Error messages: (helpful/unclear/missing)
\item
  Specific barriers found:
\end{itemize}

\subparagraph*{Keyboard Navigation}\label{docs__pandoc__latex__src__3dmake_foundation__lessons_3dmake_10__accessibility_audit_student_template.md__keyboard-navigation}

\begin{itemize}
\tightlist
\item
  All functions accessible via keyboard: Yes / No
\item
  Tab order logical: Yes / No
\item
  Specific barriers:
\end{itemize}

\subparagraph*{Recommendations for Tool 1}\label{docs__pandoc__latex__src__3dmake_foundation__lessons_3dmake_10__accessibility_audit_student_template.md__recommendations-for-tool-1}

{\def\LTcaptype{none} % do not increment counter
\begin{longtable}[]{@{}llll@{}}
\toprule\noalign{}
Recommendation & Priority & Feasibility & Impact \\
\midrule\noalign{}
\endhead
\bottomrule\noalign{}
\endlastfoot
& & & \\
& & & \\
\end{longtable}
}

\paragraph*{Tool 2: \_}\label{docs__pandoc__latex__src__3dmake_foundation__lessons_3dmake_10__accessibility_audit_student_template.md__tool-2-_}

(Repeat structure above)

\paragraph*{Tool 3: \_}\label{docs__pandoc__latex__src__3dmake_foundation__lessons_3dmake_10__accessibility_audit_student_template.md__tool-3-_}

(Repeat structure above)

\subsubsection*{Summary of Barriers}\label{docs__pandoc__latex__src__3dmake_foundation__lessons_3dmake_10__accessibility_audit_student_template.md__summary-of-barriers}

{\def\LTcaptype{none} % do not increment counter
\begin{longtable}[]{@{}llll@{}}
\toprule\noalign{}
Barrier & Frequency & Severity & Tools Affected \\
\midrule\noalign{}
\endhead
\bottomrule\noalign{}
\endlastfoot
& & & \\
& & & \\
\end{longtable}
}

\subsubsection*{Recommendations Matrix}\label{docs__pandoc__latex__src__3dmake_foundation__lessons_3dmake_10__accessibility_audit_student_template.md__recommendations-matrix}

{\def\LTcaptype{none} % do not increment counter
\begin{longtable}[]{@{}
  >{\raggedright\arraybackslash}p{(\linewidth - 8\tabcolsep) * \real{0.1449}}
  >{\raggedright\arraybackslash}p{(\linewidth - 8\tabcolsep) * \real{0.1884}}
  >{\raggedright\arraybackslash}p{(\linewidth - 8\tabcolsep) * \real{0.1159}}
  >{\raggedright\arraybackslash}p{(\linewidth - 8\tabcolsep) * \real{0.2319}}
  >{\raggedright\arraybackslash}p{(\linewidth - 8\tabcolsep) * \real{0.3188}}@{}}
\toprule\noalign{}
\begin{minipage}[b]{\linewidth}\raggedright
Priority
\end{minipage} & \begin{minipage}[b]{\linewidth}\raggedright
Feasibility
\end{minipage} & \begin{minipage}[b]{\linewidth}\raggedright
Impact
\end{minipage} & \begin{minipage}[b]{\linewidth}\raggedright
Recommendation
\end{minipage} & \begin{minipage}[b]{\linewidth}\raggedright
Implementation Steps
\end{minipage} \\
\midrule\noalign{}
\endhead
\bottomrule\noalign{}
\endlastfoot
High & Easy & High & & \\
High & Moderate & High & & \\
Moderate & Easy & Medium & & \\
\end{longtable}
}

\subsubsection*{Reflections}\label{docs__pandoc__latex__src__3dmake_foundation__lessons_3dmake_10__accessibility_audit_student_template.md__reflections}

\begin{itemize}
\tightlist
\item
  What surprised you about accessibility barriers in these tools?
\item
  How would you prioritize improvements if you had limited resources?
\item
  What role should users with disabilities play in accessibility testing?
\item
  How can accessibility become part of the design culture?
\end{itemize}

\subsubsection*{Action Plan}\label{docs__pandoc__latex__src__3dmake_foundation__lessons_3dmake_10__accessibility_audit_student_template.md__action-plan}

If implementing improvements, describe your plan:

\begin{itemize}
\tightlist
\item
  Which improvements will you tackle first?
\item
  How will you measure success?
\item
  Timeline for implementation:
\end{itemize}

\subsubsection*{Attachments}\label{docs__pandoc__latex__src__3dmake_foundation__lessons_3dmake_10__accessibility_audit_student_template.md__attachments}

\begin{itemize}
\tightlist
\item[$\square$]
  Screenshots of accessibility tests
\item[$\square$]
  Screen reader testing notes (with timestamps)
\item[$\square$]
  Keyboard navigation checklist
\item[$\square$]
  Detailed findings document
\item[$\square$]
  Recommendations matrix
\item[$\square$]
  Action plan (if implementing)
\end{itemize}

\subsubsection*{Teacher Feedback}\label{docs__pandoc__latex__src__3dmake_foundation__lessons_3dmake_10__accessibility_audit_student_template.md__teacher-feedback}

{\def\LTcaptype{none} % do not increment counter
\begin{longtable}[]{@{}lll@{}}
\toprule\noalign{}
Category & Score & Notes \\
\midrule\noalign{}
\endhead
\bottomrule\noalign{}
\endlastfoot
Problem \& Solution (0-3) & & \\
Design \& Code Quality (0-3) & & \\
Documentation (0-3) & & \\
Total (0-9) & & \\
\end{longtable}
}

Feedback:

\subsection{Accessibility Audit - Teacher Template (Extension Project)}\label{docs__pandoc__latex__src__3dmake_foundation__lessons_3dmake_10__accessibility_audit_teacher_template.md__3dmake_foundation_lessons_3dmake_10-accessibility_audit_teacher_template}

\subsubsection*{Briefing}\label{docs__pandoc__latex__src__3dmake_foundation__lessons_3dmake_10__accessibility_audit_teacher_template.md__briefing}

Students conduct a comprehensive accessibility audit of 3D printing workflows and tools. This project emphasizes universal design principles, accessibility testing, and inclusive documentation.

Key Learning: Accessibility as a design practice; testing with assistive technology; inclusive documentation.

Real-world Connection: Universal design benefits all users. Accessibility is increasingly a legal and ethical requirement in professional contexts.

\subsubsection*{Constraints}\label{docs__pandoc__latex__src__3dmake_foundation__lessons_3dmake_10__accessibility_audit_teacher_template.md__constraints}

\begin{itemize}
\tightlist
\item
  Audit must cover at least three tools or workflows (editor, slicer, terminal)
\item
  Testing must include both automated checks and user feedback
\item
  Recommendations must be specific and actionable
\item
  Documentation must support future accessibility improvements
\end{itemize}

\subsubsection*{Functional Requirements}\label{docs__pandoc__latex__src__3dmake_foundation__lessons_3dmake_10__accessibility_audit_teacher_template.md__functional-requirements}

\begin{itemize}
\tightlist
\item
  Audit identifies specific accessibility barriers with clear descriptions
\item
  Testing conducted with screen reader and keyboard-only navigation
\item
  Recommendations are prioritized by impact and feasibility
\item
  Documentation guides future accessibility improvements
\end{itemize}

\subsubsection*{Deliverables}\label{docs__pandoc__latex__src__3dmake_foundation__lessons_3dmake_10__accessibility_audit_teacher_template.md__deliverables}

\begin{itemize}
\tightlist
\item
  Completed audit template
\item
  Detailed findings for each tool/workflow
\item
  Screen reader testing log
\item
  Recommendations matrix (priority, feasibility, impact)
\item
  Reflection on accessibility challenges and opportunities
\item
  Action plan for improving course materials
\end{itemize}

\subsubsection*{Rubric}\label{docs__pandoc__latex__src__3dmake_foundation__lessons_3dmake_10__accessibility_audit_teacher_template.md__rubric}

\paragraph*{Category 1: Problem \& Solution (0-3)}\label{docs__pandoc__latex__src__3dmake_foundation__lessons_3dmake_10__accessibility_audit_teacher_template.md__category-1-problem--solution-0-3}

Audit is thorough and identifies real barriers. Recommendations are actionable.

\paragraph*{Category 2: Design \& Code Quality (0-3)}\label{docs__pandoc__latex__src__3dmake_foundation__lessons_3dmake_10__accessibility_audit_teacher_template.md__category-2-design--code-quality-0-3}

Testing methodology is rigorous. Findings are specific and well-documented.

\paragraph*{Category 3: Documentation (0-3)}\label{docs__pandoc__latex__src__3dmake_foundation__lessons_3dmake_10__accessibility_audit_teacher_template.md__category-3-documentation-0-3}

Audit template complete. Recommendations prioritized. Action plan clear.

\subsubsection*{Assessment Notes}\label{docs__pandoc__latex__src__3dmake_foundation__lessons_3dmake_10__accessibility_audit_teacher_template.md__assessment-notes}

\begin{itemize}
\tightlist
\item
  Strong submissions: Show rigorous testing methodology, specific barrier descriptions, prioritized recommendations, and sincere reflection on inclusion
\item
  Reinforce: Accessibility is everyone\textquotesingle s responsibility; small improvements compound
\item
  Extension: Implementation of recommended improvements; follow-up testing
\end{itemize}

\section{Lesson 10: Hands-On Practice Exercises and Troubleshooting}\label{docs__pandoc__latex__src__3dmake_foundation__lessons_3dmake_11__lessons_3dmake_11.md__lesson-10-hands-on-practice-exercises-and-troubleshooting}

Estimated time: 120--150 minutes

\subsection*{\texorpdfstring{Learning Objectives \footnote{OpenSCAD User Manual --- Hull and Minkowski. \url{https://en.wikibooks.org/wiki/OpenSCAD_User_Manual/Minkowski_and_Hull}}}{Learning Objectives }}\label{docs__pandoc__latex__src__3dmake_foundation__lessons_3dmake_11__lessons_3dmake_11.md__learning-objectives-}

\begin{itemize}
\tightlist
\item
  Apply skills from Lessons 1--9 in three integrated design exercises
\item
  Use calipers to measure and validate printed parts against specifications
\item
  Diagnose and fix non-manifold geometry errors
\item
  Perform tolerance stack-up analysis
\item
  Use \texttt{3dm\ describe} for non-visual validation
\end{itemize}

\subsection*{Materials}\label{docs__pandoc__latex__src__3dmake_foundation__lessons_3dmake_11__lessons_3dmake_11.md__materials}

\begin{itemize}
\tightlist
\item
  3dMake project
\item
  Printer and PLA filament
\item
  Digital calipers
\item
  Printed parts from previous lessons (or new prints from exercises below)
\end{itemize}

\subsection*{Exercise Set A: Phone Stand Refinement}\label{docs__pandoc__latex__src__3dmake_foundation__lessons_3dmake_11__lessons_3dmake_11.md__exercise-set-a-phone-stand-refinement}

\subsubsection*{A1 --- Measure and Iterate}\label{docs__pandoc__latex__src__3dmake_foundation__lessons_3dmake_11__lessons_3dmake_11.md__a1--measure-and-iterate}

Using calipers, measure your printed phone stand against the design specification:

\begin{lstlisting}[style=Alabaster]
Measurement checklist:
[ ] Base width = phone_w + 20 ± 0.3 mm
[ ] Base depth as calculated ± 0.5 mm
[ ] Back support angle (measure with angle gauge or protractor)
[ ] Lip depth = lip_h ± 0.3 mm
[ ] Phone fits and is stable (functional test)

\end{lstlisting}

For each out-of-spec dimension, calculate the correction and update the parameter in \texttt{src/main.scad}. Rebuild and reprint.

\subsubsection*{\texorpdfstring{A2 --- Tolerance Stack-Up Analysis \footnote{Digital Calipers Measurement Technique --- General metrology reference. See also: \href{https://github.com/mrhunsaker/VI_3DMake_OpenSCAD_Curriculum/3dMake_Foundation/Lessons_3dMake_11/../../assets/3dMake_Foundation/measurement_worksheet.md}{Measurement Worksheet Asset}}}{A2 --- Tolerance Stack-Up Analysis }}\label{docs__pandoc__latex__src__3dmake_foundation__lessons_3dmake_11__lessons_3dmake_11.md__a2--tolerance-stack-up-analysis-}

\begin{lstlisting}[style=Alabaster]
Scenario: phone stand cradle with three stacked parts:
- Base plate: designed 5mm, printed 5.12mm (+ 0.12mm)
- Back brace: designed 60mm, printed 59.87mm (- 0.13mm)
- Lip:        designed 15mm, printed 15.09mm (+ 0.09mm)

Total stack height: 5.12 + 59.87 + 15.09 = 80.08mm
Design intent:      5 + 60 + 15            = 80.00mm
Error:              80.08 - 80.00          = +0.08mm  (within 0.5mm spec — PASS)

Worst case (all errors same direction): 0.12 + 0.13 + 0.09 = 0.34mm — still within spec

\end{lstlisting}

Document your own measurements in a similar table.

\subsubsection*{\texorpdfstring{A3 --- Add a Cable Slot \footnote{3DMake GitHub Repository --- Command reference including \texttt{3dm\ describe}. \url{https://github.com/tdeck/3dmake}}}{A3 --- Add a Cable Slot }}\label{docs__pandoc__latex__src__3dmake_foundation__lessons_3dmake_11__lessons_3dmake_11.md__a3--add-a-cable-slot-}

Extend your phone stand design with a cable slot through the base:

\begin{lstlisting}[style=Alabaster, language=openscad]
cable_slot_w  = 12;   // mm
cable_slot_d  = 5;    // mm
cable_slot_z  = -0.001;

// Add to main difference() block:
translate([base_w/2 - cable_slot_w/2, 0, cable_slot_z])
  cube([cable_slot_w, cable_slot_d, base_h + 0.002]);

\end{lstlisting}

\subsection*{Exercise Set B: Keycap with Text}\label{docs__pandoc__latex__src__3dmake_foundation__lessons_3dmake_11__lessons_3dmake_11.md__exercise-set-b-keycap-with-text}

\subsubsection*{B1 --- Build a Mechanical Keyboard Keycap}\label{docs__pandoc__latex__src__3dmake_foundation__lessons_3dmake_11__lessons_3dmake_11.md__b1--build-a-mechanical-keyboard-keycap}

\begin{lstlisting}[style=Alabaster, language=openscad]
// Parametric keycap
key_w      = 18;
key_d      = 18;
key_h      = 7;
stem_r     = 2.75;  // MX stem: 5.5mm diameter
stem_h     = 3.8;
wall       = 1.5;
label_text = "A";

module keycap() {
  difference() {
    // Keycap body with slight top curve
    hull() {
      cube([key_w, key_d, key_h - 2], center=true);
      translate([0, 0, 1]) cube([key_w - 2, key_d - 2, key_h], center=true);
    }
    // Hollow inside
    translate([0, 0, -wall])
      cube([key_w - 2*wall, key_d - 2*wall, key_h], center=true);
    // MX stem hole
    translate([0, 0, -(key_h/2 + 0.001)])
      cylinder(r=stem_r + 0.1, h=stem_h + 0.001, $fn=16);
  }
}

module stem_mount() {
  translate([0, 0, -(key_h/2 + stem_h)])
    difference() {
      cylinder(r=stem_r + wall, h=stem_h, $fn=16);
      cylinder(r=stem_r, h=stem_h + 0.001, $fn=16);
    }
}

keycap();
stem_mount();

// Engrave label
translate([0, 0, key_h/2 - 0.8])
  linear_extrude(1.2)
    text(label_text, size=8, font="Liberation Sans:style=Bold",
         halign="center", valign="center", $fn=4);

\end{lstlisting}

\subsubsection*{B2 --- Validate with 3dm describe}\label{docs__pandoc__latex__src__3dmake_foundation__lessons_3dmake_11__lessons_3dmake_11.md__b2--validate-with-3dm-describe}

\begin{lstlisting}[style=Alabaster, language=bash]
3dm describe

\end{lstlisting}

Expected output should confirm the keycap geometry. Document what the AI description says and compare it to your design intent.

\subsubsection*{B3 --- Print and Test}\label{docs__pandoc__latex__src__3dmake_foundation__lessons_3dmake_11__lessons_3dmake_11.md__b3--print-and-test}

Print the keycap and test it on a Cherry MX switch (or compatible). If the stem is too tight, increase \texttt{stem\_r\ +\ 0.1} to \texttt{stem\_r\ +\ 0.15}. If too loose, decrease to \texttt{stem\_r\ +\ 0.05}.

\subsection*{Exercise Set C: Stackable Bins}\label{docs__pandoc__latex__src__3dmake_foundation__lessons_3dmake_11__lessons_3dmake_11.md__exercise-set-c-stackable-bins}

\subsubsection*{C1 --- Build a Three-Size Bin Set}\label{docs__pandoc__latex__src__3dmake_foundation__lessons_3dmake_11__lessons_3dmake_11.md__c1--build-a-three-size-bin-set}

Using the stackable bin module from Lesson 8, generate three sizes:

\begin{lstlisting}[style=Alabaster, language=openscad]
// Small bin
translate([0, 0, 0])
  bin_assembly(bin_w=60, bin_d=45, bin_h=30);

// Medium bin
translate([80, 0, 0])
  bin_assembly(bin_w=80, bin_d=60, bin_h=40);

// Large bin
translate([180, 0, 0])
  bin_assembly(bin_w=100, bin_d=80, bin_h=50);

\end{lstlisting}

\subsubsection*{C2 --- Diagnose and Fix Non-Manifold Geometry}\label{docs__pandoc__latex__src__3dmake_foundation__lessons_3dmake_11__lessons_3dmake_11.md__c2--diagnose-and-fix-non-manifold-geometry}

Non-manifold geometry occurs when faces share edges inconsistently (T-junctions, missing faces, zero-thickness walls). Common causes:

\begin{lstlisting}[style=Alabaster, language=openscad]
// PROBLEM: two cubes share a face exactly — may produce non-manifold edge
cube([20, 20, 10]);
translate([20, 0, 0]) cube([20, 20, 10]);  // touching at x=20 — ambiguous edge

// FIX 1: use union()
union() {
  cube([20, 20, 10]);
  translate([20, 0, 0]) cube([20, 20, 10]);
}

// FIX 2: overlap slightly
cube([20.001, 20, 10]);
translate([20, 0, 0]) cube([20, 20, 10]);

\end{lstlisting}

Diagnosis tool:

\begin{lstlisting}[style=Alabaster, language=bash]
3dm describe  # AI will often flag non-manifold geometry
# Also open STL in slicer and enable "Check for geometry errors"

\end{lstlisting}

\subsubsection*{C3 --- Advanced Geometry: hull() and minkowski()}\label{docs__pandoc__latex__src__3dmake_foundation__lessons_3dmake_11__lessons_3dmake_11.md__c3--advanced-geometry-hull-and-minkowski}

\begin{lstlisting}[style=Alabaster, language=openscad]
// hull() creates a convex envelope — useful for organic shapes
module smooth_transition() {
  hull() {
    translate([0, 0, 0]) cylinder(r=15, h=5, $fn=64);
    translate([0, 0, 30]) cylinder(r=5, h=2, $fn=64);
  }
}

smooth_transition();

// minkowski() adds the shape of a small object to every surface point
module rounded_hull() {
  minkowski() {
    hull() {
      cylinder(r=10, h=3, $fn=8);      // octagonal prism
      translate([30, 0, 0]) sphere(r=8, $fn=32);
    }
    sphere(r=2, $fn=16);  // rounds all edges by 2mm
  }
}

rounded_hull();

\end{lstlisting}

\subsection*{Quiz --- Lesson 3dMake.10 (15 questions)}\label{docs__pandoc__latex__src__3dmake_foundation__lessons_3dmake_11__lessons_3dmake_11.md__quiz--lesson-3dmake10-15-questions}

\begin{enumerate}
\tightlist
\item
  What tool do you use to measure printed part dimensions against the design specification?
\item
  What is tolerance stack-up, and why does it matter for multi-part assemblies?
\item
  What causes non-manifold geometry in OpenSCAD, and how do you detect it?
\item
  How does \texttt{hull()} differ from \texttt{union()}?
\item
  What does \texttt{3dm\ describe} help you verify about your model?
\item
  What does a Cherry MX stem measure in diameter, and what clearance would you add for a slip-fit keycap?
\item
  True or False: \texttt{find\ -newer} is an event-driven file change detection method.
\item
  If three parts each have ±0.15 mm tolerance, what is the worst-case total error for a three-part stack?
\item
  What does the \texttt{\$fn} parameter control in OpenSCAD?
\item
  Describe two methods for fixing non-manifold geometry caused by two touching (but not overlapping) shapes.
\item
  What is the difference between \texttt{hull()} and \texttt{minkowski()} for creating organic shapes? Give one use case for each.
\item
  What does \texttt{resize({[}50,\ 0,\ 0{]})} do, and why might \texttt{resize()} behave unexpectedly for non-uniform scaling?
\item
  When measuring a printed part with calipers, what is the difference between an inside measurement and an outside measurement, and when does that distinction matter for tolerance analysis?
\item
  Describe the iterative design workflow for dialing in press-fit tolerances: what do you print, what do you measure, and how do you adjust?
\item
  If \texttt{3dm\ describe} reports "the model appears non-manifold," what are three possible causes you would investigate in your OpenSCAD code?
\end{enumerate}

\subsection*{Extension Problems (15)}\label{docs__pandoc__latex__src__3dmake_foundation__lessons_3dmake_11__lessons_3dmake_11.md__extension-problems-15}

\begin{enumerate}
\tightlist
\item
  Create a tolerance sensitivity study: build 5 keycaps with stem clearance from 0.05--0.25 mm in 0.05 mm increments, print them, and record which values fit your switches.
\item
  Design a go/no-go gauge for a 10 mm nominal hole: a part with a "go" pin sized for slip-fit and a "no-go" pin sized for interference fit.
\item
  Write a printer calibration SOP (standard operating procedure): bed leveling, first-layer calibration, and dimension verification. Include a measurement checklist.
\item
  Build a three-tier stackable storage system for art supplies. Each tier has a different inner grid.
\item
  Conduct a tolerance stack-up analysis for your stackable bin system. Calculate worst-case misalignment.
\item
  Build a parametric test coupon that tests four different wall thicknesses (0.8, 1.2, 1.6, 2.0 mm) in a single print.
\item
  Design a caliper stand: a holder that holds your digital calipers at a comfortable angle for one-handed operation.
\item
  Build a non-manifold error catalog: intentionally create 5 different types of non-manifold geometry, document how each was created and how to fix it.
\item
  Use \texttt{hull()} to design a smooth ergonomic tool handle and compare it to a simple cylinder handle.
\item
  Create a printability checklist for new designs: overhangs, wall thickness, minimum feature size, support requirements. Apply it to your keycap and bin designs.
\item
  Research the \texttt{resize()} function in OpenSCAD. Build an example showing how it behaves differently from \texttt{scale()} for non-uniform resizing.
\item
  Design a multi-part assembly tutorial: a three-piece interlocking puzzle that teaches the concepts of tolerance, alignment, and slip-fit.
\item
  Build a "measurement worksheet" template in OpenSCAD: render a flat sheet that lists all key dimensions of a part as text, for printing alongside the part.
\item
  Create a chi-squared goodness-of-fit test for your printer\textquotesingle s dimensional accuracy: measure 20 prints of the same part and determine if the errors are normally distributed.
\item
  Write a comprehensive troubleshooting guide covering the 10 most common 3D printing failures you have encountered (or researched), with causes, prevention, and fixes.
\end{enumerate}

\subsection*{References and Helpful Resources}\label{docs__pandoc__latex__src__3dmake_foundation__lessons_3dmake_11__lessons_3dmake_11.md__references-and-helpful-resources}

\subsubsection*{Supplemental Resources}\label{docs__pandoc__latex__src__3dmake_foundation__lessons_3dmake_11__lessons_3dmake_11.md__supplemental-resources}

\begin{itemize}
\tightlist
\item
  \href{docs/pandoc/latex/src/assets/Programming_with_OpenSCAD.epub}{Programming with OpenSCAD EPUB Textbook} --- Troubleshooting and advanced geometry chapters
\item
  \href{https://github.com/ProgrammingWithOpenSCAD/CodeSolutions}{CodeSolutions Repository} --- Worked practice exercises
\item
  \href{https://programmingwithopenscad.github.io/quick-reference.html}{OpenSCAD Quick Reference} --- Function reference
\item
  \href{https://github.com/mrhunsaker/VI_3DMake_OpenSCAD_Curriculum/3dMake_Foundation/Lessons_3dMake_11/../../assets/3dMake_Foundation/master-rubric.md}{Master Rubric} --- Assessment criteria for practice exercises
\end{itemize}

\subsection{Stakeholder Interview Template}\label{docs__pandoc__latex__src__3dmake_foundation__lessons_3dmake_11__stakeholder_interview_template.md__3dmake_foundation_lessons_3dmake_11-stakeholder_interview_template}

Structured interview guide for gathering requirements directly from end-users and stakeholders.

\subsubsection*{Pre-Interview Preparation}\label{docs__pandoc__latex__src__3dmake_foundation__lessons_3dmake_11__stakeholder_interview_template.md__pre-interview-preparation}

Interview Details:

\begin{itemize}
\tightlist
\item
  Date: \_
\item
  Time: \_
\item
  Location: \_
\item
  Stakeholder Name: \_
\item
  Role/Title: \_
\item
  Contact: \_
\end{itemize}

Interview Objective: (What do you specifically want to learn?) \_

Estimated Duration: 45-60 minutes

\subsubsection*{Interview Sections}\label{docs__pandoc__latex__src__3dmake_foundation__lessons_3dmake_11__stakeholder_interview_template.md__interview-sections}

\paragraph*{Section 1: Context \& Background (5 minutes)}\label{docs__pandoc__latex__src__3dmake_foundation__lessons_3dmake_11__stakeholder_interview_template.md__section-1-context--background-5-minutes}

Opening Questions:

\begin{enumerate}
\item
  "Tell me about your role in this project."

  \begin{itemize}
  \tightlist
  \item
    \_
  \end{itemize}
\item
  "What problem are we trying to solve?"

  \begin{itemize}
  \tightlist
  \item
    \_
  \end{itemize}
\item
  "Who else will be affected by this design?"

  \begin{itemize}
  \tightlist
  \item
    \_
  \end{itemize}
\end{enumerate}

\paragraph*{Section 2: Current Situation (10 minutes)}\label{docs__pandoc__latex__src__3dmake_foundation__lessons_3dmake_11__stakeholder_interview_template.md__section-2-current-situation-10-minutes}

Current Process:

\begin{enumerate}
\item
  "How do you currently solve this problem?"

  \begin{itemize}
  \tightlist
  \item
    \_
  \end{itemize}
\item
  "What works well about the current approach?"

  \begin{itemize}
  \tightlist
  \item
    \_
  \end{itemize}
\item
  "What frustrates you most about the current process?"

  \begin{itemize}
  \tightlist
  \item
    \_
  \end{itemize}
\end{enumerate}

Pain Points:

\begin{itemize}
\tightlist
\item
  Primary issue: \_
\item
  Secondary issues: \_
\end{itemize}

\paragraph*{Section 3: Ideal Solution (15 minutes)}\label{docs__pandoc__latex__src__3dmake_foundation__lessons_3dmake_11__stakeholder_interview_template.md__section-3-ideal-solution-15-minutes}

Requirements (Ask open-ended):

\begin{enumerate}
\item
  "Ideally, what would this solution do?"

  \begin{itemize}
  \tightlist
  \item
    \_
  \end{itemize}
\item
  "How would you use it?"

  \begin{itemize}
  \tightlist
  \item
    \_
  \end{itemize}
\item
  "Where would it be used?"

  \begin{itemize}
  \tightlist
  \item
    \_
  \end{itemize}
\item
  "When would it be needed?"

  \begin{itemize}
  \tightlist
  \item
    \_
  \end{itemize}
\end{enumerate}

Specific Constraints:

\begin{itemize}
\tightlist
\item
  Size requirements: \_
\item
  Weight constraints:
\item
  Temperature tolerance: \_
\item
  Durability needs:
\item
  Cost budget:
\item
  Timeline: \_
\end{itemize}

\paragraph*{Section 4: Usage Patterns (10 minutes)}\label{docs__pandoc__latex__src__3dmake_foundation__lessons_3dmake_11__stakeholder_interview_template.md__section-4-usage-patterns-10-minutes}

Frequency \& Scale:

\begin{enumerate}
\item
  "How often would this be used?"

  \begin{itemize}
  \tightlist
  \item[$\square$]
    Daily  {[} {]} Weekly  {[} {]} Monthly  {[} {]} Occasionally
  \end{itemize}
\item
  "How many people need this?"

  \begin{itemize}
  \tightlist
  \item
    \_
  \end{itemize}
\item
  "For how long each use?"

  \begin{itemize}
  \tightlist
  \item
    \_
  \end{itemize}
\end{enumerate}

User Profile:

\begin{itemize}
\tightlist
\item
  Physical capabilities: \_
\item
  Technical expertise: \_
\item
  Accessibility needs: \_
\end{itemize}

\paragraph*{Section 5: Success Criteria (10 minutes)}\label{docs__pandoc__latex__src__3dmake_foundation__lessons_3dmake_11__stakeholder_interview_template.md__section-5-success-criteria-10-minutes}

What defines success? 1. 2. 3.

How will you measure if this works?

\begin{itemize}
\tightlist
\item
  Metric 1:
\item
  Metric 2:
\end{itemize}

\subsubsection*{What would be a failure?}\label{docs__pandoc__latex__src__3dmake_foundation__lessons_3dmake_11__stakeholder_interview_template.md__what-would-be-a-failure}

\paragraph*{Section 6: Preferences \& Priorities (10 minutes)}\label{docs__pandoc__latex__src__3dmake_foundation__lessons_3dmake_11__stakeholder_interview_template.md__section-6-preferences--priorities-10-minutes}

Priority Ranking (Ask: "If you could only have 3 features, which would they be?")

\begin{enumerate}
\tightlist
\item
  (Priority:  /10)
\item
  (Priority:  /10)
\item
  (Priority:  /10)
\end{enumerate}

Style/Aesthetic:

\begin{itemize}
\tightlist
\item
  Preferred look:
\item
  Must fit with existing: \_
\item
  Any visual requirements: \_
\end{itemize}

Material/Finish:

\begin{itemize}
\tightlist
\item
  Preferred materials:
\item
  Texture preference: \_
\item
  Color preferences:
\end{itemize}

\subsubsection*{Follow-Up Questions}\label{docs__pandoc__latex__src__3dmake_foundation__lessons_3dmake_11__stakeholder_interview_template.md__follow-up-questions}

Dig deeper on key points:

\begin{itemize}
\tightlist
\item
  "Tell me more about that..."
\item
  "Why is that important?"
\item
  "What would happen if that weren\textquotesingle t included?"
\item
  "Can you give me an example?"
\item
  "How would you compare that to...?"
\end{itemize}

\subsubsection*{Accessibility Considerations}\label{docs__pandoc__latex__src__3dmake_foundation__lessons_3dmake_11__stakeholder_interview_template.md__accessibility-considerations}

Always Ask:

\begin{itemize}
\tightlist
\item
  "Are there any accessibility requirements?"
\item
  "Will this be used by people with different abilities?"
\item
  "Are there sensory, mobility, or cognitive considerations?"
\item
  "What assistive devices might be used with this?"
\end{itemize}

Document:

\begin{itemize}
\tightlist
\item
  Visual considerations: \_
\item
  Motor/dexterity needs:
\item
  Hearing/audio considerations: \_
\item
  Cognitive/learning preferences:
\end{itemize}

\subsubsection*{Post-Interview Notes}\label{docs__pandoc__latex__src__3dmake_foundation__lessons_3dmake_11__stakeholder_interview_template.md__post-interview-notes}

Key Takeaways: 1. 2. 3.

Action Items:

\begin{itemize}
\tightlist
\item[$\square$]
\item[$\square$]
\end{itemize}

Questions to Follow Up On: 1. 2.

\subsubsection*{Conflicts/Concerns:}\label{docs__pandoc__latex__src__3dmake_foundation__lessons_3dmake_11__stakeholder_interview_template.md__conflictsconcerns}

\subsubsection*{Next Steps:}\label{docs__pandoc__latex__src__3dmake_foundation__lessons_3dmake_11__stakeholder_interview_template.md__next-steps}

\subsubsection*{Interview Analysis}\label{docs__pandoc__latex__src__3dmake_foundation__lessons_3dmake_11__stakeholder_interview_template.md__interview-analysis}

After conducting multiple interviews, summarize:

\paragraph*{Common Themes}\label{docs__pandoc__latex__src__3dmake_foundation__lessons_3dmake_11__stakeholder_interview_template.md__common-themes}

(What requirements appeared in multiple interviews?) 1. 2. 3.

\paragraph*{Conflicting Requirements}\label{docs__pandoc__latex__src__3dmake_foundation__lessons_3dmake_11__stakeholder_interview_template.md__conflicting-requirements}

(What requirements contradicted each other?)

\begin{itemize}
\tightlist
\item
  Stakeholder 1 wants: \_
\item
  Stakeholder 2 wants: \_
\item
  Resolution:
\end{itemize}

\paragraph*{Risk Areas}\label{docs__pandoc__latex__src__3dmake_foundation__lessons_3dmake_11__stakeholder_interview_template.md__risk-areas}

\subsubsection*{(What might be difficult or risky about this project?)}\label{docs__pandoc__latex__src__3dmake_foundation__lessons_3dmake_11__stakeholder_interview_template.md__what-might-be-difficult-or-risky-about-this-project}

\begin{itemize}
\tightlist
\item
\end{itemize}

Interview Conducted By: \_ Date Documented: \_
Approved By Stakeholder: \_ (signature/name)

\subsection{Functional Requirements Template}\label{docs__pandoc__latex__src__3dmake_foundation__lessons_3dmake_11__functional_requirements_template.md__3dmake_foundation_lessons_3dmake_11-functional_requirements_template}

Document and organize functional requirements derived from stakeholder interviews and research.

\subsubsection*{Project Information}\label{docs__pandoc__latex__src__3dmake_foundation__lessons_3dmake_11__functional_requirements_template.md__project-information}

\begin{itemize}
\tightlist
\item
  Project Name: \_
\item
  Version: \_
\item
  Date Created: \_
\item
  Last Updated: \_
\item
  Prepared By: \_
\item
  Approved By: \_
\end{itemize}

\subsubsection*{Executive Summary}\label{docs__pandoc__latex__src__3dmake_foundation__lessons_3dmake_11__functional_requirements_template.md__executive-summary}

Project Description:
(One paragraph overview of what this project is about) \_ \_

Business/User Need: (Why is this project important?) \_

Stakeholders:

\begin{itemize}
\tightlist
\item
  Primary user: \_
\item
  Secondary users:
\item
  Decision makers:
\end{itemize}

\subsubsection*{Functional Requirements}\label{docs__pandoc__latex__src__3dmake_foundation__lessons_3dmake_11__functional_requirements_template.md__functional-requirements}

\paragraph*{Organization Strategy}\label{docs__pandoc__latex__src__3dmake_foundation__lessons_3dmake_11__functional_requirements_template.md__organization-strategy}

Requirements organized by:

\begin{itemize}
\tightlist
\item[$\square$]
  User role/perspective
\item[$\square$]
  Feature/component
\item[$\square$]
  Priority level
\item[$\square$]
  Other: \_
\end{itemize}

\subsubsection*{Primary Functional Requirements}\label{docs__pandoc__latex__src__3dmake_foundation__lessons_3dmake_11__functional_requirements_template.md__primary-functional-requirements}

\emph{Core features that define the product}

\paragraph*{FR1: {[}Feature Name{]}}\label{docs__pandoc__latex__src__3dmake_foundation__lessons_3dmake_11__functional_requirements_template.md__fr1-feature-name}

Priority: {[} {]} Critical  {[} {]} High  {[} {]} Medium  {[} {]} Low

Description: What the product must do: \_

User Stories:

\begin{itemize}
\item
  "As a {[}user type{]}, I want to {[}action{]} so that {[}benefit{]}"

  \begin{itemize}
  \tightlist
  \item
    \_
  \end{itemize}
\item
  "As a {[}user type{]}, I want to {[}action{]} so that {[}benefit{]}"

  \begin{itemize}
  \tightlist
  \item
    \_
  \end{itemize}
\end{itemize}

Acceptance Criteria:

\begin{itemize}
\tightlist
\item[$\square$]
\item[$\square$]
\item[$\square$]
\end{itemize}

Related Requirements:

\paragraph*{FR2: {[}Feature Name{]}}\label{docs__pandoc__latex__src__3dmake_foundation__lessons_3dmake_11__functional_requirements_template.md__fr2-feature-name}

Priority: {[} {]} Critical  {[} {]} High  {[} {]} Medium  {[} {]} Low

Description: \_

\subsubsection*{User Stories:}\label{docs__pandoc__latex__src__3dmake_foundation__lessons_3dmake_11__functional_requirements_template.md__user-stories}

\begin{itemize}
\tightlist
\item
\end{itemize}

Acceptance Criteria:

\begin{itemize}
\tightlist
\item[$\square$]
\item[$\square$]
\end{itemize}

Related Requirements:

\paragraph*{FR3: {[}Feature Name{]}}\label{docs__pandoc__latex__src__3dmake_foundation__lessons_3dmake_11__functional_requirements_template.md__fr3-feature-name}

(Continue with additional primary requirements)

\subsubsection*{Secondary Functional Requirements}\label{docs__pandoc__latex__src__3dmake_foundation__lessons_3dmake_11__functional_requirements_template.md__secondary-functional-requirements}

\emph{Important supporting features}

\paragraph*{FR-S1: {[}Feature Name{]}}\label{docs__pandoc__latex__src__3dmake_foundation__lessons_3dmake_11__functional_requirements_template.md__fr-s1-feature-name}

Priority: {[} {]} High  {[} {]} Medium  {[} {]} Low

Description: \_

Acceptance Criteria:

\begin{itemize}
\tightlist
\item[$\square$]
\item[$\square$]
\end{itemize}

\paragraph*{FR-S2: {[}Feature Name{]}}\label{docs__pandoc__latex__src__3dmake_foundation__lessons_3dmake_11__functional_requirements_template.md__fr-s2-feature-name}

(Continue with secondary requirements)

\subsubsection*{Non-Functional Requirements}\label{docs__pandoc__latex__src__3dmake_foundation__lessons_3dmake_11__functional_requirements_template.md__non-functional-requirements}

\emph{Performance, reliability, and design properties}

\paragraph*{Performance Requirements}\label{docs__pandoc__latex__src__3dmake_foundation__lessons_3dmake_11__functional_requirements_template.md__performance-requirements}

\begin{itemize}
\tightlist
\item
  Response time: \_
\item
  Throughput: \_
\item
  Resource usage: \_
\end{itemize}

\paragraph*{Reliability Requirements}\label{docs__pandoc__latex__src__3dmake_foundation__lessons_3dmake_11__functional_requirements_template.md__reliability-requirements}

\begin{itemize}
\tightlist
\item
  Failure rate acceptable: \_
\item
  Recovery capability: \_
\item
  Data persistence: \_
\end{itemize}

\paragraph*{Physical Requirements (for 3D printed objects)}\label{docs__pandoc__latex__src__3dmake_foundation__lessons_3dmake_11__functional_requirements_template.md__physical-requirements-for-3d-printed-objects}

\begin{itemize}
\tightlist
\item
  Dimensions:
\item
  Weight capacity:
\item
  Material properties:
\item
  Temperature range:
\item
  Durability:
\end{itemize}

\paragraph*{Accessibility Requirements}\label{docs__pandoc__latex__src__3dmake_foundation__lessons_3dmake_11__functional_requirements_template.md__accessibility-requirements}

\begin{itemize}
\item[$\square$]
  Usable by people with visual impairment

  \begin{itemize}
  \tightlist
  \item
    How:
  \end{itemize}
\item[$\square$]
  Usable by people with motor impairment

  \begin{itemize}
  \tightlist
  \item
    How:
  \end{itemize}
\item[$\square$]
  Usable by people with hearing impairment

  \begin{itemize}
  \tightlist
  \item
    How:
  \end{itemize}
\item[$\square$]
  Usable by people with cognitive differences

  \begin{itemize}
  \tightlist
  \item
    How:
  \end{itemize}
\end{itemize}

\subsubsection*{Constraint Requirements}\label{docs__pandoc__latex__src__3dmake_foundation__lessons_3dmake_11__functional_requirements_template.md__constraint-requirements}

\emph{Limitations on design and implementation}

\paragraph*{Technical Constraints}\label{docs__pandoc__latex__src__3dmake_foundation__lessons_3dmake_11__functional_requirements_template.md__technical-constraints}

\begin{itemize}
\tightlist
\item
  Must work with:
\item
  Must not require:
\item
  Must be compatible with:
\end{itemize}

\paragraph*{Physical Constraints}\label{docs__pandoc__latex__src__3dmake_foundation__lessons_3dmake_11__functional_requirements_template.md__physical-constraints}

\begin{itemize}
\tightlist
\item
  Cannot exceed (size/weight/cost): \_
\item
  Must fit in/with: \_
\item
  Must be available by (date): \_
\end{itemize}

\paragraph*{Regulatory/Safety Constraints}\label{docs__pandoc__latex__src__3dmake_foundation__lessons_3dmake_11__functional_requirements_template.md__regulatorysafety-constraints}

\begin{itemize}
\tightlist
\item
  Must comply with:
\item
  Must not:
\item
  Safety considerations: \_
\end{itemize}

\paragraph*{Cost Constraints}\label{docs__pandoc__latex__src__3dmake_foundation__lessons_3dmake_11__functional_requirements_template.md__cost-constraints}

\begin{itemize}
\tightlist
\item
  Maximum budget:
\item
  Target unit cost: \_
\end{itemize}

\subsubsection*{Environmental Context}\label{docs__pandoc__latex__src__3dmake_foundation__lessons_3dmake_11__functional_requirements_template.md__environmental-context}

\paragraph*{Use Environment}\label{docs__pandoc__latex__src__3dmake_foundation__lessons_3dmake_11__functional_requirements_template.md__use-environment}

\begin{itemize}
\tightlist
\item
  Location(s): \_
\item
  Climate conditions: \_
\item
  Physical surroundings: \_
\item
  Typical usage pattern: \_
\end{itemize}

\paragraph*{Maintenance \& Lifecycle}\label{docs__pandoc__latex__src__3dmake_foundation__lessons_3dmake_11__functional_requirements_template.md__maintenance--lifecycle}

\begin{itemize}
\tightlist
\item
  Expected lifespan:
\item
  Maintenance needed:
\item
  End-of-life handling:
\end{itemize}

\subsubsection*{Dependency Mapping}\label{docs__pandoc__latex__src__3dmake_foundation__lessons_3dmake_11__functional_requirements_template.md__dependency-mapping}

Requirements that depend on other requirements:

{\def\LTcaptype{none} % do not increment counter
\begin{longtable}[]{@{}lll@{}}
\toprule\noalign{}
Requirement & Depends On & Notes \\
\midrule\noalign{}
\endhead
\bottomrule\noalign{}
\endlastfoot
FR1 & & \\
FR2 & FR1 & Cannot implement without FR1 \\
FR3 & & \\
\end{longtable}
}

External dependencies:

\begin{itemize}
\tightlist
\item
  Third-party components needed:
\item
  Integration points:
\end{itemize}

\subsubsection*{Scope Definition}\label{docs__pandoc__latex__src__3dmake_foundation__lessons_3dmake_11__functional_requirements_template.md__scope-definition}

\paragraph*{What IS In Scope}\label{docs__pandoc__latex__src__3dmake_foundation__lessons_3dmake_11__functional_requirements_template.md__what-is-in-scope}

\begin{itemize}
\tightlist
\item
\item
\item
\end{itemize}

\paragraph*{What IS NOT In Scope}\label{docs__pandoc__latex__src__3dmake_foundation__lessons_3dmake_11__functional_requirements_template.md__what-is-not-in-scope}

\begin{itemize}
\tightlist
\item
\item
\end{itemize}

\paragraph*{Future Considerations (Out of Scope but noted)}\label{docs__pandoc__latex__src__3dmake_foundation__lessons_3dmake_11__functional_requirements_template.md__future-considerations-out-of-scope-but-noted}

\begin{itemize}
\tightlist
\item
\end{itemize}

\subsubsection*{Change Control}\label{docs__pandoc__latex__src__3dmake_foundation__lessons_3dmake_11__functional_requirements_template.md__change-control}

Requirements Changes:

{\def\LTcaptype{none} % do not increment counter
\begin{longtable}[]{@{}llll@{}}
\toprule\noalign{}
Change Request & Date & Reason & Status \\
\midrule\noalign{}
\endhead
\bottomrule\noalign{}
\endlastfoot
& & & \\
& & & \\
\end{longtable}
}

\subsubsection*{Verification Plan}\label{docs__pandoc__latex__src__3dmake_foundation__lessons_3dmake_11__functional_requirements_template.md__verification-plan}

How will we verify each requirement is met?

{\def\LTcaptype{none} % do not increment counter
\begin{longtable}[]{@{}llll@{}}
\toprule\noalign{}
Requirement & Verification Method & Test Case & Status \\
\midrule\noalign{}
\endhead
\bottomrule\noalign{}
\endlastfoot
FR1 & {[}Inspection/Test/Demo{]} & & \\
FR2 & {[}Inspection/Test/Demo{]} & & \\
\end{longtable}
}

\subsubsection*{Sign-Off}\label{docs__pandoc__latex__src__3dmake_foundation__lessons_3dmake_11__functional_requirements_template.md__sign-off}

Stakeholder Approval:

{\def\LTcaptype{none} % do not increment counter
\begin{longtable}[]{@{}llll@{}}
\toprule\noalign{}
Stakeholder & Title & Signature & Date \\
\midrule\noalign{}
\endhead
\bottomrule\noalign{}
\endlastfoot
& & & \\
& & & \\
\end{longtable}
}

Requirements Baseline Approved:
Date:   Status:  Approved /  Pending /  Rejected

\subsubsection*{Reference Documents}\label{docs__pandoc__latex__src__3dmake_foundation__lessons_3dmake_11__functional_requirements_template.md__reference-documents}

\begin{itemize}
\tightlist
\item
  Interview notes:
\item
  Design sketches:
\item
  Related specifications: \_
\item
  Standards/guidelines:
\end{itemize}

Document Version History:

{\def\LTcaptype{none} % do not increment counter
\begin{longtable}[]{@{}llll@{}}
\toprule\noalign{}
Version & Date & Author & Changes \\
\midrule\noalign{}
\endhead
\bottomrule\noalign{}
\endlastfoot
1.0 & & & Initial draft \\
& & & \\
\end{longtable}
}

\subsection{Design Specification Template}\label{docs__pandoc__latex__src__3dmake_foundation__lessons_3dmake_11__design_specification_template.md__3dmake_foundation_lessons_3dmake_11-design_specification_template}

Technical design document specifying how requirements will be implemented.

\subsubsection*{Design Document Information}\label{docs__pandoc__latex__src__3dmake_foundation__lessons_3dmake_11__design_specification_template.md__design-document-information}

\begin{itemize}
\tightlist
\item
  Project: \_
\item
  Design Version: \_
\item
  Date: \_
\item
  Designed By: \_
\item
  Reviewed By: \_
\end{itemize}

\subsubsection*{Design Overview}\label{docs__pandoc__latex__src__3dmake_foundation__lessons_3dmake_11__design_specification_template.md__design-overview}

Problem Statement: (What are we solving?) \_

Solution Approach: (High-level how we\textquotesingle ll solve it) \_

Key Design Decisions: 1. 2. 3.

\subsubsection*{Component Breakdown}\label{docs__pandoc__latex__src__3dmake_foundation__lessons_3dmake_11__design_specification_template.md__component-breakdown}

\paragraph*{Component 1: {[}Component Name{]}}\label{docs__pandoc__latex__src__3dmake_foundation__lessons_3dmake_11__design_specification_template.md__component-1-component-name}

Purpose: \_

Specifications:

\begin{itemize}
\tightlist
\item
  Dimensions:
\item
  Material:
\item
  Quantity:
\end{itemize}

Design Rationale: (Why this approach?) \_

Related Components:

\begin{itemize}
\tightlist
\item
  Connects to: \_
\item
  Interfaces with:
\end{itemize}

Manufacturing Considerations:

\begin{itemize}
\tightlist
\item
  Support structures needed: \_
\item
  Orientation for print: \_
\item
  Estimated print time: \_
\end{itemize}

\paragraph*{Component 2: {[}Component Name{]}}\label{docs__pandoc__latex__src__3dmake_foundation__lessons_3dmake_11__design_specification_template.md__component-2-component-name}

(Repeat structure for each component)

\subsubsection*{Assembly Design}\label{docs__pandoc__latex__src__3dmake_foundation__lessons_3dmake_11__design_specification_template.md__assembly-design}

Overall Assembly Structure: (How components fit together) \_

Assembly Sequence: 1. 2. 3.

Fastening Methods:

\begin{itemize}
\tightlist
\item[$\square$]
  Snap fits (location: \_)
\item[$\square$]
  Threaded inserts (quantity: )
\item[$\square$]
  Glue/adhesive (type: )
\item[$\square$]
  Other: \_
\end{itemize}

Assembly Challenges \& Solutions:

\begin{itemize}
\tightlist
\item
  Challenge: \_ Solution:
\end{itemize}

\subsubsection*{Material Selection}\label{docs__pandoc__latex__src__3dmake_foundation__lessons_3dmake_11__design_specification_template.md__material-selection}

Primary Material:

Why this material?

{\def\LTcaptype{none} % do not increment counter
\begin{longtable}[]{@{}llll@{}}
\toprule\noalign{}
Property & Requirement & Selected Material & Alternative \\
\midrule\noalign{}
\endhead
\bottomrule\noalign{}
\endlastfoot
Strength & & & \\
Flexibility & & & \\
Temperature & & & \\
Cost & & & \\
Availability & & & \\
\end{longtable}
}

Material Properties:

\begin{itemize}
\tightlist
\item
  Nozzle temperature:
\item
  Bed temperature:
\item
  Print speed: \_
\item
  Support required:
\item
  Post-processing:
\end{itemize}

Material Alternatives \& Trade-offs:

\begin{itemize}
\tightlist
\item
  Option 1: \_
\item
  Option 2: \_
\item
  Selected option rationale: \_
\end{itemize}

\subsubsection*{Design Features}\label{docs__pandoc__latex__src__3dmake_foundation__lessons_3dmake_11__design_specification_template.md__design-features}

\paragraph*{Feature 1: {[}Feature Name{]}}\label{docs__pandoc__latex__src__3dmake_foundation__lessons_3dmake_11__design_specification_template.md__feature-1-feature-name}

Purpose: \_

Design Details:

\begin{itemize}
\tightlist
\item
  Dimensions:
\item
  Placement: \_
\item
  Tolerances:
\end{itemize}

Rationale: (Why designed this way?) \_

Related Requirement: (Which requirement does this fulfill?) \_

\paragraph*{Feature 2: {[}Feature Name{]}}\label{docs__pandoc__latex__src__3dmake_foundation__lessons_3dmake_11__design_specification_template.md__feature-2-feature-name}

(Repeat for each significant feature)

\subsubsection*{Tolerance \& Fit Analysis}\label{docs__pandoc__latex__src__3dmake_foundation__lessons_3dmake_11__design_specification_template.md__tolerance--fit-analysis}

Critical Dimensions:

{\def\LTcaptype{none} % do not increment counter
\begin{longtable}[]{@{}llll@{}}
\toprule\noalign{}
Dimension & Tolerance & Rationale & Risk \\
\midrule\noalign{}
\endhead
\bottomrule\noalign{}
\endlastfoot
& +/-\_mm & & \\
& +/-\_mm & & \\
\end{longtable}
}

Fit Relationships:

\begin{itemize}
\tightlist
\item
  Part A to Part B: \_
\item
  Part B to Part C: \_
\end{itemize}

Validation Plan:

\begin{itemize}
\tightlist
\item[$\square$]
  Print test version before production
\item[$\square$]
  Measure critical dimensions
\item[$\square$]
  Test assembly fit
\item[$\square$]
  Perform functional test
\end{itemize}

\subsubsection*{Manufacturing Planning}\label{docs__pandoc__latex__src__3dmake_foundation__lessons_3dmake_11__design_specification_template.md__manufacturing-planning}

\paragraph*{Print Parameters}\label{docs__pandoc__latex__src__3dmake_foundation__lessons_3dmake_11__design_specification_template.md__print-parameters}

Slicer Settings:

\begin{itemize}
\tightlist
\item
  Layer height:
\item
  Infill:
\item
  Support material:
\item
  Raft/brim: \_
\end{itemize}

Print Time \& Material:

\begin{itemize}
\tightlist
\item
  Estimated time:
\item
  Material weight: \_
\item
  Cost estimate: \_
\end{itemize}

Orientation Strategy: (How will this be oriented during printing?)

\begin{itemize}
\tightlist
\item
  Rationale: \_
\item
  Risks: \_
\item
  Mitigation:
\end{itemize}

\paragraph*{Post-Processing}\label{docs__pandoc__latex__src__3dmake_foundation__lessons_3dmake_11__design_specification_template.md__post-processing}

Cleanup:

\begin{itemize}
\tightlist
\item[$\square$]
  Remove supports (method: )
\item[$\square$]
  Remove raft/brim (method: \_)
\item[$\square$]
  Sand edges (grit: \_)
\item[$\square$]
  Smooth surfaces (method: )
\end{itemize}

Finishing:

\begin{itemize}
\tightlist
\item[$\square$]
  Paint/coat (type: )
\item[$\square$]
  Treat for durability:
\item[$\square$]
  Inspect quality:
\end{itemize}

Assembly Work:

\begin{itemize}
\tightlist
\item[$\square$]
  Insert threaded nuts/inserts
\item[$\square$]
  Assemble components
\item[$\square$]
  Test functionality
\end{itemize}

Estimated Total Time:  hours

\subsubsection*{Testing Plan}\label{docs__pandoc__latex__src__3dmake_foundation__lessons_3dmake_11__design_specification_template.md__testing-plan}

Functionality Tests:

{\def\LTcaptype{none} % do not increment counter
\begin{longtable}[]{@{}llll@{}}
\toprule\noalign{}
Test & Method & Pass Criteria & Status \\
\midrule\noalign{}
\endhead
\bottomrule\noalign{}
\endlastfoot
& & & \\
& & & \\
\end{longtable}
}

Durability Tests:

\begin{itemize}
\tightlist
\item
  Stress test: \_
\item
  Environmental exposure:
\item
  Lifecycle test: \_
\end{itemize}

Tolerance Verification:

\begin{itemize}
\tightlist
\item
  Measure key dimensions
\item
  Compare to specification
\item
  Document variance
\end{itemize}

\subsubsection*{Design Risks \& Mitigation}\label{docs__pandoc__latex__src__3dmake_foundation__lessons_3dmake_11__design_specification_template.md__design-risks--mitigation}

Risk 1:

\begin{itemize}
\tightlist
\item
  Description: \_
\item
  Probability: {[} {]} High  {[} {]} Medium  {[} {]} Low
\item
  Impact: {[} {]} Critical  {[} {]} Major  {[} {]} Minor
\item
  Mitigation:
\item
  Contingency: \_
\end{itemize}

Risk 2: (Continue for identified risks)

\subsubsection*{Design Alternatives Considered}\label{docs__pandoc__latex__src__3dmake_foundation__lessons_3dmake_11__design_specification_template.md__design-alternatives-considered}

Alternative 1:

\begin{itemize}
\tightlist
\item
  Approach: \_
\item
  Pros:
\item
  Cons:
\item
  Why not selected:
\end{itemize}

Alternative 2: (Continue for each alternative considered)

\subsubsection*{Technical Specifications}\label{docs__pandoc__latex__src__3dmake_foundation__lessons_3dmake_11__design_specification_template.md__technical-specifications}

\paragraph*{Dimensional Drawing Reference}\label{docs__pandoc__latex__src__3dmake_foundation__lessons_3dmake_11__design_specification_template.md__dimensional-drawing-reference}

(Attach or describe detailed drawings/models)

\begin{itemize}
\tightlist
\item
  CAD file:
\item
  Drawing revision:
\end{itemize}

\paragraph*{Performance Specifications}\label{docs__pandoc__latex__src__3dmake_foundation__lessons_3dmake_11__design_specification_template.md__performance-specifications}

\begin{itemize}
\tightlist
\item
  Load capacity: \_
\item
  Temperature range: \_
\item
  Lifespan:
\item
  Accuracy/precision:
\end{itemize}

\subsubsection*{Bill of Materials (BOM)}\label{docs__pandoc__latex__src__3dmake_foundation__lessons_3dmake_11__design_specification_template.md__bill-of-materials-bom}

{\def\LTcaptype{none} % do not increment counter
\begin{longtable}[]{@{}lllll@{}}
\toprule\noalign{}
Item & Part \# & Quantity & Unit Cost & Notes \\
\midrule\noalign{}
\endhead
\bottomrule\noalign{}
\endlastfoot
3D Printed Main Body & - & 1 & \$0.XX & PLA \\
& & & & \\
& & & & \\
\end{longtable}
}

Total Material Cost: \$

\subsubsection*{Design Verification Checklist}\label{docs__pandoc__latex__src__3dmake_foundation__lessons_3dmake_11__design_specification_template.md__design-verification-checklist}

Before manufacturing:

\begin{itemize}
\tightlist
\item[$\square$]
  All components specified
\item[$\square$]
  Material selected and justified
\item[$\square$]
  Assembly method documented
\item[$\square$]
  Tolerances verified
\item[$\square$]
  Manufacturing plan complete
\item[$\square$]
  Testing approach defined
\item[$\square$]
  Risks documented
\item[$\square$]
  Cost estimates calculated
\item[$\square$]
  Timeline realistic
\item[$\square$]
  Design approved by stakeholders
\end{itemize}

\subsubsection*{Design Change Log}\label{docs__pandoc__latex__src__3dmake_foundation__lessons_3dmake_11__design_specification_template.md__design-change-log}

{\def\LTcaptype{none} % do not increment counter
\begin{longtable}[]{@{}lllll@{}}
\toprule\noalign{}
Change \# & Date & Description & Reason & Impact \\
\midrule\noalign{}
\endhead
\bottomrule\noalign{}
\endlastfoot
& & & & \\
\end{longtable}
}

\subsubsection*{Sign-Off}\label{docs__pandoc__latex__src__3dmake_foundation__lessons_3dmake_11__design_specification_template.md__sign-off}

Design Approval:

{\def\LTcaptype{none} % do not increment counter
\begin{longtable}[]{@{}llll@{}}
\toprule\noalign{}
Role & Name & Signature & Date \\
\midrule\noalign{}
\endhead
\bottomrule\noalign{}
\endlastfoot
Designer & & & \\
Reviewer & & & \\
Stakeholder & & & \\
\end{longtable}
}

Status:

\begin{itemize}
\tightlist
\item[$\square$]
  Approved - Ready for manufacturing
\item[$\square$]
  Approved with conditions:
\item[$\square$]
  Not approved - Revisions needed:
\end{itemize}

\subsubsection*{Appendices}\label{docs__pandoc__latex__src__3dmake_foundation__lessons_3dmake_11__design_specification_template.md__appendices}

\begin{itemize}
\tightlist
\item
  Appendix A: Detailed CAD Models
\item
  Appendix B: Assembly Instructions
\item
  Appendix C: Material Data Sheets
\item
  Appendix D: Test Reports
\end{itemize}

\subsection{Feedback Collection Template}\label{docs__pandoc__latex__src__3dmake_foundation__lessons_3dmake_11__feedback_collection_template.md__3dmake_foundation_lessons_3dmake_11-feedback_collection_template}

Structured approach to gathering and analyzing feedback from users and stakeholders throughout the design process.

\subsubsection*{Feedback Collection Strategy}\label{docs__pandoc__latex__src__3dmake_foundation__lessons_3dmake_11__feedback_collection_template.md__feedback-collection-strategy}

\paragraph*{Collection Points}\label{docs__pandoc__latex__src__3dmake_foundation__lessons_3dmake_11__feedback_collection_template.md__collection-points}

\begin{itemize}
\tightlist
\item[$\square$]
  Prototype testing (early design)
\item[$\square$]
  Mid-development review
\item[$\square$]
  Beta testing (near completion)
\item[$\square$]
  Post-launch review
\item[$\square$]
  Long-term usage feedback
\end{itemize}

\paragraph*{Target Audience}\label{docs__pandoc__latex__src__3dmake_foundation__lessons_3dmake_11__feedback_collection_template.md__target-audience}

\begin{itemize}
\tightlist
\item
  Primary users:
\item
  Secondary users:
\item
  Stakeholders:
\end{itemize}

\subsubsection*{Prototype Feedback Form}\label{docs__pandoc__latex__src__3dmake_foundation__lessons_3dmake_11__feedback_collection_template.md__prototype-feedback-form}

\paragraph*{Feedback Session Information}\label{docs__pandoc__latex__src__3dmake_foundation__lessons_3dmake_11__feedback_collection_template.md__feedback-session-information}

\begin{itemize}
\tightlist
\item
  Feedback Date:
\item
  Tester Name:
\item
  Tester Role:
\item
  Prototype Version:
\item
  Session Duration:
\end{itemize}

\paragraph*{Section 1: First Impressions (5 minutes)}\label{docs__pandoc__latex__src__3dmake_foundation__lessons_3dmake_11__feedback_collection_template.md__section-1-first-impressions-5-minutes}

\paragraph*{Initial Reactions}\label{docs__pandoc__latex__src__3dmake_foundation__lessons_3dmake_11__feedback_collection_template.md__initial-reactions}

\begin{enumerate}
\item ~
  \subsubsection*{"What is your first impression of this design?"}\label{docs__pandoc__latex__src__3dmake_foundation__lessons_3dmake_11__feedback_collection_template.md__what-is-your-first-impression-of-this-design}
\item
  "Is it what you expected?"

  \begin{itemize}
  \tightlist
  \item[$\square$]
    More than expected  {[} {]} As expected  {[} {]} Less than expected
  \item
    Explanation:
  \end{itemize}
\item
  "Would you use this?"

  \begin{itemize}
  \tightlist
  \item[$\square$]
    Definitely  {[} {]} Maybe  {[} {]} No  {[} {]} Not sure
  \item
    Why?
  \end{itemize}
\end{enumerate}

\paragraph*{Section 2: Usability (15 minutes)}\label{docs__pandoc__latex__src__3dmake_foundation__lessons_3dmake_11__feedback_collection_template.md__section-2-usability-15-minutes}

\paragraph*{Using the Product}\label{docs__pandoc__latex__src__3dmake_foundation__lessons_3dmake_11__feedback_collection_template.md__using-the-product}

\begin{enumerate}
\item
  "Can you show me how you would use this?"

  \begin{itemize}
  \tightlist
  \item
    (Observe: difficulties, hesitations, confusion)
  \item
    Notes:
  \end{itemize}
\item
  "Was it easy to understand how to use?"

  \begin{itemize}
  \tightlist
  \item[$\square$]
    Very easy  {[} {]} Easy  {[} {]} Neutral  {[} {]} Difficult  {[} {]} Very difficult
  \item
    What was confusing?
  \end{itemize}
\item ~
  \subsubsection*{"Are there any parts that don\textquotesingle t work well together?"}\label{docs__pandoc__latex__src__3dmake_foundation__lessons_3dmake_11__feedback_collection_template.md__are-there-any-parts-that-dont-work-well-together}
\end{enumerate}

\paragraph*{Specific Usability Issues}\label{docs__pandoc__latex__src__3dmake_foundation__lessons_3dmake_11__feedback_collection_template.md__specific-usability-issues}

{\def\LTcaptype{none} % do not increment counter
\begin{longtable}[]{@{}lll@{}}
\toprule\noalign{}
Issue & Severity & Suggestion \\
\midrule\noalign{}
\endhead
\bottomrule\noalign{}
\endlastfoot
& High/Med/Low & \\
& High/Med/Low & \\
\end{longtable}
}

\paragraph*{Section 3: Functionality Assessment (10 minutes)}\label{docs__pandoc__latex__src__3dmake_foundation__lessons_3dmake_11__feedback_collection_template.md__section-3-functionality-assessment-10-minutes}

\paragraph*{Does it do what it\textquotesingle s supposed to?}\label{docs__pandoc__latex__src__3dmake_foundation__lessons_3dmake_11__feedback_collection_template.md__does-it-do-what-its-supposed-to}

{\def\LTcaptype{none} % do not increment counter
\begin{longtable}[]{@{}
  >{\raggedright\arraybackslash}p{(\linewidth - 8\tabcolsep) * \real{0.2432}}
  >{\raggedright\arraybackslash}p{(\linewidth - 8\tabcolsep) * \real{0.1622}}
  >{\raggedright\arraybackslash}p{(\linewidth - 8\tabcolsep) * \real{0.1351}}
  >{\raggedright\arraybackslash}p{(\linewidth - 8\tabcolsep) * \real{0.1892}}
  >{\raggedright\arraybackslash}p{(\linewidth - 8\tabcolsep) * \real{0.2703}}@{}}
\toprule\noalign{}
\begin{minipage}[b]{\linewidth}\raggedright
Feature/Function
\end{minipage} & \begin{minipage}[b]{\linewidth}\raggedright
Works Well
\end{minipage} & \begin{minipage}[b]{\linewidth}\raggedright
Works OK
\end{minipage} & \begin{minipage}[b]{\linewidth}\raggedright
Doesn\textquotesingle t Work
\end{minipage} & \begin{minipage}[b]{\linewidth}\raggedright
Needed But Missing
\end{minipage} \\
\midrule\noalign{}
\endhead
\bottomrule\noalign{}
\endlastfoot
& & & & - \\
& & & & - \\
& & & & - \\
\end{longtable}
}

Critical Functions:

\begin{itemize}
\tightlist
\item
  Most important for your use:
\item
  Missing capability:
\end{itemize}

\paragraph*{Section 4: Physical Design Assessment (10 minutes)}\label{docs__pandoc__latex__src__3dmake_foundation__lessons_3dmake_11__feedback_collection_template.md__section-4-physical-design-assessment-10-minutes}

\paragraph*{Appearance \& Feel}\label{docs__pandoc__latex__src__3dmake_foundation__lessons_3dmake_11__feedback_collection_template.md__appearance--feel}

\paragraph*{Visual Design}\label{docs__pandoc__latex__src__3dmake_foundation__lessons_3dmake_11__feedback_collection_template.md__visual-design}

\begin{itemize}
\tightlist
\item[$\square$]
  Appealing  {[} {]} Acceptable  {[} {]} Unappealing  {[} {]} Doesn\textquotesingle t matter
\item
  Suggestions:
\end{itemize}

\paragraph*{Size \& Comfort}\label{docs__pandoc__latex__src__3dmake_foundation__lessons_3dmake_11__feedback_collection_template.md__size--comfort}

\begin{itemize}
\tightlist
\item
  Too large / Just right / Too small (circle one)
\item
  Weight: Too heavy / Just right / Too light (circle one)
\item
  Comfort feedback:
\end{itemize}

Material \& Texture:

\begin{itemize}
\tightlist
\item
  Appropriate: {[} {]} Yes  {[} {]} No  {[} {]} Uncertain
\item
  Feel:
\item
  Durability concern:
\end{itemize}

Color/Finish:

\begin{itemize}
\tightlist
\item
  Appropriate: {[} {]} Yes  {[} {]} No
\item
  Preference:
\end{itemize}

\paragraph*{Section 5: Performance \& Reliability (10 minutes)}\label{docs__pandoc__latex__src__3dmake_foundation__lessons_3dmake_11__feedback_collection_template.md__section-5-performance--reliability-10-minutes}

During Testing:

\begin{enumerate}
\tightlist
\item
  "Did it work consistently?"

  \begin{itemize}
  \tightlist
  \item[$\square$]
    Always  {[} {]} Usually  {[} {]} Sometimes  {[} {]} Never
  \item
    Issues experienced:
  \end{itemize}
\item
  "How solid/durable does it feel?"

  \begin{itemize}
  \tightlist
  \item[$\square$]
    Very durable  {[} {]} Durable  {[} {]} Adequate  {[} {]} Fragile  {[} {]} Very fragile
  \item
    Concerns:
  \end{itemize}
\item
  "Any parts that seem weak?"
\end{enumerate}

\paragraph*{Section 6: Overall Satisfaction (5 minutes)}\label{docs__pandoc__latex__src__3dmake_foundation__lessons_3dmake_11__feedback_collection_template.md__section-6-overall-satisfaction-5-minutes}

Scoring Matrix:

{\def\LTcaptype{none} % do not increment counter
\begin{longtable}[]{@{}lll@{}}
\toprule\noalign{}
Criterion & Rating & Comments \\
\midrule\noalign{}
\endhead
\bottomrule\noalign{}
\endlastfoot
Overall satisfaction & 1 2 3 4 5 & \\
Likely to recommend & 1 2 3 4 5 & \\
Meets my needs & 1 2 3 4 5 & \\
Design quality & 1 2 3 4 5 & \\
Value for cost & 1 2 3 4 5 & \\
\end{longtable}
}

(1=Poor, 5=Excellent)

\paragraph*{Section 7: Suggestions \& Improvements (10 minutes)}\label{docs__pandoc__latex__src__3dmake_foundation__lessons_3dmake_11__feedback_collection_template.md__section-7-suggestions--improvements-10-minutes}

Open-Ended Feedback:

\begin{enumerate}
\item ~
  \subsubsection*{"What would make this better?"}\label{docs__pandoc__latex__src__3dmake_foundation__lessons_3dmake_11__feedback_collection_template.md__what-would-make-this-better}
\item ~
  \subsubsection*{"What should we change?"}\label{docs__pandoc__latex__src__3dmake_foundation__lessons_3dmake_11__feedback_collection_template.md__what-should-we-change}
\item ~
  \subsubsection*{"What should we keep as-is?"}\label{docs__pandoc__latex__src__3dmake_foundation__lessons_3dmake_11__feedback_collection_template.md__what-should-we-keep-as-is}
\item
  "Would you buy/use this?"

  \begin{itemize}
  \tightlist
  \item[$\square$]
    Yes  {[} {]} Maybe  {[} {]} No
  \item
    Why?
  \end{itemize}
\end{enumerate}

Suggestions Priority:

{\def\LTcaptype{none} % do not increment counter
\begin{longtable}[]{@{}llll@{}}
\toprule\noalign{}
Suggestion & Priority & Category & Effort \\
\midrule\noalign{}
\endhead
\bottomrule\noalign{}
\endlastfoot
& High/Med/Low & Design/Function/Material & Easy/Med/Hard \\
& High/Med/Low & Design/Function/Material & Easy/Med/Hard \\
\end{longtable}
}

\paragraph*{Section 8: Accessibility Check (5 minutes)}\label{docs__pandoc__latex__src__3dmake_foundation__lessons_3dmake_11__feedback_collection_template.md__section-8-accessibility-check-5-minutes}

If applicable to your product:

\begin{enumerate}
\item
  "Can someone with limited vision use this?"

  \begin{itemize}
  \tightlist
  \item[$\square$]
    Yes  {[} {]} With difficulty  {[} {]} No
  \item
    How:
  \end{itemize}
\item
  "Can someone with limited mobility use this?"

  \begin{itemize}
  \tightlist
  \item[$\square$]
    Yes  {[} {]} With difficulty  {[} {]} No
  \item
    How:
  \end{itemize}
\item ~
  \subsubsection*{"Are there any accessibility barriers?"}\label{docs__pandoc__latex__src__3dmake_foundation__lessons_3dmake_11__feedback_collection_template.md__are-there-any-accessibility-barriers}
\end{enumerate}

\subsubsection*{Post-Testing Notes}\label{docs__pandoc__latex__src__3dmake_foundation__lessons_3dmake_11__feedback_collection_template.md__post-testing-notes}

Tester Behavior Observations:
(What did you notice about how they used it?)

Surprises: (Anything unexpected?)

Key Insights:

\begin{enumerate}
\tightlist
\item
\item
\end{enumerate}

Critical Issues Found:

\begin{itemize}
\tightlist
\item
\item
\end{itemize}

Non-Critical Issues:

\begin{itemize}
\tightlist
\item
\end{itemize}

\subsubsection*{Feedback Analysis (After Multiple Sessions)}\label{docs__pandoc__latex__src__3dmake_foundation__lessons_3dmake_11__feedback_collection_template.md__feedback-analysis-after-multiple-sessions}

\paragraph*{Common Themes}\label{docs__pandoc__latex__src__3dmake_foundation__lessons_3dmake_11__feedback_collection_template.md__common-themes}

Frequently Mentioned Issues:

\begin{enumerate}
\tightlist
\item
  (Mentioned by of\% testers)
\item
  (Mentioned by of\%)
\item
  (Mentioned by of\%)
\end{enumerate}

Frequently Praised Aspects:

\begin{enumerate}
\tightlist
\item
\item
\end{enumerate}

\paragraph*{Data Summary}\label{docs__pandoc__latex__src__3dmake_foundation__lessons_3dmake_11__feedback_collection_template.md__data-summary}

{\def\LTcaptype{none} % do not increment counter
\begin{longtable}[]{@{}llll@{}}
\toprule\noalign{}
Metric & Average & Range & Status \\
\midrule\noalign{}
\endhead
\bottomrule\noalign{}
\endlastfoot
Overall Satisfaction & /5 & & \\
Likelihood to Recommend & /5 & & \\
Meets Needs & /5 & & \\
Usability Rating & /5 & & \\
\end{longtable}
}

\paragraph*{Priority Action Items}\label{docs__pandoc__latex__src__3dmake_foundation__lessons_3dmake_11__feedback_collection_template.md__priority-action-items}

Must Fix (High Priority - blocks usage):

\begin{enumerate}
\tightlist
\item
  \begin{itemize}
  \tightlist
  \item
    Impact if not fixed:
  \end{itemize}
\item
  \begin{itemize}
  \tightlist
  \item
    Impact if not fixed:
  \end{itemize}
\end{enumerate}

Should Fix (Medium Priority - improves experience):

\begin{enumerate}
\tightlist
\item
\item
\end{enumerate}

Nice to Have (Low Priority - enhancements):

\begin{enumerate}
\tightlist
\item
\item
\end{enumerate}

\paragraph*{Design Changes Resulting from Feedback}\label{docs__pandoc__latex__src__3dmake_foundation__lessons_3dmake_11__feedback_collection_template.md__design-changes-resulting-from-feedback}

{\def\LTcaptype{none} % do not increment counter
\begin{longtable}[]{@{}llll@{}}
\toprule\noalign{}
Original Design & Feedback & Change Made & Rationale \\
\midrule\noalign{}
\endhead
\bottomrule\noalign{}
\endlastfoot
& & & \\
& & & \\
\end{longtable}
}

\subsubsection*{Follow-Up Actions}\label{docs__pandoc__latex__src__3dmake_foundation__lessons_3dmake_11__feedback_collection_template.md__follow-up-actions}

Immediate Actions:

\begin{itemize}
\tightlist
\item[$\square$]
  Fix critical issues:
\item[$\square$]
  Notify stakeholders:
\item[$\square$]
  Update design document:
\end{itemize}

Next Test Cycle:

\begin{itemize}
\tightlist
\item[$\square$]
  Schedule follow-up testing
\item[$\square$]
  Prepare updated prototype
\item[$\square$]
  Revise feedback form if needed
\end{itemize}

Timeline for Changes:

\begin{itemize}
\tightlist
\item
  Critical fixes by:
\item
  Medium priority by:
\item
  Low priority by:
\end{itemize}

\subsubsection*{Stakeholder Communication}\label{docs__pandoc__latex__src__3dmake_foundation__lessons_3dmake_11__feedback_collection_template.md__stakeholder-communication}

Share findings with:

\begin{itemize}
\tightlist
\item[$\square$]
  Design team
\item[$\square$]
  Stakeholders
\item[$\square$]
  User community
\item[$\square$]
  Management
\end{itemize}

Format for sharing:

\begin{itemize}
\tightlist
\item[$\square$]
  Summary report
\item[$\square$]
  Presentation
\item[$\square$]
  Video highlights
\item[$\square$]
  Detailed documentation
\end{itemize}

\subsubsection*{Feedback Tracking}\label{docs__pandoc__latex__src__3dmake_foundation__lessons_3dmake_11__feedback_collection_template.md__feedback-tracking}

Overall Feedback Summary:

{\def\LTcaptype{none} % do not increment counter
\begin{longtable}[]{@{}lllll@{}}
\toprule\noalign{}
Test Round & Date & \# Testers & Key Finding & Action Taken \\
\midrule\noalign{}
\endhead
\bottomrule\noalign{}
\endlastfoot
Prototype V1 & & & & \\
Prototype V2 & & & & \\
Beta & & & & \\
\end{longtable}
}

\subsubsection*{Sign-Off}\label{docs__pandoc__latex__src__3dmake_foundation__lessons_3dmake_11__feedback_collection_template.md__sign-off}

Feedback Session Conducted By: Analysis Completed By: Reviewed By:

Status:

\begin{itemize}
\tightlist
\item[$\square$]
  Feedback incorporated into design
\item[$\square$]
  Partial incorporation (note: )
\item[$\square$]
  Under review (decision pending)
\item[$\square$]
  Declined (reason: )
\end{itemize}

Session Documentation Date: Feedback Database Location:

\subsection{Master Project Rubric - 0-9 Point Scale}\label{docs__pandoc__latex__src__3dmake_foundation__lessons_3dmake_11__master-rubric.md__3dmake_foundation_lessons_3dmake_11-master-rubric}

\emph{This rubric applies to all Unit 1, 2, and 3 projects unless otherwise noted in the project briefing. Project 0 is complete/incomplete only.}

\subsubsection*{Rubric Overview}\label{docs__pandoc__latex__src__3dmake_foundation__lessons_3dmake_11__master-rubric.md__rubric-overview}

Projects are scored from 0-9 across three categories worth 3 points each:

{\def\LTcaptype{none} % do not increment counter
\begin{longtable}[]{@{}
  >{\raggedright\arraybackslash}p{(\linewidth - 4\tabcolsep) * \real{0.1770}}
  >{\raggedright\arraybackslash}p{(\linewidth - 4\tabcolsep) * \real{0.1062}}
  >{\raggedright\arraybackslash}p{(\linewidth - 4\tabcolsep) * \real{0.7168}}@{}}
\toprule\noalign{}
\begin{minipage}[b]{\linewidth}\raggedright
Category
\end{minipage} & \begin{minipage}[b]{\linewidth}\raggedright
Max Points
\end{minipage} & \begin{minipage}[b]{\linewidth}\raggedright
What Is Evaluated
\end{minipage} \\
\midrule\noalign{}
\endhead
\bottomrule\noalign{}
\endlastfoot
Problem \& Solution & 3 &
How well the prototype meets the identified problem or functional requirements \\
Design Quality & 3 &
Complexity, originality, technical quality of the design; evidence of iteration \\
Documentation & 3 &
Completeness, accuracy, and thoughtfulness of all written deliverables \\
\end{longtable}
}

\subsubsection*{Category 1: Problem \& Solution (0-3 points)}\label{docs__pandoc__latex__src__3dmake_foundation__lessons_3dmake_11__master-rubric.md__category-1-problem--solution-0-3-points}

{\def\LTcaptype{none} % do not increment counter
\begin{longtable}[]{@{}
  >{\raggedright\arraybackslash}p{(\linewidth - 2\tabcolsep) * \real{0.0400}}
  >{\raggedright\arraybackslash}p{(\linewidth - 2\tabcolsep) * \real{0.9600}}@{}}
\toprule\noalign{}
\begin{minipage}[b]{\linewidth}\raggedright
Score
\end{minipage} & \begin{minipage}[b]{\linewidth}\raggedright
Description
\end{minipage} \\
\midrule\noalign{}
\endhead
\bottomrule\noalign{}
\endlastfoot
3 &
The prototype clearly and effectively solves the stated problem. All functional requirements are met. The solution shows evidence of testing against the requirements. \\
2 &
The prototype mostly meets the problem. Most functional requirements are met. Minor gaps between the design and the requirements. \\
1 &
The prototype partially addresses the problem. Several functional requirements are not met or were not clearly tested. \\
0 &
The prototype does not address the stated problem, or no functional requirements were established. \\
\end{longtable}
}

\subsubsection*{Category 2: Design Quality (0-3 points)}\label{docs__pandoc__latex__src__3dmake_foundation__lessons_3dmake_11__master-rubric.md__category-2-design-quality-0-3-points}

{\def\LTcaptype{none} % do not increment counter
\begin{longtable}[]{@{}
  >{\raggedright\arraybackslash}p{(\linewidth - 2\tabcolsep) * \real{0.0269}}
  >{\raggedright\arraybackslash}p{(\linewidth - 2\tabcolsep) * \real{0.9731}}@{}}
\toprule\noalign{}
\begin{minipage}[b]{\linewidth}\raggedright
Score
\end{minipage} & \begin{minipage}[b]{\linewidth}\raggedright
Description
\end{minipage} \\
\midrule\noalign{}
\endhead
\bottomrule\noalign{}
\endlastfoot
3 &
The design is original, well-considered, and technically executed. Code (if applicable) is clean, uses variables/modules appropriately, and is well-commented. The print is clean and well-made. Evidence of at least one meaningful revision or iteration. \\
2 &
The design is functional and shows thought. Code works but may lack structure (few comments, raw numbers instead of variables). Print quality is acceptable. Some iteration evident. \\
1 &
The design is basic or primarily borrowed from another source without modification. Code has issues. Print quality has significant defects that were not addressed. \\
0 &
No meaningful original design. Print is not functional or was not completed. \\
\end{longtable}
}

\subsubsection*{Category 3: Documentation (0-3 points)}\label{docs__pandoc__latex__src__3dmake_foundation__lessons_3dmake_11__master-rubric.md__category-3-documentation-0-3-points}

{\def\LTcaptype{none} % do not increment counter
\begin{longtable}[]{@{}
  >{\raggedright\arraybackslash}p{(\linewidth - 2\tabcolsep) * \real{0.0330}}
  >{\raggedright\arraybackslash}p{(\linewidth - 2\tabcolsep) * \real{0.9670}}@{}}
\toprule\noalign{}
\begin{minipage}[b]{\linewidth}\raggedright
Score
\end{minipage} & \begin{minipage}[b]{\linewidth}\raggedright
Description
\end{minipage} \\
\midrule\noalign{}
\endhead
\bottomrule\noalign{}
\endlastfoot
3 &
All required sections are present, complete, and specific. Reflections are thoughtful and reference specific decisions, problems encountered, and learning. Photos are included. Measurements are recorded. \\
2 &
Most required sections are present. Some sections are vague or missing detail. Reflections show some thought but are brief or generic. \\
1 &
Documentation is incomplete. Major sections are missing or consist of one-line responses. Reflections are minimal. \\
0 & Documentation is not submitted or is essentially empty. \\
\end{longtable}
}

\subsubsection*{Score Interpretation}\label{docs__pandoc__latex__src__3dmake_foundation__lessons_3dmake_11__master-rubric.md__score-interpretation}

{\def\LTcaptype{none} % do not increment counter
\begin{longtable}[]{@{}
  >{\raggedright\arraybackslash}p{(\linewidth - 4\tabcolsep) * \real{0.1226}}
  >{\raggedright\arraybackslash}p{(\linewidth - 4\tabcolsep) * \real{0.3491}}
  >{\raggedright\arraybackslash}p{(\linewidth - 4\tabcolsep) * \real{0.5283}}@{}}
\toprule\noalign{}
\begin{minipage}[b]{\linewidth}\raggedright
Total Score
\end{minipage} & \begin{minipage}[b]{\linewidth}\raggedright
Interpretation
\end{minipage} & \begin{minipage}[b]{\linewidth}\raggedright
Next Step
\end{minipage} \\
\midrule\noalign{}
\endhead
\bottomrule\noalign{}
\endlastfoot
8-9 & Excellent work & Move on to next project \\
6-7 & Good work with room for improvement &
Move on; instructor may suggest revisiting one element \\
4-5 & Meets basic expectations &
Resubmission of specific weak areas recommended \\
2-3 & Does not meet expectations & Resubmission required \\
0-1 & Missing major deliverables &
Meet with instructor; create a completion plan \\
\end{longtable}
}

\subsubsection*{Resubmission Policy}\label{docs__pandoc__latex__src__3dmake_foundation__lessons_3dmake_11__master-rubric.md__resubmission-policy}

Students may resubmit any project as many times as they need to improve their score. Resubmissions must include:

\begin{enumerate}
\tightlist
\item
  A one-paragraph explanation of what was changed and why
\end{enumerate}

The resubmission score replaces the original score.

\subsection{Project 3: Beaded Jewelry - OpenSCAD Design}\label{docs__pandoc__latex__src__3dmake_foundation__lessons_3dmake_11__beaded-jewelry.md__3dmake_foundation_lessons_3dmake_11-beaded-jewelry}

Accessibility: When including images or diagrams, add short alt-text and provide a comment-based walkthrough for any \texttt{.scad} examples so screen-reader users can follow the design steps.

Unit: 3 - Open-Ended Projects
Estimated Duration: 1 week (self-paced: plan milestones below)
Deliverables due: End of week (see milestones)

\subsubsection*{Project Brief}\label{docs__pandoc__latex__src__3dmake_foundation__lessons_3dmake_11__beaded-jewelry.md__project-brief}

Design and produce a wearable beaded jewelry piece that includes at least eight 3D-printed beads generated in OpenSCAD. Your design must use two distinct parametric bead modules and combine them into a completed, wearable piece (necklace, bracelet, or similar).

\paragraph*{Constraints (must follow)}\label{docs__pandoc__latex__src__3dmake_foundation__lessons_3dmake_11__beaded-jewelry.md__constraints-must-follow}

\begin{itemize}
\tightlist
\item
  Your prototype must include a 3D-printed component designed in OpenSCAD.
\item
  Code your project in a single \texttt{.scad} file that parameterizes bead shapes for repeatability.
\item
  Use at least two different bead shapes in the final assembly.
\item
  The final piece must be wearable and assembled - not just a set of loose beads.
\end{itemize}

\subsubsection*{Learning Objectives}\label{docs__pandoc__latex__src__3dmake_foundation__lessons_3dmake_11__beaded-jewelry.md__learning-objectives}

\begin{itemize}
\tightlist
\item
  Create parametric OpenSCAD modules for repeated geometry
\item
  Combine modular parts into a coherent assembled object
\item
  Document design decisions and printing notes for reproducibility
\item
  Evaluate designs against measurable functional requirements
\end{itemize}

\subsubsection*{StepbyStep Milestones}\label{docs__pandoc__latex__src__3dmake_foundation__lessons_3dmake_11__beaded-jewelry.md__stepbystep-milestones}

\begin{enumerate}
\tightlist
\item
  Project setup (Day 1)
\end{enumerate}

\begin{itemize}
\tightlist
\item
  Read this briefing and the Unit 3 lessons.
\item
  Create a folder for your project and initialize a short Design Notes document.
\end{itemize}

\begin{enumerate}
\setcounter{enumi}{1}
\tightlist
\item
  Bead module development (Days 1-2)
\end{enumerate}

\begin{itemize}
\tightlist
\item
  Implement \texttt{bead\_A(size,\ detail)} and \texttt{bead\_B(size,\ detail)} parametric modules.
\item
  Test-print a single bead from each module; record optimal print temperature/bed settings.
\end{itemize}

\begin{enumerate}
\setcounter{enumi}{2}
\tightlist
\item
  Assembly and iteration (Days 3-4)
\end{enumerate}

\begin{itemize}
\tightlist
\item
  Create an assembly script that arranges at least 8 beads and tests fit/tolerances.
\item
  Iterate bead hole diameter and test-strung spacing until beads slide but do not fall off.
\end{itemize}

\begin{enumerate}
\setcounter{enumi}{3}
\tightlist
\item
  Final prototype \& documentation (Day 5)
\end{enumerate}

\begin{itemize}
\tightlist
\item
  Print final beads, assemble the piece, photograph the result, and prepare the deliverables.
\end{itemize}

\subsubsection*{Deliverables}\label{docs__pandoc__latex__src__3dmake_foundation__lessons_3dmake_11__beaded-jewelry.md__deliverables}

Submit both digitally and physically as instructed:

\begin{itemize}
\tightlist
\item
  \texttt{.scad} file containing parametric bead modules and the assembly script
\item
  \texttt{.stl} files (exported as needed) for printed parts
\item
  Technical documentation (Google Drive link or similar) including:

  \begin{itemize}
  \tightlist
  \item
    Design Notes (ideas, measurements, param values)
  \item
    Construction / 3D printing notes (temperatures, speeds, supports)
  \item
    Photos of final prototype (multiple views)
  \end{itemize}
\item
  Physical turn-in: one assembled piece (if required by instructor)
\end{itemize}

\subsubsection*{Functional Requirements (examples - adapt to your design)}\label{docs__pandoc__latex__src__3dmake_foundation__lessons_3dmake_11__beaded-jewelry.md__functional-requirements-examples---adapt-to-your-design}

\begin{itemize}
\tightlist
\item
  The bead module must allow a hole diameter adjustable in 0.1 mm increments between 2.0-4.0 mm.
\item
  The assembly must include at least 8 beads and remain wearable (fits comfortably on intended body part).
\item
  The OpenSCAD file must be parameterized such that changing a single \texttt{scale} parameter adjusts bead size consistently.
\item
  The assembled piece must not have sharp edges that would injure skin under normal use.
\end{itemize}

\subsubsection*{Grading Rubric (simplified, 0-9 scale)}\label{docs__pandoc__latex__src__3dmake_foundation__lessons_3dmake_11__beaded-jewelry.md__grading-rubric-simplified-0-9-scale}

\begin{itemize}
\tightlist
\item
  Implementation \& parametric code: 3 points
\item
  Functionality \& wearability: 3 points
\item
  Documentation \& print notes: 2 points
\item
  Presentation \& photos: 1 point
\end{itemize}

\subsubsection*{Quiz - Project 3: Beaded Jewelry (10 questions)}\label{docs__pandoc__latex__src__3dmake_foundation__lessons_3dmake_11__beaded-jewelry.md__quiz---project-3-beaded-jewelry-10-questions}

\begin{enumerate}
\tightlist
\item
  Why use parametric modules for repeated parts? (one sentence)
\item
  What is a reliable way to test a bead hole for a given cord diameter?
\item
  Name two OpenSCAD functions or techniques useful for repeating geometry.
\item
  What printing setting is most likely to affect hole diameter accuracy?
\item
  How would you document an iteration that changed hole diameter from 2.5 mm to 2.7 mm?
\item
  True/False: Once your first bead prints successfully, all subsequent iterations will print exactly the same. (Answer: False - each print can vary due to temperature, humidity, and material batch differences)
\item
  Short answer: Explain what "wearability" means for a jewelry design. What are two specific checks you would perform to confirm a necklace is comfortable to wear?
\item
  Practical scenario: Your bead design uses a 3 mm hole diameter. When you string it on a cord, it\textquotesingle s too tight and won\textquotesingle t slide. What are two possible solutions you could implement in OpenSCAD to fix this?
\item
  Multiple choice: When documenting your design for reproducibility, what should you record? (A) Final bead dimensions only (B) All design iterations, parameter changes, and failed attempts (C) Just the successful version - Answer: B
\item
  Reflection: Describe how the iterative design process (design -\textgreater{} print -\textgreater{} test -\textgreater{} modify) applies specifically to creating a functional wearable. Why is rapid testing and documentation critical for jewelry design?
\end{enumerate}

Answer key (instructor use):

\begin{enumerate}
\tightlist
\item
  To allow repeatable, adjustable geometry and quick global changes.
\item
  Print a small tolerance test piece and measure fit; record results.
\item
  \texttt{for()} loops, \texttt{module} parameters, \texttt{translate()} and \texttt{rotate()}; \texttt{scale()}.
\item
  Nozzle diameter and extrusion multiplier; also bridging/cooling.
\item
  Note the parameter change, reproduce the print settings, measure and record fit, update Design Notes.
\end{enumerate}

\subsubsection*{Extension Problems (apply project skills)}\label{docs__pandoc__latex__src__3dmake_foundation__lessons_3dmake_11__beaded-jewelry.md__extension-problems-apply-project-skills}

\begin{enumerate}
\tightlist
\item
  Design an interlocking bead (snap-fit) and describe the tolerances required.
\item
  Create a parametric clasp module that integrates with your bead string and documents a pass/fail test.
\item
  Modify one bead to include decorative text or pattern generated procedurally in OpenSCAD.
\item
  Convert your single \texttt{.scad} project into a small library: separate bead modules into include files and demonstrate reuse.
\item
  Propose a modification to make the piece weather-resistant for outdoor wear (materials, coatings, or geometry).
\item
  Design and print a complete jewelry set: matching beads, clasp, and string connector with consistent design language.
\item
  Create a variant generator that produces 10+ different bead designs with single-parameter changes; document aesthetic and functional differences.
\item
  Build a design system document for your jewelry: modular bead library, material requirements, assembly instructions, and care guide.
\item
  Investigate material effects: print the same bead in 2+ materials; compare durability, aesthetic, and wearability.
\item
  Develop a parametric customization guide: enable users to modify size, spacing, color (if multi-material), and aesthetics through top-level variables.
\end{enumerate}

\subsubsection*{Submission Instructions}\label{docs__pandoc__latex__src__3dmake_foundation__lessons_3dmake_11__beaded-jewelry.md__submission-instructions}

Upload your digital deliverables to the course Drive folder and email the instructor with the link; bring the printed prototype to class on the due date.

\subsubsection*{Accessibility}\label{docs__pandoc__latex__src__3dmake_foundation__lessons_3dmake_11__beaded-jewelry.md__accessibility}

Provide alt-text for photos and a short written walkthrough of how your \texttt{.scad} file generates the bead shapes so screen-reader users can understand the sequence and parameters.

\subsubsection*{Notes}\label{docs__pandoc__latex__src__3dmake_foundation__lessons_3dmake_11__beaded-jewelry.md__notes}

The attached asset \texttt{project\_3\_briefing.txt} was used as the source. If you want me to also copy the original \texttt{.txt} into a \texttt{.bak} in the same folder, I can do that - tell me and I\textquotesingle ll create an adjacent \texttt{.bak} file.

\section{3dMake Foundation Final Exam}\label{docs__pandoc__latex__src__3dmake_foundation__3dmake_final_exam.md__3dmake_foundation-3dmake_final_exam}

Name:  \textbar{} Date:

Total Points: 100 (4 points per problem)

Instructions:

\begin{itemize}
\tightlist
\item
  Answer all 25 questions
\item
  For code errors, identify the specific problem and explain why it\textquotesingle s wrong
\item
  For behavioral questions, show your reasoning
\item
  For design problems, explain your approach and why it solves the stated challenge
\item
  You may reference the OpenSCAD documentation and appendices
\item
  All work must be your own
\end{itemize}

\subsection*{Section 1: Error Detection \& Code Analysis (Problems 1-10)}\label{docs__pandoc__latex__src__3dmake_foundation__3dmake_final_exam.md__section-1-error-detection--code-analysis-problems-1-10}

For each code block, identify any errors. If there is an error, explain what\textquotesingle s wrong and how to fix it. If the code is correct, state "No error" and explain what the code does.

\subsubsection*{Problem 1: Primitive Definition Error}\label{docs__pandoc__latex__src__3dmake_foundation__3dmake_final_exam.md__problem-1-primitive-definition-error}

\begin{lstlisting}[style=Alabaster, language=openscad]
cube([10, 10, 10], center=true);
sphere(r=5, $fn=16);
cylinder(h=20, r=8);

\end{lstlisting}

Question: Is there an error in this code? If yes, identify it. If no, explain what this code renders.

Answer:

\subsubsection*{Problem 2: Transform Syntax Error}\label{docs__pandoc__latex__src__3dmake_foundation__3dmake_final_exam.md__problem-2-transform-syntax-error}

\begin{lstlisting}[style=Alabaster, language=openscad]
translate([5, 5, 0])
  rotate([0, 45, 0])
    cube([10, 10, 10], center=true);

\end{lstlisting}

Question: Does this code have a syntax error? Explain what this code does.

Answer:

\subsubsection*{Problem 3: CSG Operation Error}\label{docs__pandoc__latex__src__3dmake_foundation__3dmake_final_exam.md__problem-3-csg-operation-error}

\begin{lstlisting}[style=Alabaster, language=openscad]
cube([20, 20, 20], center=true);
difference() {
  sphere(r=12);
  cylinder(h=30, r=5, center=true);
}

\end{lstlisting}

Question: What is wrong with this CSG operation? Explain the fix.

Answer:

\subsubsection*{Problem 4: Module Definition Error}\label{docs__pandoc__latex__src__3dmake_foundation__3dmake_final_exam.md__problem-4-module-definition-error}

\begin{lstlisting}[style=Alabaster, language=openscad]
module bracket(width, height, depth) {
  cube([width, height, depth], center=true);
  translate([width/2 + 2, 0, 0])
    cube([4, height, depth/2], center=true);
}
bracket(20, 15, 10);

\end{lstlisting}

Question: Is there an error in this module definition or call? Why or why not?

Answer:

\subsubsection*{Problem 5: Parameter \& Variable Scoping Error}\label{docs__pandoc__latex__src__3dmake_foundation__3dmake_final_exam.md__problem-5-parameter--variable-scoping-error}

\begin{lstlisting}[style=Alabaster, language=openscad]
wallthickness = 2;
module hollowcube(size) {
  difference() {
    cube([size, size, size], center=true);
    cube([size - wallthickness, size - wallthickness, size - wallthickness], center=true);
  }
}
hollowcube(20);

\end{lstlisting}

Question: Will this code work correctly? If not, what is the problem and how would you fix it?

Answer:

\subsubsection*{Problem 6: Loop \& Iteration Error}\label{docs__pandoc__latex__src__3dmake_foundation__3dmake_final_exam.md__problem-6-loop--iteration-error}

\begin{lstlisting}[style=Alabaster, language=openscad]
for (i = [0:5:20]) {
  translate([i, 0, 0])
    cube([4, 4, 4]);
}

\end{lstlisting}

Question: Will this code produce 5 cubes? Show the positions and explain why or why not.

Answer:

\subsubsection*{Problem 7: Center Parameter Misunderstanding}\label{docs__pandoc__latex__src__3dmake_foundation__3dmake_final_exam.md__problem-7-center-parameter-misunderstanding}

\begin{lstlisting}[style=Alabaster, language=openscad]
cube([10, 10, 20], center=false);
sphere(r=5);

\end{lstlisting}

Question: What is the relationship between these two shapes? Where would the sphere appear relative to the cube?

Answer:

\subsubsection*{Problem 8: Intersection Error}\label{docs__pandoc__latex__src__3dmake_foundation__3dmake_final_exam.md__problem-8-intersection-error}

\begin{lstlisting}[style=Alabaster, language=openscad]
intersection() {
  cube([20, 20, 20], center=true);
  sphere(r=8, $fn=32);
}

\end{lstlisting}

Question: Is there an error? What will this code render?

Answer:

\subsubsection*{Problem 9: Nested Transform Error}\label{docs__pandoc__latex__src__3dmake_foundation__3dmake_final_exam.md__problem-9-nested-transform-error}

\begin{lstlisting}[style=Alabaster, language=openscad]
translate([10, 0, 0])
  rotate([0, 0, 45])
    translate([5, 0, 0])
      cube([5, 5, 5], center=true);

\end{lstlisting}

Question: Are the transforms applied in the correct order? Trace the final position of the cube.

Answer:

\subsubsection*{Problem 10: Resolution Parameter Error}\label{docs__pandoc__latex__src__3dmake_foundation__3dmake_final_exam.md__problem-10-resolution-parameter-error}

\begin{lstlisting}[style=Alabaster, language=openscad]
sphere(r=10, $fn=4);
cylinder(h=20, r=8, $fn=3);
cube([10, 10, 10]);

\end{lstlisting}

Question: Identify the problem(s) with resolution in this code. What will happen when rendered?

Answer:

\subsection*{Section 2: Code Behavior \& Theory (Problems 11-17)}\label{docs__pandoc__latex__src__3dmake_foundation__3dmake_final_exam.md__section-2-code-behavior--theory-problems-11-17}

For each question, show your reasoning. You may draw diagrams if helpful.

\subsubsection*{Problem 11: Vertex Coordinates}\label{docs__pandoc__latex__src__3dmake_foundation__3dmake_final_exam.md__problem-11-vertex-coordinates}

Question: A cube is defined as \texttt{cube({[}10,\ 10,\ 20{]},\ center=false)}.

a) List the XYZ coordinates of all 8 vertices.

Answer:

\begin{itemize}
\tightlist
\item
  Vertex 1:
\item
  Vertex 2:
\item
  Vertex 3:
\item
  Vertex 4:
\item
  Vertex 5:
\item
  Vertex 6:
\item
  Vertex 7:
\item
  Vertex 8:
\end{itemize}

b) Now define the SAME cube with \texttt{center=true}. List the NEW coordinates of all 8 vertices.

Answer:

\begin{itemize}
\tightlist
\item
  Vertex 1:
\item
  Vertex 2:
\item
  Vertex 3:
\item
  Vertex 4:
\item
  Vertex 5:
\item
  Vertex 6:
\item
  Vertex 7:
\item
  Vertex 8:
\end{itemize}

\subsubsection*{Problem 12: Sphere Geometry}\label{docs__pandoc__latex__src__3dmake_foundation__3dmake_final_exam.md__problem-12-sphere-geometry}

Question: Explain the difference between \texttt{sphere(r=10,\ \$fn=8)} and \texttt{sphere(r=10,\ \$fn=128)}.

Which would you use for a prototype and which for final printing? Why?

Answer:

\subsubsection*{Problem 13: Transform Order}\label{docs__pandoc__latex__src__3dmake_foundation__3dmake_final_exam.md__problem-13-transform-order}

Question: Given this code:

\begin{lstlisting}[style=Alabaster, language=openscad]
translate([10, 0, 0])
  rotate([0, 0, 45])
    cube([5, 5, 5], center=true);

\end{lstlisting}

Does the order matter? What if you swap translate and rotate? Show both final positions.

Answer:

\subsubsection*{Problem 14: Boolean Operation Behavior}\label{docs__pandoc__latex__src__3dmake_foundation__3dmake_final_exam.md__problem-14-boolean-operation-behavior}

Question: You have a solid cube and you want to create a hole through it. Which CSG operation would you use: \texttt{union()}, \texttt{difference()}, or \texttt{intersection()}?

Explain your choice and write pseudocode showing how you\textquotesingle d accomplish this.

Answer:

\subsubsection*{Problem 15: Parametric Design Advantage}\label{docs__pandoc__latex__src__3dmake_foundation__3dmake_final_exam.md__problem-15-parametric-design-advantage}

Question: Compare these two approaches:

Approach A: Hard-coded cube with fixed dimensions

\begin{lstlisting}[style=Alabaster, language=openscad]
cube([10, 10, 20]);

\end{lstlisting}

Approach B: Parametric cube

\begin{lstlisting}[style=Alabaster, language=openscad]
module parametricbox(width, height, depth) {
  cube([width, height, depth], center=true);
}
parametricbox(10, 10, 20);

\end{lstlisting}

Why is Approach B better for design iteration? Give an example of how you\textquotesingle d use it.

Answer:

\subsubsection*{Problem 16: Scale Transform Behavior}\label{docs__pandoc__latex__src__3dmake_foundation__3dmake_final_exam.md__problem-16-scale-transform-behavior}

Question: If you apply \texttt{scale({[}2,\ 1,\ 0.5{]})} to a \texttt{cube({[}10,\ 10,\ 10{]},\ center=true)}, what are the NEW dimensions of the cube?

Answer: New dimensions:

Show your calculation:

\subsubsection*{Problem 17: Library Organization}\label{docs__pandoc__latex__src__3dmake_foundation__3dmake_final_exam.md__problem-17-library-organization}

Question: You\textquotesingle ve created three useful modules:

\begin{itemize}
\tightlist
\item
  \texttt{bracket(width,\ height,\ depth)}
\item
  \texttt{hollowcube(size,\ wallthickness)}
\item
  \texttt{connectorpin(diameter,\ height)}
\end{itemize}

How would you organize these into a reusable library? What file structure would you create and why?

Answer:

\subsection*{Section 3: Design \& Problem-Solving (Problems 18-25)}\label{docs__pandoc__latex__src__3dmake_foundation__3dmake_final_exam.md__section-3-design--problem-solving-problems-18-25}

These problems test your ability to design solutions, debug real-world issues, and think creatively.

\subsubsection*{Problem 18: Tolerance Design Challenge}\label{docs__pandoc__latex__src__3dmake_foundation__3dmake_final_exam.md__problem-18-tolerance-design-challenge}

Question: You\textquotesingle re designing a snap-fit connector. The male part has a thickness of 2mm. The female slot needs to accommodate this part with enough flexibility to snap but not fall out.

Should the slot be: a) Exactly 2mm wide b) 2.1mm wide c) 1.9mm wide
d) 2.5mm wide

Explain your choice and the design thinking behind it.

Answer:

\subsubsection*{Problem 19: Design Iteration Problem}\label{docs__pandoc__latex__src__3dmake_foundation__3dmake_final_exam.md__problem-19-design-iteration-problem}

Question: You print a keycap with \texttt{keysize=12} and the text embossing is too shallow to feel. Your code uses:

\begin{lstlisting}[style=Alabaster, language=openscad]
linearextrude(height=1)
  text("A", size=8);

\end{lstlisting}

What parameter(s) would you adjust to make the embossing deeper? Show your new code.

Answer:

\subsubsection*{Problem 20: Error Diagnosis}\label{docs__pandoc__latex__src__3dmake_foundation__3dmake_final_exam.md__problem-20-error-diagnosis}

Question: Your 3dMake build fails with this error: "Geometry is non-manifold." You have this code:

\begin{lstlisting}[style=Alabaster, language=openscad]
difference() {
  cube([20, 20, 20], center=true);
  cylinder(h=30, r=4, center=true);
}

\end{lstlisting}

Why might this fail? What\textquotesingle s the common fix for non-manifold geometry?

Answer:

\subsubsection*{Problem 21: Multi-Part Assembly}\label{docs__pandoc__latex__src__3dmake_foundation__3dmake_final_exam.md__problem-21-multi-part-assembly}

Question: You\textquotesingle re designing a two-part box (lid + base). The base has dimensions {[}50, 30, 20{]}. The lid should sit on top of the base.

Write parametric modules for both parts and show how you\textquotesingle d position them together. Include appropriate positioning logic.

Answer:

\begin{lstlisting}[style=Alabaster, language=openscad]
module base(length, width, height) {
  // Your code here
}
module lid(length, width, height) {
  // Your code here
}
// Positioning code here:

\end{lstlisting}

\subsubsection*{Problem 22: Optimization Challenge}\label{docs__pandoc__latex__src__3dmake_foundation__3dmake_final_exam.md__problem-22-optimization-challenge}

Question: You have a design that takes 5 minutes to render. You notice you have:

\begin{lstlisting}[style=Alabaster, language=openscad]
sphere(r=10, $fn=256);
cylinder(h=20, r=8, $fn=256);
cube([20, 20, 20]);

\end{lstlisting}

Which parameter(s) would you reduce to speed up rendering while maintaining acceptable quality for a prototype? Explain your choices.

Answer:

\subsubsection*{Problem 23: Real-World Constraint Problem}\label{docs__pandoc__latex__src__3dmake_foundation__3dmake_final_exam.md__problem-23-real-world-constraint-problem}

Question: A stakeholder requests a custom handle for a tool. They specify:

\begin{itemize}
\tightlist
\item
  Must fit a hand (approximately 80mm long)
\item
  Must accommodate fingers 60mm long inside
\item
  Wall thickness must be at least 3mm for durability
\item
  Should be ergonomic (slightly curved)
\end{itemize}

Sketch or describe a parametric design for this handle. What parameters would you expose to allow customization?

Answer:

\subsubsection*{Problem 24: Code Reusability Challenge}\label{docs__pandoc__latex__src__3dmake_foundation__3dmake_final_exam.md__problem-24-code-reusability-challenge}

Question: You\textquotesingle ve created a single keycap module. Now you need to create a keyboard with 5 keys arranged in a row. Keys are spaced 15mm apart.

Write code using a loop that creates 5 keycaps with letters A-E, properly spaced.

Answer:

\begin{lstlisting}[style=Alabaster, language=openscad]
module keycap(letter, keysize=10) {
  // Keycap code here (you can assume this exists)
}
// Your loop code here:

\end{lstlisting}

\subsubsection*{Problem 25: Design Thinking \& Iteration}\label{docs__pandoc__latex__src__3dmake_foundation__3dmake_final_exam.md__problem-25-design-thinking--iteration}

Question: You\textquotesingle ve printed Iteration 1 of a product and measured the results. The wall thickness is 3mm but feels too fragile. In Iteration 2, you increased it to 5mm, and now it feels too rigid and won\textquotesingle t flex as intended.

For Iteration 3, what thickness would you try and why? How would you make this decision more scientific/data-driven?

Answer:

\subsection*{Bonus Challenge (Optional, +5 points)}\label{docs__pandoc__latex__src__3dmake_foundation__3dmake_final_exam.md__bonus-challenge-optional-5-points}

Question: Design a parametric model for a custom assistive technology device (e.g., a tactile measuring tool, a custom gripper, an adapted eating utensil, etc.).

\begin{itemize}
\tightlist
\item
  Identify the user\textquotesingle s specific need
\item
  Specify the key dimensions and parameters
\item
  Write at least one module with realistic dimensions
\item
  Explain how the design would be tested and iterated
\end{itemize}

Answer:

\begin{lstlisting}[style=Alabaster, language=openscad]
// Your design here:

\end{lstlisting}

User Need:

Parameters:

Testing Plan:

\subsection*{Scoring Rubric}\label{docs__pandoc__latex__src__3dmake_foundation__3dmake_final_exam.md__scoring-rubric}

{\def\LTcaptype{none} % do not increment counter
\begin{longtable}[]{@{}
  >{\raggedright\arraybackslash}p{(\linewidth - 2\tabcolsep) * \real{0.1961}}
  >{\raggedright\arraybackslash}p{(\linewidth - 2\tabcolsep) * \real{0.8039}}@{}}
\toprule\noalign{}
\begin{minipage}[b]{\linewidth}\raggedright
Points per Problem
\end{minipage} & \begin{minipage}[b]{\linewidth}\raggedright
Criteria
\end{minipage} \\
\midrule\noalign{}
\endhead
\bottomrule\noalign{}
\endlastfoot
4 &
Correct answer with clear, complete explanation; demonstrates deep understanding \\
3 & Mostly correct answer; minor gaps in explanation or reasoning \\
2 &
Partially correct; shows some understanding but has significant gaps \\
1 & Minimal effort; shows limited understanding \\
0 & No answer or completely incorrect \\
\end{longtable}
}

Total Possible: 100 + 5 bonus = 105 points

\subsection*{References You May Use}\label{docs__pandoc__latex__src__3dmake_foundation__3dmake_final_exam.md__references-you-may-use}

\begin{itemize}
\tightlist
\item
  \hyperref[docs__pandoc__latex__src__3dmake_foundation__3dmake_quick_reference.md__3dmake_foundation-3dmake_quick_reference]{3dMake Quick Reference}
\item
  \hyperref[docs__pandoc__latex__src__3dmake_foundation__lessons_3dmake_2__openscad-cheat-sheet.md__3dmake_foundation_lessons_3dmake_2-openscad-cheat-sheet]{OpenSCAD Cheat Sheet}
\item
  \hyperref[docs__pandoc__latex__src__3dmake_foundation__appendix_a_comprehensive_slicing_guide.md__3dmake_foundation-appendix_a_comprehensive_slicing_guide]{Appendix A: Comprehensive Slicing Guide}
\item
  \hyperref[docs__pandoc__latex__src__3dmake_foundation__appendix_c_tolerance_qa.md__3dmake_foundation-appendix_c_tolerance_qa]{Appendix C: Tolerance Testing \& QA Matrix}
\item
  \href{https://github.com/mrhunsaker/VI_3DMake_OpenSCAD_Curriculum/3dMake_Foundation/Appendix_D_PowerShell_Integration.md}{Appendix D: PowerShell Integration}
\end{itemize}

End of Final Exam

Submission Instructions:

\begin{enumerate}
\tightlist
\item
  Answer all 25 questions completely
\item
  Show your work for calculations and reasoning
\item
  Include code samples where requested
\item
  Submit as a PDF with your name and date
\item
  Scoring will be based on correctness, clarity, and depth of understanding
\end{enumerate}

Good luck! {[}celebration{]}

\section{3dMake Foundation Quick Reference Guide}\label{docs__pandoc__latex__src__3dmake_foundation__3dmake_quick_reference.md__3dmake_foundation-3dmake_quick_reference}

Fast lookup for lessons, projects, resources, and common tasks

\subsection*{Lesson Quick Reference}\label{docs__pandoc__latex__src__3dmake_foundation__3dmake_quick_reference.md__lesson-quick-reference}

\subsubsection*{All 11 Lessons at a Glance}\label{docs__pandoc__latex__src__3dmake_foundation__3dmake_quick_reference.md__all-11-lessons-at-a-glance}

{\def\LTcaptype{none} % do not increment counter
\begin{longtable}[]{@{}
  >{\raggedright\arraybackslash}p{(\linewidth - 10\tabcolsep) * \real{0.0308}}
  >{\raggedright\arraybackslash}p{(\linewidth - 10\tabcolsep) * \real{0.2231}}
  >{\raggedright\arraybackslash}p{(\linewidth - 10\tabcolsep) * \real{0.0769}}
  >{\raggedright\arraybackslash}p{(\linewidth - 10\tabcolsep) * \real{0.1154}}
  >{\raggedright\arraybackslash}p{(\linewidth - 10\tabcolsep) * \real{0.3692}}
  >{\raggedright\arraybackslash}p{(\linewidth - 10\tabcolsep) * \real{0.1846}}@{}}
\toprule\noalign{}
\begin{minipage}[b]{\linewidth}\raggedright
\#
\end{minipage} & \begin{minipage}[b]{\linewidth}\raggedright
Title
\end{minipage} & \begin{minipage}[b]{\linewidth}\raggedright
Duration
\end{minipage} & \begin{minipage}[b]{\linewidth}\raggedright
Level
\end{minipage} & \begin{minipage}[b]{\linewidth}\raggedright
Main Topics
\end{minipage} & \begin{minipage}[b]{\linewidth}\raggedright
Key Project
\end{minipage} \\
\midrule\noalign{}
\endhead
\bottomrule\noalign{}
\endlastfoot
1 & Environmental Configuration & 60-90m & Beginner &
Setup, project structure, \texttt{3dm\ build} & None \\
2 & Geometric Primitives \& CSG & 60m & Beginner &
Primitives, CSG operations, debugging & None \\
3 & Parametric Architecture & 60m & Beginner+ &
Modules, libraries, parameters & None \\
4 & AI Verification & 45-60m & Intermediate &
\texttt{3dm\ info}, validation, design documentation & None \\
5 & Safety \& Physical Interface & 60-90m & Intermediate &
Safety protocols, materials, pre-print checks & None \\
6 & 3dm Commands \& Text & 60-90m & Intermediate &
\texttt{3dm\ describe/preview/orient/slice}, embossing & Keycap \\
7 & Parametric Transforms & 75-90m & Intermediate+ &
Transforms, multi-part design, assembly & Phone Stand \\
8 & Advanced Parametric Design & 90-120m & Advanced &
Tolerance, interlocking features, snap-fits & Stackable Bins \\
9 & Automation \& Workflows & 60-90m & Advanced &
PowerShell scripting, batch processing, CI/CD &
{[}key{]} Batch Automation \\
10 & Troubleshooting \& Mastery & 120-150m & Advanced &
Measurement, QA testing, diagnostics & {[}dice{]} QA +  Audit \\
11 & Stakeholder-Centric Design & 90-120m & Advanced+ &
Design thinking, user research, iteration &
{[}beads{]} Jewelry Holder \\
\end{longtable}
}

\subsection*{4 Reference Appendices}\label{docs__pandoc__latex__src__3dmake_foundation__3dmake_quick_reference.md__4-reference-appendices}

Quick links to comprehensive reference materials:

{\def\LTcaptype{none} % do not increment counter
\begin{longtable}[]{@{}
  >{\raggedright\arraybackslash}p{(\linewidth - 6\tabcolsep) * \real{0.5307}}
  >{\raggedright\arraybackslash}p{(\linewidth - 6\tabcolsep) * \real{0.1341}}
  >{\raggedright\arraybackslash}p{(\linewidth - 6\tabcolsep) * \real{0.0782}}
  >{\raggedright\arraybackslash}p{(\linewidth - 6\tabcolsep) * \real{0.2570}}@{}}
\toprule\noalign{}
\begin{minipage}[b]{\linewidth}\raggedright
Appendix
\end{minipage} & \begin{minipage}[b]{\linewidth}\raggedright
Focus
\end{minipage} & \begin{minipage}[b]{\linewidth}\raggedright
Size
\end{minipage} & \begin{minipage}[b]{\linewidth}\raggedright
Use When
\end{minipage} \\
\midrule\noalign{}
\endhead
\bottomrule\noalign{}
\endlastfoot
\hyperref[docs__pandoc__latex__src__3dmake_foundation__appendix_a_comprehensive_slicing_guide.md__3dmake_foundation-appendix_a_comprehensive_slicing_guide]{A: Comprehensive Slicing Guide}
& PrusaSlicer, Bambu Studio, Cura, OrcaSlicer configuration &
1,500+ lines & Slicing questions, slicer reference \\
\hyperref[docs__pandoc__latex__src__3dmake_foundation__appendix_b_material_properties.md__3dmake_foundation-appendix_b_material_properties]{B: Material Properties \& Selection Guide}
& Shrinkage data, print settings, material properties & 1,200+ lines &
Choosing material, troubleshooting prints \\
\hyperref[docs__pandoc__latex__src__3dmake_foundation__appendix_c_tolerance_qa.md__3dmake_foundation-appendix_c_tolerance_qa]{C: Tolerance Testing \& Quality Assurance Matrix}
& QA procedures, tolerance validation methods & 1,200+ lines &
Quality verification, measurement techniques \\
\href{https://github.com/mrhunsaker/VI_3DMake_OpenSCAD_Curriculum/3dMake_Foundation/Appendix_D_PowerShell_Integration.md}{D: PowerShell Integration for SCAD Workflows}
& Batch processing, automation scripts, workflow integration &
1,100+ lines & Building automation, batch processing \\
\end{longtable}
}

\subsection*{Learning Paths}\label{docs__pandoc__latex__src__3dmake_foundation__3dmake_quick_reference.md__learning-paths}

\subsubsection*{Path 1: Complete Mastery (18-22 hours)}\label{docs__pandoc__latex__src__3dmake_foundation__3dmake_quick_reference.md__path-1-complete-mastery-18-22-hours}

-\textgreater{} Lessons 1-11 + All Appendices

Best for: Complete skill development, comprehensive understanding

\subsubsection*{Path 2: Design Focus (12-15 hours)}\label{docs__pandoc__latex__src__3dmake_foundation__3dmake_quick_reference.md__path-2-design-focus-12-15-hours}

-\textgreater{} Lessons 1-3, 6-8, 11 + Appendices A, B, C

Best for: Experienced makers new to programmatic CAD

\subsubsection*{Path 3: Project-Based (14-18 hours)}\label{docs__pandoc__latex__src__3dmake_foundation__3dmake_quick_reference.md__path-3-project-based-14-18-hours}

-\textgreater{} Lessons 1-5 (Foundations) -\textgreater{} 6 (Keycap) -\textgreater{} 7 (Stand) -\textgreater{} 8 (Bins) -\textgreater{} 9 (Automation) -\textgreater{} 10 (Troubleshooting) -\textgreater{} 11 (Leadership)

Best for: Learning through building

\subsubsection*{Path 4: Safety \& Printing (10-12 hours)}\label{docs__pandoc__latex__src__3dmake_foundation__3dmake_quick_reference.md__path-4-safety--printing-10-12-hours}

-\textgreater{} Lessons 1, 2, 5, 6, 10 + Appendices A, B, C

Best for: Focus on practical printing and quality

\subsection*{3dm Command Reference}\label{docs__pandoc__latex__src__3dmake_foundation__3dmake_quick_reference.md__3dm-command-reference}

\subsubsection*{Essential Commands}\label{docs__pandoc__latex__src__3dmake_foundation__3dmake_quick_reference.md__essential-commands}

\begin{lstlisting}[style=Alabaster, language=bash]
# Setup
./3dm setup                 # Initial configuration

# Development
3dm edit-model file.scad    # Open in editor
3dm build src/main.scad     # Generate STL from SCAD

# Inspection
3dm describe file.scad      # Text analysis (AI if configured)
3dm preview file.scad       # Generate 2D tactile preview

# Optimization
3dm orient file.scad        # Suggest print orientation

# Production
3dm slice file.scad         # Generate G-code
3dm send build/main.gcode   # Send to printer

# Libraries
3dm lib list                # Show available libraries
3dm lib install BOSL2       # Install a library

\end{lstlisting}

\subsubsection*{Command Chaining}\label{docs__pandoc__latex__src__3dmake_foundation__3dmake_quick_reference.md__command-chaining}

\begin{lstlisting}[style=Alabaster, language=bash]
# Sequential with error handling
3dm build src/main.scad && 3dm slice src/main.scad && echo "Ready to print"

# Loop through files
for f in src/*.scad; do 3dm build "$f" && 3dm slice "$f"; done

\end{lstlisting}

\subsection*{OpenSCAD Quick Reference}\label{docs__pandoc__latex__src__3dmake_foundation__3dmake_quick_reference.md__openscad-quick-reference}

\subsubsection*{Primitives}\label{docs__pandoc__latex__src__3dmake_foundation__3dmake_quick_reference.md__primitives}

\begin{lstlisting}[style=Alabaster, language=openscad]
cube([width, height, depth], center=false);
sphere(r=radius, $fn=32);
cylinder(r=radius, h=height, $fn=32);

\end{lstlisting}

\subsubsection*{Transforms}\label{docs__pandoc__latex__src__3dmake_foundation__3dmake_quick_reference.md__transforms}

\begin{lstlisting}[style=Alabaster, language=openscad]
translate([x, y, z]) { ... }
rotate([x_deg, y_deg, z_deg]) { ... }
scale([x, y, z]) { ... }

\end{lstlisting}

\subsubsection*{Boolean Operations}\label{docs__pandoc__latex__src__3dmake_foundation__3dmake_quick_reference.md__boolean-operations}

\begin{lstlisting}[style=Alabaster, language=openscad]
union() { shape1; shape2; }         // Combine
difference() { shape1; shape2; }    // Subtract
intersection() { shape1; shape2; }  // Keep overlap

\end{lstlisting}

\subsubsection*{Modules}\label{docs__pandoc__latex__src__3dmake_foundation__3dmake_quick_reference.md__modules}

\begin{lstlisting}[style=Alabaster, language=openscad]
module my_shape(size) {
  cube([size, size, size]);
}
my_shape(20);   // Call module

\end{lstlisting}

\subsubsection*{Parameters}\label{docs__pandoc__latex__src__3dmake_foundation__3dmake_quick_reference.md__parameters}

\begin{lstlisting}[style=Alabaster, language=openscad]
width = 50;     // mm
height = 30;    // mm
inner = width - 2*wall;

\end{lstlisting}

\subsection*{Projects Reference}\label{docs__pandoc__latex__src__3dmake_foundation__3dmake_quick_reference.md__projects-reference}

\subsubsection*{Project 1: Parametric Keycap (Lesson 6)}\label{docs__pandoc__latex__src__3dmake_foundation__3dmake_quick_reference.md__project-1-parametric-keycap-lesson-6}

Key Parameters:

\begin{lstlisting}[style=Alabaster, language=openscad]
key_size = 18;      // mm
key_height = 12;    // mm
wall = 1.2;         // mm
letter = "A";       // Character

\end{lstlisting}

Variants to Try:

\begin{itemize}
\tightlist
\item
  Small: 12mm, 10mm height
\item
  Medium: 18mm, 12mm height
\item
  Large: 24mm, 14mm height
\end{itemize}

Files:

\begin{itemize}
\tightlist
\item
  Code: Lesson 6 (Keycap section)
\item
  Output: keycap\_X.scad, keycap\_X.stl
\end{itemize}

\subsubsection*{Project 2: Phone Stand (Lesson 7)}\label{docs__pandoc__latex__src__3dmake_foundation__3dmake_quick_reference.md__project-2-phone-stand-lesson-7}

Key Parameters:

\begin{lstlisting}[style=Alabaster, language=openscad]
phone_width = 75;   // mm
base_width = 85;    // mm
angle = 60;         // degrees
lip_height = 15;    // mm

\end{lstlisting}

Configurations:

{\def\LTcaptype{none} % do not increment counter
\begin{longtable}[]{@{}llll@{}}
\toprule\noalign{}
Phone & Width & Angle & Result \\
\midrule\noalign{}
\endhead
\bottomrule\noalign{}
\endlastfoot
iPhone & 60mm & 60 & Portrait viewing \\
iPad & 100mm & 40 & Landscape viewing \\
Tablet & 150mm & 35 & Document viewing \\
\end{longtable}
}

Files:

\begin{itemize}
\tightlist
\item
  Code: Lesson 7 (Phone Stand section)
\item
  Output: stand\_X.stl, stand\_X.gcode
\end{itemize}

\subsubsection*{Project 3: Stackable Bins (Lesson 8)}\label{docs__pandoc__latex__src__3dmake_foundation__3dmake_quick_reference.md__project-3-stackable-bins-lesson-8}

Key Parameters:

\begin{lstlisting}[style=Alabaster, language=openscad]
bin_w = 80;         // width (mm)
bin_d = 120;        // depth (mm)
bin_h = 60;         // height (mm)
wall = 2;           // thickness (mm)
stack_clear = 0.6;  // tolerance (mm) - CRITICAL

\end{lstlisting}

Tolerance Testing:

\begin{lstlisting}[style=Alabaster]
stack_clear = 0.4mm  -> Too tight (hard to stack)
stack_clear = 0.6mm  -> Ideal (smooth fit)
stack_clear = 0.8mm  -> Too loose (unstable)

\end{lstlisting}

Files:

\begin{itemize}
\tightlist
\item
  Code: Lesson 8 (Bins section)
\item
  Output: bin\_*.stl, tolerance\_matrix.md
\end{itemize}

\subsection*{Code Template Library}\label{docs__pandoc__latex__src__3dmake_foundation__3dmake_quick_reference.md__code-template-library}

\subsubsection*{Generic Parametric Part Template}\label{docs__pandoc__latex__src__3dmake_foundation__3dmake_quick_reference.md__generic-parametric-part-template}

\begin{lstlisting}[style=Alabaster, language=openscad]
// ====== PARAMETERS (customize here) ======
param1 = 50;        // mm
param2 = 30;        // mm
param3 = 5;         // mm
$fn = 32;           // Resolution (lower = faster)
// ====== CALCULATED PARAMETERS ======
derived_param = param1 - 2*param3;
// ====== MODULES ======
module my_part() {
  cube([param1, param2, param3]);
}
// ====== MAIN ======
my_part();

\end{lstlisting}

\subsubsection*{Hollow Box Template}\label{docs__pandoc__latex__src__3dmake_foundation__3dmake_quick_reference.md__hollow-box-template}

\begin{lstlisting}[style=Alabaster, language=openscad]
outer_size = 50;
inner_size = 40;
wall = 5;
difference() {
  cube([outer_size, outer_size, outer_size]);
  translate([wall, wall, wall])
    cube([inner_size, inner_size, inner_size]);
}

\end{lstlisting}

\subsubsection*{Batch Build Script Template}\label{docs__pandoc__latex__src__3dmake_foundation__3dmake_quick_reference.md__batch-build-script-template}

\begin{lstlisting}[style=Alabaster, language=bash]
#!/bin/bash
# batch_build.sh

for scad in src/*.scad; do
  name=$(basename "$scad" .scad)
  echo "Building: $name"
  3dm build "$scad" || continue
  cp "build/main.stl" "build/${name}.stl"
done

\end{lstlisting}

\subsection*{Troubleshooting Quick Fixes}\label{docs__pandoc__latex__src__3dmake_foundation__3dmake_quick_reference.md__troubleshooting-quick-fixes}

\subsubsection*{Problem: Model won\textquotesingle t build}\label{docs__pandoc__latex__src__3dmake_foundation__3dmake_quick_reference.md__problem-model-wont-build}

Diagnosis:

\begin{lstlisting}[style=Alabaster, language=bash]
3dm describe file.scad
# Look for error messages

\end{lstlisting}

Common Fixes:

\begin{itemize}
\tightlist
\item
  Check syntax (missing semicolons, parentheses)
\item
  Look for non-manifold geometry
\item
  Use \texttt{\$fn=12} for faster test renders
\end{itemize}

\subsubsection*{Problem: Parts don\textquotesingle t fit together}\label{docs__pandoc__latex__src__3dmake_foundation__3dmake_quick_reference.md__problem-parts-dont-fit-together}

Diagnosis:

\begin{itemize}
\tightlist
\item
  Print and test fit
\item
  Measure with calipers
\end{itemize}

Solution:

\begin{itemize}
\tightlist
\item
  Adjust \texttt{stack\_clear} (smaller = tighter)
\item
  Increase \texttt{wall} thickness
\item
  Test with tolerance matrix
\end{itemize}

\subsubsection*{Problem: Embossed text looks bad}\label{docs__pandoc__latex__src__3dmake_foundation__3dmake_quick_reference.md__problem-embossed-text-looks-bad}

Diagnosis:

\begin{itemize}
\tightlist
\item
  Check preview in slicer
\item
  Use \texttt{3dm\ preview} for tactile version
\end{itemize}

Solution:

\begin{itemize}
\tightlist
\item
  Increase \texttt{letter\_raise} (deeper emboss)
\item
  Use larger \texttt{\$fn} in \texttt{text()}
\item
  Simplify character or use different size
\end{itemize}

\subsubsection*{Problem: Print fails}\label{docs__pandoc__latex__src__3dmake_foundation__3dmake_quick_reference.md__problem-print-fails}

Diagnosis:

\begin{itemize}
\tightlist
\item
  Check slicer layer preview
\item
  Verify bed adhesion and temperature
\end{itemize}

Solution:

\begin{itemize}
\tightlist
\item
  Check pre-print checklist (Lesson 5)
\item
  Adjust print temperature
\item
  Verify bed is level and clean
\end{itemize}

\subsection*{Assessment Checklist}\label{docs__pandoc__latex__src__3dmake_foundation__3dmake_quick_reference.md__assessment-checklist}

\subsubsection*{Lesson Completion Criteria}\label{docs__pandoc__latex__src__3dmake_foundation__3dmake_quick_reference.md__lesson-completion-criteria}

\begin{itemize}
\tightlist
\item[$\square$]
  Watched/read entire lesson
\item[$\square$]
  Completed all step-by-step tasks
\item[$\square$]
  Reached all checkpoints
\item[$\square$]
  Answered all quiz questions (self-assessed)
\item[$\square$]
  Attempted at least 3 extension problems
\item[$\square$]
  Documented findings
\end{itemize}

\subsubsection*{Project Completion Criteria}\label{docs__pandoc__latex__src__3dmake_foundation__3dmake_quick_reference.md__project-completion-criteria}

\begin{itemize}
\tightlist
\item[$\square$]
  Code builds without errors
\item[$\square$]
  All parameters functional
\item[$\square$]
  STL generated and inspected
\item[$\square$]
  Measurements documented
\item[$\square$]
  Assembly tested (if multi-part)
\item[$\square$]
  README or documentation included
\end{itemize}

\subsubsection*{Quality Standards}\label{docs__pandoc__latex__src__3dmake_foundation__3dmake_quick_reference.md__quality-standards}

\begin{itemize}
\tightlist
\item[$\square$]
  Code is well-commented
\item[$\square$]
  Parameters have clear names and units
\item[$\square$]
  Modules are reusable
\item[$\square$]
  Design follows DRY principle
\item[$\square$]
  Documentation is complete
\end{itemize}

\subsection*{Resources \& Links}\label{docs__pandoc__latex__src__3dmake_foundation__3dmake_quick_reference.md__resources--links}

\subsubsection*{Official Docs}\label{docs__pandoc__latex__src__3dmake_foundation__3dmake_quick_reference.md__official-docs}

\begin{itemize}
\tightlist
\item
  \href{https://en.wikibooks.org/wiki/OpenSCAD_User_Manual}{OpenSCAD Manual}
\item
  \href{https://github.com/tdeck/3dmake}{3DMake GitHub}
\item
  \href{https://github.com/revarbat/BOSL2}{BOSL2 Library}
\end{itemize}

\subsubsection*{Tutorials}\label{docs__pandoc__latex__src__3dmake_foundation__3dmake_quick_reference.md__tutorials}

\begin{itemize}
\tightlist
\item
  \href{https://en.wikibooks.org/wiki/OpenSCAD_User_Manual/The_OpenSCAD_Language}{OpenSCAD Basics}
\item
  \href{https://www.prusa3d.com/support/}{3D Printing Guide}
\end{itemize}

\subsubsection*{Community}\label{docs__pandoc__latex__src__3dmake_foundation__3dmake_quick_reference.md__community}

\begin{itemize}
\tightlist
\item
  \href{https://forum.openscad.org/}{OpenSCAD Forums}
\item
  \href{https://www.reddit.com/r/3Dprinting/}{r/3Dprinting}
\end{itemize}

\subsection*{Tips \& Tricks}\label{docs__pandoc__latex__src__3dmake_foundation__3dmake_quick_reference.md__tips--tricks}

\subsubsection*{Debugging}\label{docs__pandoc__latex__src__3dmake_foundation__3dmake_quick_reference.md__debugging}

\begin{itemize}
\tightlist
\item
  Lower \texttt{\$fn} to 8-12 for faster renders during development
\item
  Use \texttt{3dm\ describe} frequently to catch issues early
\item
  Test components individually before assembling
\item
  Generate \texttt{3dm\ preview} for 2D tactile verification
\end{itemize}

\subsubsection*{Design}\label{docs__pandoc__latex__src__3dmake_foundation__3dmake_quick_reference.md__design}

\begin{itemize}
\tightlist
\item
  Keep parameters at the top of file for easy modification
\item
  Use descriptive names (not \texttt{w}, use \texttt{width})
\item
  Include units in comments
\item
  Document parameter ranges (e.g., \texttt{//\ 0-100\ mm})
\end{itemize}

\subsubsection*{Organization}\label{docs__pandoc__latex__src__3dmake_foundation__3dmake_quick_reference.md__organization}

\begin{itemize}
\tightlist
\item
  Use \texttt{src/} for SCAD files, \texttt{lib/} for modules, \texttt{build/} for outputs
\item
  Create variants by copying files and renaming
\item
  Use bash scripts for batch operations
\item
  Archive successful builds with timestamps
\end{itemize}

\subsubsection*{Accessibility}\label{docs__pandoc__latex__src__3dmake_foundation__3dmake_quick_reference.md__accessibility}

\begin{itemize}
\tightlist
\item
  Always use \texttt{3dm\ describe} to verify non-visual usability
\item
  Generate \texttt{3dm\ preview} for tactile inspection
\item
  Document measurements clearly
\item
  Test assembly without visual guidance
\end{itemize}

\subsection*{Glossary}\label{docs__pandoc__latex__src__3dmake_foundation__3dmake_quick_reference.md__glossary}

{\def\LTcaptype{none} % do not increment counter
\begin{longtable}[]{@{}
  >{\raggedright\arraybackslash}p{(\linewidth - 2\tabcolsep) * \real{0.1446}}
  >{\raggedright\arraybackslash}p{(\linewidth - 2\tabcolsep) * \real{0.8554}}@{}}
\toprule\noalign{}
\begin{minipage}[b]{\linewidth}\raggedright
Term
\end{minipage} & \begin{minipage}[b]{\linewidth}\raggedright
Definition
\end{minipage} \\
\midrule\noalign{}
\endhead
\bottomrule\noalign{}
\endlastfoot
CSG &
Constructive Solid Geometry - combining shapes using union/difference \\
Manifold & Water-tight geometry with clear inside/outside \\
Parametric & Driven by variables; changing parameters updates design \\
Tolerance & Acceptable variation in dimensions \\
Stack-up & Cumulative error from multiple tolerances \\
Module & Reusable code block in OpenSCAD \\
\$fn & Resolution parameter (higher = more detail but slower) \\
G-code & Machine instructions for 3D printer \\
STL & 3D model file format for printing \\
\end{longtable}
}

\subsection*{Quick Answers}\label{docs__pandoc__latex__src__3dmake_foundation__3dmake_quick_reference.md__quick-answers}

Q: Where do I put my SCAD files?\\
A: In the \texttt{src/} folder of your 3dMake project

Q: How do I test if my design will fit?\\
A: Use the tolerance testing matrix; print variants with different parameters

Q: What should I measure after printing?\\
A: Critical dimensions and compare to design specifications

Q: How do I fix non-manifold geometry?\\
A: Use the 0.001 offset rule: \texttt{translate({[}0,\ 0,\ 0.001{]})} before subtracting

Q: Can I combine multiple SCAD files?\\
A: Yes, use \texttt{include\ \textless{}path/to/file.scad\textgreater{}} or \texttt{use\ \textless{}path/to/file.scad\textgreater{}}

Q: How do I make designs accessible?\\
A: Use \texttt{3dm\ describe} and \texttt{3dm\ preview} and include written measurements/descriptions

\chapter{BackMatter}\label{docs__pandoc__latex__src__backmatter__backmatter.md__backmatter}

This section contains supplementary material and appendices that support the main book.

\begin{itemize}
\item
  \textbf{Command Line Appendices} --- guidance and integration notes for using the command line with SCAD workflows.

  \begin{itemize}
  \tightlist
  \item
    \hyperref[docs__pandoc__latex__src__command_line_interface_selection__appendix_a_commandline_integration.md__3dmake_foundation-appendix_f_cmd_integration]{Appendix A: CMD Integration for SCAD Workflows}
  \item
    \hyperref[docs__pandoc__latex__src__command_line_interface_selection__appendix_b_powershell_integration.md__3dmake_foundation-appendix_d_powershell_integration]{Appendix B: PowerShell Integration for SCAD Workflows}
  \item
    \hyperref[docs__pandoc__latex__src__command_line_interface_selection__appendix_c_gitbash_integration.md__3dmake_foundation-appendix_g_gitbash_integration]{Appendix C: PowerShell Integration for SCAD Workflows}
  \end{itemize}
\item
  \textbf{3D Make and OpenSCAD Appendices} --- deeper references for slicing, materials, tolerances and advanced OpenSCAD topics.

  \begin{itemize}
  \tightlist
  \item
    \hyperref[docs__pandoc__latex__src__3dmake_foundation__appendix_a_comprehensive_slicing_guide.md__3dmake_foundation-appendix_a_comprehensive_slicing_guide]{Appendix A: Comprehensive Slicing Guide - All Major Slicers}
  \item
    \hyperref[docs__pandoc__latex__src__3dmake_foundation__appendix_b_material_properties.md__3dmake_foundation-appendix_b_material_properties]{Appendix B: Material Properties \& Selection Guide}
  \item
    \hyperref[docs__pandoc__latex__src__3dmake_foundation__appendix_c_tolerance_qa.md__3dmake_foundation-appendix_c_tolerance_qa]{Appendix C: Tolerance Testing \& Quality Assurance Matrix}
  \item
    \hyperref[docs__pandoc__latex__src__3dmake_foundation__appendix_d_advanced_openscad_concepts.md__appendix-d-advanced-openscad-concepts]{Appendix D: Advanced OpenSCAD Concepts}
  \end{itemize}
\item
  ** Student Glossary** --- a concise list of terms and definitions used throughout the book with applicatiuon examples.

  \begin{itemize}
  \tightlist
  \item
    \hyperref[docs__pandoc__latex__src__backmatter__glossary.md__student--glossary]{Glossary}
  \end{itemize}
\item
  \textbf{Teacher Glossary} --- a concise list of terms and definitions used throughout the book with instructional tips.

  \begin{itemize}
  \tightlist
  \item
    \href{https://github.com/mrhunsaker/VI_3DMake_OpenSCAD_Curriculum/Backmatter/../Backmatter/teacher-glossary.md}{Glossary}
  \end{itemize}
\item
  \textbf{Further Reading} --- curated list of books, articles, tools, and resources to continue learning.

  \begin{itemize}
  \tightlist
  \item
    \hyperref[docs__pandoc__latex__src__backmatter__further_reading.md__references]{Further Reading}
  \end{itemize}
\item
  \textbf{Index} --- an alphabetical index of topics for quick lookup.

  \begin{itemize}
  \tightlist
  \item
    \href{https://github.com/mrhunsaker/VI_3DMake_OpenSCAD_Curriculum/Backmatter/../Backmatter/Index.md}{Index}
  \end{itemize}
\end{itemize}

\section*{How to use these appendices}\label{docs__pandoc__latex__src__backmatter__backmatter.md__how-to-use-these-appendices}

\begin{itemize}
\item
  Each appendix is intended as a focused reference you can consult when you need extra detail beyond the course material.
\item
  They are ordered in the book\textquotesingle s SUMMARY so they will appear after the main lessons and can be built separately or merged into the full PDF output.
\item
  \textbf{Glossary and Index:}

  \begin{itemize}
  \tightlist
  \item
    The glossary collects short definitions and cross-references to chapters where terms appear.
  \item
    The index is produced as a separate backmatter component and may be generated during PDF build using pandoc\textquotesingle s raw LaTeX index commands or a separate index generation step (for example, tagging terms with index markers and running \texttt{makeindex} during the LaTeX build).
  \end{itemize}
\end{itemize}

\section*{About the Author}\label{docs__pandoc__latex__src__backmatter__backmatter.md__about-the-author}

See the author\textquotesingle s notes and contributor information: \hyperref[docs__pandoc__latex__src__contributors.md__contributors]{About the Author}

\section{Appendices}\label{docs__pandoc__latex__src__appendices__cli_appendices.md__appendices-cli-appendices}

Comprehensive reference materials, guides, and supplemental resources for the OpenSCAD and 3dMake curriculum.

\subsection*{Command Line Interface}\label{docs__pandoc__latex__src__appendices__cli_appendices.md__command-line-interface}

\begin{itemize}
\tightlist
\item
  \hyperref[docs__pandoc__latex__src__3dmake_foundation__appendix_a_comprehensive_slicing_guide.md__3dmake_foundation-appendix_a_comprehensive_slicing_guide]{Appendix A: Comprehensive Slicing Guide - All Major Slicers} - Complete reference for PrusaSlicer, Bambu Studio, Cura, and OrcaSlicer configuration
\item
  \hyperref[docs__pandoc__latex__src__3dmake_foundation__appendix_b_material_properties.md__3dmake_foundation-appendix_b_material_properties]{Appendix B: Material Properties \& Selection Guide} - Detailed material reference including shrinkage data, print settings, and properties
\item
  \hyperref[docs__pandoc__latex__src__3dmake_foundation__appendix_c_tolerance_qa.md__3dmake_foundation-appendix_c_tolerance_qa]{Appendix C: Tolerance Testing \& Quality Assurance Matrix} - Comprehensive QA procedures and tolerance validation methods
\item
  \href{https://github.com/mrhunsaker/VI_3DMake_OpenSCAD_Curriculum/Appendices/../3dMake_Foundation/Appendix_D_PowerShell_Integration.md}{Appendix D: PowerShell Integration for SCAD Workflows} - Batch processing, automation scripts, and advanced workflow integration
\end{itemize}

\subsection{Appendix A: Command Line (CMD/Batch) Integration for SCAD Workflows}\label{docs__pandoc__latex__src__command_line_interface_selection__appendix_a_commandline_integration.md__3dmake_foundation-appendix_f_cmd_integration}

This appendix shows how traditional command-line (Windows CMD / batch) scripting automates SCAD workflows without requiring PowerShell. It mirrors the patterns in Appendix D but provides examples and idioms for \texttt{cmd} / batch files and systems that prefer native Windows command prompt scripts.

\subsubsection*{Overview: Why Automate with CMD/Batch?}\label{docs__pandoc__latex__src__command_line_interface_selection__appendix_a_commandline_integration.md__overview-why-automate-with-cmdbatch}

\begin{itemize}
\tightlist
\item
  Minimal dependencies: works on basic Windows installations.
\item
  Easy to call from other tools and CI systems that expect \texttt{.bat} helpers.
\item
  Useful for environments where PowerShell policy restrictions exist.
\end{itemize}

\subsubsection*{Prerequisites \& Setup}\label{docs__pandoc__latex__src__command_line_interface_selection__appendix_a_commandline_integration.md__prerequisites--setup}

\paragraph*{Required Software}\label{docs__pandoc__latex__src__command_line_interface_selection__appendix_a_commandline_integration.md__required-software}

\begin{lstlisting}[style=Alabaster, language=cmd]
where openscad      :: OpenSCAD (path in PATH or full path required)
where prusa-slicer  :: PrusaSlicer (or your slicer)
where python        :: Python (optional)

\end{lstlisting}

\paragraph*{Directory Structure (Windows style)}\label{docs__pandoc__latex__src__command_line_interface_selection__appendix_a_commandline_integration.md__directory-structure-windows-style}

\begin{lstlisting}[style=Alabaster, language=cmd]
C:\Projects\3dMake\
+------ src\
+------ stl\
+------ gcode\
+------ logs\
+------ scripts\
       +------ build.bat
       +------ batch_build.bat

\end{lstlisting}

\paragraph*{Notes on CMD vs PowerShell}\label{docs__pandoc__latex__src__command_line_interface_selection__appendix_a_commandline_integration.md__notes-on-cmd-vs-powershell}

\begin{itemize}
\tightlist
\item
  CMD/batch has more limited error handling and no structured objects; rely on return codes and file checks.
\item
  Use full executable paths to avoid PATH surprises.
\end{itemize}

\subsubsection*{Basic Workflow: Single-file build (batch)}\label{docs__pandoc__latex__src__command_line_interface_selection__appendix_a_commandline_integration.md__basic-workflow-single-file-build-batch}

Create \texttt{build.bat} (simple example):

\begin{lstlisting}[style=Alabaster, language=cmd]
@echo off
rem build.bat - Convert SCAD -> STL -> G-code (minimal)
setlocal enabledelayedexpansion
set PROJECTROOT=%~dp0\..\
set SCADFILE=%1
if "%SCADFILE%"=="" (
  echo Usage: build.bat file.scad
  exit /b 1
)
set SCADPATH=%PROJECTROOT%src\%SCADFILE%
set STLPATH=%PROJECTROOT%stl\%~n1.stl
set GCODEPATH=%PROJECTROOT%gcode\%~n1.gcode

rem Export STL using OpenSCAD
"C:\Program Files\OpenSCAD\openscad.exe" -o "%STLPATH%" "%SCADPATH%"
if not exist "%STLPATH%" (
  echo ERROR: STL not created
  exit /b 1
)

rem Slice (example) - adjust path to slicer
"C:\Program Files\Prusa3D\PrusaSlicer\prusa-slicer.exe" --load-config-file "%~dp0\default.ini" -o "%GCODEPATH%" "%STLPATH%"
if not exist "%GCODEPATH%" (
  echo ERROR: G-code not created
  exit /b 1
)

echo BUILD COMPLETE: %GCODEPATH%
endlocal

\end{lstlisting}

\subsubsection*{Batch (directory) build}\label{docs__pandoc__latex__src__command_line_interface_selection__appendix_a_commandline_integration.md__batch-directory-build}

\texttt{batch\_build.bat} can iterate files and call \texttt{build.bat}:

\begin{lstlisting}[style=Alabaster, language=cmd]
@echo off
set PROJECTROOT=%~dp0\..\
for /r "%PROJECTROOT%src" %%f in (*.scad) do (
  echo Processing %%~nxf
  call "%~dp0build.bat" "%%~nxf"
)

\end{lstlisting}

\subsubsection*{Parametric Sweeps (simple templating)}\label{docs__pandoc__latex__src__command_line_interface_selection__appendix_a_commandline_integration.md__parametric-sweeps-simple-templating}

Because batch is limited for text templating, a common pattern is to use small helper scripts with \texttt{sed}/\texttt{python} or write a minimal templating helper in a language such as Python. Example invocation in batch:

\begin{lstlisting}[style=Alabaster, language=cmd]
rem Example: call a Python script to generate variants, then build
python "%~dp0/generate_variants.py" "%PROJECTROOT%src\bracelet_holder.scad"
call "%~dp0/batch_build.bat"

\end{lstlisting}

\subsubsection*{Logging and CSV output}\label{docs__pandoc__latex__src__command_line_interface_selection__appendix_a_commandline_integration.md__logging-and-csv-output}

Batch scripts can append simple CSV lines using \texttt{echo} redirection:

\begin{lstlisting}[style=Alabaster, language=cmd]
echo %DATE% %TIME%,%~n1,Success >> "%PROJECTROOT%logs\build_history.csv"

\end{lstlisting}

\subsubsection*{Send to Printer (USB copy)}\label{docs__pandoc__latex__src__command_line_interface_selection__appendix_a_commandline_integration.md__send-to-printer-usb-copy}

\begin{lstlisting}[style=Alabaster, language=cmd]
rem Copy gcode to removable drive (E: example)
copy "%GCODEPATH%" E:\

\end{lstlisting}

\subsubsection*{Monitoring (networked printer) - using curl or PowerShell helper}\label{docs__pandoc__latex__src__command_line_interface_selection__appendix_a_commandline_integration.md__monitoring-networked-printer---using-curl-or-powershell-helper}

\texttt{curl} on Windows or a small PowerShell one-liner may be used to query APIs. For pure CMD, ship a small helper .exe (curl) or call \texttt{powershell\ -Command\ "Invoke-RestMethod\ ..."}.

\subsubsection*{Best Practices}\label{docs__pandoc__latex__src__command_line_interface_selection__appendix_a_commandline_integration.md__best-practices}

\begin{itemize}
\tightlist
\item
  Use full paths for all tools.
\item
  Check return codes (\texttt{if\ errorlevel\ 1}) after each step.
\item
  Keep batch scripts small; delegate complex text processing to Python/PowerShell.
\item
  Log to simple CSVs for human and machine parsing.
\end{itemize}

\begin{center}\rule{0.5\linewidth}{0.5pt}\end{center}

\subsection{Appendix B: PowerShell Integration for SCAD Workflows}\label{docs__pandoc__latex__src__command_line_interface_selection__appendix_b_powershell_integration.md__3dmake_foundation-appendix_d_powershell_integration}

This appendix shows how PowerShell (command-line scripting) streamlines 3D design workflows, automating the repetitive tasks from design through printing. It bridges PowerShell\_Foundation concepts with 3dMake\_Foundation practical applications.

Referenced in: Lessons 9 (Automation), 10 (Mastery), and any lesson requiring batch operations

Prerequisites: Completion of PowerShell\_Foundation (Lessons 1-6)

\subsubsection*{Overview: Why Automate SCAD Workflows?}\label{docs__pandoc__latex__src__command_line_interface_selection__appendix_b_powershell_integration.md__overview-why-automate-scad-workflows}

\paragraph*{Manual Workflow (Time-Consuming)}\label{docs__pandoc__latex__src__command_line_interface_selection__appendix_b_powershell_integration.md__manual-workflow-time-consuming}

\begin{enumerate}
\tightlist
\item
  Open SCAD manually -\textgreater{} Edit parameters -\textgreater{} Save -\textgreater{} Export STL
\item
  Open Slicer manually -\textgreater{} Load STL -\textgreater{} Adjust settings -\textgreater{} Slice -\textgreater{} Export G-code
\item
  Transfer G-code to printer via USB
\item
  Record results in notebook
\item
  Repeat for next variation (variant 2, variant 3, etc.)
\end{enumerate}

Time: \textasciitilde{}20-30 minutes per design iteration

\paragraph*{Automated Workflow (Fast \& Reproducible)}\label{docs__pandoc__latex__src__command_line_interface_selection__appendix_b_powershell_integration.md__automated-workflow-fast--reproducible}

\begin{enumerate}
\tightlist
\item
  Create PowerShell script with design parameters
\item
  Script automatically:

  \begin{itemize}
  \tightlist
  \item
    Generates SCAD code
  \item
    Exports to STL
  \item
    Slices to G-code
  \item
    Transfers to printer
  \item
    Logs results
  \end{itemize}
\item
  Run script once, walk away
\item
  Check results later
\end{enumerate}

Time: 2-3 minutes per iteration (plus print time)

\paragraph*{Benefits of Automation}\label{docs__pandoc__latex__src__command_line_interface_selection__appendix_b_powershell_integration.md__benefits-of-automation}

\begin{itemize}
\tightlist
\item
  {[}YES{]} Speed: 10x faster for batch operations
\item
  {[}YES{]} Consistency: No manual errors
\item
  {[}YES{]} Reproducibility: Same settings every time
\item
  {[}YES{]} Scalability: Test 5, 10, or 100 variations easily
\item
  {[}YES{]} Logging: Automatic documentation
\item
  {[}YES{]} Accessibility: Screen reader captures script output (not UI clicks)
\end{itemize}

\subsubsection*{Prerequisites \& Setup}\label{docs__pandoc__latex__src__command_line_interface_selection__appendix_b_powershell_integration.md__prerequisites--setup}

\paragraph*{Required Software}\label{docs__pandoc__latex__src__command_line_interface_selection__appendix_b_powershell_integration.md__required-software}

\begin{lstlisting}[style=Alabaster, language=powershell]
# Check what you have installed
where openscad      # OpenSCAD
where prusa-slicer  # PrusaSlicer (or your slicer)
where python3       # Python (optional, for advanced scripts)

\end{lstlisting}

\paragraph*{Directory Structure}\label{docs__pandoc__latex__src__command_line_interface_selection__appendix_b_powershell_integration.md__directory-structure}

Create a project directory:

\begin{lstlisting}[style=Alabaster]
C:\Projects\3dMake\
+------ src/
|   +------ bracelet_holder.scad
|   +------ phone_stand.scad
|   +------ stackable_bins.scad
+------ stl/
|   +------ bracelet_holder.stl
|   +------ phone_stand.stl
|   +------ stackable_bins.stl
+------ gcode/
|   +------ bracelet_holder.gcode
|   +------ phone_stand.gcode
|   +------ stackable_bins.gcode
+------ logs/
|   +------ batch_2024-01-15.log
|   +------ print_history.csv
+------ scripts/
    +------ build.ps1      (main build script)
    +------ batch_test.ps1 (test variations)
    +------ monitor.ps1    (monitor printer)

\end{lstlisting}

\paragraph*{PowerShell Execution Policy}\label{docs__pandoc__latex__src__command_line_interface_selection__appendix_b_powershell_integration.md__powershell-execution-policy}

\begin{lstlisting}[style=Alabaster, language=powershell]
# Check current policy
Get-ExecutionPolicy
# If it's "Restricted", change it (requires admin)
Set-ExecutionPolicy -ExecutionPolicy RemoteSigned -Scope CurrentUser
# Verify change
Get-ExecutionPolicy

\end{lstlisting}

\subsubsection*{Basic Workflow Automation}\label{docs__pandoc__latex__src__command_line_interface_selection__appendix_b_powershell_integration.md__basic-workflow-automation}

\paragraph*{Script 1: Single-File Build}\label{docs__pandoc__latex__src__command_line_interface_selection__appendix_b_powershell_integration.md__script-1-single-file-build}

This script takes a SCAD file and converts it through the entire workflow:

\begin{lstlisting}[style=Alabaster, language=powershell]
# build.ps1 - Convert SCAD -> STL -> G-code
param(
    [string]$ScadFile,          # Input: bracelet_holder.scad
    [string]$OutputDir = ".\",  # Output directory
    [string]$SlicerConfig = "default"
)
# Set up paths
$ProjectRoot = "C:\Projects\3dMake"
$ScadFile = Join-Path $ProjectRoot "src" $ScadFile
$StlFile = Join-Path $ProjectRoot "stl" ([System.IO.Path]::GetFileNameWithoutExtension($ScadFile) + ".stl")
$GcodeFile = Join-Path $ProjectRoot "gcode" ([System.IO.Path]::GetFileNameWithoutExtension($ScadFile) + ".gcode")
$LogFile = Join-Path $ProjectRoot "logs" "build_$(Get-Date -Format 'yyyy-MM-dd').log"
# Log function
function Write-Log {
    param([string]$Message)
    $timestamp = Get-Date -Format "yyyy-MM-dd HH:mm:ss"
    $logEntry = "[$timestamp] $Message"
    Write-Host $logEntry
    Add-Content -Path $LogFile -Value $logEntry
}
Write-Log "Starting build for $($ScadFile | Split-Path -Leaf)"
# Step 1: Export STL
Write-Log "Step 1: Exporting SCAD to STL..."
$startTime = Get-Date
& "C:\Program Files\OpenSCAD\openscad.exe" `
    -o "$StlFile" `
    "$ScadFile"
$duration = (Get-Date) - $startTime
Write-Log " STL exported in $($duration.TotalSeconds) seconds: $StlFile"
# Verify STL exists
if (-not (Test-Path $StlFile)) {
    Write-Log "[NO] ERROR: STL file not created"
    exit 1
}
# Step 2: Slice to G-code
Write-Log "Step 2: Slicing STL to G-code..."
$startTime = Get-Date
& "C:\Program Files\Prusa3D\PrusaSlicer\prusa-slicer.exe" `
    --load-config-file "$SlicerConfig" `
    --export-gcode "$GcodeFile" `
    "$StlFile"
$duration = (Get-Date) - $startTime
Write-Log " G-code generated in $($duration.TotalSeconds) seconds: $GcodeFile"
# Verify G-code exists
if (-not (Test-Path $GcodeFile)) {
    Write-Log "[NO] ERROR: G-code file not created"
    exit 1
}
# Step 3: Get file info
$stlSize = (Get-Item $StlFile).Length / 1MB
$gcodeSize = (Get-Item $GcodeFile).Length / 1MB
Write-Log "File sizes: STL=$([math]::Round($stlSize, 2))MB, G-code=$([math]::Round($gcodeSize, 2))MB"
Write-Log " BUILD COMPLETE"
Write-Log "Ready to print: $GcodeFile"

\end{lstlisting}

Usage:

\begin{lstlisting}[style=Alabaster, language=powershell]
# Run the script
.\build.ps1 -ScadFile "bracelet_holder.scad"
# Output example:
# [2024-01-15 14:30:22] Starting build for bracelet_holder.scad
# [2024-01-15 14:30:23] Step 1: Exporting SCAD to STL...
# [2024-01-15 14:30:25]  STL exported in 2.41 seconds: ...
# [2024-01-15 14:30:26] Step 2: Slicing STL to G-code...
# [2024-01-15 14:30:28]  G-code generated in 2.31 seconds: ...
# [2024-01-15 14:30:28]  BUILD COMPLETE

\end{lstlisting}

\paragraph*{Script 2: Batch Build (Multiple Files)}\label{docs__pandoc__latex__src__command_line_interface_selection__appendix_b_powershell_integration.md__script-2-batch-build-multiple-files}

\begin{lstlisting}[style=Alabaster, language=powershell]
# batch_build.ps1 - Build all SCAD files in a directory
param(
    [string]$SourceDir = "C:\Projects\3dMake\src",
    [string]$SlicerConfig = "default"
)
$ProjectRoot = Split-Path $SourceDir
$buildScript = Join-Path $ProjectRoot "scripts" "build.ps1"
# Find all SCAD files
$scadFiles = Get-ChildItem -Path $SourceDir -Filter "*.scad" -Recurse
Write-Host "Found $($scadFiles.Count) SCAD files"
if ($scadFiles.Count -eq 0) {
    Write-Host "No SCAD files found in $SourceDir"
    exit 1
}
# Process each file
$results = @()
foreach ($file in $scadFiles) {
    Write-Host "`n--- Processing: $($file.Name) ---"
    $startTime = Get-Date
    # Run build script for this file
    & $buildScript -ScadFile $file.Name -SlicerConfig $SlicerConfig
    $duration = (Get-Date) - $startTime
    $results += [PSCustomObject]@{
        FileName = $file.Name
        DurationSeconds = $duration.TotalSeconds
        Status = "Success"
        Timestamp = Get-Date
    }
}
# Summary report
Write-Host "`n=== BATCH BUILD SUMMARY ==="
Write-Host "Files processed: $($results.Count)"
Write-Host "Total time: $([math]::Round(($results | Measure-Object -Property DurationSeconds -Sum).Sum, 1)) seconds"
# Save results to CSV
$csvPath = Join-Path $ProjectRoot "logs" "batch_$(Get-Date -Format 'yyyy-MM-dd-HHmmss').csv"
$results | Export-Csv -Path $csvPath -NoTypeInformation
Write-Host "Results saved: $csvPath"

\end{lstlisting}

Usage:

\begin{lstlisting}[style=Alabaster, language=powershell]
.\batch_build.ps1 -SourceDir "C:\Projects\3dMake\src"
# Processes all .scad files automatically

\end{lstlisting}

\subsubsection*{Parametric Design Variation Testing}\label{docs__pandoc__latex__src__command_line_interface_selection__appendix_b_powershell_integration.md__parametric-design-variation-testing}

\paragraph*{The Problem: Testing Multiple Parameter Values}\label{docs__pandoc__latex__src__command_line_interface_selection__appendix_b_powershell_integration.md__the-problem-testing-multiple-parameter-values}

You want to test the same design with different parameters (e.g., peg diameter: 5mm, 6mm, 7mm):

Manually:

\begin{enumerate}
\tightlist
\item
  Open bracelet\_holder.scad
\item
  Change peg\_diameter = 5
\item
  Save, export, slice, print
\item
  Change peg\_diameter = 6
\item
  Save, export, slice, print
\item
  Change peg\_diameter = 7
\item
  Save, export, slice, print Time: 30 minutes
\end{enumerate}

\paragraph*{Automated Solution: Parametric Sweep}\label{docs__pandoc__latex__src__command_line_interface_selection__appendix_b_powershell_integration.md__automated-solution-parametric-sweep}

\begin{lstlisting}[style=Alabaster, language=powershell]
# parametric_sweep.ps1 - Test multiple parameter values
param(
    [string]$TemplateScad = "C:\Projects\3dMake\src\bracelet_holder.scad",
    [hashtable]$ParameterRanges = @{
        "peg_diameter" = @(5, 5.5, 6, 6.5, 7)
        "holder_width" = @(120, 127, 135)
    }
)
$ProjectRoot = "C:\Projects\3dMake"
$variantsDir = Join-Path $ProjectRoot "variants"
$buildScript = Join-Path $ProjectRoot "scripts" "build.ps1"
$logFile = Join-Path $ProjectRoot "logs" "parametric_sweep_$(Get-Date -Format 'yyyy-MM-dd-HHmmss').csv"
# Create variants directory
New-Item -ItemType Directory -Path $variantsDir -Force | Out-Null
# Read template
$template = Get-Content $TemplateScad -Raw
# Generate variants
$results = @()
$variantNum = 0
# For each diameter value
foreach ($dia in $ParameterRanges["peg_diameter"]) {
    # For each width value
    foreach ($width in $ParameterRanges["holder_width"]) {
        $variantNum++
        $variantName = "variant_dia${dia}_width${width}"
        $variantScad = Join-Path $variantsDir "$variantName.scad"
        # Create variant SCAD file with parameters
        $content = $template `
            -replace 'peg_diameter = [\d.]+', "peg_diameter = $dia" `
            -replace 'holder_width = [\d.]+', "holder_width = $width"
        $content | Set-Content $variantScad
        Write-Host "Generated: $variantName"
        # Build variant (STL + G-code)
        $startTime = Get-Date
        # Export to STL
        & "C:\Program Files\OpenSCAD\openscad.exe" `
            -o (Join-Path $ProjectRoot "stl" "$variantName.stl") `
            $variantScad
        $duration = (Get-Date) - $startTime
        # Record result
        $results += [PSCustomObject]@{
            Variant = $variantName
            PegDiameter = $dia
            HolderWidth = $width
            BuildTimeSeconds = $duration.TotalSeconds
            Status = "Success"
            Timestamp = Get-Date
        }
        Write-Host "   Built in $($duration.TotalSeconds) seconds"
    }
}
# Save results
$results | Export-Csv -Path $logFile -NoTypeInformation
Write-Host "`nParametric sweep complete: $($results.Count) variants generated"
Write-Host "Results: $logFile"
# Summary statistics
Write-Host "`n=== PARAMETER RANGES TESTED ==="
Write-Host "Peg diameters: $($ParameterRanges['peg_diameter'] -join ', ')"
Write-Host "Holder widths: $($ParameterRanges['holder_width'] -join ', ')"
Write-Host "Total combinations: $($results.Count)"

\end{lstlisting}

Usage:

\begin{lstlisting}[style=Alabaster, language=powershell]
$ranges = @{
    "peg_diameter" = @(5, 5.5, 6, 6.5, 7)
    "holder_width" = @(120, 127, 135)
}
.\parametric_sweep.ps1 -TemplateScad "bracelet_holder.scad" -ParameterRanges $ranges
# Generates 15 variants automatically (5 x 3)

\end{lstlisting}

Output CSV:

\begin{lstlisting}[style=Alabaster]
Variant,PegDiameter,HolderWidth,BuildTimeSeconds,Status,Timestamp
variant_dia5_width120,5,120,2.4,Success,2024-01-15 14:35:22
variant_dia5_width127,5,127,2.5,Success,2024-01-15 14:35:25
variant_dia5_width135,5,135,2.3,Success,2024-01-15 14:35:28
...

\end{lstlisting}

\subsubsection*{Automated Print Documentation}\label{docs__pandoc__latex__src__command_line_interface_selection__appendix_b_powershell_integration.md__automated-print-documentation}

\paragraph*{Script: Print Logging \& Quality Tracking}\label{docs__pandoc__latex__src__command_line_interface_selection__appendix_b_powershell_integration.md__script-print-logging--quality-tracking}

\begin{lstlisting}[style=Alabaster, language=powershell]
# log_print.ps1 - Document each print with metadata
param(
    [string]$GcodeFile,
    [string]$ProjectName,
    [string]$Material = "PLA",
    [string]$Notes = ""
)
$ProjectRoot = "C:\Projects\3dMake"
$printLogCsv = Join-Path $ProjectRoot "logs" "print_history.csv"
# Get G-code file info
$gcodeInfo = Get-Item $GcodeFile
$gcodeSize = $gcodeInfo.Length / 1MB
$estimatedTime = EstimateTime $GcodeFile  # See function below
# Calculate estimated cost
$weightG = EstimateWeight $GcodeFile
$materialCostPerKg = 20  # PLA at $20/kg
$estimatedCost = ($weightG / 1000) * $materialCostPerKg
# Create log entry
$entry = [PSCustomObject]@{
    PrintID = "PRINT_$(Get-Date -Format 'yyyyMMdd_HHmmss')"
    ProjectName = $ProjectName
    Date = Get-Date
    Material = $Material
    GcodeFile = $gcodeFile
    GcodeSizeMB = [math]::Round($gcodeSize, 2)
    EstimatedTimeHours = [math]::Round($estimatedTime / 3600, 2)
    EstimatedWeightG = [math]::Round($weightG, 1)
    EstimatedCostUSD = [math]::Round($estimatedCost, 2)
    Notes = $Notes
}
# Append to CSV
if (Test-Path $printLogCsv) {
    $entry | Export-Csv -Path $printLogCsv -NoTypeInformation -Append
} else {
    $entry | Export-Csv -Path $printLogCsv -NoTypeInformation
}
Write-Host "Print logged:"
Write-Host "  Project: $ProjectName"
Write-Host "  File: $($gcodeInfo.Name)"
Write-Host "  Estimated time: $([math]::Round($estimatedTime / 3600, 2)) hours"
Write-Host "  Estimated weight: $([math]::Round($weightG, 1))g"
Write-Host "  Estimated cost: $$([math]::Round($estimatedCost, 2))"
# Function to estimate print time from G-code (simplified)
function EstimateTime {
    param([string]$GcodeFile)
    $lines = @(Get-Content $GcodeFile | Select-String "^G1" | Measure-Object).Count
    # Rough estimate: 10 lines per second
    return $lines / 10
}
# Function to estimate weight from G-code (simplified)
function EstimateWeight {
    param([string]$GcodeFile)
    # G-code filament estimate (if supported by your slicer)
    $content = Get-Content $GcodeFile -Raw
    if ($content -match "filament used = ([\d.]+)") {
        return [double]$matches[1]
    } else {
        return 0  # Fallback
    }
}

\end{lstlisting}

Usage:

\begin{lstlisting}[style=Alabaster, language=powershell]
.\log_print.ps1 `
    -GcodeFile "C:\Projects\3dMake\gcode\bracelet_holder.gcode" `
    -ProjectName "Bracelet Holder" `
    -Material "PLA" `
    -Notes "Final design, tested with actual bracelets"

\end{lstlisting}

\subsubsection*{Printer Communication \& Monitoring}\label{docs__pandoc__latex__src__command_line_interface_selection__appendix_b_powershell_integration.md__printer-communication--monitoring}

\paragraph*{Script: Send G-code to Printer (USB)}\label{docs__pandoc__latex__src__command_line_interface_selection__appendix_b_powershell_integration.md__script-send-g-code-to-printer-usb}

\begin{lstlisting}[style=Alabaster, language=powershell]
# send_to_printer.ps1 - Copy G-code to printer USB
param(
    [string]$GcodeFile,
    [string]$PrinterUSBLetter = "E"  # E:, F:, G:, etc.
)
$printerDrive = "${PrinterUSBLetter}:\"
# Check if USB drive connected
if (-not (Test-Path $printerDrive)) {
    Write-Host "[NO] ERROR: Printer USB not found at $PrinterUSBLetter"
    Write-Host "Available drives:"
    Get-PSDrive -PSProvider FileSystem | Where-Object Name -Like "[D-Z]" | Select-Object Name
    exit 1
}
# Copy file
$filename = Split-Path $GcodeFile -Leaf
$destination = Join-Path $printerDrive $filename
Copy-Item -Path $GcodeFile -Destination $destination -Force
Write-Host " G-code copied to printer USB:"
Write-Host "  From: $GcodeFile"
Write-Host "  To: $destination"
Write-Host "`nNext steps:"
Write-Host "  1. Eject USB safely from computer"
Write-Host "  2. Insert USB into printer"
Write-Host "  3. Select file on printer screen"
Write-Host "  4. Press print"

\end{lstlisting}

Usage:

\begin{lstlisting}[style=Alabaster, language=powershell]
.\send_to_printer.ps1 -GcodeFile "C:\Projects\3dMake\gcode\bracelet_holder.gcode" -PrinterUSBLetter "E"

\end{lstlisting}

\paragraph*{Script: Monitor Printer Status (Network Printers)}\label{docs__pandoc__latex__src__command_line_interface_selection__appendix_b_powershell_integration.md__script-monitor-printer-status-network-printers}

\begin{lstlisting}[style=Alabaster, language=powershell]
# monitor_printer.ps1 - Check printer status via API
param(
    [string]$PrinterIP,
    [int]$CheckInterval = 30,    # seconds
    [int]$MaxChecks = 240        # 2 hours max
)
$printerUrl = "[https://example.com](https://example.com)
$checksPerformed = 0
Write-Host "Monitoring printer at $PrinterIP"
Write-Host "Check interval: $CheckInterval seconds"
Write-Host "Ctrl+C to stop`n"
while ($checksPerformed -lt $MaxChecks) {
    try {
        # Get printer status
        $response = Invoke-WebRequest -Uri $printerUrl -UseBasicParsing -ErrorAction Stop
        $status = $response.Content | ConvertFrom-Json
        $state = $status.state.text
        $progress = $status.progress.completion
        $timeRemaining = $status.progress.printTimeLeft
        Write-Host "[$((Get-Date).ToString("HH:mm:ss"))] State: $state | Progress: $progress% | Time remaining: $timeRemaining seconds"
        # If print completed, exit loop
        if ($state -eq "Operational") {
            Write-Host "`n Print complete!"
            break
        }
    } catch {
        Write-Host "Connection error: $_"
    }
    Start-Sleep -Seconds $CheckInterval
    $checksPerformed++
}
if ($checksPerformed -ge $MaxChecks) {
    Write-Host "Max monitoring time reached. Stopping."
}

\end{lstlisting}

Usage:

\begin{lstlisting}[style=Alabaster, language=powershell]
.\monitor_printer.ps1 -PrinterIP "192.168.1.100" -CheckInterval 30

\end{lstlisting}

\subsubsection*{Complete Workflow Integration}\label{docs__pandoc__latex__src__command_line_interface_selection__appendix_b_powershell_integration.md__complete-workflow-integration}

\paragraph*{Master Script: Design -\textgreater{} Print -\textgreater{} Log}\label{docs__pandoc__latex__src__command_line_interface_selection__appendix_b_powershell_integration.md__master-script-design---print---log}

\begin{lstlisting}[style=Alabaster, language=powershell]
# full_workflow.ps1 - Complete automation from design to printing
param(
    [string]$ProjectName,
    [string]$ScadFile,
    [string]$Material = "PLA",
    [string]$Notes = "",
    [switch]$SendToPrinter,
    [string]$PrinterUSBLetter = "E"
)
$ProjectRoot = "C:\Projects\3dMake"
Write-Host "=== 3dMake Full Workflow Automation ==="
Write-Host "Project: $ProjectName"
Write-Host ""
# Step 1: Build (SCAD -> STL -> G-code)
Write-Host "Step 1: Building..."
& "$ProjectRoot\scripts\build.ps1" -ScadFile $ScadFile
if ($LASTEXITCODE -ne 0) { exit 1 }
# Step 2: Log the print
Write-Host "`nStep 2: Logging print metadata..."
$gcodeFile = Join-Path $ProjectRoot "gcode" ([System.IO.Path]::GetFileNameWithoutExtension($ScadFile) + ".gcode")
& "$ProjectRoot\scripts\log_print.ps1" `
    -GcodeFile $gcodeFile `
    -ProjectName $ProjectName `
    -Material $Material `
    -Notes $Notes
# Step 3: Send to printer (optional)
if ($SendToPrinter) {
    Write-Host "`nStep 3: Sending to printer..."
    & "$ProjectRoot\scripts\send_to_printer.ps1" `
        -GcodeFile $gcodeFile `
        -PrinterUSBLetter $PrinterUSBLetter
}
Write-Host "`n Workflow complete!"

\end{lstlisting}

Usage:

\begin{lstlisting}[style=Alabaster, language=powershell]
# Full workflow with automatic printer transfer
.\full_workflow.ps1 `
    -ProjectName "Bracelet Holder" `
    -ScadFile "bracelet_holder.scad" `
    -Material "PLA" `
    -Notes "Final design, v2 with improved peg strength" `
    -SendToPrinter `
    -PrinterUSBLetter "E"

\end{lstlisting}

\subsubsection*{PowerShell Skills Applied to SCAD}\label{docs__pandoc__latex__src__command_line_interface_selection__appendix_b_powershell_integration.md__powershell-skills-applied-to-scad}

\paragraph*{Lesson Mapping: PowerShell -\textgreater{} SCAD Workflows}\label{docs__pandoc__latex__src__command_line_interface_selection__appendix_b_powershell_integration.md__lesson-mapping-powershell---scad-workflows}

{\def\LTcaptype{none} % do not increment counter
\begin{longtable}[]{@{}
  >{\raggedright\arraybackslash}p{(\linewidth - 4\tabcolsep) * \real{0.2432}}
  >{\raggedright\arraybackslash}p{(\linewidth - 4\tabcolsep) * \real{0.3063}}
  >{\raggedright\arraybackslash}p{(\linewidth - 4\tabcolsep) * \real{0.4505}}@{}}
\toprule\noalign{}
\begin{minipage}[b]{\linewidth}\raggedright
PowerShell Lesson
\end{minipage} & \begin{minipage}[b]{\linewidth}\raggedright
SCAD Application
\end{minipage} & \begin{minipage}[b]{\linewidth}\raggedright
Example
\end{minipage} \\
\midrule\noalign{}
\endhead
\bottomrule\noalign{}
\endlastfoot
PS 1: Navigation & Working with project directories &
Using \texttt{cd} to navigate src/ -\textgreater{} stl/ -\textgreater{} gcode/ \\
PS 2: File Manipulation & Copying, organizing SCAD files &
Copy design files, organize variants by date \\
PS 3: Piping \& Objects & Pass data between scripts &
\texttt{\$results\ \textbar{}\ Export-Csv} exports analysis \\
PS 4: Variables \& Aliases & Parameterize SCAD workflows &
Variables store file paths, material types, etc. \\
PS 5: Functions \& Modules & Reusable automation blocks &
Build, slice, log = functions in one script \\
PS Unit Test & Verify workflow correctness &
Test that STL file exists before slicing \\
\end{longtable}
}

\paragraph*{Example: File Navigation with SCAD Projects}\label{docs__pandoc__latex__src__command_line_interface_selection__appendix_b_powershell_integration.md__example-file-navigation-with-scad-projects}

\begin{lstlisting}[style=Alabaster, language=powershell]
# Navigate between SCAD project folders
$ProjectRoot = "C:\Projects\3dMake"
# Go to source directory
cd "$ProjectRoot\src"
Get-ChildItem -Filter "*.scad"  # List all SCAD files
# Process each file
foreach ($file in Get-ChildItem -Filter "*.scad") {
    Write-Host "Processing: $($file.Name)"
    # Do something with each file
}
# Go to output directory and check results
cd "$ProjectRoot\stl"
Get-ChildItem -Filter "*.stl" | Measure-Object -Sum

\end{lstlisting}

\paragraph*{Example: Working with Piping \& Objects}\label{docs__pandoc__latex__src__command_line_interface_selection__appendix_b_powershell_integration.md__example-working-with-piping--objects}

\begin{lstlisting}[style=Alabaster, language=powershell]
# Build all files, then get statistics
Get-ChildItem -Path "$ProjectRoot\src" -Filter "*.scad" |
    ForEach-Object {
        # Build each one
        & .\build.ps1 -ScadFile $_.Name
    } |
    # Analyze results
    Group-Object -Property Status |
    Select-Object Name, Count

\end{lstlisting}

\subsubsection*{Accessibility Considerations}\label{docs__pandoc__latex__src__command_line_interface_selection__appendix_b_powershell_integration.md__accessibility-considerations}

\paragraph*{Why PowerShell for SCAD?}\label{docs__pandoc__latex__src__command_line_interface_selection__appendix_b_powershell_integration.md__why-powershell-for-scad}

\begin{enumerate}
\tightlist
\item
  Screen Reader Friendly: Console output = text (not UI clicks)
\item
  Scriptable: Can run overnight, results logged
\item
  Auditable: Every step written to log file
\item
  Shareable: Scripts document the process for others
\item
  Testable: Can verify each step independently
\end{enumerate}

\paragraph*{Example: Accessible Build Output}\label{docs__pandoc__latex__src__command_line_interface_selection__appendix_b_powershell_integration.md__example-accessible-build-output}

\begin{lstlisting}[style=Alabaster]
[2024-01-15 14:30:22] Starting build for bracelet_holder.scad
[2024-01-15 14:30:23] Step 1: Exporting SCAD to STL...
[2024-01-15 14:30:25]  STL exported in 2.41 seconds
[2024-01-15 14:30:26] Step 2: Slicing STL to G-code...
[2024-01-15 14:30:28]  G-code generated in 2.31 seconds
[2024-01-15 14:30:28] File sizes: STL=2.15MB, G-code=4.32MB
[2024-01-15 14:30:28]  BUILD COMPLETE

Ready to print: C:\Projects\3dMake\gcode\bracelet_holder.gcode

\end{lstlisting}

All information is text-based and sequential-perfectly accessible to screen readers.

\subsubsection*{Best Practices \& Tips}\label{docs__pandoc__latex__src__command_line_interface_selection__appendix_b_powershell_integration.md__best-practices--tips}

\paragraph*{1. Always Log Everything}\label{docs__pandoc__latex__src__command_line_interface_selection__appendix_b_powershell_integration.md__1-always-log-everything}

\begin{lstlisting}[style=Alabaster, language=powershell]
function Write-Log {
    param(
        [string]$Message,
        [string]$LogFile
    )
    $timestamp = Get-Date -Format "yyyy-MM-dd HH:mm:ss"
    $entry = "[$timestamp] $Message"
    Write-Host $entry
    Add-Content -Path $LogFile -Value $entry
}

\end{lstlisting}

\paragraph*{2. Verify Steps Succeed Before Continuing}\label{docs__pandoc__latex__src__command_line_interface_selection__appendix_b_powershell_integration.md__2-verify-steps-succeed-before-continuing}

\begin{lstlisting}[style=Alabaster, language=powershell]
# Don't just run commands-check they worked
& "C:\Program Files\OpenSCAD\openscad.exe" -o "$StlFile" "$ScadFile"
if (-not (Test-Path $StlFile)) {
    Write-Log "ERROR: STL creation failed"
    exit 1
}
Write-Log " STL created successfully"

\end{lstlisting}

\paragraph*{3. Make Scripts Reusable}\label{docs__pandoc__latex__src__command_line_interface_selection__appendix_b_powershell_integration.md__3-make-scripts-reusable}

\begin{lstlisting}[style=Alabaster, language=powershell]
# BAD: Hardcoded paths
$scadFile = "C:\Users\John\Desktop\bracelet_holder.scad"
# GOOD: Parameters with defaults
param(
    [string]$ScadFile = "bracelet_holder.scad",
    [string]$ProjectRoot = "C:\Projects\3dMake"
)
$fullPath = Join-Path $ProjectRoot "src" $ScadFile

\end{lstlisting}

\paragraph*{4. Use Configuration Files}\label{docs__pandoc__latex__src__command_line_interface_selection__appendix_b_powershell_integration.md__4-use-configuration-files}

\begin{lstlisting}[style=Alabaster, language=powershell]
# config.ps1 - Centralized settings
$Config = @{
    ProjectRoot = "C:\Projects\3dMake"
    OpenSCADPath = "C:\Program Files\OpenSCAD\openscad.exe"
    SlicerPath = "C:\Program Files\Prusa3D\PrusaSlicer\prusa-slicer.exe"
    PrinterUSB = "E"
    DefaultMaterial = "PLA"
    DefaultInfill = 20
}
# Use in scripts:
# . .\config.ps1
# & $Config.OpenSCADPath ...

\end{lstlisting}

\subsubsection*{Troubleshooting Automated Workflows}\label{docs__pandoc__latex__src__command_line_interface_selection__appendix_b_powershell_integration.md__troubleshooting-automated-workflows}

{\def\LTcaptype{none} % do not increment counter
\begin{longtable}[]{@{}
  >{\raggedright\arraybackslash}p{(\linewidth - 4\tabcolsep) * \real{0.1695}}
  >{\raggedright\arraybackslash}p{(\linewidth - 4\tabcolsep) * \real{0.2458}}
  >{\raggedright\arraybackslash}p{(\linewidth - 4\tabcolsep) * \real{0.5847}}@{}}
\toprule\noalign{}
\begin{minipage}[b]{\linewidth}\raggedright
Problem
\end{minipage} & \begin{minipage}[b]{\linewidth}\raggedright
Cause
\end{minipage} & \begin{minipage}[b]{\linewidth}\raggedright
Solution
\end{minipage} \\
\midrule\noalign{}
\endhead
\bottomrule\noalign{}
\endlastfoot
Scripts won\textquotesingle t run & Execution policy &
\texttt{Set-ExecutionPolicy\ -ExecutionPolicy\ RemoteSigned} \\
Command not found & Tool not in PATH &
Specify full path: \texttt{\&\ "C:\textbackslash{}Program\ Files\textbackslash{}OpenSCAD\textbackslash{}openscad.exe"} \\
Files not created & Wrong working directory &
Use absolute paths with \texttt{Join-Path} \\
Script hangs & Waiting for user input &
Disable UI input; use \texttt{-o} flag for batch operations \\
CSV data corrupted & Special characters in notes &
Properly quote strings in CSV export \\
No output/logging & Path doesn\textquotesingle t exist &
Create directory first: \texttt{New-Item\ -Type\ Directory\ -Path\ ...\ -Force} \\
\end{longtable}
}

\subsubsection*{Summary: PowerShell + SCAD Integration Benefits}\label{docs__pandoc__latex__src__command_line_interface_selection__appendix_b_powershell_integration.md__summary-powershell--scad-integration-benefits}

\paragraph*{Time Savings}\label{docs__pandoc__latex__src__command_line_interface_selection__appendix_b_powershell_integration.md__time-savings}

{\def\LTcaptype{none} % do not increment counter
\begin{longtable}[]{@{}llll@{}}
\toprule\noalign{}
Task & Manual & Automated & Savings \\
\midrule\noalign{}
\endhead
\bottomrule\noalign{}
\endlastfoot
Single design build & 5 min & 2.5 min & 50\% \\
Test 10 parameter variations & 50 min & 5 min & 90\% \\
Batch processing 20 designs & 100 min & 10 min & 90\% \\
\end{longtable}
}

\paragraph*{Consistency}\label{docs__pandoc__latex__src__command_line_interface_selection__appendix_b_powershell_integration.md__consistency}

\begin{itemize}
\tightlist
\item
  {[}YES{]} Same process every time (no missed steps)
\item
  {[}YES{]} Reproducible results (exact same settings)
\item
  {[}YES{]} Comprehensive logging (documentation automatic)
\end{itemize}

\paragraph*{Scalability}\label{docs__pandoc__latex__src__command_line_interface_selection__appendix_b_powershell_integration.md__scalability}

\begin{itemize}
\tightlist
\item
  {[}YES{]} 1 design or 1,000 designs = same time commitment
\item
  {[}YES{]} Run overnight for batch operations
\item
  {[}YES{]} Easy to expand with new features
\end{itemize}

\paragraph*{Accessibility}\label{docs__pandoc__latex__src__command_line_interface_selection__appendix_b_powershell_integration.md__accessibility}

\begin{itemize}
\tightlist
\item
  {[}YES{]} All interaction is text-based (screen reader friendly)
\item
  {[}YES{]} Results logged in CSV (machine-readable, sortable)
\item
  {[}YES{]} Repeatable (can verify steps later)
\end{itemize}

\subsubsection*{Next Steps}\label{docs__pandoc__latex__src__command_line_interface_selection__appendix_b_powershell_integration.md__next-steps}

\begin{enumerate}
\tightlist
\item
  Try the basic build script (Script 1) with one of your SCAD files
\item
  Add logging (Script 3) to track what you create
\item
  Test parametric sweeps (Script 2) with design variations
\item
  Integrate full workflow (Script 5) for complete automation
\item
  Customize config.ps1 for your specific setup
\end{enumerate}

Remember: Start simple, test each script individually, then combine them into larger workflows. PowerShell is most powerful when built incrementally!

\subsection{Appendix C: Git Bash / POSIX Shell Integration for SCAD Workflows}\label{docs__pandoc__latex__src__command_line_interface_selection__appendix_c_gitbash_integration.md__3dmake_foundation-appendix_g_gitbash_integration}

This appendix mirrors Appendix D (PowerShell) but provides examples for Git Bash (mingw) or other POSIX-compatible shells on Windows, macOS, and Linux. Use these scripts when you prefer shell utilities (\texttt{bash}, \texttt{sed}, \texttt{awk}, \texttt{curl}) and Unix-like tooling.

\subsubsection*{Overview: Why Git Bash / POSIX Shell?}\label{docs__pandoc__latex__src__command_line_interface_selection__appendix_c_gitbash_integration.md__overview-why-git-bash--posix-shell}

\begin{itemize}
\tightlist
\item
  Leverage standard Unix tools for text processing and automation.
\item
  Cross-platform: same scripts often work on Linux/macOS and Windows via Git Bash.
\item
  Good fit for containerized CI like GitHub Actions.
\end{itemize}

\subsubsection*{Prerequisites \& Setup}\label{docs__pandoc__latex__src__command_line_interface_selection__appendix_c_gitbash_integration.md__prerequisites--setup}

\paragraph*{Required Software}\label{docs__pandoc__latex__src__command_line_interface_selection__appendix_c_gitbash_integration.md__required-software}

\begin{lstlisting}[style=Alabaster, language=bash]
command -v openscad   # OpenSCAD
command -v prusa-slicer # PrusaSlicer (or use CLI slicer)
command -v python3     # Python (optional)
command -v curl        # curl for API calls

\end{lstlisting}

\paragraph*{Directory Structure (POSIX-style)}\label{docs__pandoc__latex__src__command_line_interface_selection__appendix_c_gitbash_integration.md__directory-structure-posix-style}

\begin{lstlisting}[style=Alabaster]
~/projects/3dMake/
|- src/
|- stl/
|- gcode/
|- logs/
`- scripts/
  |- build.sh
  `- batch_build.sh

\end{lstlisting}

\subsubsection*{Basic Workflow: Single-file build (bash)}\label{docs__pandoc__latex__src__command_line_interface_selection__appendix_c_gitbash_integration.md__basic-workflow-single-file-build-bash}

\texttt{scripts/build.sh}:

\begin{lstlisting}[style=Alabaster, language=bash]
#!/usr/bin/env bash
set -euo pipefail
ROOT_DIR="$(cd "$(dirname "$0")/.." && pwd)"
SCAD_FILE="$1"
SCAD_PATH="$ROOT_DIR/src/$SCAD_FILE"
STL_PATH="$ROOT_DIR/stl/${SCAD_FILE%.*}.stl"
GCODE_PATH="$ROOT_DIR/gcode/${SCAD_FILE%.*}.gcode"

echo "Exporting SCAD -> STL: $SCAD_FILE"
"/c/Program Files/OpenSCAD/openscad.exe" -o "$STL_PATH" "$SCAD_PATH" || { echo "STL export failed"; exit 1; }

echo "Slicing STL -> G-code"
"/c/Program Files/Prusa3D/PrusaSlicer/prusa-slicer.exe" --load-config-file "$ROOT_DIR/scripts/default.ini" -o "$GCODE_PATH" "$STL_PATH" || { echo "Slicing failed"; exit 1; }

echo "BUILD COMPLETE: $GCODE_PATH"

\end{lstlisting}

Notes:

\begin{itemize}
\tightlist
\item
  Paths to GUI apps on Windows may need the \texttt{/c/Program\textbackslash{}\ Files/...} form under Git Bash, or call the native \texttt{.exe} with full quoted path.
\item
  On Linux/macOS adjust executable paths accordingly.
\end{itemize}

\subsubsection*{Batch Build (directory)}\label{docs__pandoc__latex__src__command_line_interface_selection__appendix_c_gitbash_integration.md__batch-build-directory}

\texttt{scripts/batch\_build.sh}:

\begin{lstlisting}[style=Alabaster, language=bash]
#!/usr/bin/env bash
set -euo pipefail
ROOT_DIR="$(cd "$(dirname "$0")/.." && pwd)"
find "$ROOT_DIR/src" -name "*.scad" -print0 | while IFS= read -r -d '' file; do
  fname="$(basename "$file")"
  echo "Building $fname"
  "$ROOT_DIR/scripts/build.sh" "$fname"
done

\end{lstlisting}

\subsubsection*{Parametric Sweeps}\label{docs__pandoc__latex__src__command_line_interface_selection__appendix_c_gitbash_integration.md__parametric-sweeps}

Use \texttt{sed} or \texttt{python} to template parameter changes, then call \texttt{build.sh} for each variant. Example (bash + python hybrid):

\begin{lstlisting}[style=Alabaster, language=bash]
python3 scripts/gen_variants.py "$ROOT_DIR/src/bracelet_holder.scad" --out variants/
for v in variants/*.scad; do
  ./scripts/build.sh "$(basename "$v")"
done

\end{lstlisting}

\subsubsection*{Logging \& CSV}\label{docs__pandoc__latex__src__command_line_interface_selection__appendix_c_gitbash_integration.md__logging--csv}

Append simple CSV lines using \texttt{printf}:

\begin{lstlisting}[style=Alabaster, language=bash]
printf "%s,%s,Success\n" "$(date +'%Y-%m-%d %H:%M:%S')" "${SCAD_FILE%.*}" >> "$ROOT_DIR/logs/build_history.csv"

\end{lstlisting}

\subsubsection*{Send to Printer (USB)}\label{docs__pandoc__latex__src__command_line_interface_selection__appendix_c_gitbash_integration.md__send-to-printer-usb}

\begin{lstlisting}[style=Alabaster, language=bash]
# Copy to mounted USB drive (example mount point /media/usb)
cp "$GCODE_PATH" "/media/usb/"

\end{lstlisting}

\subsubsection*{Monitoring (network printers)}\label{docs__pandoc__latex__src__command_line_interface_selection__appendix_c_gitbash_integration.md__monitoring-network-printers}

Use \texttt{curl} to call a printer API (e.g., OctoPrint):

\begin{lstlisting}[style=Alabaster, language=bash]
curl -s "[https://example.com](https://example.com) -H "X-Api-Key: $API_KEY" | jq .

\end{lstlisting}

\subsubsection*{Best Practices}\label{docs__pandoc__latex__src__command_line_interface_selection__appendix_c_gitbash_integration.md__best-practices}

\begin{itemize}
\tightlist
\item
  Use \texttt{set\ -euo\ pipefail} for safer scripts.
\item
  Use \texttt{mktemp} or a \texttt{variants/} directory for generated files.
\item
  Keep heavy text processing to \texttt{python} if logic becomes complex.
\item
  Make scripts executable (\texttt{chmod\ +x\ scripts/*.sh}).
\end{itemize}

\begin{center}\rule{0.5\linewidth}{0.5pt}\end{center}

\section{Appendices}\label{docs__pandoc__latex__src__appendices__3dmake_appendices.md__appendices-3dmake-appendices}

Comprehensive reference materials, guides, and supplemental resources for the OpenSCAD and 3dMake curriculum.

\subsection*{3dMake Foundation Appendices}\label{docs__pandoc__latex__src__appendices__3dmake_appendices.md__3dmake-foundation-appendices}

\begin{itemize}
\tightlist
\item
  \hyperref[docs__pandoc__latex__src__3dmake_foundation__appendix_a_comprehensive_slicing_guide.md__3dmake_foundation-appendix_a_comprehensive_slicing_guide]{Appendix A: Comprehensive Slicing Guide - All Major Slicers} - Complete reference for PrusaSlicer, Bambu Studio, Cura, and OrcaSlicer configuration
\item
  \hyperref[docs__pandoc__latex__src__3dmake_foundation__appendix_b_material_properties.md__3dmake_foundation-appendix_b_material_properties]{Appendix B: Material Properties \& Selection Guide} - Detailed material reference including shrinkage data, print settings, and properties
\item
  \hyperref[docs__pandoc__latex__src__3dmake_foundation__appendix_c_tolerance_qa.md__3dmake_foundation-appendix_c_tolerance_qa]{Appendix C: Tolerance Testing \& Quality Assurance Matrix} - Comprehensive QA procedures and tolerance validation methods
\item
  \href{https://github.com/mrhunsaker/VI_3DMake_OpenSCAD_Curriculum/Appendices/../3dMake_Foundation/Appendix_D_PowerShell_Integration.md}{Appendix D: PowerShell Integration for SCAD Workflows} - Batch processing, automation scripts, and advanced workflow integration
\end{itemize}

\subsection{Appendix A: Comprehensive Slicing Guide - All Major Slicers}\label{docs__pandoc__latex__src__3dmake_foundation__appendix_a_comprehensive_slicing_guide.md__3dmake_foundation-appendix_a_comprehensive_slicing_guide}

This appendix covers settings, workflows, and troubleshooting for 7 major 3D printer slicers. Each slicer is represented with:

\begin{itemize}
\tightlist
\item
  Recommended settings for beginners
\item
  Screen-reader-accessible settings explanation
\item
  Common troubleshooting
\item
  Accessibility features built-in
\item
  Command-line usage (for PowerShell integration)
\end{itemize}

Referenced in: Lessons 5 (Safety), 8 (Design), 10 (Verification)

\subsubsection*{Overview: What is Slicing?}\label{docs__pandoc__latex__src__3dmake_foundation__appendix_a_comprehensive_slicing_guide.md__overview-what-is-slicing}

Slicing converts a 3D model into printer instructions:

\begin{lstlisting}[style=Alabaster]
SCAD Design (bracelet_holder.scad)
   v
Export to STL (3D shape file)
   v
Load into Slicer
   v
Apply settings (temperature, speed, supports, etc.)
   v
Generate G-code (printer instructions)
   v
Send to Printer
   v
Physical part

\end{lstlisting}

\paragraph*{Core Slicing Parameters (All Slicers Share These)}\label{docs__pandoc__latex__src__3dmake_foundation__appendix_a_comprehensive_slicing_guide.md__core-slicing-parameters-all-slicers-share-these}

{\def\LTcaptype{none} % do not increment counter
\begin{longtable}[]{@{}
  >{\raggedright\arraybackslash}p{(\linewidth - 6\tabcolsep) * \real{0.1429}}
  >{\raggedright\arraybackslash}p{(\linewidth - 6\tabcolsep) * \real{0.2449}}
  >{\raggedright\arraybackslash}p{(\linewidth - 6\tabcolsep) * \real{0.1735}}
  >{\raggedright\arraybackslash}p{(\linewidth - 6\tabcolsep) * \real{0.4388}}@{}}
\toprule\noalign{}
\begin{minipage}[b]{\linewidth}\raggedright
Parameter
\end{minipage} & \begin{minipage}[b]{\linewidth}\raggedright
What It Does
\end{minipage} & \begin{minipage}[b]{\linewidth}\raggedright
Typical Range
\end{minipage} & \begin{minipage}[b]{\linewidth}\raggedright
Impact
\end{minipage} \\
\midrule\noalign{}
\endhead
\bottomrule\noalign{}
\endlastfoot
Nozzle Temp & Filament melting heat & 200-250C &
Too cold -\textgreater{} weak; too hot -\textgreater{} oozing \\
Bed Temp & Build plate heat & 50-110C &
Helps adhesion; prevents warping \\
Layer Height & Z-axis precision & 0.1-0.4mm &
Finer = slower, better detail \\
Print Speed & Movement velocity & 30-150 mm/s &
Faster = worse quality; slower = stronger \\
Infill \% & Interior density & 10-100\% & Higher = stronger + heavier \\
Support & Temporary scaffolding & On/Off &
Required for overhangs \textgreater45 \\
Bed Adhesion & First layer stickiness & Brim/Raft/Skirt &
Prevents parts lifting mid-print \\
\end{longtable}
}

\subsubsection*{1. PrusaSlicer (Prusa)}\label{docs__pandoc__latex__src__3dmake_foundation__appendix_a_comprehensive_slicing_guide.md__1-prusaslicer-prusa}

\paragraph*{Overview}\label{docs__pandoc__latex__src__3dmake_foundation__appendix_a_comprehensive_slicing_guide.md__overview}

\begin{itemize}
\tightlist
\item
  Developer: Prusa Research (Czech company)
\item
  Platforms: Windows, Mac, Linux (open-source)
\item
  Best For: Beginner-friendly, excellent support, strong community
\item
  Download: \url{https://www.prusa3d.com/page/prusaslicer_410/}
\item
  Accessibility: Good font sizes, text-based profiles
\end{itemize}

\paragraph*{Quick Setup for Beginners}\label{docs__pandoc__latex__src__3dmake_foundation__appendix_a_comprehensive_slicing_guide.md__quick-setup-for-beginners}

Step 1: Install \& Select Your Printer

\begin{lstlisting}[style=Alabaster]
1. Open PrusaSlicer
2. Go to "Help" -> "Check for Updates"
3. When prompted, select your printer model
4. Choose default profile (matches printer exactly)

\end{lstlisting}

Step 2: Load Your STL

\begin{lstlisting}[style=Alabaster]
1. Click "File" -> "Open STL Model"
2. Select your bracelet_holder.stl
3. Model appears in 3D view

\end{lstlisting}

Step 3: Essential Settings

These are the most important adjustments:

{\def\LTcaptype{none} % do not increment counter
\begin{longtable}[]{@{}
  >{\raggedright\arraybackslash}p{(\linewidth - 6\tabcolsep) * \real{0.1772}}
  >{\raggedright\arraybackslash}p{(\linewidth - 6\tabcolsep) * \real{0.2405}}
  >{\raggedright\arraybackslash}p{(\linewidth - 6\tabcolsep) * \real{0.2152}}
  >{\raggedright\arraybackslash}p{(\linewidth - 6\tabcolsep) * \real{0.3671}}@{}}
\toprule\noalign{}
\begin{minipage}[b]{\linewidth}\raggedright
Setting
\end{minipage} & \begin{minipage}[b]{\linewidth}\raggedright
Location
\end{minipage} & \begin{minipage}[b]{\linewidth}\raggedright
Beginner Value
\end{minipage} & \begin{minipage}[b]{\linewidth}\raggedright
Why
\end{minipage} \\
\midrule\noalign{}
\endhead
\bottomrule\noalign{}
\endlastfoot
Layer Height & Print Settings & 0.15mm & Balance speed \& quality \\
Infill & Print Settings & 20\% & Enough strength; fast print \\
Support & Print Settings & Yes (if needed) &
For overhangs \textgreater45 \\
Nozzle Temp & Filament Settings & 210C & Default for PLA \\
Bed Temp & Filament Settings & 60C & Standard PLA adhesion \\
Print Speed & Print Settings & 150 mm/s & Balanced quality \\
\end{longtable}
}

Step 4: Preview \& Export

\begin{lstlisting}[style=Alabaster]
1. Click "Slice now" (or G-code icon)
2. Left panel shows preview of each layer
3. Look for issues:
   - Supports covering entire model? (OK)
   - Model floating in air? (Not OK-likely error)
   - Top surface quality acceptable?
4. If satisfied, click "Export G-code"
5. Save to USB or send to printer

\end{lstlisting}

\paragraph*{Accessible Parameter Explanations}\label{docs__pandoc__latex__src__3dmake_foundation__appendix_a_comprehensive_slicing_guide.md__accessible-parameter-explanations}

When adjusting settings, use these descriptions to understand what each does:

Layer Height (0.1-0.4mm)

\begin{itemize}
\tightlist
\item
  Lower (0.1mm): Smoother surface, more layers, slower (best for detail)
\item
  Higher (0.4mm): Faster, rougher surface (best for speed)
\item
  Recommendation: 0.15mm for balanced quality
\end{itemize}

Infill Percentage (10-100\%)

\begin{itemize}
\tightlist
\item
  10\%: Fast, uses less filament, weaker (good for prototypes)
\item
  20\%: Good balance (recommended for most prints)
\item
  50\%: Stronger, slower, heavier
\item
  100\%: Solid interior, strongest, slowest (waste of plastic)
\end{itemize}

Support Type

\begin{itemize}
\tightlist
\item
  None: Fast, good surface finish, but risky if overhangs exist
\item
  Linear (Default): Good balance-easy to remove, provides support
\item
  Grid: Extra strong support, takes longer to remove
\end{itemize}

First Layer

\begin{itemize}
\tightlist
\item
  Brim: Adds border to help adhesion (recommended for beginners)
\item
  Raft: Sacrificial platform (good if bed isn\textquotesingle t level)
\item
  Skirt: Just outline, doesn\textquotesingle t help adhesion (fastest)
\end{itemize}

\paragraph*{Command-Line Usage (PowerShell Integration)}\label{docs__pandoc__latex__src__3dmake_foundation__appendix_a_comprehensive_slicing_guide.md__command-line-usage-powershell-integration}

\begin{lstlisting}[style=Alabaster, language=powershell]
# Slice a model automatically with PrusaSlicer
$model = "C:\Models\bracelet_holder.stl"
$output = "C:\GCode\bracelet_holder.gcode"
$config = "default"  # Use default printer profile
# Run PrusaSlicer in batch mode
& "C:\Program Files\Prusa3D\PrusaSlicer\prusa-slicer.exe" `
    --load-config-file "$config" `
    --export-gcode "$output" `
    "$model"
Write-Host "Slicing complete: $output"

\end{lstlisting}

\paragraph*{Troubleshooting}\label{docs__pandoc__latex__src__3dmake_foundation__appendix_a_comprehensive_slicing_guide.md__troubleshooting}

{\def\LTcaptype{none} % do not increment counter
\begin{longtable}[]{@{}
  >{\raggedright\arraybackslash}p{(\linewidth - 4\tabcolsep) * \real{0.3061}}
  >{\raggedright\arraybackslash}p{(\linewidth - 4\tabcolsep) * \real{0.2653}}
  >{\raggedright\arraybackslash}p{(\linewidth - 4\tabcolsep) * \real{0.4286}}@{}}
\toprule\noalign{}
\begin{minipage}[b]{\linewidth}\raggedright
Problem
\end{minipage} & \begin{minipage}[b]{\linewidth}\raggedright
Cause
\end{minipage} & \begin{minipage}[b]{\linewidth}\raggedright
Solution
\end{minipage} \\
\midrule\noalign{}
\endhead
\bottomrule\noalign{}
\endlastfoot
First layer not sticking & Bed not level &
Bed leveling procedure in printer manual \\
Supports everywhere & No support type selected &
Change to "Linear" or "Grid" \\
Nozzle drags through model & Z-offset too low & Raise Z-offset +0.1mm \\
Oozing strings between parts & Temp too high &
Lower nozzle temp 5-10C \\
Print breaks mid-way & Adhesion problem & Add brim; check bed level \\
\end{longtable}
}

\subsubsection*{2. Bambu Studio (Bambu Lab)}\label{docs__pandoc__latex__src__3dmake_foundation__appendix_a_comprehensive_slicing_guide.md__2-bambu-studio-bambu-lab}

\paragraph*{Overview}\label{docs__pandoc__latex__src__3dmake_foundation__appendix_a_comprehensive_slicing_guide.md__overview-1}

\begin{itemize}
\tightlist
\item
  Developer: Bambu Lab (printer manufacturer)
\item
  Platforms: Windows, Mac, Linux
\item
  Best For: Modern X1-series printers, excellent speed, AMS support
\item
  Download: \url{https://bambulab.com/en/download/studio}
\item
  Accessibility: Good contrast, keyboard navigation
\end{itemize}

\paragraph*{Quick Setup}\label{docs__pandoc__latex__src__3dmake_foundation__appendix_a_comprehensive_slicing_guide.md__quick-setup}

Step 1: Create Account \& Connect Printer

\begin{enumerate}
\tightlist
\item
  Launch Bambu Studio
\item
  Sign in with Bambu Lab account
\item
  Select printer from network
\item
  Studio auto-detects printer settings
\end{enumerate}

Step 2: Load Model \& Configure

\begin{enumerate}
\tightlist
\item
  Drag STL into workspace
\item
  Auto-arranges on bed plate
\item
  Default profile applied automatically
\end{enumerate}

Step 3: Key Settings (Bambu-Specific)

{\def\LTcaptype{none} % do not increment counter
\begin{longtable}[]{@{}
  >{\raggedright\arraybackslash}p{(\linewidth - 6\tabcolsep) * \real{0.2179}}
  >{\raggedright\arraybackslash}p{(\linewidth - 6\tabcolsep) * \real{0.1154}}
  >{\raggedright\arraybackslash}p{(\linewidth - 6\tabcolsep) * \real{0.2949}}
  >{\raggedright\arraybackslash}p{(\linewidth - 6\tabcolsep) * \real{0.3718}}@{}}
\toprule\noalign{}
\begin{minipage}[b]{\linewidth}\raggedright
Setting
\end{minipage} & \begin{minipage}[b]{\linewidth}\raggedright
Default
\end{minipage} & \begin{minipage}[b]{\linewidth}\raggedright
Adjustment
\end{minipage} & \begin{minipage}[b]{\linewidth}\raggedright
Why
\end{minipage} \\
\midrule\noalign{}
\endhead
\bottomrule\noalign{}
\endlastfoot
AMS Multi-Color & Off & On (if AMS attached) & Auto-switch filament \\
Auto-Leveling & Enabled & Keep On & Bambu feature-very reliable \\
Nozzle Temp & 220C & Keep unless specified & Bambu-optimized \\
Layer Height & 0.2mm & 0.15mm for detail & Balance speed/quality \\
Bed Temp & 65C & 60C for PLA & Standard adhesion \\
\end{longtable}
}

Step 4: Send to Printer

\begin{enumerate}
\tightlist
\item
  Click "Prepare" (bottom right)
\item
  Review preview
\item
  Click "Send to Device"
\item
  Printer receives over WiFi
\item
  Start print from printer screen
\end{enumerate}

\paragraph*{Accessible Parameter Explanations}\label{docs__pandoc__latex__src__3dmake_foundation__appendix_a_comprehensive_slicing_guide.md__accessible-parameter-explanations-1}

Auto-Leveling

\begin{itemize}
\tightlist
\item
  Bambu printers automatically level the nozzle before every print
\item
  Explanation: Saves manual calibration; extremely reliable
\item
  For VI users: Provides confidence that bed is properly prepared
\end{itemize}

Filament Calibration

\begin{itemize}
\tightlist
\item
  Before first use of new filament color, run "Filament Calibration"
\item
  This optimizes temperature and speed for that specific filament
\item
  Explanation: Ensures consistent color and strength
\end{itemize}

Multi-Material (AMS)

\begin{itemize}
\tightlist
\item
  If you have Auto Material System (AMS):

  \begin{itemize}
  \tightlist
  \item
    Load up to 4 filament colors
  \item
    Studio auto-switches during print
  \item
    Explanation: Multi-color prints without manual intervention
  \end{itemize}
\end{itemize}

\paragraph*{Command-Line Usage (PowerShell)}\label{docs__pandoc__latex__src__3dmake_foundation__appendix_a_comprehensive_slicing_guide.md__command-line-usage-powershell}

\begin{lstlisting}[style=Alabaster, language=powershell]
# Slice with Bambu Studio (command-line interface)
$model = "C:\Models\bracelet_holder.stl"
$output = "C:\GCode\bracelet_holder.3mf"  # Bambu uses .3mf format
# Bambu Studio CLI
& "C:\Program Files\BambuStudio\bambu-studio.exe" `
    --output "$output" `
    "$model"
Write-Host "Slice saved: $output"
# Send to printer directly
# (Requires API key-see Bambu documentation)

\end{lstlisting}

\paragraph*{Troubleshooting}\label{docs__pandoc__latex__src__3dmake_foundation__appendix_a_comprehensive_slicing_guide.md__troubleshooting-1}

{\def\LTcaptype{none} % do not increment counter
\begin{longtable}[]{@{}
  >{\raggedright\arraybackslash}p{(\linewidth - 4\tabcolsep) * \real{0.3407}}
  >{\raggedright\arraybackslash}p{(\linewidth - 4\tabcolsep) * \real{0.2527}}
  >{\raggedright\arraybackslash}p{(\linewidth - 4\tabcolsep) * \real{0.4066}}@{}}
\toprule\noalign{}
\begin{minipage}[b]{\linewidth}\raggedright
Problem
\end{minipage} & \begin{minipage}[b]{\linewidth}\raggedright
Cause
\end{minipage} & \begin{minipage}[b]{\linewidth}\raggedright
Solution
\end{minipage} \\
\midrule\noalign{}
\endhead
\bottomrule\noalign{}
\endlastfoot
WiFi not connecting & Network issue &
Restart printer WiFi; check SSID \\
AMS not switching & Filament not detected &
Load filament into AMS; recalibrate \\
Print quality inconsistent & Wrong filament type &
Run filament calibration \\
Nozzle crashes on first layer & Auto-level failed &
Manually check nozzle height \\
\end{longtable}
}

\subsubsection*{3. Cura (Ultimaker)}\label{docs__pandoc__latex__src__3dmake_foundation__appendix_a_comprehensive_slicing_guide.md__3-cura-ultimaker}

\paragraph*{Overview}\label{docs__pandoc__latex__src__3dmake_foundation__appendix_a_comprehensive_slicing_guide.md__overview-2}

\begin{itemize}
\tightlist
\item
  Developer: Ultimaker (Dutch company, open-source)
\item
  Platforms: Windows, Mac, Linux
\item
  Best For: Broad printer support, user-friendly, good documentation
\item
  Download: \url{https://ultimaker.com/software/ultimaker-cura}
\item
  Accessibility: Clear UI, good contrast
\end{itemize}

\paragraph*{Quick Setup}\label{docs__pandoc__latex__src__3dmake_foundation__appendix_a_comprehensive_slicing_guide.md__quick-setup-1}

Step 1: Add Your Printer

\begin{enumerate}
\tightlist
\item
  Launch Cura
\item
  Go to "Settings" (top-right)
\item
  Click "Printers" -\textgreater{} "Add Printer"
\item
  Select your printer model from list
\item
  Confirm network connection
\end{enumerate}

Step 2: Load \& Prepare Model

\begin{enumerate}
\tightlist
\item
  Drag STL into workspace
\item
  Model auto-scales if needed (confirm size)
\item
  Right-click -\textgreater{} "Support" (if overhangs need support)
\end{enumerate}

Step 3: Recommended Settings

{\def\LTcaptype{none} % do not increment counter
\begin{longtable}[]{@{}
  >{\raggedright\arraybackslash}p{(\linewidth - 4\tabcolsep) * \real{0.3143}}
  >{\raggedright\arraybackslash}p{(\linewidth - 4\tabcolsep) * \real{0.1429}}
  >{\raggedright\arraybackslash}p{(\linewidth - 4\tabcolsep) * \real{0.5429}}@{}}
\toprule\noalign{}
\begin{minipage}[b]{\linewidth}\raggedright
Setting
\end{minipage} & \begin{minipage}[b]{\linewidth}\raggedright
Value
\end{minipage} & \begin{minipage}[b]{\linewidth}\raggedright
Notes
\end{minipage} \\
\midrule\noalign{}
\endhead
\bottomrule\noalign{}
\endlastfoot
Profile & Standard & Good balance for most prints \\
Layer Height & 0.2mm & Default; change to 0.15mm for detail \\
Infill & 20\% & 100\% waste for solid parts \\
Support Angle & 50 & Auto-generates support for overhangs \\
Build Plate Adhesion & Brim & Helps first layer stick \\
Nozzle Temp & 200C & Standard PLA \\
Bed Temp & 60C & Standard PLA \\
\end{longtable}
}

Step 4: Print

\begin{enumerate}
\tightlist
\item
  Click "Slice" (bottom right)
\item
  Review layer-by-layer preview
\item
  Click "Print Over Network" or "Print to File"
\item
  Model sends to printer or saves as .gcode
\end{enumerate}

\paragraph*{Accessible Settings Explanation}\label{docs__pandoc__latex__src__3dmake_foundation__appendix_a_comprehensive_slicing_guide.md__accessible-settings-explanation}

Combing Mode

\begin{itemize}
\tightlist
\item
  Off: Nozzle retracts on every travel (quality)
\item
  All: Never retracts (faster, possible stringing)
\item
  Not in Skin: Smart compromise
\item
  For VI users: Retraction prevents nozzle oozing on visible surfaces
\end{itemize}

Z-Offset (Z Clearance)

\begin{itemize}
\tightlist
\item
  Adjusts first-layer distance
\item
  Too low -\textgreater{} nozzle scrapes bed (bad)
\item
  Too high -\textgreater{} filament doesn\textquotesingle t stick (bad)
\item
  Correct -\textgreater{} thin line of plastic sticks to bed
\end{itemize}

Gradual Infill

\begin{itemize}
\tightlist
\item
  Automatically reduces infill strength away from surface
\item
  Saves filament while maintaining strength
\item
  Explanation: The core doesn\textquotesingle t need to be solid
\end{itemize}

\paragraph*{Command-Line Usage (PowerShell)}\label{docs__pandoc__latex__src__3dmake_foundation__appendix_a_comprehensive_slicing_guide.md__command-line-usage-powershell-1}

\begin{lstlisting}[style=Alabaster, language=powershell]
# Cura engine CLI (CuraEngine)
$model = "C:\Models\bracelet_holder.stl"
$output = "C:\GCode\bracelet_holder.gcode"
$config = "default.cfg"
# Requires Cura to be installed; command-line slicing
& "C:\Program Files\Ultimaker Cura\CuraEngine.exe" `
    -c "$config" `
    -o "$output" `
    "$model"
Write-Host "Sliced: $output"

\end{lstlisting}

\paragraph*{Troubleshooting}\label{docs__pandoc__latex__src__3dmake_foundation__appendix_a_comprehensive_slicing_guide.md__troubleshooting-2}

{\def\LTcaptype{none} % do not increment counter
\begin{longtable}[]{@{}
  >{\raggedright\arraybackslash}p{(\linewidth - 4\tabcolsep) * \real{0.3765}}
  >{\raggedright\arraybackslash}p{(\linewidth - 4\tabcolsep) * \real{0.2471}}
  >{\raggedright\arraybackslash}p{(\linewidth - 4\tabcolsep) * \real{0.3765}}@{}}
\toprule\noalign{}
\begin{minipage}[b]{\linewidth}\raggedright
Problem
\end{minipage} & \begin{minipage}[b]{\linewidth}\raggedright
Cause
\end{minipage} & \begin{minipage}[b]{\linewidth}\raggedright
Solution
\end{minipage} \\
\midrule\noalign{}
\endhead
\bottomrule\noalign{}
\endlastfoot
Model appears too small on bed & Scale wrong &
Right-click -\textgreater{} Scale to fit \\
Stringing between parts & Retraction disabled &
Enable retraction in settings \\
Support doesn\textquotesingle t generate & Auto-support off &
Enable "Generate Support" \\
Printer not found & Network/USB issue &
Check connection; restart Cura \\
\end{longtable}
}

\subsubsection*{4. SuperSlicer (Modification of Prusa)}\label{docs__pandoc__latex__src__3dmake_foundation__appendix_a_comprehensive_slicing_guide.md__4-superslicer-modification-of-prusa}

\paragraph*{Overview}\label{docs__pandoc__latex__src__3dmake_foundation__appendix_a_comprehensive_slicing_guide.md__overview-3}

\begin{itemize}
\tightlist
\item
  Developer: Community fork of PrusaSlicer
\item
  Platforms: Windows, Mac, Linux
\item
  Best For: Advanced users wanting more control than Prusa offers
\item
  Download: \url{https://github.com/supermerill/SuperSlicer}
\item
  Accessibility: Similar to Prusa, more advanced options
\end{itemize}

\paragraph*{Key Differences from Prusa}\label{docs__pandoc__latex__src__3dmake_foundation__appendix_a_comprehensive_slicing_guide.md__key-differences-from-prusa}

{\def\LTcaptype{none} % do not increment counter
\begin{longtable}[]{@{}lll@{}}
\toprule\noalign{}
Feature & PrusaSlicer & SuperSlicer \\
\midrule\noalign{}
\endhead
\bottomrule\noalign{}
\endlastfoot
Arachne Engine & No & Yes-better edges \\
Seam Position & Limited options & Full control \\
Pressure Equalization & No & Yes-better bridging \\
Stealth Mode & No & Yes-quieter/higher quality \\
\end{longtable}
}

\paragraph*{When to Use SuperSlicer}\label{docs__pandoc__latex__src__3dmake_foundation__appendix_a_comprehensive_slicing_guide.md__when-to-use-superslicer}

\begin{itemize}
\tightlist
\item
  Printing challenging geometries with tight tolerances
\item
  Need advanced surface finish control
\item
  Familiar with PrusaSlicer already and want more power
\end{itemize}

\paragraph*{Quick Start}\label{docs__pandoc__latex__src__3dmake_foundation__appendix_a_comprehensive_slicing_guide.md__quick-start}

\begin{enumerate}
\tightlist
\item
  Download SuperSlicer from GitHub
\item
  Export profile from PrusaSlicer (if you have it)
\item
  Import into SuperSlicer
\item
  All settings are compatible with PrusaSlicer
\end{enumerate}

\paragraph*{Advanced Settings for SuperSlicer}\label{docs__pandoc__latex__src__3dmake_foundation__appendix_a_comprehensive_slicing_guide.md__advanced-settings-for-superslicer}

Arachne Engine

\begin{itemize}
\tightlist
\item
  Enables finer edges on walls
\item
  Results in cleaner, more accurate prints
\item
  Takes slightly longer to slice but worth it
\end{itemize}

Seam Positioning

\begin{itemize}
\tightlist
\item
  Random: Hides seams (good for aesthetic)
\item
  Aligned: Consistent location (good for debugging)
\item
  Rear: Always at back (recommended)
\end{itemize}

Pressure Equalization

\begin{itemize}
\tightlist
\item
  Helps with bridging (printing across gaps)
\item
  Reduces sagging on overhangs
\item
  Recommended: Enable for complex designs
\end{itemize}

\subsubsection*{5. OrcaSlicer (Modern Bamboo Alternative)}\label{docs__pandoc__latex__src__3dmake_foundation__appendix_a_comprehensive_slicing_guide.md__5-orcaslicer-modern-bamboo-alternative}

\paragraph*{Overview}\label{docs__pandoc__latex__src__3dmake_foundation__appendix_a_comprehensive_slicing_guide.md__overview-4}

\begin{itemize}
\tightlist
\item
  Developer: Community (independent open-source)
\item
  Platforms: Windows, Mac, Linux
\item
  Best For: Users who want Bambu features without Bambu printer
\item
  Download: \url{https://github.com/SoftFever/OrcaSlicer}
\item
  Accessibility: Modern UI, good keyboard support
\end{itemize}

\paragraph*{Why Orca?}\label{docs__pandoc__latex__src__3dmake_foundation__appendix_a_comprehensive_slicing_guide.md__why-orca}

OrcaSlicer brings Bambu Studio\textquotesingle s best features to any printer:

\begin{itemize}
\tightlist
\item
  Excellent defaults
\item
  Fast slicing
\item
  Good preview
\item
  Modern interface
\end{itemize}

\paragraph*{Quick Setup}\label{docs__pandoc__latex__src__3dmake_foundation__appendix_a_comprehensive_slicing_guide.md__quick-setup-2}

\begin{enumerate}
\tightlist
\item
  Download OrcaSlicer
\item
  Select your printer (not just Bambu models)
\item
  Load STL
\item
  Uses good defaults-usually ready to print
\end{enumerate}

\paragraph*{Key Settings}\label{docs__pandoc__latex__src__3dmake_foundation__appendix_a_comprehensive_slicing_guide.md__key-settings}

{\def\LTcaptype{none} % do not increment counter
\begin{longtable}[]{@{}lll@{}}
\toprule\noalign{}
Setting & Value & Why \\
\midrule\noalign{}
\endhead
\bottomrule\noalign{}
\endlastfoot
Wall Loops & 2 & Strong walls, visible detail \\
Internal Solid Layer & 4 & Strength for brackets/connectors \\
Infill Pattern & Grid & Balanced strength \\
Line Width & Auto & Matches nozzle diameter \\
Speed & 80 mm/s & Balanced quality/speed \\
\end{longtable}
}

\paragraph*{Accessible Features}\label{docs__pandoc__latex__src__3dmake_foundation__appendix_a_comprehensive_slicing_guide.md__accessible-features}

Filament Manager

\begin{itemize}
\tightlist
\item
  Track filament type, color, weight used
\item
  Explanation: Know when to buy new filament
\end{itemize}

Print Time Estimation

\begin{itemize}
\tightlist
\item
  Accurate prediction of print duration
\item
  Updates during slicing
\end{itemize}

Material Presets

\begin{itemize}
\tightlist
\item
  Pre-configured settings for common filaments
\item
  Just select material type; settings auto-apply
\end{itemize}

\subsubsection*{6. IdeaMaker (Raise3D)}\label{docs__pandoc__latex__src__3dmake_foundation__appendix_a_comprehensive_slicing_guide.md__6-ideamaker-raise3d}

\paragraph*{Overview}\label{docs__pandoc__latex__src__3dmake_foundation__appendix_a_comprehensive_slicing_guide.md__overview-5}

\begin{itemize}
\tightlist
\item
  Developer: Raise3D
\item
  Platforms: Windows, Mac, Linux
\item
  Best For: Raise3D printer users; advanced features
\item
  Download: \url{https://www.raise3d.com/ideamaker}
\item
  Accessibility: Professional UI, detailed settings
\end{itemize}

\paragraph*{When to Use}\label{docs__pandoc__latex__src__3dmake_foundation__appendix_a_comprehensive_slicing_guide.md__when-to-use}

\begin{itemize}
\tightlist
\item
  Using a Raise3D printer (excellent multi-nozzle support)
\item
  Need industrial-strength slicing
\item
  Want dual-extrusion (2-color) printing
\end{itemize}

\paragraph*{Quick Start}\label{docs__pandoc__latex__src__3dmake_foundation__appendix_a_comprehensive_slicing_guide.md__quick-start-1}

\begin{enumerate}
\tightlist
\item
  Launch IdeaMaker
\item
  Add printer (Raise3D models have built-in profiles)
\item
  Load STL
\item
  Adjust layer height and infill
\item
  Slice and send to printer
\end{enumerate}

\paragraph*{Key Features}\label{docs__pandoc__latex__src__3dmake_foundation__appendix_a_comprehensive_slicing_guide.md__key-features}

Dual Extrusion

\begin{itemize}
\tightlist
\item
  Two nozzles = two colors in one print
\item
  Useful for: Bracelets (core + colored rim)
\item
  Requires: Coordinating two materials
\end{itemize}

Advanced Support

\begin{itemize}
\tightlist
\item
  Tree support (uses less material)
\item
  Grid support (stronger for larger parts)
\end{itemize}

Print Balancing

\begin{itemize}
\tightlist
\item
  Optimizes nozzle movement for efficiency
\item
  Reduces print time without sacrificing quality
\end{itemize}

\subsubsection*{7. Fusion 360 (CAD + Slicer)}\label{docs__pandoc__latex__src__3dmake_foundation__appendix_a_comprehensive_slicing_guide.md__7-fusion-360-cad--slicer}

\paragraph*{Overview}\label{docs__pandoc__latex__src__3dmake_foundation__appendix_a_comprehensive_slicing_guide.md__overview-6}

\begin{itemize}
\tightlist
\item
  Developer: Autodesk
\item
  Platforms: Windows, Mac
\item
  Best For: If you already use Fusion 360 for CAD
\item
  Download: \url{https://www.autodesk.com/products/fusion-360}
\item
  Accessibility: Integrated environment; good for learning CAD+Slicing together
\end{itemize}

\paragraph*{Why Integrate CAD + Slicing?}\label{docs__pandoc__latex__src__3dmake_foundation__appendix_a_comprehensive_slicing_guide.md__why-integrate-cad--slicing}

Traditional workflow:

\begin{lstlisting}[style=Alabaster]
Design in CAD -> Export to STL -> Open in Slicer -> Slice -> Print

\end{lstlisting}

Fusion 360 workflow:

\begin{lstlisting}[style=Alabaster]
Design in Fusion -> Run Print Preparation -> Slice -> Print

\end{lstlisting}

Advantage: Stay in one program; no format conversion

\paragraph*{Quick Start}\label{docs__pandoc__latex__src__3dmake_foundation__appendix_a_comprehensive_slicing_guide.md__quick-start-2}

\begin{enumerate}
\tightlist
\item
  Design in Fusion 360 (or import STL)
\item
  Select part
\item
  Go to "3D Print" tab
\item
  Click "Print Preparation"
\item
  View auto-generated support and alignment
\item
  Adjust layer height, infill, etc.
\item
  Slice and export G-code
\end{enumerate}

\paragraph*{Integration with 3dMake}\label{docs__pandoc__latex__src__3dmake_foundation__appendix_a_comprehensive_slicing_guide.md__integration-with-3dmake}

If using Fusion 360 for CAD:

\begin{enumerate}
\tightlist
\item
  Export SCAD design to STL
\item
  3dm export-stl src/main.scad output/main.stl
\item
  Then import into Fusion 360
\item
  For more complex designs that need refinement
\end{enumerate}

\subsubsection*{Universal Troubleshooting Guide}\label{docs__pandoc__latex__src__3dmake_foundation__appendix_a_comprehensive_slicing_guide.md__universal-troubleshooting-guide}

{\def\LTcaptype{none} % do not increment counter
\begin{longtable}[]{@{}
  >{\raggedright\arraybackslash}p{(\linewidth - 4\tabcolsep) * \real{0.3652}}
  >{\raggedright\arraybackslash}p{(\linewidth - 4\tabcolsep) * \real{0.2522}}
  >{\raggedright\arraybackslash}p{(\linewidth - 4\tabcolsep) * \real{0.3826}}@{}}
\toprule\noalign{}
\begin{minipage}[b]{\linewidth}\raggedright
Problem
\end{minipage} & \begin{minipage}[b]{\linewidth}\raggedright
Diagnosis
\end{minipage} & \begin{minipage}[b]{\linewidth}\raggedright
Solution
\end{minipage} \\
\midrule\noalign{}
\endhead
\bottomrule\noalign{}
\endlastfoot
Print won\textquotesingle t stick to bed & Bed temperature too low &
Raise bed temp +5C; check bed level \\
& Bed not level & Manual leveling procedure (printer manual) \\
& Build plate dirty & Clean with isopropyl alcohol \\
Nozzle hits model mid-print & Z-offset wrong &
Adjust Z-offset; re-level bed \\
& Model placed too low on bed & Use "Arrange on Bed" tool in slicer \\
Supports won\textquotesingle t remove & Too much support generated &
Reduce support density or angle \\
& Support too strong & Reduce support material percentage \\
Stringy/Oozing & Nozzle too hot & Reduce temp by 5-10C \\
& Retraction disabled & Enable retraction in settings \\
& Travel speed too fast & Reduce travel speed \\
Layer shifting (X/Y) & Belt tension off &
Check belt tension (printer manual) \\
& Stepper motor power issue & Firmware issue-check printer logs \\
Model prints poorly but slices look good & Filament quality issue &
Try different filament batch \\
& Nozzle clogged &
Unclog nozzle (heat -\textgreater{} purge -\textgreater{} clean) \\
\end{longtable}
}

\subsubsection*{Recommended Slicer for Each Situation}\label{docs__pandoc__latex__src__3dmake_foundation__appendix_a_comprehensive_slicing_guide.md__recommended-slicer-for-each-situation}

{\def\LTcaptype{none} % do not increment counter
\begin{longtable}[]{@{}
  >{\raggedright\arraybackslash}p{(\linewidth - 4\tabcolsep) * \real{0.3111}}
  >{\raggedright\arraybackslash}p{(\linewidth - 4\tabcolsep) * \real{0.2556}}
  >{\raggedright\arraybackslash}p{(\linewidth - 4\tabcolsep) * \real{0.4333}}@{}}
\toprule\noalign{}
\begin{minipage}[b]{\linewidth}\raggedright
Scenario
\end{minipage} & \begin{minipage}[b]{\linewidth}\raggedright
Best Slicer
\end{minipage} & \begin{minipage}[b]{\linewidth}\raggedright
Why
\end{minipage} \\
\midrule\noalign{}
\endhead
\bottomrule\noalign{}
\endlastfoot
I\textquotesingle m a beginner & PrusaSlicer &
Clear defaults, excellent UI \\
I have a Bambu printer & Bambu Studio & Native, best features \\
I have an Ultimaker & Cura & Official support, broad compatibility \\
I have a Raise3D & IdeaMaker & Official, dual-extrusion support \\
I want advanced control & SuperSlicer & Maximum customization \\
I want modern/fast slicing & OrcaSlicer & Great defaults, any printer \\
I use Fusion 360 for CAD & Fusion 360 Print Prep &
Integrated workflow \\
\end{longtable}
}

\subsubsection*{PowerShell Integration: Batch Slicing}\label{docs__pandoc__latex__src__3dmake_foundation__appendix_a_comprehensive_slicing_guide.md__powershell-integration-batch-slicing}

\paragraph*{Slice Multiple Files Automatically}\label{docs__pandoc__latex__src__3dmake_foundation__appendix_a_comprehensive_slicing_guide.md__slice-multiple-files-automatically}

\begin{lstlisting}[style=Alabaster, language=powershell]
# Batch slice all SCAD designs in a project
$scadDir = "C:\Projects\3dMake\src"
$slicerConfig = "default.cfg"
$outputDir = "C:\Projects\3dMake\gcode"
# Find all SCAD files
$scadFiles = Get-ChildItem -Path $scadDir -Filter "*.scad" -Recurse
foreach ($scadFile in $scadFiles) {
    $stlFile = $scadFile.FullName -replace ".scad$", ".stl"
    $gcodeFile = Join-Path $outputDir ($scadFile.BaseName + ".gcode")
    # Export SCAD to STL
    Write-Host "Exporting: $($scadFile.Name)"
    & "C:\Program Files\OpenSCAD\openscad.exe" -o "$stlFile" "$($scadFile.FullName)"
    # Slice STL
    Write-Host "Slicing: $($scadFile.Name)"
    & "C:\Program Files\Prusa3D\PrusaSlicer\prusa-slicer.exe" `
        --load-config-file "$slicerConfig" `
        --export-gcode "$gcodeFile" `
        "$stlFile"
    Write-Host "Complete: $gcodeFile"
}
Write-Host "Batch slicing finished."

\end{lstlisting}

\paragraph*{Monitor Printing Progress}\label{docs__pandoc__latex__src__3dmake_foundation__appendix_a_comprehensive_slicing_guide.md__monitor-printing-progress}

\begin{lstlisting}[style=Alabaster, language=powershell]
# Connect to printer API and monitor print status
# (Requires printer to support API-check documentation)
$printerIP = "192.168.1.100"  # Your printer's IP
$printerPort = 8080            # Typical API port
# Get current print status
$status = Invoke-WebRequest -Uri "https://yourdolphin.com/supernova/" `
    -UseBasicParsing | ConvertFrom-Json
Write-Host "State: $($status.state.text)"
Write-Host "Progress: $($status.progress.completion)%"
Write-Host "Time remaining: $($status.progress.printTimeLeft) seconds"

\end{lstlisting}

\subsubsection*{Accessibility Best Practices for Slicing}\label{docs__pandoc__latex__src__3dmake_foundation__appendix_a_comprehensive_slicing_guide.md__accessibility-best-practices-for-slicing}

\paragraph*{Describing Sliced Models}\label{docs__pandoc__latex__src__3dmake_foundation__appendix_a_comprehensive_slicing_guide.md__describing-sliced-models}

\begin{lstlisting}[style=Alabaster, language=powershell]
# Use 3dm describe to understand the slicer's output
# (Best practice: describe after slicing to verify settings)
3dm describe output/bracelet_holder.stl
# Output includes:
# - Number of parts
# - Dimensions
# - Surface area
# - Volume

\end{lstlisting}

\paragraph*{Testing Sliced Parts}\label{docs__pandoc__latex__src__3dmake_foundation__appendix_a_comprehensive_slicing_guide.md__testing-sliced-parts}

\begin{enumerate}
\item
  Weight Check:

  \begin{itemize}
  \tightlist
  \item
    Calculate expected weight from volume + material density
  \item
    Compare to actual printed weight
  \item
    Indicates if infill is correct
  \end{itemize}
\item
  Dimensional Check:

  \begin{itemize}
  \tightlist
  \item
    Use calipers to verify critical dimensions
  \item
    Compare to SCAD parameters
  \item
    Check tolerance stack-up
  \end{itemize}
\item
  Functional Test:

  \begin{itemize}
  \tightlist
  \item
    Assemble with other parts
  \item
    Test strength by loading with known weight
  \item
    Verify support removal didn\textquotesingle t damage part
  \end{itemize}
\end{enumerate}

\subsubsection*{Summary}\label{docs__pandoc__latex__src__3dmake_foundation__appendix_a_comprehensive_slicing_guide.md__summary}

This appendix provides:

\begin{itemize}
\tightlist
\item
  {[}YES{]} 7 slicer workflows, settings, and troubleshooting
\item
  {[}YES{]} Accessible parameter explanations
\item
  {[}YES{]} Command-line integration for PowerShell
\item
  {[}YES{]} Comparison table for choosing a slicer
\item
  {[}YES{]} Batch processing automation examples
\item
  {[}YES{]} Accessibility best practices
\end{itemize}

Use this guide whenever you:

\begin{itemize}
\tightlist
\item
  Start a new print
\item
  Switch slicers
\item
  Encounter quality issues
\item
  Want to automate slicing workflows
\item
  Need troubleshooting help
\end{itemize}

\subsection{Appendix B: Material Properties \& Selection Guide}\label{docs__pandoc__latex__src__3dmake_foundation__appendix_b_material_properties.md__3dmake_foundation-appendix_b_material_properties}

This appendix covers material properties, characteristics, and selection criteria for 3D printing filaments. Each material includes:

\begin{itemize}
\tightlist
\item
  Physical properties (strength, flexibility, temperature tolerance)
\item
  Printing parameters (nozzle temp, bed temp, speed)
\item
  Suitability for different projects
\item
  Accessibility considerations (measurement-based verification)
\item
  Cost/availability comparison
\end{itemize}

Referenced in: Lessons 5 (Safety), 6-10 (Projects), 11 (Customer Requirements)

\subsubsection*{Overview: Why Material Matters}\label{docs__pandoc__latex__src__3dmake_foundation__appendix_b_material_properties.md__overview-why-material-matters}

The material you choose affects:

\begin{itemize}
\tightlist
\item
  Strength: Can the part hold weight?
\item
  Flexibility: Will it bend or break?
\item
  Temperature: Can it withstand heat?
\item
  Durability: Will it last months or years?
\item
  Cost: Budget constraints?
\item
  Printability: Ease of use for beginners?
\item
  Appearance: Texture, color, finish?
\end{itemize}

\paragraph*{The Material Selection Flowchart}\label{docs__pandoc__latex__src__3dmake_foundation__appendix_b_material_properties.md__the-material-selection-flowchart}

\begin{lstlisting}[style=Alabaster]
What's the primary requirement?
+-- Strength & Detail?
    +-- PLA (best for beginners, detail)
        or PETG (stronger, tougher)
+-- Flexibility?
    +-- TPU/TPE (rubber-like)
+-- Heat Resistance?
    +-- ABS or Polycarbonate
+-- Transparency?
    +-- PETG or Polycarbonate
+-- Food Contact?
    +-- FDA-approved PETG or PLA
+-- Cost-Conscious?
    +-- PLA (cheapest, easiest)

\end{lstlisting}

\subsubsection*{1. PLA (Polylactic Acid)}\label{docs__pandoc__latex__src__3dmake_foundation__appendix_b_material_properties.md__1-pla-polylactic-acid}

\paragraph*{Properties at a Glance}\label{docs__pandoc__latex__src__3dmake_foundation__appendix_b_material_properties.md__properties-at-a-glance}

{\def\LTcaptype{none} % do not increment counter
\begin{longtable}[]{@{}
  >{\raggedright\arraybackslash}p{(\linewidth - 4\tabcolsep) * \real{0.3038}}
  >{\raggedright\arraybackslash}p{(\linewidth - 4\tabcolsep) * \real{0.1013}}
  >{\raggedright\arraybackslash}p{(\linewidth - 4\tabcolsep) * \real{0.5949}}@{}}
\toprule\noalign{}
\begin{minipage}[b]{\linewidth}\raggedright
Property
\end{minipage} & \begin{minipage}[b]{\linewidth}\raggedright
Rating
\end{minipage} & \begin{minipage}[b]{\linewidth}\raggedright
Notes
\end{minipage} \\
\midrule\noalign{}
\endhead
\bottomrule\noalign{}
\endlastfoot
Strength & {[}3/5{]} & Good for most projects; not for dynamic loads \\
Flexibility & {[}1/5{]} & Brittle; will snap under stress \\
Temperature Resistance & {[}2/5{]} &
Softens around 60C; bad for hot environments \\
Ease of Printing & {[}5/5{]} & Most forgiving; best for beginners \\
Cost & {[}5/5{]} & Cheapest option (\textasciitilde{}\$20/kg) \\
Appearance & {[}5/5{]} & Excellent surface finish; many colors \\
Availability & {[}5/5{]} & Available everywhere \\
\end{longtable}
}

\paragraph*{Ideal Projects}\label{docs__pandoc__latex__src__3dmake_foundation__appendix_b_material_properties.md__ideal-projects}

\begin{itemize}
\tightlist
\item
  {[}YES{]} Decorative pieces (jewelry, miniatures)
\item
  {[}YES{]} Enclosures/shells (light loads)
\item
  {[}YES{]} Prototype/mockups
\item
  {[}YES{]} Low-stress connector clips
\item
  {[}YES{]} Educational demonstrations
\item
  {[}NO{]} Load-bearing brackets
\item
  {[}NO{]} Hinges or flexing parts
\item
  {[}NO{]} Heat-resistant applications
\end{itemize}

\paragraph*{Printing Parameters}\label{docs__pandoc__latex__src__3dmake_foundation__appendix_b_material_properties.md__printing-parameters}

{\def\LTcaptype{none} % do not increment counter
\begin{longtable}[]{@{}lll@{}}
\toprule\noalign{}
Parameter & Value & Tolerance \\
\midrule\noalign{}
\endhead
\bottomrule\noalign{}
\endlastfoot
Nozzle Temp & 200C & 190-210C (varies by brand) \\
Bed Temp & 60C & 50-65C \\
Print Speed & 50 mm/s & 40-60 mm/s \\
Retraction & 5mm @ 40 mm/s & Yes, prevents stringing \\
Cooling Fan & 100\% & High cooling improves quality \\
First Layer & Slower (25 mm/s) & Ensures adhesion \\
Layer Height & 0.2mm & 0.1-0.3mm depending on detail \\
\end{longtable}
}

\paragraph*{PLA Variants}\label{docs__pandoc__latex__src__3dmake_foundation__appendix_b_material_properties.md__pla-variants}

Standard PLA

\begin{itemize}
\tightlist
\item
  Most common, reliable
\item
  Best for beginners
\end{itemize}

PLA Pro / Enhanced PLA

\begin{itemize}
\tightlist
\item
  Slightly stronger than standard PLA
\item
  Same temperature parameters
\item
  \textasciitilde{}10\% cost premium
\end{itemize}

Silk PLA

\begin{itemize}
\tightlist
\item
  Glossy finish instead of matte
\item
  Same strength as standard PLA
\item
  Slightly slower to print
\end{itemize}

Marble or Color-Changing PLA

\begin{itemize}
\tightlist
\item
  Visual effects
\item
  Same printing parameters
\item
  Mostly for aesthetics
\end{itemize}

\paragraph*{Common Issues \& Solutions}\label{docs__pandoc__latex__src__3dmake_foundation__appendix_b_material_properties.md__common-issues--solutions}

{\def\LTcaptype{none} % do not increment counter
\begin{longtable}[]{@{}
  >{\raggedright\arraybackslash}p{(\linewidth - 4\tabcolsep) * \real{0.2373}}
  >{\raggedright\arraybackslash}p{(\linewidth - 4\tabcolsep) * \real{0.2797}}
  >{\raggedright\arraybackslash}p{(\linewidth - 4\tabcolsep) * \real{0.4831}}@{}}
\toprule\noalign{}
\begin{minipage}[b]{\linewidth}\raggedright
Problem
\end{minipage} & \begin{minipage}[b]{\linewidth}\raggedright
Cause
\end{minipage} & \begin{minipage}[b]{\linewidth}\raggedright
Solution
\end{minipage} \\
\midrule\noalign{}
\endhead
\bottomrule\noalign{}
\endlastfoot
Warping on corners & Bed too hot or cooling too fast &
Reduce bed temp to 50C; disable cooling for first layer \\
Stringing & Temp too high & Lower nozzle temp 5C \\
Poor layer adhesion & Nozzle too high & Lower Z-offset 0.1mm \\
Brittleness after print & Normal for PLA &
Not a problem; expected behavior \\
Nozzle clogs on retraction & Temperature inconsistency &
Ensure stable nozzle temp +/- 5C \\
\end{longtable}
}

\paragraph*{Why PLA for This Course}\label{docs__pandoc__latex__src__3dmake_foundation__appendix_b_material_properties.md__why-pla-for-this-course}

PLA is recommended for all Lessons 1-11 because:

\begin{enumerate}
\tightlist
\item
  Beginner-friendly: Most forgiving material
\item
  Predictable: Consistent across different printers
\item
  Cost-effective: Maximize printing volume on student budget
\item
  Accessibility: Easier to troubleshoot for new users
\item
  Safe: Non-toxic, low fume emission
\item
  Available: Found at any 3D printing supplier
\end{enumerate}

\subsubsection*{2. PETG (Polyethylene Terephthalate Glycol)}\label{docs__pandoc__latex__src__3dmake_foundation__appendix_b_material_properties.md__2-petg-polyethylene-terephthalate-glycol}

\paragraph*{Properties at a Glance}\label{docs__pandoc__latex__src__3dmake_foundation__appendix_b_material_properties.md__properties-at-a-glance-1}

{\def\LTcaptype{none} % do not increment counter
\begin{longtable}[]{@{}
  >{\raggedright\arraybackslash}p{(\linewidth - 4\tabcolsep) * \real{0.2963}}
  >{\raggedright\arraybackslash}p{(\linewidth - 4\tabcolsep) * \real{0.0988}}
  >{\raggedright\arraybackslash}p{(\linewidth - 4\tabcolsep) * \real{0.6049}}@{}}
\toprule\noalign{}
\begin{minipage}[b]{\linewidth}\raggedright
Property
\end{minipage} & \begin{minipage}[b]{\linewidth}\raggedright
Rating
\end{minipage} & \begin{minipage}[b]{\linewidth}\raggedright
Notes
\end{minipage} \\
\midrule\noalign{}
\endhead
\bottomrule\noalign{}
\endlastfoot
Strength & {[}4/5{]} & Tougher than PLA; better for load-bearing \\
Flexibility & {[}2/5{]} & Better than PLA but still primarily rigid \\
Temperature Resistance & {[}3/5{]} &
Softens around 85C; better than PLA \\
Ease of Printing & {[}4/5{]} & Slightly more challenging than PLA \\
Cost & {[}4/5{]} & \textasciitilde{}\$25/kg (slightly more than PLA) \\
Appearance & {[}4/5{]} &
Good surface finish; slightly glossier than PLA \\
Availability & {[}5/5{]} & Widely available \\
\end{longtable}
}

\paragraph*{When to Use PETG Instead of PLA}\label{docs__pandoc__latex__src__3dmake_foundation__appendix_b_material_properties.md__when-to-use-petg-instead-of-pla}

\begin{itemize}
\tightlist
\item
  Functional brackets or mounts (where strength matters)
\item
  Parts that may be load-bearing (shelves, holders)
\item
  Outdoors/higher temperature environments
\item
  Parts requiring transparency (clear PETG available)
\item
  Better moisture resistance (vs. PLA)
\end{itemize}

\paragraph*{Printing Parameters}\label{docs__pandoc__latex__src__3dmake_foundation__appendix_b_material_properties.md__printing-parameters-1}

{\def\LTcaptype{none} % do not increment counter
\begin{longtable}[]{@{}lll@{}}
\toprule\noalign{}
Parameter & Value & Tolerance \\
\midrule\noalign{}
\endhead
\bottomrule\noalign{}
\endlastfoot
Nozzle Temp & 235C & 230-245C \\
Bed Temp & 80C & 75-85C \\
Print Speed & 50 mm/s & 40-60 mm/s (same as PLA) \\
Retraction & 4mm @ 40 mm/s & Slightly less than PLA \\
Cooling Fan & 30-50\% & Less cooling than PLA \\
First Layer & Normal (50 mm/s) & Harder to adjust than PLA \\
Layer Height & 0.2mm & 0.1-0.3mm \\
\end{longtable}
}

\paragraph*{Comparison: PETG vs. PLA}\label{docs__pandoc__latex__src__3dmake_foundation__appendix_b_material_properties.md__comparison-petg-vs-pla}

{\def\LTcaptype{none} % do not increment counter
\begin{longtable}[]{@{}lll@{}}
\toprule\noalign{}
Aspect & PLA & PETG \\
\midrule\noalign{}
\endhead
\bottomrule\noalign{}
\endlastfoot
Nozzle Temp & 200C & 235C \\
Bed Temp & 60C & 80C \\
Strength & Good & Better (20\% stronger) \\
Flexibility & Brittle & More resilient \\
Heat Tolerance & 60C & 85C \\
Ease & Very easy & Easy (needs tweaking) \\
Cost & \$20/kg & \$25/kg \\
Outdoor Use & Not ideal & Better \\
\end{longtable}
}

\paragraph*{Projects for PETG}\label{docs__pandoc__latex__src__3dmake_foundation__appendix_b_material_properties.md__projects-for-petg}

\begin{itemize}
\tightlist
\item
  {[}YES{]} Phone stand (needs strength)
\item
  {[}YES{]} Bracket or shelf support
\item
  {[}YES{]} Flexible clip (needs resilience)
\item
  {[}YES{]} Outdoor item
\item
  {[}YES{]} Clear enclosure (if using clear PETG)
\item
  {[}NO{]} Fine detail work (slightly courser finish)
\item
  {[}NO{]} Food-contact items (not food-safe unless specified)
\end{itemize}

\subsubsection*{3. ABS (Acrylonitrile Butadiene Styrene)}\label{docs__pandoc__latex__src__3dmake_foundation__appendix_b_material_properties.md__3-abs-acrylonitrile-butadiene-styrene}

\paragraph*{Properties at a Glance}\label{docs__pandoc__latex__src__3dmake_foundation__appendix_b_material_properties.md__properties-at-a-glance-2}

{\def\LTcaptype{none} % do not increment counter
\begin{longtable}[]{@{}
  >{\raggedright\arraybackslash}p{(\linewidth - 4\tabcolsep) * \real{0.3038}}
  >{\raggedright\arraybackslash}p{(\linewidth - 4\tabcolsep) * \real{0.1013}}
  >{\raggedright\arraybackslash}p{(\linewidth - 4\tabcolsep) * \real{0.5949}}@{}}
\toprule\noalign{}
\begin{minipage}[b]{\linewidth}\raggedright
Property
\end{minipage} & \begin{minipage}[b]{\linewidth}\raggedright
Rating
\end{minipage} & \begin{minipage}[b]{\linewidth}\raggedright
Notes
\end{minipage} \\
\midrule\noalign{}
\endhead
\bottomrule\noalign{}
\endlastfoot
Strength & {[}4/5{]} & Similar to PETG; good impact resistance \\
Flexibility & {[}3/5{]} & More flexible than PETG \\
Temperature Resistance & {[}5/5{]} &
Softens around 105C; best of common materials \\
Ease of Printing & {[}2/5{]} &
Requires enclosure/heated bed; challenging \\
Cost & {[}4/5{]} & \textasciitilde{}\$25-30/kg \\
Appearance & {[}3/5{]} & Rougher than PLA; requires post-processing \\
Availability & {[}4/5{]} & Good availability; not as universal as PLA \\
\end{longtable}
}

\paragraph*{When to Use ABS}\label{docs__pandoc__latex__src__3dmake_foundation__appendix_b_material_properties.md__when-to-use-abs}

\begin{itemize}
\tightlist
\item
  High-temperature environments (near heat sources)
\item
  Mechanical parts (gears, bearings)
\item
  Durability (parts lasting years)
\item
  Post-processing (can be sanded, glued, vapor-smoothed)
\item
  Professional applications (not toys/decorative)
\end{itemize}

\paragraph*{Why ABS is Challenging}\label{docs__pandoc__latex__src__3dmake_foundation__appendix_b_material_properties.md__why-abs-is-challenging}

ABS requires:

\begin{enumerate}
\tightlist
\item
  Enclosed environment: Minimize temperature fluctuations
\item
  Heated bed: 100C+ (much higher than PLA)
\item
  Controlled cooling: Too-fast cooling causes warping
\item
  Ventilation: ABS emits fumes (acetone-like smell)
\end{enumerate}

\paragraph*{Printing Parameters}\label{docs__pandoc__latex__src__3dmake_foundation__appendix_b_material_properties.md__printing-parameters-2}

{\def\LTcaptype{none} % do not increment counter
\begin{longtable}[]{@{}lll@{}}
\toprule\noalign{}
Parameter & Value & Tolerance \\
\midrule\noalign{}
\endhead
\bottomrule\noalign{}
\endlastfoot
Nozzle Temp & 240C & 230-250C \\
Bed Temp & 100C & 95-105C \\
Enclosure & Required & Maintains heat (reduces warping) \\
Print Speed & 40 mm/s & Slower than PLA \\
Retraction & 3mm @ 30 mm/s & Very short \\
Cooling Fan & 0\% & OFF (causes warping) \\
First Layer & Slow (30 mm/s) & Critical for adhesion \\
\end{longtable}
}

\paragraph*{Projects for ABS}\label{docs__pandoc__latex__src__3dmake_foundation__appendix_b_material_properties.md__projects-for-abs}

\begin{itemize}
\tightlist
\item
  {[}YES{]} Mechanical components
\item
  {[}YES{]} Heat-resistant enclosure
\item
  {[}YES{]} Durable outdoor item
\item
  {[}YES{]} Professional prototypes
\item
  {[}NO{]} Beginner projects (too challenging)
\item
  {[}NO{]} Decorative/aesthetic work (not recommended)
\item
  {[}NO{]} Flexible parts
\end{itemize}

\paragraph*{Accessibility Note}\label{docs__pandoc__latex__src__3dmake_foundation__appendix_b_material_properties.md__accessibility-note}

ABS is not recommended for this course because:

\begin{enumerate}
\tightlist
\item
  Requires enclosed printer (expensive for beginners)
\item
  High failure rate for inexperienced users
\item
  Strong odor (ventilation concerns for some users)
\item
  Requires extra equipment (acetone for post-processing)
\end{enumerate}

\subsubsection*{4. TPU / TPE (Thermoplastic Polyurethane / Elastomer)}\label{docs__pandoc__latex__src__3dmake_foundation__appendix_b_material_properties.md__4-tpu--tpe-thermoplastic-polyurethane--elastomer}

\paragraph*{Properties at a Glance}\label{docs__pandoc__latex__src__3dmake_foundation__appendix_b_material_properties.md__properties-at-a-glance-3}

{\def\LTcaptype{none} % do not increment counter
\begin{longtable}[]{@{}
  >{\raggedright\arraybackslash}p{(\linewidth - 4\tabcolsep) * \real{0.3038}}
  >{\raggedright\arraybackslash}p{(\linewidth - 4\tabcolsep) * \real{0.1013}}
  >{\raggedright\arraybackslash}p{(\linewidth - 4\tabcolsep) * \real{0.5949}}@{}}
\toprule\noalign{}
\begin{minipage}[b]{\linewidth}\raggedright
Property
\end{minipage} & \begin{minipage}[b]{\linewidth}\raggedright
Rating
\end{minipage} & \begin{minipage}[b]{\linewidth}\raggedright
Notes
\end{minipage} \\
\midrule\noalign{}
\endhead
\bottomrule\noalign{}
\endlastfoot
Strength & {[}3/5{]} & Good; impact-resistant \\
Flexibility & {[}5/5{]} & Very flexible; rubber-like \\
Temperature Resistance & {[}3/5{]} & Softens around 80C; moderate \\
Ease of Printing & {[}3/5{]} &
Needs tweaking; flexible materials are tricky \\
Cost & {[}3/5{]} & \textasciitilde{}\$30-40/kg (expensive) \\
Appearance & {[}4/5{]} & Smooth; feels good tactilely \\
Availability & {[}4/5{]} & Growing availability \\
\end{longtable}
}

\paragraph*{What is TPU?}\label{docs__pandoc__latex__src__3dmake_foundation__appendix_b_material_properties.md__what-is-tpu}

TPU is a flexible rubber-like plastic that:

\begin{itemize}
\tightlist
\item
  Doesn\textquotesingle t crack when bent
\item
  Absorbs impact
\item
  Returns to original shape
\item
  Bridges the gap between plastic and rubber
\end{itemize}

\paragraph*{Printing Parameters}\label{docs__pandoc__latex__src__3dmake_foundation__appendix_b_material_properties.md__printing-parameters-3}

{\def\LTcaptype{none} % do not increment counter
\begin{longtable}[]{@{}
  >{\raggedright\arraybackslash}p{(\linewidth - 4\tabcolsep) * \real{0.1688}}
  >{\raggedright\arraybackslash}p{(\linewidth - 4\tabcolsep) * \real{0.2078}}
  >{\raggedright\arraybackslash}p{(\linewidth - 4\tabcolsep) * \real{0.6234}}@{}}
\toprule\noalign{}
\begin{minipage}[b]{\linewidth}\raggedright
Parameter
\end{minipage} & \begin{minipage}[b]{\linewidth}\raggedright
Value
\end{minipage} & \begin{minipage}[b]{\linewidth}\raggedright
Tolerance
\end{minipage} \\
\midrule\noalign{}
\endhead
\bottomrule\noalign{}
\endlastfoot
Nozzle Temp & 215C & 210-225C \\
Bed Temp & 60C & 50-70C \\
Print Speed & 20-30 mm/s & VERY SLOW (flexibility needs time) \\
Retraction & Minimal or Off &
0-1mm (flexible material doesn\textquotesingle t retract well) \\
Cooling Fan & 0\% & Off (material needs heat) \\
Line Width & 0.5mm & Wider than normal (flexible material bridges) \\
\end{longtable}
}

\paragraph*{Projects for TPU}\label{docs__pandoc__latex__src__3dmake_foundation__appendix_b_material_properties.md__projects-for-tpu}

\begin{itemize}
\tightlist
\item
  {[}YES{]} Phone case (needs flexibility + protection)
\item
  {[}YES{]} Flex joints / hinges
\item
  {[}YES{]} Gasket or seal
\item
  {[}YES{]} Shoe insert or orthotic
\item
  {[}YES{]} Tactile button (for accessibility)
\item
  {[}NO{]} Fine detail work (too stretchy)
\item
  {[}NO{]} Decorative items (usually not aesthetic)
\item
  {[}NO{]} Precision parts
\end{itemize}

\paragraph*{Challenges with TPU}\label{docs__pandoc__latex__src__3dmake_foundation__appendix_b_material_properties.md__challenges-with-tpu}

\begin{enumerate}
\tightlist
\item
  Very slow printing: 5-10x slower than PLA
\item
  Stringing: Flexible material tends to ooze
\item
  Flexible bed needed: Standard beds may not work
\item
  Post-processing difficult: Hard to sand/glue
\end{enumerate}

\paragraph*{Why TPU for Accessibility}\label{docs__pandoc__latex__src__3dmake_foundation__appendix_b_material_properties.md__why-tpu-for-accessibility}

TPU is excellent for accessibility because:

\begin{itemize}
\tightlist
\item
  Can create tactile buttons/indicators
\item
  Flexible grips for ergonomic handles
\item
  Gaskets that don\textquotesingle t damage delicate equipment
\item
  Accessible because: Achievable by all users if given proper guidance
\end{itemize}

\subsubsection*{5. Polycarbonate (PC)}\label{docs__pandoc__latex__src__3dmake_foundation__appendix_b_material_properties.md__5-polycarbonate-pc}

\paragraph*{Properties at a Glance}\label{docs__pandoc__latex__src__3dmake_foundation__appendix_b_material_properties.md__properties-at-a-glance-4}

{\def\LTcaptype{none} % do not increment counter
\begin{longtable}[]{@{}lll@{}}
\toprule\noalign{}
Property & Rating & Notes \\
\midrule\noalign{}
\endhead
\bottomrule\noalign{}
\endlastfoot
Strength & {[}5/5{]} & Extremely strong; impact-resistant \\
Flexibility & {[}2/5{]} & Rigid; similar to PETG \\
Temperature Resistance & {[}5/5{]} & Best; softens around 130C \\
Ease of Printing & {[}2/5{]} & Difficult; prone to warping \\
Cost & {[}2/5{]} & Most expensive (\textasciitilde{}\$50+/kg) \\
Appearance & {[}4/5{]} & Transparent/translucent options \\
Availability & {[}2/5{]} & Limited; specialty supply \\
\end{longtable}
}

\paragraph*{When to Use Polycarbonate}\label{docs__pandoc__latex__src__3dmake_foundation__appendix_b_material_properties.md__when-to-use-polycarbonate}

\begin{itemize}
\tightlist
\item
  Highest strength required
\item
  Transparent/bullet-proof enclosure needed
\item
  Extreme temperature environment
\item
  Professional/industrial applications
\end{itemize}

\paragraph*{Why Polycarbonate is Not for This Course}\label{docs__pandoc__latex__src__3dmake_foundation__appendix_b_material_properties.md__why-polycarbonate-is-not-for-this-course}

\begin{enumerate}
\tightlist
\item
  Extreme difficulty (high failure rate)
\item
  Very expensive (3-5x cost of PLA)
\item
  Requires industrial-grade printer
\item
  Post-processing complex
\end{enumerate}

\subsubsection*{6. Nylon (PA)}\label{docs__pandoc__latex__src__3dmake_foundation__appendix_b_material_properties.md__6-nylon-pa}

\paragraph*{Properties at a Glance}\label{docs__pandoc__latex__src__3dmake_foundation__appendix_b_material_properties.md__properties-at-a-glance-5}

{\def\LTcaptype{none} % do not increment counter
\begin{longtable}[]{@{}
  >{\raggedright\arraybackslash}p{(\linewidth - 4\tabcolsep) * \real{0.3380}}
  >{\raggedright\arraybackslash}p{(\linewidth - 4\tabcolsep) * \real{0.1127}}
  >{\raggedright\arraybackslash}p{(\linewidth - 4\tabcolsep) * \real{0.5493}}@{}}
\toprule\noalign{}
\begin{minipage}[b]{\linewidth}\raggedright
Property
\end{minipage} & \begin{minipage}[b]{\linewidth}\raggedright
Rating
\end{minipage} & \begin{minipage}[b]{\linewidth}\raggedright
Notes
\end{minipage} \\
\midrule\noalign{}
\endhead
\bottomrule\noalign{}
\endlastfoot
Strength & {[}5/5{]} & Very strong; can be flexible \\
Flexibility & {[}4/5{]} & More flexible than ABS \\
Temperature Resistance & {[}4/5{]} & Good; softens around 120C \\
Ease of Printing & {[}2/5{]} & Difficult; very temperature-sensitive \\
Cost & {[}3/5{]} & \textasciitilde{}\$30-40/kg \\
Appearance & {[}3/5{]} & Matte finish; less aesthetic than PLA \\
Availability & {[}3/5{]} & Growing but limited \\
\end{longtable}
}

\paragraph*{Nylon Use Cases}\label{docs__pandoc__latex__src__3dmake_foundation__appendix_b_material_properties.md__nylon-use-cases}

\begin{itemize}
\tightlist
\item
  {[}YES{]} Mechanical parts (gears, hinges)
\item
  {[}YES{]} Flexible connectors
\item
  {[}YES{]} Threads/screws
\item
  {[}YES{]} High-stress applications
\item
  {[}NO{]} Not for beginners
\end{itemize}

\subsubsection*{Material Comparison Table}\label{docs__pandoc__latex__src__3dmake_foundation__appendix_b_material_properties.md__material-comparison-table}

{\def\LTcaptype{none} % do not increment counter
\begin{longtable}[]{@{}
  >{\raggedright\arraybackslash}p{(\linewidth - 16\tabcolsep) * \real{0.1124}}
  >{\raggedright\arraybackslash}p{(\linewidth - 16\tabcolsep) * \real{0.0899}}
  >{\raggedright\arraybackslash}p{(\linewidth - 16\tabcolsep) * \real{0.0674}}
  >{\raggedright\arraybackslash}p{(\linewidth - 16\tabcolsep) * \real{0.1124}}
  >{\raggedright\arraybackslash}p{(\linewidth - 16\tabcolsep) * \real{0.1461}}
  >{\raggedright\arraybackslash}p{(\linewidth - 16\tabcolsep) * \real{0.0787}}
  >{\raggedright\arraybackslash}p{(\linewidth - 16\tabcolsep) * \real{0.0787}}
  >{\raggedright\arraybackslash}p{(\linewidth - 16\tabcolsep) * \real{0.0674}}
  >{\raggedright\arraybackslash}p{(\linewidth - 16\tabcolsep) * \real{0.2472}}@{}}
\toprule\noalign{}
\begin{minipage}[b]{\linewidth}\raggedright
Material
\end{minipage} & \begin{minipage}[b]{\linewidth}\raggedright
Nozzle
\end{minipage} & \begin{minipage}[b]{\linewidth}\raggedright
Bed
\end{minipage} & \begin{minipage}[b]{\linewidth}\raggedright
Strength
\end{minipage} & \begin{minipage}[b]{\linewidth}\raggedright
Flexibility
\end{minipage} & \begin{minipage}[b]{\linewidth}\raggedright
Heat
\end{minipage} & \begin{minipage}[b]{\linewidth}\raggedright
Ease
\end{minipage} & \begin{minipage}[b]{\linewidth}\raggedright
Cost
\end{minipage} & \begin{minipage}[b]{\linewidth}\raggedright
Best For
\end{minipage} \\
\midrule\noalign{}
\endhead
\bottomrule\noalign{}
\endlastfoot
PLA & 200C & 60C & {[}3/5{]} & {[}1/5{]} & {[}2/5{]} & {[}5/5{]} & \$ &
Beginners, detail \\
PETG & 235C & 80C & {[}4/5{]} & {[}2/5{]} & {[}3/5{]} & {[}4/5{]} & \$\$
& Functional parts \\
ABS & 240C & 100C & {[}4/5{]} & {[}3/5{]} & {[}5/5{]} & {[}2/5{]} & \$\$
& High-temp/mechanical \\
TPU & 215C & 60C & {[}3/5{]} & {[}5/5{]} & {[}3/5{]} & {[}3/5{]} &
\$\$\$ & Flexibility/tactile \\
PC & 280C & 110C & {[}5/5{]} & {[}2/5{]} & {[}5/5{]} & {[}1/5{]} &
\$\$\$\$ & Extreme strength \\
Nylon & 250C & 85C & {[}5/5{]} & {[}4/5{]} & {[}4/5{]} & {[}2/5{]} &
\$\$\$ & Mechanical/flexible \\
\end{longtable}
}

\subsubsection*{Filament Quality Factors}\label{docs__pandoc__latex__src__3dmake_foundation__appendix_b_material_properties.md__filament-quality-factors}

\paragraph*{Why Not All PLA is the Same}\label{docs__pandoc__latex__src__3dmake_foundation__appendix_b_material_properties.md__why-not-all-pla-is-the-same}

Diameter Tolerance

\begin{itemize}
\tightlist
\item
  Good filament: +/- 0.03mm
\item
  Poor filament: +/- 0.1mm or worse
\item
  Impact: Inconsistent extrusion, layer quality varies
\end{itemize}

Dryness

\begin{itemize}
\tightlist
\item
  PLA absorbs moisture from air
\item
  Wet filament = weak prints + bubbles
\item
  Solution: Store in sealed container with desiccant
\end{itemize}

Color Consistency

\begin{itemize}
\tightlist
\item
  Good brands: Same color throughout
\item
  Poor brands: Color varies spool-to-spool
\end{itemize}

Impurities

\begin{itemize}
\tightlist
\item
  Good: Minimal impurities
\item
  Poor: Visible specs/contaminants -\textgreater{} possible clogs
\end{itemize}

\paragraph*{How to Check Filament Quality (Non-Visually)}\label{docs__pandoc__latex__src__3dmake_foundation__appendix_b_material_properties.md__how-to-check-filament-quality-non-visually}

\begin{enumerate}
\item
  Weight Check

  \begin{lstlisting}[style=Alabaster]
  Known: 1kg spool
  Weigh spool + filament
  Calculate remaining filament
  Should match spool markings

  \end{lstlisting}
\item
  Diameter Check

  \begin{lstlisting}[style=Alabaster]
  Use caliper to measure multiple points
  Should be consistent +/- 0.03mm
  If varying, likely lower quality

  \end{lstlisting}
\item
  Dryness Test

  \begin{lstlisting}[style=Alabaster]
  Feel texture: Should be smooth, not tacky
  Smell: Should be neutral (not musty)
  If wet, store in sealed container with desiccant

  \end{lstlisting}
\item
  Print Test

  \begin{lstlisting}[style=Alabaster]
  Print small cube (20mm x 20mm x 20mm)
  Inspect surface: Smooth or bubbly?
  Weight: Does it match expected weight?

  \end{lstlisting}
\end{enumerate}

\subsubsection*{Storage \& Maintenance}\label{docs__pandoc__latex__src__3dmake_foundation__appendix_b_material_properties.md__storage--maintenance}

\paragraph*{Proper Filament Storage}\label{docs__pandoc__latex__src__3dmake_foundation__appendix_b_material_properties.md__proper-filament-storage}

Temperature

\begin{itemize}
\tightlist
\item
  Store between 15-25C
\item
  Avoid direct sunlight (fades color, degrades material)
\end{itemize}

Humidity

\begin{itemize}
\tightlist
\item
  Keep below 40\% humidity
\item
  Use desiccant packets in sealed containers
\item
  Change desiccant every 2-3 months
\end{itemize}

Organization

\begin{itemize}
\tightlist
\item
  Label spools with: Material, color, date opened, approx. remaining
\item
  Store vertically or on spindle (prevents kinking)
\end{itemize}

\paragraph*{Filament Degradation Signs}\label{docs__pandoc__latex__src__3dmake_foundation__appendix_b_material_properties.md__filament-degradation-signs}

{\def\LTcaptype{none} % do not increment counter
\begin{longtable}[]{@{}
  >{\raggedright\arraybackslash}p{(\linewidth - 4\tabcolsep) * \real{0.3523}}
  >{\raggedright\arraybackslash}p{(\linewidth - 4\tabcolsep) * \real{0.3182}}
  >{\raggedright\arraybackslash}p{(\linewidth - 4\tabcolsep) * \real{0.3295}}@{}}
\toprule\noalign{}
\begin{minipage}[b]{\linewidth}\raggedright
Sign
\end{minipage} & \begin{minipage}[b]{\linewidth}\raggedright
Cause
\end{minipage} & \begin{minipage}[b]{\linewidth}\raggedright
Solution
\end{minipage} \\
\midrule\noalign{}
\endhead
\bottomrule\noalign{}
\endlastfoot
Weak prints / breaking easily & Filament aged or wet &
Replace with fresh filament \\
Discoloration or spots & Oxidation or contamination &
Not usable; discard safely \\
Brittle or crumbly & Overheated or UV damage & Not usable; discard \\
Slight fading (color) & UV exposure & Still usable; just faded \\
\end{longtable}
}

\subsubsection*{Cost Analysis}\label{docs__pandoc__latex__src__3dmake_foundation__appendix_b_material_properties.md__cost-analysis}

\paragraph*{Cost per Project}\label{docs__pandoc__latex__src__3dmake_foundation__appendix_b_material_properties.md__cost-per-project}

\begin{lstlisting}[style=Alabaster]
Material cost = (Filament weight used) x (Cost per kg)

Example: Bracelet holder
- Weight: 25 grams
- Material: PLA at $20/kg
- Cost: (25g / 1000g) x $20 = $0.50

vs. PETG at $25/kg: $0.625
vs. ABS at $30/kg: $0.75

\end{lstlisting}

\paragraph*{Budget Tips}\label{docs__pandoc__latex__src__3dmake_foundation__appendix_b_material_properties.md__budget-tips}

\begin{enumerate}
\tightlist
\item
  Buy bulk: 5kg spool is cheaper per gram than 1kg
\item
  Buy sales: 30-40\% discounts common during sales
\item
  Brand matters: Premium brands slightly more but more reliable
\item
  Generic brands: Often acceptable if reviews are good
\end{enumerate}

\paragraph*{Recommended Brands (Ranked by Beginner-Friendliness)}\label{docs__pandoc__latex__src__3dmake_foundation__appendix_b_material_properties.md__recommended-brands-ranked-by-beginner-friendliness}

{\def\LTcaptype{none} % do not increment counter
\begin{longtable}[]{@{}
  >{\raggedright\arraybackslash}p{(\linewidth - 8\tabcolsep) * \real{0.0659}}
  >{\raggedright\arraybackslash}p{(\linewidth - 8\tabcolsep) * \real{0.1648}}
  >{\raggedright\arraybackslash}p{(\linewidth - 8\tabcolsep) * \real{0.3736}}
  >{\raggedright\arraybackslash}p{(\linewidth - 8\tabcolsep) * \real{0.0659}}
  >{\raggedright\arraybackslash}p{(\linewidth - 8\tabcolsep) * \real{0.3297}}@{}}
\toprule\noalign{}
\begin{minipage}[b]{\linewidth}\raggedright
Rank
\end{minipage} & \begin{minipage}[b]{\linewidth}\raggedright
Brand
\end{minipage} & \begin{minipage}[b]{\linewidth}\raggedright
Known For
\end{minipage} & \begin{minipage}[b]{\linewidth}\raggedright
Cost
\end{minipage} & \begin{minipage}[b]{\linewidth}\raggedright
Notes
\end{minipage} \\
\midrule\noalign{}
\endhead
\bottomrule\noalign{}
\endlastfoot
1 & Prusament & Reliability, Prusa compatibility & \$\$\$ &
Best; excellent support \\
2 & MatterHackers & Quality, variety & \$\$ &
Very good; educational focus \\
3 & Fillamentum & European quality & \$\$ & Excellent; eco-friendly \\
4 & Overture & Value, consistency & \$ & Good budget option \\
5 & eSUN & Variety, affordable & \$ & Decent; variable quality \\
\end{longtable}
}

\subsubsection*{Material Selection Decision Tree}\label{docs__pandoc__latex__src__3dmake_foundation__appendix_b_material_properties.md__material-selection-decision-tree}

\begin{lstlisting}[style=Alabaster]
START HERE: What's most important?

GOAL: Beginner success?
  YES - PLA (best choice)
  NO  - Next question

GOAL: Strength matters?
  YES - PETG or Nylon
  NO  - Next question

GOAL: Flexibility needed?
  YES - TPU (rubber-like)
  NO  - Next question

GOAL: High temperature?
  YES - ABS or Polycarbonate
  NO  - Next question

GOAL: Transparent?
  YES - Clear PETG or Polycarbonate
  NO  - Use PLA

FINAL CHOICE:
- If unsure, use PLA
- If needs strength, use PETG
- If needs flexibility, use TPU
- If needs heat, use ABS

\end{lstlisting}

\subsubsection*{PowerShell Integration: Track Material Usage}\label{docs__pandoc__latex__src__3dmake_foundation__appendix_b_material_properties.md__powershell-integration-track-material-usage}

\begin{lstlisting}[style=Alabaster, language=powershell]
# Track filament usage across all projects
$materialLog = @"
ProjectName,Material,ColorName,WeightUsed(g),DatePrinted,Notes
"@
# Example entries
$materialLog += "`nBraceletHolder,PLA,NaturalWhite,25.3,2024-01-15,Final design"
$materialLog += "`nPhoneStand,PETG,Black,47.2,2024-01-16,Needs strength"
$materialLog += "`nKeycap,PLA,Red,12.5,2024-01-17,Prototype"
# Save to CSV
$materialLog | Out-File "C:\Projects\material-log.csv"
# Analyze usage
$materials = Import-Csv "C:\Projects\material-log.csv"
$totalWeight = ($materials | Measure-Object -Property "WeightUsed(g)" -Sum).Sum
$avgWeight = ($materials | Measure-Object -Property "WeightUsed(g)" -Average).Average
Write-Host "Total weight used: $totalWeight grams"
Write-Host "Average per project: $avgWeight grams"
# Calculate cost (PLA at $20/kg)
$costPerKg = 20
$totalCost = ($totalWeight / 1000) * $costPerKm
Write-Host "Estimated cost: $$($totalCost.ToString("F2"))"

\end{lstlisting}

\subsubsection*{Summary}\label{docs__pandoc__latex__src__3dmake_foundation__appendix_b_material_properties.md__summary}

Key Takeaways:

\begin{enumerate}
\tightlist
\item
  Start with PLA: Best for learning; most forgiving
\item
  Understand the tradeoffs: Strength vs. ease, cost vs. quality
\item
  Match material to project: Decorative = PLA; functional = PETG
\item
  Store properly: Desiccant, cool, dark location
\item
  Track usage: Know what works for future projects
\item
  Test before committing: Print small test on new filament
\end{enumerate}

Recommended Progression:

\begin{itemize}
\tightlist
\item
  Lessons 6-7: PLA (simplest)
\item
  Lessons 8-9: PLA or PETG (functional parts)
\item
  Lesson 10: PETG or TPU (testing materials)
\item
  Lesson 11: Student choice (depends on stakeholder requirements)
\end{itemize}

\subsection{Appendix C: Tolerance Testing \& Quality Assurance Matrix}\label{docs__pandoc__latex__src__3dmake_foundation__appendix_c_tolerance_qa.md__3dmake_foundation-appendix_c_tolerance_qa}

This appendix provides measurement-based testing methodology for verifying that 3D-printed parts meet design specifications. It\textquotesingle s designed to be used non-visually-focusing on calipers, scales, and functional tests rather than visual inspection.

Referenced in: Lessons 8-10 (Complex Design, Troubleshooting, Mastery)

\subsubsection*{Overview: What is Tolerance?}\label{docs__pandoc__latex__src__3dmake_foundation__appendix_c_tolerance_qa.md__overview-what-is-tolerance}

Tolerance is the acceptable range of variation in dimensions:

\begin{lstlisting}[style=Alabaster]
Design spec:  Hole diameter = 6mm
Tolerance:    +/-0.5mm
Acceptable range:  5.5mm to 6.5mm
Actual print:  5.8mm [YES] (within tolerance)
or            7.2mm [NO] (exceeds tolerance)

\end{lstlisting}

\paragraph*{Why Tolerance Matters}\label{docs__pandoc__latex__src__3dmake_foundation__appendix_c_tolerance_qa.md__why-tolerance-matters}

\begin{enumerate}
\tightlist
\item
  Assembly: Parts must fit together
\item
  Function: Fit too tight = stuck; too loose = falls apart
\item
  Safety: Wrong tolerance = part failure
\item
  Cost: Tight tolerance = slower, more waste
\end{enumerate}

\subsubsection*{Essential Measurement Tools}\label{docs__pandoc__latex__src__3dmake_foundation__appendix_c_tolerance_qa.md__essential-measurement-tools}

\paragraph*{1. Digital Calipers}\label{docs__pandoc__latex__src__3dmake_foundation__appendix_c_tolerance_qa.md__1-digital-calipers}

What they measure:

\begin{itemize}
\tightlist
\item
  Outside diameter (part width)
\item
  Inside diameter (hole width)
\item
  Depth
\end{itemize}

How to use non-visually:

\begin{enumerate}
\tightlist
\item
  Gently close calipers until they barely touch the part
\item
  Feel the resistance (should be light, not forced)
\item
  Read digital display with audio feedback or manually
\item
  Record three measurements at different locations
\item
  Average the three measurements
\end{enumerate}

Accuracy: +/-0.05mm (very precise for 3D printing)

\paragraph*{2. Digital Scale (Kitchen Scale)}\label{docs__pandoc__latex__src__3dmake_foundation__appendix_c_tolerance_qa.md__2-digital-scale-kitchen-scale}

What it measures:

\begin{itemize}
\tightlist
\item
  Part weight (indicates infill, material type)
\end{itemize}

How to use non-visually:

\begin{enumerate}
\tightlist
\item
  Place part on scale
\item
  Wait for reading to stabilize (1-2 seconds)
\item
  Read digital display (in grams)
\item
  Compare to expected weight
\end{enumerate}

Accuracy: +/-1g (good enough for verification)

Why it matters:

\begin{itemize}
\tightlist
\item
  Too light = infill too low or void inside part
\item
  Expected weight = indicates proper slicing
\end{itemize}

\paragraph*{3. Test Jig / Go/No-Go Gauge}\label{docs__pandoc__latex__src__3dmake_foundation__appendix_c_tolerance_qa.md__3-test-jig--gono-go-gauge}

What it measures:

\begin{itemize}
\tightlist
\item
  Pass/fail tolerance testing without calipers
\end{itemize}

How to make:

\begin{lstlisting}[style=Alabaster, language=openscad]
// Go/No-Go gauge for bracelet peg holes
// Tests if hole is within acceptable range
pegdiameter = 6;        // Design spec
tolerance = 0.5;         // Tolerance
godiameter = pegdiameter - tolerance;     // Min acceptable (5.5)
nogodiameter = pegdiameter + tolerance;  // Max acceptable (6.5)
module gogauge() {
  // Part should fit through this easily
  cylinder(h=10, r=godiameter/2);
}
module nogogauge() {
  // Part should NOT fit through this
  cylinder(h=10, r=nogodiameter/2);
}
// Test jig with both gauges
union() {
  translate([0, 0, 0]) gogauge();
  translate([0, 15, 0]) nogogauge();
}

\end{lstlisting}

Use non-visually:

\begin{itemize}
\tightlist
\item
  Try inserting peg into go-gauge -\textgreater{} Should slide easily
\item
  Try inserting peg into no-go-gauge -\textgreater{} Should NOT fit
\item
  If both tests pass = tolerance correct
\end{itemize}

\subsubsection*{Quality Assurance Testing Matrix}\label{docs__pandoc__latex__src__3dmake_foundation__appendix_c_tolerance_qa.md__quality-assurance-testing-matrix}

\paragraph*{Critical Dimensions to Test}\label{docs__pandoc__latex__src__3dmake_foundation__appendix_c_tolerance_qa.md__critical-dimensions-to-test}

Create a Test Plan before printing:

{\def\LTcaptype{none} % do not increment counter
\begin{longtable}[]{@{}
  >{\raggedright\arraybackslash}p{(\linewidth - 8\tabcolsep) * \real{0.1809}}
  >{\raggedright\arraybackslash}p{(\linewidth - 8\tabcolsep) * \real{0.1915}}
  >{\raggedright\arraybackslash}p{(\linewidth - 8\tabcolsep) * \real{0.0745}}
  >{\raggedright\arraybackslash}p{(\linewidth - 8\tabcolsep) * \real{0.1170}}
  >{\raggedright\arraybackslash}p{(\linewidth - 8\tabcolsep) * \real{0.4362}}@{}}
\toprule\noalign{}
\begin{minipage}[b]{\linewidth}\raggedright
Part
\end{minipage} & \begin{minipage}[b]{\linewidth}\raggedright
Dimension
\end{minipage} & \begin{minipage}[b]{\linewidth}\raggedright
Spec
\end{minipage} & \begin{minipage}[b]{\linewidth}\raggedright
Tolerance
\end{minipage} & \begin{minipage}[b]{\linewidth}\raggedright
How to Test
\end{minipage} \\
\midrule\noalign{}
\endhead
\bottomrule\noalign{}
\endlastfoot
Bracelet Holder & Base width & 127mm & +/-2mm &
Measure with calipers (multiple points) \\
& Peg diameter & 6mm & +/-0.5mm & Test fit with go/no-go gauge \\
& Peg spacing & 8mm & +/-1mm & Measure distance between pegs \\
& Back wall height & 120mm & +/-2mm & Measure with calipers \\
Phone Stand & Slope angle & 20 & +/-3 &
Calculate from height/depth ratio \\
& Weight capacity & 200g & 150-250g & Load test (see below) \\
& Stability & N/A & Pass/fail & 1-hour load test without tipping \\
\end{longtable}
}

\paragraph*{Pre-Print Planning}\label{docs__pandoc__latex__src__3dmake_foundation__appendix_c_tolerance_qa.md__pre-print-planning}

Before slicing, define:

\begin{enumerate}
\tightlist
\item
  Critical dimensions: Which measurements matter most?
\item
  Acceptable range: What tolerance is realistic?
\item
  Test method: How will you verify?
\item
  Pass/fail criteria: What does "success" look like?
\end{enumerate}

Example Plan for Phone Stand:

\begin{lstlisting}[style=Alabaster]
# Phone Stand - Quality Assurance Plan

## Critical Dimensions
1. Slope angle: 20 +/-3
2. Base width: 80mm +/-1mm
3. Stand height: 60mm +/-2mm

## Functional Tests
1. Stability: Hold 200g for 1 hour without tipping
2. Grip: Phone doesn't slide during tilt
3. Assembly: Back brace attaches without force

## Pass/Fail Criteria
[YES] PASS if:
- All dimensions within tolerance
- Phone holds weight for 1 hour
- No cracks or layer separation

[NO] FAIL if:
- Any dimension >2mm off spec
- Phone slides or part tips
- Visible cracks

\end{lstlisting}

\subsubsection*{Measurement Procedures}\label{docs__pandoc__latex__src__3dmake_foundation__appendix_c_tolerance_qa.md__measurement-procedures}

\paragraph*{Procedure 1: Linear Dimension (Width, Height, Depth)}\label{docs__pandoc__latex__src__3dmake_foundation__appendix_c_tolerance_qa.md__procedure-1-linear-dimension-width-height-depth}

Equipment needed: Digital calipers

Steps:

\begin{enumerate}
\tightlist
\item
  Place part on flat surface
\item
  Position caliper jaws perpendicular to surface
\item
  Gently close jaws until they just touch part
\item
  Feel for light resistance (not forced)
\item
  Read digital display
\item
  Record measurement
\item
  Repeat at 3 different locations
\item
  Average the three readings
\item
  Compare to design spec +/- tolerance
\end{enumerate}

Example:

\begin{lstlisting}[style=Alabaster]
Design spec: 127mm (bracelet holder width)
Tolerance:   +/-2mm (acceptable: 125-129mm)
Measurements:
  Location 1: 126.8mm
  Location 2: 127.1mm
  Location 3: 126.5mm
  Average: 126.8mm [YES] (within tolerance)

\end{lstlisting}

\paragraph*{Procedure 2: Hole or Peg Diameter}\label{docs__pandoc__latex__src__3dmake_foundation__appendix_c_tolerance_qa.md__procedure-2-hole-or-peg-diameter}

Equipment needed: Digital calipers, Go/No-Go gauges (optional)

Method A: Direct Measurement

\begin{enumerate}
\tightlist
\item
  Insert caliper jaws into hole/around peg
\item
  Adjust jaws to gently touch surfaces
\item
  Feel for light resistance on both sides
\item
  Read digital display (inside measurement mode)
\item
  Record three measurements
\item
  Average the readings
\end{enumerate}

Method B: Go/No-Go Gauge

\begin{enumerate}
\tightlist
\item
  Print test jigs (go \& no-go gauges)
\item
  Attempt to insert peg/hole into go-gauge
  -\textgreater{} Should slide through easily with light resistance
\item
  Attempt to insert peg/hole into no-go-gauge
  -\textgreater{} Should NOT fit or fit with visible resistance
\item
  If both pass -\textgreater{} dimension is acceptable
\end{enumerate}

Example: Peg Diameter

Design spec: 6.0mm Tolerance:   +/-0.5mm (acceptable: 5.5-6.5mm)
Method A (Direct): Measurement 1: 5.9mm Measurement 2: 6.0mm
Measurement 3: 5.8mm Average: 5.9mm {[}YES{]} Method B (Go/No-Go):
Slides through go-gauge (5.5mm) -\textgreater{} {[}YES{]}
Doesn\textquotesingle t fit no-go-gauge (6.5mm) -\textgreater{} {[}YES{]}
Result: PASS

\paragraph*{Procedure 3: Surface Finish / Layer Quality}\label{docs__pandoc__latex__src__3dmake_foundation__appendix_c_tolerance_qa.md__procedure-3-surface-finish--layer-quality}

Equipment needed: Caliper, ruler, touch/texture assessment

Step 1: Surface Texture

Run fingers/hand over surface:
{[}YES{]} Smooth -\textgreater{} Good quality
Slightly rough -\textgreater{} Acceptable
{[}NO{]} Very rough/bumpy -\textgreater{} Quality issue

Step 2: Layer Line Visibility

Feel horizontal ridges (layer lines):
{[}YES{]} Barely perceptible -\textgreater{} Good (0.2mm layers)
Noticeable but even -\textgreater{} OK (0.25mm layers)
{[}NO{]} Very pronounced -\textgreater{} Quality issue

Step 3: Dimensional Consistency

\begin{lstlisting}[style=Alabaster]
Measure thickness at multiple points:
  Design spec: 3mm wall thickness
  Measure at 5 locations
  Record all measurements
  All should be within +/-0.2mm of each other
  Example:
    Pt 1: 3.0mm
    Pt 2: 3.1mm
    Pt 3: 3.0mm
    Pt 4: 2.9mm
    Pt 5: 3.1mm
  Std Dev: 0.08mm [YES] (very consistent)

\end{lstlisting}

\paragraph*{Procedure 4: Weight Verification (Confirms Infill)}\label{docs__pandoc__latex__src__3dmake_foundation__appendix_c_tolerance_qa.md__procedure-4-weight-verification-confirms-infill}

Equipment needed: Digital scale

Step 1: Calculate Expected Weight

Expected weight = Volume x Density x Infill\% For bracelet holder (PLA):
Volume = 127mm x 80mm x 120mm = 1,219,200 mm = 1,219.2 cm
PLA density = 1.24 g/cm Infill = 20\%
Expected weight = 1,219.2 x 1.24 x 0.20 = 302.4g
Acceptable range: 290-315g (+/-5\%)

Step 2: Measure Actual Weight

\begin{enumerate}
\tightlist
\item
  Place part on digital scale
\item
  Wait for reading to stabilize (1-2 seconds)
\item
  Read display in grams
\item
  Compare to expected weight
\end{enumerate}

Step 3: Interpret Result

{\def\LTcaptype{none} % do not increment counter
\begin{longtable}[]{@{}ll@{}}
\toprule\noalign{}
Actual Weight & Interpretation \\
\midrule\noalign{}
\endhead
\bottomrule\noalign{}
\endlastfoot
290-315g & {[}YES{]} Correct infill, no internal voids \\
\textless{}280g & Infill too low or significant voids \\
\textgreater320g & Infill too high (was it supposed to be 20\%?) \\
\end{longtable}
}

\subsubsection*{Functional Testing}\label{docs__pandoc__latex__src__3dmake_foundation__appendix_c_tolerance_qa.md__functional-testing}

\paragraph*{Test 1: Load Testing (Strength)}\label{docs__pandoc__latex__src__3dmake_foundation__appendix_c_tolerance_qa.md__test-1-load-testing-strength}

Purpose: Verify part can hold weight without failure

Equipment needed:

\begin{itemize}
\tightlist
\item
  Digital scale
\item
  Test weights (or books, water jugs)
\item
  Calipers (to check for deflection)
\end{itemize}

Procedure:

\begin{enumerate}
\tightlist
\item
  Measure baseline dimensions (part unloaded) Baseline height: 60mm
\item
  Place test weight on part Added weight: 200g
\item
  Wait 5 minutes (allow part to settle)
\item
  Measure dimensions again New height: 59.8mm
  Deflection: 0.2mm (acceptable)
\item
  Observe for cracks (visual or tactile) No cracks: {[}YES{]}
\item
  Remove weight and wait 5 minutes
\item
  Measure dimensions again (should return to baseline)
  Height after unload: 60.0mm Recovery: Complete {[}YES{]}
  Result: PASS (part handles load without permanent deformation)
\end{enumerate}

Acceptance Criteria for Phone Stand:

{[}YES{]} PASS if:

\begin{itemize}
\tightlist
\item
  Deflection \textless{}0.5mm under 200g load
\item
  No cracks visible/felt
\item
  Full recovery after load removed {[}NO{]} FAIL if:
\item
  Deflection \textgreater1mm
\item
  Visible cracks
\item
  Permanent deformation after unload
\end{itemize}

\paragraph*{Test 2: Assembly Testing}\label{docs__pandoc__latex__src__3dmake_foundation__appendix_c_tolerance_qa.md__test-2-assembly-testing}

Purpose: Verify parts fit together as designed

Equipment needed:

\begin{itemize}
\tightlist
\item
  Calipers
\item
  Go/No-Go gauges
\end{itemize}

Procedure for Multi-Part Assembly (e.g., Stackable Bins):

Part A (Bin body) dimensions:

\begin{itemize}
\tightlist
\item
  Top opening: 50mm x 50mm +/-1mm
\item
  Wall thickness: 2mm +/-0.2mm Part B (Stacking rim) dimensions:
\item
  Base diameter: 50mm +/-0.5mm
\item
  Should nest inside Part A Test procedure:
\end{itemize}

\begin{enumerate}
\tightlist
\item
  Measure Part A opening: 50.1mm {[}YES{]}
\item
  Measure Part B base: 49.8mm {[}YES{]}
\item
  Attempt to nest Part B into Part A
  -\textgreater{} Should fit with light resistance
  -\textgreater{} Feel for smooth insertion (no catching)
\item
  Check for rocking (part should sit stable)
\item
  Apply 500g load (simulate stacking)
  -\textgreater{} Should hold without slipping
\end{enumerate}

\paragraph*{Test 3: Durability Testing (Repeated Use)}\label{docs__pandoc__latex__src__3dmake_foundation__appendix_c_tolerance_qa.md__test-3-durability-testing-repeated-use}

Purpose: Verify part doesn\textquotesingle t fail after repeated cycles

Example: Phone Stand Tilt Cycles

\begin{enumerate}
\tightlist
\item
  Record baseline dimensions
\item
  Cycle 1: Place phone, tilt to max angle, remove
\item
  Inspect for cracks or damage
\item
  Repeat cycle 10 times
\item
  Measure dimensions again
\item
  Compare to baseline: Should be \textless{}0.1mm change
  If no damage after 10 cycles -\textgreater{} Durable {[}YES{]}
\end{enumerate}

\subsubsection*{Tolerance Stack-Up (Multi-Part Designs)}\label{docs__pandoc__latex__src__3dmake_foundation__appendix_c_tolerance_qa.md__tolerance-stack-up-multi-part-designs}

When multiple parts are assembled, tolerances add:

Design: Part A opening: 50mm +/-0.5mm Part B base: 50mm +/-0.5mm
Worst-case fit: Part A minimum: 49.5mm Part B maximum: 50.5mm
Difference: 1.0mm (VERY TIGHT or won\textquotesingle t fit!)
Solution: Increase tolerance on Part B to +/-0.3mm
Part B minimum: 49.7mm Part B maximum: 50.3mm
Now fits inside Part A: 49.5-50.5mm {[}YES{]}

\paragraph*{Tolerance Stack-Up Calculation}\label{docs__pandoc__latex__src__3dmake_foundation__appendix_c_tolerance_qa.md__tolerance-stack-up-calculation}

For N parts in assembly:

\begin{lstlisting}[style=Alabaster]
Total tolerance = (tolerance + tolerance + ... + tolerance)
Example with 3 parts (each +/-0.5mm):
Total = (0.5 + 0.5 + 0.5)
      = 0.75
      = 0.87mm

\end{lstlisting}

\subsubsection*{Common Dimensional Problems \& Fixes}\label{docs__pandoc__latex__src__3dmake_foundation__appendix_c_tolerance_qa.md__common-dimensional-problems--fixes}

{\def\LTcaptype{none} % do not increment counter
\begin{longtable}[]{@{}
  >{\raggedright\arraybackslash}p{(\linewidth - 6\tabcolsep) * \real{0.2450}}
  >{\raggedright\arraybackslash}p{(\linewidth - 6\tabcolsep) * \real{0.1788}}
  >{\raggedright\arraybackslash}p{(\linewidth - 6\tabcolsep) * \real{0.3311}}
  >{\raggedright\arraybackslash}p{(\linewidth - 6\tabcolsep) * \real{0.2450}}@{}}
\toprule\noalign{}
\begin{minipage}[b]{\linewidth}\raggedright
Problem
\end{minipage} & \begin{minipage}[b]{\linewidth}\raggedright
Typical Cause
\end{minipage} & \begin{minipage}[b]{\linewidth}\raggedright
How to Fix
\end{minipage} & \begin{minipage}[b]{\linewidth}\raggedright
Prevent Next Time
\end{minipage} \\
\midrule\noalign{}
\endhead
\bottomrule\noalign{}
\endlastfoot
Part too small (all dimensions off) & Scale wrong in slicer &
Scale STL up in CAD or slicer & Verify scale before slicing \\
Holes too small & Compensation shrinkage & Increase hole diameter +0.5mm
& Add shrinkage factor to design \\
Walls too thin & Layer squishing &
Check first layer height; may need recalibration &
Measure first layer thickness \\
Infill showing through (loose fill) & Infill too low &
Increase infill \% to 25-30\% & Recalculate for part type \\
Inconsistent across print & Nozzle clogging mid-print &
Clean nozzle; check filament quality &
Use quality filament; monitor print \\
Z-axis dimensions off & Z-axis uncalibrated &
Run Z-calibration procedure & Calibrate before critical prints \\
Dimensions change between prints & Thermal drift &
Print in stable temperature & Use enclosed printer if possible \\
\end{longtable}
}

\subsubsection*{Testing Checklist Template}\label{docs__pandoc__latex__src__3dmake_foundation__appendix_c_tolerance_qa.md__testing-checklist-template}

Print this checklist before starting a new project:

\begin{lstlisting}[style=Alabaster]
# Quality Assurance Checklist: [Project Name]

## Design Specs
- [ ] Part A width:  mm +/-  mm
- [ ] Part B height:  mm +/-  mm
- [ ] Hole diameter:  mm +/-  mm
- [ ] Assembly fit tolerance:  mm

## Pre-Print
- [ ] STL exported correctly
- [ ] Scale verified
- [ ] Supports configured
- [ ] Slice file reviewed

## Post-Print (Dimensional)
- [ ] Part A width measured (3 points): mm, mm, mm
  - [ ] Within tolerance? YES/NO
- [ ] Part B height measured (3 points): mm, mm, mm
  - [ ] Within tolerance? YES/NO
- [ ] Hole diameter measured (go/no-go): PASS/FAIL
- [ ] Part weight measured: g (expected: g +/-g)
  - [ ] Within expected weight? YES/NO

## Post-Print (Functional)
- [ ] Load test (if applicable): PASS/FAIL
- [ ] Assembly test: PASS/FAIL
- [ ] Surface quality acceptable: YES/NO
- [ ] No cracks or voids: YES/NO

## Result
- [ ] ACCEPT (all tests pass)
- [ ] REWORK (minor issue, can fix)
- [ ] REJECT (major issue, reprint)

## Notes
[Space for observations]

\end{lstlisting}

\subsubsection*{PowerShell Integration: Track Quality Metrics}\label{docs__pandoc__latex__src__3dmake_foundation__appendix_c_tolerance_qa.md__powershell-integration-track-quality-metrics}

\begin{lstlisting}[style=Alabaster, language=powershell]
# Track print quality across multiple projects
$qualityLog = @"
ProjectName,Date,Material,PartCount,DimensionsPass,FunctionalPass,Weight(g),Notes
"@
# Log entries
$qualityLog += "`nBraceletHolder,2024-01-15,PLA,1,PASS,PASS,302.1,Perfect fit"
$qualityLog += "`nPhoneStand,2024-01-16,PETG,2,PASS,PASS,487.3,Strong joints"
$qualityLog += "`nStackableBins,2024-01-17,PLA,3,FAIL,N/A,N/A,Holes too small reprint"
# Save to CSV
$qualityLog | Out-File "C:\Projects\quality-log.csv"
# Analyze pass/fail rate
$log = Import-Csv "C:\Projects\quality-log.csv"
$passCount = ($log | Where-Object { $.DimensionsPass -eq "PASS" }).Count
$totalCount = $log.Count
$passRate = ($passCount / $totalCount) * 100
Write-Host "Print success rate: $passRate%"
# Show recent failures
Write-Host "`nRecent issues:"
$log | Where-Object { $.DimensionsPass -eq "FAIL" } | Select-Object ProjectName, Notes

\end{lstlisting}

\subsubsection*{Accessibility-Focused QA Best Practices}\label{docs__pandoc__latex__src__3dmake_foundation__appendix_c_tolerance_qa.md__accessibility-focused-qa-best-practices}

\paragraph*{Why Measurement-Based Testing Matters}\label{docs__pandoc__latex__src__3dmake_foundation__appendix_c_tolerance_qa.md__why-measurement-based-testing-matters}

Traditional visual QA (looking at surface finish, checking dimensions by eye) is inherently inaccessible. This appendix prioritizes:

\begin{enumerate}
\tightlist
\item
  Caliper measurements: Objective, quantifiable
\item
  Weight verification: Numerical result
\item
  Functional testing: Pass/fail criteria
\item
  Go/No-Go gauges: Tactile pass/fail
\item
  Test jigs: Can be shared/standardized
\end{enumerate}

\paragraph*{Screen Reader Integration}\label{docs__pandoc__latex__src__3dmake_foundation__appendix_c_tolerance_qa.md__screen-reader-integration}

When documenting QA results:

\begin{lstlisting}[style=Alabaster]
[YES] GOOD: "Bracket width measured 49.8mm (spec 50+/-1mm)"
[NO] AVOID: "Bracket looks good" (not measurable)
[YES] GOOD: "Part weighs 298g (spec 300+/-10g)"
[NO] AVOID: "Feels about right" (subjective)
[YES] GOOD: "Slides through go-gauge, blocks no-go-gauge"
[NO] AVOID: "Hole looks the right size" (visual only)

\end{lstlisting}

\paragraph*{Documentation Template (Accessible)}\label{docs__pandoc__latex__src__3dmake_foundation__appendix_c_tolerance_qa.md__documentation-template-accessible}

\begin{lstlisting}[style=Alabaster]
## Part: Bracelet Holder Peg (Test Date: Jan 15, 2024)

### Dimensional Verification
| Dimension    | Specification  | Measurement 1 | Measurement 2 | Measurement 3 | Average | Result     |
|--------------|----------------|---------------|---------------|---------------|---------|------------|
| Peg Diameter | 6.0 +/- 0.5mm  | 5.9mm         | 6.0mm         | 5.8mm         | 5.9mm   | [YES] PASS |
| Peg Length   | 25.0 +/- 0.5mm | 25.1mm        | 25.0mm        | 25.0mm        | 25.03mm | [YES] PASS |

### Functional Test
- [x] Slides into bracelet loop easily (no forcing)
- [x] No cracks visible or felt
- [x] Survives 1-hour load test with 20 bracelets
- [x] No permanent deformation after load removed

### Overall Result
[YES] PASS - All dimensions within tolerance; functionally sound

\end{lstlisting}

\subsubsection*{Summary}\label{docs__pandoc__latex__src__3dmake_foundation__appendix_c_tolerance_qa.md__summary}

Key Principles:

\begin{enumerate}
\tightlist
\item
  Plan before printing: Define tolerances and test methods
\item
  Measure precisely: Use calibrated tools (digital calipers, scale)
\item
  Test functionally: Will parts assemble and work?
\item
  Document quantitatively: Use numbers, not adjectives
\item
  Iterate based on data: If test fails, adjust design or process
\end{enumerate}

When to Use Each Test:

{\def\LTcaptype{none} % do not increment counter
\begin{longtable}[]{@{}ll@{}}
\toprule\noalign{}
Situation & Recommended Test \\
\midrule\noalign{}
\endhead
\bottomrule\noalign{}
\endlastfoot
Prototype/mockup & Linear dimensions only \\
Assembly with other parts & Tolerance stack-up analysis \\
Load-bearing part & Functional load test \\
Repeated-use item & Durability cycling test \\
Multi-part design & Assembly fit test \\
\end{longtable}
}

Accessibility Reminder:

All QA testing in this appendix is measurement-based and non-visual, making it equally accessible to all users. The goal is objective, quantifiable verification-not subjective assessment.

\subsection{Appendix D: Advanced OpenSCAD Concepts}\label{docs__pandoc__latex__src__3dmake_foundation__appendix_d_advanced_openscad_concepts.md__appendix-d-advanced-openscad-concepts}

This appendix covers specialized topics for experienced users seeking to tackle complex parametric designs. These are optional topics not required for foundational mastery but valuable for professional application development.

\subsubsection*{Topic 1: Gears and Mechanical Components}\label{docs__pandoc__latex__src__3dmake_foundation__appendix_d_advanced_openscad_concepts.md__topic-1-gears-and-mechanical-components}

Gears are one of the most challenging parametric designs, requiring careful calculation of tooth geometry. Understanding gear mathematics enables creation of mechanical systems-power transmission, speed reduction, or precise positioning.

\paragraph*{Gear Terminology}\label{docs__pandoc__latex__src__3dmake_foundation__appendix_d_advanced_openscad_concepts.md__gear-terminology}

\begin{itemize}
\tightlist
\item
  Pitch Diameter (PD): The reference diameter for meshing calculations
\item
  Module: PD divided by tooth count; determines tooth size
\item
  Pressure Angle: Typically 20deg or 14.5deg; affects tooth shape and strength
\item
  Clearance: Space between teeth to allow smooth meshing
\item
  Backlash: Intentional gap to prevent binding at tolerance extremes
\end{itemize}

\paragraph*{Simple Involute Gear Algorithm}\label{docs__pandoc__latex__src__3dmake_foundation__appendix_d_advanced_openscad_concepts.md__simple-involute-gear-algorithm}

\begin{lstlisting}[style=Alabaster, language=openscad]
// Simplified gear with circular teeth (adequate for most 3D prints)
function gear_module(pitch_diameter, teeth_count) = pitch_diameter / teeth_count;
module simple_gear(pitch_diameter, teeth_count, bore_diameter, thickness) {
  module_m = gear_module(pitch_diameter, teeth_count);
  outer_diameter = pitch_diameter + 2 * module_m;
  tooth_angle = 360 / teeth_count;
  difference() {
    // Main gear body
    cylinder(r=outer_diameter/2, h=thickness, $fn=teeth_count*4);
    // Center bore
    cylinder(r=bore_diameter/2, h=thickness + 2);
  }
  // Add teeth as small protrusions
  for (i = [0:teeth_count-1]) {
    angle = i * tooth_angle;
    translate([
      (pitch_diameter/2 + module_m/2) * cos(angle),
      (pitch_diameter/2 + module_m/2) * sin(angle),
      0
    ])
      rotate([0, 0, angle])
        cube([module_m * 0.8, module_m, thickness], center=true);
  }
}
// Create meshing gear pair
module gear_pair_demo() {
  // Gear 1: 20 teeth, 40mm pitch diameter
  simple_gear(40, 20, 8, 10);
  // Gear 2: 30 teeth, positioned to mesh
  center_distance = (40 + 60) / 4;  // Sum of radii
  translate([center_distance, 0, 0])
    simple_gear(60, 30, 8, 10);
}
gear_pair_demo();

\end{lstlisting}

\paragraph*{Practical Application: Servo Gearbox}\label{docs__pandoc__latex__src__3dmake_foundation__appendix_d_advanced_openscad_concepts.md__practical-application-servo-gearbox}

\begin{lstlisting}[style=Alabaster, language=openscad]
// Parametric servo gearbox with 3:1 reduction
module servo_gearbox(motor_torque, reduction_ratio, bore_size) {
  // Input gear (motor)
  input_teeth = 12;
  input_pd = 30;
  // Output gear (load)
  output_teeth = input_teeth * reduction_ratio;
  output_pd = input_pd * reduction_ratio;
  // Gearbox housing
  housing_size = output_pd + 40;
  difference() {
    // Main box
    cube([housing_size, housing_size, 30]);
    // Interior chamber
    translate([20, 20, 5])
      cube([housing_size - 40, housing_size - 40, 25]);
  }
  // Mount input gear
  translate([housing_size/4, housing_size/2, 15]) {
    simple_gear(input_pd, input_teeth, bore_size, 15);
    // Motor coupling
    cylinder(r=bore_size/2, h=5);
  }
  // Mount output gear
  translate([3*housing_size/4, housing_size/2, 15])
    simple_gear(output_pd, output_teeth, bore_size, 15);
}
servo_gearbox(10, 3, 5);

\end{lstlisting}

\paragraph*{Belt and Pulley Systems}\label{docs__pandoc__latex__src__3dmake_foundation__appendix_d_advanced_openscad_concepts.md__belt-and-pulley-systems}

\begin{lstlisting}[style=Alabaster, language=openscad]
// Timing pulley with tooth grooves
module timing_pulley(bore_diameter, pitch_diameter, teeth_count, width) {
  module_m = pitch_diameter / teeth_count;
  tooth_height = module_m * 0.3;
  difference() {
    // Main pulley body
    cylinder(r=pitch_diameter/2 + tooth_height/2, h=width);
    // Center bore
    cylinder(r=bore_diameter/2, h=width + 2);
    // Tooth grooves
    for (i = [0:teeth_count-1]) {
      angle = i * (360 / teeth_count);
      rotate([0, 0, angle])
        translate([pitch_diameter/2, -module_m*0.3, 0])
          cube([module_m * 0.6, module_m * 0.6, width], center=true);
    }
  }
}
// Belt drive demonstration
module belt_drive_system() {
  // Motor pulley (input)
  timing_pulley(5, 20, 16, 10);
  // Load pulley (output) - positioned for belt routing
  translate([80, 0, 0])
    timing_pulley(5, 40, 32, 10);
}
belt_drive_system();

\end{lstlisting}

\subsubsection*{Topic 2: Batch Processing and Statistical Analysis}\label{docs__pandoc__latex__src__3dmake_foundation__appendix_d_advanced_openscad_concepts.md__topic-2-batch-processing-and-statistical-analysis}

Professional applications often generate many variants and need to analyze results. Batch processing enables systematic exploration of parameter spaces.

\paragraph*{Parameter Sweep Generation}\label{docs__pandoc__latex__src__3dmake_foundation__appendix_d_advanced_openscad_concepts.md__parameter-sweep-generation}

\begin{lstlisting}[style=Alabaster, language=openscad]
// Generate models with all combinations of parameters
// Define parameter ranges
wall_thicknesses = [1, 1.5, 2, 2.5, 3];
part_sizes = [20, 30, 40, 50];
materials = ["pla", "petg", "nylon"];
// Theory: Generate 5 x 4 x 3 = 60 variants automatically
// In practice, export each to separate STL for analysis

\end{lstlisting}

\paragraph*{Statistical Summary Generation Script}\label{docs__pandoc__latex__src__3dmake_foundation__appendix_d_advanced_openscad_concepts.md__statistical-summary-generation-script}

\begin{lstlisting}[style=Alabaster, language=cmd]
@echo off
rem analyze_batch.bat - Analyze batch results (CMD / Batch)
setlocal enableextensions enabledelayedexpansion
set "OUTPUT_DIR=batch_results"
set "REPORT_FILE=batch_analysis.csv"
echo Part,Wall,Size,File_Size_KB,Est_Print_Time_min,Est_Weight_g > "%REPORT_FILE%"

for %%w in (1 2 3) do (
  for %%s in (20 30 40 50) do (
    set "SCAD_FILE=part_%%s_wall%%w.scad"
    set "STL_FILE=%OUTPUT_DIR%\%%~nSCAD_FILE.stl"
    "C:\Program Files\OpenSCAD\openscad.exe" -D "part_size=%%s" -D "wall=%%w" -o "%OUTPUT_DIR%\part_%%s_wall%%w.stl" src\main.scad
    if exist "%OUTPUT_DIR%\part_%%s_wall%%w.stl" (
      for /f "usebackq" %%F in (`powershell -NoProfile -Command "(Get-Item -Path '%OUTPUT_DIR%\part_%%s_wall%%w.stl').Length"`) do set FILE_BYTES=%%F
      set /a FILE_KB=FILE_BYTES / 1024
      set /a EST_TIME=(FILE_KB / 1024) * 60
      set /a EST_WEIGHT=FILE_KB / 1024
    ) else (
      set FILE_KB=0
      set EST_TIME=0
      set EST_WEIGHT=0
    )
    echo part_%%s_wall%%w,%%w,%%s,!FILE_KB!,!EST_TIME!,!EST_WEIGHT!>>"%REPORT_FILE%"
  )
)
endlocal
echo Analysis complete. Results in %REPORT_FILE%

\end{lstlisting}

\begin{lstlisting}[style=Alabaster, language=powershell]
# analyze_batch.ps1 - Analyze batch results (PowerShell)
$OutputDir = "batch_results"
$ReportFile = "batch_analysis.csv"
"Part,Wall,Size,File_Size_KB,Est_Print_Time_min,Est_Weight_g" | Out-File -FilePath $ReportFile -Encoding utf8

foreach ($wall in @(1.0,1.5,2.0,2.5,3.0)) {
  foreach ($size in @(20,30,40,50)) {
    $scad = "part_${size}_wall${wall}.scad"
    $stl = Join-Path $OutputDir ("{0}.stl" -f ([System.IO.Path]::GetFileNameWithoutExtension($scad)))
    & "C:\Program Files\OpenSCAD\openscad.exe" -D ("part_size={0}" -f $size) -D ("wall={0}" -f $wall) -o $stl src\main.scad
    if (Test-Path $stl) {
      $bytes = (Get-Item $stl).Length
      $kb = [math]::Round($bytes / 1KB, 2)
      $est_time = [math]::Round(($kb / 1024) * 60, 2)
      $est_weight = [math]::Round($kb / 1024, 2)
    } else {
      $kb = 0; $est_time = 0; $est_weight = 0
    }
    "{0},{1},{2},{3},{4},{5}" -f ("part_{0}_wall{1}" -f $size,$wall), $wall, $size, $kb, $est_time, $est_weight | Out-File -FilePath $ReportFile -Append -Encoding utf8
  }
}
Write-Host "Analysis complete. Results in $ReportFile"

\end{lstlisting}

\begin{lstlisting}[style=Alabaster, language=bash]
#!/bin/bash
# analyze_batch.sh - Analyze batch results statistically

OUTPUT_DIR="batch_results"
REPORT_FILE="batch_analysis.csv"

echo "Part,Wall,Size,File_Size_KB,Est_Print_Time_min,Est_Weight_g" > "$REPORT_FILE"

for wall in 1.0 1.5 2.0 2.5 3.0; do
  for size in 20 30 40 50; do
    # Generate model
    SCAD_FILE="part_${size}_wall${wall}.scad"
    STL_FILE="$OUTPUT_DIR/${SCAD_FILE%.scad}.stl"
    
    # Export to STL
    openscad -D "part_size=$size" -D "wall=$wall" \
      -o "$STL_FILE" src/main.scad
    
    # Get file size
    FILE_SIZE=$(stat -c%s "$STL_FILE" 2>/dev/null | awk '{print $1/1024}')
    
    # Estimate print time (rough: 1 hour per MB)
    EST_TIME=$((FILE_SIZE / 1024 * 60))
    
    # Estimate weight (rough: 1g per MB for PLA)
    EST_WEIGHT=$((FILE_SIZE / 1024))
    
    # Log results
    echo "part_${size}_wall${wall},$wall,$size,$FILE_SIZE,$EST_TIME,$EST_WEIGHT" >> "$REPORT_FILE"
  done
done

echo "Analysis complete. Results in $REPORT_FILE"

\end{lstlisting}

\paragraph*{Data-Driven Design Selection}\label{docs__pandoc__latex__src__3dmake_foundation__appendix_d_advanced_openscad_concepts.md__data-driven-design-selection}

\begin{lstlisting}[style=Alabaster, language=openscad]
// Load results and select optimal configuration
// wall=2.0mm gives best balance of strength/weight for most sizes
function optimal_configuration(target_size, priority) =
  priority == "strength" ? 3.0 :
  priority == "speed" ? 1.0 :
  priority == "balanced" ? 2.0 :
  2.0;  // Default
module batch_optimized_part(size, priority) {
  wall = optimal_configuration(size, priority);
  echo(str("Generating optimized part: size=", size, " wall=", wall, " priority=", priority));
  // Create part with optimal parameters
  difference() {
    cube([size, size, size]);
    translate([wall, wall, wall])
      cube([size - 2*wall, size - 2*wall, size]);
  }
}
// Generate three variants with different priorities
batch_optimized_part(40, "strength");    // thickest walls
translate([60, 0, 0]) batch_optimized_part(40, "speed");    // thinnest walls
translate([120, 0, 0]) batch_optimized_part(40, "balanced"); // medium walls

\end{lstlisting}

\subsubsection*{Topic 3: Performance Optimization}\label{docs__pandoc__latex__src__3dmake_foundation__appendix_d_advanced_openscad_concepts.md__topic-3-performance-optimization}

Complex models can take hours to render. Strategic optimization keeps iteration cycles fast.

\paragraph*{Measuring Render Performance}\label{docs__pandoc__latex__src__3dmake_foundation__appendix_d_advanced_openscad_concepts.md__measuring-render-performance}

\begin{lstlisting}[style=Alabaster, language=bash]
# Time how long rendering takes
time openscad -o output.stl -D "quality=32" src/main.scad
# Output: real 5m 32s, user 5m 28s, sys 0m 2s

\end{lstlisting}

\paragraph*{Resolution Parameter Strategy}\label{docs__pandoc__latex__src__3dmake_foundation__appendix_d_advanced_openscad_concepts.md__resolution-parameter-strategy}

\begin{lstlisting}[style=Alabaster, language=openscad]
// Use lower resolution during development, high during export
quality = $preview ? 16 : 64;  // 16 segments in preview, 64 in export
module efficient_sphere(radius) {
  sphere(r=radius, $fn=quality);
}
// In array operations, efficiency matters most
module circular_array(radius, count) {
  for (i = [0:count-1]) {
    angle = (360 / count) * i;
    x = radius * cos(angle);
    y = radius * sin(angle);
    translate([x, y, 0])
      efficient_sphere(5);  // Uses quality parameter
  }
}
circular_array(30, 100);  // 100 spheres: slow with quality=64!

\end{lstlisting}

\paragraph*{Cache Complex Calculations}\label{docs__pandoc__latex__src__3dmake_foundation__appendix_d_advanced_openscad_concepts.md__cache-complex-calculations}

\begin{lstlisting}[style=Alabaster, language=openscad]
// Pre-compute expensive calculations outside loops
// SLOW: Recalculates sin/cos for every iteration
module slow_spiral() {
  for (i = [0:200]) {
    angle = i * 2;
    x = 40 * cos(angle);
    y = 40 * sin(angle);
    z = i * 0.5;
    translate([x, y, z])
      sphere(r=1);
  }
}
// FAST: Pre-compute positions, then use them
spiral_positions = [for (i = [0:200]) [
  40 * cos(i * 2),
  40 * sin(i * 2),
  i * 0.5
]];
module fast_spiral() {
  for (pos = spiral_positions)
    translate(pos)
      sphere(r=1);
}
// Use the fast version
fast_spiral();

\end{lstlisting}

\paragraph*{Simplification During Preview}\label{docs__pandoc__latex__src__3dmake_foundation__appendix_d_advanced_openscad_concepts.md__simplification-during-preview}

\begin{lstlisting}[style=Alabaster, language=openscad]
// Complex model that simplifies in preview mode
module detailed_model() {
  if ($preview) {
    // Simplified version for fast preview
    cube([100, 100, 50]);
  } else {
    // Detailed version for export
    difference() {
      cube([100, 100, 50]);
      // Complex inner geometry
      for (x = [10:10:90])
        for (y = [10:10:90])
          translate([x, y, 20])
            cylinder(h=15, d=3, $fn=32);
    }
  }
}
detailed_model();

\end{lstlisting}

\paragraph*{Profile-Driven Optimization}\label{docs__pandoc__latex__src__3dmake_foundation__appendix_d_advanced_openscad_concepts.md__profile-driven-optimization}

\begin{lstlisting}[style=Alabaster, language=bash]
#!/bin/bash
# profile_render.sh - Identify slow rendering hotspots

echo "Profiling render performance..."

for fn_value in 8 16 32 64 128; do
  echo ""
  echo "Testing with \$fn=$fn_value..."
  
  /usr/bin/time -v openscad -D "fn=$fn_value" \
    -o test_$fn_value.stl src/main.scad 2>&1 | \
    grep "Maximum resident set size"
done

\end{lstlisting}

\subsubsection*{Topic 4: Print Orientation and Support Structure Algorithms}\label{docs__pandoc__latex__src__3dmake_foundation__appendix_d_advanced_openscad_concepts.md__topic-4-print-orientation-and-support-structure-algorithms}

Strategic orientation dramatically affects support material, print time, and surface quality.

\paragraph*{Strength Orientation Analysis}\label{docs__pandoc__latex__src__3dmake_foundation__appendix_d_advanced_openscad_concepts.md__strength-orientation-analysis}

\begin{lstlisting}[style=Alabaster, language=openscad]
// Different orientations affect strength differently
// SCENARIO: Bracket with cantilever load
module bracket() {
  difference() {
    cube([50, 100, 10]);
    translate([25, 50, 5]) cylinder(r=5, h=10);
  }
}
// Orientation A: Lay flat (XY plane) - WEAK for cantilever
echo("Orientation A: Flat - Low strength for cantilever");
bracket();
// Orientation B: Stand tall (Z axis) - STRONG for cantilever
translate([70, 0, 0]) {
  echo("Orientation B: Tall - High strength for cantilever");
  rotate([90, 0, 0]) bracket();
}
// Orientation C: Diagonal - MODERATE strength
translate([0, 150, 0]) {
  echo("Orientation C: Diagonal - Moderate strength");
  rotate([45, 0, 0]) bracket();
}

\end{lstlisting}

\paragraph*{Support Material Minimization}\label{docs__pandoc__latex__src__3dmake_foundation__appendix_d_advanced_openscad_concepts.md__support-material-minimization}

\begin{lstlisting}[style=Alabaster, language=openscad]
// Design features to reduce supports needed
// POOR: Overhanging feature needs extensive supports
module poor_design() {
  cube([50, 50, 50]);
  translate([25, 25, 50])
    cube([30, 30, 10]);  // 30mm overhang!
}
// GOOD: Bridge angle keeps overhang under 45deg
module good_design() {
  cube([50, 50, 50]);
  // Add ramp instead of overhang
  translate([20, 20, 50])
    rotate([20, 0, 0])
      cube([30, 30, 10]);
}
// Compare
poor_design();
translate([80, 0, 0])
  good_design();

\end{lstlisting}

\paragraph*{Bridge Span Calculation}\label{docs__pandoc__latex__src__3dmake_foundation__appendix_d_advanced_openscad_concepts.md__bridge-span-calculation}

\begin{lstlisting}[style=Alabaster, language=openscad]
// Calculate maximum unsupported span for given material/settings
function max_bridge_span(material) =
  material == "pla" ? 10 :
  material == "petg" ? 12 :
  material == "nylon" ? 15 :
  material == "abs" ? 8 :
  10;  // Default
module bridged_connector(material, gap_width) {
  max_span = max_bridge_span(material);
  if (gap_width <= max_span) {
    echo(str("Bridge OK: gap ", gap_width, "mm fits within ", max_span, "mm limit"));
    // Create part with bridge
    union() {
      cube([50, 10, 10]);  // Side A
      translate([gap_width, 0, 0])
        cube([50, 10, 10]);  // Side B
      // Bridge connecting them
      translate([0, 5, 10])
        cube([gap_width + 100, 2, 1]);
    }
  } else {
    echo(str("ERROR: gap ", gap_width, "mm exceeds ", max_span, "mm max bridge span"));
  }
}
bridged_connector("pla", 8);      // OK: 8 < 10
// bridged_connector("pla", 15);  // ERROR: 15 > 10

\end{lstlisting}

\paragraph*{Slicing Parameter Calculation}\label{docs__pandoc__latex__src__3dmake_foundation__appendix_d_advanced_openscad_concepts.md__slicing-parameter-calculation}

\begin{lstlisting}[style=Alabaster, language=openscad]
// Calculate optimal slicing parameters based on geometry
function recommended_layer_height(nozzle_diameter, quality_priority) =
  quality_priority == "fast" ? nozzle_diameter * 0.75 :
  quality_priority == "balanced" ? nozzle_diameter * 0.5 :
  quality_priority == "detailed" ? nozzle_diameter * 0.25 :
  nozzle_diameter * 0.5;
function recommended_infill(structural_load, safety_factor) =
  structural_load < 1 ? 10 :    // Decorative: minimal fill
  structural_load < 5 ? 20 :    // Light duty: low fill
  structural_load < 20 ? 50 :   // Medium duty: standard fill
  100;                           // High duty: solid
module optimized_part(
  nozzle_0p4,
  quality_priority,
  structural_load,
  safety_factor
) {
  layer_h = recommended_layer_height(nozzle_0p4, quality_priority);
  infill_pct = recommended_infill(structural_load, safety_factor);
  echo(str("Recommended layer height: ", layer_h, "mm"));
  echo(str("Recommended infill: ", infill_pct, "%"));
  // Create part
  cube([50, 50, 30]);
}
optimized_part(0.4, "detailed", 15, 2);

\end{lstlisting}

\subsubsection*{Topic 5: Recursive Function Patterns}\label{docs__pandoc__latex__src__3dmake_foundation__appendix_d_advanced_openscad_concepts.md__topic-5-recursive-function-patterns}

Recursion enables elegant solutions to problems with self-similar structure: trees, fractals, nested components.

\paragraph*{Basic Recursive Pattern}\label{docs__pandoc__latex__src__3dmake_foundation__appendix_d_advanced_openscad_concepts.md__basic-recursive-pattern}

\begin{lstlisting}[style=Alabaster, language=openscad]
// Tree structure: branches recursively smaller
function tree_depth(level) = level > 0 ? level + tree_depth(level - 1) : 0;
// Calculate: tree_depth(5) = 5 + 4 + 3 + 2 + 1 = 15
result = tree_depth(5);
echo(result);  // Prints: 15
// Recursive tree drawing
module tree_branch(length, angle, recursion_depth) {
  if (recursion_depth > 0) {
    // Draw this branch
    rotate([angle, 0, 0])
      cube([2, 2, length]);
    // Recursively draw sub-branches
    translate([0, 0, length])
      rotate([-angle/2, 0, 0])
        tree_branch(length * 0.7, angle, recursion_depth - 1);
    translate([0, 0, length])
      rotate([angle/2, 0, 0])
        tree_branch(length * 0.7, angle, recursion_depth - 1);
  }
}
tree_branch(30, 25, 4);  // Tree with 4 levels of recursion

\end{lstlisting}

\paragraph*{Fractal Generation}\label{docs__pandoc__latex__src__3dmake_foundation__appendix_d_advanced_openscad_concepts.md__fractal-generation}

\begin{lstlisting}[style=Alabaster, language=openscad]
// Sierpinski Triangle fractal
function sierpinski_triangle(size, depth) =
  depth == 0 ?
    [[0, 0], [size, 0], [size/2, size * sqrt(3)/2]] :
    // Recursive case: three smaller triangles at corners
    concat(
      sierpinski_triangle(size/2, depth - 1),
      sierpinski_triangle(size/2, depth - 1),
      sierpinski_triangle(size/2, depth - 1)
    );
// Create 3D fractal structure
module fractal_spiral(base_size, depth, height_per_level) {
  if (depth > 0) {
    // Create current level
    cube([base_size, base_size, 5], center=true);
    // Recursively create smaller level on top
    translate([0, 0, height_per_level])
      scale([0.6, 0.6, 1])
        fractal_spiral(base_size, depth - 1, height_per_level);
  }
}
fractal_spiral(40, 5, 10);  // 5-level spiral

\end{lstlisting}

\paragraph*{Nested Component Assembly}\label{docs__pandoc__latex__src__3dmake_foundation__appendix_d_advanced_openscad_concepts.md__nested-component-assembly}

\begin{lstlisting}[style=Alabaster, language=openscad]
// Russian doll boxes: each box contains a smaller copy
module russian_doll(size, wall, nesting_depth) {
  if (nesting_depth > 0) {
    // Current level: hollow box
    difference() {
      cube([size, size, size]);
      translate([wall, wall, wall])
        cube([size - 2*wall, size - 2*wall, size - wall]);
    }
    // Recursively place next smaller doll inside
    interior_size = size - 4*wall;
    translate([2*wall, 2*wall, wall])
      russian_doll(interior_size, wall, nesting_depth - 1);
  } else {
    // Smallest doll (solid)
    cube([size, size, size]);
  }
}
russian_doll(60, 2, 4);  // 4 nested boxes

\end{lstlisting}

\paragraph*{Performance Considerations}\label{docs__pandoc__latex__src__3dmake_foundation__appendix_d_advanced_openscad_concepts.md__performance-considerations}

\begin{lstlisting}[style=Alabaster, language=openscad]
// Recursion can be expensive - monitor depth
// EFFICIENT: Tail recursion (result computed immediately)
function sum_to(n, accumulator) =
  n == 0 ? accumulator :
  sum_to(n - 1, accumulator + n);
result = sum_to(100, 0);  // Computes: 5050
// INEFFICIENT: Deep recursion without optimization
function fibonacci(n) =
  n <= 1 ? n :
  fibonacci(n-1) + fibonacci(n-2);
// fibonacci(20) recalculates sub-problems thousands of times
// Don't use for depth > 20 without memoization
// BETTER: Use iteration where possible
function fibonacci_iter(n) =
  let(
    fib = [for (i = [0:n])
      i <= 1 ? i :
      let(prev = [for (j = [0:i-1]) i==j ? 1 : i==j+1 ? 0 : 0])
        1  // Placeholder
    ]
  ) fib[n];

\end{lstlisting}

\paragraph*{Practical Example: Cable Management}\label{docs__pandoc__latex__src__3dmake_foundation__appendix_d_advanced_openscad_concepts.md__practical-example-cable-management}

\begin{lstlisting}[style=Alabaster, language=openscad]
// Recursive cable tray with branching paths
module cable_tray(width, depth, height, levels, branch_angle) {
  if (levels > 0) {
    // Main tray section
    difference() {
      cube([width, depth, height]);
      translate([2, 2, 2])
        cube([width - 4, depth - 4, height - 2]);
    }
    // Left branch (recursively smaller)
    translate([-width/2, 0, height])
      rotate([0, 0, -branch_angle])
        cable_tray(width * 0.6, depth, height, levels - 1, branch_angle);
    // Right branch
    translate([width/2, 0, height])
      rotate([0, 0, branch_angle])
        cable_tray(width * 0.6, depth, height, levels - 1, branch_angle);
  }
}
cable_tray(40, 10, 5, 3, 30);  // 3-level branching cable tray

\end{lstlisting}

\subsubsection*{Summary: When to Use Each Advanced Technique}\label{docs__pandoc__latex__src__3dmake_foundation__appendix_d_advanced_openscad_concepts.md__summary-when-to-use-each-advanced-technique}

{\def\LTcaptype{none} % do not increment counter
\begin{longtable}[]{@{}
  >{\raggedright\arraybackslash}p{(\linewidth - 6\tabcolsep) * \real{0.2989}}
  >{\raggedright\arraybackslash}p{(\linewidth - 6\tabcolsep) * \real{0.3563}}
  >{\raggedright\arraybackslash}p{(\linewidth - 6\tabcolsep) * \real{0.1379}}
  >{\raggedright\arraybackslash}p{(\linewidth - 6\tabcolsep) * \real{0.2069}}@{}}
\toprule\noalign{}
\begin{minipage}[b]{\linewidth}\raggedright
Technique
\end{minipage} & \begin{minipage}[b]{\linewidth}\raggedright
Use Case
\end{minipage} & \begin{minipage}[b]{\linewidth}\raggedright
Complexity
\end{minipage} & \begin{minipage}[b]{\linewidth}\raggedright
Performance
\end{minipage} \\
\midrule\noalign{}
\endhead
\bottomrule\noalign{}
\endlastfoot
Gears & Mechanical power transmission & High & Medium \\
Batch Processing & Design space exploration & Medium &
Depends on scope \\
Performance Optimization & Reducing render time & Medium &
High payoff \\
Orientation Analysis & Strength optimization & Medium &
Simulation cost \\
Recursion & Fractal/nested structures & High & Can be expensive \\
\end{longtable}
}

\subsubsection*{References and Further Learning}\label{docs__pandoc__latex__src__3dmake_foundation__appendix_d_advanced_openscad_concepts.md__references-and-further-learning}

\begin{itemize}
\tightlist
\item
  OpenSCAD Documentation - Advanced Features: \url{https://en.wikibooks.org/wiki/OpenSCAD_User_Manual}
\item
  Gear Design Theory: \url{https://www.mekanizmalar.com/}
\item
  3D Print Orientation Optimization: \url{https://stratasys.com/}
\item
  Fractal Geometry: \url{https://en.wikipedia.org/wiki/Fractal}
\item
  Recursive Algorithms: \url{https://www.khanacademy.org/computing/computer-science/algorithms}
\end{itemize}

For Educators and Students

These advanced topics are intended for:

\begin{itemize}
\tightlist
\item
  Experienced users tackling professional applications
\item
  Specialized problem domains (robotics, mechanical design, etc.)
\item
  Research and optimization work
\item
  Custom library and framework development
\end{itemize}

For accessibility:

\begin{itemize}
\tightlist
\item
  All recursive examples include base cases clearly marked
\item
  Each section provides simplified versions before advanced variants
\item
  Code comments explain both "what" and "why"
\item
  Practical examples show real-world applications
\item
  Performance considerations documented to prevent frustration
\end{itemize}

\section{PTECHNICAL REFERENCE}\label{docs__pandoc__latex__src__backmatter__troubleshooting.md__ptechnical-reference}

\subsection*{Parametric 3D Fabrication Systems}\label{docs__pandoc__latex__src__backmatter__troubleshooting.md__parametric-3d-fabrication-systems}

\subsubsection*{OpenSCAD • 3dMake • PowerShell • Command Prompt • Git Bash}\label{docs__pandoc__latex__src__backmatter__troubleshooting.md__openscad--3dmake--powershell--command-prompt--git-bash}

\begin{center}\rule{0.5\linewidth}{0.5pt}\end{center}

\section*{PART I --- SYSTEMS ARCHITECTURE}\label{docs__pandoc__latex__src__backmatter__troubleshooting.md__part-i--systems-architecture}

\subsection*{1.1 Digital Fabrication Pipeline}\label{docs__pandoc__latex__src__backmatter__troubleshooting.md__11-digital-fabrication-pipeline}

\begin{lstlisting}[style=Alabaster]
Source (.scad)

    ↓

OpenSCAD Engine

    ↓

STL (Mesh Geometry)

    ↓

Slicer

    ↓

G-code

    ↓

FDM Printer Firmware

    ↓

Physical Object


\end{lstlisting}

Automation Overlay:

\begin{lstlisting}[style=Alabaster]
CLI → Script → Build Directory → Version Control → Reproducible Artifact


\end{lstlisting}

\begin{center}\rule{0.5\linewidth}{0.5pt}\end{center}

\section*{PART II --- OPENSCAD PROFESSIONAL MODELING REFERENCE}\label{docs__pandoc__latex__src__backmatter__troubleshooting.md__part-ii--openscad-professional-modeling-reference}

\subsection*{2.1 Language Architecture}\label{docs__pandoc__latex__src__backmatter__troubleshooting.md__21-language-architecture}

OpenSCAD is:

\begin{itemize}
\tightlist
\item
  Declarative
\item
  CSG-based
\item
  Deterministic
\item
  Single-pass evaluation
\item
  Non-interactive
\end{itemize}

Implication: Models must be architected intentionally.

\begin{center}\rule{0.5\linewidth}{0.5pt}\end{center}

\subsection*{2.2 Advanced Geometry Techniques}\label{docs__pandoc__latex__src__backmatter__troubleshooting.md__22-advanced-geometry-techniques}

\subsubsection*{2.2.1 Transform Stack Order}\label{docs__pandoc__latex__src__backmatter__troubleshooting.md__221-transform-stack-order}

Transformations apply from inside outward.

\begin{lstlisting}[style=Alabaster, language=openscad]
translate([10,0,0])

    rotate([0,0,45])

        cube(10);


\end{lstlisting}

Order matters.

\begin{center}\rule{0.5\linewidth}{0.5pt}\end{center}

\subsubsection*{2.2.2 Hull()}\label{docs__pandoc__latex__src__backmatter__troubleshooting.md__222-hull}

Creates convex hull of objects.

\begin{lstlisting}[style=Alabaster, language=openscad]
hull() {

    translate([0,0,0]) sphere(5);

    translate([20,0,0]) sphere(5);

}


\end{lstlisting}

\begin{center}\rule{0.5\linewidth}{0.5pt}\end{center}

\subsubsection*{2.2.3 Minkowski()}\label{docs__pandoc__latex__src__backmatter__troubleshooting.md__223-minkowski}

Expands object by shape kernel.

\begin{lstlisting}[style=Alabaster, language=openscad]
minkowski() {

    cube([10,10,10]);

    sphere(1);

}


\end{lstlisting}

Warning: Computationally expensive.

\begin{center}\rule{0.5\linewidth}{0.5pt}\end{center}

\subsubsection*{2.2.4 Linear and Rotational Extrusion}\label{docs__pandoc__latex__src__backmatter__troubleshooting.md__224-linear-and-rotational-extrusion}

\begin{lstlisting}[style=Alabaster, language=openscad]
linear_extrude(height=10)

    square(5);


rotate_extrude()

    translate([10,0,0])

        circle(5);


\end{lstlisting}

\begin{center}\rule{0.5\linewidth}{0.5pt}\end{center}

\subsubsection*{2.2.5 Projection}\label{docs__pandoc__latex__src__backmatter__troubleshooting.md__225-projection}

\begin{lstlisting}[style=Alabaster, language=openscad]
projection(cut=true)

    sphere(10);


\end{lstlisting}

\begin{center}\rule{0.5\linewidth}{0.5pt}\end{center}

\subsection*{2.3 CLI Operation of OpenSCAD}\label{docs__pandoc__latex__src__backmatter__troubleshooting.md__23-cli-operation-of-openscad}

\subsubsection*{Render STL (All shells)}\label{docs__pandoc__latex__src__backmatter__troubleshooting.md__render-stl-all-shells}

\begin{lstlisting}[style=Alabaster, language=bash]
openscad -o output.stl input.scad


\end{lstlisting}

\subsubsection*{Specify Variables via CLI}\label{docs__pandoc__latex__src__backmatter__troubleshooting.md__specify-variables-via-cli}

\begin{lstlisting}[style=Alabaster, language=bash]
openscad -D width=50 -o model.stl model.scad


\end{lstlisting}

\begin{center}\rule{0.5\linewidth}{0.5pt}\end{center}

\section*{PART III --- 3DMAKE PROFESSIONAL USAGE}\label{docs__pandoc__latex__src__backmatter__troubleshooting.md__part-iii--3dmake-professional-usage}

\subsection*{3.1 3dMake Overview}\label{docs__pandoc__latex__src__backmatter__troubleshooting.md__31-3dmake-overview}

3dMake automates model generation and batch workflows.

Core capabilities:

\begin{itemize}
\tightlist
\item
  Batch STL generation
\item
  File processing pipelines
\item
  Integration with OpenSCAD CLI
\item
  Script-driven builds
\end{itemize}

Repository: https://github.com/tdeck/3dmake

\begin{center}\rule{0.5\linewidth}{0.5pt}\end{center}

\subsection*{3.2 Example Automated Pipeline (PowerShell)}\label{docs__pandoc__latex__src__backmatter__troubleshooting.md__32-example-automated-pipeline-powershell}

\begin{lstlisting}[style=Alabaster, language=powershell]
New-Item -ItemType Directory -Force build

openscad -D width=40 -o build/model.stl src/model.scad

if ($LASTEXITCODE -ne 0) { exit 1 }


\end{lstlisting}

\begin{center}\rule{0.5\linewidth}{0.5pt}\end{center}

\subsection*{3.3 Batch Processing (cmd.exe)}\label{docs__pandoc__latex__src__backmatter__troubleshooting.md__33-batch-processing-cmdexe}

\begin{lstlisting}[style=Alabaster, language=cmd]
for %%f in (src\*.scad) do openscad -o build\%%~nf.stl %%f


\end{lstlisting}

\begin{center}\rule{0.5\linewidth}{0.5pt}\end{center}

\subsection*{3.4 Batch Processing (Git Bash)}\label{docs__pandoc__latex__src__backmatter__troubleshooting.md__34-batch-processing-git-bash}

\begin{lstlisting}[style=Alabaster, language=bash]
for f in src/*.scad; do

  openscad -o build/$(basename "$f" .scad).stl "$f"

done


\end{lstlisting}

\begin{center}\rule{0.5\linewidth}{0.5pt}\end{center}

\section*{PART IV --- WINDOWS COMMAND PROMPT (CMD.EXE) FULL REFERENCE}\label{docs__pandoc__latex__src__backmatter__troubleshooting.md__part-iv--windows-command-prompt-cmdexe-full-reference}

\subsection*{4.1 cmd Architecture}\label{docs__pandoc__latex__src__backmatter__troubleshooting.md__41-cmd-architecture}

\begin{itemize}
\tightlist
\item
  Text-based
\item
  Legacy DOS-compatible syntax
\item
  String-based (not object-based)
\end{itemize}

Executable:

\begin{lstlisting}[style=Alabaster]
C:\Windows\System32\cmd.exe


\end{lstlisting}

\begin{center}\rule{0.5\linewidth}{0.5pt}\end{center}

\subsection*{4.2 File System Operations}\label{docs__pandoc__latex__src__backmatter__troubleshooting.md__42-file-system-operations}

\subsubsection*{List Files}\label{docs__pandoc__latex__src__backmatter__troubleshooting.md__list-files}

\begin{lstlisting}[style=Alabaster, language=cmd]
dir


\end{lstlisting}

\subsubsection*{Change Directory}\label{docs__pandoc__latex__src__backmatter__troubleshooting.md__change-directory}

\begin{lstlisting}[style=Alabaster, language=cmd]
cd builds


\end{lstlisting}

\subsubsection*{Create Directory}\label{docs__pandoc__latex__src__backmatter__troubleshooting.md__create-directory}

\begin{lstlisting}[style=Alabaster, language=cmd]
mkdir build


\end{lstlisting}

\subsubsection*{Remove Directory}\label{docs__pandoc__latex__src__backmatter__troubleshooting.md__remove-directory}

\begin{lstlisting}[style=Alabaster, language=cmd]
rmdir /s /q build


\end{lstlisting}

\begin{center}\rule{0.5\linewidth}{0.5pt}\end{center}

\subsection*{4.3 Running Programs}\label{docs__pandoc__latex__src__backmatter__troubleshooting.md__43-running-programs}

\begin{lstlisting}[style=Alabaster, language=cmd]
openscad.exe -o model.stl model.scad


\end{lstlisting}

\begin{center}\rule{0.5\linewidth}{0.5pt}\end{center}

\subsection*{4.4 PATH Inspection}\label{docs__pandoc__latex__src__backmatter__troubleshooting.md__44-path-inspection}

\begin{lstlisting}[style=Alabaster, language=cmd]
echo %PATH%


\end{lstlisting}

\begin{center}\rule{0.5\linewidth}{0.5pt}\end{center}

\subsection*{4.5 Temporary Environment Variables}\label{docs__pandoc__latex__src__backmatter__troubleshooting.md__45-temporary-environment-variables}

\begin{lstlisting}[style=Alabaster, language=cmd]
set MYVAR=value


\end{lstlisting}

\begin{center}\rule{0.5\linewidth}{0.5pt}\end{center}

\subsection*{4.6 Exit Codes}\label{docs__pandoc__latex__src__backmatter__troubleshooting.md__46-exit-codes}

\begin{lstlisting}[style=Alabaster, language=cmd]
echo %ERRORLEVEL%


\end{lstlisting}

Conditional:

\begin{lstlisting}[style=Alabaster, language=cmd]
if %ERRORLEVEL% NEQ 0 echo Build failed


\end{lstlisting}

\begin{center}\rule{0.5\linewidth}{0.5pt}\end{center}

\subsection*{4.7 Batch Files (.bat)}\label{docs__pandoc__latex__src__backmatter__troubleshooting.md__47-batch-files-bat}

Example build.bat:

\begin{lstlisting}[style=Alabaster, language=cmd]
@echo off

mkdir build

openscad -o build\model.stl src\model.scad

if %ERRORLEVEL% NEQ 0 exit /b 1


\end{lstlisting}

\begin{center}\rule{0.5\linewidth}{0.5pt}\end{center}

\section*{PART V --- POWERSHELL PROFESSIONAL REFERENCE}\label{docs__pandoc__latex__src__backmatter__troubleshooting.md__part-v--powershell-professional-reference}

\subsection*{5.1 Architecture}\label{docs__pandoc__latex__src__backmatter__troubleshooting.md__51-architecture}

\begin{itemize}
\tightlist
\item
  Object-based pipeline
\item
  .NET-backed
\item
  Strong scripting support
\end{itemize}

\begin{center}\rule{0.5\linewidth}{0.5pt}\end{center}

\subsection*{5.2 Execution Policy}\label{docs__pandoc__latex__src__backmatter__troubleshooting.md__52-execution-policy}

\begin{lstlisting}[style=Alabaster, language=powershell]
Get-ExecutionPolicy

Set-ExecutionPolicy RemoteSigned -Scope CurrentUser


\end{lstlisting}

\begin{center}\rule{0.5\linewidth}{0.5pt}\end{center}

\subsection*{5.3 Structured Error Handling}\label{docs__pandoc__latex__src__backmatter__troubleshooting.md__53-structured-error-handling}

\begin{lstlisting}[style=Alabaster, language=powershell]
try {

    openscad -o model.stl model.scad

}

catch {

    Write-Host "Failure"

}


\end{lstlisting}

\begin{center}\rule{0.5\linewidth}{0.5pt}\end{center}

\section*{PART VI --- GIT BASH (UNIX-LIKE SHELL)}\label{docs__pandoc__latex__src__backmatter__troubleshooting.md__part-vi--git-bash-unix-like-shell}

\subsection*{6.1 POSIX-Compatible Commands}\label{docs__pandoc__latex__src__backmatter__troubleshooting.md__61-posix-compatible-commands}

\begin{lstlisting}[style=Alabaster, language=bash]
ls

pwd

mkdir

rm -rf build


\end{lstlisting}

\begin{center}\rule{0.5\linewidth}{0.5pt}\end{center}

\subsection*{6.2 Shell Variables}\label{docs__pandoc__latex__src__backmatter__troubleshooting.md__62-shell-variables}

\begin{lstlisting}[style=Alabaster, language=bash]
VAR=value

echo $VAR


\end{lstlisting}

\begin{center}\rule{0.5\linewidth}{0.5pt}\end{center}

\section*{PART VII --- ADVANCED TROUBLESHOOTING COMPENDIUM}\label{docs__pandoc__latex__src__backmatter__troubleshooting.md__part-vii--advanced-troubleshooting-compendium}

\begin{center}\rule{0.5\linewidth}{0.5pt}\end{center}

\section*{A. OpenSCAD Failures}\label{docs__pandoc__latex__src__backmatter__troubleshooting.md__a-openscad-failures}

\subsection*{A.1 Segmentation Fault}\label{docs__pandoc__latex__src__backmatter__troubleshooting.md__a1-segmentation-fault}

Cause:

\begin{itemize}
\tightlist
\item
  Deep Minkowski
\item
  Memory exhaustion
\end{itemize}

Fix:

\begin{itemize}
\tightlist
\item
  Reduce \$fn
\item
  Simplify geometry
\end{itemize}

\begin{center}\rule{0.5\linewidth}{0.5pt}\end{center}

\subsection*{A.2 Empty STL Output}\label{docs__pandoc__latex__src__backmatter__troubleshooting.md__a2-empty-stl-output}

Cause:

\begin{itemize}
\tightlist
\item
  Geometry evaluates to null
\end{itemize}

Diagnosis: Add temporary primitive to confirm render pipeline.

\begin{center}\rule{0.5\linewidth}{0.5pt}\end{center}

\subsection*{A.3 Boolean Artifacts}\label{docs__pandoc__latex__src__backmatter__troubleshooting.md__a3-boolean-artifacts}

Cause:

\begin{itemize}
\tightlist
\item
  Coplanar faces
\end{itemize}

Fix:

\begin{itemize}
\tightlist
\item
  Add small offsets (0.01mm)
\end{itemize}

\begin{center}\rule{0.5\linewidth}{0.5pt}\end{center}

\section*{B. 3dMake Failures}\label{docs__pandoc__latex__src__backmatter__troubleshooting.md__b-3dmake-failures}

\subsection*{B.1 Command Not Found}\label{docs__pandoc__latex__src__backmatter__troubleshooting.md__b1-command-not-found}

Verify:

\begin{lstlisting}[style=Alabaster, language=cmd]
where openscad


\end{lstlisting}

\begin{center}\rule{0.5\linewidth}{0.5pt}\end{center}

\subsection*{B.2 Silent Failure}\label{docs__pandoc__latex__src__backmatter__troubleshooting.md__b2-silent-failure}

Check exit code:

\begin{lstlisting}[style=Alabaster, language=cmd]
echo %ERRORLEVEL%


\end{lstlisting}

\begin{center}\rule{0.5\linewidth}{0.5pt}\end{center}

\section*{C. CMD-Specific Failures}\label{docs__pandoc__latex__src__backmatter__troubleshooting.md__c-cmd-specific-failures}

\subsection*{C.1 "Access is denied"}\label{docs__pandoc__latex__src__backmatter__troubleshooting.md__c1-access-is-denied}

Cause:

\begin{itemize}
\tightlist
\item
  Insufficient privileges
\end{itemize}

Fix: Run as Administrator.

\begin{center}\rule{0.5\linewidth}{0.5pt}\end{center}

\subsection*{C.2 Incorrect Variable Expansion in Loop}\label{docs__pandoc__latex__src__backmatter__troubleshooting.md__c2-incorrect-variable-expansion-in-loop}

In batch files use:

\begin{lstlisting}[style=Alabaster]
%%f


\end{lstlisting}

In interactive shell use:

\begin{lstlisting}[style=Alabaster]
%f


\end{lstlisting}

\begin{center}\rule{0.5\linewidth}{0.5pt}\end{center}

\section*{D. PowerShell-Specific Failures}\label{docs__pandoc__latex__src__backmatter__troubleshooting.md__d-powershell-specific-failures}

\subsection*{D.1 Script Not Running}\label{docs__pandoc__latex__src__backmatter__troubleshooting.md__d1-script-not-running}

Fix:

\begin{lstlisting}[style=Alabaster, language=powershell]
Set-ExecutionPolicy RemoteSigned


\end{lstlisting}

\begin{center}\rule{0.5\linewidth}{0.5pt}\end{center}

\subsection*{D.2 Command Returns But File Not Created}\label{docs__pandoc__latex__src__backmatter__troubleshooting.md__d2-command-returns-but-file-not-created}

Check:

\begin{lstlisting}[style=Alabaster, language=powershell]
Test-Path build\model.stl


\end{lstlisting}

\begin{center}\rule{0.5\linewidth}{0.5pt}\end{center}

\section*{E. Git Bash Issues}\label{docs__pandoc__latex__src__backmatter__troubleshooting.md__e-git-bash-issues}

\subsection*{E.1 Windows Path Confusion}\label{docs__pandoc__latex__src__backmatter__troubleshooting.md__e1-windows-path-confusion}

Use forward slashes:

\begin{lstlisting}[style=Alabaster, language=bash]
openscad -o build/model.stl src/model.scad


\end{lstlisting}

\begin{center}\rule{0.5\linewidth}{0.5pt}\end{center}

\section*{F. 3D Printing Mechanical Failures}\label{docs__pandoc__latex__src__backmatter__troubleshooting.md__f-3d-printing-mechanical-failures}

\subsection*{F.1 Warping}\label{docs__pandoc__latex__src__backmatter__troubleshooting.md__f1-warping}

Solutions:

\begin{itemize}
\tightlist
\item
  Increase bed temp
\item
  Use enclosure
\item
  Add brim
\end{itemize}

\begin{center}\rule{0.5\linewidth}{0.5pt}\end{center}

\subsection*{F.2 Under-Extrusion}\label{docs__pandoc__latex__src__backmatter__troubleshooting.md__f2-under-extrusion}

Cause:

\begin{itemize}
\tightlist
\item
  Clogged nozzle
\item
  Incorrect flow rate
\end{itemize}

\begin{center}\rule{0.5\linewidth}{0.5pt}\end{center}

\subsection*{F.3 Dimensional Error}\label{docs__pandoc__latex__src__backmatter__troubleshooting.md__f3-dimensional-error}

Test with calibration cube.

\begin{center}\rule{0.5\linewidth}{0.5pt}\end{center}

\section*{G. Performance Diagnostics}\label{docs__pandoc__latex__src__backmatter__troubleshooting.md__g-performance-diagnostics}

\subsection*{G.1 Slow Render}\label{docs__pandoc__latex__src__backmatter__troubleshooting.md__g1-slow-render}

Reduce:

\begin{itemize}
\tightlist
\item
  \$fn
\item
  Minkowski usage
\item
  Nested difference()
\end{itemize}

\begin{center}\rule{0.5\linewidth}{0.5pt}\end{center}

\subsection*{G.2 High CPU Usage}\label{docs__pandoc__latex__src__backmatter__troubleshooting.md__g2-high-cpu-usage}

Expected during full render.

\begin{center}\rule{0.5\linewidth}{0.5pt}\end{center}

\section*{H. SYSTEM DIAGNOSTIC CHECKLIST}\label{docs__pandoc__latex__src__backmatter__troubleshooting.md__h-system-diagnostic-checklist}

\begin{enumerate}
\tightlist
\item
  Is OpenSCAD in PATH?
\item
  Does CLI produce STL?
\item
  Is STL manifold?
\item
  Does slicer preview correctly?
\item
  Is G-code generated?
\item
  Does printer firmware accept G-code?
\item
  Is first layer adhering?
\item
  Are dimensions within tolerance?
\end{enumerate}

\begin{center}\rule{0.5\linewidth}{0.5pt}\end{center}

\section*{I. FULL BUILD VALIDATION SCRIPT (CMD)}\label{docs__pandoc__latex__src__backmatter__troubleshooting.md__i-full-build-validation-script-cmd}

\begin{lstlisting}[style=Alabaster, language=cmd]
@echo off

echo Validating environment...

where openscad >nul 2>nul

if %ERRORLEVEL% NEQ 0 (

    echo OpenSCAD not found.

    exit /b 1

)


mkdir build

openscad -o build\model.stl src\model.scad

if %ERRORLEVEL% NEQ 0 (

    echo Build failed.

    exit /b 1

)


echo Build succeeded.


\end{lstlisting}

\section{Student  Glossary}\label{docs__pandoc__latex__src__backmatter__glossary.md__student--glossary}

\emph{OpenSCAD • 3dMake • 3D Printing • PowerShell • Command Prompt • Git Bash}

This glossary provides:

\begin{itemize}
\tightlist
\item
  Formal definitions
\item
  Operational context
\item
  Cross-system distinctions
\item
  CLI examples
\item
  Practical implications in 3D printing workflows
\end{itemize}

\begin{center}\rule{0.5\linewidth}{0.5pt}\end{center}

\section*{SECTION I --- Additive Manufacturing \& Fabrication}\label{docs__pandoc__latex__src__backmatter__glossary.md__section-i--additive-manufacturing--fabrication}

\subsection*{Additive Manufacturing}\label{docs__pandoc__latex__src__backmatter__glossary.md__additive-manufacturing}

A manufacturing methodology in which objects are fabricated layer-by-layer from digital models.\\
In desktop contexts, this typically refers to FDM (Fused Deposition Modeling).

\subsubsection*{Technical Characteristics}\label{docs__pandoc__latex__src__backmatter__glossary.md__technical-characteristics}

\begin{itemize}
\tightlist
\item
  Layer-based deposition
\item
  Toolpath-generated geometry
\item
  STL-to-G-code workflow
\item
  Thermoplastic extrusion (in FDM)
\end{itemize}

\subsubsection*{Workflow Position}\label{docs__pandoc__latex__src__backmatter__glossary.md__workflow-position}

OpenSCAD → STL → Slicer → G-code → Printer → Physical object

\begin{center}\rule{0.5\linewidth}{0.5pt}\end{center}

\subsection*{FDM (Fused Deposition Modeling)}\label{docs__pandoc__latex__src__backmatter__glossary.md__fdm-fused-deposition-modeling}

A thermoplastic extrusion process in which filament is melted and deposited in sequential layers.

\subsubsection*{Critical Variables}\label{docs__pandoc__latex__src__backmatter__glossary.md__critical-variables}

\begin{itemize}
\tightlist
\item
  Nozzle temperature
\item
  Bed temperature
\item
  Layer height
\item
  Print speed
\item
  Cooling rate
\end{itemize}

\begin{center}\rule{0.5\linewidth}{0.5pt}\end{center}

\subsection*{Layer Height}\label{docs__pandoc__latex__src__backmatter__glossary.md__layer-height}

The vertical thickness of each printed layer.

\begin{itemize}
\tightlist
\item
  Smaller values increase surface fidelity.
\item
  Larger values increase speed.
\end{itemize}

Example:

\begin{itemize}
\tightlist
\item
  0.2 mm = standard
\item
  0.1 mm = high resolution
\end{itemize}

\begin{center}\rule{0.5\linewidth}{0.5pt}\end{center}

\subsection*{Infill}\label{docs__pandoc__latex__src__backmatter__glossary.md__infill}

Internal structural lattice inside a model.

Typical values:

\begin{itemize}
\tightlist
\item
  10--20\% for visual models
\item
  40\%+ for structural parts
\end{itemize}

\begin{center}\rule{0.5\linewidth}{0.5pt}\end{center}

\subsection*{Tolerance}\label{docs__pandoc__latex__src__backmatter__glossary.md__tolerance}

Intentional dimensional offset to ensure mechanical fit.

Example: A 10mm peg may require a 10.2mm hole for clearance.

\begin{center}\rule{0.5\linewidth}{0.5pt}\end{center}

\subsection*{Manifold (Watertight Model)}\label{docs__pandoc__latex__src__backmatter__glossary.md__manifold-watertight-model}

A printable model with:

\begin{itemize}
\tightlist
\item
  No self-intersections
\item
  No zero-thickness faces
\item
  No holes in mesh
\end{itemize}

Non-manifold geometry causes slicer failure.

\begin{center}\rule{0.5\linewidth}{0.5pt}\end{center}

\section*{SECTION II --- OpenSCAD (Parametric Modeling)}\label{docs__pandoc__latex__src__backmatter__glossary.md__section-ii--openscad-parametric-modeling}

\subsection*{OpenSCAD}\label{docs__pandoc__latex__src__backmatter__glossary.md__openscad}

A script-based solid modeling system using Constructive Solid Geometry (CSG).

Official site: https://openscad.org

\begin{center}\rule{0.5\linewidth}{0.5pt}\end{center}

\subsection*{Constructive Solid Geometry (CSG)}\label{docs__pandoc__latex__src__backmatter__glossary.md__constructive-solid-geometry-csg}

Modeling technique combining primitives using Boolean operations.

\begin{center}\rule{0.5\linewidth}{0.5pt}\end{center}

\subsection*{Primitive}\label{docs__pandoc__latex__src__backmatter__glossary.md__primitive}

Basic geometric object.

Examples:

\begin{lstlisting}[style=Alabaster, language=openscad]
cube([10,10,10]);
sphere(5);
cylinder(h=20, d=10);

\end{lstlisting}

\begin{center}\rule{0.5\linewidth}{0.5pt}\end{center}

\subsection*{Boolean Operations}\label{docs__pandoc__latex__src__backmatter__glossary.md__boolean-operations}

\subsubsection*{union()}\label{docs__pandoc__latex__src__backmatter__glossary.md__union}

Combines shapes.

\begin{lstlisting}[style=Alabaster, language=openscad]
union() {
    cube([10,10,10]);
    sphere(6);
}

\end{lstlisting}

\begin{center}\rule{0.5\linewidth}{0.5pt}\end{center}

\subsubsection*{difference()}\label{docs__pandoc__latex__src__backmatter__glossary.md__difference}

Subtracts shapes.

\begin{lstlisting}[style=Alabaster, language=openscad]
difference() {
    cube([20,20,20]);
    cylinder(h=25, d=5);
}

\end{lstlisting}

\begin{center}\rule{0.5\linewidth}{0.5pt}\end{center}

\subsubsection*{intersection()}\label{docs__pandoc__latex__src__backmatter__glossary.md__intersection}

Keeps overlapping geometry.

\begin{lstlisting}[style=Alabaster, language=openscad]
intersection() {
    sphere(10);
    cube([15,15,15], center=true);
}

\end{lstlisting}

\begin{center}\rule{0.5\linewidth}{0.5pt}\end{center}

\subsection*{Variable}\label{docs__pandoc__latex__src__backmatter__glossary.md__variable}

A named parameter used to control geometry.

\begin{lstlisting}[style=Alabaster, language=openscad]
width = 30;
cube([width,10,5]);

\end{lstlisting}

\begin{center}\rule{0.5\linewidth}{0.5pt}\end{center}

\subsection*{Parametric Design}\label{docs__pandoc__latex__src__backmatter__glossary.md__parametric-design}

Geometry controlled by variables.

Advantages:

\begin{itemize}
\tightlist
\item
  Rapid iteration
\item
  Reusability
\item
  Scalable models
\end{itemize}

\begin{center}\rule{0.5\linewidth}{0.5pt}\end{center}

\subsection*{Module}\label{docs__pandoc__latex__src__backmatter__glossary.md__module}

Reusable geometry block.

\begin{lstlisting}[style=Alabaster, language=openscad]
module peg(d, h) {
    cylinder(d=d, h=h);
}

peg(5, 20);

\end{lstlisting}

\begin{center}\rule{0.5\linewidth}{0.5pt}\end{center}

\subsection*{\$fn (Fragment Number)}\label{docs__pandoc__latex__src__backmatter__glossary.md__fn-fragment-number}

Controls circular resolution.

\begin{lstlisting}[style=Alabaster, language=openscad]
$fn = 100;
sphere(10);

\end{lstlisting}

Higher values increase smoothness and render time.

\begin{center}\rule{0.5\linewidth}{0.5pt}\end{center}

\subsection*{Preview vs Render}\label{docs__pandoc__latex__src__backmatter__glossary.md__preview-vs-render}

\begin{itemize}
\tightlist
\item
  F5 = Preview (OpenGL approximation)
\item
  F6 = Render (full CSG evaluation)
\end{itemize}

CLI equivalent:

\begin{lstlisting}[style=Alabaster, language=bash]
openscad -o model.stl model.scad

\end{lstlisting}

\begin{center}\rule{0.5\linewidth}{0.5pt}\end{center}

\section*{SECTION III --- 3dMake (Automation Tool)}\label{docs__pandoc__latex__src__backmatter__glossary.md__section-iii--3dmake-automation-tool}

\subsection*{3dMake}\label{docs__pandoc__latex__src__backmatter__glossary.md__3dmake}

A command-line tool for automating 3D model generation and processing.

Repository: https://github.com/tdeck/3dmake

\begin{center}\rule{0.5\linewidth}{0.5pt}\end{center}

\subsection*{Headless Rendering}\label{docs__pandoc__latex__src__backmatter__glossary.md__headless-rendering}

Running OpenSCAD without GUI.

\begin{lstlisting}[style=Alabaster, language=bash]
openscad -o output.stl input.scad

\end{lstlisting}

\begin{center}\rule{0.5\linewidth}{0.5pt}\end{center}

\subsection*{Automated Build Pipeline}\label{docs__pandoc__latex__src__backmatter__glossary.md__automated-build-pipeline}

Example PowerShell pipeline:

\begin{lstlisting}[style=Alabaster, language=powershell]
openscad -o model.stl model.scad

\end{lstlisting}

Example Bash:

\begin{lstlisting}[style=Alabaster, language=bash]
openscad -o model.stl model.scad

\end{lstlisting}

\begin{center}\rule{0.5\linewidth}{0.5pt}\end{center}

\section*{SECTION IV --- Command-Line Systems}\label{docs__pandoc__latex__src__backmatter__glossary.md__section-iv--command-line-systems}

\section*{CLI (Command-Line Interface)}\label{docs__pandoc__latex__src__backmatter__glossary.md__cli-command-line-interface}

A text-based interface for interacting with the operating system.

\begin{center}\rule{0.5\linewidth}{0.5pt}\end{center}

\section*{Shell}\label{docs__pandoc__latex__src__backmatter__glossary.md__shell}

A program that interprets commands.

Examples:

\begin{itemize}
\tightlist
\item
  PowerShell
\item
  Command Prompt
\item
  Git Bash
\end{itemize}

\begin{center}\rule{0.5\linewidth}{0.5pt}\end{center}

\section*{PowerShell}\label{docs__pandoc__latex__src__backmatter__glossary.md__powershell}

Object-oriented Windows shell.

\subsubsection*{List Files}\label{docs__pandoc__latex__src__backmatter__glossary.md__list-files}

\begin{lstlisting}[style=Alabaster, language=powershell]
Get-ChildItem

\end{lstlisting}

\subsubsection*{Change Directory}\label{docs__pandoc__latex__src__backmatter__glossary.md__change-directory}

\begin{lstlisting}[style=Alabaster, language=powershell]
Set-Location Documents

\end{lstlisting}

\subsubsection*{Run Script}\label{docs__pandoc__latex__src__backmatter__glossary.md__run-script}

\begin{lstlisting}[style=Alabaster, language=powershell]
.\build.ps1

\end{lstlisting}

\subsubsection*{Check Exit Code}\label{docs__pandoc__latex__src__backmatter__glossary.md__check-exit-code}

\begin{lstlisting}[style=Alabaster, language=powershell]
$LASTEXITCODE

\end{lstlisting}

\begin{center}\rule{0.5\linewidth}{0.5pt}\end{center}

\section*{Command Prompt (cmd.exe)}\label{docs__pandoc__latex__src__backmatter__glossary.md__command-prompt-cmdexe}

Traditional Windows shell.

\subsubsection*{List Files}\label{docs__pandoc__latex__src__backmatter__glossary.md__list-files-1}

\begin{lstlisting}[style=Alabaster, language=cmd]
dir

\end{lstlisting}

\subsubsection*{Change Directory}\label{docs__pandoc__latex__src__backmatter__glossary.md__change-directory-1}

\begin{lstlisting}[style=Alabaster, language=cmd]
cd Documents

\end{lstlisting}

\subsubsection*{Run Program}\label{docs__pandoc__latex__src__backmatter__glossary.md__run-program}

\begin{lstlisting}[style=Alabaster, language=cmd]
openscad.exe -o model.stl model.scad

\end{lstlisting}

\subsubsection*{Check Exit Code}\label{docs__pandoc__latex__src__backmatter__glossary.md__check-exit-code-1}

\begin{lstlisting}[style=Alabaster, language=cmd]
echo %ERRORLEVEL%

\end{lstlisting}

\begin{center}\rule{0.5\linewidth}{0.5pt}\end{center}

\section*{Git Bash}\label{docs__pandoc__latex__src__backmatter__glossary.md__git-bash}

Unix-like shell for Windows.

\subsubsection*{List Files}\label{docs__pandoc__latex__src__backmatter__glossary.md__list-files-2}

\begin{lstlisting}[style=Alabaster, language=bash]
ls

\end{lstlisting}

\subsubsection*{Change Directory}\label{docs__pandoc__latex__src__backmatter__glossary.md__change-directory-2}

\begin{lstlisting}[style=Alabaster, language=bash]
cd Documents

\end{lstlisting}

\subsubsection*{Make Directory}\label{docs__pandoc__latex__src__backmatter__glossary.md__make-directory}

\begin{lstlisting}[style=Alabaster, language=bash]
mkdir builds

\end{lstlisting}

\begin{center}\rule{0.5\linewidth}{0.5pt}\end{center}

\section*{Working Directory}\label{docs__pandoc__latex__src__backmatter__glossary.md__working-directory}

The directory in which commands execute.

\subsubsection*{Print Working Directory}\label{docs__pandoc__latex__src__backmatter__glossary.md__print-working-directory}

\begin{lstlisting}[style=Alabaster, language=bash]
pwd

\end{lstlisting}

\begin{lstlisting}[style=Alabaster, language=powershell]
Get-Location

\end{lstlisting}

\begin{center}\rule{0.5\linewidth}{0.5pt}\end{center}

\section*{PATH (Environment Variable)}\label{docs__pandoc__latex__src__backmatter__glossary.md__path-environment-variable}

Tells system where executables are located.

\subsubsection*{View PATH (PowerShell)}\label{docs__pandoc__latex__src__backmatter__glossary.md__view-path-powershell}

\begin{lstlisting}[style=Alabaster, language=powershell]
$env:PATH

\end{lstlisting}

\subsubsection*{View PATH (cmd)}\label{docs__pandoc__latex__src__backmatter__glossary.md__view-path-cmd}

\begin{lstlisting}[style=Alabaster, language=cmd]
echo %PATH%

\end{lstlisting}

\begin{center}\rule{0.5\linewidth}{0.5pt}\end{center}

\section*{Environment Variable}\label{docs__pandoc__latex__src__backmatter__glossary.md__environment-variable}

A system-level configuration variable.

\subsubsection*{Set Temporarily (PowerShell)}\label{docs__pandoc__latex__src__backmatter__glossary.md__set-temporarily-powershell}

\begin{lstlisting}[style=Alabaster, language=powershell]
$env:TESTVAR="hello"

\end{lstlisting}

\subsubsection*{Set Temporarily (cmd)}\label{docs__pandoc__latex__src__backmatter__glossary.md__set-temporarily-cmd}

\begin{lstlisting}[style=Alabaster, language=cmd]
set TESTVAR=hello

\end{lstlisting}

\begin{center}\rule{0.5\linewidth}{0.5pt}\end{center}

\section*{Exit Code}\label{docs__pandoc__latex__src__backmatter__glossary.md__exit-code}

Numeric status returned by command.

\begin{itemize}
\tightlist
\item
  0 = Success
\item
  Non-zero = Error
\end{itemize}

Example:

\begin{lstlisting}[style=Alabaster, language=powershell]
if ($LASTEXITCODE -ne 0) { Write-Host "Build Failed" }

\end{lstlisting}

\begin{lstlisting}[style=Alabaster, language=bash]
if [ $? -ne 0 ]; then echo "Build Failed"; fi

\end{lstlisting}

\begin{center}\rule{0.5\linewidth}{0.5pt}\end{center}

\section*{Pipe}\label{docs__pandoc__latex__src__backmatter__glossary.md__pipe}

Pass output from one command to another.

\begin{lstlisting}[style=Alabaster, language=powershell]
Get-ChildItem | Select-String ".scad"

\end{lstlisting}

\begin{lstlisting}[style=Alabaster, language=bash]
ls | grep ".scad"

\end{lstlisting}

\begin{center}\rule{0.5\linewidth}{0.5pt}\end{center}

\section*{SECTION V --- Git \& Version Control}\label{docs__pandoc__latex__src__backmatter__glossary.md__section-v--git--version-control}

\subsection*{Git}\label{docs__pandoc__latex__src__backmatter__glossary.md__git}

Distributed version control system.

\begin{center}\rule{0.5\linewidth}{0.5pt}\end{center}

\subsection*{Repository}\label{docs__pandoc__latex__src__backmatter__glossary.md__repository}

Tracked project folder.

\begin{lstlisting}[style=Alabaster, language=bash]
git init

\end{lstlisting}

\begin{center}\rule{0.5\linewidth}{0.5pt}\end{center}

\subsection*{Commit}\label{docs__pandoc__latex__src__backmatter__glossary.md__commit}

Snapshot of changes.

\begin{lstlisting}[style=Alabaster, language=bash]
git add .
git commit -m "Initial model"

\end{lstlisting}

\begin{center}\rule{0.5\linewidth}{0.5pt}\end{center}

\subsection*{Branch}\label{docs__pandoc__latex__src__backmatter__glossary.md__branch}

Parallel development path.

\begin{lstlisting}[style=Alabaster, language=bash]
git branch feature-peg
git checkout feature-peg

\end{lstlisting}

\begin{center}\rule{0.5\linewidth}{0.5pt}\end{center}

\subsection*{Clone}\label{docs__pandoc__latex__src__backmatter__glossary.md__clone}

Download remote repository.

\begin{lstlisting}[style=Alabaster, language=bash]
git clone https://github.com/user/project.git

\end{lstlisting}

\begin{center}\rule{0.5\linewidth}{0.5pt}\end{center}

\subsection*{Status}\label{docs__pandoc__latex__src__backmatter__glossary.md__status}

Check changes.

\begin{lstlisting}[style=Alabaster, language=bash]
git status

\end{lstlisting}

\begin{center}\rule{0.5\linewidth}{0.5pt}\end{center}

\section*{SECTION VI --- File Formats}\label{docs__pandoc__latex__src__backmatter__glossary.md__section-vi--file-formats}

\subsection*{STL}\label{docs__pandoc__latex__src__backmatter__glossary.md__stl}

Triangle mesh file used in 3D printing.

Generated by:

\begin{lstlisting}[style=Alabaster, language=bash]
openscad -o model.stl model.scad

\end{lstlisting}

\begin{center}\rule{0.5\linewidth}{0.5pt}\end{center}

\subsection*{G-code}\label{docs__pandoc__latex__src__backmatter__glossary.md__g-code}

Machine instructions generated by slicer.

Example fragment:

\begin{lstlisting}[style=Alabaster]
G1 X10 Y10 Z0.2 F1500

\end{lstlisting}

\begin{center}\rule{0.5\linewidth}{0.5pt}\end{center}

\section*{SECTION VII --- Automation Concepts}\label{docs__pandoc__latex__src__backmatter__glossary.md__section-vii--automation-concepts}

\subsection*{Script}\label{docs__pandoc__latex__src__backmatter__glossary.md__script}

File containing executable instructions.

Examples:

\begin{itemize}
\tightlist
\item
  \texttt{.ps1}
\item
  \texttt{.sh}
\item
  \texttt{.scad}
\end{itemize}

\begin{center}\rule{0.5\linewidth}{0.5pt}\end{center}

\subsection*{Reproducibility}\label{docs__pandoc__latex__src__backmatter__glossary.md__reproducibility}

Ability to regenerate identical outputs from identical inputs.

\begin{center}\rule{0.5\linewidth}{0.5pt}\end{center}

\subsection*{Automation}\label{docs__pandoc__latex__src__backmatter__glossary.md__automation}

Using scripts instead of manual steps.

Example build script (PowerShell):

\begin{lstlisting}[style=Alabaster, language=powershell]
openscad -o build/model.stl src/model.scad

\end{lstlisting}

\begin{center}\rule{0.5\linewidth}{0.5pt}\end{center}

\section*{SECTION VIII --- Accessibility \& Documentation}\label{docs__pandoc__latex__src__backmatter__glossary.md__section-viii--accessibility--documentation}

\subsection*{Markdown}\label{docs__pandoc__latex__src__backmatter__glossary.md__markdown}

Lightweight markup language.

Example:

\begin{lstlisting}[style=Alabaster]
# Heading
- Bullet

\end{lstlisting}

\begin{center}\rule{0.5\linewidth}{0.5pt}\end{center}

\subsection*{Semantic Structure}\label{docs__pandoc__latex__src__backmatter__glossary.md__semantic-structure}

Proper heading hierarchy:

\begin{lstlisting}[style=Alabaster]
# Title
## Section
### Subsection

\end{lstlisting}

Important for screen readers.

\begin{center}\rule{0.5\linewidth}{0.5pt}\end{center}

\subsection*{Screen Reader}\label{docs__pandoc__latex__src__backmatter__glossary.md__screen-reader}

Software that reads text aloud.

Examples:

\begin{itemize}
\tightlist
\item
  NVDA
\item
  JAWS
\item
  Orca
\item
  VoiceOver
\end{itemize}

\begin{center}\rule{0.5\linewidth}{0.5pt}\end{center}

\section*{SECTION IX --- Mechanical Design Concepts}\label{docs__pandoc__latex__src__backmatter__glossary.md__section-ix--mechanical-design-concepts}

\subsection*{Clearance Fit}\label{docs__pandoc__latex__src__backmatter__glossary.md__clearance-fit}

Loose fit allowing movement.

\subsection*{Press Fit}\label{docs__pandoc__latex__src__backmatter__glossary.md__press-fit}

Tight fit requiring force.

\subsection*{Overhang}\label{docs__pandoc__latex__src__backmatter__glossary.md__overhang}

Unsupported geometry exceeding safe angle (≈45° for FDM).

\begin{center}\rule{0.5\linewidth}{0.5pt}\end{center}

\section*{SECTION X --- Advanced Workflow Concepts}\label{docs__pandoc__latex__src__backmatter__glossary.md__section-x--advanced-workflow-concepts}

\subsection*{Headless CI Build}\label{docs__pandoc__latex__src__backmatter__glossary.md__headless-ci-build}

Automated rendering using scripts or CI tools.

\begin{lstlisting}[style=Alabaster, language=bash]
openscad -o output.stl model.scad

\end{lstlisting}

\begin{center}\rule{0.5\linewidth}{0.5pt}\end{center}

\subsection*{Deterministic Modeling}\label{docs__pandoc__latex__src__backmatter__glossary.md__deterministic-modeling}

Same input → same output every time.

OpenSCAD is deterministic.

\begin{center}\rule{0.5\linewidth}{0.5pt}\end{center}

\subsection*{Dependency}\label{docs__pandoc__latex__src__backmatter__glossary.md__dependency}

External program required for workflow.

Example:

\begin{itemize}
\tightlist
\item
  OpenSCAD must be in PATH.
\end{itemize}

\begin{center}\rule{0.5\linewidth}{0.5pt}\end{center}

\section{Master Instructor Glossary with Pedagogical Notes}\label{docs__pandoc__latex__src__backmatter__teacher-glossary__master-instructor-glossary-with-pedagogical-notes}

\emph{OpenSCAD + 3dMake + CLI-Based 3D Printing Curriculum}

This glossary expands technical definitions with:

\begin{itemize}
\tightlist
\item
  Teaching emphasis
\item
  Common misconceptions
\item
  Demonstration strategies
\item
  Assessment indicators
\item
  Cross-module integration notes
\end{itemize}

\begin{center}\rule{0.5\linewidth}{0.5pt}\end{center}

\section*{1. General Design \& Engineering Concepts}\label{docs__pandoc__latex__src__backmatter__teacher-glossary__1-general-design--engineering-concepts}

\subsection*{Additive Manufacturing}\label{docs__pandoc__latex__src__backmatter__teacher-glossary__additive-manufacturing}

\textbf{Definition:} Manufacturing process that builds objects layer by layer.

\textbf{Teaching Emphasis:}\\
Contrast with subtractive manufacturing (CNC milling). Students should understand material addition vs removal.

\textbf{Common Misconception:}\\
Students often assume ``3D printing'' is a single technology; clarify FDM vs resin vs industrial processes.

\textbf{Demonstration:}\\
Show failed print cross-section to visualize layers.

\textbf{Assessment Cue:}\\
Can student explain how layer height affects surface quality?

\begin{center}\rule{0.5\linewidth}{0.5pt}\end{center}

\subsection*{Parametric Design}\label{docs__pandoc__latex__src__backmatter__teacher-glossary__parametric-design}

\textbf{Definition:} Designing models driven by adjustable variables.

\textbf{Teaching Emphasis:}\\
This is the core philosophical difference between OpenSCAD and GUI CAD tools.

\textbf{Common Misconception:}\\
Students confuse ``parameter'' with ``hardcoded variable.'' Emphasize reusability.

\textbf{Demonstration:}\\
Live-edit a dimension variable and re-render.

\textbf{Assessment Cue:}\\
Can student refactor a fixed-dimension model into a parametric one?

\begin{center}\rule{0.5\linewidth}{0.5pt}\end{center}

\subsection*{Tolerance}\label{docs__pandoc__latex__src__backmatter__teacher-glossary__tolerance}

\textbf{Definition:} Acceptable dimensional variation to ensure parts fit.

\textbf{Teaching Emphasis:}\\
Tie directly to printer calibration and material shrinkage.

\textbf{Common Misconception:}\\
Students assume digital dimensions equal physical results.

\textbf{Demonstration:}\\
Print tolerance test blocks.

\textbf{Assessment Cue:}\\
Can student calculate clearance needed for a press fit?

\begin{center}\rule{0.5\linewidth}{0.5pt}\end{center}

\section*{2. 3D Printing (FDM)}\label{docs__pandoc__latex__src__backmatter__teacher-glossary__2-3d-printing-fdm}

\subsection*{Layer Height}\label{docs__pandoc__latex__src__backmatter__teacher-glossary__layer-height}

\textbf{Definition:} Vertical thickness of each printed layer.

\textbf{Teaching Emphasis:}\\
Trade-off between print speed and surface quality.

\textbf{Common Misconception:}\\
Lower layer height does not automatically mean stronger prints.

\textbf{Demonstration:}\\
Compare 0.2mm vs 0.1mm prints.

\textbf{Assessment Cue:}\\
Can student justify layer height choice for functional vs aesthetic part?

\begin{center}\rule{0.5\linewidth}{0.5pt}\end{center}

\subsection*{Infill}\label{docs__pandoc__latex__src__backmatter__teacher-glossary__infill}

\textbf{Definition:} Internal support structure expressed as a percentage.

\textbf{Teaching Emphasis:}\\
Strength comes from perimeters + infill pattern choice.

\textbf{Common Misconception:}\\
Students assume 100\% infill = strongest in all cases.

\textbf{Demonstration:}\\
Cut apart failed prints to show internal structure.

\textbf{Assessment Cue:}\\
Can student select appropriate infill for load-bearing bracket?

\begin{center}\rule{0.5\linewidth}{0.5pt}\end{center}

\subsection*{Warping}\label{docs__pandoc__latex__src__backmatter__teacher-glossary__warping}

\textbf{Definition:} Edge lifting due to uneven cooling.

\textbf{Teaching Emphasis:}\\
Connect to material properties (PLA vs ABS).

\textbf{Assessment Cue:}\\
Can student propose mitigation strategies (brim, enclosure, temperature)?

\begin{center}\rule{0.5\linewidth}{0.5pt}\end{center}

\section*{3. OpenSCAD --- Modeling \& Language}\label{docs__pandoc__latex__src__backmatter__teacher-glossary__3-openscad--modeling--language}

\subsection*{CSG (Constructive Solid Geometry)}\label{docs__pandoc__latex__src__backmatter__teacher-glossary__csg-constructive-solid-geometry}

\textbf{Definition:} Modeling by combining primitives with boolean operations.

\textbf{Teaching Emphasis:}\\
Encourage students to think in logical operations, not sculpting.

\textbf{Common Misconception:}\\
Students attempt to ``draw'' instead of combining primitives.

\textbf{Demonstration:}\\
Build a complex object from cube + cylinder + difference().

\textbf{Assessment Cue:}\\
Can student decompose a real-world object into primitives?

\begin{center}\rule{0.5\linewidth}{0.5pt}\end{center}

\subsection*{Module}\label{docs__pandoc__latex__src__backmatter__teacher-glossary__module}

\textbf{Definition:} Reusable block of geometry.

\textbf{Teaching Emphasis:}\\
Modules = abstraction. Introduce early to prevent code duplication.

\textbf{Common Misconception:}\\
Students treat modules as simple macros; clarify parameter scope.

\textbf{Assessment Cue:}\\
Can student create a reusable module with defaults?

\begin{center}\rule{0.5\linewidth}{0.5pt}\end{center}

\subsection*{\$fn}\label{docs__pandoc__latex__src__backmatter__teacher-glossary__fn}

\textbf{Definition:} Resolution variable controlling roundness.

\textbf{Teaching Emphasis:}\\
Performance vs visual quality trade-off.

\textbf{Demonstration:}\\
Render cylinder with \$fn=20 vs 100.

\textbf{Assessment Cue:}\\
Can student explain why high \$fn slows render time?

\begin{center}\rule{0.5\linewidth}{0.5pt}\end{center}

\subsection*{Preview vs Render}\label{docs__pandoc__latex__src__backmatter__teacher-glossary__preview-vs-render}

\textbf{Definition:} F5 (preview) vs F6 (full geometry evaluation).

\textbf{Teaching Emphasis:}\\
Preview ≠ printable model.

\textbf{Common Misconception:}\\
Students export from preview state without full render.

\textbf{Assessment Cue:}\\
Does student verify manifold geometry before export?

\begin{center}\rule{0.5\linewidth}{0.5pt}\end{center}

\section*{4. 3dMake \& Automation}\label{docs__pandoc__latex__src__backmatter__teacher-glossary__4-3dmake--automation}

\subsection*{Headless Rendering}\label{docs__pandoc__latex__src__backmatter__teacher-glossary__headless-rendering}

\textbf{Definition:} Running OpenSCAD via command line without GUI.

\textbf{Teaching Emphasis:}\\
Reproducibility and automation are professional practices.

\textbf{Demonstration:}\\
Run OpenSCAD CLI from PowerShell.

\textbf{Assessment Cue:}\\
Can student produce STL via script without GUI?

\begin{center}\rule{0.5\linewidth}{0.5pt}\end{center}

\subsection*{Reproducible Build}\label{docs__pandoc__latex__src__backmatter__teacher-glossary__reproducible-build}

\textbf{Definition:} Identical output from version-controlled inputs.

\textbf{Teaching Emphasis:}\\
Critical for collaboration and grading consistency.

\textbf{Assessment Cue:}\\
Can another student regenerate identical artifact from repo?

\begin{center}\rule{0.5\linewidth}{0.5pt}\end{center}

\section*{5. Command Line \& Shell}\label{docs__pandoc__latex__src__backmatter__teacher-glossary__5-command-line--shell}

\subsection*{Environment Variable}\label{docs__pandoc__latex__src__backmatter__teacher-glossary__environment-variable}

\textbf{Definition:} Named value available to processes (e.g., PATH).

\textbf{Teaching Emphasis:}\\
Students must understand PATH for CLI tools.

\textbf{Common Misconception:}\\
Confusion between system vs user variables.

\textbf{Assessment Cue:}\\
Can student diagnose ``command not found'' error?

\begin{center}\rule{0.5\linewidth}{0.5pt}\end{center}

\subsection*{Exit Code}\label{docs__pandoc__latex__src__backmatter__teacher-glossary__exit-code}

\textbf{Definition:} Numeric return status of command (0 = success).

\textbf{Teaching Emphasis:}\\
Foundation for scripting logic.

\textbf{Assessment Cue:}\\
Can student use conditional logic based on exit status?

\begin{center}\rule{0.5\linewidth}{0.5pt}\end{center}

\subsection*{Pipe}\label{docs__pandoc__latex__src__backmatter__teacher-glossary__pipe}

\textbf{Definition:} Redirect output of one command into another.

\textbf{Teaching Emphasis:}\\
Core Unix philosophy --- small tools combined.

\textbf{Demonstration:}\\
\texttt{Get-ChildItem\ \textbar{}\ Select-String}

\begin{center}\rule{0.5\linewidth}{0.5pt}\end{center}

\section*{6. PowerShell-Specific}\label{docs__pandoc__latex__src__backmatter__teacher-glossary__6-powershell-specific}

\subsection*{Cmdlet}\label{docs__pandoc__latex__src__backmatter__teacher-glossary__cmdlet}

\textbf{Definition:} Native PowerShell command.

\textbf{Teaching Emphasis:}\\
Verb-Noun naming convention.

\textbf{Assessment Cue:}\\
Can student identify correct cmdlet via \texttt{Get-Command}?

\begin{center}\rule{0.5\linewidth}{0.5pt}\end{center}

\subsection*{Execution Policy}\label{docs__pandoc__latex__src__backmatter__teacher-glossary__execution-policy}

\textbf{Definition:} Controls whether scripts can run.

\textbf{Teaching Emphasis:}\\
Security implications.

\textbf{Assessment Cue:}\\
Can student safely modify policy for development?

\begin{center}\rule{0.5\linewidth}{0.5pt}\end{center}

\section*{7. Git \& Version Control}\label{docs__pandoc__latex__src__backmatter__teacher-glossary__7-git--version-control}

\subsection*{Commit}\label{docs__pandoc__latex__src__backmatter__teacher-glossary__commit}

\textbf{Definition:} Snapshot of project state.

\textbf{Teaching Emphasis:}\\
Encourage small, meaningful commits.

\textbf{Assessment Cue:}\\
Does student use descriptive commit messages?

\begin{center}\rule{0.5\linewidth}{0.5pt}\end{center}

\subsection*{Branch}\label{docs__pandoc__latex__src__backmatter__teacher-glossary__branch}

\textbf{Definition:} Parallel development path.

\textbf{Teaching Emphasis:}\\
Introduce feature branching in advanced module.

\begin{center}\rule{0.5\linewidth}{0.5pt}\end{center}

\section*{8. Accessibility}\label{docs__pandoc__latex__src__backmatter__teacher-glossary__8-accessibility}

\subsection*{Screen Reader}\label{docs__pandoc__latex__src__backmatter__teacher-glossary__screen-reader}

\textbf{Definition:} Software that reads digital text aloud.

\textbf{Teaching Emphasis:}\\
All documentation must be keyboard navigable.

\textbf{Assessment Cue:}\\
Can site be navigated without mouse?

\begin{center}\rule{0.5\linewidth}{0.5pt}\end{center}

\subsection*{Semantic Structure}\label{docs__pandoc__latex__src__backmatter__teacher-glossary__semantic-structure}

\textbf{Definition:} Logical heading hierarchy in Markdown/HTML.

\textbf{Teaching Emphasis:}\\
Accessibility and maintainability.

\begin{center}\rule{0.5\linewidth}{0.5pt}\end{center}

\section*{9. Safety \& Risk}\label{docs__pandoc__latex__src__backmatter__teacher-glossary__9-safety--risk}

\subsection*{Hierarchy of Controls}\label{docs__pandoc__latex__src__backmatter__teacher-glossary__hierarchy-of-controls}

\textbf{Definition:} Safety prioritization model (eliminate → PPE).

\textbf{Teaching Emphasis:}\\
Ventilation \textgreater{} mask alone.

\begin{center}\rule{0.5\linewidth}{0.5pt}\end{center}

\section*{10. AI \& Generative Design (Advanced)}\label{docs__pandoc__latex__src__backmatter__teacher-glossary__10-ai--generative-design-advanced}

\subsection*{Prompt Engineering}\label{docs__pandoc__latex__src__backmatter__teacher-glossary__prompt-engineering}

\textbf{Definition:} Designing effective AI instructions.

\textbf{Teaching Emphasis:}\\
Students must validate AI-generated geometry.

\textbf{Assessment Cue:}\\
Can student identify unsafe AI-generated model?

\begin{center}\rule{0.5\linewidth}{0.5pt}\end{center}

\section*{Instructor Meta-Notes}\label{docs__pandoc__latex__src__backmatter__teacher-glossary__instructor-meta-notes}

\subsubsection*{Spiral Curriculum Recommendation}\label{docs__pandoc__latex__src__backmatter__teacher-glossary__spiral-curriculum-recommendation}

Reintroduce:

\begin{itemize}
\tightlist
\item
  Parametric design
\item
  Automation
\item
  Reproducibility at increasing levels of complexity.
\end{itemize}

\subsubsection*{Capstone Integration}\label{docs__pandoc__latex__src__backmatter__teacher-glossary__capstone-integration}

Final project should require:

\begin{itemize}
\tightlist
\item
  Parametric OpenSCAD module
\item
  3dMake automated build
\item
  Git version control
\item
  Printed physical artifact
\item
  Accessibility-compliant documentation
\end{itemize}

\subsubsection*{Evaluation Matrix Suggestions}\label{docs__pandoc__latex__src__backmatter__teacher-glossary__evaluation-matrix-suggestions}

Assess students on:

\begin{itemize}
\tightlist
\item
  Code structure
\item
  Parametric flexibility
\item
  CLI competency
\item
  Print quality
\item
  Documentation clarity
\item
  Accessibility compliance
\end{itemize}

\section{References}\label{docs__pandoc__latex__src__backmatter__further_reading.md__references}

\subsection*{Peer-Reviewed and Scholarly Works}\label{docs__pandoc__latex__src__backmatter__further_reading.md__peer-reviewed-and-scholarly-works}

Gonzalez Avila, J. F., Pietrzak, T., Girouard, A., \& Casiez, G. (2024). \emph{Understanding the challenges of OpenSCAD users for 3D printing.} In \emph{Proceedings of the CHI Conference on Human Factors in Computing Systems (CHI '24)} (Article 351, pp. 1--20). Association for Computing Machinery. https://doi.org/10.1145/3613904.3642566 :contentReference{[}oaicite:0{]}\{index=0\}

Li, G. (2024). \emph{Chinese dragon modeling based on OpenSCAD.} \emph{Applied and Computational Engineering, 67}, 1--12. https://doi.org/10.54254/2755-2721/67/20240585 :contentReference{[}oaicite:1{]}\{index=1\}

Kwon, N., Sun, T., Gao, Y., Zhao, L., Wang, X., Kim, J., \& Hong, S. R. (2024). \emph{3DPFIX: Improving remote novices' 3D printing troubleshooting through human-AI collaboration.} arXiv. https://doi.org/10.48550/arXiv.2401.15877 :contentReference{[}oaicite:2{]}\{index=2\}

\subsection*{Government \& Standards}\label{docs__pandoc__latex__src__backmatter__further_reading.md__government--standards}

Centers for Disease Control and Prevention, National Institute for Occupational Safety and Health. (2020, May 14). \emph{Health and safety considerations for 3D printing.} https://blogs.cdc.gov/niosh-science-blog/2020/05/14/3d-printing/

Centers for Disease Control and Prevention, National Institute for Occupational Safety and Health. (2024). \emph{Approaches to safe 3D printing.} https://www.cdc.gov/niosh/blogs/2024/safe-3d-printing.html

National Institute of Standards and Technology. (2023). \emph{Artificial Intelligence Risk Management Framework (AI RMF 1.0).} https://www.nist.gov/system/files/documents/2023/01/26/AI\%20RMF\%201.0.pdf

Occupational Safety and Health Administration. (n.d.). \emph{Hierarchy of controls.} https://www.osha.gov/hierarchy-of-controls

Washington State Department of Health. (n.d.). \emph{3D printers in schools.} https://doh.wa.gov/community-and-environment/schools/3d-printers

\subsection*{Industry \& Commercial Documentation}\label{docs__pandoc__latex__src__backmatter__further_reading.md__industry--commercial-documentation}

All3DP. (n.d.). \emph{3D printer filament types explained.} https://all3dp.com/1/3d-printer-filament-types-3d-printing-3d-filament/

All3DP. (n.d.). \emph{FDM 3D printing tolerances.} https://all3dp.com/2/fdm-3d-printing-tolerances/

Anycubic. (n.d.). \emph{Anycubic.} https://www.anycubic.com/

Autodesk. (n.d.). \emph{Fusion 360.} https://www.autodesk.com/products/fusion-360

Bambu Lab. (n.d.). \emph{Downloads.} https://bambulab.com/en/download

Bambu Lab. (n.d.). \emph{Bambu Studio.} https://bambulab.com/en/download/studio

Bambu Lab Wiki. (n.d.). \emph{Bambu Lab Wiki.} https://wiki.bambulab.com

Hubs. (n.d.). \emph{What is FDM 3D printing?} https://www.hubs.com/knowledge-base/what-is-fdm-3d-printing/

IDEA Maker (Raise3D). (n.d.). \emph{ideaMaker.} https://www.raise3d.com/ideamaker

MatterHackers. (n.d.). \emph{3D printer filament comparison.} https://www.matterhackers.com/3d-printer-filament-compare

Prusa Research. (n.d.). \emph{PrusaSlicer 4.2.4.} https://www.prusa3d.com/page/prusaslicer424/

Prusa Research. (n.d.). \emph{PrusaSlicer 4.1.0.} https://www.prusa3d.com/page/prusaslicer\_410/

Prusa Research. (n.d.). \emph{Support.} https://www.prusa3d.com/support/

Stratasys. (n.d.). \emph{Stratasys.} https://stratasys.com/

UL Research Institutes. (n.d.). \emph{3D printing emissions study.} https://www.ul.com/news/ul-research-institutes-releases-3d-printing-emissions-study

Ultimaker. (n.d.). \emph{Ultimaker Cura.} https://ultimaker.com/software/ultimaker-cura

ZenML. (n.d.). \emph{LLM-powered 3D model generation for 3D printing.} https://www.zenml.io/llmops-database/llm-powered-3d-model-generation-for-3d-printing

\subsection*{Open-Source Software \& Technical Documentation}\label{docs__pandoc__latex__src__backmatter__further_reading.md__open-source-software--technical-documentation}

Anthropic. (n.d.). \emph{Prompt engineering overview.} https://docs.anthropic.com/en/docs/build-with-claude/prompt-engineering/overview

Apache Software Foundation. (2004). \emph{Apache License, Version 2.0.} https://www.apache.org/licenses/LICENSE-2.0

BelfrySCAD. (n.d.). \emph{BOSL2.} https://github.com/BelfrySCAD/BOSL2

Deck, T. (2025). \emph{3DMake repository.} https://github.com/tdeck/3dmake

Free Software Foundation. (n.d.). \emph{GNU Bash manual.} https://www.gnu.org/software/bash/manual/

Free Software Foundation. (n.d.). \emph{GNU Coreutils manual.} https://www.gnu.org/software/coreutils/manual/

Free Software Foundation. (n.d.). \emph{GNU Grep manual.} https://www.gnu.org/software/grep/manual/grep.html

Git SCM. (n.d.). \emph{Git documentation.} https://git-scm.com/docs

Google Cloud. (2025). \emph{Get started with Gemini models.} https://docs.cloud.google.com/vertex-ai/generative-ai/docs/start/get-started-with-gemini-3

Microsoft. (n.d.). \emph{PowerShell documentation.} https://learn.microsoft.com/powershell/

Microsoft. (n.d.). \emph{Windows commands documentation.} https://learn.microsoft.com/en-us/windows-server/administration/windows-commands/

OpenSCAD. (n.d.). \emph{OpenSCAD documentation.} https://openscad.org/documentation.html

OpenSCAD User Manual Contributors. (n.d.). \emph{OpenSCAD User Manual.} https://en.wikibooks.org/wiki/OpenSCAD\_User\_Manual

Programming with OpenSCAD. (n.d.). \emph{Programming with OpenSCAD.} https://programmingwithopenscad.github.io/

Rust Project Developers. (n.d.). \emph{mdBook.} https://rust-lang.github.io/mdBook/

SuperSlicer. (n.d.). \emph{SuperSlicer repository.} https://github.com/supermerill/SuperSlicer

The Linux Documentation Project. (n.d.). \emph{Bash Beginner's Guide.} https://tldp.org/LDP/Bash-Beginners-Guide/html/

\subsection*{Educational \& Community Resources}\label{docs__pandoc__latex__src__backmatter__further_reading.md__educational--community-resources}

A11Y-101. (n.d.). \emph{Inclusive documentation design.} https://www.a11y-101.com/design/inclusive-documentation

Class Central. (n.d.). \emph{OpenSCAD courses.} https://www.classcentral.com/subject/openscad

Khan Academy. (n.d.). \emph{Algorithms.} https://www.khanacademy.org/computing/computer-science/algorithms

Salt Lake City Public Library. (n.d.). \emph{Creative Lab.} https://services.slcpl.org/creativelab

Salt Lake County Library. (n.d.). \emph{Create spaces.} https://www.slcolibrary.org/what-we-have/create

University of Utah. (n.d.). \emph{ProtoSpace.} https://lib.utah.edu/protospace.php

\subsubsection*{Accessibility \& Assistive Tech}\label{docs__pandoc__latex__src__backmatter__further_reading.md__accessibility--assistive-tech}

Freedom Scientific. (n.d.). \emph{JAWS screen reader.} https://www.freedomscientific.com/products/software/jaws/

Microsoft. (n.d.). \emph{Narrator guide.} https://support.microsoft.com/narrator

NV Access. (n.d.). \emph{NVDA screen reader.} https://www.nvaccess.org/

Your Dolphin. (n.d.). \emph{SuperNova.} https://yourdolphin.com/supernova/

\subsubsection*{Community \& Media}\label{docs__pandoc__latex__src__backmatter__further_reading.md__community--media}

Discord. (n.d.). \emph{3D printing community server.} https://discord.gg/F2Nx2VxTB7

Reddit. (n.d.). \emph{r/3Dprinting subreddit.} https://www.reddit.com/r/3Dprinting/

Reddit. (n.d.). \emph{r/OpenSCAD subreddit.} https://www.reddit.com/r/openscad/

YouTube. (n.d.). \emph{Luke's Lab channel.} https://www.youtube.com/@LukesBlab

YouTube. (n.d.). \emph{Teaching Tech channel.} https://www.youtube.com/@TeachingTech

\section{Index}\label{docs__pandoc__latex__src__index_page.md__index}

::: \{.content-visible when-format="pdf"\}

\begin{lstlisting}[style=Alabaster]
\printindex

:::

\end{lstlisting}

\chapter*{Contributors to this Curriculum}\label{docs__pandoc__latex__src__contributors.md__contributors}
\addcontentsline{toc}{chapter}{Contributors to this Curriculum}

\pandocbounded{\includegraphics[keepaspectratio,alt={Headshot of Michael Ryan Hunsaker, M.Ed., Ph.D.}]{docs/pandoc/latex/src/assets/images/bio-photo.jpg}}

Michael Ryan Hunsaker, M.Ed., Ph.D. (he/him)

\section*{About Dr. Hunsaker}\label{docs__pandoc__latex__src__contributors.md__about-dr-hunsaker}

Michael Ryan Hunsaker holds a Ph.D. in neuroscience (behavioral/translational neuroscience focus) from UC Davis and a Master\textquotesingle s degree in Special Education from the University of Utah with a credential to teach blind and visually impaired students. His career uniquely bridges scientific research, clinical experience, and classroom practice-positioning him to design accessible educational materials grounded in both neuroscience and practical pedagogy.

\subsection*{Research \& Clinical Background}\label{docs__pandoc__latex__src__contributors.md__research--clinical-background}

Following his doctoral work at UC Davis (2012), Dr. Hunsaker completed postdoctoral research in primate neural development at the UC Davis MIND Institute, where he developed sophisticated research methods to evaluate developmental disorders. During this period, he gained extensive clinical experience working directly with children and adults across a broad spectrum of diagnoses, including autism spectrum disorder, cerebral palsy, genetic conditions (fragile X syndrome, Williams syndrome, 22q11.2DS, Down syndrome), ADHD, and Tourette\textquotesingle s syndrome. This diverse exposure to different neurological profiles taught him a crucial lesson: each learner\textquotesingle s disability uniquely shapes their cognitive strengths, learning preferences, and educational needs.

\subsection*{Transition to Education}\label{docs__pandoc__latex__src__contributors.md__transition-to-education}

In 2013, recognizing the gap between academic research and classroom reality, Dr. Hunsaker made a deliberate career shift to direct education. Rather than pursue traditional academic pathways, he committed to applying his scientific training to help students in real classroom environments. As a special education teacher, he focused on understanding the individual relationship between disability and learning-developing instructional approaches that enable students to access their full potential rather than working around perceived limitations.

\subsection*{Expertise in Accessibility \& Inclusive Design}\label{docs__pandoc__latex__src__contributors.md__expertise-in-accessibility--inclusive-design}

His specialization in teaching blind and visually impaired students, particularly those with additional disabilities, has given Dr. Hunsaker deep expertise in accessible pedagogy and non-visual problem-solving. He grounds all curriculum design in evidence-based practice informed by neuroscience, accessibility research, and years of direct classroom experience with diverse learners. He understands that accessibility is not an afterthought or accommodation; it is the foundation upon which effective education for all students is built.

\subsection*{Creator of Accessible Educational Resources}\label{docs__pandoc__latex__src__contributors.md__creator-of-accessible-educational-resources}

Motivated by observing persistent gaps in affordable, practical, and culturally responsive accessible educational materials, Dr. Hunsaker created TVI Resources, a collection of free, openly licensed instructional materials designed to meet the real needs of educators and students. This commitment to accessible education informed his development of this curriculum: a comprehensive, text-based approach to 3D design and digital fabrication specifically engineered for blind and visually impaired learners, emphasizing independence, technical mastery, and real-world application.

\section*{Acknowledgements}\label{docs__pandoc__latex__src__contributors.md__acknowledgements}

This curriculum stands on the shoulders of innovators, educators, and open-source maintainers whose commitment to accessibility and open education made this work possible.

\subsection*{3DMake}\label{docs__pandoc__latex__src__contributors.md__3dmake}

Special thanks to Troy Deck, the creator and maintainer of \href{https://github.com/tdeck/3dmake}{3DMake}, an open-source command-line tool that automates the workflow from code-based 3D models to print-ready files. 3DMake is essential to this curriculum-it eliminates the visual GUI entirely and makes parametric 3D design accessible to screen reader users through text-based commands and configuration files. Without 3DMake, a truly non-visual pathway from design to fabrication would not exist.

\subsection*{Texas School for the Blind and Visually Impaired (TSBVI)}\label{docs__pandoc__latex__src__contributors.md__texas-school-for-the-blind-and-visually-impaired-tsbvi}

This curriculum was directly inspired by the innovative work at TSBVI, a pioneer in accessible education for blind and visually impaired students. The foundational idea of using text-driven tools and command-line workflows for technical education originated from  collaboration with TSBVI educators. Their commitment to demonstrating that blind and visually impaired students can excel in technical fields provided the vision for this project.

\subsection*{OpenSCAD Community}\label{docs__pandoc__latex__src__contributors.md__openscad-community}

Deep appreciation to the OpenSCAD project maintainers and contributors for creating and maintaining a code-driven, parametric 3D modeling system that is inherently more accessible than traditional GUI-based CAD tools. OpenSCAD\textquotesingle s text-based programming model and screen-reader-compatible approach to geometry design makes it uniquely suited for learners who use assistive technology. The OpenSCAD community\textquotesingle s commitment to open-source principles and accessibility embodies the philosophy that drives this curriculum.

\subsection*{Open Education \& Accessibility Movement}\label{docs__pandoc__latex__src__contributors.md__open-education--accessibility-movement}

Finally, this curriculum is part of a broader movement toward open educational resources that prioritize accessibility from the ground up. It reflects the conviction that education should be barrier-free, that students with disabilities bring unique perspectives and strengths, and that accessible design benefits all learners. The work of accessibility advocates, neurodivergent educators, and disability justice activists informed every decision in building this material.


% Render the index at the very end of the book
\cleardoublepage
\phantomsection
\addcontentsline{toc}{chapter}{Index}
\clearpage
\printindex

\end{document}
